% !TEX root = ./AM_2C-Aulas_Anotações.5.tex
\providecommand\mainfilename{"./AM_2C-Aulas_Anotações.tex"}
\providecommand \subfilename{}
\renewcommand   \subfilename{"./AM_2C-Aulas_Anotações.5.tex"}
\documentclass[\mainfilename]{subfiles}

% \graphicspath{{\subfix{../images/}}}
% \tikzset{external/force remake=true} % - remake all

\begin{document}

\mymakesubfile{5}
[AM\,2C]
{Aula 24/10/2022: Maximos e Mínimos} % Subfile Title
{Aula 24/10/2022: Maximos e Mínimos} % Part Title

\begin{exampleBox}1{} % E1
    
    \begin{BM}
        f(x,y)
        = (3,x-y^2,x^3-3\,y^2)
    \end{BM}

    \begin{flalign*}
        &
            \det J\,f(1,1)
            = \begin{pmatrix}
                3 & -2\,y
                \\
                3\,x^2 & - 6\,y
            \end{pmatrix}_{(1,1)}
            = \begin{pmatrix}
                3 & -2
                \\
                3 & - 6
            \end{pmatrix}
            = -18 + 6 = -12 \neq 0
            &\\&
            f(1,1) = (2,-2)
            &\\&
            \begin{bmatrix}
                    \pdv{x}{u} 
                &   \pdv{x}{v}
                \\  \pdv{y}{u} 
                &   \pdv{y}{v}
            \end{bmatrix}
            = J f^{-1}(2,-2)
            = J f^{-1}(f(1,1))
            % = &\\&
            = (J f(1,1))^{-1}
            = \begin{bmatrix}
                    \pdv{u}{x} 
                &   \pdv{u}{y}
                \\  \pdv{v}{x} 
                &   \pdv{v}{y}
            \end{bmatrix}^{-1}
            = &\\&
            = \begin{bmatrix}
                     3
                &   -2
                \\   3
                &   -6
            \end{bmatrix}^{-1}
            = -12^{-1}
            \begin{bmatrix}
                -6 & 2
                \\
                -3 & 3
            \end{bmatrix}
            = -12^{-1}
            \begin{bmatrix}
                1/2 & -1/6
                \\
                1/4 & -1/4
            \end{bmatrix}
        &
    \end{flalign*}
    
\end{exampleBox}

\begin{definitionBox}1{Invérsa por Adjunta} % DEF1
    
    \begin{BM}
        A^{-1}
        = \frac{\hat{A}^T}{\lvert A \rvert}
    \end{BM}

    \begin{definitionBox}3{Exemplo} % DEF1 (i)
        
        Para matriz 2\times 2
        \begin{flalign*}
            &
                A^{-1}
                = \begin{bmatrix}
                    a & b
                    \\
                    c & d
                \end{bmatrix}^{-1}
                = \left(
                    a\,d - b\,c
                \right)^{-1}
                \begin{bmatrix}
                    d & -b
                    \\
                    -c & a
                \end{bmatrix}
            &
        \end{flalign*}
        
    \end{definitionBox}
    
\end{definitionBox}

\begin{definitionBox}1{Extremos} % DEF2
    
    Seja \(f:D\subset\mathbb{R}^n\to\mathbb{R}\) e \(x_0\in\int{D}\). Diz-se que \(f(x_0)\) é um mínimo local ou relativo da função \textit{f}, se existir \(B(x_0,\delta)\) tal que
    \begin{BM}
        f(x_0)\leq f(x),\ \forall\,x\in B(x_0,\delta)
    \end{BM}
    \begin{itemize}
        \item Ao ponto \(x_0\) chama-se um minimizante local da função f.
        \item Diz-se que \(f(x_0)\) é um extremo local ou relativo de \textit{f} se for um mínimo local ou um máximo local
    \end{itemize}
    
\end{definitionBox}

\begin{definitionBox}1{Teorema de Weierstraß} % DEF3
    
    Seja \(D\subset\mathbb{R}^n\) um conjunto fechado e limitado (compacto) e \(f:D\subset\mathbb{R}^n\to\mathbb{R}\) uma função contínua. Então \textit{f} tem em \textit{D} um máximo e um mínimo.

    \begin{definitionBox}3{Exemplo} % DEF3 (i)
        
        \begin{BM}
            y = x^2,\ \mathbb{R}
        \end{BM}

    \tikzset{external/remake next=true}
    \begin{center}
        \pgfplotsset{
            height=4cm, 
            width =10cm}
        \begin{tikzpicture}
            \begin{axis}
                [
                    xmajorgrids = true,
                    % legend pos  = north west
                    axis lines=center, % 3D center/box/left/right
                    hide axis
                ]
                % Legends
                % \addlegendimage{empty legend}
                % \addlegendentry[Red]{\( x \)}

                % Circunferencia
                \addplot[
                    smooth,
                    thick,
                    red\Light,
                    domain  = -1:1,
                    samples = 0.4*\mysampledensityFancy,
                ]{  x^2 };
                
            \end{axis}
        \end{tikzpicture}
    \end{center}
        
    \end{definitionBox}
    
\end{definitionBox}

\begin{definitionBox}1{} % DEF 4
    
    \begin{BM}
        f:D\subset\mathbb{R}^n\to \mathbb{R}
        \quad
        a\in\int{D}
    \end{BM}
    O ponto \textit{a} diz-se um ponto crítico ou de estacionaridade de \textit{f} se:
    \begin{enumerate}
        \item Pelo menos uma das dervadas \(\pdv{f}{x_i}(a),i\in\{1,\dots,n\}\) não existir, ou
        \item \(\pdv{f}{x_i}(a)= 0,\quad\forall\,i\in\{1,\dots,n\}\)
    \end{enumerate}
    
\end{definitionBox}

\begin{exampleBox}1{} % E2
    
    \begin{BM}
        f(x,y)=x^2-y^2
        \\
        \left(
            \pdv{f}{x},
            \pdv{f}{y}
        \right)
        = (0,0)
        \iff
        (2\,x,-2\,y)
        = (0,0)
        \\
        \therefore
        (0,0)\text{ é o único ponto crítico}
    \end{BM}
    
\end{exampleBox}

\end{document}
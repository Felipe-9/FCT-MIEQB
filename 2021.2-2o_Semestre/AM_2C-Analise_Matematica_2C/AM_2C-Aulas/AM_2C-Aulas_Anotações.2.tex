% !TEX root = ./AM_2C-Aulas_Anotações.2.tex
\providecommand\mainfilename{"./AM_2C-Aulas_Anotações.tex"}
\providecommand \subfilename{}
\renewcommand   \subfilename{"./AM_2C-Aulas_Anotações.2.tex"}
\documentclass[\mainfilename]{subfiles}

% \graphicspath{{\subfix{../images/}}}

\begin{document}

\mymakesubfile{2}{03/21 -- Limites}{03/21 -- Limites}

\begin{questionBox}1{}
    
    \begin{BM}
        f(x,y) =
        \begin{cases}
            5\,x - y &\quad se\ x-y\leq 2
        \\  \frac{x^2-y^2+4\,x+8}{x-y} &\quad se\ x-y>2
        \end{cases}
    \end{BM}

    \begin{questionBox}3{}
        
        \begin{BM}
            \lim_{
                \begin{subarray}{c}
                    (x,y)\to(a,b)
                \\  (x,y)\in D_1
                \end{subarray}
            } f(x,y) :
            \left\{
            \begin{aligned}
                &
                    (a,b)\in\fronteira D_1=\fronteira D_2
                \ldiv
                    (a-b=2)
                &
            \end{aligned}
            \right\}
        \end{BM}

        \begin{flalign*}
            &
               =\lim_{
                    \begin{subarray}{c}
                        (x,y)\to(a,b)
                    \\  (x,y)\in D_2
                    \end{subarray}
                }
                \frac{x^2-y^2+4\,x+8}{x-y}
               =\frac{a^2-b^2+4\,a+8}{a-b}
               =&\\&
               =\frac{(a+b)(a-b)+4\,a+8}{2}
               =3\,a+b+4
               =3\,a+(a-2)+4
               =4\,a+2
            &
        \end{flalign*}
        
    \end{questionBox}

\end{questionBox}

\begin{sectionBox}1{Testes para encontrar Limites}
    
    \begin{enumerate}
        \item Iterados
        \item Direcionais
        \item Parabolas
        \item Provar por definição
    \end{enumerate}
    
\end{sectionBox}

\begin{questionBox}1m{}
    
    \begin{BM}
        f(x,y)
       =\frac{x\,y}{\sqrt{x^2+y^2}}
       :\begin{cases}
           \dominio = \mathbb{R}^2\backslash\{0,0\}
        \end{cases}
    \end{BM}

    \begin{multicols}{2}

    \begin{questionBox}3{}
        \begin{flalign*}
            &
                \lim_{x\to0}
                \left(
                    \lim_{y\to0}
                    f(x,y)
                \right)
               =0
            &
        \end{flalign*}
    \end{questionBox}

    \begin{questionBox}3{}
        \begin{flalign*}
            &
                \lim_{y\to0}
                \left(
                    \lim_{x\to0}
                    f(x,y)
                \right)
               =0
            &
        \end{flalign*}
    \end{questionBox}

    \end{multicols}

    \begin{questionBox}3{\(y=m\,x:m\in\mathbb{R}\backslash\{0\}\land x>0\)}
        \begin{flalign*}
            &
                \lim_{x\to0}
                \frac{m\,x^2}{\sqrt{x^2+m^2\,x^2}}
               =\lim_{x\to0}
                \frac{m\,x^2}{x\sqrt{1+m^2}}
               =0
            &
        \end{flalign*}
    \end{questionBox}

    \begin{questionBox}3{\(y=m\,x^2:m\in\mathbb{R}\backslash\{0\}\)}
        \begin{flalign*}
            &
                \lim_{x\to 0}
                \frac{x\,(m\,x^2)}{\sqrt{x^2+m^2\,x^4}}
               =\lim_{x\to 0}
                \frac{m\,x^3}{x\sqrt{1+m^2\,x^2}}
               =\lim_{x\to 0}
                \frac{m\,x^2}{\sqrt{1+m^2\,x^2}}
               =0
            &
        \end{flalign*}
    \end{questionBox}

    \begin{questionBox}3{Definição}

        \begin{BM}
            \begin{aligned}
                &
                    (x,y)\in\dominio
                \ldiv
                    \lVert
                        (x,y)-(0,0)
                    \rVert
                =\sqrt{x^2+y^2}\leq\varepsilon
                &
            \end{aligned}
        \end{BM}

        \begin{flalign*}
            &
                \Bigg\lvert
                    \frac{x\,y}{\sqrt{x^2+y^2}}-0
                \Bigg\rvert
               =\frac{\lvert x\,y \rvert}{\sqrt{x^2+y^2}}
               =\frac{\lvert x \rvert\lvert y \rvert}{\sqrt{x^2+y^2}}
            \leq\frac{\sqrt{x^2+y^2}\sqrt{x^2+y^2}}{\sqrt{x^2+y^2}}
               =&\\&
               =\sqrt{x^2+y^2}
            \leq\varepsilon=\delta
            &
        \end{flalign*}
    \end{questionBox}
    
\end{questionBox}

\begin{questionBox}1{}
    
    \begin{BM}
        f(x,y,z) 
       =\frac{z}{\sqrt{(x-1)^2+(y+1)^2+z^2}}:
        \begin{cases}
            \dominio=\mathbb{R}^3\backslash\{1,-1,0\}
        \end{cases}
    \end{BM}
    
\end{questionBox}

\begin{questionBox}1{}
    
    \begin{BM}
        f(x,y) = x^2+y^2
    \end{BM}

    \begin{questionBox}3{}
        
        \begin{flalign*}
            &
                \lim_{(x,y)\to(2,1)}f(x,y)
               =2^2+1^2 = 5
            &
        \end{flalign*}
        
    \end{questionBox}

    \begin{questionBox}3{Definição}
        
        \begin{BM}
            \forall\delta>0\,\exists\,\varepsilon>0:
            (x,y)\in\mathbb{R}^2\land\sqrt{(x-2)^2+(y-1)^2}\leq\varepsilon
        \end{BM}

        \begin{flalign*}
            &
                \lvert
                    x^2 + y^2 - 5
                \rvert
               =\lvert
                   (x-2)^2 + (y-1)^2 - 4 + 4\,x-1+2\,y-5
                \rvert
               =&\\&
               =\lvert
                   (x-2) + (y-1)^2+ 4\,x-2\,y-10
                \rvert
            \leq&\\&
            \leq\lvert
                   (x-2) + (y-1)^2
                \rvert
               +\lvert
                    4\,(x-2) + 2\,(y-1)+8+2-10
                \rvert
            \leq&\\&
            \leq(x-2) + (y-1)^2
               +4\,\lvert x-2 \rvert 
               +2\,\lvert y-1 \rvert
               <\varepsilon^2+6\,\varepsilon
               =\delta
                \implies &\\&
                \implies 
                \varepsilon
               =\frac{ -6\pm\sqrt{36+4\,\delta} }{2}
               =-3\pm\sqrt{9+\delta}
                \varepsilon
               =-3+\sqrt{9+\delta}
               >0
            &
        \end{flalign*}

    \end{questionBox}
    
\end{questionBox}

\begin{questionBox}1{}
    
    \begin{BM}
        f(x,y) = \frac{ x^3\,y }{ 2\,x^6 + y^2 }
       :\begin{cases}
           \dominio = \mathbb{R}^2\backslash\{(0,0)\}
        \\ (0,0)\in\overline{f(x,y)}
        \end{cases}
    \end{BM}

    \begin{questionBox}3{\(y=x^3\)}
        \begin{flalign*}
            &
                \lim_{x\to0}
                \frac{x^6}{2\,x^6+x^6}
               =1/3
            &
        \end{flalign*}
    \end{questionBox}
    
\end{questionBox}

\end{document}
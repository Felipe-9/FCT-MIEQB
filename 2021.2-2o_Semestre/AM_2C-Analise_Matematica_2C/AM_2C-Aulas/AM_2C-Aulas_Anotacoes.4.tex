% !TEX root = ./AM_2C-Aulas_Anotações.4.tex
\providecommand\mainfilename{"./AM_2C-Aulas_Anotações.tex"}
\providecommand \subfilename{}
\renewcommand   \subfilename{"./AM_2C-Aulas_Anotações.4"}
\documentclass[\mainfilename]{subfiles}

% \graphicspath{{\subfix{./.build/figures/\subfilename/}}}
% \graphicspath{./.build/figures/AM_2C-Aulas_Anotações.4/}
% \tikzset{external/force remake=true} % - remake all

\begin{document}

\mymakesubfile{4}
[AM\,2C]
{Aula 21/10} % Subfile Title
{Aula 21/10} % Part Title

\part*{funções implícitas}
% \begin{definitionBox}1{Função Implícita} % DEF1
    
%     \begin{center}
%         \includegraphics[\width=.9\textwidth]{
%             ./.build/figures/AM_2C-Aulas_Anotações.4/Screenshot 2022-10-21 at 11.52.50.png
%         }
%     \end{center}
    
% \end{definitionBox}

\begin{exampleBox}1m{} % E1
    
    \begin{BM}
        \begin{cases}
            a^3 + u\,b-v = 0
            \\
            b^3 + v\,a-u = 0
        \end{cases}
    \end{BM}

    Defina implicitamente numa viz de (0,1,1,-1) e \textit{a} e \textit{b} como funções de \textit{u} e \textit{v}

    \begin{align*}
        1 \quad & \begin{cases}
            f_1(u,v,a,b) = a^3 + u\,b - v
            \\
            f_2(u,v,a,b) = b^3+v\,a-u
        \end{cases}
        \\
        2 \quad & \{f_1,f_2\}\in c^1(\mathbb{R}^4)
        \\
        3 \quad & f_1(0,1,1,-1) 
        = 1^3 -(-1)
        = 1^3 + 1
        = f_2(0,1,1,-1)
        \\
        4 \quad & 
        \pdv{f_1,f_2}{a,b}(0,1,1,-1)
        = \begin{pmatrix}
            3\,a^2 & u
            \\
            v & 3\,b^2
        \end{pmatrix}
        (0,1,1,-1)
        = \begin{pmatrix}
            3 & 0
            \\
            1 & 3
        \end{pmatrix}
        = 9
    \end{align*}

    \begin{flalign*}
        &
            \begin{cases}
                a = \psi_1(u,v)
                \\
                b = \psi_2(u,v)
            \end{cases}
            &\\&
            \pdv{a}{u}(0,1)
            = -\frac{1}{\Delta}
            \pdv{f_1,f_2}{u,b}
            (0,1,1,-1)
            = -\frac{1}{\Delta}
            \begin{pmatrix}
                b & u
                \\
                -1 & 3\,b^2
            \end{pmatrix}
            (0,1,1,-1)
            = 1/3
            &\\&
            \left\{
                \begin{aligned}
                    \phi_1(u,v)^3 + u\,\phi_2(u,v) -v = 0
                    \\
                    \phi_2(u,v)^3 + v\,\phi_1(u,v) -u = 0
                \end{aligned}
            \right\}
            \implies &\\&
            \implies
            \left\{
                \begin{aligned}
                    3\phi_1(u,v)^2\pdv{\phi_1}{u}(u,v) 
                    + \phi_2(u,v) 
                    + \pdv{\phi_2}{u}(u,v) = 0
                    \\
                    3\phi_2(u,v)^2\pdv{\phi_2}{u}(u,v) 
                    + v\,\pdv{\phi_1(u,v)}{u}(u,v)
                    - u = 0
                \end{aligned}
            \right\}
            \implies &\\&
            \implies
            \left\{
                \begin{aligned}
                    3\pdv{\phi_1}{u}(0,1)
                    - 1 + 0 = 0
                    \\
                    3\pdv{\phi_2}{u}(0,1)
                    +\pdv{\phi_1}{u}(0,1)
                    - 1 
                    =0
                \end{aligned}
            \right\}
            % \implies &\\&
            \implies
            \left\{
                \begin{aligned}
                    \pdv{\phi_1}{u}(0,1) = 1/3
                    \\
                    \pdv{\phi_2}{u}(0,1) = 2/9
                \end{aligned}
            \right\}
        &
    \end{flalign*}

\end{exampleBox}

\begin{definitionBox}1{Teorema da F invérsa} % DEF2
    
    Seja \textit{D} um conjunto aberto de \(\mathbb{R}^n, F:D\subset\mathbb{R}^n\to\mathbb{R}^n\), uma função de classe \(C^1\) em \textit{D}, definida pelas \textit{n} condições \(y_i = F_i(x_1,\dots,x_n)\), com \(i=1,\dots,n\).\\

    Dado \(a\in D\) suponhamos que \(\det(F'(a))=\pdv{F_1,\dots,F_n}{x_1,\dots,x_n}(a)\neq 0\)

    \begin{enumerate}
        \item \(a \in A,F(a)\in B\)
        \item \(F_{/a}(F \text{ restrita a } A)\) é uma bijecção de \textit{A} sobre \textit{B}
        \item A função invérsa de \(F_{/A}\) é de classe \(C^1\) em \textit{B}
        \item \((F^{-1})'(y) = (F'(x))^{-1} \text{, onde } x = F^{-1}(y),y\in B\)
    \end{enumerate}
    
\end{definitionBox}

\begin{exampleBox}1{} % E2
    
    \begin{BM}[align*]
        F:D \subseteq\mathbb{R}^2 & \to\mathbb{R}^2\\
        (x,y) & \mapsto(u,v) = (e^x\,\cos(y),e^x\,\sin(y))
    \end{BM}

    \begin{flalign*}
        &
            F_1,F_2\in C^1(\mathbb{R}^2)
        &\\&
            \begin{pmatrix}
                \pdv{F_1}{x} & \pdv{F_1}{y}\\
                \pdv{F_2}{x} & \pdv{F_2}{y}
            \end{pmatrix}
            (x_0,y_0)
            = \dots
        &
    \end{flalign*}
    
\end{exampleBox}

\begin{exampleBox}1{} % E3
    
    \begin{BM}
        F(0,\pi/4) = \left(
            \frac{\sqrt{2}}{2},
            \frac{\sqrt{2}}{2}
        \right)
    \end{BM}

    \begin{flalign*}
        &
            \begin{bmatrix}
                \pdv{x}{u} & \pdv{x}{v}\\
                \pdv{y}{u} & \pdv{y}{v}
            \end{bmatrix}
            _{
                \left(
                    \frac{\sqrt{2}}{2},
                    \frac{\sqrt{2}}{2}
                \right)
            }
            = - J_{F^{-1}}
            \,\left(
                \frac{\sqrt{2}}{2},
                \frac{\sqrt{2}}{2}
            \right)
            \dots
        &
    \end{flalign*}
    
\end{exampleBox}

\end{document}
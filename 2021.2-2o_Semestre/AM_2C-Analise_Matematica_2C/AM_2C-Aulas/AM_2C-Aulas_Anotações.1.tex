% !TEX root = ./AAM_2C-Aulas_Anotações.1.tex
\providecommand\mainfilename{"./AM_2C-Aulas_Anotações.tex"}
\providecommand \subfilename{}
\renewcommand   \subfilename{"./AM_2C-Aulas_Anotações.1.tex"}
\documentclass[\mainfilename]{subfiles}

% \graphicspath{{\subfix{../images/}}}

\begin{document}

\mymakesubfile{1}
{AM\,2C Aula 14/mar Anotações}
{AM\,2C Aula 14/mar Anotações}

% Slide 1 e 2
\part*{Geometria em R2 e revisão AM 1 (Slide 1)}
\begin{definitionBox}1{Ponto Fronteiro}
    
    \begin{BM}
        X_0\in\mathbb{R}^n:X_0\nin\{ \exterior A\cup\interior A \}
    \end{BM}
    
\end{definitionBox}

\begin{definitionBox}1{Fronteira}
    \begin{BM}
        \fronteira A = \rho A = \dots
    \end{BM}
\end{definitionBox}

\begin{exampleBox}1{}
    
    \begin{flalign*}
        &
            X_0\in A\cap A'\therefore\dots
            X_0\text{ não é ponto isolado}
        &
    \end{flalign*}

\end{exampleBox}

\begin{sectionBox}1{Caracterização de Conjuntos}
    
    \begin{sectionBox}2{Conjunto Limitado}
        
        \begin{BM}
            A\subset B(x,r)\quad
            \begin{aligned}
            \ldiv{}
                r>0
            \ldiv{}
                x\in\mathbb{R}^n
            \end{aligned}
        \end{BM}
        
    \end{sectionBox}
    
\end{sectionBox}

\begin{questionBox}1{}
    
    \begin{BM}
        A = \{
            X\in\mathbb{R}^n : 0 < \lVert X \rVert \leq 1
        \}
        \cup \{(0,2)\}
    \end{BM}

    [Inserir função graficos]
    
\end{questionBox}

\part*{Conexão}

\begin{sectionBox}1{Conjuntos Separados}
    
    \begin{exampleBox}*1{}
        
        \begin{BM}
            \frac{x^2}{4}+y^2 \leq 1\land x^2>y^2
        \end{BM}

        \begin{center}
            \begin{tikzpicture}
                \begin{axis}
                    [
                        axis lines = {middle},
                        axis on top,
                    ]
                    
                    \addplot [
                        % opacity = 0.5,
                        fill opacity = 0.2,
                        DarkGreen\Light,
                        fill = DarkGreen\Light,
                        domain y = 0:1,
                    ] gnuplot { \sqrt{1-0.25*x**2} }
                    
                    \addplot [
                        % opacity = 0.5,
                        fill opacity = 0.2,
                        DarkGreen\Light,
                        fill = DarkGreen\Light,
                        domain y = -1:0,
                    ] gnuplot { 0\sqrt{1-0.25*x**2} }
                    
                    
                \end{axis}
            \end{tikzpicture}
        \end{center}
        
        \begin{exampleBox}*2{\(\bar{A}\cap B=\emptyset\)?}
            
            \begin{flalign*}
                &
                    0\in\bar{A}\land 0\in B\therefore \bar{A}\cap B=\{ 0 \}
                &
            \end{flalign*}
            
        \end{exampleBox}

        \begin{exampleBox}*2{\( \bar{A}\cap B=\emptyset \)?}
            
            \begin{flalign*}
                &
                    0\in\bar{A}\land 0\in B\therefore \bar{A}\cap B=\{ 0 \}
                &
            \end{flalign*}
            
        \end{exampleBox}

        \begin{exampleBox}*2{\( A\cap \bar{B}=\emptyset \)?}
            
            Mesma dem da anterior
            
        \end{exampleBox}

    \end{exampleBox}
    
\end{sectionBox}


\end{document}
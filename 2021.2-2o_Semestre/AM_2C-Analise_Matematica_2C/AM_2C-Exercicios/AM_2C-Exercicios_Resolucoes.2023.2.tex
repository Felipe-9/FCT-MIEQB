% !TEX root = ./AM_2C-Exercicios_Resolucoes.2023.2.tex
\providecommand\mainfilename{"./AM_2C-Exercicios_Resolucoes.tex"}
\providecommand \subfilename{}
\renewcommand   \subfilename{"./AM_2C-Exercicios_Resolucoes.2023.2.tex"}
\documentclass[\mainfilename]{subfiles}

% \tikzset{external/force remake=true} % - remake all

\begin{document}

% \graphicspath{{\subfix{./.build/figures/AM_2C-Exercicios_Resolucoes.2023.2}}}
% \tikzsetexternalprefix{./.build/figures/AM_2C-Exercicios_Resolucoes.2023.2/}

\mymakesubfile{2}
[AM 2C]
{Exericios: Coordenadas polares, esféricas e cilíndricas} % Subfile Title
{Coordenadas polares, esféricas e cilíndricas} % Part Title

\begin{questionBox}1{ % Q1
    Escreva em coordenadas polares e represente geometricamente o conjunto definido pelas seguintes inequacções:
    \begin{BM}
        0< x^2+y^2\leq 1 
        \quad\land\quad
        -x\,\sqrt{3}\leq y\leq x\,\frac{\sqrt{3}}{3}
    \end{BM}
} % Q1
    \answer{}
    \begin{flalign*}
        &
            \left\{
                (x,y)\in\mathbb{R}^2:
                \left\{
                    \begin{aligned}
                        &
                            0< x^2+y^2\leq 1
                        \ldiv{}
                            -x\,\sqrt{3}\leq y\leq x\,\frac{\sqrt{3}}{3}
                        &
                    \end{aligned}
                \right\}
            \right\}
            = &\\&
            = \left\{
                (r,\theta)\in\mathbb{R}^+\times\myrange{-\pi,\pi}:
                \left\{
                    \begin{aligned}
                        &
                            (x,y)=(r\cos\theta,r\sin\theta)
                        \ldiv{}
                            (x,y)\in\mathbb{R}^2
                        \ldiv{}
                            0< x^2+y^2\leq 1
                        \ldiv{}
                            -x\,\sqrt{3}\leq y\leq x\,\frac{\sqrt{3}}{3}
                        &
                    \end{aligned}
                \right\}
            \right\}
            = &\\&
            = \left\{
                (r,\theta)\in\mathbb{R}^+\times\myrange{-\pi,\pi}:
                \left\{
                    \begin{aligned}
                        &
                            0<
                            \left(r\cos\theta\right)^2
                            +\left(r\sin\theta\right)^2
                            \leq 1
                        \ldiv{}
                            -r\cos\theta\,\sqrt{3}
                            \leq 
                            r\sin\theta
                            \leq 
                            r\cos\theta\,\frac{\sqrt{3}}{3}
                        &
                    \end{aligned}
                \right\}
            \right\}
            = &\\&
            = \left\{
                (r,\theta)\in\mathbb{R}^+\times\myrange{-\pi,\pi}:
                \left\{
                    \begin{aligned}
                        &
                            0<
                            r^2\left(
                                \cos^2(\theta)+\sin^2(\theta)
                            \right)
                            \leq 1
                        \ldiv{}
                            -\sqrt{3}
                            \leq 
                            \tan\theta
                            \leq 
                            \frac{\sqrt{3}}{3}
                        &
                    \end{aligned}
                \right\}
            \right\}
            = &\\&
            = \left\{
                (r,\theta)\in\mathbb{R}^+\times\myrange{-\pi,\pi}:
                \left\{
                    \begin{aligned}
                        &
                            0<r\leq 1
                        \ldiv{}
                            -\pi/3
                            \leq 
                            \theta
                            \leq 
                            \pi/6
                        &
                    \end{aligned}
                \right\}
            \right\}
        &
    \end{flalign*}
    % \begin{center}
    %     \tikzset{external/remake next=true}
    %     % \pgfplotsset{height=7cm, width= .6\textwidth}
    %     \begin{tikzpicture}
    %     \begin{axis}
    %         [
    %             % xmajorgrids = true,
    %             % legend pos  = north west
    %             % domain=0:4,
    %             % xlabel={},
    %             % ylabel={},
    %         ]
    %         % Legends
    %         % \addlegendimage{empty legend}
    %         % \addlegendentry[Red]{\( x \)}
            
    %         % % Plot from csv file
    %         % \addplot[smooth, thick, mark=*] % mesh for colormap
    %         % table[
    %         %     col sep=comma,  % csv: comma,
    %         %     header=true,
    %         %     x index=3,      % x column on file
    %         %     y index=4,      % y column on file
    %         %     point meta=x,   % value to colormap
    %         % ]{  file.csv };
            
    %         % Plot from equation
    %         \addplot[
    %             smooth,
    %             thick,
    %             % Graph,
    %             x=cos(t),
    %             y=sin(t),
    %             domain  = -\pi/3:\pi/6,
    %             samples = \mysampledensityFancy,
    %         ]{ x };
            
    %     \end{axis}
    %     \end{tikzpicture}
    % \end{center}
\end{questionBox}

\begin{questionBox}1{ % Q2
    Escreva em coordenadas cartesianas e represente geometricamente os conjuntos definidos em coordenadas polares por:
} % Q2
    \begin{questionBox}2{} % Q2.1
        \begin{BM}
            \theta=\pi
        \end{BM}
        \answer{}
        \begin{flalign*}
            &
                \left\{
                    (x,y)\in\mathbb{R}^2:
                    \left\{
                        \begin{aligned}
                            &
                                (x,y)=(r\,\cos\theta,r\,\sin\theta)
                            \ldiv{}
                                (r,\theta)
                                \in\mathbb{R}^+\times\myrange{0,2\,\pi}
                            \ldiv{}
                                \theta=\pi
                            &
                        \end{aligned}
                    \right\}
                \right\}
                = &\\&
                = \left\{
                    (x,y)\in\mathbb{R}^2:
                    \left\{
                        \begin{aligned}
                            &
                                (x,y)=(r\,\cos\pi,r\,\sin\pi)
                            \ldiv{}
                                r\in\mathbb{R}^+
                            &
                        \end{aligned}
                    \right\}
                \right\}
                = &\\&
                = \left\{
                    (x,y)\in\mathbb{R}^2:
                    \left\{
                        \begin{aligned}
                            &
                                (x,y)=(-r,0)
                            \ldiv{}
                                r\in\mathbb{R}^+
                            &
                        \end{aligned}
                    \right\}
                \right\}
                = &\\&
                = \left\{
                    (x,y)\in\mathbb{R}^-\times\{0\}
                \right\}
            &
        \end{flalign*}
    \end{questionBox}

    \begin{questionBox}2{} % Q2.2
        \begin{BM}
            \pi\leq\theta<\pi\,5/3
            \quad\land\quad
            1<r\leq3
        \end{BM}
        \answer{}
        \begin{flalign*}
            &
                \left\{
                    (x,y)\in\mathbb{R}^2:
                    \left\{
                        \begin{aligned}
                            &
                                (r,\theta)=\left(
                                    \sqrt{x^2+y^2},
                                    \tan^{-1}(y/x)
                                \right),x\neq0
                            \ldiv{}
                                (r,\theta)\in\mathbb{R}^+\times\myrange{0,2\,\pi}
                            \ldiv{}
                                \pi\leq\theta<\pi\,5/3
                            \ldiv{}
                                1<r\leq3
                            &
                        \end{aligned}
                    \right\}
                \right\}
                = &\\&
                = \left\{
                    (x,y)\in\mathbb{R}^2:
                    \left\{
                        \begin{aligned}
                            &
                                0\leq
                                y/x
                                < -\sqrt{3}
                            \ldiv{}
                                1<x^2+y^2\leq3
                            &
                        \end{aligned}
                    \right\}
                \right\}
                = &\\&
                = \left\{
                    (x,y)\in\mathbb{R}^2:
                    \left\{
                        \begin{aligned}
                            &
                                0\leq
                                y
                                < -x\sqrt{3}
                            \ldiv{}
                                1<x^2+y^2\leq3
                            &
                        \end{aligned}
                    \right\}
                \right\}
            &
        \end{flalign*}
    \end{questionBox}
\end{questionBox}

\begin{questionBox}1{ % Q3
    Escreva as seguintes equações em coordenadas polares:
} % Q3
    
    \begin{multicols}{2}
        
        \begin{questionBox}2{} % Q3.1
            \begin{BM}
                y=3
            \end{BM}
            \answer{}
            \begin{flalign*}
                &
                    r\,\sin\theta=3
                &
            \end{flalign*}
        \end{questionBox}
    
        \begin{questionBox}2{} % Q3.2
            \begin{BM}
                x^2+y^2=9
            \end{BM}
            \answer{}
            \begin{flalign*}
                &
                    r=3
                &
            \end{flalign*}
        \end{questionBox}
    
        \begin{questionBox}2{} % Q3.3
            \begin{BM}
                x^2+(y-2)^2=4
            \end{BM}
            \answer{}
            \begin{flalign*}
                &
                    r^2+4-4\,r\,\sin\theta
                    =4
                    \implies &\\&
                    \implies
                    r=4\,\sin\theta
                &
            \end{flalign*}
        \end{questionBox}
    
        \begin{questionBox}2{} % Q3.4
            \begin{BM}
                x=y
            \end{BM}
            \answer{}
            \begin{flalign*}
                &
                    \cos\theta=\sin\theta
                    % \implies &\\&
                    \implies
                    \myvert{\theta}=\pi/4
                &
            \end{flalign*}
        \end{questionBox}
    
        \begin{questionBox}2{} % Q3.5
            \begin{BM}
                x^2-y^2=4
            \end{BM}
            \answer{}
            \begin{flalign*}
                &
                    r^2(\cos^2\theta-\sin^2\theta)=4
                &
            \end{flalign*}
        \end{questionBox}
    
        \begin{questionBox}2{} % Q3.6
            \begin{BM}
                (x^2+y^2)^2=2\,x\,y
            \end{BM}
            \answer{}
            \begin{flalign*}
                &
                    r^2=2\,\cos\theta\,\sin\theta
                &
            \end{flalign*}
        \end{questionBox}

    \end{multicols}

\end{questionBox}

\begin{questionBox}1{ % Q4
    Identifique as curvas, representadas em coordenadas polares, e escreva as equações em coordenadas cartesianas:
} % Q4
    
    \begin{multicols}{2}
        
        \begin{questionBox}2{} % Q4.1
            \begin{BM}
                r\,\cos\theta=4
            \end{BM}
            \answer{}
            \begin{flalign*}
                &
                    x=4
                &
            \end{flalign*}
        \end{questionBox}
    
        \begin{questionBox}2{} % Q4.2
            \begin{BM}
                r=\frac{2}{1-\cos\theta}
            \end{BM}
            \answer{}
            \begin{flalign*}
                &
                    2
                    = r-r\cos\theta
                    = &\\&
                    = \sqrt{x^2+y^2}
                    - x
                    \implies &\\&
                    \implies
                    x^2 + 4\,x + 4
                    = x^2 + y^2
                    \implies &\\&
                    \implies
                    y^2
                    = 4\,x+4
                &
            \end{flalign*}
        \end{questionBox}
    
        \begin{questionBox}2{} % Q4.3
            \begin{BM}
                r=4\,\sin(\theta+\pi)
            \end{BM}
            \answer{}
            \begin{flalign*}
                &
                    x^2+y^2
                    =-4\,y
                    \implies &\\&
                    \implies
                    (x-2)(x+2)=-(y+4)^2
                &
            \end{flalign*}
        \end{questionBox}
    
        \begin{questionBox}2{} % Q4.4
            \begin{BM}
                \tan\theta=2
            \end{BM}
            \answer{}
            \begin{flalign*}
                &
                    y=2\,x
                &
            \end{flalign*}
        \end{questionBox}
    
        \begin{questionBox}2{} % Q4.5
            \begin{BM}
                \theta=\pi\,5/4
            \end{BM}
            \answer{}
            \begin{flalign*}
                &
                    y = x
                    \land x\leq 0
                &
            \end{flalign*}
        \end{questionBox}

    \end{multicols}

\end{questionBox}

\begin{questionBox}1{ % Q5
    Seja
} % Q5
    \begin{BM}
        f(x,y)
        =\sqrt{4-x^2-y^2} + \log{y}
    \end{BM}

    \begin{questionBox}2{ % Q5.1
        Determine o dominio \textit{D} de \textit{f}
    } % Q5.1
        \answer{}
        \begin{flalign*}
            &
                D_{f}
                = \left\{
                    (x,y)\in\mathbb{R}^2:
                    \left\{
                        \begin{aligned}
                            &
                                4-x^2-y^2\geq0
                            \ldiv{}
                                y>0
                            &
                        \end{aligned}
                    \right\}
                \right\}
                = &\\&
                = \left\{
                    (x,y)\in\mathbb{R}\times\mathbb{R}^{0+}
                    :4\geq x^2+y^2
                \right\}
            &
        \end{flalign*}
    \end{questionBox}

    \begin{questionBox}2{ % Q5.2
        Represente o conjunto \textit{D} em coordenadas polares
    } % Q5.2
        \answer{}
        \begin{flalign*}
            &
                D_{f} 
                = \left\{
                    (x,y)\in\mathbb{R}\times\mathbb{R}^{0+}
                    :4\geq x^2+y^2
                \right\}
                = &\\&
                = \left\{
                    (r,\theta)\in\mathbb{R}^+\times\myrange{-\pi,\pi}
                    : \left\{
                        \begin{aligned}
                            &
                                (r,\theta)=(\sqrt{x^2+y^2},\tan^{-1}(y/x))
                            \ldiv{}
                                (x,y)\in\mathbb{R}\times\mathbb{R}^{0+}
                            \ldiv{}
                                4\geq x^2+y^2
                            &
                        \end{aligned}
                    \right\}
                \right\}
                = &\\&
                = \left\{
                    (r,\theta)\in\mathbb{R}^+\times\myrange{-\pi,\pi}
                    : \left\{
                        \begin{aligned}
                            &
                                r\sin\theta>0
                            \ldiv{}
                                2\geq r
                            &
                        \end{aligned}
                    \right\}
                \right\}
                = &\\&
                = \left\{
                    (r,\theta)\in\mathbb{R}^+\times\myrange{-\pi,\pi}
                    : \left\{
                        \begin{aligned}
                            &
                                r > 0
                            \ldiv{}
                                0 < \theta < \pi
                            \ldiv{}
                                2\geq r
                            &
                        \end{aligned}
                    \right\}
                \right\}
                = &\\&
                = \left\{
                    (r,\theta)
                    \in   \myrange l{0,2}
                    \times\myrange*{0,\pi}
                \right\}
            &
        \end{flalign*}
    \end{questionBox}
\end{questionBox}

\begin{questionBox}1{ % Q6
    Descreva em coordenadas cilíndricas:
} % Q6
    \begin{questionBox}2{ % Q6.1
        o sólido de \(\mathbb{R}^3\) definido pelas condições
    } % Q6.1
        body
    \end{questionBox}

    \begin{questionBox}2{ % Q6.2
        o sólido em \(\mathbb{R}^3\) limitado pelo parabolóide de equação \(x^2+y^2 = 1-z\) e pela folha superior da superfície cónica \((z + 1)^2 = x^2 + y^2\);
    } % Q6.2
        \answer{}
        \begin{flalign*}
            &
                1-z
                = x^2+y^2
                = (z+1)^2
                \implies
                1-z=(z+1)^2
                \implies &\\&
                \implies
                z(z+3)=0
                \implies
                z\in\{0,-3\}
                ; &\\[3ex]&
                \begin{cases}
                    z
                    =1-x^2+y^2
                    =1-r^2
                    \\
                    z
                    =-1+\sqrt{x^2+y^2}
                    =-1+r
                \end{cases}
                &\\&
                \begin{cases}
                    \theta\in\myrange{0,2\,\pi}
                    \\
                    r\in\myrange{0,1}:r=\sqrt{x^2+y^2}
                    \\
                    z\in\myrange{-1+r,1-r^2}
                \end{cases}
                &\\[3ex]&
                \therefore
                \left\{
                    \begin{aligned}
                        \theta&\in\myrange{0,2\,\pi}
                        \\
                        r&\in\myrange{0,1}
                        \\
                        z&\in\myrange{-1+r,1-r^2}
                    \end{aligned}
                \right\}
            &
        \end{flalign*}
    \end{questionBox}
\end{questionBox}

\setcounter{question}{8}
\begin{questionBox}1{ % Q9
    Escreva em coordenadas esféricas:
} % Q9
    \begin{questionBox}2{ % Q9.2
        o sólido em \(\mathbb{R}^3\) limitado pelas superfícies
        \begin{BM}
            z=\sqrt{1-x^2-y^2}
            \quad\text{e}\quad
            z=\sqrt{x^2+y^2}
        \end{BM}
    } % Q9.2
        \answer{}
        \begin{flalign*}
            &
                \begin{cases}
                    x = \rho\,\sin{\phi}\,\cos{\theta}
                    \\
                    y = \rho\,\sin{\phi}\,\sin{\theta}
                    \\
                    z = \rho\,\cos{\phi}
                \end{cases}
                &\\[3ex]&
                \begin{cases}
                    z=\sqrt{1-x^2-y^2}
                    \text{ é uma circuferencia centrada em }(0,0)
                    \\
                    z=\sqrt{x^2+y^2}
                    \text{ é uma paraboloide}
                \end{cases}
                &\\[3ex]&
                z = \rho\,\cos{\phi}
                = \sqrt{x^2+y^2}
                = \sqrt{
                    (\rho\,\sin{\phi}\,\cos{\theta})^2
                    +(\rho\,\sin{\phi}\,\sin{\theta})^2
                }
                = &\\&
                = \rho\,\sin\phi\sqrt{\sin{\theta}
                    (\cos{\theta})^2
                    +(\sin{\theta})^2
                }
                = \rho\,\sin\phi
                \implies &\\&
                \implies
                \phi=\tan^{-1}{1}=\pi/4
                ; &\\[3ex]&
                \left\{
                    \begin{aligned}
                        \rho&\in\myrange{0,1}
                        \\
                        \phi&\in\myrange{0,\pi/4}
                        \\
                        \theta&\in\myrange{0,2\,\pi}
                    \end{aligned}
                \right\}
            &
        \end{flalign*}
    \end{questionBox}
\end{questionBox}

\end{document}
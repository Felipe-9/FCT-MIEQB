% !TEX root = ./AM_2C-Exercicios_Resolucoes.2023.4.tex
\providecommand\mainfilename{"./AM_2C-Exercicios_Resolucoes.tex"}
\providecommand \subfilename{}
\renewcommand   \subfilename{"./AM_2C-Exercicios_Resolucoes.2023.4.tex"}
\documentclass[\mainfilename]{subfiles}

% \tikzset{external/force remake=true} % - remake all

\begin{document}

% \graphicspath{{\subfix{./.build/figures/AM_2C-Exercicios_Resolucoes.2023.4}}}
% \tikzsetexternalprefix{./.build/figures/AM_2C-Exercicios_Resolucoes.2023.4/graphics/}

\mymakesubfile{4}
[AM 2C]
{Curvas e funções vetoriais} % Subfile Title
{Curvas e funções vetoriais} % Part Title

\setcounter{question}{4}

\begin{questionBox}1{ % Q5
    Determine uma representação paramétrica da curva indicada e a reta tangente à curva referida no ponto \(P_0\):
} % Q5
    \begin{questionBox}2{ % Q5.1
        \begin{BM}
            x^2/4+2\,y^2=1
            \quad\text{e}\quad
            z=2,
            P_0=\left(\sqrt{2},1/2,2\right)
        \end{BM}
    } % Q5.1
    \end{questionBox}

    \begin{questionBox}2{ % Q5.2
        \begin{BM}
            z
            =x^2 + y^2/4
            \quad\text{e}\quad
            z=4\,x,P_0 = (4,0,16).
        \end{BM}
    } % Q5.2
        \answer{}
        \begin{flalign*}
            &
                z
                = x^2 + y^2/4
                = 4\,x
                % \implies &\\&
                \implies
                \begin{cases}
                    \left(
                        \frac{x-2}{2}
                    \right)^2
                    + \left(
                        \frac{y^2}{4}
                    \right)^2
                    \\
                    z=4\,x
                \end{cases}
                % \implies &\\&
                \implies
                \begin{cases}
                    x = 2+2\,\cos{t}
                    \\
                    y=4\,\sin{t}
                    \\
                    z=4\,x = 8+8\,\cos{t}
                    \\
                    t\in\myrange{0,2\,\pi}
                \end{cases};
                &\\[3ex]&
                \vv{r}(t)
                = \left(
                    \begin{aligned}
                        &
                            \hat{\imath}
                            \,(2+2\,\cos{t})
                        &+\\+&
                            \hat{\jmath}
                            \,(4\,\sin{t})
                        &+\\+&
                            \hat{k}
                            \,(4\,x = 8+8\,\cos{t})
                        &
                    \end{aligned}
                \right)
            &
        \end{flalign*}
    \end{questionBox}
\end{questionBox}

\begin{questionBox}1{ % Q6
    Para cada uma das seguintes curvas determine a reta tangente no ponto \(P_0\):
} % Q6
    \begin{questionBox}2{ % Q6.1
        \begin{BM}
            y = 3\,x-2\text{ em }\mathbb{R}^2,P_0=(2/3,0)
        \end{BM}
    } % Q6.1
    \end{questionBox}

    \begin{questionBox}2{ % Q6.2
        \begin{BM}
            x=y^2=z^2+1
            \text{ em }\mathbb{R}^2, P_0=(1,1,0)
        \end{BM}
    } % Q6.2
    \end{questionBox}
\end{questionBox}

\begin{questionBox}1{ % Q7
    Seja a curva \textit{C} a curva em \(\mathbb{R}^3\) definida pelas equações paramétricas
    \begin{BM}
        \begin{cases}
                x=\cos(2\,t)
            \\  y=(2/3)\sqrt{t^3}
            \\  z=\sin(2\,t)
            \\  t\in\myrange{0,4}
        \end{cases}
    \end{BM}
    Suponha que a curva \textit{C} corresponde à trajetória de uma partícula P.
} % Q7
    \begin{questionBox}2{ % Q7.1
        Calcule o comprimento do caminho percorrido pela partícula P.
    } % Q7.1
        \answer{}
        \begin{flalign*}
            &
                \vv{r}(t)
                = \left(
                    \begin{aligned}
                        \cos(2\,t),
                    \\  (2/3)\sqrt{t^3},
                    \\  \sin(2\,t)
                    \end{aligned}
                \right)
                \implies &\\[3ex]&
                \implies 
                \vv{r}'(t)
                = \left(
                    \begin{aligned}
                        - 2\,\sin(2\,t),
                    \\  t^{1/2},
                    \\  2\,\cos(2\,t)
                    \end{aligned}
                \right)
                \implies &\\[3ex]&
                \implies
                \myVert{\vv{r}'(t)}
                = \sqrt{\left(
                    \begin{aligned}
                        &
                            (- 2\,\sin(2\,t))^2
                        &+\\+&
                            (t^{1/2})^2
                        &+\\+&
                            (2\,\cos(2\,t))^2
                        &
                    \end{aligned}
                \right)}
                = \sqrt{4+t}
                \implies &\\[3ex]&
                \implies
                \begin{cases}
                    1=t^2+1
                    \\
                    1=(t^2+1)^{1/3}
                    \\
                    0=t
                \end{cases}
                , t=0
                ; &\\[3ex]&
                \vv{r}'(0)
                = \hat{k}
                ; &\\[6ex]&
                \therefore
                \text{Reta tangente}
                = P_0+\lambda\,\myVert{\vv{r}'(P_0)}
                = \begin{bmatrix}
                    1\\1\\0
                \end{bmatrix}
                + \lambda
                \begin{bmatrix}
                    0\\0\\1
                \end{bmatrix}
            &
        \end{flalign*}
    \end{questionBox}

    \begin{questionBox}2{ % Q7.2
        Determine uma equação da reta tangente à trajetória no ponto \(0,\sqrt{\pi^3}/12,1\).
    } % Q7.2
    \end{questionBox}
\end{questionBox}

\setcounter{question}{12}

\begin{questionBox}1{ % Q13
    Considere a função vetorial
    \begin{BM}
        \vv{\sigma}(t)
        = \frac{t}{2}\,\hat{\imath}
        + \cos(\sqrt{2}\,t)\,\hat{\jmath}
        + \sin(\sqrt{2}\,t)\,\hat{k},
        \quad t\in\myrange{0,2\,\pi}
    \end{BM}
    e designe por \textit{C} a curva paramétrica definida por \(\vv{\sigma}\). Suponha que \textit{C} corresponde à trajetória de uma partícula \textit{P}.
} % Q13
    \begin{questionBox}2{ % Q13.1
        Determine o vetor tangente unitário em cada ponto da curva \textit{C}
    } % Q13.1
        \answer{}
        \begin{flalign*}
            &
                T(t) 
                = \frac{\vv{\sigma}(t)}{\myVert{\vv{\sigma}(t)}}
                = \frac{
                    \frac{t}{2}\,\hat{\imath}
                    + \cos(\sqrt{2}\,t)\,\hat{\jmath}
                    + \sin(\sqrt{2}\,t)\,\hat{k}
                }{
                    \sqrt{
                        \left(
                            \frac{t}{2}
                        \right)^2
                        + \left(
                            \cos(\sqrt{2}\,t)
                        \right)^2
                        + \left(
                            \sin(\sqrt{2}\,t)
                        \right)^2
                    }
                }
            &
        \end{flalign*}
    \end{questionBox}

    \begin{questionBox}2{ % Q13.2
        Calcule o comprimento da curva \textit{C}
    } % Q13.2
        \answer{}
    \end{questionBox}

    \begin{questionBox}2{ % Q13.3
        Determine a função comprimento de arco da curva \textit{C}.
    } % Q13.3
        body
    \end{questionBox}

    \begin{questionBox}2{ % Q13.4
        Determine a parametrização de \textit{C} por comprimento de arco \(s(t)\) verificando \(s(0)=0\)
    } % Q13.4
        \answer{}
        \begin{flalign*}
            &
                \vv{\sigma}\left(
                    \frac{2}{3}\,s
                \right)
                = \frac{s\,2/3}{2}\,\hat{\imath}
                + \cos(\sqrt{2}\,s\,2/3)\,\hat{\jmath}
                + \sin(\sqrt{2}\,s\,2/3)\,\hat{k}
            &
        \end{flalign*}
    \end{questionBox}

    \begin{questionBox}2{ % Q13.5
        Determine as posições inicial e final da partícula \textit{P}
    } % Q13.5
        \answer{}
        \begin{flalign*}
            &
                P_i
                =\vv{\sigma}(0)
                = \hat{\jmath}
                \implies (0,1,0)
                ; &\\[3ex]&
                P_f
                = \vv{\sigma}(2\,\pi)
            &
        \end{flalign*}
    \end{questionBox}

    \begin{questionBox}2{ % Q13.6
        Determine a dinstância percorrida pela partícula \textit{P} para \(t=\pi/2\).
    } % Q13.6
        \answer{}
        \begin{flalign*}
            &
                s(\pi/2)
                =(3/2)\,\pi/2
                = (3/4)\,\pi
            &
        \end{flalign*}
    \end{questionBox}

    \begin{questionBox}2{ % Q13.7
        Em que ponto se encontra a partícula \(P\) após ter percorrido uma distância de \(\pi/2\sqrt{2}\)
    } % Q13.7
        \answer{}
        \begin{flalign*}
            &
                \vv{\sigma}(t): s(t)=(3/2)\,t=\pi/2\sqrt{2}
                \implies &\\[3ex]&
                \implies
                t
                = (2/3)\,\pi/2\sqrt{2}
                = (\sqrt{2}/6)\,\pi
                \implies &\\[3ex]&
                \implies
                \vv{\sigma}\left(
                    (\sqrt{2}/6)\,\pi
                \right)
                = \dots
            &
        \end{flalign*}
    \end{questionBox}
\end{questionBox}

\end{document}
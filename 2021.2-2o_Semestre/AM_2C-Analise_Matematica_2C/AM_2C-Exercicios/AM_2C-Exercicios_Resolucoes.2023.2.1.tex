% !TEX root = ./AM_2C-Exercicios_Resolucoes.2023.2.1.tex
\providecommand\mainfilename{"./AM_2C-Exercicios_Resolucoes.tex"}
\providecommand \subfilename{}
\renewcommand   \subfilename{"./AM_2C-Exercicios_Resolucoes.2023.2.1.tex"}
\documentclass[\mainfilename]{subfiles}

% \tikzset{external/force remake=true} % - remake all

\begin{document}

% \graphicspath{{\subfix{./.build/figures/AM_2C-Exercicios_Resolucoes.2023.2.1}}}
% \tikzsetexternalprefix{./.build/figures/AM_2C-Exercicios_Resolucoes.2023.2.1/graphics/}

\mymakesubfile{1}
[AM 2C]
{Exercicios} % Subfile Title
{Exercicios} % Part Title

\setcounter{question}{2}
\begin{questionBox}1{ % Q3
    \begin{BM}
        A = (1,-1,2),
        \quad B=(2,1,-1),
        \quad C=(1,1,5)
    \end{BM}
} % Q3
    \begin{questionBox}2{ % Q3.1
        Plano definido por \textit{A},\textit{B} e \textit{C}
    } % Q3.1
        \begin{flalign*}
            &
                \vv{A\,B} = B-A = (1,2,-3);
                \quad
                \vv{A\,C} = C-A = (0,2,3);
                &\\&
                \mathcal{P}
                = \left\{
                    (x,y,z)\in\mathbb{R}^3
                    : (x,y,z) 
                    = (1,-1,2)
                    + \lambda\,(1,2,-3)
                    + \mu\,(0,2,3)
                \right\}
                = &\\&
                = \left\{
                    (x,y,z)\in\mathbb{R}^3
                    : 
                    \left(
                        \begin{aligned}
                            &
                                x = 1 + \lambda
                            &\\&
                                y = -1+2\,\lambda+2\,\mu
                            &\\&
                                z = 2-3\,\lambda+3\,\mu
                            &
                        \end{aligned}
                    \right)
                \right\}
                ;&\\&
                \begin{cases}
                    \lambda = \dots
                \end{cases}
            &
        \end{flalign*}
    \end{questionBox}
\end{questionBox}

\setcounter{question}{5}

\begin{questionBox}1{ % Q6
    question
} % Q6
    \begin{questionBox}2{ % Q6.1
        \begin{BM}
            x^2+2\,x+y^2-2\,y=2
        \end{BM}
    } % Q6.1
        \begin{flalign*}
            &
                2
                = x^2+2\,x+y^2-2\,y
                = (x^2+2\,x+1)
                + (y^2-2\,y+1)-2
                = &\\&
                = (x+1)^2
                + (y-1)^2-2
                \implies &\\[3ex]&
                \implies
                4
                = (x+1)^2
                + (y-1)^2
                % \implies &\\&
                \implies
                \begin{cases}
                    \text{Circuferencia}
                    \\ r=\sqrt{4}=2
                    \\ X_0=(-1,1)
                \end{cases}
            &
        \end{flalign*}
    \end{questionBox}

    \begin{questionBox}2{ % Q6.2
        \begin{BM}
            a\,x^2+18\,x-y^2+2\,y-8=0
        \end{BM}
    } % Q6.2
        \begin{flalign*}
            &
                0
                = 9\,x^2+18\,x
                - y^2+2\,y
                - 8
                = 9(x^2+2\,x+1) 
                - 9
                - (y^2+2\,y+1) + 1
                - 8
                = &\\&
                = 9(x+1)^2 
                - (y+1)^2
                - 16
                \implies &\\&
                \implies
                1
                = \left(
                    \frac{x+1}{4/3}
                \right)^2
                - \left(
                    \frac{y+1}{4}
                \right)^2
                \implies &\\[3ex]&
                \implies
                \begin{cases}
                    \text{Hiperbole}
                \end{cases}
            &
        \end{flalign*}
    \end{questionBox}
\end{questionBox}

\end{document}
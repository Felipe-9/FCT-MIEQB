% !TEX root = ./AM_2C-Exercicios_Resolucoes.2023.5.tex
\providecommand\mainfilename{"./AM_2C-Exercicios_Resolucoes.tex"}
\providecommand \subfilename{}
\renewcommand   \subfilename{"./AM_2C-Exercicios_Resolucoes.2023.5.tex"}
\documentclass[\mainfilename]{subfiles}

% \tikzset{external/force remake=true} % - remake all

\begin{document}

% \graphicspath{{\subfix{./.build/figures/AM_2C-Exercicios_Resolucoes.2023.5}}}
% \tikzsetexternalprefix{./.build/figures/AM_2C-Exercicios_Resolucoes.2023.5/graphics/}

\mymakesubfile{5}
[AM 2C]
{Limites, Continuidade e cálculo diferencial} % Subfile Title
{Limites, Continuidade e cálculo diferencial} % Part Title

\setcounter{question}{1}

\begin{questionBox}1{ % Q2
    Estude a existencia dos seguintes limites:
} % Q2
    \setcounter{subquestion}{4}
    \begin{questionBox}2{ % Q2.5
        \begin{BM}
            \lim_{(x,y)\to(0,0)}{
                \frac{
                    4\,x^2+3\,y^2+x^3\,y^3
                }{
                    x^2+y^2+x^4+y^4
                }
            }
        \end{BM}
    } % Q2.5
        \answer{}
        \begin{flalign*}
            &
                % \lim_{(x,y)\to(0,0)}{
                %     \frac{
                %         4\,x^2+3\,y^2+x^3\,y^3
                %     }{
                %         x^2+y^2+x^4+y^4
                %     }
                % }
                = \lim_{x\to0}{
                    \lim_{y\to0}{
                        \frac{
                            4\,x^2+3\,y^2+x^3\,y^3
                        }{
                            x^2+y^2+x^4+y^4
                        }
                    }
                }
                = \lim_{x\to0}{
                    \frac{
                        4\,x^2
                    }{
                        x^2+x^4
                    }
                }
                = &\\&
                = \lim_{x\to0}{
                    \frac{
                        4
                    }{
                        1+x^2
                    }
                }
                = 4
                \neq &\\[3ex]&
                \neq \lim_{y\to0}{
                    \lim_{x\to0}{
                        \frac{
                            4\,x^2+3\,y^2+x^3\,y^3
                        }{
                            x^2+y^2+x^4+y^4
                        }
                    }
                }
                = \lim_{y\to0}{
                    \frac{
                        3\,y^2
                    }{
                        y^2+y^4
                    }
                }
                = \lim_{y\to0}{
                    \frac{
                        3
                    }{
                        1+y^2
                    }
                }
                = 3
            &
        \end{flalign*}
        \(\therefore\) não existe
        \vspace{3ex}
        \answer{alternativa: usando limite direcional}
        \begin{flalign*}
            &
                y=m\,x
                \implies &\\&
                \implies
                \lim_{(x,y)\to(0,0)}{
                    \frac{
                        4\,x^2+3\,y^2+x^3\,y^3
                    }{
                        x^2+y^2+x^4+y^4
                    }
                }
                =\lim_{(x)\to0}{
                    \frac{
                        4\,x^2+3\,(m\,x)^2+x^3\,(m\,x)^3
                    }{
                        x^2+(m\,x)^2+x^4+(m\,x)^4
                    }
                }
                = &\\&
                = \lim_{(x)\to0}{
                    \frac{
                        (3\,m^2+4)\,x^2
                        + m^3\,x^6
                    }{
                        (1+m^2)\,x^2
                        + (1+m^4)\,x^4
                    }
                }
                = \lim_{(x)\to0}{
                    \frac{
                        (3\,m^2+4)
                        + m^3\,x^4
                    }{
                        (1+m^2)
                        + (1+m^4)\,x^2
                    }
                }
                = &\\&
                =\frac{
                    3\,m^2+4
                }{
                    1+m^2
                }
            &
        \end{flalign*}
        \(\therefore\) como o valor do limite depende de \textit{m}, o limite não existe
    \end{questionBox}
\end{questionBox}

\begin{questionBox}1{ % Q3
    Estude as seguintes funções quanto à continuidade:
} % Q3
    \setcounter{subquestion}{1}
    \begin{questionBox}2{ % Q3.2
        \begin{BM}
            f(x,y)
            = \begin{cases}
                1 + \frac{(x-2)^3}{(x-2)^2+(y-1)^2}
                ,\quad&\text{se }(x,y)\neq(2,1)
                \\
                1,\quad&\text{se }(x,y)=(2,1)
            \end{cases}
        \end{BM}
    } % Q3.2
        \answer{}
        \begin{flalign*}
            &
                \lim_{(x,y)\to(2,1)}{f(x,y)}
                = f(2,1) = 1
                \iff &\\&
                \iff
                0\leq 
                \left(
                    \lim_{(x,y)\to(2,1)}{f(x,y)}
                \right)
                -f(2,1)
                \leq g(x,y)
                \implies &\\[3ex]&
                \implies
                \myvert{
                    \left(
                        \lim_{(x,y)\to(2,1)}{f(x,y)}
                    \right)
                    -f(2,1)
                }
                = \myvert{
                    1 + \frac{(x-2)^3}{(x-2)^2+(y-1)^2}
                    -1
                }
                = &\\&
                = \myvert{
                    \frac{(x-2)^3}{(x-2)^2+(y-1)^2}
                }
                \leq \myvert{
                    \frac{
                        (x-2)\,((x-2)^2+(y-1)^2)
                    }{
                        (x-2)^2+(y-1)^2
                    }
                }
                = &\\&
                =\myvert{x-2}
                \leq\sqrt{
                    (x-2)^2+(y-1)^2
                }
                \implies &\\&
                \implies
                \lim_{(x,y)\to(2,1)}{f(x,y)}
                \leq
                \lim_{(x,y)\to(2,1)}{
                    \sqrt{(x-2)^2+(y-1)^2}
                }
                =0
            &
        \end{flalign*}
    \end{questionBox}
\end{questionBox}

\setcounter{question}{25}

\begin{questionBox}1{ % Q26
    Estude as seguintes funções quanto à diferenciabilidade no ponot \((0,0)\):
} % Q26
    \setcounter{subquestion}{1}
    \begin{questionBox}2{ % Q26.2
        \begin{BM}
            f(x,y)
            \begin{cases}
                \frac{x\,y^2}{x^2+y^2}
                ,\quad&\se(x,y)\neq(0,0)
                \\
                0,\quad&
                \se(x,y)=(0,0)
            \end{cases}
        \end{BM}
    } % Q26.2
        \answer{}
        \begin{flalign*}
            &
                f\text{ é diferenciavel em }(0,0)
                \iff &\\&
                \iff
                \lim_{(h,k)\to(0,0)}{
                    \frac{
                        \myvert{
                            f_{(0+h,0+k)}
                            - f_{(0,0)}
                            - h\,\pdv{f}{x}_{(0,0)}
                            - k\,\pdv{f}{y}_{(0,0)}
                        }
                    }{
                        \sqrt{k^2+h^2}
                    }
                }
                = &\\&
                \lim_{(h,k)\to(0,0)}{
                    \frac{
                        \myvert{
                            f_{(0+h,0+k)}
                            - f_{(0,0)}
                            - h\,\left(
                                \lim_{h\to0}{
                                    \frac{f_{(h,0)-f(0,0)}}{h}
                                }
                            \right)
                            - k\,\left(
                                \lim_{k\to0}{
                                    \frac{f_{(0,k)-f(0,0)}}{k}
                                }
                            \right)
                        }
                    }{
                        \sqrt{k^2+h^2}
                    }
                }
                = &\\&
                = \lim_{(h,k)\to(0,0)}{
                    \frac{
                        \myvert{
                            f_{(h,k)}
                        }
                    }{
                        \sqrt{k^2+h^2}
                    }
                }
                = 0
            &
        \end{flalign*}
    \end{questionBox}
\end{questionBox}

\end{document}
% !TEX root = ./AM_2C-Testes_Resolucoes.2023.1.2.tex
\providecommand\mainfilename{"./AM_2C-Testes_Resolucoes.tex"}
\providecommand \subfilename{}
\renewcommand   \subfilename{"./AM_2C-Testes_Resolucoes.2023.1.2.tex"}
\documentclass[\mainfilename]{subfiles}

% \tikzset{external/force remake=true} % - remake all

\begin{document}

\graphicspath{{\subfix{./.build/figures/AM_2C-Testes_Resolucoes.2023.1.2}}}
\tikzsetexternalprefix{./.build/figures/AM_2C-Testes_Resolucoes.2023.1.2/graphics/}

\mymakesubfile{2}
[AM 2C]
{Teste Resolução} % Subfile Title
{Teste Resolução} % Part Title

\begin{questionBox}1{ % Q1
    Considere a superfície não limitada \(\mathcal{S}\) e o ponto \textit{P}. Escolha a afirmação correta
} % Q1
    \begin{BM}
        \mathcal{S}=\{(x,y,z)\in\mathbb{R}^3:x^2+y^2-z^2=1\}
        \\
        P = (2\sqrt{2},0,0)
    \end{BM}

    \begin{enumerate}[label=\alph{enumi})]
        \item Existem exatamente {\color{Graph21}2} pontos de \(\mathcal{S}\) à distancia mínima, {\color{Graph22}\(\sqrt{3}\)}, de \textit{P}
        \item Existem exatamente {\color{Graph21}4} pontos de \(\mathcal{S}\) à distancia mínima, {\color{Graph22}\(\sqrt{3}\)}, de \textit{P}
        \item Existem apenas {\color{Graph21}1} ponto de \(\mathcal{S}\) à distancia mínima, {\color{Graph22}3}, de \textit{P}
        \item Existem exatamente {\color{Graph21}2} pontos de \(\mathcal{S}\) à distancia mínima, {\color{Graph22}\(2\sqrt{2}-1\)}, de \textit{P}
        \item Nenhuma das afirmações é verdadeira
    \end{enumerate}

    \answer{}
    \begin{flalign*}
        &
            \left\{
                \begin{aligned}
                    \gdif{f(X)}=\lambda\,\gdif{g(X)}
                    \\
                    g(X)=0
                \end{aligned}
            \right\}
            &\\[3ex]&
            f(X)
            = {d(P,X)}^2
            =\myVert{P-X}^2
            =\myVert{\left(
                2\sqrt{2}-x,
                0-y,
                0-z
            \right)}^2
            = &\\&
            =
            (2\sqrt{2}-x)^2
            +y^2
            +z^2
            % =(2\,(2\sqrt{2}-x),2\,y,2\,z)
            ; &\\[3ex]&
            g(X)
            = x^2+y^2-z^2-1
            % = (2\,x,2\,y,-2\,z)
            &\\[3ex]&
            \implies
            \left\{
                \begin{aligned}
                    \gdif{f(X)}=\lambda\,\gdif{g(X)}
                    \\
                    g(X)=0
                \end{aligned}
            \right\}
            =\left\{
                \begin{aligned}
                    (-2\,(2\sqrt{2}-x),2\,y,2\,z)
                    = \lambda (2\,x,2\,y,-2\,z)
                    \\
                    x^2+y^2-z^2=1
                \end{aligned}
            \right\}
            = &\\&
            = \left\{
                \begin{aligned}
                    -2\,(2\sqrt{2}-x) =  2\,\lambda\,x\\
                    2\,y=  2\,\lambda\,y\\
                    2\,z= -2\,\lambda\,z
                    \\
                    x^2+y^2-z^2=1
                \end{aligned}
            \right\}
            % = &\\&
            = \left\{
                \begin{aligned}
                    x=\frac{2\sqrt{2}}{(1-\lambda)}
                    \\
                    y= y\,\lambda
                    \\
                    z=-z\,\lambda
                    \\
                    x^2+y^2-z^2=1
                \end{aligned}
            \right\}
            \implies &\\[3ex]&
            \implies
            \left\{
                \begin{aligned}
                    &
                        x=2\,\sqrt{2}/(1-\lambda)
                    \ldiv{}
                        \lambda\neq1
                    \ldiv{}
                        \left(
                            \begin{aligned}
                                &
                                    y=0
                                \ldiv[\lor]{}
                                    (y\neq0\land\lambda=1)
                                &
                            \end{aligned}
                        \right)
                    \ldiv{}
                        \left(
                            \begin{aligned}
                                &
                                    z=0
                                \ldiv[\lor]{}
                                    (z\neq0\land\lambda=-1)
                                &
                            \end{aligned}
                        \right)
                    \ldiv{}
                        x^2+y^2-z^2=1
                    &
                \end{aligned}
            \right\}
            = &\\&
            = \left\{
                \begin{aligned}
                    &
                        x=2\,\sqrt{2}/(1-\lambda)
                    \ldiv{}
                        \lambda\neq1
                    \ldiv{}
                        y=0
                    \ldiv{}
                        \left(
                            \begin{aligned}
                                &
                                    z=0
                                \ldiv[\lor]{}
                                    (z\neq0\land\lambda=-1)
                                &
                            \end{aligned}
                        \right)
                    \ldiv{}
                        x^2+y^2-z^2=1
                    &
                \end{aligned}
            \right\}
            = &\\&
            = \left\{
                \begin{aligned}
                    &
                        X=(+1,0,0)
                    \ldiv*[\lor]{}
                        X=(-1,0,0)
                    \ldiv[\lor]{}
                        X=\left(\sqrt{2},0,+1\right)
                    \ldiv[\lor]{}
                        X=\left(\sqrt{2},0,-1\right)
                    &
                \end{aligned}
            \right\}
            \implies &\\&
            \implies
            \left\{
                \begin{aligned}
                    &d(+1,0,0) 
                    &=& \sqrt{
                        (2\sqrt{2}-1)^2
                        + 0^2 + 0^2
                    }
                    &=2\sqrt{2}-1
                    \\
                    &d(-1,0,0) 
                    &=& \sqrt{
                        (2\sqrt{2}+1)^2
                        + 0^2 + 0^2
                    }
                    &=2\sqrt{2}+1
                    \\
                    &d(\sqrt{2},0,+1) 
                    &=& \sqrt{
                        (2\sqrt{2}+\sqrt{2})^2
                        + 0^2 + 1^2
                    }
                    &=\sqrt{3}
                    \\
                    &d(\sqrt{2},0,-1) 
                    &=& \sqrt{
                        (2\sqrt{2}+\sqrt{2})^2
                        + 0^2 + (-1)^2
                    }
                    &=\sqrt{3}
                \end{aligned}
            \right\}
            &\\[3ex]&
            \sqrt{3}<2\,\sqrt{2}\pm1
            \,\therefore \text{a)}
        &
    \end{flalign*}
\end{questionBox}

\begin{questionBox}1{ % Q2
    Seja \textit{D} a região do plano limitada pelas curvas \(y=x^2\text{ e }x=y^2\). O valor do integral \(\iint_D{x\,y\,\odif{x,y}}\) é:
} % Q2
    \begin{enumerate}[label=\Alph{enumi}.]
        \begin{multicols}{4}
            \item 1/2
            \item -1/4
            \item -1/2
            \item 1/4
        \end{multicols}
        \item Nenhum dos anteriores
    \end{enumerate}

    \answer{}
    \begin{center}
        % \tikzset{external/remake next=true}
        % \pgfplotsset{height=7cm, width= .6\textwidth}
        \begin{tikzpicture}
        \begin{axis}
            [
                axis lines={left},
                % xmajorgrids = true,
                legend pos={south east},
                % xmin={},
                % xmax={},
                % ymin={},
                % ymax={},
                % domain=0:4,
                % xlabel={},
                % ylabel={},
            ]
            % Legends
            \addlegendimage{empty legend}
            \addlegendentry[Graph]{\( 
                {\color{Graph}x^2},
                {\color{GraphC}\sqrt{x}} 
            \)}
            
            % Plot from equation
            \addplot[
                name path={A},
                smooth,
                thick,
                Graph,
                domain  = 0:1.1,
                samples = \mysampledensitySimple,
            ]{ x^2 };
            % Plot from equation
            \addplot[
                name path={B},
                smooth,
                thick,
                GraphC,
                domain  = 0:1.1,
                samples = 2\mysampledensitySimple,
            ]{ sqrt(x) };

            % ================ fillbetween =============== %
            \addplot [
                fill opacity=0.1,
                foreground,
            ] fill between [
                of={A and B},
                soft clip={domain={0:1}},
            ];
            
        \end{axis}
        \end{tikzpicture}
    \end{center}
    \begin{flalign*}
        &
            \iint_D{x\,y\,\odif{x,y}}
            =\int_{0}^{1}{
                \int_{y^2}^{\sqrt{y}}{
                    x\,y\,\odif{x,y}
                }
            }
            =\int_{0}^{1}{
                \adif{(x^2)}\big\vert_{y^2}^{\sqrt{y}}
                \,y\,\odif{y}/2
            }
            =\int_{0}^{1}{
                (y-y^4)\,y\,\odif{y}/2
            }
            = &\\&
            =\int_{0}^{1}{
                y^2\,\odif{y}/2
            }
            -\int_{0}^{1}{
                y^5\,\odif{y}/2
            }
            =\adif{(y^3)}\big\vert_{0}^{1}/6
            -\adif{(y^6)}\big\vert_{0}^{1}/12
            =1/12
            &\\&
            \therefore\text{ E.}
        &
    \end{flalign*}
\end{questionBox}

\begin{questionBox}1{ % Q3
    Considere o integral \textit{I}. Tem-se:
} % Q3
    \begin{BM}
        I
        =\int_{0}^{1}{
            \int_{y^{1/2}}^{\sqrt{2-y^2}}{
                y\,\odif{x,y}
            }
        }
    \end{BM}

    \begin{enumerate}[label=\Alph{enumi}.]
        % \begin{multicols}{2}
            \item \(
                I=\int_{0}^{1}{\int_{0}^{{\color{Graph21}x^2}}{
                    y\,\odif{y,x}
                }}
                +\int_{1}^{{\color{Graph21}\sqrt{2}}}{\int_{0}^{{\color{Graph31}x^2+y^2}}{
                    y\,\odif{y,x}
                }}
            \)
            \item \(
                I=\int_{0}^{1}{\int_{0}^{{\color{Graph22}\sqrt{2-x^2}}}{
                    y\,\odif{y,x}
                }}
                +\int_{1}^{{\color{Graph21}\sqrt{2}}}{\int_{0}^{{\color{Graph32}x^2}}{
                    y\,\odif{y,x}
                }}
            \)
            \item \(
                I=\int_{0}^{1}{\int_{0}^{{\color{Graph21}x^2}}{
                    y\,\odif{y,x}
                }}
                +\int_{1}^{{\color{Graph21}\sqrt{2}}}{\int_{0}^{{\color{Graph33}\sqrt{2-x^2}}}{
                    y\,\odif{y,x}
                }}
            \)
            \item \(
                I=\int_{0}^{1}{\int_{0}^{{\color{Graph22}\sqrt{2-x^2}}}{
                    y\,\odif{y,x}
                }}
                +\int_{1}^{{\color{Graph22}2}}{\int_{0}^{{\color{Graph32}x^2}}{
                    y\,\odif{y,x}
                }}
            \)
            \item Nenhum dos casos anteriores
        % \end{multicols}
    \end{enumerate}

    \answer{}
    \begin{flalign*}
        &
            \begin{cases}
                x=\sqrt{2-y^2}\implies y^2=2-x^2
                \\
                x=\sqrt{y}\implies y=x^2
                \\
                \sqrt{2-y^2}=\sqrt{y}
                \implies
                X=(1, 1);
            \end{cases}
        &
    \end{flalign*}
    \begin{center}
        % \tikzset{external/remake next=true}
        % \pgfplotsset{height=7cm, width= .6\textwidth}
        \begin{tikzpicture}
        \begin{axis}
            [
                axis lines={center},
                % xmin={},
                % xmax={},
                % ymin={},
                % ymax={},
                % xmajorgrids = true,
                % legend pos  = north west
                % domain=0:1.5,
                % y domain=0:1.1,
                % xlabel={},
                % ylabel={},
            ]
            % Legends
            \addlegendimage{empty legend}
            \addlegendentry{\(
                {\color{Graph} x^2},
                {\color{GraphC}\sqrt{2-x^2}}
            \)}

            % Plot from equation
            \addplot[
                name path={A},
                smooth,
                thick,
                Graph,
                domain  = 0:1,
                samples = \mysampledensitySimple,
            ]{ x^2 };

            % Plot from equation
            \addplot[
                name path={B},
                smooth,
                thick,
                GraphC,
                domain  = 1:sqrt(2),
                samples = 2\mysampledensitySimple,
            ]{ sqrt(2-x^2) };

            % \path [name path={B}]
            % (\pgfkeysvalueof(/pgfplots/xmin),0)
            % -- (\pgfkeysvalueof(/pgfplots/xmax),0);

            % % ================ fillbetween =============== %
            % \addplot [
            %     fill opacity=0.1,
            %     foreground,
            % ] fill between [
            %     of={B and A1},
            %     soft clip={domain={0:1}},
            % ];
            % \addplot [
            %     fill opacity=0.1,
            %     foreground,
            % ] fill between [
            %     of={B and A2},
            %     soft clip={domain={1:1.414}},
            % ];
            
        \end{axis}
        \end{tikzpicture}
    \end{center}

    \begin{flalign*}
        &
            I
            =\int_{0}^{1}{
                \int_{y^{1/2}}^{\sqrt{2-y^2}}{
                    y\,\odif{x,y}
                }
            }
            =\int_{0}^{1}{
                \int_{0}^{x^2}{
                    y\,\odif{y,x}
                }
            }
            +\int_{1}^{\sqrt{2}}{
                \int_{0}^{\sqrt{2-x^2}}{
                    y\,\odif{y,x}
                }
            }
            \quad
            \therefore\text{ C.}
        &
    \end{flalign*}
\end{questionBox}

\begin{questionBox}1{ % Q4
    Considere o domínio de \(\mathbb{R}^2\) definido por
} % Q4
    \begin{BM}
        R=
        \{
            (x,y)\in\mathbb{R}^2
            : x\geq 0
            , y\geq 0
            , 1\leq x + y \leq 4
        \}
    \end{BM}

    Usando a transformação de variáveis \(x=u-u\,v\text{, }y=u\,v\). Tem-se:

    \begin{enumerate}[label=\Alph{enumi}.]
        \item \(
            \iint_{R}{\frac{y}{x+y}\,\odif{x,y}}
            =\int_{1}^{4}{\int_{0}^{1}{v\,\odif{vu}}}
        \)
        \item \(
            \iint_{R}{\frac{1}{x+y}\,\odif{x,y}}
            =3
        \)
        \item \(
            \iint_{R}{\frac{y}{x+y}\,\odif{x,y}}
            =\int_{0}^{4}{\int_{0}^{1}{v\,\odif{vu}}}
        \)
        \item \(
            \iint_{R}{\frac{y}{x+y}\,\odif{x,y}}
            =\int_{1}^{4}{\int_{0}^{1}{u^{-1}\,\odif{vu}}}
        \)
        \item Nenhum dos casos anteriores.
    \end{enumerate}

    \answer{}
    \begin{flalign*}
        &
            \begin{cases}
                y\geq 1-x
                \\
                y\leq4-x
            \end{cases}
        &
    \end{flalign*}
    \begin{center}
        % \tikzset{external/remake next=true}
        % \pgfplotsset{height=7cm, width= .6\textwidth}
        \begin{tikzpicture}
        \begin{axis}
            [
                % xmajorgrids = true,
                % legend pos={north east},
                % domain=0:4,
                % xlabel={},
                % ylabel={},
                axis lines={center}, % left|right|center|box
                xmin={-0.1},
                xmax={ 4.1},
                ymin={-0.1},
                ymax={ 4.1},
            ]
            Legends
            \addlegendimage{empty legend}
            \addlegendentry[Graph]{\( 
                {\color{Graph41}C_1},
                {\color{Graph42}C_2}, 
                {\color{Graph43}C_3}, 
                {\color{Graph44}C_4} 
            \)}
            
            % Plot from equation
            \addplot[
                % name path={A},
                smooth,
                thick,
                Graph41,
                domain  = 0:1,
                samples = \mysampledensityDouble,
            ]{ 1-x };
            % Plot from equation
            \draw[
                % name path={A},
                % smooth,
                thick,
                Graph42,
                % domain  = 0:1,
                % samples = \mysampledensityDouble,
            ] (1,0) -- (4,0);
            % Plot from equation
            \addplot[
                % name path={B},
                smooth,
                thick,
                Graph43,
                domain  = 0:4,
                samples = 2\mysampledensityDouble,
            ]{ 4-x };
            % Plot from equation
            \draw[
                % name path={A},
                % smooth,
                thick,
                Graph44,
                % domain  = 0:1,
                % samples = \mysampledensityDouble,
            ] (0,1) -- (0,4);
            
            % % ================ fillbetween =============== %
            % \addplot [
            %     fill opacity=0.1,
            %     foreground,
            % ] fill between [
            %     of={A and B},
            %     soft clip={domain={0:1}},
            % ];
            
        \end{axis}
        \end{tikzpicture}
    \end{center}

    \begin{flalign*}
        &
            \left\{
                \begin{aligned}
                    x& =u-u\,v
                    \\
                    y& =u\,v
                \end{aligned}
            \right\}
            = \left\{
                \begin{aligned}
                    x& =u-y
                    \\
                    y& =u\,v
                \end{aligned}
            \right\}
            = \left\{
                \begin{aligned}
                    u& =x+y
                    \\
                    y& =(x+y)\,v
                \end{aligned}
            \right\}
            = \left\{
                \begin{aligned}
                    u& =x+y
                    \\
                    v& =\frac{y}{x+y}
                \end{aligned}
            \right\}
            &\\[3ex]&
            \left\{
                \begin{aligned}
                    C_1
                    & = \{y=1-x,& x\in\myrange{0,1}\}
                    \\
                    C_2
                    & = \{y=0,& x\in\myrange{1,4}\}
                    \\
                    C_3
                    & = \{y=4-x,& x\in\myrange{0,4}\}
                    \\
                    C_4
                    & = \{x=0,& y\in\myrange{1,4}\}
                \end{aligned}
            \right\}
            \implies &\\[6ex]&
            % ================== C_{1,*} ================= %
            C_{1,*}
            % =&\\&
            = \{u\,v=1-(u-u\,v), (u-u\,v)\in\myrange{0,1}\}
            = \{u=1, 0\leq 1-v \leq 1\}
            =&\\&
            = \{u=1, v\in\myrange{0,1}\}
            ;&\\[3ex]&
            % ================== C_{2,*} ================= %
            C_{2,*}
            % =&\\&
            = \{u\,v=0, (u-u\,v)\in\myrange{1,4}\}
            % =&\\&
            = \left\{
                \begin{aligned}
                    &
                        \{u=0,(u-u\,v)\in\myrange{1,4}\}
                    \ldiv[\lor]{}
                        \{v=0,u\in\myrange{1,4}\}
                    &
                \end{aligned}
            \right\}
            = &\\&
            = \{v=0, u\in\myrange{1,4}\}
            ;&\\[3ex]&
            % ================== C_{3,*} ================= %
            C_{3,*}
            % =&\\&
            = \{u\,v=4-(u-u\,v), (u-u\,v)\in\myrange{0,4}\}
            = \{u=4, 0 \leq 4-4\,v \leq 4\}
            = &\\&
            = \{u=4, v\in\myrange{0,1}\}
            ;&\\[3ex]&
            % ================== C_{4,*} ================= %
            C_{4,*}
            = \{u-u\,v=0, (u\,v)\in\myrange{1,4}\}
            = \{u(1-v)=0, u\,v\in\myrange{1,4}\}
            =&\\&
            = \left\{
                \begin{aligned}
                    &
                        u=0\land u\,y\in\myrange{1,4}
                    \ldiv[\lor]{}
                        v=1\land u\in\myrange{1,4}
                    &
                \end{aligned}
            \right\}
            = &\\&
            = \{v=1,  u\in\myrange{1,4}\}
            \implies &\\[3ex]&
            \implies
            \left\{
                \begin{aligned}
                    C_{1,*}
                    & = \{u=1, v\in\myrange{0,1}\}
                    \\
                    C_{2,*}
                    & = \{v=0, u\in\myrange{1,4}\}
                    \\
                    C_{3,*}
                    & = \{u=4, v\in\myrange{0,1}\}
                    \\
                    C_{4,*}
                    & = \{v=1,  u\in\myrange{1,4}\}
                \end{aligned}
            \right\}
            &
        \end{flalign*}

        \begin{center}
            % \tikzset{external/remake next=true}
            % \pgfplotsset{height=7cm, width= .6\textwidth}
            \begin{tikzpicture}
            \begin{axis}
                [
                    % xmajorgrids = true,
                    legend pos={north east},
                    % domain=0:2,
                    xmin={0.9}, 
                    xmax={4.1},
                    ymin={-0.1}, 
                    ymax={1.3},
                    % xlabel={},
                    % ylabel={},
                    axis lines={center}, % left|right|center|box
                ]
                % Legends
                \addlegendimage{empty legend}
                \addlegendentry{\( 
                    {\color{Graph41}C_{1,*}},
                    {\color{Graph42}C_{2,*}},
                    {\color{Graph43}C_{3,*}},
                    {\color{Graph44}C_{4,*}},
                \)};
                
                \draw[Graph41,thick] (1,0) -- (1,1);
                \draw[Graph42,thick] (1,0) -- (4,0);
                \draw[Graph43,thick] (4,0) -- (4,1);
                \draw[Graph44,thick] (1,1) -- (4,1);

                % % Plot from equation
                % \addplot[
                %     % name path={A},
                %     smooth,
                %     thick,
                %     Graph41,
                %     domain  = 0:1,
                %     samples = \mysampledensityDouble,
                % ]{ 1 };
                % % Plot from equation
                % \addplot[
                %     % name path={B},
                %     smooth,
                %     thick,
                %     GraphC,
                %     domain  = 0:1.1,
                %     samples = \mysampledensitySimple,
                % ]{ x^2 };
                
                % % ================ fillbetween =============== %
                % \addplot [
                %     fill opacity=0.1,
                %     foreground,
                % ] fill between [
                %     of={A and B},
                %     soft clip={domain={0:1}},
                % ];
                
            \end{axis}
            \end{tikzpicture}
        \end{center}

        \begin{flalign*}
            &
                \det\jacobiana_f
                = \begin{pmatrix}
                    \pdv{x}{u} & \pdv{x}{v}
                    \\
                    \pdv{y}{u} & \pdv{y}{v}
                \end{pmatrix}
                = \begin{pmatrix}
                    1-v & -u
                    \\
                    v & u
                \end{pmatrix}
                = u-u\,v + (u\,v)
                = u
                ; &\\[3ex]&
                \iint_{R}{\frac{1}{x+y}\,\odif{x,y}}
                = \iint_{R_{*}}{
                    \frac{1}{u}\,\det{J}\,\odif{u,v}
                }
                = \int_{1}^{4}{
                    \int_{0}^{1}{
                        \frac{1}{u}\,u\,\odif{v,u}
                    }
                }
                = \int_{1}^{4}{
                    \odif{u}
                }
                = 3
                &\\&
                \therefore\text{ B.}
            &
        \end{flalign*}
\end{questionBox}

\begin{questionBox}1{ % Q5
    Recorrendo a coordenadas esféricas \(\rho,\theta,\phi\), o volume do sólido
} % Q5
    \begin{BM}
        S
        = \left\{
            (x,y,z)\in\mathbb{R}^3
            : x^2+y^2+z^2\leq 16
            \land
            z^2\leq 3\,x^2+3\,y^2
        \right\}
    \end{BM}
    é igual a:
    \begin{enumerate}[label=\Alph{enumi}.]
        \item \(
            \int_{0}^{2\,\pi}{
                \int_{0}^{4}{
                    \int_{\pi/4}^{3\,\pi/4}{
                        \rho^2\,\sin\phi
                        \,\odif{\phi,\rho,\theta}
                    }
                }
            }
        \)
        \item \(
            \int_{0}^{2\,\pi}{
                \int_{0}^{4}{
                    \int_{\pi/6}^{5\,\pi/6}{
                        \rho^2\,\cos\phi
                        \,\odif{\phi,\rho,\theta}
                    }
                }
            }
        \)
        \item \(
            4\,\int_{0}^{\pi/2}{
                \int_{0}^{4}{
                    \int_{\pi/6}^{5\,\pi/6}{
                        \rho^2\,\sin\phi
                        \,\odif{\phi,\rho,\theta}
                    }
                }
            }
        \)
        \item \(
            4\,\int_{0}^{\pi/2}{
                \int_{0}^{4}{
                    \int_{\pi/3}^{2\,\pi/3}{
                        \rho^2\,\sin\phi
                        \,\odif{\phi,\rho,\theta}
                    }
                }
            }
        \)
        \item \(
            \int_{0}^{2\,\pi}{
                \int_{0}^{4}{
                    \int_{\pi/3}^{2\,\pi/3}{
                        \rho^2\,\cos\phi
                        \,\odif{\phi,\rho,\theta}
                    }
                }
            }
        \)
    \end{enumerate}

    \answer{}
    \begin{flalign*}
        &
            y=0\begin{cases}
                z^2\leq 16-x^2
                \\
                z^2\leq 3\,x^2
            \end{cases}
            % &\\&
            ;\quad
            16-x^2=3\,x^2
            \implies
            (x,z)
            =\begin{cases}
                   ( 2, 4),
                \\ (-2, 4),
                \\ ( 2,-4),
                \\ (-2,-4),
            \end{cases}
        &
    \end{flalign*}
    \begin{center}
        % \tikzset{external/remake next=true}
        % \pgfplotsset{height=7cm, width= .6\textwidth}
        \begin{tikzpicture}
        \begin{axis}
            [
                % xmajorgrids = true,
                legend pos={south east},
                % domain=0:4,
                % xlabel={},
                % ylabel={},
                xmin={-4.1},
                xmax={ 4.1},
                ymin={-4.5},
                ymax={ 4.5},
                axis lines={center}, % left|right|center|box
            ]
            % Legends
            % \addlegendimage{empty legend}
            % \addlegendentry[Graph]{\( 
            %     {\color{Graph}x},
            %     {\color{GraphC}x^2} 
            % \)}
            
            % Plot from equation
            \addplot[
                % name path={A},
                smooth,
                thick,
                Graph,
                domain={-4:4},
                samples = 50*\mysampledensitySimple,
            ]{ sqrt(16-x^2) };
            % Plot from equation
            \addplot[
                % name path={A},
                smooth,
                thick,
                Graph,
                domain={-4:4},
                samples = 50*\mysampledensitySimple,
            ]{ -sqrt(16-x^2) };
            % Plot from equation
            \addplot[
                % name path={B},
                smooth,
                thick,
                GraphC,
                domain={-2.1:2.1},
                samples = \mysampledensitySimple,
            ]{ x*sqrt(3) };
            % Plot from equation
            \addplot[
                % name path={B},
                smooth,
                thick,
                GraphC,
                domain={-2.1:2.1},
                samples = \mysampledensitySimple,
            ]{ -x*sqrt(3) };
            
            % % ================ fillbetween =============== %
            % \addplot [
            %     fill opacity=0.1,
            %     foreground,
            % ] fill between [
            %     of={A and B},
            %     soft clip={domain={0:1}},
            % ];
            
        \end{axis}
        \end{tikzpicture}
    \end{center}

    \begin{flalign*}
        &
            R_*
            =\left\{
                \begin{aligned}
                       x&=\rho\sin\phi\cos\theta
                    \\ y&=\rho\sin\phi\sin\theta
                    \\ z&=\rho\cos\phi
                \end{aligned}
            \right\}
            = &\\&
            =\left\{
                \begin{aligned}
                       \theta & \in\myrange{0,2\,\phi}
                    \\ \rho   & \in\myrange{0,4}
                    \\ \phi   & \in\myrange{\pi/2-\tan^{-1}{\sqrt{3}},\pi/2+\tan^{-1}{\sqrt{3}}}
                \end{aligned}
            \right\}
            =\left\{
                \begin{aligned}
                       \theta & \in\myrange{0,2\,\phi}
                    \\ \rho   & \in\myrange{0,4}
                    \\ \phi   & \in\myrange{\pi/6,\pi\,5/6}
                \end{aligned}
            \right\}
            &\\&
            \int_{0}^{2\,\pi}{
                \int_{0}^{4}{
                    \int_{\pi/6}^{\pi\,5/6}{
                        \odif{\phi,\rho,\theta}
                    }
                }
            }
        &
    \end{flalign*}
\end{questionBox}

\begin{questionBox}1{ % Q6
} % Q6
    \begin{BM}
        I=\int_C{\vv{F}\cdot\odif{\vv{r}}}
    \end{BM}

    Se I é conservativo

    \begin{BM}
        I = f(B)-f(A)
    \end{BM}

    \begin{flalign*}
        &
            \pdv{f}{x} = F_x;
            \pdv{f}{y} = F_y;
            \implies
            f = (...) + h(y);
            h'(y)= 0; h(y) = c (constante), escolhemos c=0 e resolvemos
        &
    \end{flalign*}
\end{questionBox}

\end{document}
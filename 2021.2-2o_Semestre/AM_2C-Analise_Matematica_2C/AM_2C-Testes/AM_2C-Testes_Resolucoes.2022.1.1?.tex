% !TEX root = ./AM_2C-Testes_Resolucoes-2022.1.1.tex
% !TEX root = ./AM_2C_-_Testes_Resolucoes.tex
\providecommand\mainfilename{"./AM_2C-Testes_Resolucoes.tex"}
\providecommand \subfilename{}
\renewcommand   \subfilename{"./AM_2C-Testes_Resolucoes-2022.1.1.tex"}
\documentclass[\mainfilename]{subfiles}

% \graphicspath{{\subfix{../images/}}}

\begin{document}

\mymakesubfile{1}
[AM\,2C]
{Teste Resolução}
{Teste Resolução}

\group{}

\begin{questionBox}1{} % I Q1
    
    % A elipse centrada em (1,-4), com um dos vértices em \((1,-1)\) e um dos focos em \((1,-4+\sqrt{5})\) tem por equação:

    \begin{BM}
        \text{Elipse}:
        \begin{cases}
            P' = (1,-4)
         \\ \text{Vertice} = (1,-1)
         \\ f_1 = (1,-4+\sqrt{5})
        \end{cases}
    \end{BM}

    % \begin{answerBox}{} % RS 
        \begin{flalign*}
            &
                1
                =
                \left(
                    \frac{x-x'}{r_1}
                \right)^2
                +
                \left(
                    \frac{y-y'}{r_2}
                \right)^2
                = &\\&
                =
                \left(
                    \frac{x-1}{
                        \sqrt{
                            (-4-(-1))^2
                            -(-4-(-4+\sqrt{5}))^2
                        }
                    }
                \right)^2
                +
                \left(
                    \frac{
                        y+4
                    }{
                        (\lvert-4-(-1)\rvert)
                    }
                \right)^2
                = &\\&
                =
                \left(
                    \frac{x-1}{2}
                \right)^2
                +
                \left(
                    \frac{y+4}{3}
                \right)^2
                \implies &\\&
                \implies
                0
                =
                9
                \left(
                    x^2
                    -2\,x
                    +1
                \right)
                +
                4\left(
                    y^2
                    +8\,y
                    +16
                \right)
                -9*4
                = &\\&
                =
                9\,x^2
                -18\,x
                +4y^2
                +32\,y
                +37
            &
        \end{flalign*}
    % \end{answerBox}
    
\end{questionBox}

\begin{questionBox}1m{} % I Q2

    Relativamente a função
    
    \begin{BM}
        f(x,y) = (3\,x+3\,x^3)(y^2+2\,y)
    \end{BM}

    Tem-se que

    \begin{itemize}[label=\square]
        \item O ponto (0,0) é de sela e o ponto (0,2) é maximizante,
        \item O ponto (0,0) é de sela e o ponto (0,-2) é de sela,
        \item O ponto (0,0) é de sela e o ponto (0,-2) é maximizante,
        \item O ponto (0,0) é de sela e o ponto (0,2) é de sela,
        \item O ponto (0,0) é maximizante e o ponto (0,-2) é de sela,
        \item O ponto (0,0) é maximizante e o ponto (0,-2) é de sela,
    \end{itemize}

    % \begin{answerBox}{} % RS 
        \paragraph*{Pontos de inflexão}
        \begin{flalign*}
            &
            \nabla f
            = \left(
                \pdv{f}{x},
                \pdv{f}{y}
            \right)
            = \left(
                \begin{aligned}
                    \left(
                        3+9\,x^2
                    \right)(y^2+2\,y),
                    \\
                    (3\,x+3\,x^3)
                    \left(
                        2\,y + 2
                    \right)
                \end{aligned}
            \right)
            = &\\&
            = \left(
                \begin{aligned}
                    y\,\left(
                        3+9\,x^2
                    \right)(y+2),
                    \\
                    6\,x(1+x^2)
                    \left(
                        y + 1
                    \right)
                \end{aligned}
            \right)
            = (0,0)
            \implies &\\&
            \implies
            x=0\lor y=-1
            ;
            \begin{cases}
                x=0 & \implies y=-2\lor y=0
                \\
                y=-1 & \implies x=\sqrt{-1/3}
            \end{cases}
            &
        \end{flalign*}

        \paragraph*{Caracterizando pontos}
        \begin{flalign*}
            &
                \det H_f 
                = \begin{vmatrix}
                        \pdv{f}{x,x}
                    &   \pdv{f}{x,y}
                    \\  \pdv{f}{y,x}
                    &   \pdv{f}{y,y}
                \end{vmatrix}
                = \begin{vmatrix}
                        18\,x\,(y^2+2\,y)
                    &   (3+9\,x^2)(2\,y+2)
                    \\  (3+9\,x^2)(2\,y+2)
                    &   2\,(3\,x+3\,x^3)
                \end{vmatrix}
                &\\&
                \begin{cases}
                    \det h_f(0,0) = \begin{vmatrix}
                        0 & 6 \\ 6 & 0
                    \end{vmatrix}
                    = -36 < 0
                    \\
                    \det h_f(0,-2) = \begin{vmatrix}
                        0 & -6 \\ -6 & 0
                    \end{vmatrix}
                    = -36 < 0
                \end{cases}
            &
        \end{flalign*}
    % \end{answerBox}
    
\end{questionBox}

\begin{questionBox}1{} % I Q3
    
    Considere a função \(f:\mathbb{R}^2\to\mathbb{R}\) definida por
    \begin{BM}
        f(x,y) = (4\,x+y^3,x^2+3\,y)
    \end{BM}
    Invertível numa vizinhança do ponto (-1,1). Tem-se que:
    \begin{BM}
        J_{f^{-1}}(-3,4) = ?
    \end{BM}

    % \begin{answerBox}{} % RS 
        \begin{flalign*}
            &
                J_{f^{-1}}(-3,4)
                J_{f^{-1}}\left(
                    % (4\,(-1)+(1)^3,(-1)^2+3\,(1))
                    % (-4+1,1+3)
                    f(-1,1)
                \right)
                (J_{f}\left(f(-1,1)\right))^{-1}
                =\begin{bmatrix}
                        \pdv{f_1}{x}(-1,1)
                    &   \pdv{f_1}{y}(-1,1)
                    \\  \pdv{f_2}{x}(-1,1)
                    &   \pdv{f_2}{y}(-1,1)
                \end{bmatrix}^{-1}
                = &\\&
                =\begin{bmatrix}
                         4  &   3
                    \\  -2 &   3
                \end{bmatrix}^{-1}
                =\left(
                    12-(-6)
                \right)^{-1}
                \begin{pmatrix}
                    3 & -3
                    \\
                    2 & 4
                \end{pmatrix}
                =
                \begin{pmatrix}
                    \sfrac{1}{6} & \sfrac{-1}{6}
                    \\
                    \sfrac{1}{9} & \sfrac{2}{9}
                \end{pmatrix}
            &
        \end{flalign*}
    % \end{answerBox}

\end{questionBox}

\begin{questionBox}1{} % I Q4
    
    Pretende-se determinar o ponto da porção da superfície cónica \(z=\sqrt{(x-1)^2+y^2}\) que se encontra à distância mínima da origem. A função de Lagrange para o problema considerado é:
    
    \begin{answerBox}{} % RS 
        \begin{flalign*}
            &
                L(x,y,z,\lambda)
                = x^2+y^2+z^2
                +\lambda\left(
                    z-\sqrt{(x-1)^2+y^2}
                \right)
            &
        \end{flalign*}
    \end{answerBox}

\end{questionBox}

\begin{questionBox}1{} % I Q5
    Seja \(f(x,y):\mathbb{R}^2\to\mathbb{R}\) uma função diferenciável e \(x=\varphi(s,t)=s+t,y=\psi(s,t)=s-t\). Considere a função composta \(u=f\circ g\), onde \(g=(\varphi,\psi)\). Para que a igualdade

    \begin{BM}
        \pdv[order=2]{f}{x}
        +\pdv[order=2]{f}{y}
        = A\left(
            \pdv[order=2]{u}{s}
            +\pdv[order=2]{u}{t}
        \right)
    \end{BM}

    Seja verificada, \textit{A} tem de ser igual a:

    \begin{answerBox}{} % RS 
        \begin{flalign*}
            &
                A
                =\frac{
                    \pdv[order=2]{f}{x}
                    +\pdv[order=2]{f}{y}
                }{
                    \pdv[order=2]{u}{s}
                    +\pdv[order=2]{u}{t}
                };
                &\\[1.5ex]&
                % 
                % 
                \pdv{u}{s}
                = 
                \pdv{f}{x}\pdv{x}{s}
                +\pdv{f}{y}\pdv{y}{s}
                =
                \pdv{f}{x}
                +\pdv{f}{y}
                &\\[1.5ex]&
                % 
                % 
                \pdv{u}{t}
                = 
                \pdv{f}{x}\pdv{x}{t}
                +\pdv{f}{y}\pdv{y}{t}
                =
                \pdv{f}{x}
                -\pdv{f}{y}
                &\\[1.5ex]&
                \pdv[order=2]{u}{s}
                = \pdv{}{s}\left(
                    \pdv{f}{x}
                    +
                    \pdv{f}{y}
                \right)
                =
                \pdv[order=2]{f}{x}
                \,\pdv{x}{s}
                +\pdv{f}{x,y}
                \,\pdv{y}{s}
                +\pdv{f}{y,x}
                \,\pdv{x}{s}
                +
                \pdv[order=2]{f}{y}
                \pdv{y}{s}
                = &\\&
                =
                \pdv[order=2]{f}{x}
                +2\,\pdv{f}{x,y}
                +\pdv[order=2]{f}{y}
                % 
                % 
                % 
                &\\[1.5ex]&
                \pdv[order=2]{u}{t}
                = \pdv{}{t}\left(
                    \pdv{f}{x}
                    +
                    \pdv{f}{y}
                \right)
                =
                \pdv[order=2]{f}{x}
                \,\pdv{x}{t}
                +\pdv{f}{x,y}
                \,\pdv{y}{t}
                -\pdv{f}{y,x}
                \,\pdv{x}{t}
                -\pdv[order=2]{f}{y}
                \,\pdv{y}{t}
                = &\\&
                =
                \pdv[order=2]{f}{x}
                -2\,\pdv{f}{x,y}
                +\pdv[order=2]{f}{y}
                &\\[2ex]&
                \therefore
                A
                =\cfrac{
                    \pdv[order=2]{f}{x}
                    +\pdv[order=2]{f}{y}
                }{
                    \pdv[order=2]{f}{x}
                    +2\,\pdv{f}{x,y}
                    +\pdv[order=2]{f}{y}
                    +
                    \pdv[order=2]{f}{x}
                    -2\,\pdv{f}{x,y}
                    +\pdv[order=2]{f}{y}
                } 
                % = &\\&
                =\frac{1}{2} 
                % 
                % 
            &
        \end{flalign*}
    \end{answerBox}
\end{questionBox}

\group{}

\begin{questionBox}1{} % QII 1
    
    Considere a função real \textit{g}, de duas variáveis reais, definida por
    \begin{BM}
        g(x,y)
        =\begin{cases}
            \frac{x^2-2\,y^2\,\sin(x^2+y^2)}{\sqrt{x^2+y^2}}
            & \text{se } (x,y)\neq (0,0)
            \\
            0
            & \text{se } (x,y)= (0,0)
        \end{cases}
    \end{BM}

\end{questionBox}

\begin{questionBox}2{} % QII 1.1

    Considere que dado um número real positivo \(\delta\) existe um número real positivo \(\epsilon\), tal que se \((x,y)\neq (0,0)\) e \(\sqrt{x^2+y^2}<\epsilon\), então \(\lvert g(x,y) \rvert < \delta\).

    \begin{answerBox}{} % RS 
        \begin{flalign*}
            &
                \forall\,\delta>0\,\exists\,\epsilon>0:
                (x,y)\neq(0,0)\land\sqrt{x^2+y^2}<\epsilon
                \implies
                \lvert g(x,y)-0 \rvert
                <\delta
                \implies &\\&
                \implies
                \left\lvert
                    \frac{x^2-2\,y^2\,\sin(x^2+y^2)}{\sqrt{x^2+y^2}}
                \right\rvert
                \leq
                \frac{x^2+2\,y^2\,\lvert\sin(x^2+y^2)\rvert}{\sqrt{x^2+y^2}}
                \leq
                \frac{x^2+2\,y^2}{\sqrt{x^2+y^2}}
                \leq &\\&
                \leq
                \frac{2\,x^2+2\,y^2}{\sqrt{x^2+y^2}}
                \leq
                2\,\sqrt{x^2+y^2}
                \leq
                2\,\epsilon
                = \delta
                &\\&
                \therefore
                \exists \epsilon=\delta/2;
                % &\\&
                \qquad
                \lim_{(x,y)\to(0,0)}{
                    \frac{x^2-2\,y^2\,\sin(x^2+y^2)}{\sqrt{x^2+y^2}}
                }
                =0
            &
        \end{flalign*}
    \end{answerBox}

\end{questionBox}



\begin{questionBox}2{} % QII 1.2

    Determine, por definição, caso existam,

    \begin{BM}
        \pdv{g}{y}(0,0)\text{ e }
        \mdif{\left(
            \frac{2}{\sqrt{5}},
            \frac{1}{\sqrt{5}},
        \right)}
        \,g(0,0)
    \end{BM}
    
    Diga, justificando, se \textit{g} é diferenciável no ponto (0,0).

    \begin{answerBox}{} % RS 
        \begin{flalign*}
            &
                \pdv{g}{y}(0,0)
                = \lim_{h\to0}\frac{g(0,h)-0}{h}
                = \lim_{h\to0}{
                    \cfrac{
                        \cfrac{-2\,h^2\,\sin{h^2}}{\sqrt{h^2}}
                    }{h}
                }
            &
        \end{flalign*}
    \end{answerBox}

\end{questionBox}

\end{document}
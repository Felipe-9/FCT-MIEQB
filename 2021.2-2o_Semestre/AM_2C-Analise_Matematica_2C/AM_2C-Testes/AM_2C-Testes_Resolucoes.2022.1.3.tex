% !TEX root = ./AM_2C-Testes_Resolucoes-2022.1.3.tex
\providecommand\mainfilename{"./AM_2C-Testes_Resolucoes.tex"}
\providecommand \subfilename{}
\renewcommand   \subfilename{"./AM_2C-Testes_Resolucoes-2022.1.3.tex"}
\documentclass[\mainfilename]{subfiles}

\graphicspath{{\subfix{../images/AM_2C-Testes_Resolucoes-2022.1.3/}}}
% \tikzset{external/force remake=true} % - remake all

\begin{document}

\mymakesubfile{3}
[AM\,2C]
{Teste de Exame}
{Teste de Exame}

\group{}

% Q1
\begin{questionBox}1{}
    
    \begin{BM}
        \mathcal{P}ar_2:\left\{
            F = (2,0)
            \land
            D = \{(x,y)\in\mathbb{R}^2:x=-2\}
        \right\}
        \\
        \mathcal{E}li_2:\left\{
            P' = (0,1)
            \land
            F= (\sqrt{7},1)
            \land
            V = (4,1)
        \right\}
    \end{BM}

    \begin{questionBox}3{\(\mathcal{P}ar_2\)}
        
        \begin{flalign*}
            &
                x 
                = \frac{y^2}{4*2}
                = \frac{y^2}{8}
            &
        \end{flalign*}
        
    \end{questionBox}

    \begin{questionBox}3{\(\mathcal{E}li_2\)}
        
        \begin{flalign*}
            &
                1
                = \frac{x^2}{r_1^2}
                + \frac{(y-1)^2}{r_2^2}
                = \frac{4^2}{r_1^2}
                + \frac{(1-1)^2}{r_2^2}
                = \frac{16}{r_1^2}
                \land &\\&
                \land
                r_2^2 
                = r_1^2 - c^2
                = 16 - \sqrt{7}^2
                = 9
                \implies &\\&
                \implies
                1
                = \frac{x^2}{16}
                + \frac{(y-1)^2}{9}
                \implies
                9*16
                = 144
                = 9\,x^2
                + 16(y-1)^2
            &
        \end{flalign*}
        
    \end{questionBox}
    
\end{questionBox}

% Q2
\begin{questionBox}1{}
    
    \begin{BM}
        f(x,y):\mathbb{R}^2\to\mathbb{R} 
        = (3\,x-y^2,x^3-3\,y^2) = (u,v)
        \\
        \visinhanca(f(1,1))\text{ é invertivel}
    \end{BM}
    
    \begin{flalign*}
        &
            f(1,1)
            = (3*1-1^2, 1^3-3*1^2) 
            = (2,-2) 
            = (u,v)
            &\\[2ex]&
            \pdv{f(1,1)}{u}
            = \left(
                3\,\pdv{x}{u}
                -(1^2),
                1^3 - 3*2*1\,\pdv{y}{u}
            \right)
            = \left(
                3\,\pdv{x}{u}
                -1,
                1 - 6\,\pdv{y}{u}
            \right)
            &\\[2ex]&
            \pdv{f(1,1)}{v}
            = \left(
                3\,\pdv{x}{v}
                -(1^2),
                1^3 - 3*2*1\,\pdv{y}{v}
            \right)
            = \left(
                3\,\pdv{x}{v}
                -1,
                1 - 6\,\pdv{y}{v}
            \right)
            &\\[2ex]&
            \implies
            % 
        &
    \end{flalign*}
    
\end{questionBox}

% Q3
\begin{questionBox}1{}
    
    \begin{BM}
        x^4 + y^4 + (x+1)e^z + 8\sin(z) - 4 = 0
        \\
        (x',y',z') = (1,-1,0)
    \end{BM}

    \begin{flalign*}
        &
            % 0 
            % = 1^4 + (-1)^4 + (1+1)e^0 + 8\sin(0) - 4
            % = 1 + 1 + -2
            &\\[2ex]&
            % \pdv{f(1,-1,0)}{x}
            % = 
            \left(
                4\,x^3
                + y^4
                + e^z(1)
                + (x+1)
                \,e^z
                \,\pdv{z}{x}
                - 8\cos(z)\pdv{z}{x}
            \right)
            (1,-1,0)
            = &\\&
            = 4
            + 1
            + 1
            + 2\pdv{z}{x}
            - 8\pdv{z}{x}
            = 6
            - 6\pdv{z}{x}
            &\\[2ex]&
            \pdv{f(1,-1,0)}{y}
        &
    \end{flalign*}
    
\end{questionBox}

% Q4
\begin{questionBox}1{}
    
    \begin{BM}
        f(x,y,z) 
        = (
            x\,g(x/z),
            h(x^2\,y)
        )
    \end{BM}

    \begin{flalign*}
        &
            J 
            = (f\circ g)'(x)\in\mathcal{M}_{p\times n}
            : (f\circ g)'(x)_{(i,j)}
            = \jdv{f(g(x))_i}{x_j}
            \implies &\\&
            \implies
            J
            = \begin{bmatrix}
                  \pdv{f(g(-1,0,1))_1}{x}
                & \pdv{f(g(-1,0,1))_1}{y}
                & \pdv{f(g(-1,0,1))_1}{z}
                \\ 
                  \pdv{f(g(-1,0,1))_2}{x}
                & \pdv{f(g(-1,0,1))_2}{y}
                & \pdv{f(g(-1,0,1))_2}{z}
            \end{bmatrix}
            % = &\\&
            % = \begin{bmatrix}
            %       \pdv{x\,g(x/z)}{x}
            %     & \pdv{f(g(x))_1}{y}
            %     & \pdv{f(g(x))_1}{z}
            %     \\ 
            %       \pdv{f(g(x))_2}{x}
            %     & \pdv{f(g(x))_2}{y}
            %     & \pdv{f(g(x))_2}{z}
            % \end{bmatrix}
            = &\\&
            = \begin{bmatrix}
                g(1)
                +g'(1)
                & 
                0
                &
                -g'(1)
                \\
                0
                & 
                h'(0)
                & 
                0
            \end{bmatrix}
            % \pdv{(x+1)^2}{x}
            % = \pdv{x^2}{(x+1)}
            % \,\pdv{x+1}{x}
        &
    \end{flalign*}
    
\end{questionBox}

% Q5
\begin{questionBox}1{}
    
    \begin{BM}
        A = \left\{
            (x,y)\in\mathbb{R}^2
            : x^2 + y^2 \geq 1
            \land
            (x-1)^2 + y^2 \leq 1
        \right\}
        \\[2ex]
        \int_L {
            \left(
                x^3 - \frac{y^2}{2}
            \right)
            \odif{x}
            + 
            \left(
                y^5-1
            \right)
            \odif{y}
        }
    \end{BM}

    em coordenadas polares

    % \begin{center}
    %     %\pgfplotsset{height=7cm, width= .6\textwidth}
    %     \begin{tikzpicture}
    %     \begin{axis}
    %         [
    %             % xmajorgrids = true,
    %             % legend pos  = north west
    %         ]
    %         % Legends
    %         % \addlegendimage{empty legend}
    %         % \addlegendentry[Red]{\( x \)}
            
    %         % % Plot from csv file
    %         % \addplot[smooth, thick, mark=*] % mesh for colormap
    %         % table[
    %         %     col sep=comma,  % csv: comma,
    %         %     header=true,
    %         %     x index=3,      % x column on file
    %         %     y index=4,      % y column on file
    %         %     point meta=x,   % value to colormap
    %         % ]{  file.csv };
            
    %         % Plot from equation
    %         \addplot[
    %             smooth,
    %             thick,
    %             % Red,
    %             domain  = -2:2,
    %             samples = 0.4*\mysampledensity,
    %         ]{ sqrt(x^2-1) };
    %         \addplot[
    %             smooth,
    %             thick,
    %             % Red,
    %             domain  = -2:2,
    %             samples = 0.4*\mysampledensity,
    %         ]{ sqrt((x-1)^2-1) };
            
    %     \end{axis}
    %     \end{tikzpicture}
    % \end{center}

    \begin{flalign*}
        &
            x^2 + y^2
            = (x-1)^2 + y^2
            \implies
            \lvert x \rvert 
            = \lvert (x-1) \rvert
            \implies
            \left\{
                \begin{aligned}
                    x &= 1/2
                    \\
                    y &
                    = \pm\sqrt{1-1/4}
                    = \pm\sqrt{3}/2
                \end{aligned}
            \right\}
            % &\\[2ex]&
            % \int_L {
            %     \left(
            %         x^3 - \frac{y^2}{2}
            %     \right)
            %     \odif{x}
            %     + 
            %     \left(
            %         y^5-1
            %     \right)
            %     \odif{y}
            % }:
            % \left(
            %     \begin{aligned}
            %         &
            %         x = \cos(\theta)
            %         \ldiv{}
            %         y = \sin(\theta)
            %         \ldiv{}
            %         \lVert (x,y) \rVert
            %         = r
            %         &
            %     \end{aligned}
            % \right)
            % = &\\&
            % \int_L {
            %     \left(
            %         \cos(\theta)^3 - \frac{y^2}{2}
            %     \right)
            %     \odif{x}
            %     + 
            %     \left(
            %         y^5-1
            %     \right)
            %     \odif{y}
            % }
            &\\[2ex]&
            L = \left\{
                (x,y):
                \left(
                    \begin{aligned}
                        x &=\cos(\theta):
                        \quad&  0\leq&\theta\leq \pi/2
                        \\ 
                        x &=\cos(\theta)+1: 
                        \quad& \pi/2\leq&\theta\leq 2\,\pi
                        \\
                        y &=\sin(\theta)
                    \end{aligned}
                \right)
            \right\}
            &\\[2ex]&
            \int_L {
                \left(
                    x^3 - \frac{y^2}{2}
                \right)
                \odif{x}
                + 
                \left(
                    y^5-1
                \right)
                \odif{y}
            }
            % = &\\&
            % \int_
            = &\\&
            = \int_0^{\pi/2}{
                \left(
                    \left(
                        \cos(\theta)+1
                    \right)^3-
                    \left(
                        \sin(\theta)
                    \right)^2
                \right)
                (-\sin(\theta))
                \odif{\theta}
                + \left(
                    \left(
                        \sin(\theta)
                    \right)^5 - 1
                \right)
                \cos(\theta)
                \odif{\theta}
            }
            + &\\&
            + \int_{\pi/2}^{2\pi}{
                \left(
                    \left(
                        \cos(\theta)
                    \right)^3-
                    \left(
                        \sin(\theta)
                    \right)^2
                \right)
                (-\sin(\theta))
                \odif{\theta}
                + \left(
                    \left(
                        \sin(\theta)
                    \right)^5 - 1
                \right)
                \cos(\theta)
                \odif{\theta}
            }
            % 
            = &\\&
            % 
            = \int_0^{\pi/2}{
                \left(
                    \left(
                        \cos(\theta)+1
                    \right)^3-
                    \left(
                        \sin(\theta)
                    \right)^2
                \right)
            }
            + &\\&
            + \int_{\pi/2}^{2\pi}{
                \left(
                    \left(
                        \cos(\theta)
                    \right)^3-
                    \left(
                        \sin(\theta)
                    \right)^2
                \right)
                (-\sin(\theta))
                \odif{\theta}
            }
            + &\\&
            + \int_{0}^{2\pi}{
                \left(
                    \left(
                        \sin(\theta)
                    \right)^5 - 1
                \right)
                \cos(\theta)
                \odif{\theta}
            }
        &
    \end{flalign*}
    
\end{questionBox}

% Q6
\begin{questionBox}1{}
    
    \begin{BM}
        D = 
        (x,y,z):\\
        \left\{
            \begin{aligned}
                &
                \limsup D = \{(x,y,z): z = \sqrt{1-x^2-y^2}\}
                \ldiv{}
                \liminf D = \{(x,y,z): z = 1 - \sqrt{x^2+y^2}\}
                \ldiv{}
                y \geq 0
                \land
                y\leq x
                &
            \end{aligned}
        \right\}
    \end{BM}
    
\end{questionBox}

% Q7
\begin{questionBox}1{}
    
    \begin{BM}
        \int_{-2}^{0} {
            \left(
                \int_{0}^{x^2}{
                    x^2
                    \odif{y}
                }
            \right)
            \odif{x}
        }
        + \int_{0}^{2} {
            \left(
                \int_{0}^{x^2} {
                    x^2
                    \odif{y}
                }
            \right)
            \odif{x}
        }
    \end{BM}

    % \begin{center}
    %     %\pgfplotsset{height=7cm, width= .6\textwidth}
    %     % \tikzset{external/remake next=true}
    %     \begin{tikzpicture}
    %     \begin{axis}
    %         [
    %             % xmajorgrids = true,
    %             % legend pos  = north west
    %         ]
    %         % Legends
    %         % \addlegendimage{empty legend}
    %         % \addlegendentry[Red]{\( x \)}
            
    %         % % Plot from csv file
    %         % \addplot[smooth, thick, mark=*] % mesh for colormap
    %         % table[
    %         %     col sep=comma,  % csv: comma,
    %         %     header=true,
    %         %     x index=3,      % x column on file
    %         %     y index=4,      % y column on file
    %         %     point meta=x,   % value to colormap
    %         % ]{  file.csv };
            
    %         % Plot from equation
    %         \addplot[
    %             smooth,
    %             thick,
    %             % Red,
    %             domain  = -2:2,
    %             samples = 0.4*\mysampledensity,
    %         ]{  x^2 };
            
    %     \end{axis}
    %     \end{tikzpicture}
    % \end{center}

    \begin{flalign*}
        &
            y = x^2 \implies x = \pm\sqrt{y}
            &\\[2ex]&
            \int_{-2}^{0} {
                \left(
                    \int_{0}^{x^2}{
                        x^2
                        \odif{y}
                    }
                \right)
                \odif{x}
            }
            + \int_{0}^{2} {
                \left(
                    \int_{0}^{x^2} {
                        x^2
                        \odif{y}
                    }
                \right)
                \odif{x}
            }
            = &\\&
            = \int_0^4 {
                \left(
                    \int_{-2}^{-\sqrt{y}} {
                        x^2
                        \odif{y}
                    }
                    +\int_{\sqrt{y}}^{2} {
                        x^2
                        \odif{y}
                    }
                \right)
                \odif{y}
            }
        &
    \end{flalign*}
    
\end{questionBox}

% Q8
\begin{questionBox}1{}
    
    \begin{BM}
        L = \fronteira (D)
        \\[2ex]
        D = \left\{
            (x,y) \in\mathbb{R}^2
            : \left(
                \begin{aligned}
                    &
                    (x^2+y^2) \leq 4
                    \ldiv{}
                    x^2 + (y-1)^2 \geq 1
                    \ldiv{}
                    x \leq 0
                    \land
                    y \geq 0
                    &
                \end{aligned}
            \right)
        \right\}
    \end{BM}

    % \begin{center}
    %     %\pgfplotsset{height=7cm, width= .6\textwidth}
    %     \tikzset{external/remake next=true}
    %     \begin{tikzpicture}
    %     \begin{axis}
    %         [
    %             % xmajorgrids = true,
    %             % legend pos  = north west
    %             % domain = -5:5,
    %         ]
    %         % Legends
    %         % \addlegendimage{empty legend}
    %         % \addlegendentry[Red]{\( x \)}
            
    %         % % Plot from csv file
    %         % \addplot[smooth, thick, mark=*] % mesh for colormap
    %         % table[
    %         %     col sep=comma,  % csv: comma,
    %         %     header=true,
    %         %     x index=3,      % x column on file
    %         %     y index=4,      % y column on file
    %         %     point meta=x,   % value to colormap
    %         % ]{  file.csv };
            
    %         % % Plot from equation
    %         % \addplot[
    %         %     smooth,
    %         %     thick,
    %         %     % Red,
    %         %     domain  = -2:2,
    %         %     samples = 0.4*\mysampledensity,
    %         % ]{  x };

    %         \draw[white] (0,0) circle (4);
            
    %     \end{axis}
    %     \end{tikzpicture}
    % \end{center}

    \begin{flalign*}
        &
            \left\{
                \begin{aligned}
                    x &= \sin(t) \quad \pi \leq t \leq 2\pi
                    \\ 
                    y &= 1+\cos(t)
                \end{aligned}
            \right.
        &
    \end{flalign*}
    
\end{questionBox}

\group{}

\begin{questionBox}1{}
    
    \begin{BM}
        g(x,y)
        = \left\{
            \begin{aligned}
                \frac{
                    x^3 - 3\,x\,y^2
                }{
                    x^2 - y^2
                }
                \quad& se (x,y)\neq(0,0)
                \\
                0
                \quad& se (x,y)=(0,0)
            \end{aligned}
        \right\}
    \end{BM}
    
\end{questionBox}

\setcounter{subquestion}{1}

\begin{questionBox}2{}
    
    \begin{questionBox}3{}
        
        \begin{flalign*}
            &
                \pdv{g}{x}(0,0)
                = \frac{
                    3\,x^2 - 3\,y^2
                }{
                    x^2 - y^2
                }(0,0)
                = \frac{
                    3\,(x^2 - y^2)
                }{
                    x^2 - y^2
                }(0,0)
                = 3
            &
        \end{flalign*}
        
    \end{questionBox}

    \begin{questionBox}3{}
        
        \begin{flalign*}
            &
                \pdv{g}{y}
                = \frac{
                    3\,x\,(2\,y)
                }{
                    2\,y
                }(0,0)
                = 3\,(0)
                = 0
            &
        \end{flalign*}
        
    \end{questionBox}
    
\end{questionBox}

\begin{questionBox}2{}
    
    \begin{flalign*}
        &
            \pdv{g}{x} = 3
            &\\[2ex]&
            \pdv{g}{y} = 0
            &\\[2ex]&
            \pdv{g}{x,x}
            = \pdv{
                \frac{
                    3\,x^2 - 3\,y^2
                }{
                    2\,x - y^2
                }
            }{x}
        &
    \end{flalign*}
    
\end{questionBox}

\begin{questionBox}1{}
    
    \begin{BM}
        f:\mathbb{R}^2 
        \to \mathbb{R}^2
        \\
        (x,y)
        \to x^2+x\,y+y^2
    \end{BM}

    \begin{flalign*}
        &
            0 = \pdv{f(x,y)}{x}
            = 2\,x+y
            \implies
            y=-2\,x
            &\\[2ex]&
            \pdv{f(x,y)}{x,x}
            = \pdv{2\,x+y}{x}
            = 2
            &\\[2ex]&
            0
            = \pdv{f(x,y)}{y}
            = x+2\,y
            \implies y=-x/2
            &\\[2ex]&
            \pdv{f(x,y)}{y,y}
            = \pdv{x+2\,y}{y}
            = 2
            &\\[4ex]&
            \left\{
                (x,y)\in\mathbb{R}^2
                : y=-2\,x \lor y = -x/2
            \right\}
            \text{ são extremos inferiores}
        &
    \end{flalign*}
    
\end{questionBox}

\group{}

\begin{questionBox}1{}
    
    \begin{BM}
        x^2+y^2+(z-2)^2 = 4
        \\
        z = x^2+y^2
    \end{BM}

    \begin{flalign*}
        &
            (z-2)^2 = 4
            \implies
            z = 4
            &\\&
            z + (z-2)^2 
            = z^2-3\,z + 4
            = z(z-3) + 4
            = 4
            \implies
            z = 3 \lor z=0
            &\\&
            \int_{0}^{3} {
                2\,\pi\,\sqrt{z}
                \odif{z}
            }
            = 2\,\pi
            \,\int_{0}^{3} {
                z^{1/2}
                \odif{z}
            }
            = 2\,\pi
            (2/3)(z^{3/2})\big\vert_0^3
            = (4/3)\,\pi
            (3^{3/2})
            = 4\pi\sqrt{3}
        &
    \end{flalign*}
    
\end{questionBox}

\begin{questionBox}1{}
    
    \begin{BM}
        D\subset\mathbb{R}^3
        : D
        = \left\{
            \begin{aligned}
                &
                (x,y,z)\in\mathbb{R}^3
                : x^2+y^2+z^2
                \leq 0
                \ldiv{}
                3\,x^2+3\,y^2\leq z^2
                \land
                z\geq0
                &
            \end{aligned}
        \right\}
    \end{BM}

    \begin{questionBox}3{}

        \begin{BM}
            \iiint_D{
                (x^2+y^2)
                \odif{x,y,z}
            }
        \end{BM}
        
        \begin{flalign*}
            &
                z^2 
                = 3(z^2-9)
                = 3z^2-27
                \implies
                3\sqrt{3/2} = z
                \implies
                \rho = \sin^{-1}(3\sqrt{3/2})
                &\\&
                \iiint_D{
                    (x^2+y^2)
                    \odif{x,y,z}
                }
                = \int_{0}^{2\pi} {
                    \left(
                        \int_{0}^{\sin^{-1}(3\sqrt{3/2})} {
                            3
                            \odif{\rho}
                        }
                    \right)
                    \odif{\theta}
                }
            &
        \end{flalign*}
        
    \end{questionBox}
    
\end{questionBox}

\begin{questionBox}1{}
    
    \begin{BM}
        z = 1+x^2+y^2:\quad 1\leq z\leq 2
        \\[2ex]
        \iint_S {
            \nabla\times\left(
                y\,\hat{\imath}
                + x\,z\,\hat{\jmath}
                + \hat{k}
            \right)
            \cdot
            \vec{n}\odif{S}
        }
    \end{BM}
    
\end{questionBox}

\group{}

\begin{questionBox}1{}
    
    \begin{BM}
        C\subset S
        = \left\{
            (x,y,z):
            a\,x+b\,y+c\,z+d=0
        \right\}
        \\[2ex]
        \frac{1}{2\,\lVert \vec{N} \rVert}
        \int_C {
            (b\,z-c\,y)
            \odif{x}
            + (c\,x-a\,z)
            \odif{y}
            + (a\,y-b\,x)
            \odif{z}
        }
    \end{BM}
    
\end{questionBox}

\end{document}
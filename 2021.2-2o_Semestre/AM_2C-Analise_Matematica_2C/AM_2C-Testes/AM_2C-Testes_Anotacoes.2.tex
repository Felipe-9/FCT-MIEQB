% !TEX root = ./AM_2C-Testes_Anotacoes.2.tex
\documentclass["./AM_2C-Testes_Anotacoes.tex"]{subfiles}

% \tikzset{external/force remake=true} % - remake all

\begin{document}

% \graphicspath{{\subfix{./.build/figures/AM_2C-Testes_Anotacoes.2}}}
% \tikzsetexternalprefix{./.build/figures/AM_2C-Testes_Anotacoes.2/}

\mymakesubfile{2}[AM2C]
{Anotações Teste 2} % Subfile Title
{Anotações Teste 2} % Part Title

\begin{sectionBox}1m{Produto Interno}

  \begin{BM}
    p(X,Y) = X\vert Y = \sum_{k=1}^{n} x_k\,y_k \quad\{X,Y\}\in E
    \\ P:E\times E\to\mathbb{R}
  \end{BM}

  \begin{itemize}
    \item \( X\vert X\geq 0 \land (X\vert X=0\iff X=0) \)
    \item \( X\vert Y = Y\vert X \cftdotfill{\cftdotsep}\forall\,\{X,Y\}\in E \)
    \item \(
        \lambda\,X\vert Y = \lambda\,(X\vert Y)\cftdotfill{\cftdotsep}
        \forall\,\lambda\in\mathbb{R}
        \land\forall\,\{X,Y\}\in E
      \)
    \item \( (X+Y)\vert Z = (X\vert Z) + (Y\vert Z) \cftdotfill{\cftdotsep}\forall\,\{X,Y,Z\}\in E \)
  \end{itemize}

\end{sectionBox}

\begin{sectionBox}1m{Norma}

  \begin{BM}
    N(X)_p
    =   ||X||_p
    :=  \left(\sum_{k=1}^n |x_i|^p\right)^{1/p}
    =   \sqrt{X\vert X}
    \quad X\in\mathbb{R}^n
    \\
    N(\cdot): E\to\mathbb{R}
  \end{BM}

  \begin{itemize}
    \item \( N(X)\geq 0 \left(\land N(X)=0\iff x=0 \right)\)
    \item \( N(\lambda\,X) = \myvert{\lambda}\,N(X) \quad\forall\,\lambda\in\mathbb{R} \)
    \item \( N(X+Y)\leq N(X)+N(Y) \quad \forall\,\{X,Y\}\in E \)
  \end{itemize}

\end{sectionBox}

\begin{sectionBox}1m{Multiplicadores de Lagrange} % S

  \begin{BM}
    \gdif{f(x_0,y_0)}
    = \lambda\,\gdif{g(x_0,y_0)}
  \end{BM}

  \paragraph*{Motivação} Pretende-se maximizar a função \(f(x_0,y_0)\) sugeita a restrição \(g(x_0,y_0)=0\), o ponto \((x_0,y_0)\) pertençe tanto a uma curva de nível de \textit{f} quanto a \textit{g} onde a reta tangente das curvas é igual, coincidentemente os vetores normáis tambem coencidem 

\end{sectionBox}

\end{document}

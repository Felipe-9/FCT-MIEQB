% !TEX root = ./AM_2C-Testes_Resolucoes.2022.2.1.tex
\providecommand\mainfilename{"./AM_2C-Testes_Resolucoes.tex"}
\providecommand \subfilename{}
\renewcommand   \subfilename{"./AM_2C-Testes_Resolucoes.2022.2.1.tex"}
\documentclass[\mainfilename]{subfiles}

% \tikzset{external/force remake=true} % - remake all

\begin{document}

% \graphicspath{{\subfix{./.build/figures/AM_2C-Testes_Resolucoes.2022.2.1}}}
% \tikzsetexternalprefix{./.build/figures/AM_2C-Testes_Resolucoes.2022.2.1/}

\mymakesubfile{2}
[AM 2C]
{02/11/22 - Teste 2 Resolução} % Subfile Title
{02/11/22 - Teste 2 Resolução} % Part Title

\group{}

\begin{questionBox}1{ % Q1
    Seja \textit{E} um espaço vectorial onde está definido um produto interno notado com o símbolo \(\vert\) e seja \(\myVert{\cdot}\) a norma induzida pelo produto interno. Apenas uma das seguintes afirmações é verdadeira. Indique qual.
    \paragraph*{Nota: }  No que segue \textit{u,v} são elementos arbitrários de \textit{E}
} % Q1
    \begin{enumerate}
        \item \(\myVert{u+v}<\myvert{u}+\myvert{v}\)
        \item se \(
            u\neq0
            ,\myVert{u}+\myVert{-u}=0
        \)
        \item \(
            (\lambda\,u\vert v)=\myvert{\lambda}(u\vert v)
            ,\forall\,\lambda\in\mathbb{R}
        \)
        \item se \(
            u\neq0\land v=\frac{u}{\myVert{u}}
            \implies u\vert v=1
        \)
        \item se \(
            \myVert{u}=\myVert{v}=1
            \implies \myVert{u+v}=2
        \)
        \item se \(
            \myVert{u}=1
            \implies-\myVert{v}\leq(u\vert v)\leq\myVert{v}
        \)
    \end{enumerate}
    \answer{}
\end{questionBox}

\group{}

\begin{questionBox}1{ % Q2.1
    Considere a função real \textit{g}, de duas variáveis reais, definda por
    \begin{BM}
        g_{(x,y)}
        =\begin{cases}
            \frac{x^4\,\cos(2\,x)-2\,x^2\,y^2}{(x^2+y^2)^{3/2}}
            ,\quad&\text{se } (x,y)\neq(0,0)
            \\
            0,\quad&\text{se } (x,y)=(0,0)
        \end{cases}
    \end{BM}
} % Q1
    \begin{questionBox}2{ % Q2.1.1
        Montre, por definiçao, que \textit{g(x,y)} é continua em (0,0).
    } % Q2.1.1
        \answer{}
        \begin{flalign*}
            &
                \forall\,\delta>0\,\exists\epsilon>0
                :(x,y)\neq0\land\sqrt{x^2+y^2}<\epsilon
                \implies &\\&
                \implies
                \myvert{g(x,y)-0}
                =
                \myvert{
                    \frac{
                        x^4\,\cos(2\,x)-2\,x^2\,y^2
                    }{
                        (x^2+y^2)^{3/2}
                    }
                }
                = &\\&
                =
                \frac{
                    x^4\,\myvert{\cos(2\,x)}+2\,x^2\,y^2
                }{
                    (x^2+y^2)^{3/2}
                }
                \leq &\\&
                \leq
                \frac{
                    x^4*1+2\,x^2\,y^2 + y^4
                }{
                    (x^2+y^2)^{3/2}
                }
                = &\\&
                =
                \frac{
                    (x^2+y^2)^2
                }{
                    (x^2+y^2)^{3/2}
                }
                = &\\&
                = (x^2+y^2)^{1/2}
                <\epsilon
                =\delta
                \implies &\\&
                \implies
                \lim_{(x,y)\to(0,0)}{
                    g(x,y)
                }=0=g(0,0)
                &\\&\therefore g\text{ é continua em }(0,0)
            &
        \end{flalign*}
    \end{questionBox}
    \begin{questionBox}2{ % Q2.1.2
        Determine, por definição, \(\pdv{g}{x}(0,0)\text{ e }\pdv{g}{y}(0,0)\).
    } % Q2.1.2
        \answer{}
        \begin{flalign*}
            &
                \pdv{g}{x}(0,0)
                = \lim_{h\to0}{
                    \frac{g(h,0)-g(0,0)}{h}
                }
                = \lim_{h\to0}{
                    \frac{
                        \frac{h^4\,\cos(2\,h)-2\,h^2\,0^2}{(h^2+0^2)^{3/2}}
                        -0
                    }{h}
                }
                = &\\&
                = \lim_{h\to0}{
                    \frac{
                        h^4\,\cos(2\,h)
                    }{
                        h\,\myvert{h}^3
                    }
                }
                = \lim_{h\to0}{
                    \frac{
                        \myvert{h}^4\,\cos(2\,h)
                    }{
                        h\,\myvert{h}^3
                    }
                }
                = \lim_{h\to0}{
                    \frac{\myvert{h}}{h}
                    \,\cos(2\,h)
                }
                = &\\&
                = \lim_{h\to0}{
                    \sgn(h)
                    \,\cos(2\,h)
                }
                = \sgn(h)
                % ============= Partial on y ============= %
                \neq &\\[3ex]&
                \neq \pdv{g}{y}(0,0)
                = \lim_{h\to0}{
                    \frac{g(0,h)-g(0,0)}{h}
                }
                % = &\\&
                = \lim_{h\to0}{
                    \frac{
                        \frac{0^4\,\cos(2*0)-2*0^2\,h^2}{(0^2+h^2)^{3/2}}
                        -0
                    }{h}
                }
                = &\\&
                = \lim_{h\to0}{
                    \frac{0}{h\,\myvert{h}^3}
                }
                = 0
            &
        \end{flalign*}
    \end{questionBox}
    \begin{questionBox}2{ % Q2.1.3
        Estude a diferenciabilidade em \textit{g} no ponto (0,0).
    } % Q2.1.3
    \end{questionBox}
\end{questionBox}

\end{document}
% !TEX root = ./AM_2C-Tests_Resolutions.2023.1.4.tex
\providecommand\mainfilename{"./AM_2C-Tests_Resolutions.tex"}
\providecommand \subfilename{}
\renewcommand   \subfilename{"./AM_2C-Tests_Resolutions.2023.1.4.tex"}
\documentclass[\mainfilename]{subfiles}

% \tikzset{external/force remake=true} % - remake all

\begin{document}

% \graphicspath{{\subfix{./.build/figures/AM_2C-Tests_Resolutions.2023.1.4}}}
% \tikzsetexternalprefix{./.build/figures/AM_2C-Tests_Resolutions.2023.1.4/graphics/}

\mymakesubfile{4}
[AM 2C]
{Exame Epoca Especial 2023 Resolução} % Subfile Title
{Exame Epoca Especial 2023 Resolução} % Part Title

\begin{questionBox}1{ % Q1
    Considere a função vetorial
    \begin{BM}
        \vec{r}:\myrange{0,4\,\pi}\to\mathbb{R}^3,
        \vec{r}(t)
        = \left(
            \begin{aligned}
                &
                \cos(\pi+t)\,\hat{\imath}
                &+\\+&
                \sin(\pi+t)\,\hat{\jmath}
                &+\\+&
                \sqrt{3}\,t\,\hat{k}
                &
            \end{aligned}
        \right)
    \end{BM}
    e designe por \textit{C} a curva orientada definida por \(\vec{r}\). A equação vetorial da reta tangente à curva \textit{C} no ponto \((1,0,3\,\sqrt{3}\,\pi)\) é:
} % Q1
    % Reta tg a linha simples
    % P0=(f(t0),g(t0),z(t0))
    \begin{flalign*}
        &
            % Encontrar t0 que tangencia a curva
            \begin{cases}
                \cos(\pi+t_0)=1
                \\
                \sin(\pi+t_0)=0
                \\
                \sqrt{3}\,t_0 = 3\,\sqrt{3}\,\pi\implies t_0=3\,\pi
            \end{cases}
            &\\[3ex]&
            % Descobrir o r' derivada de r(t)
            r'(t_0)
            = \odv{\cos(\pi+t)}{t}(t_0)\hat{\imath}
            + \odv{\sin(\pi+t)}{t}(t_0)\hat{\jmath}
            + \odv{\sqrt{3}\,t}{t}(t_0)\hat{k}
            = &\\&
            = -\sin(\pi+3\,\pi)\hat{\imath}
            + \cos(\pi+3\,\pi)\hat{\jmath}
            + \sqrt{3}\hat{k}
            = &\\&
            = \hat{\jmath}
            + \sqrt{3}\hat{k}
            \implies &\\[3ex]&
            \implies
            (x,y,z)
            = (1,0,3\,\sqrt{3}\pi)
            + t\,(0,1,\sqrt{3})
        &
    \end{flalign*}
\end{questionBox}

\begin{questionBox}1{ % Q2
    A equação do plano tangente à superfície de nível
    \begin{BM}
        x^3+y^3+(x+1)\,e^z=2
    \end{BM}
    no ponto \((1,-1,0)\) é:
} % Q2
    \answer{}
    \begin{flalign*}
        &
            \left(
                \begin{aligned}
                    &
                        \pdv{f}{x}(x_0)\,(x-x_0)
                    &+\\+&
                        \pdv{f}{y}(y_0)\,(y-y_0)
                    &+\\+&
                        \pdv{f}{z}(z_0)\,(z-z_0)
                    &
                \end{aligned}
            \right)
            = \left(
                \begin{aligned}
                    &
                        (3\,(1)^2+e^0)\,(x-1)
                    &+\\+&
                        (3\,(-1)^2)\,(y+1)
                    &+\\+&
                        (1+1)e^(0)\,z
                    &
                \end{aligned}
            \right)
            = &\\&
            = 4\,x
            + 3\,y
            + 2\,z
            - 1
            = 0
        &
    \end{flalign*}
    % Para cada caso de formula de superficie temos diff formas de fazer
    % Nesse caso usamos f(x,y,z)=0
    % \answer{}
    % \begin{flalign*}
    %     &
    %         \nabla{F}{(x,y,z)}(x-a,y-b,z-c)
    %         = \left(
    %             \begin{aligned}
    %                 3\,x^2+e^z,
    %                 \\
    %                 3\,y^2
    %                 \\
    %                 (x+1)\,e^z
    %             \end{aligned}
    %         \right)(1,-1,0)
    %         = (4,3,2)
    %         \implies &\\&
    %         \implies
    %         4\,(x-1)
    %         +3\,(y+1)
    %         +2\,z
    %         = 4\,x
    %         +3\,y
    %         +2\,z
    %         -1
    %     &
    % \end{flalign*}
    Resposta D.
\end{questionBox}

\begin{questionBox}1{ % Q3
    Considere o subconjunto de \(\mathbb{R}^2\) definido por
    \begin{BM}
        C=\left\{
            (x,y)\in\myrange{-1,1}\times\myrange{-1,1}
            : x^2+y^2>1
        \right\}
        \cup \{(0,0)\}
    \end{BM}
    Tem-se (\(C'\) é o conjunto dos pontos de acumulação de \(C\text{ e }\bar{C}\) é a aderência de \textit{C})
} % Q3
    % \begin{itemize}
    %     % \item \(\int(C)\neq C:\exists c\in\fronteira(C)\neq Int(C)\)
    % \end{itemize}
\end{questionBox}

\begin{questionBox}1{ % Q4
    Seja \textit{C} o segmineto de reta \textit{[AB]} com \(A=(0,0)\text{ e }B=(1,1)\), percorrido de \textit{A} para \textit{B}. O valor do integral curvilíneo
    \begin{BM}
        \int_C{
            x^2\,\odif{x}
            + y^2\,\odif{y}
        }
    \end{BM}
    % Integrais de curva: Paremtrizar do segimento e multiplicar pela derivada da parametrização
    % \begin{BM}
    %     \int_{C,t}f(s)\odif{s}
    %     = \int_a^b{
    %         f(\phi(t))\myVert{\phi'(t)}\odif{t}
    %     }
    % \end{BM}
    % Integral de linha/fronteira
    % \int (f(x,y)\odif{x}+g(x,y)\odif{y})
    % usar riemman-green
} % Q4
    \answer{}
    \begin{flalign*}
        &
            \int_C{
                x^2\,\odif{x}
                + y^2\,\odif{y}
            }
            = \int_C{
                x^2\,\odif{x}
                + y^2\,\odif{y}
            }
        &
    \end{flalign*}
    \begin{flalign*}
        &
            \phi(t)
            =A+(B-A)
            =(0,0)+((1,1)-(0,0))\,t
            =(1,1)\,t
            =(t,t)
            :t\in\myrange{0,1}
            \implies &\\&
            \implies
            \int_{C,t}f(s)\odif{s}
            = \int_a^b{
                f(\phi(t))\myVert{\phi'(t)}\odif{t}
            }
            = \int_0^1{
                f((t,t))\myVert{(1,1)}\odif{t}
            }
            = \int_0^1{
                (t^2+t^2)\,\sqrt{1^1+1^1}\odif{t}
            }
            = \int_0^1{
                2\,t^2\,\sqrt{2}\odif{t}
            }
            = 2\sqrt{2}\int_0^1{
                t^2\,\odif{t}
            }
            = 2\sqrt{2}\,t^3/3\big\vert_0^1
        &
    \end{flalign*}
\end{questionBox}

\begin{questionBox}1{ % Q5
    \begin{itemize}
        \item limites direcionais:
    \end{itemize}
} % Q5
    \answer{}
    \begin{questionBox}2{ % Q1
        B
    } % Q1
        \begin{flalign*}
            &
                \myvert{
                    f(x,y)-0
                }
                = \myvert{
                    \frac{x\,y^2}{x^4+y^2}
                }
                = \frac{y^2\,\myvert{
                    x
                }}{
                    x^4+y^2
                }
                \leq 
                \myVert{(x,y)}
                \sqrt{x^2+y^2}
                &\\[3ex]&
                \lim_{(x,y)\to(0,0)}\frac{x\,y^2}{x^4+y^2}
                = \lim_{r\to0}\frac{
                    r\,\cos(\theta)\,(r\,\sin(\theta))^2
                }{(r\,\cos(\theta))^4+(r\,\sin(\theta))^2}
                = &\\&
                = \lim_{r\to0}\frac{
                    r\,\cos(\theta)\,\sin^2(\theta)
                }{
                    r^2\,\cos^4(\theta)+\sin^2(\theta)
                }
                = \frac{0}{0+\sin^2(\theta)}
                = 0
                &\\&
                \therefore f \text{ é continua em } (0,0)
            &
        \end{flalign*}
    \end{questionBox}
\end{questionBox}

\begin{questionBox}1{ % Q6
    \begin{BM}
        \lim_{(x,y)\to(1,1)}{
            \frac
            {3(y-1)\,\exp(-(x-1)^2)}
            {\sqrt{x-1}^2+(y-1)^2}
        }
    \end{BM}
} % Q6
    \answer{}
    \begin{flalign*}
        &
            \lim_{(x,y)\to(1,1)}{
                \frac
                {3(y-1)\,\exp(-(x-1)^2)}
                {\sqrt{(x-1)^2+(y-1)^2}}
            }
            = &\\&
            = \lim_{x\to1,y=m(x-1)+1}{
                \frac
                {3(m(x-1)+1-1)\,\exp(-(x-1)^2)}
                {\sqrt{(x-1)^2+(m(x-1)+1-1)^2}}
            }
            = &\\&
            = \lim_{x\to1,y=m(x-1)+1}{
                \frac
                {3(m(x-1))\,\exp(-(x-1)^2)}
                {\sqrt{(m^2+1)(x-1)^2}}
            }
            = &\\&
            = \lim_{x\to1,y=m(x-1)+1}{
                \frac
                {3\,m\,\exp(-(x-1)^2)}
                {\frac{\myvert{x-1}}{x-1}\sqrt{(m^2+1)}}
            }
            = &\\&
            = \lim_{x\to1,y=m(x-1)+1}{
                \frac
                {3\,m\,\exp(-(x-1)^2)}
                {\frac{\myvert{x-1}}{x-1}\sqrt{(m^2+1)}}
            }
            \implies &\\[3ex]&
            \implies
            \lim_{x\to1^+,y=m(x-1)+1}{
                \frac
                {3\,m\,\exp(-(x-1)^2)}
                {\frac{\myvert{x-1}}{x-1}\sqrt{(m^2+1)}}
            }
            = \frac
                {3\,m\,\exp(0)}
                {\sqrt{(m^2+1)}}
            = \frac
                {3\,m}
                {\sqrt{(m^2+1)}}
            ; &\\[3ex]&
            \lim_{x\to1^-,y=m(x-1)+1}{
                \frac
                {3\,m\,\exp(-(x-1)^2)}
                {\frac{\myvert{x-1}}{x-1}\sqrt{(m^2+1)}}
            }
            = \frac
                {3\,m\,\exp(0)}
                {-\sqrt{(m^2+1)}}
            = \frac
                {3\,m}
                {-\sqrt{(m^2+1)}}
        &
    \end{flalign*}
\end{questionBox}

\begin{questionBox}1{ % Q7
    \begin{BM}
        f(x,y)
        =\begin{cases}
            \frac{x\,y^2}{x^4+y^2}&:(x,y)\neq(0,0)
            \\
            0&:(x,y)=(0,0)
        \end{cases}
    \end{BM}
} % Q7
    \begin{flalign*}
        &
            \pdv{f}{x}(0,0)
            = \lim_{h\to0}{
                \frac{
                f(h,0)
                - f(0,0)
                }{
                    h
                }
            }
            = \lim_{h\to0}{
                \frac{
                \frac{h*0^2}{h^4+0^2}
                }{
                    h
                }
            }
            = 0
            ; &\\[3ex]&
            \pdv{f}{y}(0,0)
            = \lim_{k\to0}{
                \frac{
                f(0,k)
                - f(0,0)
                }{
                    k
                }
            }
            = \lim_{k\to0}{
                \frac{
                \frac{0*k^2}{0^4+k^2}
                }{
                    k
                }
            }
            = 0
            ; &\\[3ex]&
            D_{\vec{u}}f(0,0)
            = \lim_{t\to0}{
                \frac{
                f(0+t/\sqrt{2},0+t/\sqrt{2})
                - f(0,0)
                }{
                    t
                }
            }
            = &\\&
            = \lim_{t\to0}{
                \frac{
                    \frac{(t/\sqrt{2})*(t/\sqrt{2})^2}{(t/\sqrt{2})^4+(t/\sqrt{2})^2}
                }{
                    t
                }
            }
            = \lim_{t\to0}{
                \frac{1/2\sqrt{2}}{t^2/4+1/2}
            }
            = \frac{1/2\,\sqrt{2}}{1/2}
            = 1/\sqrt{2}
        &
    \end{flalign*}
\end{questionBox}

\begin{questionBox}1{ % Q8
    \begin{BM}
        f:\mathbb{R}^2\to\mathbb{R}; C^1\in\mathbb{R}^2
        :\nabla{f(1/2,0)}=(-1,1)
        \\
        H(x,y)
        =f\left(
            \frac{\sin{y}}{1+x^2}
            ,x+\cos(2\,y)
        \right)
    \end{BM}
    % C^n: deriv parciais de ordem n são contínuas em R2
} % Q8
    \answer{}
    \begin{flalign*}
        &
            H(x,y)
            =f\left(
                \frac{\sin{y}}{1+x^2}
                ,x+\cos(2\,y)
            \right)
            =f\left(
                \phi(x,y)
                ,\rho(x,y)
            \right)
            ; &\\[3ex]&
            \odv{H}{x}(1,\pi/2)
            = \odv{f}{\phi(x,y)}
            \left(
                \frac{\sin{\pi/2}}{1+1^2}
                ,1+\cos(2\,\pi/2)
            \right)
            \,\odv{\phi}{x}(1,\pi/2)
            = &\\&
            = -1
            \,\frac{\sin{\pi/2}}{(1+1^2)^2}
            \,2(1+1^2)
            = -1
            ; &\\[3ex]&
            \odv{H}{x}(1,\pi/2)
            = \odv{f}{x}
            \left(
                \frac{\sin{\pi/2}}{1+1^2}
                ,1+\cos(2\,\pi/2)
            \right)
            \,\odv{\frac{\sin{y}}{1+x^2}}{x}(1,\pi/2)
            = &\\&
            = \odv{f}{x}
            \left(\frac{1}{2},0\right)
            \frac{\sin{\pi/2}}{(1+1^2)^2}\,2*1
            = \odv{f}{x}
            \left(\frac{1}{2},0\right)
            \frac{\sin{\pi/2}}{(1+1^2)^2}\,2*1
            = -2
        &
    \end{flalign*}
\end{questionBox}

\begin{questionBox}1{ % Q9
    Seja \(\varphi:\mathbb{R}\to\mathbb{R}\) uma função de classe \(
        C^2
        \text{ em }
        \mathbb{R}
        \text{ tal que }
        \varphi'(0)=5,
        \varphi"(0)=-1
        \text{. Considere }
        u(x,t) = \varphi(x^2-2\,t)
    \)
    . Tem-se:
} % Q9
    \answer{}
    \begin{flalign*}
        &
            \pdv[order=2]{u}{x}(2,2)
            =\pdv[order=2]{(\varphi(x^2-2\,t))}{x}(2,2)
            =\pdv[order=1]{(\varphi'(x^2-2\,t)\,2\,x)}{x}(2,2)
            = &\\&
            = 2\,\left(
                \varphi"(2^2-2*2)\,2
                + \varphi'(2^2-2*2)
            \right)
            = &\\&
            = 2\,\left(
                -1*2
                + 5
            \right)
            = 6
        &
    \end{flalign*}
\end{questionBox}

\begin{questionBox}1{ % Q10
    Seja
    \begin{BM}
        \int_0^2{
            \int_y^2{
                \exp(x^2)\,\odif{x,y}
            }
        }
    \end{BM}
    Tem-se:
} % Q10
    \paragraph*{Sugestão:} Troque a ordem de integração
    \answer{}
    \begin{flalign*}
        &
            \begin{cases}
                y\in\myrange{0,2}
                \\
                x\in\myrange{y,2}
            \end{cases}
            &\\&
            \begin{cases}
                y\in\myrange{0,x}
                \\
                x\in\myrange{0,2}
            \end{cases}
            ; &\\[3ex]&
            \int_0^2{
                \int_y^2{
                    \exp(x^2)\,\odif{x,y}
                }
            }
            = \int_0^2{
                \int_0^x{
                    \exp(x^2)\,\odif{y,x}
                }
            }
            = \int_0^2{
                \exp(x^2)\,(x-0)\odif{x}
            }
            = \int_0^2{
                \exp(x^2)\,x\odif{x}
            }
            = &\\&
            = \int_0^2{
                \exp(x^2)\,2\,x\odif{x}/2
            }
            = (\exp{2^2}-\exp(0^2))/2
            = (\exp{4}-1)/2
        &
    \end{flalign*}
\end{questionBox}

\begin{questionBox}1{ % Q11
    A equação
    \begin{BM}
        \exp(x\,z)
        +y\,\sin{x}
        -y^2+z^3+2\,x
        =2\,\pi
    \end{BM}
    define implicitamente \textit{x} como função de \textit{y} e \textit{z} numa vizinhança do ponto \((x_0, y_0, z_0)=(\pi,1,0)\). Para essa função tem-se:
} % Q11
    \answer{}
    \begin{flalign*}
        &
            % % Define implicitamente:
            % \begin{cases}
            %     f(x,y,z)=0
            %     \\
            %     f\in C^1
            %     \\
            %     \pdv{f}{x}(x_0,y_0,z_0)\neq0
            % \end{cases}
            % &\\[3ex]&
            % Pelo teorema da função implícita
            \pdv{x}{y}(1,0)
            = -\frac
            {\pdv{f}{y}(\pi,1,0)}
            {\pdv{f}{x}(\pi,1,0)}
            = -\frac
            {\sin(\pi)-2*1}
            {\exp(\pi*0)\,0+1\,\cos{\pi}+2}
            = 2
        &
    \end{flalign*}
    \vspace{5ex}
    \begin{flalign*}
        &
            \pdv{x}{y}(\pi,1,0)
            : 
            \exp(x\,z)
            +y\,\sin{x}
            -y^2+z^3+2\,x
            =2\,\pi
            \implies &\\&
            \implies
            \pdv{(
                \exp(x\,z)
                +y\,\sin{x}
                -y^2+z^3+2\,x
            )}{y}(\pi,1,0)
            = &\\&
            = \left(
                \begin{aligned}
                    &
                        \pdv{
                            \exp(x\,z)
                        }{y}(\pi,1,0)
                    &+\\+&
                        \pdv{
                            y\,\sin{x}
                        }{y}(\pi,1,0)
                    &+\\-&
                        \pdv{
                            y^2
                        }{y}(\pi,1,0)
                    &+\\+&
                        \pdv{
                            z^3
                        }{y}(\pi,1,0)
                    &+\\+&
                        \pdv{
                            2\,x
                        }{y}(\pi,1,0)
                    &
                \end{aligned}
            \right)
            % = &\\&
            = \left(
                \begin{aligned}
                    &
                        0
                    &+\\+&
                        \sin{\pi}
                        1\,\cos{\pi}\,\pdv{x}{y}(1,0)
                    &+\\-&
                        2\,(1)
                    &+\\+&
                        3\,(0)^2\,\pdv{z}{y}(\pi,1)
                    &+\\+&
                        2\,\pdv{x}{y}(1,0)
                    &
                \end{aligned}
            \right)
            = &\\&
            = 
            -2+\pdv{x}{y}(1,0)
            = &\\[3ex]&
            =\pdv{(2\,\pi)}{x}
            = 0
            \implies &\\[3ex]&
            \implies
            \pdv{x}{y}(1,0)=2
        &
    \end{flalign*}
\end{questionBox}

\begin{questionBox}1{ % Q12
    Considere a função
    \begin{BM}
        f(x,y)
        = y^4/2-x\,y^2+x^2-4\,x
    \end{BM}
    Escolha a affirmativa correta
} % Q12
    \answer{}
    \begin{flalign*}
        &
            \det H_f=\begin{vmatrix}
                  \pdv{f}{x,x}
                & \pdv{f}{x,y}
                \\\pdv{f}{y,x}
                & \pdv{f}{y,y}
            \end{vmatrix}
            =\begin{vmatrix}
                    2    & 2\,y
                \\ -2\,y & 6\,y^2-x\,2
            \end{vmatrix}
            = &\\&
            = 12\,y^2-4\,x
            + 4\,y^2
            = 16\,y^2-4\,x
            ; &\\[3ex]&
            \det H_f(4,-2)
            = 16\,(-2)^2-4*4
            = 48
            &\\&
            \therefore
            \text{Crítico mínimo Local}
            ; &\\[3ex]&
            \det H_f(2,0)
            = 16\,(0)^2-4*2
            = -8
            &\\&
            \therefore
            \text{Ponto de sela}
        &
    \end{flalign*}
\end{questionBox}

\begin{questionBox}1{ % Q13
    Considere a superfície
    \begin{BM}
        \rho
        =\left\{
            (x,y,z)\in\mathbb{R}^3
            : z=\sqrt{3}\,x
            , (x,y)\in\myrange{0,1}\times\myrange{0,2}
        \right\}
    \end{BM}
    orientada com a terceira componente do campo vetorial normal não negativa, e o campo vetorial
    \begin{BM}
        \vec{F}(x,y,z)
        = -3\left(
            x^2+\sqrt{z^2}{4}\hat{\jmath}
        \right)
    \end{BM}
    Seja \(\mathcal{L}\) o bordo de \(\sigma\) orientado positivamente de acordo com \(\sigma\). O valor do integral curvilíneo
    \begin{BM}
        \int_{\mathcal{L}}{
            - \frac{z^3}{4}\odif{x}
            + x^3\,\odif{z}
        }
    \end{BM}
} % Q13
    body
\end{questionBox}

\begin{questionBox}1{ % Q14
    Seja \(f(x,y)\) uma função contínua em \(\mathbb{R}^2\). Considere a igualdade
    \begin{BM}
        \iint_{\mathcal{R}}{
            f(x,y)\,\odif{x,y}
        }
        = \int_0^1{
            \int_{y^2}^{\sqrt{2-y^2}}{
                f(x,y)
                \,\odif{x,y}
            }
        }
    \end{BM}
    Tem se:
} % Q14
    \answer{}
    \begin{flalign*}
        &
            \begin{cases}
                x=\sqrt{2-y^2}\implies \myvert{y}=\sqrt{2-x^2}
                \\
                x=y^2\implies \sqrt{x}=\myvert{y}
                \\
                \text{Integra com}
            \end{cases}
            &\\[3ex]&
            \int_0^1{
                \int_{y^2}^{\sqrt{2-y^2}}{
                    f(x,y)
                    \,\odif{x,y}
                }
            }
            = &\\&
            = \int_0^1{
                \int_0^{\sqrt{x}}{
                    f(x,y)\,\odif{y,x}
                }
            }
            + \int_{1}^{\sqrt{2}}{
                \int_0^{\sqrt{2-x^2}}{
                    f(x,y)\,\odif{y,x}
                }
            }
        &
    \end{flalign*}
\end{questionBox}

\begin{questionBox}1{ % Q15
    Seja
    \begin{BM}
        I=\iint_{\mathcal{R}}{
            \exp\left(
                \frac{y-x}{y+x}
            \right)
            \odif{x,y}
        }
    \end{BM}
    Onde \(\mathcal{R}\) é o triangulo definido pelas retas de equações
    \begin{BM}
        x=0,y=0\text{ e }x+y=4.
    \end{BM}
    Tem se:
} % Q15
    \paragraph*{Sugestão:} Considere a mudança de variáveis \(y=x-y,v=x+y\)
\end{questionBox}

\setcounter{question}{17}

\begin{questionBox}1{ % Q18
    \begin{BM}
        \begin{aligned}
            C_1:\,&
            y=1, 
            0\leq x\leq 2
            \\
            C_2:& 
            x^2+y^2=4, 
            1\leq x\leq 2
            \land y\geq0
            \\
            C_3:& 
            y=\sqrt{3}\,x, 
            0\leq x\leq 1
        \end{aligned}
    \end{BM}
} % Q18
    % Teorema de green
    \answer{}
    \begin{flalign*}
        &
            \begin{cases}
                \phi_1(t)=(t,0),& 0\leq t\leq 2
                \\
                \phi_2(t)=(2\,\cos{t},2\,\sin{t}),& 0\leq t\leq \pi/3
                \\
                \phi_3(t)=(t,\sqrt{3}\,t)
                & 0\leq t\leq 1
                \\
                \phi_1'(t)=(1,0)
                \\
                \phi'_2(t)=(-2\,\sin{t},2\,\cos{t})
                \\
                \phi'_3(t)=(1,\sqrt{3})
                &
            \end{cases}
            &\\[3ex]&
            \oint_C{
                0\,\odif{x} 
                + x\,y\,\odif{y}
            }
            % teorema de green
            = \iint_{C_{int}}{
                y\,\odif{A}
            }
            % Separa
            = &\\&
            = 
            \iint_{C_1}{
                y\,\odif{A}
            }
            + \iint_{C_2}{
                y\,\odif{A}
            }
            + \iint_{C_3}{
                y\,\odif{A}
            }
            % \int{f \odif{s}}
            % = \int{f(phi(t))||\phi'(t)||\odif{t}}
            = &\\&
            = 
            \left(
                \begin{aligned}
                    &
                    \int_{0}^{2}{
                        0*\myVert{1,0}\,\odif{t}
                    }
                    &+\\+&
                    \int_{0}^{\pi/3}{
                        2\,\sin{t}
                        \,\myVert{(-2\sin{t},2\,\cos{t})}
                        \,\odif{t}
                    }
                    &+\\+&
                    \int_{0}^{1}{
                        \sqrt{3}\,t
                        \,\myVert{(1,\sqrt{3})}
                        \,\odif{t}
                    }
                \end{aligned}
            \right)
        &
    \end{flalign*}
\end{questionBox}

\setcounter{question}{20}

\begin{questionBox}1{ % Q21
    \begin{BM}
        0\leq z\leq 3,
        1\leq x^2+y^2,
        x^2+y^2\leq z+1
    \end{BM}
} % Q21
    \answer{}
    \begin{flalign*}
        &
            \begin{cases}
                0\leq z\leq 3
                \\
                1\leq r^2
                \\
                r^2\leq z+1
            \end{cases}
            &\\&
            \iiint_{\varepsilon}{
                \odif{V}
            }
            = \int_0^3{
                \int_0^{2\,\pi}{
                    \int_1^{\sqrt{z+1}}{
                        r\odif{r,\theta,z}
                    }
                }
            }
            = \dots
            = 2\,\pi\,(3^2/2)/2
            = \pi\,9/2
        &
    \end{flalign*}
\end{questionBox}

\begin{questionBox}1{ % Q22
    Considere o sólido \(\mathcal{D}\text{ de }\mathbb{R}^2\) definido por
    \begin{BM}
        \mathcal{D}
        = \left\{
            (x,y,z)\in\mathbb{R}^3
            : x^2\leq y\leq 1
            \land 0\leq z\leq 2
        \right\}
    \end{BM}
    \begin{BM}
        \vec{F}(x,y,z)
        = \left(
            \cos{y}+x^3,
            y+1,
            2\,x^3-z
        \right)
    \end{BM}
    O valor do fluxo \(\iint_{\sigma}{\vec{F}\cdot\vec{n}}\odif{S}\)
} % Q22
    \answer{}
    \begin{flalign*}
        &
            % \begin{cases}
            %     (r\cos\theta)^2
            %     \leq
            %     r\sin\theta
            %     \leq
            %     1 
            %     \\
            %     0<z<2
            %     \\
            %     \myvert{et J}=r
            % \end{cases}
            % &\\&
            % \begin{cases}
            %     (r\cos\theta)^2
            %     \leq
            %     r\sin\theta
            %     \leq
            %     1 
            %     \\
            %     0<z<2
            %     \\
            %     \myvert{et J}=r
            % \end{cases}
            % &\\[3ex]&
            \iint_{\sigma}{\vec{F}\cdot\vec{n}}\odif{S}
            = \iiint{div{F}\,\odif{V}}
            = \iiint{
                (3\,x^2+1-1)
                \,\odif{V}
            }
            = &\\&
            = \int_0^2{
                \int_0^1{
                    \int_{x^2}^{1}{
                        3\,x^2
                        \,\odif{y,x,z}
                    }
                }
            }
            = &\\&
            = \int_0^2{
                \int_0^1{
                    3\,x^2\,(1-x^2)
                    \,\odif{x,z}
                }
            }
            = &\\&
            = \int_0^2{
                3\,(1^3-0^3)/3-3\,(1^5-0^5)/5
                \,\odif{z}
            }
            = &\\&
            = 2(1-3/5)
            = 4/5
        &
    \end{flalign*}
\end{questionBox}

\end{document}
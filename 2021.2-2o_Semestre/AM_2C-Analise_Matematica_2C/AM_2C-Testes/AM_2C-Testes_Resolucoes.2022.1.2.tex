% !TEX root = ./AM_2C-Testes_Resolucoes-2022.1.2.tex
\providecommand\mainfilename{"./AM_2C-Testes_Resolucoes.tex"}
\providecommand \subfilename{}
\renewcommand   \subfilename{"./AM_2C-Testes_Resolucoes-2022.1.2.tex"}
\documentclass[\mainfilename]{subfiles}

% \graphicspath{{\subfix{../images/}}}
% \tikzset{external/force remake=true}



\begin{document}

\mymakesubfile{2}
[AM\,2C]
{Teste 2 2022.1 Resolução}
{Teste 2 2022.1 Resolução}



%   ,ad8888ba,                                                              88
%  d8"'    `"8b                                                           ,d88
% d8'                                                                   888888
% 88             8b,dPPYba,  88       88  8b,dPPYba,    ,adPPYba,           88
% 88      88888  88P'   "Y8  88       88  88P'    "8a  a8"     "8a          88
% Y8,        88  88          88       88  88       d8  8b       d8          88
%  Y8a.    .a88  88          "8a,   ,a88  88b,   ,a8"  "8a,   ,a8"          88
%   `"Y88888P"   88           `"YbbdP'Y8  88`YbbdP"'    `"YbbdP"'           88
%                                         88
%                                         88

\group{}

\begin{questionBox}1{ % 1 Q1
} % 1 Q1

    \paragraph{Considere o conjunto}
    \begin{BM}
        A = \left\{
            (x,y)\in\mathbb{R}^2
            :x^2+y^2\geq 1 
            \land (x-1)^2 + y^2\leq 1
        \right\}
    \end{BM}
    E seja \textit{L} a fronteira de \textit{A} percorrida no sentido direto. O integral de linha
    \begin{BM}
        \int_L{
            \left(
                x^3-y^2/2
            \right)
            \odif{x}
            + (y^5-1)
            \odif{y}
        }
    \end{BM}
    pode ser calculado, utilizando coordenadas polares, a partir do seguinte integral repetido:

    \begin{flalign*}
        &
            % ------------------------------ Meeting points ------------------------------ %
            (x,y):
            x^2+y^2
            =(x-1)^2+y^2
            \implies
            (x,y):
            \myvert{x}
            =\myvert{(x-1)}
            \implies &\\&
            \implies
            (x,y)=
            (0.5,\pm\sqrt{3}/2)
            \implies &\\&
            \implies
            (\rho,\theta) 
            =\left(
                \sqrt{(0.5)^2+(\sqrt{3}/2)^2},
                \arccos\left(
                    \frac{\sqrt{3}/2}{\rho}
                \right)
            \right)
            = (1,\pm\pi/3)
            &\\[3ex]&
            \begin{cases}
                1 
                = x^2+y^2
                = (\rho\,\cos(\theta))^2
                + (\rho\,\sin(\theta))^2
                = \rho^2
                \\
                1
                = (x-1)^2+y^2
                = \rho^2 - 2\,\rho\,\cos(\theta) + 1
                \implies
                \rho=2\,\cos(\theta)
            \end{cases}
            &\\[3ex]&
            % ------------------------------ Limite superior ----------------------------- %
            (x-1)^2+y^2
            = (\rho\,\cos(\theta)-1)^2+(\rho\,\sin(\theta))^2
            = &\\&
            = \rho^2\,\cos^2(\theta)
            -2\rho\,\cos(\theta)
            +1
            +\rho^2\,\sin^2(\theta)
            = \rho^2
            -2\rho\,\cos(\theta)
            +1
            =1
            \implies &\\&
            \implies
            \rho=2\cos\theta
            &\\[3ex]&
            % --------------------------------- Jacobiana -------------------------------- %
            \myvert{\Jacobian}
            = \begin{vmatrix}
                \pdv{x}{\rho} & \pdv{x}{\theta}
                \\
                \pdv{y}{\rho} & \pdv{y}{\theta}
            \end{vmatrix}
            = \begin{vmatrix}
                \cos\theta & -\rho\sin\theta
                \\
                \sin\theta & \rho\cos\theta
            \end{vmatrix}
            = \rho\cos^2(\theta)
            + \rho\sin^2(\theta)
            = \rho
            &\\[3ex]&
            % ----------------------------------- Start ---------------------------------- %
            \int_L{
                \left(
                    x^3-y^2/2
                \right)
                \odif{x}
                + (y^5-1)
                \odif{y}
            }
            \overset{Riemmann-Green}{=} &\\&
            =
            \iint_A{
                \left(
                    \pdv{\psi(x,y)}{x}
                    -\pdv{\varphi(x,y)}{y}
                \right)
                \odif{x,y}
            }
            = &\\&
            =
            \iint_A{
                \left(
                    \pdv{}{x}\left(
                        y^5-1
                    \right)
                    -\pdv{}{y}\left(
                        x^3-y^2/2
                    \right)
                \right)
                \odif{x,y}
            }
            = &\\&
            = \iint_A{
                y
                \odif{x,y}
            }
            = &\\&
            = \int_{-\pi/3}^{\pi/3}{
                \int_{1}^{2\,\cos\theta}{
                    \left(
                        \rho
                        \,\sin{\theta}
                    \right)
                    (\rho)
                    \,\odif{\rho}
                }
                \,\odif{\theta}
            }
        &
    \end{flalign*}
\end{questionBox}

\begin{questionBox}1{ % 1 Q2
    Utilizando coordenadas polares, o volume de um domínio fechado \(\dominio\), limitado superiormente pela semisuperfície esférica \(z=\sqrt{1-x^2-y^2}\), inferiormente pela superfície cónica \(z=1-\sqrt{x^2+y^2}\) e compreendida entre os planos \(y=0\) e \(y=x\) pode ser calculado, utilizando coordenadas polares, a partir do seguinte integral repetido:
} % 1 Q2
    \begin{flalign*}
        &
            z
            =\sqrt{1-x^2-y^2}
            =\sqrt{1-(\rho\cos\theta)^2-{\rho\sin\theta}^2}
            =\sqrt{1-\rho^2}
            &\\[3ex]&
            z
            = 1- \sqrt{x^2+y^2}
            = 1- \sqrt{(\rho\cos\theta)^2+(\rho\sin\theta)^2}
            = 1 - \rho
            &\\[3ex]&
            z
            =\sqrt{1-x^2-y^2}
            =\sqrt{1-(1-z)^2}
            =\sqrt{2z-z^2}
            \implies
            z^2=2\,z-z^2
            \implies z=1
            &\\&
            z-1=0=-\sqrt{x^2+y^2}
            \implies
            (x,y)=(0,0)
            \implies &\\&
            \implies
            (0,0,1)\to()
            \implies &\\[3ex]&
            \implies
            \int_{0}^{\pi/4}{
                \int_{0}^{1}
                (
                    \sqrt{1-\rho^2}
                    -(1-\rho)
                )
                \,\rho
                \,\odif{\rho}
            }\,\odif{\theta}
            = 
            \int_{0}^{\pi/4}{
                \int_{0}^{1}
                (
                    \sqrt{1-\rho^2}
                    -1+\rho
                )
                \,\rho
                \,\odif{\rho}
            }\,\odif{\theta}
        &
    \end{flalign*}
\end{questionBox}

\begin{questionBox}1{ % 1 Q3
    O integral repetido
} % 1 Q3
    \begin{BM}
        \int_{-2}^{0}{
            \int_{0}^{x^2}{
                x^2\,\odif{y}
            }
            \odif{x}
        }
        + \int_{0}^{2}{
            \int_{0}^{x^2}{
                x^2\,\odif{y}
            }\odif{x}
        }
    \end{BM}

    Ultilizando a ordem de integração inversa da apresentada, pode ser calculado a partir de:

    \begin{flalign*}
        &
            \left\{
                \begin{aligned}
                    y_1: & 0\to x^2
                    \\
                    x_1: & -2\to0
                    \\
                    y_2: & 0\to x^2
                    \\
                    x_2: & 0\to2
                \end{aligned}
            \right\}
            \implies
            \left\{
                \begin{aligned}
                    x_1: & -2\to -\sqrt{y}
                    \\
                    y_1: & 0\to(-2)^2
                    \\
                    x_2: & \sqrt{y}\to 2
                    \\
                    y_2: & 0\to2^2
                \end{aligned}
            \right\}
            &\\[3ex]&
            \left\{
                (x,y)\in\mathbb{R}^2:
                \left(
                    \begin{aligned}
                        &
                            0\leq y\leq4
                        \ldiv{}
                            -2\leq x \leq -\sqrt{y}
                        &
                    \end{aligned}
                \right)
            \right\}
            \cup 
            \left\{
                (x,y)\in\mathbb{R}^2:
                \left(
                    \begin{aligned}
                        &
                            0\leq y\leq4
                        \ldiv{}
                            \sqrt{y}\leq x \leq 2
                        &
                    \end{aligned}
                \right)
            \right\}
            &\\[3ex]&
            \therefore
            \int_{0}^{4}{
                \int_{-2}^{-\sqrt{y}}{
                    x^2\odif{x}
                }\odif{y}
            }
            +\int_{0}^{4}{
                \int_{\sqrt{y}}^{2}{
                    x^2\odif{x}
                }\odif{y}
            }
        &
    \end{flalign*}
\end{questionBox}

\end{document}
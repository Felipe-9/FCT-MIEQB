% !TEX root = ./AM_2C-Testes_Resolucoes.2022.1.1.tex
\providecommand\mainfilename{"./AM_2C-Testes_Resolucoes.tex"}
\providecommand \subfilename{}
\renewcommand   \subfilename{"./AM_2C-Testes_Resolucoes.2022.1.1.tex"}
\documentclass[\mainfilename]{subfiles}



\begin{document}

\graphicspath{{\subfix{.build/figures/AM_2C-Testes_Resolucoes.2022.1.1}}}
% \tikzsetexternalprefix{./.build/figures/AM_2C-Testes_Resolucoes.2022.1.1/graphics/}

\mymakesubfile{1}
[AM 2C]
{Teste Resolução 2022.1}
{Teste Resolução 2022.1}

\group{}

\begin{questionBox}1{ % 1 Q1
    A parábola com foco em (2,0) e recta directriz \(x=-2\), e a elipse com centro em (0,1), foco em \((\sqrt{7},1)\) e vértice em (4,1) admitem, respectivamente, as seguintes equações:
} % 1 Q1
    \answer{}
    % --------------------------------- Parabola --------------------------------- %
    \subsubquestion{Parabola}
    \begin{flalign*}
        &
            P 
            = \left\{
                (x,y)\in\mathbb{R}^2:
                \left(
                    \begin{aligned}
                        &
                            (y-y') = (x-x')^2/(4\,a)
                        \ldiv{}
                            f' = (x'+a,y') = (2,0)
                        \ldiv{}
                            L \subset\mathbb{R}^2: x = x'-a = -2
                        \ldiv{}
                            y' = 0
                        \ldiv{}
                            x'+a+x'-a = 2\,x' = 2-2 = 0 
                        \ldiv{}
                            a = 2+x' = 2
                        &
                    \end{aligned}
                \right)
            \right\}
            = &\\&
            = \left\{
                (x,y)\in\mathbb{R}^2:y = x^2/8
            \right\}
        &
    \end{flalign*}

    \subsubquestion{Elipse}
    \begin{flalign*}
        &
            E
            = \left\{
                (x,y)\in\mathbb{R}^2:
                \left(
                    \begin{aligned}
                        &
                            \frac{(x-x')^2}{r_1^2}
                            + \frac{(y-y')^2}{r_2^2}
                            = 1
                        \ldiv{}
                            (x',y') = (0,1)
                        \ldiv{}
                            f_2 = (x'+c,y') = (c,1) = (\sqrt{7},1)
                        \ldiv{}
                            P_x = (x'+r_1,y') = (r_1,y') = (4,1)
                        \ldiv{}
                            r_2^2
                            = r_1^2-c^2
                            = 16-7
                            = 9
                        &
                    \end{aligned}
                \right)
            \right\}
            = &\\&
            = \left\{
                (x,y)\in\mathbb{R}^2:
                \frac{x^2}{16}
                + \frac{(y-1)^2}{9}
                = 9\,x^2+16(y-1)^2
                = 1*9*16
                = 144
            \right\}
        &
    \end{flalign*}

\end{questionBox}

\begin{questionBox}1{ % 1 Q2
    Considere-se as seguintes funções reais de variável real:
} % 1 Q2
    \begin{BM}
        \textcolor{red\Light}{f}(x,y) = \frac{y\,x^3}{y+x^3};
        % \\
        \qquad
        \textcolor{green\Light}{g}(x,y) = \frac{\sin(x^2+y^2)}{e^{(x^2+y^2)^{-1}}};
        \\
        \textcolor{blue\Light}{h}(x,y) = \frac{2\,x+y}{x-y}
    \end{BM}

    Relativamente ao limite de cada uma delas no ponto (0,0) tem-se que:

    \begin{alternativelist}
        \item \textit{\textcolor{red\Light}{f}} e \textit{\textcolor{green\Light}{g}} têm limite zero, e \textit{\textcolor{blue\Light}{h}} não tem limite.
        \item \textit{\textcolor{red\Light}{f}} e \textit{\textcolor{green\Light}{g}} têm limite zero, e \textit{\textcolor{blue\Light}{h}} tem limite 2.
        \item \textit{\textcolor{red\Light}{f}} e \textit{\textcolor{green\Light}{g}} têm limite zero, e \textit{\textcolor{blue\Light}{h}} tem limite -1.
        \item \textit{\textcolor{red\Light}{f}} têm limite zero, \textit{\textcolor{blue\Light}{h}} tem limite 2 e o limite de \textit{\textcolor{green\Light}{g}} é infinito.
        \item \textit{\textcolor{red\Light}{f}} e \textit{\textcolor{blue\Light}{h}} não têm limite e o limite de \textit{\textcolor{green\Light}{g}} é infinito.
        \item \textit{\textcolor{red\Light}{f}} e \textit{\textcolor{blue\Light}{h}} não têm limite e o limite de \textit{\textcolor{green\Light}{g}} é zero.
    \end{alternativelist}

    \subsubquestion{\textcolor{red\Light}{f}}
    \begin{flalign*}
        &
            \lim_{(x,y)\to(0,0)}
            {f(x,y)}
            = \lim_{(x,y)\to(0,0)}{
                \frac{y\,x^3}{y+x^3}
            }
            = \lim_{x\to0}{f(x,0)}
            = \lim_{x\to0}{
                \frac{0}{x^3}
            }
            = 0
            = &\\&
            = \lim_{x\to0}{f(x,x^6-x^3)}
            = \lim_{x\to0}{
                \frac{
                    (x^6-x^3)\,x^3
                }{
                    (x^6-x^3)+x^3
                }
            }
            = \lim_{x\to0}{x^3-1}
            = -1
        &
    \end{flalign*}

    \subsubquestion{\textcolor{green\Light}{g}}
    \begin{flalign*}
        &
            -1 \leq \sin(x^2+y^2) \leq 1
            \land
            \lim_{(x^2+y^2)\to0}{e^{(x^2+y^2)^{-1}}} = +\infty
            % &\\&
            \therefore
            \lim_{(x,y)\to(0,0)}{g(x,y)} = 0
        &
    \end{flalign*}

    \subsubquestion{\textcolor{blue\Light}{h}}
    \begin{flalign*}
        &
            \lim_{x\to0}{\left(
                \lim_{y\to0}{
                    h(x,y)
                }
            \right)
            }
            =\lim_{x\to0}{\left(
                \lim_{y\to0}{
                    \frac{2\,x+y}{x-y}
                }
            \right)
            }
            = 2
            \neq
            \lim_{y\to0}{\left(
                \lim_{x\to0}{
                    \frac{2\,x+y}{x-y}
                }
            \right)
            }
            = -1
        &
    \end{flalign*}

\end{questionBox}

\begin{questionBox}1{ % 1 Q3
    Considere a função \(f:\mathbb{R}^2\to\mathbb{R}^2\) definida por
} % 1 Q3
    \begin{BM}
        f(x,y) 
        = (3\,x-y^2,x^3-3\,y^2)
        f(u,v) 
    \end{BM}

    Que é invertivel numa vizinhança do ponto (1,1). Tem-se que:

    \begin{alternativelist}
        \begin{multicols}{2}
            \item \(\textcolor{blue\Light}{\pdv{x}{u}}(2,-2) =   \textcolor{green\Light}{\pdv{x}{v}}(2,-2)\)
            \item \(\textcolor{blue\Light}{\pdv{x}{u}}(2,-2) = - \textcolor{green\Light}{\pdv{x}{v}}(2,-2)\)
            \item \(\textcolor{ red\Light}{\pdv{y}{u}}(2,-2) = - \textcolor{orange\Light}{\pdv{y}{v}}(2,-2)\)
            \item \(\textcolor{blue\Light}{\pdv{x}{u}}(2,-2) =   \textcolor{ red\Light}{\pdv{y}{u}}(2,-2)\)
            \item \(\textcolor{ red\Light}{\pdv{y}{u}}(2,-2) =   \textcolor{orange\Light}{\pdv{y}{v}}(2,-2)\)
            \item \(\textcolor{green\Light}{\pdv{x}{v}}(2,-2) =   \textcolor{orange\Light}{\pdv{y}{v}}(2,-2)\)
        \end{multicols}
    \end{alternativelist}

    \begin{flalign*}
        &
            f(1,1)
            = (3*1-1^2,1^3-3*1^2)
            = (2,-2)
            &\\&
            \begin{bmatrix}
                \textcolor{blue\Light}   {\pdv{x}{u}} 
                & \textcolor{green\Light}{\pdv{x}{v}}
                \\
                \textcolor{red\Light}     {\pdv{y}{u}} 
                & \textcolor{orange\Light}{\pdv{y}{v}}
            \end{bmatrix}_{(2,2)}
            = \Jacobian_{f^{-1}}{(2,-2)}
            = \Jacobian_{f^{-1}}{(f(1,1))}
            = \Jacobian_{f}     {(f(1,1))^{-1}}
            = \begin{bmatrix}
                \pdv{u}{x} & \pdv{u}{y}
                \\
                \pdv{v}{x} & \pdv{v}{y}
            \end{bmatrix}^{-1}_{(1,1)}
            = &\\&
            = \begin{bmatrix}
                3 & -2\,y
                \\
                3\,x^2 & -6\,y
            \end{bmatrix}^{-1}_{(1,1)}
            % = &\\&
            = \begin{bmatrix}
                3 & -2
                \\
                3 & -6
            \end{bmatrix}^{-1}
            = \begin{vmatrix}
                3 & -2
                \\
                3 & -6
            \end{vmatrix}^{-1}
            \adj\begin{bmatrix}
                3 & -2
                \\
                3 & -6
            \end{bmatrix}
            = -12^{-1}
            \begin{bmatrix}
                -6 & 2
                \\
                -3 & 3
            \end{bmatrix}
            = &\\&
            = \begin{bmatrix}
                \textcolor{blue\Light}   { 1/2}
                & \textcolor{green\Light}{-1/6}
                \\
                \textcolor{red\Light}     { 1/4}
                & \textcolor{orange\Light}{-1/4}
            \end{bmatrix}
        &
    \end{flalign*}

\end{questionBox}

\begin{questionBox}1{ % 1 Q4
    A equação define \textit{z} como função de \textit{x} e de \textit{y} numa vizinhança do ponto \((x_0,y_0,z_0)=(1,-1,0)\). Então:
} % 1 Q4
    \begin{BM}
        x^4+y^4+(x+1)\,e^z + 8\,x\,\sin(z)-4=0
    \end{BM}

    \begin{alternativelist}
        \begin{multicols}{2}
            \item \(\pdv{z}{x}(1,-1)=+1/2, \pdv{z}{y}(1,-1)=+2/5\)
            \item \(\pdv{z}{x}(1,-1)=-1/2, \pdv{z}{y}(1,-1)=-2/5\)
            \item \(\pdv{z}{x}(1,-1)=-1/2, \pdv{z}{y}(1,-1)=+2/5\)
            \item \(\pdv{z}{x}(1,-1)=+1/2, \pdv{z}{y}(1,-1)=+2\)
            \item \(\pdv{z}{x}(1,-1)=-1/2, \pdv{z}{y}(1,-1)=+2\)
            \item \(\pdv{z}{x}(1,-1)=-1/2, \pdv{z}{y}(1,-1)=-2\)
        \end{multicols}
    \end{alternativelist}

    \subsubquestion{\(\pdv{z}{x}\)}
    \begin{flalign*}
        &
            \pdv{z}{x}(1,-1)
            = -\frac{
                \pdv{f}{x}(1,-1,0)
            }{
                \pdv{f}{z}(1,-1,0)
            }
            = -\frac{
                \left(
                    4\,x^3+e^z + 8\,\sin(z)
                \right)(1,-1,0)
            }{
                \left(
                    (x+1)\,e^z + 8\,x\,\cos(z)
                \right)(1,-1,0)
            }
            = -\frac{
                4*(1)^3+e^0 + 8*\sin(0)
            }{
                (1+1)*e^0 + 8*1*\cos(0)
            }
            = -\frac{1}{2}
        &
    \end{flalign*}
    % \begin{flalign*}
    %     &
    %         0
    %         = \pdv{}{x}\left(
    %             x^4+y^4+(x+1)\,e^z + 8\,x\,\sin(z)-4
    %         \right)\left(
    %             1,-1,0
    %         \right)
    %         = &\\&
    %         = 
    %         \left(
    %             4\,x^3
    %             +\left(
    %                 (x+1)\,e^z\,\pdv{z}{x}
    %                 +
    %                 e^z
    %             \right)
    %             +8\left(
    %                 \sin(z)
    %                 + x\,\cos(z)\,\pdv{z}{x}
    %             \right)
    %         \right)\left(
    %             1,-1,0
    %         \right)
    %         = &\\&
    %         = 
    %         4
    %         + 2\,\pdv{z}{x} 
    %         + 1
    %         + 8\,\pdv{z}{x}
    %         \implies
    %         \pdv{z}{x} 
    %         = -1/2
    %     &
    % \end{flalign*}

    \subsubquestion{\(\pdv{z}{y}\)}
    \begin{flalign*}
        &
            \pdv{z}{y}(1,-1)
            = -\frac{
                \pdv{f}{y}(1,-1,0)
            }{
                \pdv{f}{z}(1,-1,0)
            }
            = -\frac{
                \left(
                    4\,y^3
                \right)(1,-1,0)
            }{
                \left(
                    (x+1)\,e^z + 8\,x\,\cos(z)
                \right)(1,-1,0)
            }
            = &\\&
            = -\frac{
                4\,(-1)^3
            }{
                (1+1)*e^{(0)} + 8*1*\cos(0)
            }
            = \frac{2}{5}
        &
    \end{flalign*}
    % \begin{flalign*}
    %     &
    %         0
    %         = \pdv{}{y}\left(
    %             x^4+y^4+(x+1)\,e^z + 8\,x\,\sin(z)-4
    %         \right)\left(
    %             1,-1,0
    %         \right)
    %         = &\\&
    %         =\left(
    %             +4\,y^3
    %             +(x+1)\,e^z\,\pdv{z}{y}
    %             + 8\,x\,\cos(z)\,\pdv{z}{y}
    %         \right)\left(
    %             1,-1,0
    %         \right)
    %         = &\\&
    %         = 4*(-1)^3
    %         + (1+1)*e^0*\pdv{z}{y}
    %         + 8*1*\cos(0)*\pdv{z}{y}
    %         % = &\\&
    %         = -4
    %         + 10\,\pdv{z}{y}
    %         \implies &\\&
    %         \implies
    %         \pdv{z}{y}
    %         = 4/10=2/5
    %     &
    % \end{flalign*}

\end{questionBox}

\begin{questionBox}1{ % 1 Q5
    Sejam \textit{g} e \textit{h} duas funções reais e de classe \(C^1\) em \(\mathbb{R}\). Considere a função \(f:\mathbb{R}^3\to\mathbb{R}^2\) definida por
} % 1 Q5
    \begin{BM}
        f(x,y,z) = \left(
            x\,g\,\left(
                \frac{x}{z}
            \right)
            ,
            h\,(x^2\,y)
        \right)
    \end{BM}

    Determine a matriz jacobiana de \textit{f} no ponto (-1,0,-1) em função das derivadas de \textit{g} e de \textit{h}.

    \begin{flalign*}
        &
            J_f(-1,0,-1)
            = \begin{bmatrix}
                    \pdv{f_1}{x} 
                &   \pdv{f_1}{y}
                &   \pdv{f_1}{z}
                \\
                    \pdv{f_2}{x} 
                &   \pdv{f_2}{y}
                &   \pdv{f_2}{z}
            \end{bmatrix}_{(-1,0,-1)}
            = &\\&
            = \begin{bmatrix}
                    g\left(
                        \frac{x}{z}
                    \right)
                    +x\,g'\left(
                        \frac{x}{z}
                    \right)\,z^{-1}
                &   0
                &   -g'\left(
                        \frac{x}{z}
                    \right)\,z^{-2}
                \\
                    h'(x^2\,y)\,y\,2\,x 
                &   h'(x^2\,y)\,x^2
                &   0
            \end{bmatrix}_{(-1,0,-1)}
            = &\\&
            = \begin{bmatrix}
                    g\left(
                        \frac{-1}{-1}
                    \right)
                    +(-1)\,g'\left(
                        \frac{-1}{-1}
                    \right)(-1)^{-1}
                &   0
                &   -g'\left(
                        \frac{-1}{-1}
                    \right)\,(-1)^{-2}
                \\
                    h'((-1)^2\,0)\,(0)\,2\,(-1) 
                &   h'((-1)^2\,0)\,(-1)^2
                &   0
            \end{bmatrix}
            = &\\&
            = \begin{bmatrix}
                    g(1)+g'(1)
                &   0
                &   -g'(1)
                \\
                    0
                &   h'(0)
                &   0
            \end{bmatrix}
        &
    \end{flalign*}

\end{questionBox}

\group{}

\begin{questionBox}1{ % 2 Q1
    Considere a função real \textit{g}, de duas variáveis reais, definida por
} % 2 Q1
    \begin{BM}
        g(x,y)
        = \begin{cases}
            \frac{x^3-3\,x\,y^2}{x^2+y^2}
            \quad& (x,y)\neq(0,0)
            \\
            0 \quad& (x,y)=(0,0)
        \end{cases}
    \end{BM}
\end{questionBox}

\begin{questionBox}2{ % 2 Q1.1
    Estude, por definição, a continuidade de \textit{g(x,y)} em (0,0).
} % 2 Q1.1
    \begin{flalign*}
        &
            \forall\,\delta>0\,
            \exists\,\varepsilon>0:
            \left(
                \left(
                    \forall (x,y)\neq(0,0)
                    \land
                    \myVert{\sqrt{x^2+y^2}}<\varepsilon
                \right)
                \implies
                \myvert{
                    g(x,y)-0
                }<\delta
            \right)
            \implies &\\&
            \implies
            \myvert{
                \frac{
                    x^3-3\,x\,y^2
                }{
                    x^2+y^2
                }
            }
            \leq 
            \myvert{x}
            \frac{
                (x^2+3\,y^2)
            }{
                x^2+y^2
            }
            \leq 
            \myvert{x}
            \frac{
                (3\,x^2+3\,y^2)
            }{
                x^2+y^2
            }
            \leq &\\&
            \leq 
            3\,\myvert{x}
            \leq 
            3\,\sqrt{x^2+y^2}
            \leq
            3\,\varepsilon=\delta
            \implies \varepsilon=\delta/3
        &
    \end{flalign*}
\end{questionBox}

\begin{questionBox}2{ % 2 Q1.2
    Determine \(\pdv{g}{x}(0,0) e \pdv{g}{y}(0,0)\)
} % 2 Q1.2
    \subsubquestion{}
    \begin{flalign*}
        &
            \pdv{g}{x}(0,0)
            = \lim_{h\to0}{
                \frac{g(h,0)-g(0,0)}{h}
            }
            = \lim_{h\to0}{
                \frac{
                    \frac{h^3-3*h*0^2}{h^2+0^2}
                }{h}
            }
            = \lim_{h\to0}{
                1
            }
            = 1
        &
    \end{flalign*}

    \subsubquestion{}
    \begin{flalign*}
        &
            \pdv{g}{y}(0,0)
            = \lim_{h\to0}{
                \frac{g(0,h)-g(0,0)}{h}
            }
            = \lim_{h\to0}{
                \frac{
                    \frac{0^3-3*0*h^2}{0^2+h^2}
                }{h}
            }
            = \lim_{h\to0}{
                0
            }
            = 0
        &
    \end{flalign*}
\end{questionBox}

\begin{questionBox}2{ % 2 Q1.3
    Estude a diferenciabilidade de \textit{g} no ponto (0,0).
} % 2 Q1.3
    \begin{flalign*}
        &
            g(a,b)-g(0,0)
            = \frac{a^3-3\,a\,b^2}{a^2+b^2}
            = 
            \pdv{g}{x}(0,0)\,a
            + \pdv{g}{y}(0,0)\,b
            + \varepsilon(a,b)\,\sqrt{a^2+b^2}
            = &\\&
            = 1\,a
            + \varepsilon(a,b)\,\sqrt{a^2+b^2}
            \implies
            \varepsilon(a,b)
            = 
            \frac{
                -4\,a\,b^2
            }{
                (a^2+b^2)^{3/2}
            }
            \implies &\\&
            \implies
            \lim_{a\to0^+}{\varepsilon(a,a)}
            = \lim_{a\to0^+}{
                \frac{
                    -4*a*a^2
                }{
                    (a^2+a^2)^{3/2}
                }
            }
            = \lim_{a\to0^+}{
                -2^{1/2}
            }
            = -\sqrt{2} \neq 0
        &
    \end{flalign*}
\end{questionBox}

\begin{questionBox}1{ % 2 Q2
    Considere a função real \textit{f} de duas variáveis reais, definida por
} % 2 Q2
    \begin{BM}
        f(x,y)
        = \frac{
            \log(x^2+y^2-1)\,\log(x^2-y^2)
        }{
            \sqrt{4-x^2-y^2}
        }
    \end{BM}

    Indique o seu domínio \(\dominio\) e esboce-o. Caracterize \(\interior\dominio\) usando coordenadas polares. Diga, justificando, se \(\dominio\) é um conjunto aberto ou fechado.

    \begin{flalign*}
        &
            \dominio{(f)}
            = \left\{
                (x,y)\in\mathbb{R}^2:
                \left(
                    \begin{aligned}
                        &
                            4-x^2-y^2\geq0
                        \ldiv{}
                            \sqrt{4-x^2-y^2}\neq0
                        \ldiv{}
                            x^2+y^2-1>0
                        \ldiv{}
                            x^2-y^2>0
                        &
                    \end{aligned}
                \right)
            \right\}
            = &\\&
            = \left\{
                (x,y)\in\mathbb{R}^2:
                1 < x^2+y^2 < 4
                \land
                \myvert{x}>\myvert{y}
            \right\}
        &
    \end{flalign*}

    \begin{center}
        \includegraphics[width=.6\textwidth]{Screenshot 2023-01-10 at 22.07.02.png}
    \end{center}

    \begin{flalign*}
        &
            \interior{(\dominio)} = \dominio
            = \left\{
                (\rho,\theta)\in
                \myrange*{1,4}\times
                \myrange*{-\pi/4,\pi/4}
            \right\}
            \cup \left\{
                (\rho,\theta)\in
                \myrange*{1,4}\times
                \myrange*{3\pi/4,5\pi/4}
            \right\}
        &
    \end{flalign*}

\end{questionBox}

\group{}

\begin{questionBox}1{ % 3 Q1
    Considere a função:
} % 3 Q1
    \begin{BM}[align*]
        f:\dominio\subset\mathbb{R}^2 & \to\mathbb{R}^2
        \\
        (x,y) & \to x^2+x\,y+y^2
    \end{BM}
\end{questionBox}

\begin{questionBox}2{ % 3 Q1.1
    Calcule os extremos locais de \(f(x,y)\) quando \(\dominio=\mathbb{R}^2\).
} % 3 Q1.1
    \begin{flalign*}
        &
            \left\{
                (x,y)\in\mathbb{R}^2:
                \left\{
                    \begin{aligned}
                        \pdv{f}{x}=2\,x+y=0
                        \\
                        \pdv{f}{y}=x+2\,y=0
                    \end{aligned}
                \right\}
            \right\}
            = &\\&
            = \left\{
                (x,y)\in\mathbb{R}^2:
                \left\{
                    \begin{aligned}
                        -3\,y=0
                        \\
                        x=-2\,y
                    \end{aligned}
                \right\}
            \right\}
            = \left\{(0,0)\right\}
            &\\[3ex]&
            \det\Hessiana(f(x,y))
            = \det\begin{bmatrix}
                \pdv[order=2]{f}{x}
                & \pdv{f}{x,y}
                \\
                \pdv[order=2]{f}{y}
                & \pdv{f}{y,x}
            \end{bmatrix}_{(0,0)}
            = 
            \det\begin{bmatrix}
                2 & 1
                \\
                1 & 2
            \end{bmatrix}_{(0,0)}
            = 4-1 = 3
            &\\[3ex]&
            \therefore (0,0)\text{ é um mínimo local}
        &
    \end{flalign*}
\end{questionBox}

\begin{questionBox}2{ % 3 Q1.2
    No conjunto \(D={(x,y)\in\mathbb{R}^2:x^2+y^2=1}\) a função \(f(x,y)\) admite mínimo e máximo absolutos. Escreva a função lagrangiana associada a este problema e determine os referidos extremos.
} % 3 Q1.2
    \begin{flalign*}
        &
            \Lagrangiana
            = f + \lambda\,g
            = x^2+x\,y+y^2
            + \lambda(x^2+y^2-1)
            \implies &\\&
            \implies
            \left\{
                \begin{aligned}
                    \pdv{\Lagrangiana}{x} & = 0
                    \\
                    \pdv{\Lagrangiana}{y} & = 0
                    \\
                    \pdv{\Lagrangiana}{\lambda} & = 0
                \end{aligned}
            \right\}
            = \left\{
                \begin{aligned}
                    2\,x+y+\lambda(2\,x) & = 0
                    \\
                    x+2\,y+\lambda(2\,y) & = 0
                    \\
                    x^2+y^2-1 & = 0
                \end{aligned}
            \right\}
            = \left\{
                \begin{aligned}
                    2\,x(\lambda+1)+y & = 0
                    \\
                    2(\lambda+1)(x-y)
                    & = x-y
                    \\
                    x^2+y^2 & = 1
                \end{aligned}
            \right\}
            = &\\&
            = \left\{
                \begin{aligned}
                    \left(
                        \begin{aligned}
                            &
                                (
                                    \lambda = -1/2 
                                    \land 
                                    x=y
                                )
                            \ldiv[\lor]{}
                                (
                                    x = -y 
                                    \land 
                                    \lambda = -3/2
                                )
                            &
                        \end{aligned}
                    \right)
                    \\
                    x^2+y^2 = 1
                \end{aligned}
            \right\}
            \implies 
            C_f = \left\{
                \begin{aligned}
                        &\textcolor{blue\Light}  {(+2^{-1/2},+2^{-1/2})},
                    \\  &\textcolor{green\Light} {(+2^{-1/2},-2^{-1/2})},
                    \\  &\textcolor{orange\Light}{(-2^{-1/2},+2^{-1/2})},
                    \\  &\textcolor{red\Light}   {(-2^{-1/2},-2^{-1/2})}
                \end{aligned}
            \right\}
            \implies &\\&
            \implies
            \begin{cases}
                    f\textcolor{blue\Light}  {(+2^{-1/2},+2^{-1/2})}=3/2
                \\  f\textcolor{green\Light} {(+2^{-1/2},-2^{-1/2})}=1/2
                \\  f\textcolor{orange\Light}{(-2^{-1/2},+2^{-1/2})}=1/2
                \\  f\textcolor{red\Light}   {(-2^{-1/2},-2^{-1/2})}=3/2
            \end{cases}
            &\\&
            \therefore
            \begin{cases}
                \text{Minimizantes:}\{
                    \textcolor{green\Light} {(+2^{-1/2},-2^{-1/2})},
                    \textcolor{orange\Light}{(-2^{-1/2},+2^{-1/2})}
                \}
                \\
                \text{Maximizantes:}\{
                    \textcolor{blue\Light}{(+2^{-1/2},+2^{-1/2})},
                    \textcolor{red\Light} {(-2^{-1/2},-2^{-1/2})}
                \}
            \end{cases}
        &
    \end{flalign*}
\end{questionBox}

\group{}

\begin{questionBox}1{ % 4 Q1
    Determine os pontos da superfície de equação \(x^2+2\,y^2-3\,z^2=1\) nos quais o plano tangente é paralelo ao plano de equação \(3\,x-2\,y+3\,z=1\)
} % 1 Q4
    \begin{flalign*}
        &
            % Plano paralelo
            ( 2\,x_0)(x-x_0)
            + ( 4\,y_0)(y-y_0)
            + (-6\,z_0)(z-z_0)
            = &\\&
            = 
            \begin{bmatrix}
                x\\y\\z
            \end{bmatrix}^T
            \begin{bmatrix}
                -2\,x_0
                \\ 4\,y_0
                \\ -6\,z_0
            \end{bmatrix}
            -2\,x_0^2
            -4\,y_0^2
            +6\,z_0^2
            = 0
            &\\[3ex]&
            3\,x-2\,y+3\,z-1
            = \begin{bmatrix}
                x\\y\\z
            \end{bmatrix}^T
            \begin{bmatrix}
                3\\-2\\3
            \end{bmatrix}
            -1
            =0
            \implies &\\[3ex]&
            \implies
            % Garantir que vetores são paralelos
            \begin{bmatrix}
                -2\,x_0
                \\ 4\,y_0
                \\-6\,z_0
            \end{bmatrix}
            = \alpha
            \begin{bmatrix}
                3\\-2\\3
            \end{bmatrix}
            \implies
            \begin{bmatrix}
                x_0
                \\ y_0
                \\ z_0
            \end{bmatrix}
            =
            \begin{bmatrix}
                3\,\alpha/2
                \\-\alpha/2
                \\-\alpha/2
            \end{bmatrix}
            \implies &\\&
            \implies
            (3\,\alpha/2)^2
            +2\,(-\alpha/2)^2
            -3\,(-\alpha/2)^2
            -1
            = &\\&
            = 9\,\alpha^2
            +2\,\alpha^2
            -3\,\alpha^2
            -4
            = 8\,\alpha^2
            -4
            =0
            \implies &\\&
            \implies
            \alpha = \pm\sqrt{1/2}
            &\\[3ex]&
            \therefore
            \left\{
                \begin{aligned}
                    &(+3/2^{3/2},-2^{-3/2},-2^{-3/2}),
                    \\
                    &(-3/2^{3/2},+2^{-3/2},+2^{-3/2})
                \end{aligned}
            \right\}
        &
    \end{flalign*}
\end{questionBox}

\begin{minipage}{1\textwidth}
    \ 
    \vspace{10cm}
\end{minipage}

\end{document}
% !TEX root = ./AM_2C_-_Teste_2021.1_2_Resolução.tex
% !TEX root = ./AM_2C_-_Testes_Anteriores_Resolucoes.tex
\providecommand\mainfilename{"./AM_2C_-_Testes_Anteriores_Resolucoes.tex"}
\providecommand \subfilename{}
\renewcommand   \subfilename{"./AM_2C_-_Teste_2021.1_2_Resolução.tex"}
\documentclass[\mainfilename]{subfiles}

% \graphicspath{{\subfix{../images/}}}
\tikzset{external/force remake = true} % remake all

\begin{document}

\mymakesubfile{2}
{AM\,2C -- Teste 2 Repetição 2021.1}
{AM\,2C -- Teste 2 Repetição 2021.1}

\group{}
\begin{questionBox}1m{}

    O integral repetido
    \begin{BM}
        \int_0^1 \left(
            \int_{\sqrt{1-x^2}}^{\sqrt{9-9\,x^2}}
            \cos(x\,y)
            \odif{y}
        \right)
        \odif{x}
    \end{BM}

    Utilizando a ordem de integração inversa da apresentdada, pode ser calculada a partir de:

    \begin{flalign*}
        &
            \left.
                \begin{aligned}
                    y = \sqrt{9-9\,x^2}
                    \implies 
                    \left(
                        \frac{y}{3}
                    \right)^2
                    + x^2 = 1
                 \\ y = \sqrt{1-x^2}
                    \implies
                    y^2+x^2 = 1
                 \\ x=1
                 \\ x=0
                \end{aligned}
            \right\}
            \implies &\\&
            \implies
            \int_0^1
            \left(
                \int_{\sqrt{1-y^2}}^{1-y^2/9}
                \cos(x\,y)
                \odif{x}
            \right)
            \odif{y}
            + \int_1^3
            \left(
                \int_0^{\sqrt{1-y^2/9}}
                \cos(x,y)
                \odif{x}
            \right)
            \odif{y}
        &
    \end{flalign*}

    % \begin{center}
    %     %\pgfplotsset{height=7cm, width= .6\textwidth}
    %     \begin{tikzpicture}
    %     \begin{axis}
    %         [
    %             % xmajorgrids = true,
    %             % legend pos  = north west
    %         ]
    %         % Legends
    %         % \addlegendimage{empty legend}
    %         % \addlegendentry[Red]{\( x \)}
            
    %         % % Plot from csv file
    %         % \addplot[smooth, thick, mark=*] % mesh for colormap
    %         % table[
    %         %     col sep=comma,  % csv: comma,
    %         %     header=true,
    %         %     x index=3,      % x column on file
    %         %     y index=4,      % y column on file
    %         %     point meta=x,   % value to colormap
    %         % ]{  file.csv };
            
    %         % Plot from equation
    %         \addplot[
    %             smooth,
    %             thick,
    %             red\Light,
    %             % domain  = -2:2,
    %             % samples = 0.4*\mysampledensity,
    %             % variable y = x,
    %         ]{  sqrt{9-9*x^2} };
            
    %         % Plot from equation
    %         \addplot[
    %             smooth,
    %             thick,
    %             red\Light,
    %             % domain  = -2:2,
    %             % samples = 0.4*\mysampledensity,
    %             % variable y = x,
    %         ]{  sqrt{1-x^2} };
            
    %     \end{axis}
    %     \end{tikzpicture}
    % \end{center}
    
\end{questionBox}

\begin{questionBox}1m{}
    
    Seja \(D\subset\mathbb{R}^3\) o domínio fechado, limitado superiormente pelo parabolóide \(z=1-(x^2+y^2)\) e inferiormente pelo parabolóide \(z=x^2+y^2\). O volume do domínio \textit{D} pode ser calculado a partir do seguinte integral triplo (seguindo a ordem z,y,x)

    \begin{flalign*}
        &
            \left.
                \begin{aligned}
                    z = 1-(x^2+y^2)
                 \\ z = x^2 + y^2
                 \\ 1-(x^2+y^2) = x^2 + y^2 
                    \implies
                    1/2 = x^2 + y^2
                \end{aligned}
            \right\}
            \implies
        &
    \end{flalign*}

    % \begin{center}
    %     % \pgfplotsset{height=0.3\pageheight, width= .95\textwidth}
        
    %     % \tikzset{external/remake next=true}
    %     \begin{tikzpicture}
    %         \begin{axis}
    %         [
    %             z buffer = {sort}, % default|none|auto|sort|reverse x seq|reverse y seq|reverse xy seq
    %             % xlabel={\(x\)}, ylabel={\(y\)}, zlabel={\(z\)}
    %             % xmajorgrids = true,
    %             % legend pos  = north west,
    %             % 3d view,
    %             % perspective,
    %             % view = {10}{10}, % rot/elevation
    %             hide axis,
    %             % axis lines = {center}, % 3D center/box/left/right
    %             % axis on top,
    %             % ticks = {none} % minor/major/both/none,
    %             domain = -1:2,
    %         ]
    %             % Legends
    %             % \addlegendimage{empty legend}
    %             % \addlegendentry[Red]{\( x \)}
                
    %             % Conical surface
    %             \addplot3 [
    %                 % Aparence
    %                 surf,
    %                 opacity      = 0.2,
    %                 fill opacity = 0.4,
    %                 faceted color = White,
    %                 shader = faceted interp,
    %                 % Scope
    %                 data cs = cart, % cart/polar/polarrad
    %                 samples = \mysampledensity,
    %                 % Variable
    %                 % variable   = z,
    %                 % variable y = z,
    %                 % domain   = -1:1,
    %                 % domain x = -1:2,
    %             ]{
    %                 x^2+y^2
    %             };

    %             % Conical surface
    %             \addplot3 [
    %                 % Aparence
    %                 surf,
    %                 opacity      = 0.2,
    %                 fill opacity = 0.4,
    %                 faceted color = White,
    %                 shader = faceted interp,
    %                 % Scope
    %                 data cs = cart, % cart/polar/polarrad
    %                 samples = \mysampledensity,
    %                 % Variable
    %                 % variable   = z,
    %                 % variable y = z,
    %                 % domain   = -1:1,
    %                 % domain y = -1:1,
    %             ]{
    %                 1-(x^2+y^2)
    %             };

    %             % % Conical surface
    %             % \addplot3 [
    %             %     % Aparence
    %             %     surf,
    %             %     opacity      = 0.2,
    %             %     fill opacity = 0.4,
    %             %     faceted color = White,
    %             %     shader = faceted interp,
    %             %     % Scope
    %             %     data cs = cart, % cart/polar/polarrad
    %             %     samples = \mysampledensity,
    %             %     % Variable
    %             %     % variable   = z,
    %             %     % variable y = z,
    %             %     % domain   = -1:1,
    %             %     % domain y = -1:1,
    %             % ]{
    %             %     0.5 = x^2 + y^2
    %             % };


                
            
    %         \end{axis}
    %     \end{tikzpicture}
    % \end{center}
        
    \begin{flalign*}
        &
            \implies
            \int_{-\sqrt{1/2}}^{\sqrt{1/2}}
            \left(
                \int_{-\sqrt{1/2 - x^2}}^{\sqrt{1/2 - x^2}}
                \left(
                    \int_{x^2+y^2}^{1-(x^2+y^2)}
                    \odif{z}
                \right)
                \odif{y}
            \right)
            \odif{x}
        &
    \end{flalign*}
    
\end{questionBox}

% Question 3
\begin{questionBox}1m{}
    
    Seja \textit{L} uma linha admitindo a representação paramétrica regular
    \begin{BM}[align*]
        \left\{
            \begin{aligned}
                x =& \cos(t)
             \\ y =& \sin(t)
             \\ z =& \log(1+t)
            \end{aligned}
        \right.
      & \quad
        0 \leq t \leq \pi/2
    \end{BM}

    Percorrida no sentido crescente do parâmetro \textit{t}, \(\varphi(x,y,z)=\exp(x\,y + y\,z+z\,x)\) e
    \begin{BM}
        \vec{u} 
        = u_1(x,y,z)\hat{\imath}
        + u_2(x,y,z)\hat{\jmath}
        + u_3(x,y,z)\hat{k}
        = \nabla\varphi(x,y,z)
    \end{BM}

    Seja
    \begin{BM}
        I
        = \int_L
          u_1(x,y,z)\odif{x}
        + u_2(x,y,z)\odif{y}
        + u_3(x,y,z)\odif{z}
    \end{BM}

    Então: \(I =\) ?

    \begin{flalign*}
        &
            I
            = \int_L
              u_1(x,y,z)\odif{x}
            + u_2(x,y,z)\odif{y}
            + u_3(x,y,z)\odif{z}
            = &\\&
            = \int_L
              (
                u_1(x,y,z)\hat{\imath}
                + u_2(x,y,z)\hat{\jmath}
                + u_3(x,y,z)\hat{k}
            )
            \cdot \vec{\odif{r}}
            = &\\&
            = \int_L
              \vec{u}\cdot \vec{\odif{r}}
            % = &\\&
            = \int_L
              \nabla\varphi(x,y,z)
              \cdot \vec{\odif{r}}
            = &\\&
            = \varphi(0,1,\log(1+\pi/2))
            - \varphi(1,0,0)
            % = &\\&
            = \exp(0 + \log(1+\pi/2) + 0)
            - \exp(0)
            = &\\&
            = \pi/2
        &
    \end{flalign*}
    
\end{questionBox}

\begin{questionBox}1{}
    
    Seja \(g(x,y,z)\) um campo escalar, definido e admitindo derivadas parciais contínuas até à segunda ordem num subconjunto aberto \textit{A} de \(\mathbb{R}^3\). uma esperssão de \(\nabla\cdot\nabla g^2\) em função de \(g,\lVert \nabla g\rVert\) e \(\nabla^2 g\)
    
\end{questionBox}

\end{document}
% !TEX root = ./AM_2C-Tests_Resolutions.2023.2.3.tex.tex
\providecommand\mainfilename{"./AM_2C-Tests_Resolutions.tex"}
\providecommand \subfilename{}
\renewcommand   \subfilename{"./AM_2C-Tests_Resolutions.2023.2.3.tex.tex"}
\documentclass[\mainfilename]{subfiles}

% \tikzset{external/force remake=true} % - remake all

\begin{document}

% \graphicspath{{\subfix{./.build/figures/AM_2C-Tests_Resolutions.2023.2.3.tex}}}
% \tikzsetexternalprefix{./.build/figures/AM_2C-Tests_Resolutions.2023.2.3.tex/graphics/}

\mymakesubfile{3}
[AM 2C]
{Exame 2023.2 Resolução} % Subfile Title
{Exame 2023.2 Resolução} % Part Title

\begin{questionBox}1{ % Q1
    Plano tangente a sup
    \begin{BM}
        \ln(y^3)+(x^2+1)\,e^z=1-x^3
        \\
        (-1,1,0)
    \end{BM}
} % Q1
    \answer{}
    \begin{flalign*}
        &
            f(x,y,z)
            = \ln(y^3)+(x^2+1)\,e^z-(1-x^3)
            = &\\&
            = \ln(y^3)+(x^2+1)\,e^z-1+x^3
            = 0
            ;&\\[3ex]&
            \left(
                \begin{aligned}
                    &
                           \pdv{f}{x}(p_0)\,(x-x_0)
                    &+\\+& \pdv{f}{y}(p_0)\,(y-y_0)
                    &+\\+& \pdv{f}{z}(p_0)\,(z-z_0)
                    &
                \end{aligned}
            \right)
            = \left(
                \begin{aligned}
                    &
                           (e^z\,2\,x+3\,x^2)(p_0)\,(x+1)
                    &+\\+& (y^{-3}\,3\,y^2)(p_0)\,(y-1)
                    &+\\+& ((x^2+1)\,e^z)(p_0)\,(z-0)
                    &
                \end{aligned}
            \right)
            = &\\&
            = \left(
                \begin{aligned}
                    &
                           e^0\,2*(-1)+3\,(-1)^2\,(x+1)
                    &+\\+& ((1)^{-3}\,3\,(1)^2)\,(y-1)
                    &+\\+& (((-1)^2+1)\,e^(0))(p_0)\,(z-0)
                    &
                \end{aligned}
            \right)
            = &\\&
            = 
            x + 1
            + 3\,(y-1)
            + 2\,z
            = 
            x
            + 3\,y
            + 2\,z
            - 2
            = 0
        &
    \end{flalign*}
\end{questionBox}

\begin{questionBox}1{ % Q2
    \begin{BM}
        \mathcal{D}
        = \left\{
            (x,y)\in\mathbb{R}^2
            : \left(
                \begin{aligned}
                       & x^2+y^2 \leq 4
                    \\ & x^2+y^2/4 \geq 1
                    \\ & 0\leq y\leq x\,\sqrt{3}/3
                \end{aligned}
            \right)
        \right\}
    \end{BM}
    Escreva em coordenadas polares
} % Q2
    \answer{}
    \begin{flalign*}
        &
            \left(
                \begin{aligned}
                    x^2+y^2 \leq 4
                \\  x^2+y^2/4 \geq 1
                \\  0\leq y\leq x\,\sqrt{3}/3
                \end{aligned}
            \right)
            = &\\&
            = \left(
                \begin{aligned}
                &
                    (r\,cos\theta)^2
                    + (r\,\sin\theta)^2 
                    = r^2
                    \leq 4
                    \implies \myvert{r}\leq 2
                &\\& 
                    (r\,\cos\theta)^2
                    + (r\,\sin\theta)^2/4 
                    = r^2\,\cos^2{\theta}
                    + r^2\,\sin^2{\theta}/4
                    \geq 1
                &\\&
                    0\leq r\,\sin\theta\leq (r\,\cos\theta)\,\sqrt{3}/3
                \implies
                    0\leq \tan\theta\leq \sqrt{3}/3
                &
                \end{aligned}
            \right)
            = &\\&
            = \left(
                \begin{aligned}
                &
                    \myvert{r}\leq 2
                &\\& 
                    r\geq \sqrt{\frac{2}{\cos^2{\theta}3+1}}
                &\\&
                    0\leq \theta\leq \pi/6
                &
                \end{aligned}
            \right)
        &
    \end{flalign*}
\end{questionBox}

\begin{questionBox}1{ % Q3
    \begin{BM}
        f(x,y)
        =\ln(x^2+y^2)
        , \mathbb{R}^2\backslash\{(0,0)\}
    \end{BM}
    Tem-se:
} % Q3
    \answer{}
    \begin{flalign*}
        &
            \gdif{f}(1,1)
            = \left(
                (x^2+y^2)^{-1}
                \,2\,x
                , (x^2+y^2)^{-1}
                \,2\,y
            \right)(1,1)
            = (1, 1)
            &\\[3ex]&
            D_{\vec{u}}f(1,1)
            = \lim_{t\to0}{
                \frac{
                    f(1+t/\sqrt{2},1+t/\sqrt{2})
                    + f(1,1)
                }{
                    t
                }
            }
            = &\\&
            = \lim_{t\to0}{
                \frac{
                    \ln\left(
                        (1+t/\sqrt{2})^2
                        + (1+t/\sqrt{2})^2
                    \right)
                    + \ln(1^2+1^2)
                }{
                    t
                }
            }
            = &\\&
            = \lim_{t\to0}{
                2\frac{
                    \ln2
                    + \ln\left(
                        1+t/\sqrt{2}
                    \right)
                }{
                    t
                }
            }
        &
    \end{flalign*}
\end{questionBox}

\begin{questionBox}1{ % Q4
    Seja a curva C
    \begin{BM}
        \begin{cases}
            C_1: & z^2=7-x^2+2\,x-4\,y^2
            \\
            C_2: & z=2
        \end{cases}
    \end{BM}
    Parametrização regular
} % Q4
    \answer{}
    \begin{flalign*}
        &
            \begin{cases}
                z^2
                =7-x^2+2\,x-4\,y^2
                \\
                z=2
            \end{cases}
            &\\&
            \begin{cases}
                2
                =7-(r\,\cos{\theta})^2
                + 2\,(r\,\cos(\theta))
                - 4\,(r\,\sin(\theta))^2
                \\
                z=2
            \end{cases}
            &\\[3ex]&
            -5
            =
            - r^2\,3\,\sin^2\theta
            + 2\,r\,\cos\theta
            &\\[3ex]&
            \frac{((1+2\,\cos(t))-1)^2}{2^2}
            +\frac{(\sin(t))^2}{1^2}
            = \cos^2(t)
            + \sin^2(t)
            = 1
        &
    \end{flalign*}
\end{questionBox}

\begin{questionBox}1{ % Q5
    Considere a função  \(f:\mathbb{R}^2\to\mathbb{R}\) definida por
    \begin{BM}
        f(x,y)=\begin{cases}
            \frac{y(x^2-y^2)}{\sqrt{x^2+y^2}}
            , \quad& (x,y)\neq(0,0)
            \\
            0
            , \quad& (x,y)=(0,0)
        \end{cases}
        \\
        \vec{u}=(2/\sqrt{5},1/\sqrt{5})
    \end{BM}
    Tem-se:
} % Q5
    \answer{}
    \begin{flalign*}
        &
            \pdv{f}{x}(0,0)
            = \lim_{h\to0}{
                \frac{f(h,0)-f(0,0)}{h}
            }
            = \lim_{h\to0}{
                \frac{
                    \frac{0(h^2-0^2)}{\sqrt{h^2+0^2}}
                    -0
                }{h}
            }
            = 0
            % 
            % 
            % 
            &\\[3ex]&
            \pdv{f}{y}(0,0)
            = \lim_{h\to0}{
                \frac{f(0,h)-f(0,0)}{h}
            }
            = &\\&
            = \lim_{h\to0}{
                \frac{
                    \frac{h(0^2-h^2)}{\sqrt{0^2+h^2}}
                    -0
                }{h}
            }
            = &\\&
            = \lim_{h\to0}{
                \frac{
                    \frac{-h^3}{\myvert{h}}
                }{h}
            }
            = &\\&
            = \lim_{h\to0}{
                \frac{-h^1}{\sgn{h}}
            }
            = 0
            % 
            % 
            % 
            &\\[3ex]&
            D_{\vec{u}}(0,0)
            = \lim_{t\to0}{
                \frac{
                    f(0+t\,2/\sqrt{5},0+t/\sqrt{5})
                    - f(0,0)
                }{t}
            }
            = &\\&
            = \lim_{t\to0}{
                \frac{
                    \frac{
                        (t/\sqrt{5})
                        (
                            (t\,2/\sqrt{5})^2
                            -(t/\sqrt{5})^2
                        )
                    }{
                        \sqrt{
                            (t\,2/\sqrt{5})^2+(t/\sqrt{5})^2
                        }
                    }
                    - 0
                }{t}
            }
            = &\\&
            = \lim_{t\to0}{
                \frac{
                    3\,t/5\,\sqrt{5}
                }{
                    \sgn{t}
                }
            }
            = 0
        &
    \end{flalign*}
\end{questionBox}

\begin{questionBox}1{ % Q6
    Considere
    \begin{BM}
        \lim_{(x,y)\to(1,-1)}{
            \frac{
                2(y+1)
                \,\cos(x-1)
            }{
                \sqrt{
                    (x-1)^2
                    +(y+1)^2
                }
            }
        }
    \end{BM}
} % Q6
    \answer{}
    \begin{flalign*}
        &
            \lim_{(x,y)\to(1,-1)}{
                \frac{
                    2(y+1)
                    \,\cos(x-1)
                }{
                    \sqrt{
                        (x-1)^2
                        +(y+1)^2
                    }
                }
            }
            = \lim_{x\to1,y=-x}{
                \frac{
                    2(y+1)
                    \,\cos(x-1)
                }{
                    \sqrt{
                        (x-1)^2
                        +(y+1)^2
                    }
                }
            }
            = &\\&
            = \lim_{x\to1}{
                \frac{
                    2(-x+1)
                    \,\cos(x-1)
                }{
                    \sqrt{
                        (x-1)^2
                        +(-x+1)^2
                    }
                }
            }
            = &\\&
            = \lim_{x\to1}{
                \frac{
                    -2\,\cos(x-1)
                }{
                    \sqrt{2}
                }
            }
            = -\sqrt{2}
        &
    \end{flalign*}
\end{questionBox}

\begin{questionBox}1{ % Q7
    Designe por \(\sigma\) a sup em \(\mathbb{R}^3\) def por:
    \begin{BM}
        \sigma = \left\{
            (x,y,z)\in\mathbb{R}^3
            : \left(
                \begin{aligned}
                    & (x,y)\in\myrange{0,1}\times\myrange{0,2}
                    \\ & z=x^2+y
                \end{aligned}
            \right)
        \right\}
    \end{BM}
    Suponha \(\sigma\) orientada pela norma \(\vec{n}\)\\
    O valor do fluxo do campo vetorial \(\vec{F}(x,y,z)=-6\,x^2\,\hat{\jmath}\) através da superfície \(\sigma\)
} % Q7
\end{questionBox}

\begin{questionBox}1{ % Q8
    Seja \(f:\mathbb{R}^2\mapsto\mathbb{R}\) uma função de classe \(C^1\) em \(\mathbb{R}^2\) tal que \(\gdif{f}(2,1)=(-2,1)\).
    \\Considere a função
    \begin{BM}
        g(x,y)
        = f\left(
            2\,x,
            \frac{2\,x}{y^2+1}
        \right)
    \end{BM}
    Tem-se
} % Q8
    \answer{}
    \begin{flalign*}
        &
            \begin{cases}
                \phi(x,y) = 2\,x
                \\
                \rho(x,y)
                = \frac{2\,x}{y^2+1}
            \end{cases}
            &\\[3ex]&
            \pdv{g}{x}
            = \pdv{f}{\phi}(2,1)
            \,\pdv{\phi}{x}
            = -2*2
            = -4
        &
    \end{flalign*}
\end{questionBox}

\begin{questionBox}1{ % Q9
    Seja \(\varphi:\mathbb{R}^3\to\mathbb{R}\) uma função de classe \(
        C^1
        \text{ em }
        \mathbb{R}
        \text{ e }
        c\in\mathbb{R}
    \).
    \\ Considere a função \(
        u(x,t)
        = x\,\varphi(x-c\,t)
        \text{. Para todo o }
        (x,t)\in\mathbb{R}^2
    \), Tem-se:
} % Q9
    \answer{}
    \begin{flalign*}
        &
            c\,\pdv{u}{x}(x,t)
            + \pdv{u}{t}(x,t)
            = c\,\left(
                \varphi(x-c\,t)
                + x\,\varphi'(x-c\,t)
            \right)(x,t)
            + \left(
                x\,\varphi'(x-c\,t)
                (-c)
            \right)(x,t)
            = c\,\varphi(x-c\,t)
        &
    \end{flalign*}
\end{questionBox}

\begin{questionBox}1{ % Q10
    Seja \(f:\mathbb{R}^3\mapsto\mathbb{R}\) uma função da classe \(C^1\) em \(\mathbb{R}^3\) tal que \(\gdif{f}(0,1,\ln{2})=(1,1,1)\).
    \\ Considere a função:
    \begin{BM}
        g:\mathbb{R}^2\to\mathbb{R}^3
        \\
        g(x,y)
        = \left(
            \sin(x\,y),
            x-2\,y,
            \ln(x^2+1)
        \right)
    \end{BM}
} % Q10
    \answer{}
    \begin{flalign*}
        &
            h =
            f\circ h
            = f\begin{bmatrix}
                \sin(1*0)
                \\ 1-2*0,
                \\ \ln(1^2+1)
            \end{bmatrix}
            = f(0,1,\ln2)
            &\\[3ex]&
            J(h)
            = \begin{bmatrix}
                \pdv{x}{u} & \pdv{x}{v}
            \end{bmatrix}_{(0,1)}
        &
    \end{flalign*}
\end{questionBox}

\begin{questionBox}1{ % Q11
    \begin{BM}
        F(x,y,z)
        = x\,\cos(y\,z)
        - z\,\exp(x-y-z)
        +x^2
        +3\,y
        -1
        \\
        P=(1,0,1)
    \end{BM}
    \(F(x,y,z)=0\) def x como func de y e z
} % Q11
    \answer{}
    \begin{flalign*}
        &
            \pdv{x}{y}
            = -\frac
            {\pdv{F}{y}(1,0,1)}
            {\pdv{F}{x}(1,0,1)}
            = -\frac
            {
                -(1)\,\sin(0*1)*1
                -1\,\exp(1-0-1)(-1)
                +3
            }
            {
                \cos(0*1)
                -1\,\exp(1-0-1)
                +2\,1
            }
            = -2
        &
    \end{flalign*}
\end{questionBox}

\begin{questionBox}1{ % Q12
    Considere a função
    \begin{BM}
        (y-2)\,x^2-y^2
    \end{BM}
    Tem-se:
} % Q12
    \answer{}
    \begin{flalign*}
        &
            \det H_f
            =\begin{vmatrix}
                  \pdv{f}{x,x}
                & \pdv{f}{x,y}
                \\\pdv{f}{y,x}
                & \pdv{f}{y,y}
            \end{vmatrix}
            =\begin{vmatrix}
                   2\,y-4
                &  2\,x
                \\ 2\,x
                &  -2
            \end{vmatrix}
            = -4\,y+8-4\,x^2
            &\\[3ex]&
            \begin{cases}
                \det H_f(0,0)
                =8
                , \pdv[order=2]{f}{x}
                = -4
                \therefore
                \text{Máximo local}
                \\
                \det H_f(2,2)
                = \det H_f(-2,2)
                = -16
                \quad\therefore
                \text{Sela}
                hessi
            \end{cases}
        &
    \end{flalign*}
\end{questionBox}

\begin{questionBox}1{ % Q13
    Seja \textit{D} a região do plano definida pelas condições: \(y\leq \leq 2-y^2,y\leq0\).
    \\O valor do integral
    \begin{BM}
        \iint_{D}{y\,\odif{x,y}}
    \end{BM}
} % Q13
    \answer{}
    \begin{flalign*}
        &
            \begin{cases}
                y+y^2-2
                = (y-1)(2+y)
                = 0
                \\
                y=0\implies x=2
                \\
                x=0\implies\myvert{y}=\sqrt{2}
            \end{cases}
            &\\&
            \iint_{D}{y\,\odif{x,y}}
            = \int_0^2{
                \int_y^{2-y^2}{
                    y\,\odif{x,y}
                }
            }
            = \int_0^2{
                    ((2-y^2)-y)
                    y\,\odif{y}
            }
            = \int_0^2{
                    (2\,y-y^3-y^2)
                    \odif{y}
            }
            = 2\,(2^2-0^2)/2
            -(2^4-0^4)/4
            -(2^3-0^3)/3
            = -8/3
        &
    \end{flalign*}
\end{questionBox}

\begin{questionBox}1{ % Q14
    Considere o seguinte integral triplo
    \begin{BM}
        I
        =\int_0^1{
            \int_{x^2}^{3-2\,x}{
                x\,y\,\odif{y,x}
            }
        }
    \end{BM}
    Inverta a integração
} % Q14
    \answer{}
    \begin{flalign*}
        &
            \begin{cases}
                x^2=3-2\,x
                \implies
                x^2-3+2\,x
                =(x-3)(x-1)
                =0
            \end{cases}
            &\\[3ex]&
            I 
            =\int_0^1{
                \int_{x^2}^{3-2\,x}{
                    x\,y\,\odif{y,x}
                }
            }
            =\int_1^3{
                \int_0^{(3-y)/2}{
                    x\,y\,\odif{x,y}
                }
            }
            +\int_0^1{
                \int_0^{\sqrt{x}}{
                    x\,y\,\odif{x,y}
                }
            }
        &
    \end{flalign*}
\end{questionBox}

\begin{questionBox}1{ % Q15
    Volume do sólido
    \begin{BM}
        S=\left\{
            (x,y,z)\in\mathbb{R}^3
            : \left(
                \begin{aligned}
                    0\leq z\leq 2-x^2-y^2
                    \\ \land
                    0\leq y\leq -x
                \end{aligned}
            \right)
        \right\}
    \end{BM}
} % Q15
    \answer{}
    \begin{flalign*}
        &
            \begin{cases}
                x=r\cos\theta
                \\
                y=r\sin\theta
                \\
                z=z
                \\
                x^2+y^2=r^2
                \\
                \tan{\theta}=y/x
                \\
                \myvert{det J}=r
            \end{cases}
            &\\&
            \begin{cases}
                0\leq z\leq 
                2
                -(r\,\cos\theta)^2
                -(r\,\sin\theta)^2
                = 2-r^2
                \\ 
                0
                \leq  r\,\sin\theta
                \leq -r\,\cos\theta
                \implies
                0
                \leq  \tan\theta
                \leq -1
                \implies
                \pi
                \geq  \theta
                \geq 3\pi/4
            \end{cases}
            &\\[3ex]&
            \int_0
        &
    \end{flalign*}
\end{questionBox}

\begin{questionBox}1{ % Q16
    Considere a função \(f:\mathbb{R}^2\to\mathbb{R}\)
    \begin{BM}
        f(x,y)
        = \begin{cases}
            \frac{x^2\,e^x+y^2}{x^2+y^2}
            , \quad& (x,y)\neq(0,0)
            \\
            1
            , \quad& (x,y)=(0,0)
        \end{cases}
    \end{BM}
    Tem-se:
} % Q16
    \answer{}
    \begin{flalign*}
        &
            \pdv{f}{x}(0,0)
            = \lim_{h\to0}{
                \frac{f(h,0)-f(0,0)}{h}
            }
            = \lim_{h\to0}{
                \frac{
                    \frac{h^2\,e^h+0^2}{h^2+0^2}
                    -1
                }{h}
            }
            = \lim_{h\to0}{
                \frac{e^h-1}{h}
            }
            % 
            % 
            % 
            &\\[3ex]&
            \pdv{f}{y}(0,0)
            = \lim_{h\to0}{
                \frac{f(0,h)-f(0,0)}{h}
            }
            = \lim_{h\to0}{
                \frac{
                    \frac{0^2\,e^0+h^2}{0^2+h^2}
                    -1
                }{h}
            }
            = 0
            &\\[3ex]&
            \lim_{(x,y)\to(0,0)}{f(x,y)}
            = \lim_{x\to0,y=m\,x}{
                \frac{x^2\,e^x+y^2}{x^2+y^2}
            }
            = \lim_{x\to0}{
                \frac{x^2\,e^x+(m\,x)^2}{x^2+(m\,x)^2}
            }
            = &\\&
            = \lim_{x\to0}{
                \frac{e^x+m^2}{(m^2+1)}
            }
            = 1
        &
    \end{flalign*}
\end{questionBox}

\begin{questionBox}1{ % Q17
    Considere o sólido
    \begin{BM}
        \mathcal{E}
        =\left\{
            (x,y,z)\in\mathbb{R}^3
            : \left(
                \begin{aligned}
                    &
                        x^2+y^2+z^2\leq9
                    &\\&
                        z\geq\sqrt{x^2+y^2}
                    &\\&
                        y\geq0
                    &
                \end{aligned}
            \right)
        \right\}
    \end{BM}
} % Q17
    \answer{}
    \begin{flalign*}
        &
            \begin{cases}
                x=r\sin\phi\,\cos\theta
                \\
                y=r\sin\phi\,\sin\theta
                \\
                z=\cos\phi
                \\
                \myvert{et J}=r^2\sin\phi
            \end{cases}
            &\\[3ex]&
            \begin{cases}
                     (r\,\sin\phi\,\cos\theta)^2
                    +(r\,\sin\phi\,\sin\theta)^2
                    +(r\,\cos\phi)^2
                    =r^2
                    \leq9
                \\
                    r\,\cos{\phi}
                    \geq
                    \sqrt{
                        (r\,\sin\phi\,\cos\theta)^2
                        +(r\,\sin\phi\,\sin\theta)^2
                    }
                    = \myvert{r\,\sin\phi}
                    \implies
                \\  \implies
                    1
                    \geq
                    \tan\phi
                \\
                    r\,\sin\phi\,\sin\theta\geq0
                    \implies
                    \sin\phi\,\sin\theta
                    \geq0
                    \implies
                    \theta
                    \in\myrange{0,\pi}
            \end{cases}
            &\\[3ex]&
        &
    \end{flalign*}
\end{questionBox}

\begin{questionBox}1{ % Q18
    Teorema de green
    \begin{BM}
        \Sigma
        = \left\{
            (x,y)\in\mathbb{R}^2
            : x^2+y^2\leq1
            , 0\leq y\leq x
        \right\}
    \end{BM}
} % Q18
    \answer{}
\end{questionBox}

\setcounter{question}{19}
\begin{questionBox}1{ % Q20
    Integral seg de reta entre 2 pontos
    \begin{BM}
        p_0=(0,1)
        ,p_1=(2,0)
        \\
        I=\int_C{
            y\,\odif{x}
            2\,x\,\odif{y}
        }
    \end{BM}
} % Q20
    \answer{}
    \begin{flalign*}
        &
            \phi(t)
            =A+(B-A)
            =(0,1)+((2,0)-(0,1))\,t
            =(0,1)+(2,-1)\,t
            =(2\,t,1-t)
            &\\[3ex]&
            I
            =\int_0^1{
                (1-t)\,2\,\odif{t}
            }
        &
    \end{flalign*}
\end{questionBox}

\begin{questionBox}1{ % Q21
    question
} % Q21
    body
\end{questionBox}

\end{document}
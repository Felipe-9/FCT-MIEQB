% !TEX root = ./AM_2C-Testes_Resolucoes.2023.1.1.tex
\providecommand\mainfilename{"./AM_2C-Testes_Resolucoes.tex"}
\providecommand \subfilename{}
\renewcommand   \subfilename{"./AM_2C-Testes_Resolucoes.2023.1.1.tex"}
\documentclass[\mainfilename]{subfiles}

% \tikzset{external/force remake=true} % - remake all

\begin{document}

% \graphicspath{{\subfix{./.build/figures/AM_2C-Testes_Resolucoes.2023.1.1}}}

\mymakesubfile{1}
[AM 2C]
{Teste 2023} % Subfile Title
{Teste 2023} % Part Title

\begin{questionBox}1{ % Q1
    Seja \(\mathcal{D}=\{(x,y)\in\mathbb{R}^2:x^2+y^2\leq 2,y\leq-1\}\) e r, \(\theta\) as coordenadas polares. Tem se:
} % Q1
    \begin{enumerate}[label=\Alph{enumi}.]
        \item \(\mathcal{D}=\{(x,y)=(r\,\cos\theta,r\,\sin\theta):\pi\,1/4\leq\theta\leq\pi\,3/4,-\sin^{-1}\theta\leq{r}\leq\sqrt{2}\}\)
        \item \(\mathcal{D}=\{(x,y)=(r\,\cos\theta,r\,\sin\theta):\pi\,3/4\leq\theta\leq\pi\,5/4,+\sin^{-1}\theta\leq{r}\leq\sqrt{2}\}\)
        \item \(\mathcal{D}=\{(x,y)=(r\,\cos\theta,r\,\sin\theta):\pi\,5/4\leq\theta\leq\pi\,7/4,-\sin^{-1}\theta\leq{r}\leq\sqrt{2}\}\)
        \item \(\mathcal{D}=\{(x,y)=(r\,\cos\theta,r\,\sin\theta):\pi\,5/4\leq\theta\leq\pi\,7/4,+\sin^{-1}\theta\leq{r}\leq\sqrt{2}\}\)
        \item Nenhum dos casos anteriores
    \end{enumerate}
    \answer{}
    \begin{flalign*}
        &
            \begin{cases}
                x = r\cos\theta
                \\
                y = r\sin\theta
            \end{cases}
        &\\[3ex]&    
            x^2+y^2
            = (r\cos\theta)^2+(r\sin\theta)^2
            = r^2\cos^2\theta+r^2\sin^2\theta
            = r^2(\cos^2\theta+\sin^2\theta)
            = r^2
            \leq 2
            \implies &\\&
            \implies
            r \leq \pm\sqrt{2}
            \iff
            \myvert{r} \leq \sqrt{2}
        &\\[3ex]&
            y
            =r\sin\theta
            \leq \sqrt{2}\sin\theta
            \leq -1
            \implies &\\&
            \implies
            \theta 
            \leq \sin^{-1}(-1/\sqrt{2})
            = \sin^{-1}(-\sqrt{2}/2)
            = -\pi/4
            = -\pi/4 + 2\pi
            = 7\pi/4
        &\\[3ex]&
            y 
            = r\sin\theta 
            \leq -1
            \implies
            r\leq -1/\sin\theta
        &
    \end{flalign*}
\end{questionBox}

\begin{questionBox}1{ % Q2
    Plano tangente no ponto \(0,0,1\)
} % Q2
    \begin{BM}
        f(x,y) = e^{-y^2}+\sin(2\,x-y)
    \end{BM}

    
\end{questionBox}

\begin{questionBox}1{ % Q3
    Considere a função \textit{f}. A curva de nível de valor -4 de \textit{f} é
} % Q3
    \begin{BM}
        f(x,y) = \frac{y^2}{1-x^2}
    \end{BM}

    \begin{flalign*}
        &
            \frac{y^2}{1-x^2}
            =-4
            \implies &\\&
            \implies
            y^2 = -4+(2\,x)^2
            \implies &\\&
            \implies
            -y^2/4+x^2=1
            &\\[3ex]&
            \myvert{x}\neq 1
        &
    \end{flalign*}
\end{questionBox}

\begin{questionBox}1{ % Q4
    Seja \(T(\rho,\theta,\phi)\) onde \(\rho,\theta,\phi\) são as c esfericas. a região do espaço limitada pelas superficies
} % Q4
    \begin{BM}
        T(\rho,\theta,\phi)
        = \left(
            \begin{aligned}
                & \rho\,\cos\theta\,\sin\phi,
                \\
                & \rho\,\sin\theta\,\sin\phi,
                \\
                & \rho\,\cos\phi
            \end{aligned}
        \right)
        \\[1.5ex]
        z=\sqrt{3-x^2-y^2}
        \qquad
        z^2=3(x^2+y^2)
    \end{BM}

    \begin{flalign*}
        &
            3-x^2-y^2
            = \pm\left(
                3(x^2+y^2)
            \right)
            = 
            \pm 3\,x^2\pm 3\,y^2
            % \implies &\\&
            \implies
            3=x^2(1\pm3)+y^2(1\pm3)
            \implies &\\&
            \implies
            \begin{cases}
                3=+4\,(x^2+y^2)
                \\
                3=-2\,(x^2+y^2)
            \end{cases}
            &\\[3ex]&
            3=4\,((x)^2+y^2)
            &\\[3ex]&
            \rho\cos\phi
            = \sqrt{
                3
                -\left(
                    \rho\,\cos\theta\,\sin\phi
                \right)^2
                -\left(
                    \rho\,\sin\theta\,\sin\phi
                \right)^2
            }
            = &\\&
            = \sqrt{
                3
                -\rho^2\,\sin^2\phi\left(
                    \cos^2\theta+\sin^2\theta
                \right)
            }
            = \sqrt{
                3-\rho^2\,\sin^2\phi
            }
            \implies &\\&
            \implies
            \rho^2\cos^2\phi
            =3-\rho^2\,\sin^2\phi
            \implies
            \rho^2(\cos^2\phi+\sin^2\phi)
            \rho^2
            =3
            % \implies &\\&
            \implies
            \rho = \pm\sqrt{3}
            &\\[3ex]&
            \rho^2\,\cos^2\theta
            =3\left(
                (\rho\,\cos\theta\,\sin\phi)^2
                +
                (\rho\,\sin\theta\,\sin\phi)^2
            \right)
            =3\left(
                \rho^2\,\cos^2\theta\,\sin^2\phi
                +
                \rho^2\,\sin^2\theta\,\sin^2\phi
            \right)
            =3\rho^2\,\sin^2\phi
            \implies
            \cos^2\theta = 3\sin^2\phi
            &\\&
            3-x^2-y^2
            = 3
            -\left(
                \rho\,\cos\theta\,\sin\phi
            \right)^2
            -\left(
                \rho\,\sin\theta\,\sin\phi
            \right)^2
            = 3-\rho^2\,\sin^2\theta
            \geq 0
            \implies
            \rho^2\,\sin^2\theta
            \leq 3
            \implies &\\&
            \implies
            \rho\leq\sqrt{3}
            &\\[3ex]&
            3(x^2+y^2)
            =3\left(
                (\rho\,\cos\theta\,\sin\phi)^2
                +
                (\rho\,\sin\theta\,\sin\phi)^2
            \right)
            =3\left(
                \rho^2\,\cos^2\theta\,\sin^2\phi
                +
                \rho^2\,\sin^2\theta\,\sin^2\phi
            \right)
            =3\rho^2\,\sin^2\phi
            \leq 0
            \implies &\\&
            \implies
        &
    \end{flalign*}
\end{questionBox}

\begin{questionBox}1{ % Q5
    Seja \textit{C} a curva em \(\mathbb{R}^3\) definida por
} % Q5
    \begin{BM}
        z = 8-4\,x^2-y^2;
        \qquad
        z = 4
    \end{BM}

    tem-se:
    \begin{flalign*}
        &
            4
            = 8-4\,x^2-y^2
            \implies
            4\,x^2+y^2
            = 4
            &\\&
            4\,(2\,\cos t)^2
            +\sin^2t
            = 15\,\cos^2t
            + \cos^2t
            + \sin^2t
            = 15\,\cos^2t
            = 4
            \implies &\\&
            \implies
            t = \cos^{-1}\sqrt(4/15)
            &\\[6ex]&
            \nabla{f(x,y)}(1/2,\sqrt{3})
            = \left(
                \begin{aligned}
                    &
                        -8\,x
                    &\\&
                        -2\,y
                    &
                \end{aligned}
            \right)(1/2,\sqrt{3})
            = \left(
                \begin{aligned}
                    &
                        -4
                    &\\&
                        -2\,\sqrt{3}
                    &
                \end{aligned}
            \right)
        &
    \end{flalign*}
\end{questionBox}

\begin{questionBox}1{ % Q6
    Comprimento da curva
} % Q6
    \begin{flalign*}
        &
            \begin{cases}
                y=x^2
                \\
                x^2+y^2=2
            \end{cases}
        &
    \end{flalign*}
\end{questionBox}

\begin{questionBox}1{ % Q7
    Considere
} % Q7
    \begin{BM}
        \lim_{(x,y)\to(0,0)}{
            \frac{
                -x^2-y^2-2\,x^4
            }{
                x^2+y^2
            }
        }
    \end{BM}

    \begin{flalign*}
        &
            \lim_{(x,y)\to(0,0)}{
                \frac{
                    -x^2-y^2-2\,x^4
                }{
                    x^2+y^2
                }
            }
            = &\\[3ex]&
            = \lim_{y\to0}{
                \lim_{x\to0}{
                    \frac{
                        -x^2-y^2-2\,x^4
                    }{
                        x^2+y^2
                    }
                }
            }
            = \lim_{y\to0}{
                \frac{
                    -y^2
                }{
                    y^2
                }
            }
            = \lim_{y\to0}{
                -1
            }
            =-1
            = &\\[3ex]&
            = \lim_{x\to0}{
                \lim_{y\to0}{
                    \frac{
                        -x^2-y^2-2\,x^4
                    }{
                        x^2+y^2
                    }
                }
            }
            = \lim_{x\to0}{
                \frac{
                    -x^2-2\,x^4
                }{
                    x^2
                }
            }
            = \lim_{x\to0}{
                \frac{
                    -x^2(1+2\,x^2)
                }{
                    x^2
                }
            }
            = \lim_{x\to0}{
                -(1+2\,x^2)
            }
            = -1
            = &\\[3ex]&
            = \lim_{x\to0}{
                \lim_{y\to x}{
                    \frac{
                        -x^2-y^2-2\,x^4
                    }{
                        x^2+y^2
                    }
                }
            }
            = \lim_{x\to0}{
                \frac{
                    -x^2-x^2-2\,x^4
                }{
                    x^2+x^2
                }
            }
            = \lim_{x\to0}{
                -(1-x^2)
            }
            = -1
            = &\\&
            = \lim_{x\to0}{
                \lim_{y\to x^2}{
                    \frac{
                        -x^2-y^2-2\,x^4
                    }{
                        x^2+y^2
                    }
                }
            }
            = \lim_{x\to0}{
                \frac{
                    -x^2-x^4-2\,x^4
                }{
                    x^2+x^4
                }
            }
            = \lim_{x\to0}{
                \frac{
                    -x^2-3\,x^4
                }{
                    x^2+x^4
                }
            }
            = \lim_{x\to0}{
                \frac{
                    -2\,x^4
                }{
                    x^2+x^4
                }
                - \frac{
                    x^2+x^4
                }{
                    x^2+x^4
                }
            }
            = \lim_{x\to0}{
                \frac{
                    -2\,x^2
                }{
                    1+x^2
                }
                - 1
            }
            = -1
        &
    \end{flalign*}
\end{questionBox}

\begin{questionBox}1{ % Q8
    A eq do plano tg a sup no ponto \((1,-1,0)\)
} % Q8
    \begin{BM}
        y^3+(x+1)\,e^z=2-x^3
    \end{BM}

    \begin{flalign*}
        &
            (1,-1,0)
            +\nabla(f(x))(1,-1,0)
        &
    \end{flalign*}
\end{questionBox}

\begin{questionBox}1{ % Q9
    Considere a função \(f:\mathbb{R}^2\to\mathbb{R}\) definia por
} % Q9
    \begin{BM}
        f(x,y)
        = \begin{cases}
            \frac{x^3-y^5}{x^2+2\,y^4},\quad& (x,y)\neq(0,0)
            \\
            0\quad& (x,y)=(0,0)
        \end{cases}
        \\
        \vec{u} = (2^{-1/2},2^{-1/2})
        \qquad
        \nabla f(0,0) = (1,-1/2)
    \end{BM}

    \begin{flalign*}
        &
            1
        &
    \end{flalign*}
\end{questionBox}

\begin{questionBox}1{ % Q10
    \(\odv{f}{x}(0,0)\) e \(\odv{f}{y}(0,1)\) de \textit{f}
} % Q10
    \begin{BM}
        f(x,y)
        = \begin{cases}
            \frac{5\,x^3+3\,y^4}{x^2+y^2}
            \quad& (x,y)\neq(0,0)
            \\
            0
            \quad& (x,y)=(0,0)
        \end{cases}
    \end{BM}

    \begin{flalign*}
        &
            \odv{f}{x}(0,0)
            = \frac{
                15\,x^2\,(x^2+y^2)
                - (5\,x^3+3\,y^4)(2\,x)
            }{
                (x^2+y^2)^2
            }(0,0)
            = &\\&
            = \frac{
                15\,x^4
                - 10\,x^4
                15\,x^2\,y^2
                - 6\,y^4\,x
            }{
                x^4
                + 2\,x^2\,y^2
                + y^4
            }(0,0)
            = &\\&
            = \frac{
                5\,x^4
                + 15\,x^2\,y^2
                - 6\,x\,y^4
            }{
                x^4
                + 2\,x^2\,y^2
                + y^4
            }(0,0)
            = &\\&
            = \frac{
                5\,x^4
            }{
                x^4
                + 2\,x^2\,y^2
                + y^4
            }(0,0)
            + \frac{
                15\,x^2\,y^2
            }{
                x^4
                + 2\,x^2\,y^2
                + y^4
            }(0,0)
            + \frac{
                - 6\,x\,y^4
            }{
                x^4
                + 2\,x^2\,y^2
                + y^4
            }(0,0)
            = &\\&
            = \frac{
                5
            }{
                1
                + 2\,y^2/x^2
                + y^4/x^4
            }(0,0)
            + \frac{
                15
            }{
                x^2/y^2
                + 2
                + y^2/x^2
            }(0,0)
            + \frac{
                - 6\,x
            }{
                x^4/y^4
                + 2\,x^2/y^2
                + 1
            }(0,0)
            = &\\&
            = \frac{
                5
            }{
                1
                + 2
                + 1
            }(0,0)
            + \frac{
                15
            }{
                1
                + 2
                + 1
            }(0,0)
            + \frac{
                - 6\,x
            }{
                1
                + 2
                + 1
            }(0,0)
            = &\\&
            = \frac{5}{4}
            + \frac{15}{4}
            = 5
            &\\[3ex]&
            \odv{f}{y}(0,1)
            = \frac{
                (12\,y^3)(x^2+y^2)
                - (5\,x^3+3\,y^4)(2\,y)
            }{
                x^4
                + 2\,x^2\,y^2
                + y^4
            }(0,1)
            = &\\&
            = \frac{
                12\,y^3\,x^2
                + 12\,y^5
                - 10\,x^3\,y
                - 6\,y^5
            }{
                x^4
                + 2\,x^2\,y^2
                + y^4
            }(0,1)
            = &\\&
            = \frac{
                + 6\,y^5
                12\,y^3\,x^2
                - 10\,x^3\,y
            }{
                x^4
                + 2\,x^2\,y^2
                + y^4
            }(0,1)
            = &\\&
            = 6
        &
    \end{flalign*}

\end{questionBox}

\setcounter{question}{14}

\begin{questionBox}1{ % Q15
    \(\odv{z}{x}\)
} % Q11
    \begin{BM}
        F(x,y,z) 
        = \frac{y}{z}
        - \frac{3\,x}{y}
        - \frac{2\,z}{x}
        \\
        z=z(x,y)
        : P(1,-1,1)\land F(x,y,z)=0
    \end{BM}
    
    \begin{flalign*}
        &
            0
            = \frac{y}{z}
            - \frac{3\,x}{y}
            - \frac{2\,z}{x}
            = y^2\,x
            - 3\,x^2\,z
            - 2\,z^2\,y
            &\\&
            0
            = \odv{y^2\,x}{x}(1,-1,1)
            - \odv{3\,x^2\,z}{x}(1,-1,1)
            - \odv{2\,z^2\,y}{x}(1,-1,1)
            = 1
            - 6
            - 3\odv{z}{x}(1,-1)
            + 4 \odv{z}{x}(1,-1)
            - 2
            = -7
            + \odv{z}{x}(1,-1)
            \implies &\\&
            \implies
            \odv{z}{x}(1,-1)
            = 7
        &
    \end{flalign*}
\end{questionBox}



\end{document}
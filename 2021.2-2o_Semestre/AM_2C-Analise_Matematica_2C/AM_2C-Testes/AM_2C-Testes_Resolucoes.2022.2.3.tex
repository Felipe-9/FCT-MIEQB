% !TEX root = ./AM_2C-Testes_Resolucoes.2022.2.3.tex
\providecommand\mainfilename{"./AM_2C-Testes_Resolucoes.tex"}
\providecommand \subfilename{}
\renewcommand   \subfilename{"./AM_2C-Testes_Resolucoes.2022.2.3.tex"}
\documentclass[\mainfilename]{subfiles}

% \tikzset{external/force remake=true} % - remake all

\date{11/01/2023}

\begin{document}

\graphicspath{{\subfix{./.build/figures/AM_2C-Testes_Resolucoes.2022.2.3/}}}

\mymakesubfile{1}
[AM\,2C]
{Exame 3: Resolução} % Subfile Title
{Exame 3: Resolução} % Part Title

\group{}

\begin{questionBox}1{ % 1 Q1
    Parabola\to Vertice, foco e diretriz
} % 1 Q1
    \begin{BM}
        x = 2-y-y^2/4
    \end{BM}

    \begin{flalign*}
        &
            x
            = \frac{(y-y')^2}{4\,a}+x'
            % &\\[3ex]&
            \begin{cases}
                a 
                = (4*(-1/4))^{-1}
                = -1
                \\
                y' 
                = -(-1)*2*(-1)
                = -2
                \\
                x'
                = 2 - \frac{(-2)^2}{4*(-1)}
                = 3
            \end{cases}
            &\\&
            \therefore
            x 
            = \frac{(y+2)^2}{-4}+3
            \begin{cases}
                X' = (3,-2)
                \\
                F = (x'+a,y') = (3-1,-2) = (2,-2)
                \\
                L \subset\mathbb{R}^2: 
                x
                =x'-a
                =3+1 = 4
            \end{cases}
        &
    \end{flalign*}

\end{questionBox}

\begin{questionBox}1{ % 1 Q2
    Seja \(f:\mathbb{R}^2\to\mathbb{R}\) uma norma.
} % 1 Q2
    b) \(f(\lambda\,x) = \lambda\,f(x)\)
\end{questionBox}

\begin{questionBox}1{ % 1 Q3
    Considere o sistema de equações
} % 1 Q3
    \begin{BM}
        \begin{cases}
            \log(x\,u^2)-y+2\,v=0
            \\
            x\,e^v - y\,v^2 + u = 0
        \end{cases}
    \end{BM}
    
    Define u e v como funções de x e y na viz de \(P_0=(x_0,y_0,u_0,v_0)=(1,0,-1,0)\)

    \begin{flalign*}
        &
            \frac{\pdv{f_1}{x}}{\pdv{f_1}{u}}
            = \frac{
                \frac{(u^2)}{x\,u^2}\log(e)
            }{
                \frac{x\,2\,u}{x\,u^2}\log(e)
            }
            = \frac{
                \log(e)
            }{
                -2\log(e)
            }
            =-1/2
            &\\&
            \frac{\pdv{f_1}{x}}{\pdv{f_1}{v}}
            = \frac{
                \frac{(u^2)}{x\,u^2}\log(e)
            }{
                2
            }
            = \frac{
                \log(e)
            }{
                2
            }
        &
    \end{flalign*}

    \begin{flalign*}
        &  
            \pdv{f_1}{x}(P_0)
            = \frac{
                \left(
                    u^2
                    + x\,2\,u\,\pdv{u}{x}
                \right)
            }{x\,u^2}
            \,\log(e)
            + 2\,\pdv{v}{x}
            = \frac{u_0^2\,\log(e)}{x_0\,u_0^2}
            + \frac{x_0\,2\,u_0\,\pdv{u}{x}\,\log(e)}{x_0\,u_0^2}
            + 2\,\pdv{v}{x}
            = &\\& 
            = \frac{\log(e)}{1}
            + \frac{2\,\pdv{u}{x}\,\log(e)}{(-1)}
            + 2\,\pdv{v}{x} 
            = \log(e)
            - 2\,\pdv{u}{x}\,\log(e)
            + 2\,\pdv{v}{x} 
            = 0
            &\\[3ex]&
            \pdv{f_2}{x}(P_0)
            = e^{v_0}+x_0\,e^{v_0}\,\pdv{v}{x}
            -y_0\,2\,v_0\,\pdv{v}{x}
            +\pdv{u}{x}
            = 
            1+\pdv{v}{x}+\pdv{u}{x}
            =0
            \implies &\\[3ex]&
            \implies
            \pdv{u}{x}
            = \pdv{v}{x}\log(e)^{-1}+1/2
            = \left(
                -\pdv{u}{x}-1
            \right)\log(e)^{-1}+1/2
            = &\\&
            =
            -\pdv{u}{x}\log(e)^{-1}
            -\log(e)^{-1}
            1/2
            \implies
            \pdv{u}{x}
            =
            % 
            \pdv{u}{x}
            =
            \frac{
                -\log(e)^{-1}
                1/2
            }{
                \log(e)^{-1}+1
            }
        &
    \end{flalign*}

\end{questionBox}

\begin{questionBox}1{ % 1 Q4
    Seja \(g(s,t)=f(u,v)\) em q f é dif e \(u=s^2-t^2,v=t^2-s^2\). Sabendo q g satisfaz a eq
} % 1 Q4
    \begin{BM}
        (t+2)\pdv{g}{s}
        +(s+2)\pdv{g}{t}
        =h(s,t)\left(
            \pdv{f}{u}
            -\pdv{f}{v}
        \right)
    \end{BM}

    % \begin{flalign*}
    %     &
    %     (t+2)\pdv{g}{s}
    %     +(s+2)\pdv{g}{t}
    %     = (t+2)\frac{\pdf{f}{s}}{\pdv{f}{g}}
    %     + (s+2)\pdv{g}{t}
    %     = 
    %     h(s,t)\left(
    %         \pdv{f}{u}
    %         -\pdv{f}{v}
    %     \right)
    %     &
    % \end{flalign*}

\end{questionBox}

\begin{questionBox}1{ % 1 Q5
    Considere o conjunto
} % 1 Q5
    \begin{BM}
        A = \left\{
            (x,y)\in\mathbb{R}^2:
            y\geq x 
            \land
            x\geq y^2
        \right\}
    \end{BM}

    Seja \textit{L} a fronteira de A percorrida no sentido +. O integral de linha

    \begin{BM}
        \int_L{
            (x^3-2\,y)
            \odif{x}
            +(2\,x-y^3)
            \odif{y}
        }
    \end{BM}

    Pode ser calc usando coord polares

    \begin{flalign*}
        &
            \begin{cases}
                P_1 = (0,0)
                \\
                P_2:y=x=y^2\implies P_2 = (1,1)
            \end{cases}
            &\\&
            \begin{cases}
                P_1 = (0,0)
                \\
                P_2 
                = (\sqrt{1+1},\arccos(1/\sqrt{2}))
                = (\sqrt{2},\pi/4)
            \end{cases}
            &\\[3ex]&
            % --------------------------------- Jacobiana -------------------------------- %
            \det\Jacobian
            = \rho
            &\\[3ex]&
            x=y^2
            \implies 
            \rho\cos\theta = \rho^2\sin^2\theta
            \implies 
            \rho
            =\frac{\cos\theta}{\sin^2{\theta}}
            &\\[3ex]&
            \therefore
            \int_L{
                (x^3-2\,y)
                \odif{x}
                +(2\,x-y^3)
                \odif{y}
            }
            = \iint_A{
                4 \odif{x,y}
            }
            = \int_{\pi/4}^{\pi/2}{
                \int_{0}^{\frac{\cot(\theta)}{\sin^{\theta}}}{
                    4\rho
                    \odif{\rho}
                }\odif{\theta}
            }
        &
    \end{flalign*}

\end{questionBox}

\begin{questionBox}1{ % 1 Q6
    plano tg ao cone elip \(x^2+4\,y^2=z^2\) no p (3,2,5)
} % 1 Q6
    \begin{flalign*}
        &   
            (x)(2\,x')
            + (y)(8\,y')
            - (z)(2\,z')
            = &\\&
            =
            2\,x'\,x
            + 8\,y'\,y
            - 2\,z'\,z
            -2\,x'^2
            -8\,y'^2
            +2\,z'^2
            = &\\&
            = 6\,x
            + 16\,y
            - 10\,z
            = &\\&
            = 6\,x
            + 16\,y
            - 10\,z
            =0
            \implies &\\&
            \implies
            3\,x
            + 8\,y
            - 5\,z
            =0
        &
    \end{flalign*}
\end{questionBox}

\begin{questionBox}1{ % 1 Q7
    O integral repetido
} % 1 Q7
    \begin{BM}
        \int_{-2}^{0}{
            \int_{x}^{0}{
                x^2\,\odif{x}
            }
        }
        + \int_{0}^{2}{
            \int_{0}^{x}{
                x^2\,\odif{x}
            }
        }
    \end{BM}

    \begin{flalign*}
        &
            \int_{-2}^{0}{
                \int_{-2}^{y}{
                    x^2\odif{x}
                }\odif{y}
            }
            +\int_{0}^{2}{
                \int_{y}^{2}{
                    x^2\odif{x}
                }\odif{y}
            }
        &
    \end{flalign*}

\end{questionBox}

\group{}

\begin{questionBox}1{ % 2 Q1
    Considere a função real g, continua de 2 var
} % 2 Q1
    \begin{BM}
        g(x,y)
        =\begin{cases}
            \frac{x^4+y^4}{x^2+y^2} 
            \quad &:(x,y)\neq(0,0)
            \\
            0
            \quad &:(x,y)=(0,0)
        \end{cases}
    \end{BM}
\end{questionBox}

\begin{questionBox}2{ % 2 Q1.1
    Determine \(\pdv{g}{x}(x,y),\forall\,(x,y)\in\mathbb{R}^2\)
} % 2 Q1.1
    \begin{flalign*}
        &
            \pdv{g}{x}(x,y)
            = \frac{
                4\,x^3(x^2+y^2)
                - 2\,x(x^4+y^4)
            }{
                (x^2+y^2)^2
            }
            &\\&
        &
    \end{flalign*}
\end{questionBox}

\begin{questionBox}2{ % 2 Q1.2
    Estude a continuidade de \(\pdv{g}{x}(x,y)\) em (0,0)
} % 2 Q1.2
    \begin{flalign*}
        &
            \forall\,\delta>0\,
            \exists\,\varepsilon>0:
            \left(
                \left(
                    \forall (x,y)\neq(0,0)
                    \land
                    \myVert{\sqrt{x^2+y^2}}<\varepsilon
                \right)
                \implies
                \myvert{
                    g(x,y)-0
                }<\delta
            \right)
            \implies &\\&
            \implies
            \myvert{
                \frac{
                    x^4+y^4
                }{
                    x^2+y^2
                }
            }
            \leq \frac{
                x^4
                +2\,x^2\,y^2
                +y^4
            }{
                x^2+y^2
            }
            = \frac{
                (x^2+y^2)^2
            }{
                x^2+y^2
            }
            = x^2+y^2
            \leq \varepsilon^2
            =\delta
            &\\&
            \therefore
            \varepsilon=\sqrt{\delta}
        &
    \end{flalign*}
\end{questionBox}

\begin{questionBox}2{ % 2 Q1.3
    Estude a dif de g em (0,0)
} % 2 Q1.3
    \begin{flalign*}
        &
            \pdv{g}{x}(0,0)
            = \lim_{h\to0}{
                \frac{g(h,0)-g(0,0)}{h}
            }
            = \lim_{h\to0}{
                \frac{g(h,0)-g(0,0)}{h}
            }
            = &\\&
            = \lim_{h\to0}{
                \frac{
                    h^2
                }{h}
            }
            = \lim_{h\to0}{
                h
            }
            =0
            =\pdv{g}{y}(0,0)
            &\\[3ex]&
            g(a,b)-g(0,0)
            = \frac{a^4+b^4}{a^2+b^2}-0
            = &\\&
            = \pdv{g}{x}(0,0)\,a
            + \pdv{g}{y}(0,0)\,b
            + \varepsilon(a,b)\,\sqrt{a^2+b^2}
            = \varepsilon(a,b)\,\sqrt{a^2+b^2}
            \implies &\\&
            \implies
            \varepsilon(a,b)
            = \frac{a^4+b^4}{(a^2+b^2)^2}
            = \frac{a^4+b^4}{a^4+2\,a^2\,b^2+b^4}
            \implies &\\&
            \implies
            \lim_{a\to0^+}{\varepsilon(a,a)}
            = \lim_{a\to0^+}{
                \frac{a^4+a^4}{(a^2+a^2)^2}
            }
            = \lim_{a\to0^+}{
                \frac{2\,a^4}{4\,a^4}
            }
            =1/2\neq0
        &
    \end{flalign*}
\end{questionBox}

\begin{questionBox}1{ % 2 Q2
    Considere a função \(f:\mathbb{R}^2\to\mathbb{R}\) definida por
} % 2 Q2
    \begin{BM}
        f(x,y)=2\,x^3+x\,y^2+2\,x\,y
    \end{BM}
\end{questionBox}

\begin{questionBox}2{ % 2 Q2.1
    Determine os extremos locais de f
} % 2 Q2.1
    \begin{flalign*}
        &
            \left\{
                (x,y)\in\mathbb{R}^2:
                \left(
                    \begin{aligned}
                        & \pdv{f}{x} = 6\,x+y^2+2\,y=0
                        \\
                        & \pdv{f}{y} = 2\,x\,y+2\,x=0
                    \end{aligned}
                \right)
            \right\}
            = &\\&
            = \left\{
                (x,y)\in\mathbb{R}^2:
                \left(
                    \begin{aligned}
                        & x = -\frac{y(y+2)}{6}
                        % & y(y+2)/-6=x
                        \\
                        & 2\,x\,(y+1)=0
                        \\
                        & y(y+2)(y+1)=0
                    \end{aligned}
                \right)
            \right\}
            = &\\&
            = \left\{
                (0,0),
                (1/6,-1),
                (0,-2)
            \right\}
            &\\[3ex]&
            \det\Hessiana(f(x,y))
            = \begin{vmatrix}
                \pdv[order=2]{f}{x}
                & \pdv{f}{x,y}
                \\
                \pdv[order=2]{f}{y}
                & \pdv{f}{y,x}
            \end{vmatrix}
            = \begin{vmatrix}
                6
                & 2\,y+2
                \\
                2\,x
                & 2\,y+2
            \end{vmatrix}
            = (12- 4\,x)(y+1)
            &\\[3ex]&
            \begin{cases}
                \det\Hessiana(f(0,0)) = 12 \land \pdv[order=2]{f}{x} = 6 & \therefore \text{ minimo local}
                \\
                \det\Hessiana(f(1/6,-1)) = 0 & \therefore \text{ indeterminado}
                \\
                \det\Hessiana(f(0,-2)) = -12 & \therefore \text{ ponto de cela}
            \end{cases}
        &
    \end{flalign*}
\end{questionBox}

\begin{questionBox}2{ % 2 Q2.2
    Extremos locais restrita a
} % 2 Q2.2
    \begin{BM}
        \left\{
            (x,y)\in\mathbb{R}^2:
            x-y=1
        \right\}
    \end{BM}

    e com x,y verificando \(\myvert{x},\myvert{y}<4\)

\end{questionBox}

\group{}

\begin{questionBox}1{ % 3 Q1
    Calcule o integral de linha
} % 2 Q2.3
    \begin{BM}
        \int_L{
            2\,y\,e^{z^2}\odif{x}
            +(x^2+y-z)\odif{y}
            +(y+z)\odif{z}
        }
    \end{BM}
\end{questionBox}

\begin{minipage}{\textwidth}
    \ \vspace{10cm}
\end{minipage}

\end{document}
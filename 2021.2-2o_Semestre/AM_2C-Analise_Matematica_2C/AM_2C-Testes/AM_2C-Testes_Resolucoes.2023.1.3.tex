b% !TEX root = ./AM_2C-Testes_Resolucoes.2023.2.3.tex
\providecommand\mainfilename{"./AM_2C-Testes_Resolucoes.tex"}
\providecommand \subfilename{}
\renewcommand   \subfilename{"./AM_2C-Testes_Resolucoes.2023.2.3.tex"}
\documentclass[\mainfilename]{subfiles}

% \tikzset{external/force remake=true} % - remake all

\begin{document}

\graphicspath{{\subfix{./.build/figures/AM_2C-Testes_Resolucoes.2023.2.3}}}
\tikzsetexternalprefix{./.build/figures/AM_2C-Testes_Resolucoes.2023.2.3/graphics/}

\mymakesubfile{3}
[AM 2C]
{Exame Resolução} % Subfile Title
{Exame Resolução} % Part Title

\begin{questionBox}1{ % Q1
    A equação do plano angente à superfície de equação
} % Q1
    \begin{BM}
        e^{x,y}
        +y\,\sin{x}
        -y^2+z^2+2\,x=2\,\pi
    \end{BM}

    \answer{B.}
\end{questionBox}

\begin{questionBox}1{ % Q2
    Considere f, a curva de nível em 1/2 de f é
} % Q2
    \begin{BM}
        f(x,y)
        = \frac{y^2-4}{4\,x^2+y^2-4}
    \end{BM}

    \answer{D.}
    \begin{flalign*}
        &
            f(x,y)
            = \frac{y^2-4}{4\,x^2+y^2-4}
            = 1/2
            \implies
            y^2 = 4\,x^2+4
            ; &\\[3ex]&
            0
            \neq 
            4\,x^2+y^2-4
            % \implies &&
            \implies
            y^2
            \neq 
            4-4\,x^2
            \implies &\\&
            \implies
            4-4\,x^2\neq 4+4\,x^2
            \implies
            0\neq x
            \implies
            y\neq\pm2
            ; &\\[3ex]&
            \therefore
            \left\{
                (x,y)\in\mathbb{R}^2\backslash\{(0,2),(0,-2)\}:
                y^2/4-x^2=1
            \right\}
        &
    \end{flalign*}
\end{questionBox}

\begin{questionBox}1{ % Q3
} % Q3
    \answer{A.}
\end{questionBox}

\begin{questionBox}1{ % Q4
    Considere a curva \textit{C} em \(\mathbb{R}^2\) no sentido horario
} % Q4
    \begin{BM}
        C = \left\{
            (x,y,z)\in\mathbb{R}^3:
            \left\{
                \begin{aligned}
                    z+x^2+3\,y^2=3
                    \\
                    z=6\,y
                \end{aligned}
            \right\}
        \right\}
    \end{BM}

    \answer{}
    \begin{flalign*}
        &
            x=0
            \implies\begin{cases}
                z=3-3\,y^2
                \\
                z=6\,y
            \end{cases}
            &\\&
            6\,y=3-3\,y^2
            \implies
            0=3\,y^2+6\,y-3
            \begin{cases}
                a=(4*3)^{-1}=1/12
                \\
                y'
                =-6*2*1/12
                =-1
                \\
                x'
                = -3-(-1)^2/(4*1/12)
                = -6
            \end{cases}
            &\\&
            \therefore
            0=3(y+1)^2-6
            \implies
            y=\pm\sqrt{2}-1
        &
    \end{flalign*}
    \begin{center}
        % \tikzset{external/remake next=true}
        % \pgfplotsset{height=7cm, width= .6\textwidth}
        \begin{tikzpicture}
        \begin{axis}
            [
                % xmajorgrids = true,
                % legend pos={south east},
                % domain=0:4,
                % xlabel={},
                % ylabel={},
                axis lines={center}, % left|right|center|box
            ]
            % Legends
            % \addlegendimage{empty legend}
            % \addlegendentry[Graph]{\( 
            %     {\color{Graph}x},
            %     {\color{GraphC}x^2} 
            % \)}
            
            % Plot from equation
            \addplot[
                name path={A},
                smooth,
                thick,
                Graph,
                domain={-2.5:.5},
                samples = \mysampledensitySimple,
            ]{ 3-3*x^2 };
            % Plot from equation
            \addplot[
                name path={B},
                smooth,
                thick,
                GraphC,
                domain={-2.5:.5},
                samples = \mysampledensityDouble,
            ]{ 6*x };
            % % ================ fillbetween =============== %
            % \addplot [
            %     fill opacity=0.1,
            %     foreground,
            % ] fill between [
            %     of={A and B},
            %     soft clip={domain={0:1}},
            % ];
            
        \end{axis}
        \end{tikzpicture}
    \end{center}
\end{questionBox}

\begin{questionBox}1{ % Q5
} % Q5
    \answer{D.}
\end{questionBox}

\begin{questionBox}1{ % Q6
    O valor do limite é
} % Q6
    \begin{BM}
        \lim_{(x,y)\to(1,-1)}{
            \frac{
                (x-1)(y+1)\,e^{(x-1)^2}
            }{
                \sqrt{
                    (x-1)^2+(y+1)^2
                }
            }
        }
    \end{BM}
    \answer{D.}
\end{questionBox}

\begin{questionBox}1{ % Q7
    Considere a função real de duas variáveis reais, definida por
} % Q7
    \begin{BM}
        f(x,y)
        =\begin{cases}
            \frac{3\,x^2\,y+2\,x^5}{x^4+y^2}
            \quad&\text{se }(x,y)\neq(0,0)
            \\0
            \quad&\text{se }(x,y)=(0,0)
        \end{cases}
    \end{BM}

    Relativamente a \textit{f} temos:
    \answer{C.}

    \begin{flalign*}
        &
            \pdv{f}{x}(0,0)
            =\pdv{}{x}{(0,0)}
            \left(
                \frac{3\,x^2\,y+2\,x^5}{x^4+y^2}
            \right)
            = &\\&
            =\pdv{}{x}{(0,0)}
            \left(
                \frac{3\,x^2\,y}{x^4+y^2}
                + \frac{2\,x^5}{x^4+y^2}
            \right)
            = &\\&
            =3\,y\pdv{}{x}{(0,0)}
            \left(
                \frac{x^2}{x^4+y^2}
            \right)
            + 2\pdv{}{x}{(0,0)}
            \left(
                \frac{x^5}{x^4+y^2}
            \right)
            = &\\&
            =\left\{
                \begin{aligned}
                    & 
                        3\,y
                        \frac{
                            \left(
                                \pdv{}{x}{(0,0)}(x^2)(x^4+y^2)
                                - \pdv{}{x}{(0,0)}(x^4+y^2)(x^2)
                            \right)
                        }
                        {(x^4+y^2)^2}
                    &+\\+&
                        2\pdv{}{x}{(0,0)}
                        \frac{
                            \left(
                                \pdv{}{x}{(0,0)}(x^5)(x^4+y^2)
                                - \pdv{}{x}{(0,0)}(x^4+y^2)(x^5)
                            \right)
                        }{(x^4+y^2)^2}
                    &
                \end{aligned}
            \right\}
            = &\\&
            =\left\{
                \begin{aligned}
                    & 
                        3\,y
                        \frac{
                            \left(
                                2\,x\,(x^4+y^2)
                                - 4\,x^3\,(x^2)
                            \right)
                        }
                        {(x^4+y^2)^2}
                    &+\\+&
                        2\pdv{}{x}{(0,0)}
                        \frac{
                            \left(
                                5\,x^4(x^4+y^2)
                                - 5\,x^3(x^5)
                            \right)
                        }{(x^4+y^2)^2}
                    &
                \end{aligned}
            \right\}
            = &\\&
            =\left\{
                \begin{aligned}
                    & 
                        3\,y
                        \frac{
                            +2\,x\,y^2
                            -2\,x^5
                        }
                        {(x^4+y^2)^2}
                    &+\\+&
                        \frac{
                            10\,x^4\,y^2
                        }{(x^4+y^2)^2}
                    &
                \end{aligned}
            \right\}
            = &\\&
            =
            \frac{6\,x\,y^3}
            {
                x^8
                +2x^4\,y^2
                + y^4
            }
            -\frac{2\,x^5\,y}
            {
                x^8
                +2x^4\,y^2
                + y^4
            }
            + \frac{5\,x^4\,y^2}
            {
                x^8
                +2x^4\,y^2
                + y^4
            }
            ; &\\&
            \lim_{(x\to0),(y\to x^2)}{
                \frac{
                    6\,x\,y^3
                    -2\,x^5\,y
                    +5\,x^4\,y^2
                }
                {
                    x^8
                    +2x^4\,y^2
                    + y^4
                }
            }
            = &\\&
            = \lim_{(x\to0)}{
                \frac{
                    6\,x\,x^6
                    -2\,x^5\,x^2
                    +5\,x^4\,x^4
                }
                {
                    x^8
                    +2x^4\,x^4
                    + x^8
                }
            }
            = &\\&
            = \lim_{(x\to0)}{
                \frac{
                    6\,x^7
                    -2\,x^7
                    +5\,x^8
                }
                {
                    4x^8
                }
            }
            = &\\&
            = \lim_{(x\to0)}{
                \frac{
                    6
                    -2
                    +5\,x
                }
                {
                    4x
                }
            }
        &
    \end{flalign*}
\end{questionBox}

\begin{minipage}{1\textwidth}
    \question{D.} % Q8
    \question{C.} % Q9
    \question{B.} % Q10
    \question{D.} % Q11
\end{minipage}

\begin{questionBox}1{ % Q12
    Considere a função \textit{f}. Escolha a afirm correta
} % Q12
    \begin{BM}
        f(x,y)
        = x^3+x^2\,y-y^2-4\,y
    \end{BM}

    \answer{E.}
    \begin{flalign*}
        &
            \begin{cases}
                \pdv{f}{x}
                = 3\,x^2+2\,x\,y
                \\
                \pdv{f}{y}
                = x^2-2\,y-4
            \end{cases}
            &\\&
            \det\hessiana{f(x,y)}
            = \begin{pmatrix}
                \pdv{f(x,y)}{x,x} 
                & \pdv{f(x,y)}{x,y}
                \\
                \pdv{f(x,y)}{y,x} 
                & \pdv{f(x,y)}{y,y}
            \end{pmatrix}
            = \begin{pmatrix}
                6\,x+2\,y & 2\,x
                \\
                2\,x & -2
            \end{pmatrix}
            = &\\&
            = -2(6\,x+2\,y)
            -4\,x^2
            = -4\,x^2
            -12\,x
            +4\,y
            ; &\\[3ex]&
            \left\{
                \begin{aligned}
                    \det\hessiana(f)(0,-2)
                    & = -8
                    \quad& \text{Ponto de sela}
                    \\ \det\hessiana(f)(-4,6)
                    & = 8
                    \quad& \text{Critico local}
                    \\ \det\hessiana(f)(1,-3/2)
                    & = -10
                    \quad& \text{Ponto de sela}
                    \\ \det\hessiana(f)(0,0)
                    &= 0
                    \quad& \text{Indeterminável}
                \end{aligned}
            \right\}
            &\\&
            \pdv[order=2]{f}{x}(1,-3/2)
            = 6+2\,(-3/2)
            = 6-3
            = 3
            \therefore\text{Mínimo local}
        &
    \end{flalign*}
\end{questionBox}

\begin{questionBox}1{ % Q13
    Seja \textit{D}, calcule o integral
} % Q13
    \begin{BM}
        D = \left\{
            (x,y)\in\mathbb{R}^2
            : y\leq x\leq 2-y^2
            \land y\geq0
        \right\}
        \\
        \iint_D{4\,y\,\odif{x,y}}
    \end{BM}
    \answer{B.}
    \begin{flalign*}
        &
            \begin{cases}
                y\leq x
                \\
                y^2\geq 2-x
                \\
                y\geq0
            \end{cases}
            &\\&
            x^2=2-x
            \implies
            x^2+x-2=0
            \implies &\\&
            \implies
            \left\{
                \begin{aligned}
                    a=1/4
                    \\
                    x'=-1/2
                    \\
                    y'=-9/4
                \end{aligned}
            \right\}
            \implies
            (x+1/2)^2=9/4
            \implies &\\&
            \implies
            x=\pm(3/2)-1/2;
            X=\{(1,1),(-2,-2)\}
        &
    \end{flalign*}
    \begin{center}
        % \tikzset{external/remake next=true}
        % \pgfplotsset{height=7cm, width= .6\textwidth}
        \begin{tikzpicture}
        \begin{axis}
            [
                % xmajorgrids = true,
                legend pos={south east},
                % domain=0:4,
                % xlabel={},
                % ylabel={},
                axis lines={center}, % left|right|center|box
            ]
            % Legends
            % \addlegendimage{empty legend}
            % \addlegendentry[Graph]{\( 
            %     {\color{Graph}x},
            %     {\color{GraphC}x^2} 
            % \)}
            
            % Plot from equation
            \addplot[
                name path={A},
                smooth,
                thick,
                Graph,
                domain={-0.1:1.1},
                samples = \mysampledensitySimple,
            ]{ x };
            % Plot from equation
            \addplot[
                name path={B},
                smooth,
                thick,
                GraphC,
                domain={0.9:2},
                samples = 10\mysampledensitySimple,
            ]{ sqrt(2-x) };
            
        \end{axis}
        \end{tikzpicture}
    \end{center}
    \begin{flalign*}
        &
            \int_{0}^{1}{
                \int_{y}^{2-y^2}{
                    4\,y\,\odif{x,y}
                }
            }
            = \int_{0}^{1}{
                4\,y((2-y^2)-y)\,\odif{y}
            }
            = &\\&
            = \adif(8\,y^2/2-4\,y^4/4-4\,y^3/3)\big\vert_0^1
            = 4-1-4/3
            = 5/3
        &
    \end{flalign*}
\end{questionBox}

\begin{questionBox}1{ % Q14
    Seja \textit{f} uma função contínua em \(\mathbb{R}^2\). Considere a igualdade
} % Q14
    \begin{BM}
        \iint_{\mathcal{R}}{
            f(x,y)\,\odif{x,y}
        }
        = \left\{
            \begin{aligned}
                &
                    \int_{0}^{1}{
                        \int_{-\sqrt{x}}^{\sqrt{x}}{
                            f(x,y)\,\odif{y,x}
                        }
                    }
                &+\\+&
                    \int_{1}^{\sqrt{2}}{
                        \int_{-\sqrt{2-x^2}}^{\sqrt{2-x^2}}{
                            f(x,y)\,\odif{y,x}
                        }
                    }
                &
            \end{aligned}
        \right\}
    \end{BM}
    
    \answer{C.}
    \begin{flalign*}
        &
            \left\{
                \begin{aligned}
                    x \leq y^2
                    \quad& x\in\myrange{0,1}
                    \\
                    2-x^2\leq y^2
                    \quad& x\in\myrange{1,\sqrt{2}}
                \end{aligned}
            \right\}
            = \left\{
                \begin{aligned}
                    x \leq y^2
                    \quad& y\in\myrange{0,1}
                    \\
                    x^2\leq 2-y^2
                    \quad& y\in\myrange{0,1}
                \end{aligned}
            \right\}
            % ; &\\[3ex]&
            % 0=x^2+x-2
            % \implies &\\&
            % \implies
            % \left\{
            %     \begin{aligned}
            %         a=1/4
            %         \\
            %         x'=-1/2
            %         \\
            %         y'=-9/4
            %     \end{aligned}
            % \right\}
            % \implies
            % (x+1/2)^2=9/4
            % \implies &\\&
            % \implies
            % x=\pm(3/2)-1/2;
            % X=\{(1,1),(-2,-2)\}
        &
    \end{flalign*}
    \begin{center}
        % \tikzset{external/remake next=true}
        % \pgfplotsset{height=7cm, width= .6\textwidth}
        \begin{tikzpicture}
        \begin{axis}
            [
                % xmajorgrids = true,
                legend pos={south east},
                % domain=0:4,
                % xlabel={},
                % ylabel={},
                axis lines={left}, % left|right|center|box
            ]
            % Legends
            % \addlegendimage{empty legend}
            % \addlegendentry[Graph]{\( 
            %     {\color{Graph}x},
            %     {\color{GraphC}x^2} 
            % \)}
            
            % Plot from equation
            \addplot[
                name path={B},
                smooth,
                thick,
                Graph,
                domain  = 1:1.5,
                samples = 50\mysampledensitySimple,
            ]{ sqrt(2-x^2) };
            % Plot from equation
            \addplot[
                name path={B},
                smooth,
                thick,
                Graph,
                domain  = 1:1.5,
                samples = 50\mysampledensitySimple,
            ]{ -sqrt(2-x^2) };
            % Plot from equation
            \addplot[
                name path={B},
                smooth,
                thick,
                GraphC,
                domain  = 0:1.1,
                samples = 50\mysampledensitySimple,
            ]{ sqrt(x) };
            % Plot from equation
            \addplot[
                name path={B},
                smooth,
                thick,
                GraphC,
                domain  = 0:1.1,
                samples = 50\mysampledensitySimple,
            ]{ -sqrt(x) };
            
        \end{axis}
        \end{tikzpicture}
    \end{center}
    \begin{flalign*}
        &
            \therefore \int_{-1}^{1}{
                \int_{y^2}^{\sqrt{2-y^2}}{
                    f(x,y)\odif{x,y}
                }
            }
        &
    \end{flalign*}
\end{questionBox}

\begin{questionBox}1{ % Q15
    Considere o domínio de \(\mathbb{R}^2\) definido por
} % Q15
    \begin{BM}
        D=\left\{
            (x,y)\in\mathbb{R}^2
            : x\geq0,y\geq0,x+y\leq1
        \right\}
    \end{BM}
    Usando transformação de variáveis:
    \begin{BM}
        \begin{cases}
            y-x=u
            \\
            x+y=v
        \end{cases}
    \end{BM}

    Determine:
    \begin{BM}
        I = \iint_D{3(y-x)^2\sqrt{y+x}\,\odif{x,y}}
    \end{BM}

    \answer{D.}
    \begin{flalign*}
        &
            \begin{cases}
                x\geq0
                \\
                y\geq0
                \\
                x+y\leq1
            \end{cases}
            &\\&
            X=\left\{
                (0,0),
                (0,1),
                (1,0)
            \right\}
        &
    \end{flalign*}
    \begin{center}
        % \tikzset{external/remake next=true}
        % \pgfplotsset{height=7cm, width= .6\textwidth}
        \begin{tikzpicture}
        \begin{axis}
            [
                % xmajorgrids = true,
                legend pos={north east},
                % domain=0:4,
                % xlabel={},
                % ylabel={},
                xmin={-0.1},
                xmax={1.1},
                ymin={-0.1},
                ymax={1.1},
                axis lines={center}, % left|right|center|box
            ]
            % Legends
            \addlegendimage{empty legend}
            \addlegendentry{\( 
                {\color{Graph31}C_1},
                {\color{Graph32}C_2}, 
                {\color{Graph33}C_3}, 
            \)}
            
            \draw[Graph31,thick] (0,0) -- (0,1);
            \draw[Graph32,thick] (0,0) -- (1,0);
            % Plot from equation
            \addplot[
                name path={A},
                smooth,
                thick,
                Graph33,
                domain  = 0:1,
                samples = \mysampledensitySimple,
            ]{ 1-x };
            
        \end{axis}
        \end{tikzpicture}
    \end{center}
    \begin{flalign*}
        &
            \left\{
                \begin{aligned}
                    x-y=u
                    \\
                    x+y=v
                \end{aligned}
            \right\}
            = \left\{
                \begin{aligned}
                    x=(u+v)/2
                    \\
                    y=(v-u)/2
                \end{aligned}
            \right\}
            &\\&
            \left\{
                \begin{aligned}
                    C_1 & = \{x=0,y\in\myrange{0,1}\},
                    \\
                    C_2 & = \{y=0,x\in\myrange{0,1}\},
                    \\
                    C_3 & = \{y=1-x,x\in\myrange{0,1}\},
                \end{aligned}
            \right\}
            ; &\\[6ex]&
            % ================== C_{1,*} ================= %
            C_{1,*} 
            = \{u+v=0,u-v\in\myrange{0,1}\}
            = \{u=-v,v\in\myrange{-1/2,0}\}
            ; &\\[3ex]&
            % ================== C_{2,*} ================= %
            C_{2,*}
            = \left\{ u-v=0,u+v\in\myrange{0,1} \right\}
            = \left\{ u=v,v\in\myrange{0,1/2} \right\}
            ; &\\[3ex]&
            % ================== C_{3,*} ================= %
            C_{3,*}
            = \left\{ u-v=1-u-v,u+v\in\myrange{0,1} \right\}
            = \left\{ u=1/2,1/2+v\in\myrange{0,1} \right\}
            = &\\&
            = \left\{ u=1/2,v\in\myrange{-1/2,1/2} \right\}
            &\\[3ex]&
            \implies
            \left\{
                \begin{aligned}
                    C_{1,*} & = \{u=-v,v\in\myrange{-1/2,0}\},
                    \\
                    C_{2,*} & = \{u=v,v\in\myrange{0,1/2}\},
                    \\
                    C_{3,*} & = \{ u=1/2,v\in\myrange{-1/2,1/2}\}
                \end{aligned}
            \right\}
        &
    \end{flalign*}

    \begin{center}
        % \tikzset{external/remake next=true}
        % \pgfplotsset{height=7cm, width= .6\textwidth}
        \begin{tikzpicture}
        \begin{axis}
            [
                % xmajorgrids = true,
                legend pos={south east},
                % domain=0:4,
                % xlabel={},
                % ylabel={},
                xmin={-0.6},
                xmax={0.6},
                ymin={-0.1},
                ymax={0.6},
                axis lines={center}, % left|right|center|box
            ]
            % Legends
            \addlegendimage{empty legend}
            \addlegendentry{\( 
                {\color{Graph31}C_{1,*}},
                {\color{Graph32}C_{2,*}},
                {\color{Graph33}C_{3,*}} 
            \)}
            
            % Plot from equation
            \addplot[
                name path={A},
                smooth,
                thick,
                Graph31,
                domain={-.5:0},
                samples = \mysampledensitySimple,
            ]{ -x };
            % Plot from equation
            \addplot[
                name path={A},
                smooth,
                thick,
                Graph32,
                domain={0:.5},
                samples = \mysampledensitySimple,
            ]{ x };

            \draw[Graph33,thick]
            (-0.5,.5) -- (0.5,.5);
            
        \end{axis}
        \end{tikzpicture}
    \end{center}

    \begin{flalign*}
        &
            % ================= Jacobiana ================ %
            &\\[6ex]&
            \det\jacobiana{f}
            =\begin{pmatrix}
                \pdv{x}{u} & \pdv{x}{v}
                \\
                \pdv{y}{u} & \pdv{y}{v}
            \end{pmatrix}
            =\begin{pmatrix}
                1/2 & 1/2
                \\
                -1/2 & 1/2
            \end{pmatrix}
            = 1/4+1/4 = 1/2
            ; &\\[3ex]&
            I 
            = \iint_D{3(y-x)^2\sqrt{y+x}\,\odif{x,y}}
            = \int_{0}^{0.5}{
                \int_{-v}^{v}{
                    3\,u^2\sqrt{v}
                    (1/2)
                    \odif{u,v}
                }
            }
            = &\\&
            = \int_{0}^{0.5}{
                (3/2)\,(v^3+v^3)(1/3)\sqrt{v}
                \odif{v}
            }
            = (1/2)\int_{0}^{0.5}{
                v^{7/2}
                \odif{v}
            }
            = \frac{(1/2)(0.5)^{9/2}}{9/2}
            = \frac{(0.5)^{9/2}}{9}
        &
    \end{flalign*}
\end{questionBox}

\begin{questionBox}1{ % Q16
    Denote por \(\lambda\) a área do domínio \textit{D} no plano, Tem-se:
} % Q16
    \begin{BM}
        D = \left\{
            (x,y)\in\mathbb{R}^2
            : x^2+y^2\geq1
            \land
            x^2+y^2\leq2\,x
        \right\}
    \end{BM}

    \answer{D.}
    \begin{flalign*}
        &
            \left\{
                \begin{aligned}
                    y^2=1-x^2
                    \\
                    y^2=2\,x-x^2
                \end{aligned}
            \right\}
            ; &\\[3ex]&
            1-x^2=2\,x-x^2
            \implies
            1/2=x
            \implies
            X'=\left\{
                (1/2, \sqrt{3}/2),
                (1/2,-\sqrt{3}/2)
            \right\}
        &
    \end{flalign*}
    \begin{center}
        % \tikzset{external/remake next=true}
        % \pgfplotsset{height=7cm, width= .6\textwidth}
        \begin{tikzpicture}
        \begin{axis}
            [
                % xmajorgrids = true,
                legend pos={south east},
                % domain=0:4,
                % xlabel={},
                % ylabel={},
                axis lines={center}, % left|right|center|box
            ]
            % Legends
            % \addlegendimage{empty legend}
            % \addlegendentry[Graph]{\( 
            %     {\color{Graph}x},
            %     {\color{GraphC}x^2} 
            % \)}
            
            % Plot from equation
            \addplot[
                name path={A},
                smooth,
                thick,
                Graph,
                domain  = .4:1,
                samples = 50\mysampledensitySimple,
            ]{ sqrt(1-x^2) };
            % Plot from equation
            \addplot[
                name path={A},
                smooth,
                thick,
                Graph,
                domain  = .4:1,
                samples = 50\mysampledensitySimple,
            ]{ -sqrt(1-x^2) };

            % Plot from equation
            \addplot[
                name path={B},
                smooth,
                thick,
                GraphC,
                domain  = .4:2,
                samples = 50\mysampledensitySimple,
            ]{ sqrt(2*x-x^2) };
            \addplot[
                name path={B},
                smooth,
                thick,
                GraphC,
                domain={0.4:2},
                samples = 50\mysampledensitySimple,
            ]{ -sqrt(2*x-x^2) };
            
        \end{axis}
        \end{tikzpicture}
    \end{center}
    \begin{flalign*}
        &
            \begin{cases}
                x=r\cos\theta
                \\
                y=r\sin\theta
                \\
                r=\sqrt{x^2+y^2}
                \\
                \theta
                =\tan^{-1}{y/x}
            \end{cases}
            &\\&
            r^2\,\cos^2\theta
            + r^2\,\sin^2\theta
            = 2\,r\,\cos\theta
            \implies &\\&
            \implies
            r = 2\,\cos\theta
            &\\&
            \left\{
                \begin{aligned}
                    (x,y)=(1/2,\sqrt{3}/2)
                    \iff
                    (r,\theta) = (1,\pi/3)
                \end{aligned}
            \right\}
            ; &\\[3ex]&
            \lambda
            =\int_{0}^{\pi/3}{
                \int_{1}^{2\,\cos\theta}{
                    \odif{r,\theta}
                }
            }
            +\int_{\pi\,5/3}^{2\,\pi}{
                \int_{1}^{2\,\cos\theta}{
                    \odif{r,\theta}
                }
            }
        &
    \end{flalign*}
\end{questionBox}

\begin{questionBox}1{ % Q17
    Considere o sólido \(\mathcal{E}\) sabendo que tem por função de densidade \(d(x,y,z)\). a massa desse sóldio é
} % Q17
    \begin{BM}
        \mathcal{E}
        = \left\{
            (x,y,z)\in\mathbb{R}^3
            : x^2+y^2+z^2\leq9,
            z\leq\sqrt{x^2+y^2},
            z\leq0
        \right\}
        \\
        d(x,y,z) = z
    \end{BM}

    \answer{A.}
    \begin{flalign*}
        &
            \left\{
                \begin{aligned}
                       x & =\rho\sin\phi\cos\theta
                    \\ y & =\rho\sin\phi\sin\theta
                    \\ z & =\rho\cos\phi
                \end{aligned}
            \right\}
            y=0\implies
            \begin{cases}
                z^2\leq 9-x^2
                \\
                z\leq x
            \end{cases}
            &\\&
            x^2=9-x^2
            \implies
            X' = \left\{
                ( 3\sqrt{2}/2, 3\sqrt{2}/2),
                (-3\sqrt{2}/2,-3\sqrt{2}/2)
            \right\}
        &
    \end{flalign*}
    \begin{center}
        % \tikzset{external/remake next=true}
        % \pgfplotsset{height=7cm, width= .6\textwidth}
        \begin{tikzpicture}
        \begin{axis}
            [
                % xmajorgrids = true,
                legend pos={south east},
                % domain=0:4,
                % xlabel={},
                % ylabel={},
                axis lines={center}, % left|right|center|box
            ]
            % Legends
            % \addlegendimage{empty legend}
            % \addlegendentry[Graph]{\( 
            %     {\color{Graph}x},
            %     {\color{GraphC}x^2} 
            % \)}
            
            % Plot from equation
            \addplot[
                name path={A},
                smooth,
                thick,
                Graph,
                domain  = 0:3,
                samples = 50\mysampledensitySimple,
            ]{ sqrt(9-x^2) };
            % Plot from equation
            \addplot[
                name path={B},
                smooth,
                thick,
                GraphC,
                domain  = 0:2.2,
                samples = \mysampledensityDouble,
            ]{ x };
            
        \end{axis}
        \end{tikzpicture}
    \end{center}
\end{questionBox}

\begin{minipage}{1\textwidth}
    \question{D.} % Q18
    \question{C.} % Q19
    \question{D.} % Q20
\end{minipage}

\begin{questionBox}1{ % Q21
    Volume do sólido
} % Q21
    \begin{BM}
        E=\left\{
            (x,y,z)
            \in\mathbb{R}^3
            :0\leq z\leq x^2+y^2
            \land
            x^2+y^2\leq 1
            \land
            y\geq0
        \right\}
    \end{BM}
    \answer{D.}
    \begin{flalign*}
        &
            y=0\implies
            \begin{cases}
                z\geq0
                \\
                z\leq x^2
                \\
                x^2\leq 1
                \implies x\in\myrange{-1,1}
                \\
                y\geq 0
            \end{cases}
        &
    \end{flalign*}
    \begin{center}
        % \tikzset{external/remake next=true}
        % \pgfplotsset{height=7cm, width= .6\textwidth}
        \begin{tikzpicture}
        \begin{axis}
            [
                % xmajorgrids = true,
                legend pos={south east},
                % domain=0:4,
                % xlabel={},
                % ylabel={},
                axis lines={center}, % left|right|center|box
            ]
            % Legends
            % \addlegendimage{empty legend}
            % \addlegendentry[Graph]{\( 
            %     {\color{Graph}x},
            %     {\color{GraphC}x^2} 
            % \)}

            % Plot from equation
            \addplot[
                name path={B},
                smooth,
                thick,
                Graph31,
                domain  = -1:1,
                samples = \mysampledensitySimple,
            ]{ x^2 };
            
        \end{axis}
        \end{tikzpicture}
        % \tikzset{external/remake next=true}
        \begin{tikzpicture}
        \begin{axis}
            [
                % xmajorgrids = true,
                legend pos={south east},
                % domain=0:4,
                % xlabel={},
                % ylabel={},
                axis lines={center}, % left|right|center|box
            ]
            % Legends
            % \addlegendimage{empty legend}
            % \addlegendentry[Graph]{\( 
            %     {\color{Graph}x},
            %     {\color{GraphC}x^2} 
            % \)}

            % Plot from equation
            \addplot[
                name path={B},
                smooth,
                thick,
                Graph32,
                domain  = 0:1,
                samples = \mysampledensitySimple,
            ]{ x^2 };
            
        \end{axis}
        \end{tikzpicture}
        % \tikzset{external/remake next=true}
        \begin{tikzpicture}
        \begin{axis}
            [
                % xmajorgrids = true,
                legend pos={south east},
                % domain=0:4,
                % xlabel={},
                % ylabel={},
                axis lines={center}, % left|right|center|box
            ]
            % Legends
            % \addlegendimage{empty legend}
            % \addlegendentry[Graph]{\( 
            %     {\color{Graph}x},
            %     {\color{GraphC}x^2} 
            % \)}

            % Plot from equation
            \addplot[
                name path={B},
                smooth,
                thick,
                Graph33,
                domain  = -1:1,
                samples = 50\mysampledensitySimple,
            ]{ sqrt(1-x^2) };
            
        \end{axis}
        \end{tikzpicture}
    \end{center}
    \begin{flalign*}
        &
            \int_{0}^{1}{
                \int_{-\sqrt{1-y^2}}^{\sqrt{1-y^2}}{
                    \int_{x^2+y^2}^{1}{
                        \odif{z,x,y}
                    }
                }
            }
            = &\\&
            = \int_{0}^{1}{
                \int_{-\sqrt{1-y^2}}^{\sqrt{1-y^2}}{
                    x^2\odif{x,y}
                }
                +\int_{-\sqrt{1-y^2}}^{\sqrt{1-y^2}}{
                    (y^2-1)\odif{x,y}
                }
                -\int_{-\sqrt{1-y^2}}^{\sqrt{1-y^2}}{
                    \odif{x,y}
                }
            }
            = &\\&
            = \int_{0}^{1}{
                (2/3)((\sqrt{1-y^2})^3)\odif{y}
                +(y^2-1)(2\sqrt{1-y^2})\odif{y}
                -2\sqrt{1-y^2}\odif{y}
            }
        &
    \end{flalign*}
\end{questionBox}

\begin{minipage}{1\textwidth}
    \question{A.} % Q22
    \question{A.} % Q23
    \question{B.} % Q24
    \question{C.} % Q25
\end{minipage}

\end{document}
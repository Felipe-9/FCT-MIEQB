% !TEX root = ./AM_2C-Slides_Anotacoes.3.tex
\providecommand\mainfilename{"./AM_2C-Slides_Anotacoes.tex"}
\providecommand \subfilename{}
\renewcommand   \subfilename{"./AM_2C-Slides_Anotacoes.3.tex"}
\documentclass[\mainfilename]{subfiles}

% \tikzset{external/force remake=true} % - remake all

\begin{document}

% \graphicspath{{\subfix{./.build/figures/AM_2C-Slides_Anotacoes.3}}}
% \tikzsetexternalprefix{./.build/figures/AM_2C-Slides_Anotacoes.3/graphics/}

\mymakesubfile{3}
[AM 2C]
{Slides Anotações} % Subfile Title
{Slides Anotações} % Part Title

\begin{exampleBox}1{ % E
    Determinar a reta tangente à elipse \(x^2/4+y^2=1\) no ponto \(P = (\sqrt{2},1/\sqrt{2})\)
} % E
    \answer{}
    \begin{flalign*}
        &
            \begin{cases}
                x = 2\,\cos(t)
                \\
                y = \sin{t}
            \end{cases}
            &\\&
            \vv{r}(t) 
            = 2\,\cos(t)\,\hat{\imath}
            + \sin(t)\,\hat{\jmath}
            ; &\\[3ex]&
            t
            \left\{
                \begin{aligned}
                    \sqrt{2}=2\,cos(t)
                    \implies
                    \cos(t)=\sqrt{2}/2
                    \\
                    1/\sqrt{2}=cos(t)
                    \implies\sin(t)=\sqrt{2}/2
                \end{aligned}
            \right\}
            \implies 
            t=\pi/4
            ; &\\[3ex]&
            \vv{r}'(\pi/4)
            = -2\sin(\pi/4)\,\hat{\imath}
            + \cos(\pi/4)\,\hat{\imath}
            = -\sqrt{2}\,\hat{\imath}
            + \frac{\sqrt{2}}{2}\,\hat{\imath}
        &
    \end{flalign*}
    
\end{exampleBox}

\end{document}
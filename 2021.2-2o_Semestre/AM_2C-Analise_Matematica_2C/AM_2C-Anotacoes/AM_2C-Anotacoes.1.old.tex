% !TEX root = ./AM_2C-Anotacoes.1.tex
\documentclass["./AM_2C-Anotacoes.tex"]{subfiles}

\tikzset{external/force remake=true} % - remake all

\begin{document}

% \graphicspath{{\subfix{./.build/figures/AM_2C-Anotações.1}}}
\tikzsetexternalprefix{./.build/figures/AM_2C-Anotações.1/graphics/}

\mymakesubfile{1}[AM\,2C]
{Anotações 0: Cónicas} % Subfile Title
{Cónicas} % Part Title

\begin{sectionBox}1{Cônicas}

  \subsection*{Definições}
  \begin{sectionBox}*2{Intersecção de um plano com um cone}

    % \tikzset{external/remake next=true}
    \begin{center}
      \pgfplotsset{
        height=0.45\pageheight,
        width =0.6\textwidth
      }

      \begin{tikzpicture}

        % % Axes 3D
        % \draw[-stealth] (-0.5,0,0) -- (1,0,0) node[pos=1.1]{\(x\)};
        % \draw[-stealth] (0,-0.5,0) -- (0,1,0) node[pos=1.1]{\(y\)};
        % \draw[-stealth] (0,0,-0.5) -- (0,0,1) node[pos=1.1]{\(z\)};

        \begin{axis}
          [
            % xlabel={\(x\)}, ylabel={\(y\)}, zlabel={\(z\)}
            % xmajorgrids = true,
            % legend pos  = north west,
            % 3d view,
            % perspective,
            view = {-10}{15}, % rot/elevation
            % axis lines=center, % 2D middle/box/left/right
            axis lines=center, % 3D center/box/left/right
            axis on top,
            ticks = {none},
            % z buffer = {sort}, % default|none|auto|sort|reverse x seq|reverse y seq|reverse xy seq
          ]
          % Legends
          % \addlegendimage{empty legend}
          % \addlegendentry[Red]{\( x \)}


          % Conical surface
          \addplot3 [
            % Aparence
            surf,
            opacity=0.3,
            shader=interp,
            thick,
            % Scope
            data cs = cart, % cart/polar/polarrad
            samples = \mysampledensitySimple,
            % Variable
            % variable = z,
            domain   = -1:1,
            domain y = -1:1,
          ]
          {sqrt(x^2+y^2)};
          \addplot3 [
            % Aparence
            surf,
            opacity=0.3,
            shader=interp,
            thick,
            % Scope
            data cs = cart, % cart/polar/polarrad
            samples = \mysampledensitySimple,
            % variable = z,
            domain   = -1:1, % variable
            domain y = -1:1, % y
          ]
          {-sqrt(x^2+y^2)};

          % plano do Ponto
          \addplot3 [
            % Aparence
            surf,
            opacity=0.3,
            shader=interp,
            thick,
            colormap = {cool}{
              color=(red!20!foreground);
              color=(red!40!foreground);
              color=(red!90!foreground)
            },
            % Scope
            data cs = cart, % cart/polar/polarrad
            samples = 2,
            % variable = z,
            domain   = -1:1, % variable
            domain y = -1:1, % y
          ]{
            0
          };

          % ponto
          \addplot3[mark=*,red] coordinates { (0,0,0) };

          % Reta
          \addplot3 [
            % Aparence
            surf,
            opacity=0.3,
            shader=interp,
            thick,
            colormap = {cool}{
              color=(green!20!foreground);
              color=(green!40!foreground);
              color=(green!90!foreground)
            },
            % Scope
            data cs = cart, % cart/polar/polarrad
            samples = 2,
            % variable = z,
            domain   = -1:1, % variable
            domain y = -1:1, % y
          ]{
            x
          };
          \addplot3 [
            % Aparence
            surf,
            % opacity=0.9,
            % shader=interp,
            thick,
            % green,
            colormap = {cool}{
              % color(0cm)=(green!20!foreground);
              % color(1cm)=(green!40!foreground);
              color=(green!90!foreground);
              color=(green!90!foreground)
            },
            % Scope
            data cs = cart, % cart/polar/polarrad
            samples = 2,
            variable = t,
            domain   = -1:1, % variable
            domain y = -1:1, % y
          ](
            t,0,t
          );

          % % Modulo
          % \addplot3 [
          %     % Aparence
          %     surf,
          %     opacity=0.3,
          %     shader=interp,
          %     thick,
          %     % green,
          %     colormap = {cool}{
          %         color(0cm)=(blue!20!foreground);
          %         color(1cm)=(blue!40!foreground);
          %         color(2cm)=(blue!90!foreground)
          %     },
          %     % Scope
          %     data cs = cart, % cart/polar/polarrad
          %     samples = \mysampledensitySimple,
          %     % variable = t,
          %     domain   = -1:1, % variable
          %     domain y = -1:1, % y
          % ]{
          %     x
          % };

        \end{axis}
      \end{tikzpicture}
    \end{center}

  \end{sectionBox}

  \begin{sectionBox}*2{Focos e Diretrizes}

    \begin{BM}
      \mathcal{C}on_m\subset\mathbb{R}^m
      :\lvert \vv{P\,F} \rvert
      = e \lvert \vv{P\,D} \rvert
      \\[2ex]
      \left(
        \begin{aligned}
          &
          P\in\mathcal{C}on_m
          \ldiv{}
          F\in\mathbb{R}^m
          \ldiv{}
          D\subset\mathbb{R}^m
          \ldiv{}
          e\in\mathbb{R}
          &
        \end{aligned}
      \right)
    \end{BM}

  \end{sectionBox}

\end{sectionBox}

\begin{sectionBox}2{Ecentricidade}

  Podemos classificar as conicas em grupos distintos pela ecentricidade
  \begin{BM}
    \begin{aligned}
      e=0   \quad&\implies\quad \text{Circunferencia}
      \\ 0<e<1 \quad&\implies\quad \text{Elipse}
      \\ e=1   \quad&\implies\quad \text{Parabola}
      \\ e>1   \quad&\implies\quad \text{Hipérbole}
      \\ \lim(e)=\infty\quad&\implies\quad \text{Reta}
    \end{aligned}
  \end{BM}

  % \tikzset{external/remake next=true}
  \begin{center}
    \pgfplotsset{
      height=10cm, 
      width =10cm
    }
    \begin{tikzpicture}
      \begin{axis}
        [
          % xmajorgrids = true,
          % legend pos  = north west
          % axis lines=center, % 3D center/box/left/right
          hide axis
        ]
        \draw[red] (0,0) circle (1)
        % Legends
        % \addlegendimage{empty legend}
        % \addlegendentry[Red]{\( x \)}

        % Circunferencia
        % \addplot[
        %   smooth,
        %   thick,
        %   red\Light,
        %   domain  = -1:1,
        %   samples = 0.4*\mysampledensityFancy,
        % ]{  sqrt(1-x^2) };
        % \addplot[
        %   smooth,
        %   thick,
        %   red\Light,
        %   domain  = -1:1,
        %   samples = 0.4*\mysampledensityFancy,
        % ]{  -sqrt(1-x^2) };

      \end{axis}
    \end{tikzpicture}
  \end{center}

  Para encontrar a eccentricidade de uma conica podemos usar imágens simétricas de mesma eccentricidade para que esta seja nula
  \begin{BM}
    {\lvert \vv{P\,F_1} \rvert} \Big/ {\lvert \vv{P\,D_1} \rvert}
    = {\lvert \vv{P\,F_2} \rvert} \Big/ {\lvert \vv{P\,D_2} \rvert}
  \end{BM}

  \paragraph{Notas:}
  \begin{itemize}
    \item Em parabolas temos \(e=1\) desnessitando tal ferramenta
    \item Em elipses a imagem da original é sobreposta com sua simétrica
  \end{itemize}

\end{sectionBox}

\begin{sectionBox}1{Degeneração}

  Quando o plano que intersecta a superfície cônica contem o vértice da superfície cônica está é considerada degenerada

  \begin{multicols}{2}

    \begin{sectionBox}*2{Degeneradas}
      \begin{itemize}
        \item Ponto
        \item Reta
        \item Módulo
      \end{itemize}

      % \begin{center}
      %   \pgfplotsset{height=0.3\pageheight}
      %
      %   \begin{tikzpicture}
      %
      %     % % Axes 3D
      %     % \draw[-stealth] (-0.5,0,0) -- (1,0,0) node[pos=1.1]{\(x\)};
      %     % \draw[-stealth] (0,-0.5,0) -- (0,1,0) node[pos=1.1]{\(y\)};
      %     % \draw[-stealth] (0,0,-0.5) -- (0,0,1) node[pos=1.1]{\(z\)};
      %
      %     \begin{axis}
      %       [
      %         % xlabel={\(x\)}, ylabel={\(y\)}, zlabel={\(z\)}
      %         % xmajorgrids = true,
      %         % legend pos  = north west,
      %         % 3d view,
      %         % perspective,
      %         view = {-10}{15}, % rot/elevation
      %         % axis lines=center, % 2D middle/box/left/right
      %         axis lines=center, % 3D center/box/left/right
      %         axis on top,
      %         ticks = {none},
      %         % z buffer = {sort}, % default|none|auto|sort|reverse x seq|reverse y seq|reverse xy seq
      %       ]
      %       % Legends
      %       % \addlegendimage{empty legend}
      %       % \addlegendentry[Red]{\( x \)}
      %
      %
      %       % Conical surface
      %       \addplot3 [
      %         % Aparence
      %         surf,
      %         opacity=0.3,
      %         shader=interp,
      %         thick,
      %         % Scope
      %         data cs = cart, % cart/polar/polarrad
      %         samples = \mysampledensitySimple,
      %         % Variable
      %         % variable = z,
      %         domain   = -1:1,
      %         domain y = -1:1,
      %       ]
      %       {sqrt(x^2+y^2)};
      %       \addplot3 [
      %         % Aparence
      %         surf,
      %         opacity=0.3,
      %         shader=interp,
      %         thick,
      %         % Scope
      %         data cs = cart, % cart/polar/polarrad
      %         samples = \mysampledensitySimple,
      %         % variable = z,
      %         domain   = -1:1, % variable
      %         domain y = -1:1, % y
      %       ]
      %       {-sqrt(x^2+y^2)};
      %
      %       % plano do Ponto
      %       \addplot3 [
      %         % Aparence
      %         surf,
      %         opacity=0.3,
      %         shader=interp,
      %         thick,
      %         colormap = {cool}{
      %           color=(red!20!foreground);
      %           color=(red!40!foreground);
      %           color=(red!90!foreground)
      %         },
      %         % Scope
      %         data cs = cart, % cart/polar/polarrad
      %         samples = 2,
      %         % variable = z,
      %         domain   = -1:1, % variable
      %         domain y = -1:1, % y
      %       ]{
      %         0
      %       };
      %
      %       % ponto
      %       \addplot3[mark=*,red] coordinates { (0,0,0) };
      %
      %       % Reta
      %       \addplot3 [
      %         % Aparence
      %         surf,
      %         opacity=0.3,
      %         shader=interp,
      %         thick,
      %         colormap = {cool}{
      %           color=(green!20!foreground);
      %           color=(green!40!foreground);
      %           color=(green!90!foreground)
      %         },
      %         % Scope
      %         data cs = cart, % cart/polar/polarrad
      %         samples = 2,
      %         % variable = z,
      %         domain   = -1:1, % variable
      %         domain y = -1:1, % y
      %       ]{
      %         x
      %       };
      %       \addplot3 [
      %         % Aparence
      %         surf,
      %         % opacity=0.9,
      %         % shader=interp,
      %         thick,
      %         % green,
      %         colormap = {cool}{
      %           % color(0cm)=(green!20!foreground);
      %           % color(1cm)=(green!40!foreground);
      %           color=(green!90!foreground);
      %           color=(green!90!foreground)
      %         },
      %         % Scope
      %         data cs = cart, % cart/polar/polarrad
      %         samples = 2,
      %         variable = t,
      %         domain   = -1:1, % variable
      %         domain y = -1:1, % y
      %       ](
      %         t,0,t
      %       );
      %
      %       % % Modulo
      %       % \addplot3 [
      %       %     % Aparence
      %       %     surf,
      %       %     opacity=0.3,
      %       %     shader=interp,
      %       %     thick,
      %       %     % green,
      %       %     colormap = {cool}{
      %       %         color(0cm)=(blue!20!foreground);
      %       %         color(1cm)=(blue!40!foreground);
      %       %         color(2cm)=(blue!90!foreground)
      %       %     },
      %       %     % Scope
      %       %     data cs = cart, % cart/polar/polarrad
      %       %     samples = \mysampledensitySimple,
      %       %     % variable = t,
      %       %     domain   = -1:1, % variable
      %       %     domain y = -1:1, % y
      %       % ]{
      %       %     x
      %       % };
      %
      %     \end{axis}
      %   \end{tikzpicture}
      % \end{center}
    \end{sectionBox}

    \begin{sectionBox}*2{Não Degeneradas}
      \begin{itemize}
        \item Parábolas
        \item Elípses
        \item Hipérboles
      \end{itemize}
    \end{sectionBox}

  \end{multicols}

\end{sectionBox}

\begin{sectionBox}1{Expressão geral}

  % A partir desta expressão se pode encontrar qualquer equação canônica, especificando para cada classe uma particularidade como \(z=0\) para equações bidimensionais

  \begin{BM}
    \mathcal{C}on_m\subset\mathbb{R}^n:
    \left\{
      \begin{aligned}
        &
        (\lambda^T + (p-p')^T\,A)(p-p')=k
        &
      \end{aligned}
    \right\}
    \\[2ex]
    \left(
      \begin{aligned}
        &
        \{\lambda, p'\}\subset\mathbb{R}^n
        \ldiv{}
        p\in\mathcal{C}_m
        \ldiv{}
        A\in\mathcal{M}_{n\times n}
        \ldiv{}
        k\in\mathbb{R}
        &
      \end{aligned}
    \right)
  \end{BM}

  \begin{flalign*}
    &
    \lvert \vv{P\,F} \rvert
    =e\,\lvert \vv{P\,D} \rvert
    \implies &\\&
    \implies
    \sqrt{(x-x')^2 + (y-(y'+c))^2}
    = e\,\sqrt{(x-x')^2 + (y-(y'-e\,c))^2}
    \dots
    &
  \end{flalign*}

  \begin{sectionBox}*2{Forma canônica}

    \begin{BM}
      \mathcal{C}on_m\subset\mathbb{R}^n:
      \left\{
        (\lambda^T + p^T\,A)p=k
      \right\}
      \\[2ex]
      \left(
        \begin{aligned}
          &
          \{\lambda, p'\}\subset\mathbb{R}^n
          \ldiv{}
          p\in\mathcal{C}on_m
          \ldiv{}
          A\in\mathcal{M}_{n\times n}:
          \begin{cases}
            a_{i,j}=0\forall\,i\neq j
          \end{cases}
          \ldiv{}
          k\in\mathbb{R}
          &
        \end{aligned}
      \right)
    \end{BM}

    \paragraph{Despresa:}
    \begin{itemize}
      \item Ponto central (\(p'\))
      \item Rotações (apenas matrizes \textit{A} diagonais)
    \end{itemize}

  \end{sectionBox}

  \begin{sectionBox}*2{para \(\mathbb{R}^2\) e \(p'=0_2\)}
    \begin{BM}[flalign*]
      &
      \left(
        \begin{bmatrix}
          d\\e
        \end{bmatrix}^T
        +   \begin{bmatrix}
          x\\y
        \end{bmatrix}
        \begin{bmatrix}
          a   & c/2
          \\  c/2 & b
        \end{bmatrix}
      \right)
      \begin{bmatrix}
        x\\y
      \end{bmatrix}
      =   -f
      \implies &\\&
      \implies
      a\,x^2+b\,y^2+c\,x\,y+d\,x+e\,y+f=0
      &
    \end{BM}
  \end{sectionBox}

  \begin{sectionBox}*2{Para \(\mathbb{R}^3\) e \(p'=0_3\)}
    \begin{BM}[flalign*]
      &
      \left(
        \begin{bmatrix}
          g\\h\\i
        \end{bmatrix}^T
        +   \begin{bmatrix}
          x\\y\\z
        \end{bmatrix}^T
        \,  \begin{bmatrix}
          a   & d/2 & e/2
          \\  d/2 & b   & f/2
          \\  e/2 & f/2 & c
        \end{bmatrix}
      \right)
      \,  \begin{bmatrix}
        x\\y\\z
      \end{bmatrix}
      =   -j
      \implies &\\&
      \implies
      a\,x^2
      +   b\,y^2
      +   c\,z^2
      +   d\,x\,y
      +   e\,x\,z
      +   f\,y\,z
      + &\\&
      +   g\,x
      +   h\,y
      +   i\,z
      +   j
      =   0
      &
    \end{BM}
  \end{sectionBox}

\end{sectionBox}

\begin{sectionBox}1{Espaço}

  \subsection*{Equação vetorial}
  \begin{BM}
    E_m\subset\mathbb{R}^n:
    \left\{
      P = A + \sum_{i=1}^{m}\lambda_i\,U_i
    \right\}
    \\[2ex]
    \left(
      \begin{array}{lll}
        &
        \{P,A\}\subset E
        &\land\\\land&
        \{i,n,m\}\subset\mathbb{N}
        &\land\\\land&
        m\leq n
        &
      \end{array}
    \right)
  \end{BM}

  \begin{sectionBox}*2{Equações paramétricas}
    \begin{BM}
      P_i = A_i + \sum_{j=1}^{m}\lambda_{j,i}\,U_{j,i}
    \end{BM}
  \end{sectionBox}

  \paragraph{Exemplos}
  \begin{itemize}
    \begin{multicols}{2}
      \item \(m=0\implies\) ponto
      \item \(m=1\implies\) reta
      \item \(m=2\implies\) plano
      \end{multicols}
  \end{itemize}

\end{sectionBox}

\begin{sectionBox}1{Reta}

  \subsection*{Equação vetorial}
  \begin{BM}
    r\subset\mathbb{R}^n:
    \left\{
      P=A+\lambda\,U
    \right\}
    \\[2ex]
    \left(
      \begin{aligned}
        &  
        \{P,A\}\in r
        \ldiv{}
        U\in\mathbb{R}^n\backslash\{0_{\mathbb{R}^n}\}
        \ldiv{}
        \lambda\in\mathbb{R}
        &
      \end{aligned}
    \right)
  \end{BM}

  \subsubsection*{Definem uma reta:}
  \begin{itemize}
    \begin{multicols}{2}
      \item 2 Pontos
      \item 1 Ponto e 1 vetor
      \end{multicols}
  \end{itemize}

  \paragraph{Vetor diretor:} Vetor não nulo paralelo a reta

  \subsection*{Equações cartezianas}

  \begin{sectionBox}3{Equações paramétricas}
    \begin{BM}
      P_i = A_i + \lambda\,U_i \quad\forall\,i\leq n
    \end{BM}
  \end{sectionBox}

  \begin{sectionBox}3{Equações normais}
    \begin{BM}[flalign*]
      &
      \left\{
        \begin{aligned}
          &
          P_i = A_i + \lambda\,U_i
          \ldiv{}
          P_j = A_j + \lambda\,U_j
          &
        \end{aligned}
      \right\}
      % \implies &\\&
      \implies
      \lambda 
      =   \frac{P_i - A_i}{U_i}
      =   \frac{P_j - A_j}{U_j}
      &\\&
      \left(
        \begin{aligned}
          &
          \{i,j\}\leq n
          \ldiv{}
          \{U_i,U_j\}\neq 0
          &
        \end{aligned}
      \right)
      &
    \end{BM}
  \end{sectionBox}

  \begin{sectionBox}3{Equações reduzidas}
    \begin{BM}
      r\subset\mathbb{R}^n:
      \left\{
        P_i = A'_i + P_j\,U'_i
      \right\}
      \\[2ex]
      \left(
        \begin{aligned}
          &
          \{i,j\}\in\mathbb{N}
          \ldiv{}
          \{i,j\}\leq n
          \ldiv{}
          i\neq j
          \ldiv{}
          U'_i = U_i/U_j
          \ldiv{}
          A'_i = A_i - A_j\,U'_i
          \ldiv{}
          U_j\neq 0
          &
        \end{aligned}
      \right)
    \end{BM}

    \begin{flalign*}
      &
      \left\{
        \begin{aligned}
          &
          P_i = A_i + \lambda\,U_i
          \ldiv{}
          \lambda = (P_j-A_j)/U_j : U_j\neq 0
          &
        \end{aligned}
      \right\}
      \implies &\\&
      \implies
      P_i 
      =   A_i + U_i\,(P_j-A_j)/U_j
      = &\\&
      =   A'_i + U'_i\,P_j
      :   \left\{
        \begin{aligned}
          &
          U'_i = U_i/U_j
          \ldiv{}
          A'_i = A_i - A_j\,U'_i
          \ldiv{}
          U_j\neq 0
          &
        \end{aligned}
      \right\}
      &
    \end{flalign*}

    Reduzida pois possui uma equação a menos comparando com as equações cartezianas

  \end{sectionBox}

\end{sectionBox}

\begin{sectionBox}1{Plano}

  \subsection*{Equação vetorial}
  \begin{BM}
    \pi\in\mathbb{R}^n:
    \left\{
      P = A + \sum_{j=1}^{2}\lambda_j\,U_j
    \right\}
    \\[2ex]
    \left(
      \begin{aligned}
        &
        \{P,A\}\in\pi
        \ldiv{}
        \{n,j\}\in\mathbb{N}
        \ldiv{}
        n\geq2
        \ldiv{}
        \lambda\in\mathbb{R}^2
        \ldiv{}
        U_j\in\mathbb{R}^n\,\forall\,j
        &
      \end{aligned}
    \right)
  \end{BM}

  \paragraph{Definem um plano:}
  \begin{itemize}
    \begin{multicols}{2}
      \item 3 Pontos
      \item 1 Ponto e 2 vetores não paralelos
      \item 1 Ponto e 1 Vetor
      \end{multicols}
  \end{itemize}

  \paragraph{Vetores diretores:} Vetores não nulos e não colineares paralelos ao plano.

  \subsection*{Equações cartesianas}

  \begin{multicols}{2}

    \begin{sectionBox}3{Equações paramétricas}
      \begin{BM}
        P_i = A_i + \sum_{j=1}^{2}\lambda_{j,i}\,U_{j,i}
      \end{BM}
    \end{sectionBox}

    \begin{sectionBox}3{Equação geral}
      \begin{BM}
        A + \sum_{i=1}^{n}\lambda_i\,P_i = 0
      \end{BM}
    \end{sectionBox}

  \end{multicols}

\end{sectionBox}

\begin{sectionBox}1{Parabolas e Paraboloides}

  \begin{multicols}{2}\centering

    \pgfplotsset{
      height = 0.3\pageheight,
      width  = 0.6\textwidth, 
      % title style = {at={(0.5,0)}, anchor={north}},
    }
    % \begin{figure}[h]
    % \centering
    \begin{tikzpicture}
      \begin{axis}
        [
          % xlabel={\(x\)}, ylabel={\(y\)}, zlabel={\(z\)}
          % xmajorgrids = true,
          % legend pos  = north west,
          % 3d view,
          % perspective,
          % view = {10}{10}, % rot/elevation
          hide axis,
          % title = {Paraboloide},
          z buffer = {sort}, % default|none|auto|sort|reverse x seq|reverse y seq|reverse xy seq
        ]
        % Legends
        % \addlegendimage{empty legend}
        % \addlegendentry[Red]{\( x \)}

        % Conical surface
        \addplot3 [
          % Aparence
          surf,
          opacity=0.5,
          fill opacity= 1,
          faceted color = foreground,
          shader=faceted interp,
          % thick,
          % Scope
          data cs = cart, % cart/polar/polarrad
          samples = \mysampledensitySimple,
          % Variable
          % variable = z,
          domain   = -1:1,
          domain y = -1:1,
        ]{
          x^2+y^2
        };
      \end{axis}
    \end{tikzpicture}
    Elíptico
    % \caption{Elíptico}
    % \end{figure}

    % \begin{figure}
    % \centering

    \begin{tikzpicture}
      \begin{axis}
        [
          % xlabel={\(x\)}, ylabel={\(y\)}, zlabel={\(z\)}
          % xmajorgrids = true,
          % legend pos  = north west,
          % 3d view,
          % perspective,
          % view = {10}{10}, % rot/elevation
          hide axis,
          % title = {Paraboloide Hiperbólico},
          z buffer = {sort}, % default|none|auto|sort|reverse x seq|reverse y seq|reverse xy seq
        ]
        % Legends
        % \addlegendimage{empty legend}
        % \addlegendentry[Red]{\( x \)}

        % Conical surface
        \addplot3 [
          % Aparence
          surf,
          opacity={0.5},
          fill opacity={1},
          faceted color={foreground},
          shader={faceted interp},
          % thick,
          % Scope
          data cs={cart}, % cart/polar/polarrad
          samples={\mysampledensitySimple},
          % Variable
          % variable = z,
          domain   ={-1:1},
          domain y ={-1:1},
        ]{
          x^2-y^2
        };

      \end{axis}
    \end{tikzpicture}
    Hiperbólico
    %     \caption{Hiperbólico}
    % \end{figure}
  \end{multicols}

  \section*{Definição}
  \begin{BM}
    \mathcal{P}ar_m\subset\mathcal{C}on_m
    : e = 1
  \end{BM}
  % \begin{BM}
  %     \mathcal{P}ar_m\subset\mathcal{C}on_m:
  %     \lvert\vv{p\,f}\rvert
  %     = \lvert\vv{p\,L}\rvert
  %     \\[2ex]
  %     \left(
  %         \begin{aligned}
  %         &
  %             p\in \mathcal{P}ar
  %         \ldiv{}
  %             f\in\mathbb{R}^m
  %         \ldiv{}
  %             L \subset \mathbb{R}^m
  %         \ldiv{}
  %             L \in E_{m-1}
  %         \ldiv{}
  %             \{m\}\in\mathbb{N}
  %         \ldiv{}
  %             2 \leq m
  %         &
  %         \end{aligned}
  %     \right)
  % \end{BM}

  Um conjunto que consiste em todos os pontos em um plano equidistantes de um determinado ponto fixo e uma determinada linha fixa no plano é uma parábola. O ponto fixo é o foco da parábola. A linha fixa é a diretriz.

  % \begin{BM}
  %     P\subset\mathbb{R}^n
  %     \left\{
  %         \begin{aligned}
  %         &   
  %             \left\{
  %                 \begin{aligned}
  %                 &
  %                     |\vv{p\,f}|=|\vv{p\,L}|
  %                 \land\\\land&
  %                     p\in P
  %                 \land\\\land&
  %                     f\in\mathbb{R}^n
  %                 \land\\\land&
  %                     L\subset\mathbb{R}^n:L \text{ é um espaço em R}
  %                 \end{aligned}
  %             \right\}
  %         \lor\\\lor&
  %             \left\{
  %                 \begin{aligned}
  %                 &   p_n - p'_n=\sum_{i=1}^{n-1}\left(\frac{p_i-p'_i}{\lambda_i}\right)^2
  %                 \land\\\land&
  %                     \{p',\lambda\}\subset\mathbb{R}^n
  %                 \land\\\land&
  %                     p\in P
  %                 \end{aligned}
  %             \right\}
  %         \end{aligned}
  %     \right\}
  % \end{BM}

  % \begin{sectionBox}*2{Paraboloide Hiperbólico}

  %     \begin{BM}[flalign*]
  %     &
  %         \mathcal{P}ar_\mathcal{H}\subset\mathbb{R}^3:
  %     &\\&
  %         \left\{
  %             \begin{aligned}
  %             &
  %                 (p-p')^T + A(p-p') = 0
  %             \ldiv{}
  %                 A \in\mathcal{M}_{n\times n}:
  %                 \begin{cases}
  %                     a_{i,j}=\lambda'_i/\lambda_i\,&\forall\, i =j
  %                 \\  a_{i,j}=0\,&\forall\, i\neq j
  %                 \end{cases}
  %             &
  %             \end{aligned}
  %         \right\}
  %     &
  %     \end{BM}

  %     \begin{center}
  %         \pgfplotsset{height=0.3\pageheight}

  %             \begin{tikzpicture}\begin{axis}
  %             [
  %                 % xlabel={\(x\)}, ylabel={\(y\)}, zlabel={\(z\)}
  %                 % xmajorgrids = true,
  %                 % legend pos  = north west,
  %                 % 3d view,
  %                 % perspective,
  %                 % view = {10}{10}, % rot/elevation
  %                 hide axis,
  %             ]
  %                 % Legends
  %                 % \addlegendimage{empty legend}
  %                 % \addlegendentry[Red]{\( x \)}

  %                 % Conical surface
  %                 \addplot3 [
  %                     % Aparence
  %                     surf,
  %                     opacity=0.5,
  %                     fill opacity= 1,
  %                     faceted color = foreground,
  %                     shader=faceted interp,
  %                     % thick,
  %                     % Scope
  %                     data cs = cart, % cart/polar/polarrad
  %                     samples = \mysampledensitySimple,
  %                     % Variable
  %                     % variable = z,
  %                     domain   = -1:1,
  %                     domain y = -1:1,
  %                 ]{
  %                     x^2-y^2
  %                 };

  %             \end{axis}
  %         \end{tikzpicture}
  %     \end{center}

  % \end{sectionBox}

\end{sectionBox}

\begin{sectionBox}2{Parabolas em \(\mathbb{R}^2\)}

  \begin{BM}
    \mathcal{P}ar_2 \subset\mathcal{C}on_2
    : (\lambda^T + (p-p')A)(p-p') = k
    \implies \\
    \implies
    \left(
      \begin{bmatrix}
        0\\-1
      \end{bmatrix}^T
      + \begin{bmatrix}
        x - x' \\ y - y'
      \end{bmatrix}^T
      \begin{bmatrix}
        \frac{1}{4\,a} & 0
        \\  0 & 0
      \end{bmatrix}
    \right)
    \begin{bmatrix}
      x - x' \\ y - y'
    \end{bmatrix}
    =   0
    \implies \\
    \implies
    \left\{
      \begin{aligned}
        & y = \frac{(x-x')^2}{4\,a} + y'
        \\ & \lor 
        \\ & x = \sqrt{4\,a(y-y')} + x'
      \end{aligned}
    \right\}
    \implies \\
    \implies
    y 
    = x^2\, \left( \frac{1}{4\,a} \right)
    + x\,   \left( \frac{-x'}{2\,a} \right)
    + \left(
      \frac{x'^2}{4\,a} 
      + y'
    \right)
    \\[2ex]
    \begin{aligned}
      f \in     \mathbb{R}^2 &: (x', y'+a)
      \\ L \subset \mathbb{R}^2 &: y =  y'-a
    \end{aligned}
  \end{BM}

  \begin{sectionBox}*2{Demonstração}
    \begin{flalign*}
      &
      \lvert\vv{p\,f}\rvert
      = \sqrt{(x-x')^2 + (y-(a+y'))^2}
      = &\\&
      = \lvert\vv{p\,L}\rvert
      = \sqrt{(x-x)^2 + (y-(-a+y'))^2}
      \implies &\\&
      \implies
      (x-x')^2 + ((y-y')-a)^2
      = (x-x')^2 
      + (y-y')^2
      - 2\,a(y-y')
      + a^2
      = &\\&
      = ((y-y')+a)^2
      = (y-y')^2
      + 2\,a(y-y')
      + a^2
      \implies &\\&
      \implies
      y = \frac{(x-x')^2}{4\,a} + y'
      \lor
      x = \sqrt{4\,a(y-y')} + x'
      &
    \end{flalign*}
  \end{sectionBox}

  \begin{sectionBox}*2{Partindo da forma reduzida}
    \begin{BM}
      % &
      y 
      = \alpha\,x^2 + \beta\,x + \gamma
      \implies \\[2ex]
      \implies
      \left\{
        \begin{aligned}
          a  &= (4\,\alpha)^{-1}
          \\ x' &
          = -\beta\,2\,a 
          = -\beta/2\,\alpha
          \\ y' &
          = \gamma - x'^2/4\,a 
          = \gamma - (\beta^2/4\,\alpha)
        \end{aligned}
      \right\}
      % &
    \end{BM}
  \end{sectionBox}

\end{sectionBox}

\begin{sectionBox}2{Parabolas em \(\mathbb{R}^3\)}

  \begin{BM}
    % &
    % (\lambda^T + (p-p')^T\,A)(p-p')=k
    \mathcal{P}ar_3\subset\mathcal{C}on_3
    : (\lambda^T + (p-p')A)(p-p') = k
    \implies \\
    \implies
    \left(
      \begin{bmatrix}
        0\\0\\-1
      \end{bmatrix}^T
      + \begin{bmatrix}
        x-x'\\y-y'\\z-z'
      \end{bmatrix}^T
      \begin{bmatrix}
        4\,a & 0    & 0
        \\ 0    & 4\,b & 0
        \\ 0    & 0    & 0
      \end{bmatrix}
    \right)
    \begin{bmatrix}
      x-x'\\y-y'\\z-z'
    \end{bmatrix}
    = \\
    = 0
    % \implies \\
    \implies
    {
      z 
      = \frac{(x-x')^2}{4\,a}
      + \frac{(y-y')^2}{4\,b}
      + z'
    }
    % \implies \\
    % \implies
    % {
    %     z 
    %     = \frac{x^2}{4\,a}
    %     + \frac{-2\,x\,x'}{4\,a}
    %     + \frac{x'^2}{4\,a}
    %     + \frac{y^2}{4\,b}
    %     + \frac{-2\,y\,y'}{4\,b}
    %     + \frac{y'^2}{4\,b}
    %     + z'
    % }
    % &
  \end{BM}

  \begin{sectionBox}*2{Demonstração}
    \begin{flalign*}
      &
      p \in \left\{
        p\in\mathcal{P}ar_3:
        y = y'
      \right\}
      \implies &\\&
      \implies
      \lvert \vv{p\,f} \rvert
      = \sqrt{
        (x-x')^2
        + (y'-y')^2
        + (z-(z'+a))^2
      }
      = &\\&
      = \lvert \vv{p\,L} \rvert
      = \sqrt{
        (x-x)^2
        + (y'-y')^2
        + (z-(z'-a))^2
      }
      \implies &\\&
      \implies
      {
        (x-x')^2
        + (z-z')^2
        - 2\,a(z-z')
        + a^2
      }
      = &\\&
      = {
        (z-z')^2
        + 2\,a(z-z')
        + a^2
      }
      \implies &\\&
      \implies
      z
      = \frac{(x-x')^2}{4\,a}
      + z'
      &\\[2ex]&
      p \in \left\{
        p\in\mathcal{P}ar_3:
        x=x'
      \right\}
      \implies &\\&
      \implies
      \lvert \vv{p\,f} \rvert
      = \sqrt{
        (x'-x')^2
        + (y-y')^2
        + (z-(z'+b))^2
      }
      = &\\&
      = \lvert \vv{p\,L} \rvert
      = \sqrt{
        (x'-x')^2
        + (y-y')^2
        + (z-(z'-b))^2
      }
      \implies &\\&
      \implies
      {
        (y-y')^2
        + (z-z')^2
        - 2\,b(z-z')
        + b^2
      }
      = &\\&
      = {
        (z-z')^2
        + 2\,b(z-z')
        + b^2
      }
      \implies &\\&
      \implies
      z
      = \frac{(y-y')^2}{4\,b}
      + z'
      &\\[2ex]&
      \therefore
      z 
      = \frac{(x-x')^2}{4\,a}
      + \frac{(y-y')^2}{4\,b}
      + z'
      &
    \end{flalign*}
  \end{sectionBox}

  \paragraph{Paraboloide Hiperbólico:} Possume um dos termos negativos.
  \begin{BM}
    \mathcal{P}ar_3: b < 0 \lxor a < 0
  \end{BM}

\end{sectionBox}

\begin{sectionBox}1{Elipse e Elipsóides}

  \begin{center}
    \pgfplotsset{height=0.4\pageheight}
    \begin{tikzpicture}
      \begin{axis}[
          % xlabel={\(x\)}, ylabel={\(y\)}, zlabel={\(z\)}
          % xmajorgrids = true,
          % legend pos  = north west,
          % 3d view,
          % perspective,
          view = {0}{30}, % rot/elevation
          hide axis,
          z buffer = {sort}, % default|none|auto|sort|reverse x seq|reverse y seq|reverse xy seq
        ]
        % Legends
        % \addlegendimage{empty legend}
        % \addlegendentry[Red]{\( x \)}

        % Elipse
        \addplot3 [
          % Aparence
          surf,
          opacity = 0.5,
          fill opacity = 1,
          faceted color = foreground,
          shader = faceted interp,
          % thick,
          % Scope
          data cs = cart, % cart/polar/polarrad
          samples = 2*\mysampledensitySimple,
          % Variable
          variable   = \u,
          variable y = \v,
          domain   = -90:90,
          domain y =   0:360,
        ](
          % {-1-x*x-y*y}
          {cos(u)*cos(v)}, % x
          {cos(u)*sin(v)}, % y
          {sin(u)} % z
        );

      \end{axis}
    \end{tikzpicture}
  \end{center}

  \section*{Definição}
  \begin{BM}
    \mathcal{E}li_m\subset\mathcal{C}on_m
    : 0<e<1
    \implies \\
    \implies
    \lvert \vv{f_1\,p} \rvert
    + \lvert \vv{f_2\,p} \rvert
    = k
    % \\[2ex]
    \qquad
    \left(
      % \begin{aligned}
      % &
      k\in\mathbb{R}
      % &
      % \end{aligned}
    \right)
  \end{BM}
  % \begin{BM}
  %     % &
  %         \mathcal{E}li_m\subset\mathcal{C}on_m
  %         : \lvert \vv{f_1\,p} \rvert
  %         + \lvert \vv{f_2\,p} \rvert
  %         = k
  %         \\[2ex]
  %         \left(
  %             \begin{aligned}
  %                 &
  %                     m\in\mathbb{N}: m\geq 2
  %                 \ldiv{}
  %                     k\in\mathbb{R}
  %                 \ldiv{}
  %                     p\in\mathcal{E}li_m
  %                 \ldiv{}
  %                     \{f_1,f_2\}\subset\mathbb{R}^m
  %                 &
  %             \end{aligned}
  %         \right)
  %     % &
  % \end{BM}

  Uma elipse é o conjunto de pontos em um plano cujas distâncias de dois pontos fixos no plano têm uma soma constante. Os dois pontos fixos são os focos da elipse.

  A linha através dos focos de uma elipse é o eixo focal da elipse. O ponto no eixo a meio caminho entre os focos é o centro. Os pontos onde o eixo focal e a elipse se cruzam são os vértices da elipse.

  % Conjunto de pontos os quais a soma da distancia de qualquer ponto a ambos os focos é igual para qualquer ponto.

  \begin{sectionBox}*2{Ecentricidade}

    \vspace{-2ex}
    \begin{BM}
      e 
      = \frac{
        \lvert \vv{p'\,f} \rvert
      }{
        \max(\lvert \vv{p'\,p} \rvert)
      }
      % = \frac{c}{r_1}
      % = \sqrt{
      %     1
      %     - (r_2/r_1)^2
      % }
    \end{BM}

  \end{sectionBox}

\end{sectionBox}

\begin{sectionBox}2{Elipses em \(\mathbb{R}^2\)}

  \begin{BM}
    \mathcal{E}li_2\subset\mathcal{C}on_2
    : ((p-p')^T\,A)\,(p-p') = 1
    \implies \\
    \implies
    \left(
      \begin{bmatrix}
        x-x'\\y-y'
      \end{bmatrix}^T
      \begin{bmatrix}
        r_1^{-2} & 0
        \\  0 & r_2^{-2}
      \end{bmatrix}
    \right)
    \begin{bmatrix}
      x-x'\\y-y'
    \end{bmatrix}
    =   1
    \implies \\
    \implies
    \frac{(x-x')^2}{r_1^2}
    + \frac{(y-y')^2}{r_2^2}
    = 1
    % 
    \\[2ex]
    A \cup \{c\}\subset\mathbb{R}^+ : c<r_1>r_2
    \\ \begin{aligned}
      r_1 &= \lvert \vv{p'\,p} \rvert : p = (\max(x),y')
      \\ r_2 &= \lvert \vv{p'\,p} \rvert : p = (x',\max(y))
      \\ c   &= \lvert \vv{p'\,f} \rvert = \sqrt{r_1^2-r_2^2}
    \end{aligned}
    \\[2ex]
    f_1\in\mathbb{R}^2 : f_1 = (x'-c, y')
    \\ f_2\in\mathbb{R}^2 : f_1 = (x'+c, y')
  \end{BM}

  % \paragraph{\textit{a}} Distancia dentre o centro e o máximo x
  % \paragraph{\textit{b}} Distancia dentre o centro e o máximo y
  % \paragraph{\textit{c}} Distancia dentre o centro e um dos focos

  \begin{sectionBox}*2{Demonstração}
    \begin{flalign*}
      &
      \lvert \vv{p\,f_1} \rvert
      + \lvert \vv{p\,f_2} \rvert
      = k 
      \land 
      \{f_1,f_2\} \subset \mathbb{R}^2
      : \lvert \vv{p'\,f} \rvert = c
      \land y = y'
      \implies &\\&
      \implies
      \sqrt{
        (x - (x'-c))^2
        + (y - y')^2
      }
      + \sqrt{
        (x-(x'+c))^2
        + (y - y')^2
      }
      % = &\\&
      = 2\,r_1
      \implies &\\&
      \implies
      \frac{x^2}{r_1^2}
      + \frac{y^2}{r_1^2-c^2}
      = 1
      \land
      r_2^2 + c^2 = r_1^2
      \implies
      \frac{x^2}{r_1^2}
      + \frac{y^2}{r_2^2}
      = 1
      &
    \end{flalign*}
  \end{sectionBox}

\end{sectionBox}

\begin{sectionBox}2{Elípses em \(\mathbb{R}^3\)}

  \begin{BM}
    \mathcal{E}li_3\subset\mathcal{C}on_3
    :((p-p')^T\,A)(p-p') = 1
    \implies \\
    \implies
    \left(
      \begin{bmatrix}
        x-x'\\y-y'\\z-z'
      \end{bmatrix}^T
      \begin{bmatrix}
        r_1^{-2} & 0 & 0
        \\ 0 & r_2^{-2} & 0
        \\ 0 & 0 & r_3^{-2}
      \end{bmatrix}
    \right)
    \begin{bmatrix}
      x-x'\\y-y'\\z-z'
    \end{bmatrix}
    =   1
    \implies \\
    \implies
    \frac{(x-x')^2}{r_1^2}
    + \frac{(y-y')^2}{r_2^2}
    + \frac{(z-z')^2}{r_3^2}
    = 1
    % 
    \\[2ex] 
    A\cup\{c\}\subset\mathbb{R}^+ : c<r_1 \land r_2<r_1>r_3
    \\ \begin{aligned}
      r_1 &= \lvert \vv{p'\,p} \rvert : p = (\max(x),y',z')
      \\ r_2 &= \lvert \vv{p'\,p} \rvert : p = (x',\max(y),z')
      \\ r_3 &= \lvert \vv{p'\,p} \rvert : p = (x',y',\max(z))
      \\ c   &= \lvert \vv{p'\,f} \rvert
    \end{aligned}
    % 
    \\[2ex] f_1\in\mathbb{R}^3 : f_1 = (x'-c, y', z')
    \\      f_2\in\mathbb{R}^3 : f_1 = (x'+c, y', z')
  \end{BM}

\end{sectionBox}

\begin{sectionBox}2{Parametrização em \(\mathbb{R}^3\)}

  \begin{BM}
    \mathcal{E}li_3\subset\mathbb{R}^3:
    e = e' + \lambda\,
    \begin{bmatrix}
      \cos(\theta)\,\cos(\lambda)
      \\  \cos(\theta)\,\sin(\lambda)
      \\  \sin(\theta)
    \end{bmatrix}
    \\[2ex]
    \left(
      \begin{aligned}
        &
        e\in\mathcal{E}li_3
        \ldiv{}
        \{e',\lambda\}\subset\mathbb{R}^3
        \ldiv{}
        \{\theta,\lambda\}\subset\mathbb{R}
        \ldiv{}
        \lvert\theta\rvert\leq \pi/2
        \ldiv{}
        0\leq\lambda\leq2\,\pi
        &
      \end{aligned}
    \right)
  \end{BM}

\end{sectionBox}

% ---------------------------------------------------------------------------- %
%                                  Hiperboles                                  %
% ---------------------------------------------------------------------------- %

\begin{sectionBox}1{Hiperboles e Hiperbolóides}


  \begin{multicols}{2}\centering

    \pgfplotsset{
      height = 0.4\pageheight, 
      width  = 0.5\pagewidth,
    }

    \begin{tikzpicture}
      \begin{axis}
        [
          % xlabel={\(x\)}, ylabel={\(y\)}, zlabel={\(z\)}
          % xmajorgrids = true,
          % legend pos  = north west,
          % 3d view,
          % perspective,
          % view = {10}{10}, % rot/elevation
          hide axis,
          % axis lines = {center}, % 3D center/box/left/right
          % axis on top,
          % ticks = {none} % minor/major/both/none,
          z buffer = {sort}, % default|none|auto|sort|reverse x seq|reverse y seq|reverse xy seq
        ]

        % Conical surface
        \addplot3 [
          % Aparence
          surf,
          opacity      = 0.5,
          fill opacity = 1,
          faceted color = foreground,
          shader = faceted interp,
          % Scope
          data cs = cart, % cart/polar/polarrad
          samples = \mysampledensitySimple,
          % Variable
          variable   = \u,
          variable y = \v,
          domain   = -1:1,
          domain y = 0:360,
        ](
          {cosh(u)*cos(v)},
          {cosh(u)*sin(v)},
          {sinh(u)}
        );

      \end{axis}
    \end{tikzpicture}

    \begin{tikzpicture}
      \begin{axis}
        [
          % xlabel={\(x\)}, ylabel={\(y\)}, zlabel={\(z\)}
          % xmajorgrids = true,
          % legend pos  = north west,
          % 3d view,
          % perspective,
          % view = {10}{10}, % rot/elevation
          hide axis,
          % axis lines = {left}, % center/box/left/right
          % axis on top,
          % ticks = {none},
          z buffer = {sort}, % default|none|auto|sort|reverse x seq|reverse y seq|reverse xy seq
        ]
        % Legends
        % \addlegendimage{empty legend}
        % \addlegendentry[Red]{\( x \)}

        % hiperbole 2 folhas
        \addplot3 [
          % Aparence
          surf,
          opacity      = 0.5,
          fill opacity = 1,
          faceted color = foreground,
          shader = faceted interp,
          % Scope
          data cs = cart, % cart/polar/polarrad
          samples = \mysampledensitySimple,
          % Variable
          variable   = \u,
          variable y = \v,
          domain   = 0:2,
          domain y = 0:360,
        ](
          {sinh(u)*cos(v)},
          {sinh(u)*sin(v)},
          {cosh(u)}
        );
        \addplot3 [
          % Aparence
          surf,
          opacity      = 0.5,
          fill opacity = 1,
          faceted color = foreground,
          shader = faceted interp,
          % Scope
          data cs = cart, % cart/polar/polarrad
          samples = \mysampledensitySimple,
          % Variable
          variable   = \u,
          variable y = \v,
          domain   = 0:2,
          domain y = 0:360,
        ](
          {sinh(u)*cos(v)},
          {sinh(u)*sin(v)},
          {-cosh(u)}
        );

      \end{axis}
    \end{tikzpicture}

    % \tikzset{external/force remake=false}

  \end{multicols}

  \section*{Definição}
  \begin{BM}
    \mathcal{H}ip_m\subset\mathcal{C}on_m
    : e>1
    \implies \\
    \implies 
    \Big\lvert
      \lvert \vv{P\,F_1} \rvert
      -\lvert \vv{P\,F_2} \rvert
    \Big\rvert
    =k\quad (k\in\mathbb{R})
  \end{BM}

  Uma hipérbole é o conjunto de pontos em um plano cujas distâncias de dois pontos fixos no plano têm uma diferença constante. Os dois pontos fixos são os focos da hipérbole.
  \\

  A linha através dos focos de uma hipérbole é o eixo focal. O ponto no eixo a meio caminho entre os focos é o centro da hipérbole. Os pontos onde o eixo focal e a hipérbole se cruzam são os vértices

  \begin{sectionBox}*3{Eccentricidade}

    \begin{BM}
      e = 
      % \frac{
      \left\lvert
        \vv{P'\,F}
      \right\rvert
      \Bigg/
      % }{
      \left\lvert
        \vv{P'\,V}
      \right\rvert
      % }
    \end{BM}

  \end{sectionBox}

\end{sectionBox}

% ----------------------------- Hiperboles em R2 ----------------------------- %

\begin{sectionBox}2{Hipérboles em \(\mathbb{R}^2\)}
  \begin{BM}
    \mathcal{H}ip_2\subset\mathcal{C}on_2
    : 
    \left(
      (P-P')^T
      \,A
    \right)
    (P-P')
    = 1
    % \implies \\
    % \implies
    % \lvert \vv{P\,F_1}
    \implies \\
    \implies
    \left(
      \begin{bmatrix}
        x-x'\\y-y'
      \end{bmatrix}^T
      \begin{bmatrix}
        1/r_1^2 & 0
        \\ 0     & -1/r_2^2
      \end{bmatrix}
    \right)
    \begin{bmatrix}
      x-x'\\y-y'
    \end{bmatrix}
    = 1
    \implies \\
    \implies
    \left(
      \frac{x}{r_1}
    \right)^2
    -\left(
      \frac{y}{r_2}
    \right)^2
    = 1
    \\[2ex]
    c 
    = \lvert \vv{P'\,F} \rvert
    = \sqrt{r_1^2+r_2^2}
    \\
    V = (x'\pm r_1,y')
    \\
    f = (x'\pm c, y')
  \end{BM}

  \begin{sectionBox}*3{Demonstração}
    \begin{flalign*}
      &
      2\,r_1
      = 
      \lvert P\,F_2 \rvert
      -\lvert P\,F_1 \rvert
      = &\\&
      = 
      \sqrt{
        (x-(x'-c))^2
        + (y-y')^2
      }
      -\sqrt{
        (x-(x'+c))^2
        + (y-y')^2
      }
      \implies &\\[1.5ex]&
      % 
      % 
      % 
      \implies
      \left(
        2\,r_1
        + \sqrt{
          (x-(x'+c))^2
          + (y-y')^2
        }
      \right)^2
      = &\\&
      =
      4\,r_1^2 
      + 4\,r_1
      \sqrt{
        (x-x'-c)^2
        + (y-y')^2
      }
      + \left(
        (x-(x'+c))^2
        + (y-y')^2
      \right)
      = &\\&
      =
      4\,r_1^2 
      + 4\,r_1
      \left(
        \sqrt{
          \begin{aligned}
            &  (x-x')^2
            \\ - &  2\,c\,(x-x')+c^2
            \\ + &  (y-y')^2
          \end{aligned}
        }
      \right)
      % \, + &\\&
      + \left(
        \begin{aligned}
          + & (x-x')^2
          \\ - & 2\,c(x-x')
          \\ + & c^2
          \\ + & (y-y')^2
        \end{aligned}
      \right)
      % 
      % 
      % 
      = &\\[1.5ex]&
      = \left(
        \sqrt{
          (x-(x'-c))^2
          + (y-y')^2
        }
      \right)^2
      % = &\\&
      = 
      \left(
        \begin{aligned}
          & (x-x')^2
          \\ + & 2\,c\,(x-x')
          \\ + & c^2
          \\ + & (y-y')^2
        \end{aligned}
      \right)
      \implies
      &
    \end{flalign*}
  \end{sectionBox}

  \begin{sectionBox}{}
    \begin{flalign*}
      &
      \implies
      \frac{c\,(x-x')}{r_1}-r_1
      =
      \sqrt{
        \begin{aligned}
          & (x-x')^2
          \\ - & 2\,c\,(x-x')
          \\ + & c^2
          \\ + & (y-y')^2
        \end{aligned}
      }
      \implies &\\[1.5ex]&
      \implies
      \left(
        \frac{c\,(x-x')}{r_1}-r_1
      \right)^2
      = 
      \left(
        \begin{aligned}
          & \frac{
            c^2\,(x-x')^2
          }{
            r_1^2
          }
          \\ - & 2\,c\,(x-x')
          \\ + & r_1^2
        \end{aligned}
      \right)
      =
      \left(
        \begin{aligned}
          & (x-x')^2
          \\ - & 2\,c\,(x-x')
          \\ + & c^2
          \\ + & (y-y')^2
        \end{aligned}
      \right)
      \implies &\\[1.5ex]&
      \implies
      \left(
        \frac{x-x'}{r_1}
      \right)^2
      \left(
        c^2
        -r_1^2
      \right)
      - (y-y')^2
      =
      c^2 - r_1^2
      \implies &\\[1.5ex]&
      \implies 
      \left(
        \frac{x-x'}{r_1}
      \right)^2
      - \left(
        \frac{
          y-y'
        }{
          \sqrt{c^2-r_1^2}
        }
      \right)^2
      = 
      \left(
        \frac{x-x'}{r_1}
      \right)^2
      - \left(
        \frac{y-y'}{r_2}
      \right)^2
      =
      1
      &
    \end{flalign*}
  \end{sectionBox}


\end{sectionBox}

% -------------------------------- Assintotas -------------------------------- %

\begin{sectionBox}3{Assintotas}

  \begin{BM}
    A(\mathcal{H}ip_2) = \left\{
      (x,y):
      (y-y') = \pm\frac{r_2}{r_1}(x-x')
    \right\}
  \end{BM}

\end{sectionBox}

% ----------------------------- Hiperboles em R3 ----------------------------- %

\begin{sectionBox}2{Hipérboles em \(\mathbb{R}^3\)}

  \subsubsection*{Hiperbole de 1 folha}
  \begin{BM}[flalign*]
    &
    % (\lambda^T + (p-p')^T\,A)(p-p')=k
    \left(
      \begin{bmatrix}
        x\\y\\z
      \end{bmatrix}^T
      \begin{bmatrix}
        1/a^2 & 0 & 0
        \\  0 & 1/b^2 & 0
        \\  0 & 0 & -1/c^2
      \end{bmatrix}
    \right)
    \begin{bmatrix}
      x\\y\\z
    \end{bmatrix}
    =   1
    \implies &\\&
    \implies
    \frac{x^2}{a^2} 
    +   \frac{y^2}{b^2} 
    -   \frac{z^2}{c^2} 
    =   1
    &
  \end{BM}

  \subsubsection*{Hiperbole de 2 folha}
  \begin{BM}[flalign*]
    &
    % (\lambda^T + (p-p')^T\,A)(p-p')=k
    \left(
      \begin{bmatrix}
        x\\y\\z
      \end{bmatrix}^T
      \begin{bmatrix}
        1/a^2 & 0 & 0
        \\  0 & -1/b^2 & 0
        \\  0 & 0 & -1/c^2
      \end{bmatrix}
    \right)
    \begin{bmatrix}
      x\\y\\z
    \end{bmatrix}
    =   1
    \implies &\\&
    \implies
    \frac{x^2}{a^2} 
    -   \frac{y^2}{b^2} 
    -   \frac{z^2}{c^2} 
    =   1
    &
  \end{BM}
\end{sectionBox}

\begin{sectionBox}3{Equações paramétricas para \(\mathbb{R}^3\)}
  \begin{multicols}{2}
    \subsection*{unica folha}
    \begin{BM}
      \begin{cases}
        x = a\hspace{-0.9em}&\cosh(v)\cos(\theta)
        \\  y = b\hspace{-0.9em}&\cosh(v)\sin(\theta)
        \\  z = c\hspace{-0.9em}&\sinh(v)
      \end{cases}
      \\  \begin{cases}
        v\in(-\infty,\infty)
        \\  \theta\in[0,2\,pi)
      \end{cases}
    \end{BM}

    \subsection*{duas folhas}
    \begin{BM}
      \begin{cases}
        x = \phantom{\pm}a\hspace{-0.9em}&\sinh(v)\cos(\theta)
        \\  y = \phantom{\pm}b\hspace{-0.9em}&\sinh(v)\sin(\theta)
        \\  z = \pm c\hspace{-0.9em}&\cosh(v)
      \end{cases}
      \\  \begin{cases}
        v\in[0,\infty)
        \\  \theta\in[0,2\,pi)
      \end{cases}
    \end{BM}
  \end{multicols}
\end{sectionBox}

\end{document}

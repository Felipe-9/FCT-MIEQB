% !TEX root = ./AM_2C-Anotações.2.tex
\providecommand\mainfilename{"./AM_2C-Anotações.tex"}
\providecommand \subfilename{}
\renewcommand   \subfilename{"./AM_2C-Anotações.2.tex"}
\documentclass[\mainfilename]{subfiles}

% \tikzset{external/force remake=true} % - remake all

\begin{document}

% \graphicspath{{\subfix{./.build/figures/AM_2C-Anotações.2}}}
% \tikzsetexternalprefix{./.build/figures/AM_2C-Anotações.2/graphics/}

\mymakesubfile{2}
[AM\,2C]
{Anotações 2} % Subfile Title
{Anotações} % Part Title

\part*{Preface}

\begin{definitionBox}1{Matriz Adjunta} % S2
    
    \begin{BM}
        \adj{A} = C^T(A)
        \\
        \adj{A}\in\mathcal{M}_{m\times m}:
        (\adj{A})_{i,j}=(C_{(A)})_{j,i}=(-1)^{i+j}\,(M_{(A)})_{j,i}
        \\
        A\in\mathcal{M}_{m\times m}
    \end{BM}

    \begin{definitionBox}*2{Determinação da invérsa} % S2.1
        
        \begin{BM}
            \adj{A}*A = A*\adj{A} = \det{A}*I
            \\
            A^{-1} = (\det A)^{-1}*\adj{A}
        \end{BM}
        
    \end{definitionBox}
    
\end{definitionBox}

\part*{Conteúdo}

\begin{sectionBox}1{Matriz Jacobiana} % S2
    
    \begin{BM}
        J_f\in\mathbb{M}_{n\times m}:
        (J_f)_{i,j} = \pdv{f_i}{x_j}
        \\
        f:\mathbb{R}^n\to\mathbb{R}^m
    \end{BM}
    
\end{sectionBox}

\begin{sectionBox}1{Matriz Hessiana} % S3
    
    \begin{BM}
        H(f(x)) = J(\nabla f(x))
        \\
        H_{f}\in\mathcal{M}_{n\times n}
        :(H_f)_{i,j}=\pdv{f}{x_i,x_j}
        \\
        f:\mathbb{R}^n\to\mathbb{R}
    \end{BM}

    É usada para estudar os pontos extremos locais de uma função

    \begin{sectionBox}{Determinante}
        
        \begin{BM}
            \det H_f
            \begin{cases}
                > 0\implies\text{crítico local}
                \\ = 0\implies\text{Indeterminável}
                \\ < 0\implies\text{Ponto de Sela}
            \end{cases}
            \\
            \det H_f > 0 \land \pdv[order=2]{f}{x}
            \begin{cases}
                > 0 \implies \text{Mínimo local}
                \\
                < 0 \implies \text{Máximo local}
            \end{cases}
        \end{BM}
        
    \end{sectionBox}
    
\end{sectionBox}

\end{document}
\part{Soluções de ácidos e de bases - titulação ácido-base}

\begin{sectionBox}1{Preparação}

    Preparar uma solução de \ch{HCl} 0.1M e duas soluções de \ch{NaOH} 0.1M e 4\% e titular as soluções de \ch{NaOH} com \ch{HCl} 1M usando fenolftaleína.

    \begin{table}[H]\centering
        \begin{tabular}{l l c r}

                \multicolumn{1}{c}{Nome}
            &   \multicolumn{1}{c}{Formula}
            &   \multicolumn{1}{c}{Molecula}
            &   \multicolumn{1}{c}{massa molar}
            &   \multicolumn{1}{c}{Segurança}
            &   \multicolumn{1}{c}{\dots}

            \\ \toprule

                % HCl : 36.46094    = 1.00794 + 35.453
                % NaOH: 39.99710928 = 22.98976928 + 15.9994 + 1.00794

                Ácido Clorídrico   & \ch{HCl}  & \chemfig[atom sep = 8mm, angle increment = 15]{H-Cl}          & \num{36.46094}    & \multirow{3}{*}{\dots}
            \\  Hidróxido de Sódio & \ch{NaOH} & \chemfig[atom sep = 5mm, angle increment = 15]{Na-[2]O-[-2]H} & \num{39.99710928} &

            \\ \bottomrule

        \end{tabular}
    \end{table}

    \begin{sectionBox}*2m{Cálculos}
        \begin{sectionBox}*3{\ch{NaOH\aq{}}(0.1M)}
            \begin{flalign*}
                &
                    m\,\unit{\gram\of{\ch{NaOH\sld{}}}}
                =   \frac
                        {\qty{39.99710928}{\gram\of{\ch{NaOH\sld{}}}}}
                        {\unit{\mole\of{\ch{NaOH}}}}
                \,  \frac
                        {0.1\,\unit{\mole\of{\ch{NaOH}}}}
                        {\unit{\litre\of{\ch{NaOH\aq{}}}}}
                \,  100\,\unit{\milli\litre\of{\ch{NaOH\aq{}}}}
                \cong &\\&
                \cong
                    \qty {0.3999710928}
                        {\gram\of{\ch{NaOH\sld{}}}}
                &
            \end{flalign*}
        \end{sectionBox}

        \begin{sectionBox}*3{\ch{NaOH\aq{}}4\%}
            \begin{flalign*}
                &
                    m\,\unit{\gram\of{\ch{NaOH\sld{}}}}
                =   \frac
                        {0.04\unit{\gram\of{\ch{NaOH\sld{}}}}}
                        {    \unit{\milli\litre\of{\ch{NaOH\aq{}}}}}
                \,  100\,\unit{\milli\litre\of{\ch{NaOH\aq{}}}}
                \cong &\\&
                \cong
                    \qty {4}
                        {\gram\of{\ch{NaOH\sld{}}}}
                &
            \end{flalign*}
        \end{sectionBox}

        \begin{sectionBox}*3{\ch{HCl\aq{}}0.1M}
            \begin{flalign*}
                &
                    v\,\unit{\milli\litre\of{\ch{HCl\aq{}}}}
                =   \frac
                        {      \unit{\milli\litre\of{\ch{HCl\aq{}}}}}
                        {1.19\,\unit{\gram\of{\ch{HCl\aq{}}}}}
                \,  \frac
                        {    \unit{\gram\of{\ch{HCl\aq{}}}}}
                        {0.37\unit{\gram\of{\ch{HCl\sld{}}}}}
                \,  \frac
                        {\qty{36.46094}{\gram\of{\ch{HCl\aq{}}}}}
                        {          \unit{\mole\of{\ch{HCl}}}}
                \,  \frac
                        {0.1\,\unit{\mole\of{\ch{HCl}}}}
                        {\unit{\litre\of{\ch{HCl\aq{}}}}}
                \,  250\,\unit{\milli\litre\of{\ch{HCl\aq{}}}}
                \cong &\\&
                \cong
                    \qty {2.070232795821031}
                        {\milli\litre\of{\ch{HCl\aq{}}}}
                &
            \end{flalign*}
        \end{sectionBox}

    \end{sectionBox}

\end{sectionBox}


\section{Experimento}

\begin{sectionBox}*2{\ch{NaOH} 0.1M}
    \begin{itemize}
        \item[Peso:] 0.40\,\unit{\gram\of{\ch{NaOH\sld{}}}}
        \item[Vol:]  100\,\unit{\milli\litre\of{\ch{NaOH\aq{}}}}
    \end{itemize}
\end{sectionBox}

\begin{sectionBox}*2{\ch{NaOH} 4\%}
    \begin{itemize}
        \item[Peso:] 4.00\,\unit{\gram\of{\ch{NaOH\sld{}}}}
        \item[Vol:]  100\,\unit{\milli\litre\of{\ch{NaOH\aq{}}}}
    \end{itemize}
\end{sectionBox}

\begin{sectionBox}*2{\ch{HCl} 0.1M}
    \begin{itemize}
        \item[Vol:] 2.01\,\unit{\milli\litre\of{\ch{HCl\aq{37\%}}}}
    \end{itemize}
\end{sectionBox}

\begin{sectionBox}*2{Titulação \ch{NaOH\aq{0.1M}}}
    \begin{enumerate}
        \item 3.1 \unit{\milli\litre\of{\ch{HCl\aq{37\%}}}}
        \item 2.5 \unit{\milli\litre\of{\ch{HCl\aq{37\%}}}}
        \item 2.3 \unit{\milli\litre\of{\ch{HCl\aq{37\%}}}}
    \end{enumerate}
\end{sectionBox}

\begin{sectionBox}*2{Titulação \ch{NaOH\aq{4\%}}}
    \begin{enumerate}
        \item 19.8 \unit{\milli\litre\of{\ch{HCl\aq{37\%}}}}
    \end{enumerate}
\end{sectionBox}





% \begin{sectionBox}2{}
%
%
%
%     \subsection*{\ch{HCl\aq{}}}
%     % Titulação com primeira solução de NaOH (mais fraca)
%     \begin{enumerate}
%         \item \( 3.1\,\unit{\milli\litre\of{\ch{HCl\aq{}}}} \)
%         \item \( 2.5\,\unit{\milli\litre\of{\ch{HCl\aq{}}}} \)
%         \item \( 2.3\,\unit{\milli\litre\of{\ch{HCl\aq{}}}} \)
%     \end{enumerate}
%
%     % Titulação com a segunda solução de NaOH (mais forte)
%     \begin{enumerate}
%         \item \( 19.8\,\unit{\milli\litre\of{\ch{HCl\aq{}}}} \)
%     \end{enumerate}
%
% \end{sectionBox}



% a) prep NaOH 0.1 mol/L
    % peso NaOH = 0.40 g     (balança só possuía 2 casas decimais, justificar erro)
    % volume    = 100.00 mL

% b) prep NaOH 4%
    % peso NaOH = 4.00 g
    % volume    = 100.00 mL

% c) prep HCl
    % HCl  2.01 mL

% d) titulação
    % NaOH 1 mol/L
    % 3.1 mL
    % 2.5 mL
    % 2.3 mL

    % NaOH 4%
    % 19.8 mL

% Lab 1
\part{Determinação de constantes físicas}

\vspace{6mm}

\section{Determinação da densidade com densímetro}

\begin{sectionBox}{}

    Usar um densímetro e uma proveta para medir a densidade de 3 amostras

    \begin{table}[H]\centering
        \begin{tabular}{l l c r}

                \multicolumn{1}{c}{Nome}
            &   \multicolumn{1}{c}{Formula}
            &   \multicolumn{1}{c}{Molecula}
            &   \multicolumn{1}{c}{\dots}

            \\ \toprule

                Água          & \ch{H2O}           & \chemfig[atom sep = 8mm, angle increment = 15]{H-[2]O-[-2]H}           & \multirow{3}{*}{\dots}
            \\  Diclorometano & \ch{CH2Cl2}        & \chemfig[atom sep = 5mm, angle increment = 15]{Cl-[2]-[-2]Cl}          &
            \\  n-Hexano      & \ch{CH3(CH2)4CH3}  & \chemfig[atom sep = 5mm, angle increment = 15]{-[2]-[-2]-[2]-[-2]-[2]} &

            \\ \bottomrule

        \end{tabular}
    \end{table}

    \begin{table}[H]\centering
        \begin{tabular}{l r l l}

                \multicolumn{1}{c}{\dots}
            &   \multicolumn{1}{c}{Densidade}
            &   \multicolumn{1}{c}{(temp. (\unit{\celsius}))}
            &   \multicolumn{1}{c}{Segurança}

            \\ \toprule

                \multirow{3}{*}{\dots} & \( 1.000 \) & \( (3.98) \) & \ref{diclorometano}
            \\                         & \( 1.33  \) & \( (20)   \) &
            \\                         & \( 0.66  \) & \( (25)   \) &

            \\ \bottomrule

        \end{tabular}
    \end{table}

\end{sectionBox}


\begin{sectionBox}*2m{Segurança}

    % \begin{multicols}{2}

        % \begin{sectionBox}*3{Diclorometano (\ch{CH2Cl2})}
        % \begin{multicols}{2}
        %
        %     % Pictogramas
        %     \begin{sectionBox}{}
        %         \begin{multicols}{3}
        %             \ghspic{exclam} % \par
        %             \ghspic{health}
        %         \end{multicols}
        %     \end{sectionBox}
        %
        %     % Hs
        %     \begin{sectionBox}*3{H statements}
        %         \begin{itemize}
        %             \item \ghs{h}{315}
        %             \item \ghs{h}{319}
        %             \item \ghs{h}{336}
        %             \item \ghs{h}{351}
        %         \end{itemize}
        %     \end{sectionBox}
        %
        %     % Ps
        %     % \begin{sectionBox}*3m{P statements}
        %
        %         \begin{sectionBox}*3{P.Prevenção}
        %             \begin{itemize}
        %                 \item \ghs{p}{261}
        %                 \item \ghs{p}{280}
        %             \end{itemize}
        %         \end{sectionBox}
        %
        %         \begin{sectionBox}*3{P.Resposta}
        %             \begin{itemize}
        %                 \item \ghs{p}{302+352}
        %                 \item \ghs{p}{305+351+338}
        %                 \item \ghs{p}{308+313}
        %             \end{itemize}
        %         \end{sectionBox}
        %
        %     % \end{sectionBox}
        %
        % \end{multicols}
        % \end{sectionBox}


        % \begin{sectionBox}*2{n-Hexano (\ch{CH3(CH2)4CH3})}
        % \begin{multicols}{2}
        %
        %     % Pictogramas
        %     \begin{sectionBox}{}
        %         \ghspic{flame}
        %         \ghspic{exclam}
        %         \ghspic{health}
        %         \ghspic{aqpol}
        %     \end{sectionBox}
        %
        %     % Hs
        %     \begin{sectionBox}*3{H Statements}
        %         \ghs[hide]{h}{225},
        %         \ghs[hide]{h}{304},
        %         \ghs[hide]{h}{315},
        %         \ghs[hide]{h}{336},
        %         \ghs[hide]{h}{361},
        %         \ghs[hide]{h}{373},
        %         \ghs[hide]{h}{411}
        %     \end{sectionBox}
        %
        %     % Ps
        %     \begin{sectionBox}*3{P Statements}
        %         \subsubsection*{P.Prevensão}
        %         \begin{itemize}
        %             \item \ghs{p}{202}
        %             \item \ghs{p}{280}
        %         \end{itemize}
        %
        %         \subsubsection*{P.Resposta}
        %         \begin{itemize}
        %             \item \ghs{p}{303+361+353}
        %             \item \ghs{p}{304+340}
        %             \item \ghs{p}{308+313}
        %         \end{itemize}
        %     \end{sectionBox}
        %
        %     % Extinção
        %     \begin{sectionBox}*3{Extinção}
        %         \begin{itemize}
        %             \item Agua Pulverizada
        %             \item Espuma
        %             \item Pó Químico
        %             \item Dióxido de Carbono (\ch{CO2})
        %             \item[x] Jato d'água
        %         \end{itemize}
        %     \end{sectionBox}
        %
        % \end{multicols}
        % \end{sectionBox}

    % \end{multicols}

\end{sectionBox}



\newpage



% 1.2 - Determinação do índice de refração
\begin{sectionBox}1{Determinação índice de refração}

    Usar o refratometro para medir o índice de refração de 3 amostras

    \begin{table}[H]\centering
        \begin{tabular}{l l c l}

                \multicolumn{1}{c}{Nome}
            &   \multicolumn{1}{c}{Formula}
            &   \multicolumn{1}{c}{Molecula}
            &   \multicolumn{1}{c}{Índice de Refração (\unit{n_{cl}^t})}

            \\ \toprule

                Ciclo-Hexano  & \ch{C6H12}         & \chemfig[atom sep=5mm, angle increment = 15, ]{*6(------)}           & \( 1.42662 \)
            \\  Diclorometano & \ch{CH2Cl2}        & \chemfig[atom sep=5mm, angle increment = 15]{Cl-[-2]-[2]Cl}          & \( 1.4241  \)
            \\  n-Hexano      & \ch{CH3(CH2)4CH3}  & \chemfig[atom sep=5mm, angle increment = 15]{-[2]-[-2]-[2]-[-2]-[2]} & \( 1.3749  \)

            \\ \bottomrule

        \end{tabular}
    \end{table}

\end{sectionBox}


% Segurança
\begin{sectionBox}*2m{Segurança}

    \begin{multicols}{2}

        \begin{sectionBox}*3{Diclorometano (\ch{CH2Cl2})}
            Ver Página: 3
        \end{sectionBox}

        \begin{sectionBox}*3{n-Hexano (\ch{CH3(CH2)4CH3})}
            Ver Página: 5
        \end{sectionBox}

    \end{multicols}

    \begin{sectionBox}*3{Ciclo-Hexano (\ch{C6H12})}
        \begin{multicols}{2}

            % Pictogramas
            \begin{sectionBox}{}
                \ghspic{flame}
                \ghspic{exclam}
                \ghspic{health}
                \ghspic{aqpol}
            \end{sectionBox}

            % Hs
            \begin{sectionBox}*3{H Statements}
                \ghs[hide]{h}{225},
                \ghs[hide]{h}{304},
                \ghs[hide]{h}{315},
                \ghs[hide]{h}{336},
                \ghs[hide]{h}{410}
            \end{sectionBox}

            % Ps
            \begin{sectionBox}*3{P Statements}
                \subsubsection*{P.Prevenção}
                \begin{itemize}
                    \item \ghs{p}{210}
                    \item \ghs{p}{273}
                \end{itemize}

                \subsubsection*{P.Resposta}
                \begin{itemize}
                    \item \ghs{p}{301+330+331}
                    \item \ghs{p}{302+352}
                \end{itemize}
            \end{sectionBox}

            \begin{sectionBox}*3{Extinção}
                \begin{itemize}
                    \item Água Pulverizada
                    \item Espuma
                    \item Pó Seco para extinção
                    \item Dióxido de Carbono (\ch{CO2})
                    \item[x] Jato d'água
                \end{itemize}
            \end{sectionBox}

        \end{multicols}
    \end{sectionBox}


\end{sectionBox}



\newpage



% L1.3 - Determinação do ponto de fusão
\begin{sectionBox}1{Ponto de Fusão}

\end{sectionBox}


% TL1.1 - conclusão
% n-hexano:
    % \delta(\rho): |0.662/0.66 - 1| =~ 0.303 %
    % \delta( T  ): |23/25      - 1| =~ 8 %

% diclorometano:
    % \delta (\rho): |1.321/1.33 - 1| =~ 0.68 %
    % \delta (T)   : |22/20      - 1| =~ 10   %

% Agua
    % \delta(\rho): |0.985/1.000 - 1| =~ 1.5 %
    % \delta( T  ): |23/3.98     - 1| =~ 477.89 %

    % Apesar de ter grande grande variação de temperatura percebemos pouca variação na densidade da agua, isso indica que a variação de temperatura varia pouco a densidade da agua.

% Ao comparar os valores experimentais com o tabelado percebemos grande proximidade dos valores, isso é uma boa indicação que as amostras são semelhantes aos elementos usados para a tabelar os valores literários.

% TL1.2 - conclusão
% Ciclo Hexando
    % \delta (n_{cl}^t): | 1.4629/1.42662 - 1 | =~ 2.5431 %

% Diclorometano
    % \delta (n_{cl}^t): | 1.4201/1.4241 - 1 | =~ 0.2809 %

% n-hexano
    % \delta (n_{cl}^t): | 1.3747/1.3749 - 1 | =~ 0.0145 %

% Ao comparar os valores experimentais com o tabelado percebemos grande proximidade dos valores, isso é uma boa indicação que as amostras experimentais apresentam alto grau de pureza.

% TL1.3 - conclusão
% Acido benzoico
    % \Delta t = 3oC
    % t_fusão = 124.0 +- 3oC
    % \delta (t) = |124.0/122.4 - 1| =~ 1.31 %

% Para uma amostra ser considerada pura deve ter uma variação de temperatura durante a fusão de 2oC, nossa amostra possuí variação de 3oC o que indica que possui alto grau de pureza.

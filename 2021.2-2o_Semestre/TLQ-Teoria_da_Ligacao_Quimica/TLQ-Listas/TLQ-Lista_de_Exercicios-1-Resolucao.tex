% !TEX root = ./TLQ-Lista_de_Exercicios-1-Resolução.tex
% !TEX root = ./TLQ-Lista_de_Exericicios-Resoluções.tex
\providecommand\mainfilename{"./TLQ-Lista_de_Exericicios-Resoluções.tex"}
\providecommand \subfilename{}
\renewcommand   \subfilename{"./TLQ-Lista_de_Exercicios-1-Resolução1.tex"}
\documentclass[\mainfilename]{subfiles}

% \graphicspath{{\subfix{../images/}}}
% \tikzset{external/force remake=true} % - remake all

\begin{document}

\mymakesubfile{1}
[TLQ]
{Resolução}
{Resolução}

% Q1
\begin{questionBox}1{}
    % A energia mínima necessária à remoção de electrões do metal magnésio, \ch{^12Mg}, é de 738\,\unit{\kilo\joule\per\mole}.

    \begin{questionBox}2{}

        % Determine a energia cinética de um electrão ejectado da superfície do magnésio se sobre ele incidir radiação electromagnética de frequência \(\nu=2.63*10^{16}\,\unit{\hertz}\).

        \begin{flalign*}
            &
                E_c
            =   E - E_{ion}
            =   (h\,\nu - 738*10^{3}/\num{6.02214076e23})
                \unit{\joule}
                \frac{\unit{\eV}}{\qty{1.6027733e-19}{\joule}}
            = &\\&
            =   (\num{6.62607004e-34}*2.63*10^{16} - \num{1.225477831574299e-18})\unit{\eV}/\num{1.6027733e-19}
            \cong
                \qty{95.107537469121186}{\eV}
            &
        \end{flalign*}
    \end{questionBox}

    \begin{questionBox}2{}
        \begin{BM}
            E_c = E - E_{ion}:E>E_{ion}
        \end{BM}
    \end{questionBox}

\end{questionBox}


% Q2
\begin{questionBox}1m{}

    % Q2.1
    \begin{questionBox}2{}
        \begin{flalign*}
            &
                E_{ion}
            =   E - E_c
            =   h\,\nu - E_c
            =   \num{6.62607004e-34}*4*10^{14}
            \cong
                \qty{2.650428016e-19}{\joule}
            &
        \end{flalign*}
    \end{questionBox}

    % Q2.2
    \begin{questionBox}2{}
        \begin{flalign*}
            &
                \lambda_{max}
            =   c/\nu
            =   \num{299792458}/4*10^{14}
            \cong
                \qty{749.481145}{\nano\meter}
            &
        \end{flalign*}
    \end{questionBox}

    % Q2.3
    \begin{questionBox}2{}
        Não ha relação da intencidade de uma radiação com a energia cinética adquirída pelos elétron, esta irá apenas variar o numero de elétrons afetados
    \end{questionBox}

\end{questionBox}


% Q3
\begin{questionBox}1{}

    \begin{questionBox}2{}
        \begin{flalign*}
            &
                \mathrm{E} = h\,v; \lambda\,v = c
                \implies &\\&
                \implies
                E = h\,c/\lambda
                = 6.626*10^{-34}\unit{\joule\second}\,2.9979*10^{8}\,\unit{\metre\per\second}/600\,\unit{\nano\metre}
            &
        \end{flalign*}
    \end{questionBox}

\end{questionBox}



% Q5
\begin{questionBox}1{}
    E
\end{questionBox}

\end{document}

% !TEX root = ./TLQ - Slides Anotações.tex
% TLQ - Slide 21 Anotações

\setcounter{part}{20}
\part{Reatividade Química}

\part*{Reatividade Química e Simetria Orbital}

\begin{sectionBox}1{Requisitos de reação de duas moléculas segundo a TOF}
    
    Para duas moléculas reagirem fácilmente, devem obedecer as seguintes condições:
    \begin{enumerate}[label = \Roman*]
        \item \textbf{Fluxo de elétrons} da \HOMO{} do doador eletrônico para a \LUMO{} do aceitador eletrônico a medida que as duas moléculas reagentes se aproximam.
        \item \textbf{Interação construtiva entre as OMs} da \HOMO{} do doador e a \LUMO{} do aceitador na aproximação.
        \item \textbf{Energias próximas} dos orbitais \HOMO{} do doador e \LUMO{} do aceitador
        \item \textbf{Formação e quebra de ligações químicas} é resultado da mistura \HOMO{}\rightarrow\LUMO{}
    \end{enumerate}
    
    \paragraph{Reações que obedeçem as quatro regras} são ditas \textcolor{Emph}{permitidas por simetria} e, geralmente ocorrem com energias de ativação relativamente pequenas.

\end{sectionBox}

\begin{sectionBox}2{I Critério}
    

    \textbf{Fluxo de elétrons} da \HOMO{} do doador eletrônico para a \LUMO{} do aceitador eletrônico a medida que as duas moléculas reagentes se aproximam.

    % TODO: desenho orbitais moleculáres
    [Desenho]
    
\end{sectionBox}

\begin{sectionBox}2{II Critério}
    
    \textbf{Interação construtiva entre as OMs} da \HOMO{} do doador e a \LUMO{} do aceitador na aproximação.

    % TODO: desenho orbitais moleculáres
    [Desenho]
    
    Interação \textcolor{green\Light}{Positivo--Positivo} e \textcolor{red\Light}{Negativo--Negativo} dentre \HOMO{} e \LUMO{}

    \begin{sectionBox}*3{Movimento de elétrons}
        
        Apresentando a seta que representa o movimento eletrônico temos um \textcolor{Emph}{aparente} impedimento por simetria.

        [Desenho]

        Como ambas as moléculas são independentes podemos inerter os sinais dos orbitais moleclares.
        
    \end{sectionBox}

\end{sectionBox}

\begin{sectionBox}2{III Critério}
    
    \textbf{Energias próximas} dos orbitais \HOMO{} do doador e \LUMO{} do aceitador

    % TODO: desenho MO
    \begin{center}
        \begin{modiagram}
            \setlength\AtomVScale{0.5cm}
            % \setlength\MoleculeVScale{1.5cm}
    
            % C
            \AO(1cm){s}{0\AtomVScale;}       % S2  AO1: -19.4
            \AO(1cm){s}{5\AtomVScale;}       % S2  AO2: -19.4
    
            % O
            \AO(5cm){s}{-2\AtomVScale;}       % S2  AO3: -32.3
            \AO(5cm){s}{1\AtomVScale;}       % S2  AO4: -32.3
    
            \connect{
                % left
                AO1  & AO4,
                AO2  & AO3,
            }
            
            \node at (1cm, 0\AtomVScale)[below]{\LUMO};
            \node at (1cm, 5\AtomVScale)[above]{\HOMO};
            \node at (5cm,-2\AtomVScale)[below]{\LUMO};
            \node at (5cm, 1\AtomVScale)[below]{\HOMO};

            \node at (1cm, -4\AtomVScale){A};
            \node at (5cm, -4\AtomVScale){B};
    
            \EnergyAxis[title=E, head=stealth]
    
        \end{modiagram}
    \end{center}
    
    \paragraph{Catalizadores} são usados para promover as reações ajustando o sistema seguindo essa regra.

\end{sectionBox}

\begin{sectionBox}2{IV Critério}
    
    \textbf{Formação e quebra de ligações químicas} é resultado da mistura \HOMO{}\rightarrow\LUMO{}

    [desenho?]
    
\end{sectionBox}

\begin{questionBox}1{Tróca isotópica entre \ch{H2} e \ch{D2}}
    
    \begin{center}
        \ch{H2 + D2 -> 2 HD}
    \end{center}

    ...
    
\end{questionBox}
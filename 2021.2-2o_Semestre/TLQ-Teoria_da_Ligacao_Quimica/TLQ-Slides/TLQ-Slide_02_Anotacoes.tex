\part{Teoria Atômica II}

\begin{sectionBox}1{Teoria orbital de Bohr (cont.)}

    \begin{sectionBox}*2{Energia de Ionização do Hidrogênio}
        \begin{flalign*}
            &
                E_i
            =   \lim_{n\to\infty}|E_{t\,0} - E_{t\,n}|
            =   \lim_{n\to\infty}k\,
                \left|
                    1^{-2}-n^{-2}
                \right|
            =   k
            \cong
                \num[round-precision=5, exponent-to-prefix=false]
                    {2.179872251033439e-18}
            &
        \end{flalign*}
    \end{sectionBox}

    \begin{sectionBox}*2{Espectro de emissão do Hidrogênio}

        Apesar das derivações da teoria de Bohr, ela permite prever o espectro de emissão do hidrogênio

        ...
    \end{sectionBox}

\end{sectionBox}



\section*{Fenômenos Ondulatórios}

\begin{sectionBox}1{Refração}

    Quando uma onda passa a se propagar em um meio diferente.

    Consequencias
    \begin{itemize}
        \item Alteração do comprimento de onda e consequentemente a velocidade de propagação (a frequência depende apenas da origem).
        \item Alteração da direção de propagação.
    \end{itemize}

    regra geral matéria interfere negativamente com a propagação de onda, diminuindo sua velocidade e comprimento de onda e aumentando o desvio.

\end{sectionBox}

\begin{sectionBox}1{Interferencia}

    Interação dentre ondas de mesmas características. % deve existir melhor nome


    \begin{sectionBox}*2{Tipos:}
        \begin{multicols}{2}

            \begin{sectionBox}*3{Construtiva}
                Momento de sincronização onde suas amplitudes se somam
            \end{sectionBox}

            \begin{sectionBox}*3{Destrutiva}
                Momento de sincronização onde suas amplitudes se subtraem
            \end{sectionBox}

        \end{multicols}
    \end{sectionBox}

\end{sectionBox}

\begin{sectionBox}1{Difração}

    Ocorre quando uma onda passa por um obstáculo com ordem de grandeza próxima do comprimento de onda.

\end{sectionBox}

\begin{sectionBox}1m{Efeito fotoelétrico}

    Fenômeno que ocorre quando placas carregadas recebem emissão elétro-magnéticas liberam elétrons

    \begin{itemize}
        \item Energia cinética dos elétrons depende da frequência da onda elétro-magnética
        \item A quantidade de elétrons depende da intensidade da onda.
    \end{itemize}

    \begin{sectionBox}*2{Foton (Quanta)}

        Energia associada a uma dada frequência.

        Explicando o efeito fotoelétrico pela característica dos elétrons de apenas receberem uma quantidade fixa de energia (chamada de ``quanta'' por Einstein e substituida por ``foton'' pela aceitação do conceito da dualidade partícula-onda dos elétrons proposto pelo príncipe Louis de Broglie (1924))

        \begin{BM}
            E=h\,\nu
        \end{BM}

        \begin{itemize}
            \item[\(h\):] Constante de Plank = \qty{6.62607004e-34}{\joule\second}
            \item[\nu:]   Frequencia da radiação
        \end{itemize}

    \end{sectionBox}

    \begin{sectionBox}*2{Dualidade partícula-onda (Príncipe Louis de Broglie)}
        \begin{flalign*}
            &
                \lambda\,\nu=c
            \land
                E = h\,\nu
            \implies
                E = \frac{h\,c}{\lambda} = m_e\,c^2
            \implies
                E = \frac{h\,v}{\lambda} = m_e\,v^2
            \implies &\\&
            \implies
                \lambda = h/m_e\,v = h/P
            &
        \end{flalign*}
    \end{sectionBox}

\end{sectionBox}



\begin{sectionBox}1{Comprimento de onda do eletron}

    O perímetro de uma determinada orbita de um eletron ao redor do núcleo terá de ser sempre um múltiplo inteiro do comprimento de onda do eletron

    \begin{flalign*}
        &
            P
        =   n\,\lambda
        =   2\,\pi\,r
        \implies
            \lambda
        =   2\,\pi\,r/n
        &
    \end{flalign*}

    \begin{sectionBox}*2{Quantização do momento angular}
        \begin{flalign*}
            &
                |\vec{L}|
            =   |m_e\,\vec{v}\times\vec{r}|
            =   m_e\,v\,r
            \land
                \frac{m_e\,v}{h}=\frac{n}{2\,\pi\,r}=\lambda^{-1}
            \implies &\\&
            \implies
                |\vec{L}|
            =   n\frac{h}{2\,\pi}
            =   n\,k
            &
        \end{flalign*}
    \end{sectionBox}

\end{sectionBox}


\begin{sectionBox}1{Equação de Rydberg}
    \begin{BM}
        \bar{\nu}_H=R_H(n^{-2}-m^{-2})\quad\forall\,\{n,m\}\in\mathbb{K}:m>n
    \end{BM}

    \begin{itemize}
        \item[\(R_H\):] Constante de Rydberg 109678.746\,\unit{\centi\meter^{-1}}
        \item[\(\bar{\nu}\):] Número de onda
    \end{itemize}

    \begin{sectionBox}*2{Comparação com Variação de energia do elétron}
        \begin{flalign*}
            &
                \bar{\nu}=\lambda^{-1}
            \land
                \lambda\,\nu=c
            \land
                h\,\nu
            =   \Delta E
            =   k|n_1^{-2}-n_2^{-2}|
            \implies &\\&
            \implies
                \bar{\nu}
            =   \frac{k}{h\,c}|n_1^{-2}-n_2^{-2}|
            &
        \end{flalign*}


        %     R
        % =   k/h\,c
        % =   \frac{m_e\,e^4}{4\,\varepsilon_0^2\,h^3\,c}
        % \cong
        %     109737.3\,\unit{\centi\meter^{-1}}
        % \neq
        %     109678.764\,\unit{\per\centi\meter}
        % =   R_H
    \end{sectionBox}

    \begin{sectionBox}*2{Massa reduzida do átomo}

        Levando em consideração a massa do elétron e átomo

        \begin{flalign*}
            &
                R_H
            =   109678.764\,\unit{\per\centi\meter}
            \cong
                R\,\mu_H
            =   (k/h\,c)
                \left(
                    \frac{m_e\,m_p}{m_e+m_p}
                \right)
            = &\\&
            =   \left(
                    \frac{m_e\,e^4}{8\,\varepsilon_0^2\,h^3\,c}
                \right)
                \left(
                    \frac{m_e\,(1836\,m_e)}{m_e+(1836\,m_e)}
                \right)
            =   \frac{m_e^2\,e^4}{8\,\varepsilon_0^2\,h^3\,c}
                \frac{1836}{1837}
            \cong
                109677.6\,\unit{\per\centi\meter}
            &
        \end{flalign*}

    \end{sectionBox}


\end{sectionBox}


\begin{sectionBox}1{Generalizando para além do hidrogênio}

    \begin{BM}
            F_e = \frac{-e^2}{4\,\pi\,\varepsilon_0\,r^2}
        \to
            \frac{-Z\,e^2}{4\,\pi\,\varepsilon_0\,r^2}\quad\forall\,\{Z\}\in\mathbb{K}
    \end{BM}

\end{sectionBox}

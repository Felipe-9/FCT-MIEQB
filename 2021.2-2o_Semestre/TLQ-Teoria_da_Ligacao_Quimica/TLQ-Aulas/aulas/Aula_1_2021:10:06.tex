% Aula 1 - Introdução a TLQ e Teoria Atômica I
\part{Teoria Atômica I}

\section*{História}

% Grécia Antiga
\begin{sectionBox}1m{Do grego: átomos} % άτομο

    No \(5^o\) Século A.C o filosofo Leucippus de Miletus originou a filosofia atômica, seu discípulo Democritus de Abdera nomeou átomo significando literalmente indivisível, e caracterizou os átomos por possuirem tamanhos e formas diferentes atribuindo a matéria que eles formam suas características.

    A filosofia atômica nunca foi aceita por Aristotles e como sua filosofia deu origem a igreja cristã na europa, a igreja perseguiu aqueles que iam contra a filosofia aristotélica, atrasando bastante o desenvolvimento da teoria atômica.

\end{sectionBox}



% John Dalton
\begin{sectionBox}1m{Teoria Atómica de Dalton}

    Apenas no século 19 d.c a teoria atómica foi retomada com a publicação do livro \textit{A New System of Chemical Philosophy} de John Dalton com base no princípio da conservação de massa em reações químicas de Lavoisier, elevando o conceito filosófico de átomo para uma teoria química. Dentre os conteúdos de sua publicação se discutiam o seguinte:

    \begin{sectionBox}2{Postulados}

        \begin{enumerate}
            \item Elementos consistem de minusculas particulas sem carga, indestrutíveis e indivisíveis;
            \item Todos os átomos do mesmo elemento são iguais, diferentes elementos possuem diferentes tipos de átomos;
            \item Átomos não são nem criados nem destruídos;
            \item Diferentes átomos podem se juntar em simples proporções para formar ``átomos compostos''.
        \end{enumerate}

    \end{sectionBox}

\end{sectionBox}


% J. J. Thompson
\begin{sectionBox}1{J. J. Thomson: modelo pudim de passas}

    \begin{sectionBox}2{Pretexto: Experimentos com ampola de Crooks}

        Ampolas alongadas e vedadas onde se podia inserir gases e reduzir sua pressão com uma bomba de vácuo, alem de possuir um cátodo e um anodo de pilhas em cada uma das suas extremidades.

        Ao diminuir a pressão á 10\,\unit{\milli\meter\ch{Hg}} no interior de uma ampola preenchida com hidrogênio uma luz rosa passou a ser emitida pela ampola.

        ...

        % profit

    \end{sectionBox}

    \begin{itemize}
        \item átomos possuem pequenas partículas carregadas negativamente (elétrons)
        \item núcleo positivo constitui praticamente toda a massa do átomo
    \end{itemize}

\end{sectionBox}

\begin{sectionBox}1{Experimento da folha de ouro de Rutherford}

    Bloco contendo Rádio emissor de partículas \alpha (positivas) que colidem com uma fina lamina de ouro, verificando o desvio das partículas \alpha que poderiam apenas ser explicadas pela interação elétrica com o campo gerado pelo núcleo dos átomos de ouro, que para possuir um campo suficientemente forte, precisa ser pequeno.

    Para que o núcleo seja pequeno, os elétrons tem que orbitar ao seu redor, e se esse for o caso pelas leis de Maxuel partículas carregadas em aceleração perdem energia em forma de ondas eletromagnéticas e os elétrons se colidiria com o núcleo causando a % <nome bnt p colizão eletron e nucleo>

\end{sectionBox}

% Hantaro Nagaoka...primeira proposição do modelo planetário

% Teoria dos orbitais de Bohr
\begin{sectionBox}1m{Teoria orbital de Bohr}

    Usando o ``\textit{insight}'' de Rutherford

    % Postulados de Bohr
    \begin{itemize}
        \item Eletrons existem em estados estacionários
        \item Qualquer variação do eletron no estado estacionario implica em absorção e emição de ondas eletromagnéticas
        \item momento angular do eletron é quantizado
    \end{itemize}

    % Busca do raio atómico



\end{sectionBox}


% Termino slide 1.61

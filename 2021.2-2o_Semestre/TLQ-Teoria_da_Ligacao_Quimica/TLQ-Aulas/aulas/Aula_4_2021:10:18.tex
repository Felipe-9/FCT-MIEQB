% Aula 4 - Teoria Atômica IV
\part{}

% Revisão aula passada:
    % Teoria de shrondinger numa caixa bidimensional

    % Probabilidade de encontrar o eletron:
    % integral da função ao quadrado

% fim da revisão

% Aplicação da equação de onda para caixa bidimensional
    % psi agora tem x e y como parametros

    % \begin{BM}
    %     \mathscr{H}\psi_{(x,y)} = \frac{8\,\pi^2\,m(E_t-V)}{h^2}\,\psi_{(x,y)}
    %     \\
    %     \implies E = \frac{h^2}{8\,m}\left( \frac{n_1^2}{L_x^2} + \frac{n_2^2}{L_y^2} \right) : V = 0
    % \end{BM}

    % Aplicando para caixa bidimensional c numeros atômicos: (2,2), (2,1), (1,2)

% Aplicando para caixa cúbica


    \begin{BM}
            \mathscr{H}\psi_{(x,y,z)} = \frac{8\,\pi^2\,m(E_t-V)}{h^2}\,\psi_{(x,y,z)}
        \\
        \implies
            \psi_{(\mathbb{R}^3)}
        =   \prod_{k=1}^{3} \sqrt{2/L_{x_k}}\,\sin\left(\frac{n_k\,\pi}{L_{x_k}}\,x_k\right)
        \\
            \implies E
        =   \frac{h^2}{8\,m}
            \left(
                \sum_{k=1}^{3}\frac{n_k^2}{L_{x_k}^2}
            \right) : V = 0
    \end{BM}

% Perg: Para dimensões > 3 temos que prob de encontrar eletron < 1?
% Perg: resumir equações

% Hamiltonian
    \begin{BM}
        \mathscr{H}\,\psi_{\mathbb{R}^i}
    =   \left(
            \left(
                -\frac{h^2}{8\,\pi^2\,m_e}
            \right)\nabla^2
        +   V
        \right)
        \psi_{\mathbb{R}^i}
    =   E\,\psi_{\mathbb{R}^i}
    \end{BM}

% Coordenadas esféricas (\theta,\phi,\mathrm{r})
    % r      = raio
    % \theta = Vertical
    % \phi   = horizontal

% \begin{sectionBox}1{Coordenadas esféricas}
%     \begin{BM}
%
%     \end{BM}
% \end{sectionBox}

% Função de onda em coordenadas esféricas

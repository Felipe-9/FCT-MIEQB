% Aula 2 - Teoria Atômica II
\part{Teoria Atômica II}

\begin{multicols}{2}

    % Cont. teoria planetária dos atomos (bohr)
    \begin{sectionBox}1m{}

        % Niveis de energia (slide 1.70)
    \end{sectionBox}

    % Maxuel e ondas (Slide ~1.76 -> ~1.88)
    \begin{sectionBox}1m{}

        % Propriedades de ondas

        % relação invérsa com o raio

        % Refração
        \begin{sectionBox}2{}
            % alteração da amplitude e deslocação
        \end{sectionBox}

        % Interferencia
        \begin{sectionBox}2{}
            % se mantem mesma direção e sentido

            % Int. construtiva

            % Int. Destrutiva

        \end{sectionBox}

        % Difração
        \begin{sectionBox}2{}
            % Ordem de grandeza da largura do orificio =~ ampliude da onda
        \end{sectionBox}

        % Efeito fotoelétrico
        \begin{sectionBox}2{}

            % relação a quantidade de energica recebida e numero de eletrons emitidos

        \end{sectionBox}

    \end{sectionBox}

    % Le brog (slide ~100 -> ~120)
    \begin{sectionBox}1m{}

        % modificação da teoria de einstein: c -> v

        % representação de elentrons como onda

        % desenvolvimento de equações

    \end{sectionBox}

    % experimento comprovando q elentron é onda
    \begin{sectionBox}1m{}

        % Jönson
        % usando difração e interferencia
        % eletrons em 2 orificios gerando interferencias destrutivas e construtivas

        % Experiencia refeita com melhor equipamento
        % Merli Missiroli e Pozzi

    \end{sectionBox}

    % Estudo espectro de emissão (Slide ~1.140 -> 170)
    \begin{sectionBox}1m{}

        % Equação de Rydberg

        % Desenvolvimento de equações
    \end{sectionBox}


\end{multicols}

% Finalizou slide 170

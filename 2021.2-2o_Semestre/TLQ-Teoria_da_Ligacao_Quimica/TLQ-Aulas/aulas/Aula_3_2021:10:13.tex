% Aula 3 - Teoria Atômica III
\part{Teoria Atômica III}

% Heinzemberg

% Quanto menor o tempo que o elentron se contra excitado maior a incerteza da medida de sua energia quando ele liberar-la

% Energia              x tempo excitado
% Velocidade(momento?) x Posição

% Transição de orbitas => Orbita*i*s

% Teoria de Schrödinger

% Equação do maxuel q descreve fenomenos ondulatórios
\begin{sectionBox}2{Função de onda de Maxuel}
    \begin{BM}
        \odv[order={2}]{\psi(x)}{x}=-k\,\psi(x)
    \end{BM}
\end{sectionBox}

% ...

% Debrogli
\begin{sectionBox}{Debrogli}

    \begin{BM}
        \lambda = \frac{h}{p}=\frac{h}{m\,v}
    \end{BM}

    % desenvolvimento da equação de onda de maxuel usando o debrogli

    % A uma função de onda está relacionado com a energia
\end{sectionBox}

% Aplicar a teoria encontrada até agora em abstrações
% caixa unidimensional

% Finalizou a teoria para caixa unidimensional

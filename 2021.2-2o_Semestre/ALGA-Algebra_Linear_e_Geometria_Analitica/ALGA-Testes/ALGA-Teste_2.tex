% !TEX root = ./ALGA - Testes Resolução.tex
% ALGA - Teste 2 Resolução

\setcounter{part}{1}
\part{}

\section*{Enunciado das 3 questões seguintes}

\begin{BM}
    \begin{array}{c}
        F = \{(a,b,c,d) \in\mathbb{R}^4:b=a-d\land b+3\,c=0
    \\  G = \langle(0,1,1,0),(1,4,0,-2),(-1,2,6,2)\rangle
    \end{array}
\end{BM}

\begin{questionBox}1{}
    \begin{BM}
        S_i\in\base F\implies
        \left\{
            \begin{array}{lll}
            &   
                S_i \text{ é l.i.}
            &\land\\\land&
                S_i \subset F
            &\land\\\land&
                S_i \dots
            &
            \end{array}
        \right\}
    \end{BM}
    \begin{flalign*}
        &
            % (S3)
            % (i)
            1-(-1) \neq -1
            % (S4)
            % (i)
            3-0 = 3
            3 = -1*3
            % (ii)
            2 -(-1) =3
            0 -(-3) = 3
        &
    \end{flalign*}
\end{questionBox}

\begin{questionBox}1{}
    \begin{BM}
        \dim(F\Cap G)
    \end{BM}
    
    \begin{flalign*}
        &
            C_i\nin F\quad\forall\,i
        &
    \end{flalign*}
\end{questionBox}

\begin{questionBox}1{}
    \begin{BM}
        \base (F + G)
    \end{BM}
    
    \begin{flalign*}
        &
            S_i\subset \base(F+G)
        \implies
            \left\{
                \begin{array}{lll}
                &
                    S_i\subset\base F
                &\land\\\land&
                    S_i\subset\base G
                &
                \end{array}
            \right\{
        \implies
            S_2\in\base G
        &
    \end{flalign*}
\end{questionBox}

\begin{questionBox}1{}
    \begin{BM}
        A = 
        \begin{bmatrix}
             0 & 2 & 2 \\
            -1 & 3 & 2 \\
             0 & 0 & 1
        \end{bmatrix}
    \end{BM}
    
    \begin{flalign*}
        &
            \subespaco A:
            \left\{
                \begin{array}{lll}
                &
                    \subespaco A \text{ é l.i}
                &\land\\\land&
                    \subespaco A
                &
                \end{array}
            \right\}
        &
    \end{flalign*}
\end{questionBox}

\setcounter{question}{5}

\section*{Enunciado das seguintes 3 questões}
\begin{BM}
    \begin{array}{c}
        (x,y,z)\in r:
        \begin{cases}
            x=2 \\ y+z=2
        \end{cases}
    \\  
        (x,y,z)\in \pi_1: 
        (x,y,z) = (1,0,1) + \lambda(2,-1,0) + \mu(-2,0,1)\quad\{\lambda,\mu\}\in\mathbb{R}
    \\ 
        (x,y,z)\in\pi_2:3\,x-2\,y+z-1=0
    \end{array}
\end{BM}

% Q6
\begin{questionBox}1{}
    \begin{BM}
        \pi:
        \left\{
        \begin{array}{lll}
        &
            (1,-2,0)\in\pi
        &\land\\\land&
            (x,y,z)\in\pi: (x,y,z) = \alpha_1\,x+2\,y+\alpha_2\,z+\alpha_3=0
        &\land\\\land&
            \text{paralelo a }\pi_1
        &
        \end{array}
        \right.
    \end{BM}
    \begin{flalign*}
        &
            (1,-2,0)\in\pi
        \implies
            \alpha_1+2\,(-2)+\alpha_3 = 0
        \implies
            \alpha_2 = 4-\alpha_3
        &\\&
            (x,y,z)\in\pi_1:(x,y,z)= 3\,x-2\,y+z-1=0
        \implies
            \{
                (0,0,1),
                (1/3,0,0),
                (0,-1/2,0)
            \}\in\pi_1;
        \implies
        %
            (x,y,z)\in\pi_1:(x,y,z) = (0,0,1) + \lambda_1(-1/3,0,1) + \lambda_2(0,-1/2,1)
        \implies
            (x,y,z)\in\pi:
            (x,y,z) = (1,-2,0) + \lambda_1(-1/3,0,1) + \lambda_2(0,-1/2,1)
        \implies
            \left\{
                \begin{array}{l}
                    x=1-\lambda_1
                \\  y=-2-\lambda_2/2
                \\  z=\lambda_1+\lambda_2
                \end{array}
            \right\}
        \implies
            x + 2y + z + 3=0
        &
    \end{flalign*}
\end{questionBox}

\begin{questionBox}1{}
    Distancia entre \(r\) e \(\pi_1\)
    
    \begin{flalign*}
        &
            
        &
    \end{flalign*}
\end{questionBox}

\setcounter{question}{10}

% Q11
\beign{questionBox}1{}
    \begin{BM}
        f:\mathbb{R}^3\to\mathbb{R}_2[x]
    \end{BM}
    
    % a
    \begin{BM}
        \mathcal{M} (f;\mathcal{B},(x^2,x,1))
        =   \begin{bmatrix}
                0 & -1 & 2
            \\  1 & -2 & 1
            \\  -2 & 1 & 0
            \end{bmatrix}
    \end{BM}
    \begin{flalign*}
        &
            \begin{bmatrix}
                0 & -1 & 2
            \\  1 & -2 & 1
            \\  -2 & 1 & 0
            \end{bmatrix}
            \begin{bmatrix}
                1 \\ 0 \\ 1
            \end{bmtraix}
        =   -2\,x^2 + 2\,x - 2
        \neq x-2
        &
    \end{flalign*}
    
    % b
    \begin{BM}
        ((0,-1,3)) = \nuc f
    \end{BM}
    \begin{flalign*}
        &
            \implies f(0,-1,3) 
        =   
        =   0_{\mathbb{R}_2[x]}
        &
    \end{flalign*}
    
\end{questionBox}
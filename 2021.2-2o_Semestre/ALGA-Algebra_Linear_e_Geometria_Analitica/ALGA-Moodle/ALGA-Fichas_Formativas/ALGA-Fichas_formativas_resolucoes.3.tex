% !TEX root = ./ALGA-Fichas_formativas_resoluções.3.tex
\providecommand\mainfilename{"./ALGA-Fichas_formativas_resoluções.tex"}
\providecommand \subfilename{}
\renewcommand   \subfilename{"./ALGA-Fichas_formativas_resoluções.3.tex"}
\documentclass[\mainfilename]{subfiles}

% \graphicspath{{\subfix{../images/}}}
% \tikzset{external/force remake=true} % - remake all

\begin{document}

\mymakesubfile{3}
[ALGA]
{Ficha Formativa}
{Ficha Formativa}

% Q1
\begin{questionBox}1m{}

    % Q1 - a)
    \begin{questionBox}2m{}

        % i
        \begin{questionBox}3{A}
            \begin{flalign*}
                &
                    \begin{bmatrix}
                         1 & 1 & 0 &  1
                    \\   0 & 1 & 2 &  0
                    \\  -1 & 1 & 3 & -1
                    \\   1 & 0 & 1 &  2
                    \end{bmatrix}
                    \xrightarrow[
                        \begin{array}{c}
                            l_3 \mathrel{{+}{=}} l_1
                        \\  l_4 \mathrel{{+}{=}} -l_1
                        \\  l_2 \mathrel{{+}{=}} l_4
                        \\  l_3 \mathrel{{+}{=}} 2\,l_4
                        \end{array}
                    ]{}
                    \begin{bmatrix}
                         1 &  1 & 0 &  1
                    \\   0 &  0 & 3 &  1
                    \\   0 &  0 & 5 &  2
                    \\   0 & -1 & 1 &  1
                    \end{bmatrix}
                &\\&
                    \therefore\exists\,A^{-1}
                &
            \end{flalign*}
        \end{questionBox}

        % ii
        \begin{questionBox}3{M}
            \begin{flalign*}
                &
                    \begin{bmatrix}
                         2 &  1 & 4
                    \\   1 & -1 & 1
                    \\  -1 &  4 & 1
                    \end{bmatrix}
                    \xrightarrow[
                        \begin{array}{c}
                            l_3 \mathrel{{+}{=}} l_2
                        \\  l_1 \mathrel{{+}{=}} -2\,l_2
                        \end{array}
                    ]{}
                    \begin{bmatrix}
                         0 &  3 & 2
                    \\   1 & -1 & 1
                    \\   0 &  3 & 2
                    \end{bmatrix}
                &\\&
                    \therefore\nexists\,M^{-1}
                &
            \end{flalign*}
        \end{questionBox}

    \end{questionBox}
    \begin{questionBox}m{}
        
        % iii
        \begin{questionBox}3{N}
            \begin{flalign*}
                &
                    N\in\mathcal{M}_{3\times4}
                % &\\&
                    \therefore\nexists\,N^{-1}
                &
            \end{flalign*}
        \end{questionBox}

        % iv
        \begin{questionBox}3{P}
            \begin{flalign*}
                &
                    \begin{bmatrix}
                        1 & 0 & 1
                    \\  1 & 0 & 2
                    \\  1 & 1 & 1
                    \end{bmatrix}
                    \xrightarrow[
                        \begin{array}{c}
                            l_3 \mathrel{{+}{=}} -l_1
                            l_2 \mathrel{{+}{=}} -l_1
                        \end{array}
                    ]{}
                    \begin{bmatrix}
                        1 & 0 & 1
                    \\  0 & 0 & 1
                    \\  0 & 1 & 0
                    \end{bmatrix}
                &\\&
                    \therefore\exists\,P^{-1}
                &
            \end{flalign*}
        \end{questionBox}

        % Q
        \begin{questionBox}3{Q}
            \begin{flalign*}
                &
                    \begin{bmatrix}
                        2 & 2 & 1
                    \\  1 & 1 & 2
                    \\  1 & 2 & 1
                    \end{bmatrix}
                    \xrightarrow[
                        \begin{array}{c}
                            l_1 \mathrel{{+}{=}} -2\,l_2
                            l_3 \mathrel{{+}{=}} -l_2
                            l_2 \mathrel{{+}{=}} -l_3
                        \end{array}
                    ]{}
                    \begin{bmatrix}
                        0 & 0 & -3
                    \\  1 & 0 &  3
                    \\  0 & 1 & -1
                    \end{bmatrix}
                &\\&
                    \therefore\exists\,Q^{-1}
                &
            \end{flalign*}
        \end{questionBox}

    \end{questionBox}

    % Q1 - b)
    \begin{questionBox}2{}
        Pelas mesmas rasões que a): A, P, Q.
    \end{questionBox}

    % Q1 - c)
    \begin{questionBox}2{}
        \begin{BM}
            \begin{bmatrix}
                 1 & 1 & 0 &  1
            \\   0 & 1 & 2 &  0
            \\  -1 & 1 & 3 & -1
            \\   1 & 0 & 1 &  2
            \end{bmatrix}
            \begin{bmatrix}
                x_1 \\ x_2 \\ x_3 \\ x_4
            \end{bmatrix}
            =
            \begin{bmatrix}
                1 \\ 3 \\ 3 \\ 2
            \end{bmatrix}
        \end{BM}

        \begin{flalign*}
            &
                \begin{bmatrix}
                     1 & 1 & 0 &  1  &|&  1
                \\   0 & 1 & 2 &  0  &|&  3
                \\  -1 & 1 & 3 & -1  &|&  3
                \\   1 & 0 & 1 &  2  &|&  2
                \end{bmatrix}
                \xrightarrow[
                    \begin{array}{l}
                        l_3 \mathrel{{+}{=}} l_1
                    \\  l_4 \mathrel{{+}{=}} -l_1
                    \\  l_2 \mathrel{{+}{=}} l_4
                    \\  l_3 \mathrel{{+}{=}} 2\,l_4
                    \\  l_3 \mathrel{{+}{=}} -2\,l_2
                    \\  l_2 \mathrel{{+}{=}} 3\,l_3
                    \\  l_4 \mathrel{{+}{=}} l_3 - l_2
                    \\  l_1 \mathrel{{+}{=}} -l_2 + l_4
                \end{array}
                ]{}
                \begin{bmatrix}
                     1 &  0 &  0 & 0  &|&   6
                \\   0 &  0 &  0 & 1  &|&  -2
                \\   0 &  0 & -1 & 0  &|&  -2
                \\   0 & -1 &  0 & 0  &|&   3
                \end{bmatrix}
            &\\&
                \therefore X =
                \begin{bmatrix}
                    6 \\ -3 \\ 2 \\ -2
                \end{bmatrix}
            &
        \end{flalign*}
    \end{questionBox}

    % Q1 - d)
    \begin{questionBox}2{}
        \begin{BM}
            \begin{bmatrix}
                 1 & 1 & 0 & 1
            \\   0 & 2 & 1 & 1
            \\  -1 & 1 & 1 & 3
            \end{bmatrix}
            \begin{bmatrix}
                x_1 \\ x_2 \\ x_3
            \end{bmatrix}
            =
            \begin{bmatrix}
                2 \\ -1 \\ 0
            \end{bmatrix}
        \end{BM}

        \begin{flalign*}
            &
                \begin{bmatrix}
                     1 & 1 & 0 & 1  &|&   2
                \\   0 & 2 & 1 & 1  &|&  -1
                \\  -1 & 1 & 1 & 3  &|&   0
                \end{bmatrix}
                \xrightarrow[
                    \begin{array}{l}
                        l_3 \mathrel{{+}{=}} l_1
                    \\  l_3 \mathrel{{+}{=}} -l_2
                    \end{array}
                ]{}
                \begin{bmatrix}
                     1 & 1 & 0 & 1  &|&   2
                \\   0 & 2 & 1 & 1  &|&  -1
                \\   0 & 0 & 0 & 3  &|&   3
                \end{bmatrix}
            &\\&
                \implies r(N) = r(N|C) < 4
            &\\&
            \quad\therefore\text{Sistema possível indeterminado com grau de indeterminação 1}
            &
        \end{flalign*}
    \end{questionBox}

\end{questionBox}

\begin{questionBox}1{}

    \begin{BM}
        A_{(\alpha)}
    =   \begin{bmatrix}
            1 & 0 & 1
        \\  2 & 1 & \alpha
        \\  3 & 1 & 2\,\alpha
        \end{bmatrix}:\alpha\in\mathbb{R}
    \end{BM}

    \begin{questionBox}2{}
        \begin{flalign*}
            &
                \begin{bmatrix}
                    1 & 0 & 1
                \\  2 & 1 & \alpha
                \\  3 & 1 & 2\,\alpha
                \end{bmatrix}
                \xrightarrow[
                    \begin{array}{l}
                        l_3 \mathrel{{+}{=}} -2\,l_2
                    \\  l_3 \mathrel{{+}{=}} l_1
                    \\  l_2 \mathrel{{+}{=}} -2\,l_1 + l_3
                    \end{array}
                ]{}
                \begin{bmatrix}
                     1 &  0 & 1
                \\   0 &  0 & \alpha-1
                \\   0 & -1 & 1
                \end{bmatrix}
            &\\&
                \therefore\alpha\neq1
            &
        \end{flalign*}
    \end{questionBox}

    \begin{questionBox}2{}
        \begin{flalign*}
            &
                \begin{bmatrix}
                        1 & 0 & 1  &|&  1 & 0 & 0
                    \\  2 & 1 & 2  &|&  0 & 1 & 0
                    \\  3 & 1 & 4  &|&  0 & 0 & 1
                \end{bmatrix}
                \xrightarrow[
                    \begin{array}{l}
                        l_2 \mathrel{{+}{=}} -2\,l_1
                    \\  l_3 \mathrel{{+}{=}} -3\,l_1
                    \\  l_3 \mathrel{{+}{=}} -l_2
                    \\  l_1 \mathrel{{+}{=}} -l_3
                    \end{array}
                ]{}
                \begin{bmatrix}
                        1 & 0 & 0  &|&   2 &  1 & -1
                    \\  0 & 1 & 0  &|&  -2 &  1 &  0
                    \\  0 & 0 & 1  &|&  -1 & -1 &  1
                \end{bmatrix}
            &
        \end{flalign*}
    \end{questionBox}


\end{questionBox}


\begin{questionBox}1{}

    \begin{BM}
        \begin{bmatrix}
            1 & 0     & 0     & 1         &|&  1
        \\  2 & 1     & 0     & \alpha    &|&  2
        \\  0 & 1     & \beta & \alpha-2  &|&  \beta
        \\  4 & \beta & 0     & 4         &|&  \alpha-4
        \end{bmatrix}
    \end{BM}

    \begin{flalign*}
        &
            \begin{bmatrix}
                    1 & 0     & 0     & 1         &|&  1
                \\  2 & 1     & 0     & \alpha    &|&  2
                \\  0 & 1     & \beta & \alpha-2  &|&  \beta
                \\  4 & \beta & 0     & 4         &|&  \alpha-4
            \end{bmatrix}
            \xrightarrow[
                \begin{array}{l}
                    l_4 \mathrel{{+}{=}} -4\,l_1
                \\  l_3 \mathrel{{+}{=}} -l_2
                \\  l_2 \mathrel{{+}{=}} -2\,l_1
                \\  l_3 \mathrel{{+}{=}} 2\,l_1
                \\  l_3 \to l_3/\beta
                \end{array}
            ]{}
            \begin{bmatrix}
                     1 & 0     & 0 &  1           &|&  1
                \\   0 & 1     & 0 &  \alpha-2    &|&  0
                \\   0 & 0     & 1 &  0           &|&  1
                \\   0 & \beta & 0 &  0           &|&  \alpha
            \end{bmatrix}
        &
    \end{flalign*}


\end{questionBox}

\end{document}
% !TEX root = ./ALGA-Fichas_formativas_resoluções.6.tex
\providecommand\mainfilename{"./ALGA-Fichas_formativas_resoluções.tex"}
\providecommand \subfilename{}
\renewcommand   \subfilename{"./ALGA-Fichas_formativas_resoluções.6.tex"}
\documentclass[\mainfilename]{subfiles}

% \graphicspath{{\subfix{../images/}}}
% \tikzset{external/force remake=true} % - remake all

\begin{document}

\mymakesubfile{6}
[ALGA]
{Ficha Formativa}
{Ficha Formativa}

\begin{questionBox}1m{}
    \begin{BM}[align*]
        &
            \left.
                \begin{array}{l}
                    u_1=(1,1,0,1)
                \\  u_2=(0,0,1,0)
                \\  u_3=(1,0,1,0)
                \end{array}
            \right\}
            \in\mathbb{R}^4
        \\&
            \mathcal{B}\phantom{'}
        =   (
                (0, 1,  0, 0),
                (0, 0, -2, 0),
                (1, 1,  0, 0),
                (0, 0,  0, 2)
            )
        \\&
            \mathcal{B}'
        =   (
                (2,  2, -4,  1),
                (0,  0,  1,  0),
                (0, -1,  2,  1),
                (0,  0,  2, -3)
            )
    \end{BM}

    \begin{questionBox}2{}
        \begin{BM}
            y(w):(x,y,z,w)\in\langle u_1,u_2,u_3\rangle
        \end{BM}
        \begin{flalign*}
            &
                \begin{bmatrix}
                    1 & 0 & 1
                \\  1 & 0 & 0
                \\  0 & 1 & 1
                \\  1 & 0 & 0
                \end{bmatrix}
                \begin{bmatrix}
                    \alpha_1\\\alpha_2\\\alpha_3
                \end{bmatrix}
            =   \begin{bmatrix}
                    x\\ y\\ z\\ w
                \end{bmatrix}
            \implies
                \alpha_3=y
            \land
                \alpha_3=w
            \implies
                y=w
            &
        \end{flalign*}
    \end{questionBox}

    \begin{questionBox}2{}
        \begin{BM}
            t:(u_1,u_2,u_3,(1,0,3,t))\text{ é base de }\mathbb{R}^4
        \end{BM}

        \begin{flalign*}
            &
                t:r
                \left(
                    \begin{bmatrix}
                        1 & 0 & 1 & 1
                    \\  1 & 0 & 0 & 0
                    \\  0 & 1 & 1 & 3
                    \\  1 & 0 & 0 & t
                    \end{bmatrix}
                \right)
            =   4;
            \begin{bmatrix}
                1 & 0 & 1 & 1
            \\  1 & 0 & 0 & 0
            \\  0 & 1 & 1 & 3
            \\  1 & 0 & 0 & t
            \end{bmatrix}
            \xrightarrow[
                \begin{subarray}{l}
                    l_4 += -l_2
                \\  l_1 += -l_4/t-l_2
                \\  l_3 += -l_1-l_4\,3/t
                \end{subarray}
            ]{}
            \begin{bmatrix}
                0 & 0 & 1 & 0
            \\  1 & 0 & 0 & 0
            \\  0 & 1 & 0 & 0
            \\  0 & 0 & 0 & t
            \end{bmatrix}
            \implies &\\&
            \implies
                t = \mathbb{R}\backslash\{1\}
            &
        \end{flalign*}
    \end{questionBox}

    \begin{questionBox}2{}
        \begin{BM}
            v\in\mathbb{R}^4:
            \begin{array}{l}
                (u_1,u_2,u_3,v)\text{ é base de }\mathbb{R}^4
            \,\land\\\land
                \langle u_1,u_2,u_3,v \rangle (3,0,-1,-1)=(2,-1,1,-1)
            \end{array}
        \end{BM}

        \begin{flalign*}
            &
                v=(v_1,v_2,v_3,v_4)
            \land
                \begin{bmatrix}
                    0 & 0 & 1 & v_1
                \\  1 & 0 & 0 & v_2
                \\  0 & 1 & 0 & v_3
                \\  0 & 0 & 0 & v_4
                \end{bmatrix}
                \begin{bmatrix}
                    3\\0\\-1\\-1
                \end{bmatrix}
            =   \begin{bmatrix}
                    2\\-1\\1\\-1
                \end{bmatrix}
            \implies &\\&
            \implies
                v
            =   \begin{bmatrix}
                    -1-2
                \\  3+1
                \\  -1
                \\  1
                \end{bmatrix}
            =   (-3,4,-1,1)
            &
        \end{flalign*}
    \end{questionBox}

    \begin{questionBox}2{}
        \begin{BM}
            x:\mathcal{B}'v=x\land \mathcal{B}v=(0,4,4,1)
        \end{BM}

        \begin{flalign*}
            &
                x
            =   \mathcal{B}'\mathcal{B}^{-1}(0,4,4,1)
            =   \begin{bmatrix}
                    2 &  2 & -4 &  1
                \\  0 &  0 &  1 &  0
                \\  0 & -1 &  2 &  1
                \\  0 &  0 &  2 & -3
                \end{bmatrix}
                \frac{\adj \mathcal{B}}{\det\mathcal{B}}
                \begin{bmatrix}
                    0\\4\\4\\1
                \end{bmatrix}
            = &\\&
            =   \begin{bmatrix}
                    2 &  2 & -4 &  1
                \\  0 &  0 &  1 &  0
                \\  0 & -1 &  2 &  1
                \\  0 &  0 &  2 & -3
                \end{bmatrix}
                \left(
                    \left(
                        1*(-1)^{1+2}
                        \begin{vmatrix}
                            0 & -2 & 0
                        \\  1 &  0 & 0
                        \\  0 &  0 & 2
                        \end{vmatrix}
                    \right)^{-1}
                    \begin{bmatrix}
                         4 &  0 & -4 &  0
                    \\  -4 &  0 &  8 &  0
                    \\   0 & -2 &  0 &  0
                    \\   0 &  0 &  0 &  2
                    \end{bmatrix}
                \right)
            &\\&
                \begin{bmatrix}
                    0\\4\\4\\1
                \end{bmatrix}
            =   \begin{bmatrix}
                    2 &  2 & -4 &  1
                \\  0 &  0 &  1 &  0
                \\  0 & -1 &  2 &  1
                \\  0 &  0 &  2 & -3
                \end{bmatrix}
                \begin{bmatrix}
                    -4/4 &  0   &  4/4 &  0
                \\   4/4 &  0   & -8/4 &  0
                \\   0   &  2/4 &  0   &  0
                \\   0   &  0   &  0   & -2/4
                \end{bmatrix}
                \begin{bmatrix}
                    0\\4\\4\\1
                \end{bmatrix}
            = &\\&
            =   \begin{bmatrix}
                    2 &  2 & -4 &  1
                \\  0 &  0 &  1 &  0
                \\  0 & -1 &  2 &  1
                \\  0 &  0 &  2 & -3
                \end{bmatrix}
                \begin{bmatrix}
                     4
                \\  -8
                \\   2
                \\  -2/4
                \end{bmatrix}
            =   \begin{bmatrix}
                    -33/2
                \\  2
                \\  23/2
                \\  11/2
                \end{bmatrix}
            &
        \end{flalign*}
    \end{questionBox}


\end{questionBox}

\end{document}

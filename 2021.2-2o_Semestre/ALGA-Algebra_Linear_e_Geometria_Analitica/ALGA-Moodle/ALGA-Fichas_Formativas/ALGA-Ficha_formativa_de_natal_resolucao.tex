% !TEX root = ./ALGA - Fichas formativas resoluções.tex
% ALGA - Resolução da ficha formativa de natal

\addtocounter{part}
\renewcommand\thepart{Natal}
\part{}

% Q1
\begin{questionBox}1m{}
    
    \begin{questionBox}3{}
        
        \begin{BM}
            (x+x^2,1+x,1,1+x+x^2)
        \end{BM}
        
        é uma sequencia linearmente independente de vetores de \(\mathbb{R}_2[x]\)
        
        \begin{flalign*}
            &
                (1+x)*x = x+x^2
            &\\&
                \therefore
                \text{False}
            &
        \end{flalign*}
        
    \end{questionBox}
    
    \begin{questionBox}3{}
        
        \begin{BM}
            (1-x+x^2,1-2\,x+x^2)
        \end{BM}
        
        é uma subsequencia de alguma base de \(\mathbb{R}_2[x]\)
        
        \begin{flalign*}
            &
                (1-x+x^2,1-2\,x+x^2)
            =   
            &
        \end{flalign*}
        
    \end{questionBox}
    
    \begin{questionBox}3{}
        
        \begin{BM}
            (0+0\,x+0\,x^2)
        \end{BM}
        
        é uma base do subespaço nulo de \(\mathbb{R}_2[x]\)
        
        falso por pertencer a \(\mathbb{R}[x]\)
        
    \end{questionBox}
    
    \begin{questionBox}3{}
        
        \begin{BM}
                \lang
                    x+x^2,1+x,1,1+x+x^2
                \rang
            =
                \mathbb{R}_2[x]
        \end{BM}
        
        \begin{flalign*}
            &
            \implies
                0_{\mathbb{R}_2[x]}\in
                \lang
                    x+x^2,1+x,1,1+x+x^2
                \rang
            \land
                \begin{bmatrix}
                    x+x^2
                \\  1+x
                \\  1
                \\  1+x+x^2
                \end{bmatrix}
                \begin{bmatrix}
                    
                \end{bmatrix}
            &
        \end{flalign*}
        
    \end{questionBox}
    
\end{questionBox}










% !TEX root = ./ALGA-Listas_Resolução.1.tex
\providecommand\mainfilename{"./ALGA-Listas_Resolução.tex"}
\providecommand \subfilename{}
\renewcommand   \subfilename{"./ALGA-Listas_Resolução.1.tex"}
\documentclass[\mainfilename]{subfiles}

% \graphicspath{{\subfix{../images/}}}
% \tikzset{external/force remake=true} % - remake all

\begin{document}

\mymakesubfile{1}
[ALGA]
{Lista: Matrizes}
{Lista: Matrizes}

% Q1.1
\begin{questionBox}1{}
    \begin{multicols}{2}
        \begin{questionBox}2{}
            B, E, F, H, I
        \end{questionBox}
        \begin{questionBox}2{}
            B, E, F, H, I
        \end{questionBox}
        \begin{questionBox}2{}
            B, E, F, I
        \end{questionBox}
        \begin{questionBox}2{}
            B, E
        \end{questionBox}
    \end{multicols}
\end{questionBox}



% 1.2
\begin{questionBox}1{}
    \begin{multicols}{2}

        \begin{questionBox}2{}
            \begin{flalign*}
                &
                    \begin{bmatrix}
                        0 & 1 & 1
                    \\  1 & 0 & 1
                    \\  1 & 1 & 0
                    \end{bmatrix}
                &
            \end{flalign*}
        \end{questionBox}


        \begin{questionBox}2{}
            \begin{flalign*}
                &
                    \begin{bmatrix}
                        0 & -1 & -1
                    \\  1 &  0 & -1
                    \\  1 &  1 &  0
                    \end{bmatrix}
                &
            \end{flalign*}
        \end{questionBox}


        \begin{questionBox}2{}
            \begin{flalign*}
                &
                    \begin{bmatrix}
                         1 & -1 &  1
                    \\  -1 &  1 & -1
                    \\   1 & -1 &  1
                    \end{bmatrix}
                &
            \end{flalign*}
        \end{questionBox}

    \end{multicols}
\end{questionBox}



% 1.3
\begin{questionBox}1m{}
    \begin{multicols}{2}

        \begin{questionBox}2{}
            \begin{flalign*}
                &
                =   \begin{bmatrix}
                         4 & 1 &  5
                    \\  -2 & 1 & -1
                    \end{bmatrix}
                &
            \end{flalign*}
        \end{questionBox}


        \begin{questionBox}2{}
            \begin{flalign*}
                &
                =   \begin{bmatrix}
                         8 & 2 &  10
                    \\  -2 & 2 & - 2
                    \end{bmatrix}
                &
            \end{flalign*}
        \end{questionBox}

        \begin{questionBox}2{}
            \begin{flalign*}
                &
                =   \begin{bmatrix}
                        2 &  1 & -4
                    \\  2 & -1 &  0
                    \end{bmatrix}
                &
            \end{flalign*}
        \end{questionBox}


        \begin{questionBox}2{}
            \begin{flalign*}
                &
                =   \begin{bmatrix}
                         0 & 2 & -15
                    \\  11 & 2 & - 2
                    \end{bmatrix}
                &
            \end{flalign*}
        \end{questionBox}

    \end{multicols}
\end{questionBox}



% Q1.4
\begin{questionBox}1{}
    \begin{flalign*}
        &
        =   \begin{bmatrix}
                2 & 1 & 1
            \\  1 & 2 & 1
            \\  1 & 1 & 2
            \end{bmatrix}
        &
    \end{flalign*}
\end{questionBox}



% 1.10
\setcounter{question}{9}
\begin{questionBox}1m{}

    \setcounter{subquestion}{2}
    % c)
    \begin{questionBox}2{}

        \begin{BM}
            (AB)^{(i)}=(AB)^{(j)}\because A^{(j)}=A^{(i)}\quad\forall\,i\neq j
        \end{BM}

        \begin{flalign*}
            &
                A_i = A_j
            \land
                AB_{k_1,k_2} = \sum_{k=1}^{n} a_{k_1,k}\,b_{k,k_2}
            \implies &\\&
            \implies
                (AB)_{i,k_2} = \sum_{k=1}^{n} a_{i,k}\,b_{k,k_2}
                = &\\&
                = \sum_{k=1}^{n} a_{j,k}\,b_{k,k_2}
                = (AB)_{j,k_2}
            \implies &\\&
            \implies
                (AB)_i = (AB)_j
            &
        \end{flalign*}

    \end{questionBox}

    % d)
    \begin{questionBox}2{}
        \begin{BM}
                B^k = B^l : k\neq l
            \implies \\
            \implies
                (AB)^k = (AB)^l
        \end{BM}
    \end{questionBox}

\end{questionBox}



% Q1.9
\setcounter{question}{8}
\begin{questionBox}1{}

    \begin{BM}[align*]
        &
            \{D,D'\} \in \mathcal{M}_{n\times n}(\mathrm{K}) :
        \\&
            d_{i,j} = 0
        \land
            d'_{i,j}\forall\,i\neq j
        \implies \\&
        \implies
            (DD')_{i,j}=0\quad\forall\,i\neq j
    \end{BM}

    \begin{flalign*}
        &
            \{D,D'\} \in \mathcal{M}_{n\times n}(\mathrm{K}) : d_{i,j} = 0 \land d'_{i,j} \ \forall\,i\neq j;
        &\\&
            (DD')_{i,j} = \sum_{k=1}^{n} d_{i,k}\,d'_{k,j}
        \implies &\\&
        \implies
            (DD')_{i,j} = 0\ \forall\,\{i,j\}\in\mathbb{K}:i\neq j
        &
    \end{flalign*}

\end{questionBox}



% Q 1.20
\begin{questionBox}1{Indique\dots}
    \begin{questionBox}2{Uma Condição para que uma matriz diag. seja invert.}
        \begin{flalign*}
            &
                A\in\mathcal{M}_{n\times n}:a_{i,j}=0\quad\forall\,i\neq j
            \land &\\&
            \land
                \exists\,A^{-1}: AA^{-1} = I_n
            \impliedby &\\&
            \impliedby
                a_{i,j}\neq 0\quad\forall\,i=j
            &
        \end{flalign*}
    \end{questionBox}

    \begin{questionBox}2{}
        % (D)_{i,i} = (d_{i,i})^{-1}
    \end{questionBox}
\end{questionBox}



% Q 1.108
\setcounter{question}{107}
\begin{questionBox}1{}

    \begin{BM}
        J_n\in\mathcal{M}_{n\times s}(\mathbb{K}): (J_n)_{i,j} = 1
        \\
        \forall\,\{i,j\}\in\mathbb{K}
    \end{BM}

\end{questionBox}



% Q 1.22
\setcounter{question}{21}
\begin{questionBox}1{}

    \begin{BM}
        A\in\mathcal{M}_{n\times n}(\mathbb{K})
    \end{BM}

    \begin{questionBox}2{}
        \begin{BM}
            A^3=I_n
        \end{BM}
    \end{questionBox}

    \begin{questionBox}2{}
        \begin{BM}
            A^2 + 2\,A = I_n
        \end{BM}
    \end{questionBox}

    \begin{questionBox}2{}
        \begin{BM}
            A^2+\alpha\,A+\beta\,I_n = 0
            \\
            :\alpha\in\mathbb{K}\land\beta\in\mathbb{K}\backslash\{0\}
        \end{BM}

    \end{questionBox}

    % ...
\end{questionBox}



% Q1.34
\setcounter{question}{33}
\begin{questionBox}1{}

    \begin{questionBox}2{}
        \(A\) e \(C\)
    \end{questionBox}

    \begin{questionBox}2{}
        \(E\)
    \end{questionBox}
\end{questionBox}

% Q1.37
\setcounter{question}{36}
\begin{questionBox}1{}
    \begin{questionBox}2{}
        \begin{flalign*}
            &
                A\in\mathcal{M}_{n\times n}(\mathbb{K}): a_{i,j} = 0\quad\forall\{i,j\}
            &
        \end{flalign*}
    \end{questionBox}

    \begin{questionBox}2{}
        ...
    \end{questionBox}

\end{questionBox}


% Q1.129
\setcounter{question}{128}
\begin{questionBox}1{}

    \begin{questionBox}2{}

        \begin{questionBox}3{}
            \begin{flalign*}
                &
                 (A + A^T)
                =   ((A + A^T)^T)^T
                =   (A^T+ A)^T
                =   (A + A^T)^T
                &\\&
                \therefore\,(A + A^T)\text{ é simétrica}
                &
            \end{flalign*}
        \end{questionBox}

        \begin{questionBox}3{}
            %\begin{flalign*}
             %   &

             %   &
            %\end{flalign*}
        \end{questionBox}

     \end{questionBox}

     \begin{questionBox}2{}

     \end{questionBox}
\end{questionBox}


% Q1.42
\setcounter{question}{41}
\begin{questionBox}1{}
    a = s, III
    b = s, II
    c = s, I
    d = n
    e = s, II
\end{questionBox}

% Q1.43
\begin{questionBox}1{}
    \begin{multicols}{2}

        \begin{questionBox}2{}
            \begin{BM}
                \begin{bmatrix}
                    0 & 0 & 1
                \\  0 & 1 & 0
                \\  1 & 0 & 0
                \end{bmatrix}
            \end{BM}
        \end{questionBox}

        \begin{questionBox}2{}
            \begin{BM}
                \begin{bmatrix}
                    6 & 0 & 0
                \\  0 & 1 & 0
                \\  0 & 0 & 1
                \end{bmatrix}
            \end{BM}
        \end{questionBox}

        \begin{questionBox}2{}
            \begin{BM}
                \begin{bmatrix}
                    1 & 0   & 0
                \\  0 & 1   & 0
                \\  0 & 1/5 & 1
                \end{bmatrix}
            \end{BM}
        \end{questionBox}

    \end{multicols}
\end{questionBox}


% Q1.45 a)
\setcounter{question}{44}
\begin{questionBox}1{}

    \begin{questionBox}2{}
        \begin{BM}
            A =
            \begin{bmatrix}
                0 & 1 & 0
            \\  2 & 0 & 1
            \\  0 & 0 & 1
            \end{bmatrix}
        \end{BM}
    \end{questionBox}

\end{questionBox}

% Q1.48
\setcounter{question}{47}
\begin{questionBox}1{}
    A = s
    B = n
    C = s
    D = n
\end{questionBox}


% Q1.51
\setcounter{question}{50}
\begin{questionBox}1{}
    A = s
    B = s
    C = n
    D = s
    E = s
\end{questionBox}


% Q1.49
\setcounter{question}{48}
\begin{questionBox}1{}
    \begin{questionBox}2{}
        \begin{flalign*}
            &
                A'
            =   \begin{bmatrix}
                     1 &  2 &  1
                \\   0 & -1 & -2
                \\   0 &  2 &  2
                \end{bmatrix}
            \xleftarrow[
                \begin{array}{l l}
                    l_2 \to l_2 - 2\,l_1
                \end{array}
            ]{}
            \begin{bmatrix}
                     1 &  2 &  1
                \\   0 & -1 & -2
                \\   0 &  2 &  2
                \end{bmatrix}
            \xleftarrow[
                \begin{array}{l l}
                    l_2 \to l_2 - 2\,l_1
                \\  l_3 \to l_3 + l_1
                \end{array}
            ]{}
            &\\&
            \leftarrow
                \begin{bmatrix}
                     1 & 2 & 1
                \\   2 & 1 & 0
                \\  -1 & 0 & 1
                \end{bmatrix}
            &
        \end{flalign*}
    \end{questionBox}
\end{questionBox}


% Q1.171
\setcounter{question}{170}
\begin{questionBox}1{}
    \begin{flalign*}
        &
            [A|I]
        =   \begin{bmatrix}
                1 & -1 & 0 & 1 & 0 & 0
            \\  2 &  0 & 1 & 0 & 1 & 0
            \\  0 &  2 & 0 & 0 & 0 & 1
            \end{bmatrix}
         \xrightarrow[l2 += -2\,l1]{}
        &
    \end{flalign*}
\end{questionBox}

\end{document}

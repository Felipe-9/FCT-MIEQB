% Aula 1 2021/10/12 - ...
\part{}

% boa parte da aula teve anotação perdida, revisar slides e/ou livro

\begin{multicols}{2}

\begin{sectionBox}1m{Matrizes Invertíveis}

    % ...

    % Propriedades

    % \begin{sectionBox}{}
    %
    % \end{sectionBox}

    % (A B)(B^{-1} A^{-1})
    \begin{sectionBox}2{\( (A\,B)(B^{-1}\,A^{-1}) \)}
        \begin{flalign*}
            &
            =   A\,(B\,B^{-1})\,A^{-1}
            =   A\,I\,A^{-1}
            =   A\,A^{-1}
            =   I
            &
        \end{flalign*}
    \end{sectionBox}


\end{sectionBox}

% Matriz Transposta
\begin{sectionBox}1m{Matriz Transposta}

    \begin{BM}
        A\in\mathcal{M}_{m\times n}
    \end{BM}

    \subsection*{Propriedades}

    % ...

    \begin{sectionBox}2{\( \exists\,A^{-1}\implies \exists\,(A^T)^{-1} \)}

    \end{sectionBox}

    % Tipos de matrizes com base em transpostas

    \subsection*{Tipos de Matrizes com referencia a transposição}

    \begin{sectionBox}2{Simétrica}
        \begin{BM}
            A=A^T
        \end{BM}
        ou
        \begin{BM}
            A\in\mathcal{M}_{n\times n}:a_{i\,j}=a_{j\,i}
        \end{BM}
    \end{sectionBox}

    \begin{sectionBox}{Hemi-Simétrica}
        \begin{BM}
            A=-A^T
        \end{BM}
        ou
        \begin{BM}
            A\in\mathcal{M}_{n\times n}:a_{i\,j}=-a_{j\,i}
        \end{BM}
    \end{sectionBox}




\end{sectionBox}

\end{multicols}

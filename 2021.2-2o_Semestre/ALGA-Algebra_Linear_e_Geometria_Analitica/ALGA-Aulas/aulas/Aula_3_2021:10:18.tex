% Aula 3 - ...
\part{}

% Revisão aula passada
    % Transposição

    % Simetria          => A =  A^T
    % hemi-simetria     => A = -A^T

% Fim da revisão

\begin{sectionBox}1{Matriz Conjugada}

    \begin{sectionBox}2{Conjugado}
        \begin{BM}
            \overline{a + b\,i} = a - b\,i
        \end{BM}
    \end{sectionBox}

    \begin{BM}
        \overline{A} \implies a_{i,j} \to \overline{a_{i,j}}
    \end{BM}

    \begin{sectionBox}2{Propriedades}
        \begin{itemize}
            \item \( \overline{\overline{A}}     = A \)
            \item \( \overline{A + B}            = \overline{A} + \overline{B} \)
            \item \( \overline{\alpha\,A}        = \overline{\alpha}\,\overline{A} \)
            \item \( \overline{A^k}              = \big(\overline{A}\big)^k \hfill : A\text{ é quadrada} \)
            \item \( \big(\overline{A}\big)^{-1} = \overline{A^{-1}}        \hfill : \exists\,A^{-1} \)
            \item \( \big(\overline{A}\big)^T    = \overline{A^T} \)
        \end{itemize}
    \end{sectionBox}


\end{sectionBox}


% Conjugado -> Inversão da parte imaginária do valor
% a + b\,i -> a - b\,i
% Matriz conjugada: \overline{A} -> a_{i,j} => \overline{a_{i,j}}
% propriedades lineares do conjugado:
    % \overline{\overline{A}} = A
    % \overline{A + B}        = \overline{A} + \overline{B}
    % \overline{\alpha\,A}    = \overline{\alpha}\,\overline{A}
    % \overline{A^k}          = \big(\overline{A}\big)^k : A é quadrada
    % (\overline{A})^{-1}     = \overline{A^{-1}}        : \quad\exists\,A^{-1}
    % (\overline{A})^T        = \overline{A^T}

% Matriz transconjugada -> A => \big(\overline{A}\big)^T = A^*


% Simetria Hermítica:

    % NOTA: puchar definição de simetria para junto daquí no material final
    % NOTA: inserir link para simetria

    % Hermítica      -> A =  A^*
    % Hemi-Hermítica -> A = -A^*


% Transformações Elementáres
    %   I troca de linhas:                      i <=> j            : i \neq j
    %  II multiplicação de linha por escalar:   i ==> \alpha\,i    : \alpha\in\mathbb{K}\backslash\{0\}
    % III troca de colunas:                     i ==> i + \beta\,j : j \neq i

    % OBS: II + III: i ==> \alpha\,i + \beta\,j : j \neq i \land \alpha\in\mathbb{K}\backslash\{0\}

    % Notação de transformação: A \xrightarrow[T]{} B : T = transf. linear

    % OBS: transformações elementares são revertiveis

    % Matriz Elementar : Matriz q com uma op. elem. vira uma matriz identidade

% Teoremas p matrizes elementares
    % A \xrightarrow[T ]{} EA  : I \xrightarrow[T] {} E  (Troca de linhas  em A)
    % A \xrightarrow[T']{} AE' : I \xrightarrow[T']{} E' (troca de colunas em A)

    % Concl.: Matrizes elementares são operadores elementares

    % Qualquer matriz elem. é invertível

% Pivôt: Primeiro elemento não nulo da linha de uma matriz

% Forma escada: organização da matriz c pivores formando um 'triangulo superior'

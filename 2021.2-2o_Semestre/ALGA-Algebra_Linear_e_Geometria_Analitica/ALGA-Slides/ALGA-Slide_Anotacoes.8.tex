% !TEX root = ./ALGA-Slides_Anotacoes.8.tex
% !TEX root = ./ALGA-Slides_Anotacoes.tex
\providecommand\mainfilename{"./ALGA-Slides_Anotacoes.tex"}
\providecommand \subfilename{}
\renewcommand   \subfilename{"./ALGA-Slides_Anotacoes.8.tex"}
\documentclass[\mainfilename]{subfiles}

% \graphicspath{{\subfix{../images/}}}

\begin{document}

\mymakesubfile{8}[ALGA]
{Reta e Plano}
{Reta e Plano}

\part*{Elementos}
\addcontentsline{toc}{section}{\bullet\qquad Elementos}

\begin{sectionBox}1m{Espaço}
    
    \subsection*{Equação vetorial}
    \begin{BM}
        \mathcal{E}sp_m\subset\mathbb{R}^n:
        \left\{
            \begin{aligned}
            &
                P = A + \sum_{i=1}^{m}\lambda_i\,U_i
            \ldiv{} \{P,A\}\subset\mathcal{E}sp_m
            \ldiv{} \{i,n,m\}\subset\mathbb{N}
            \ldiv{} m\leq n
            \ldiv{} \{\lambda,U_i\}\subset\mathbb{R}^n
            &
            \end{aligned}
        \right\}
    \end{BM}
    
    \begin{sectionBox}*2{Equações paramétricas}
        \begin{BM}
            P_i = A_i + \sum_{j=1}^{m}\lambda_{j,i}\,U_{j,i}
        \end{BM}
    \end{sectionBox}

    \paragraph{Exemplos}
    \begin{itemize}
        \begin{multicols}{2}
            \item \(m=0\implies\) ponto
            \item \(m=1\implies\) reta
            \item \(m=2\implies\) plano
        \end{multicols}
    \end{itemize}

\end{sectionBox}

\begin{sectionBox}1m{Reta}

    \subsection*{Equação vetorial}
    \begin{BM}
        \mathcal{R}eta_m\subset\mathcal{E}sp_m:
        \left\{
            \begin{aligned}
            &
                P=A+\lambda\,U
            \ldiv{} \{P,A\}\in\mathcal{R}eta_m
            &
            \end{aligned}
        \right\}
    \end{BM}
    
    \subsubsection*{Definem uma reta:}
    \begin{itemize}
        \begin{multicols}{2}
            \item 2 Pontos
            \item 1 Ponto e 1 vetor
        \end{multicols}
    \end{itemize}

    \paragraph{Vetor diretor:} Vetor não nulo parapelo a reta

    \subsection*{Equações cartezianas}

    \begin{sectionBox}3{Equações paramétricas}
        \begin{BM}
            P_i = A_i + \lambda\,U_i \quad\forall\,i\leq n
        \end{BM}
    \end{sectionBox}

    \begin{sectionBox}3{Equações normais}
        \begin{BM}[flalign*]
        &
            \left.
                \begin{aligned}
                &
                      P_i = A_i + \lambda\,U_i
                \ldiv{} P_j = A_j + \lambda\,U_j
                &
                \end{aligned}
            \right\}
        \implies &\\&
        \implies
            \lambda 
        =   \frac{P_i - A_i}{U_i}
        =   \frac{P_j - A_j}{U_j}
            \left\{
                \begin{aligned}
                &
                    \{i,j\}\leq n
                \ldiv{} 
                    \{U_i,U_j\}\neq 0
                &
                \end{aligned}
            \right.
        &
        \end{BM}
    \end{sectionBox}

    \begin{sectionBox}3{Equações reduzidas}
        \begin{BM}
            P_i = A'_i + P_j\,U'_i:
            \left\{
                \begin{aligned}
                &
                      \{i,j\}\subset\mathbb{N}
                \ldiv{} \{i,j\}\leq n
                \ldiv{} i\neq j
                \ldiv{} U'_i = U_i/U_j
                \ldiv{} A'_i = A_i - A_j\,U'_i
                \ldiv{} U_j\neq 0
                &
                \end{aligned}
            \right\}
        \end{BM}

        \begin{flalign*}
        &
            \left.
                \begin{aligned}
                &
                    P_i = A_i + \lambda\,U_i
                \ldiv{}
                    \lambda = (P_j-A_j)/U_j : U_j\neq 0
                &
                \end{aligned}
            \right\}
        \implies &\\&
        \implies
            P_i 
        =   A_i + U_i\,(P_j-A_j)/U_j
        = &\\&
        =   A'_i + U'_i\,P_j
        :   \left\{
                \begin{aligned}
                &
                      U'_i = U_i/U_j
                \ldiv{} A'_i = A_i - A_j\,U'_i
                \ldiv{} U_j\neq 0
                &
                \end{aligned}
            \right.
        &
        \end{flalign*}

        Reduzida pois possui uma equação a menos comparando com as equações cartezianas

    \end{sectionBox}

\end{sectionBox}

\begin{sectionBox}1m{Plano}

    \subsection*{Equação vetorial}
    \begin{BM}
        \mathcal{P}lano_m\subset\mathcal{E}sp_m:
        \left\{
            \begin{aligned}
            &
                  \{P,A\}\subset\mathcal{P}lano_m
            \ldiv*{} n\geq2
            &
            \end{aligned}
        \right\}
    \end{BM}

    \begin{sectionBox}*3{Definição 1 ponto e 2 vetores não paralelos}
        \begin{BM}
            P = A + \sum_{j=1}^{2}\lambda_j\,U_j
        \end{BM}
    \end{sectionBox}

    \begin{sectionBox}*3{Definição 1 ponto e 1 vetor perpendicular ao plano}
        \begin{BM}
            \vec{AP}|(\lambda_1\times\lambda_2) 
        =   \left|
                P-A,\lambda_1,\lambda_2
            \right| 
        =   \vec{AP}|\lambda'
        = \\
        =   A\,\lambda'+d
        =   0
        \\  \left\{
                \begin{aligned}
                &
                    d\in\mathbb{R}
                \ldiv{}
                    \lambda'\in\mathbb{R}^n:\lambda'\perp\mathcal{P}lano
                &
                \end{aligned}
            \right\}
        \end{BM}
    \end{sectionBox}
    
    \paragraph{Definem um plano:}
    \begin{itemize}
        \begin{multicols}{2}
            \item 3 Pontos
            \item 1 Ponto e 2 vetores não paralelos
            \item 1 Ponto e 1 Vetor
        \end{multicols}
    \end{itemize}

    \paragraph{Vetores diretores:} Vetores não nulos e não colineares paralelos ao plano.
    
    \subsection*{Equações cartesianas}

    \begin{sectionBox}3{Equações paramétricas}
        \begin{BM}
            P_i = A_i + \sum_{j=1}^{2}\lambda_{j,i}\,U_{j,i}
        \end{BM}
    \end{sectionBox}

    \begin{sectionBox}3{Equação geral}
        \begin{BM}
            A + \sum_{i=1}^{n}\lambda_i\,P_i = 0:
            \left\{
            \begin{aligned}
            &
                A\in\mathbb{R}
            \ldiv{}   
                \lambda\in\mathbb{R}^n
            &
            \end{aligned}
            \right.
        \end{BM}
    \end{sectionBox}

\end{sectionBox}

\part*{Distâncias}
\addcontentsline{toc}{section}{\bullet\qquad Distâncias}

\begin{sectionBox}1{Distancia dentre 2 espaços}
    \begin{BM}
        \distancia(\mathcal{E}_1,\mathcal{E}_2) = \min(|| \vec{e_1\,e_2} ||):
        \left\{
            \begin{aligned}
            &
                    \mathcal{E}_1\in\mathcal{E}sp_m
            \ldiv{} \mathcal{E}_2\in\mathcal{E}sp_n
            \ldiv{} e_1\in\mathcal{E}_1
            \ldiv{} e_2\in\mathcal{E}_2
            &
            \end{aligned}
        \right\}
    \end{BM}
\end{sectionBox}

\begin{sectionBox}2{Distancia entre 1 Ponto e 1 Reta}
    \begin{BM}
        \distancia(P,\mathcal{R}eta)
    =   \Bigl\lVert{\vec{A\,P}}\Bigr\rVert\sin(\theta)
    =   \Bigl\lVert{\vec{A\,P}}\Bigr\rVert\lVert{u}\rVert\sin{\theta}/\lVert{u}\rVert
    = \\
    =   \Bigl\lVert{\vec{A\,P}\times u}\Bigr\rVert\big/\lVert{u}\rVert:
        \theta = \measuredangle\Bigl(u,\vec{A\,P}\Bigr)
    \end{BM}
\end{sectionBox}

\begin{sectionBox}2{Distancia dentre 1 ponto e 1 plano}
    \begin{BM}
        \distancia(P,\mathcal{P}lano)
    =   \Bigl\lVert{\vec{A\,P}}\Bigr\rVert\lvert\cos(\theta)\rvert
    =   \Bigl\lVert{\vec{A\,P}}\Bigr\rVert\lVert{w}\rVert\lvert\cos(\theta)\rvert\big/\lVert{w}\rVert
    = \\
    =   \Bigl\lvert{\vec{A\,P}\big| w}\Bigr\rvert\big/\lVert{w}\rVert
    \end{BM}
\end{sectionBox}

\begin{questionBox}1{}
    
    Conclua que \(\mathcal{R}eta\) é extritamente paralela ao \(\mathcal{P}lano\)

    \begin{BM}
        \mathcal{R}eta=\{(x,y,z)\in\mathbb{R}^3:x=2\land2\,y-6=z\}
    \\  \{(1,0,-1),(2,2,0),(1,1,1)\}\subset\mathcal{P}lano
    \end{BM}

    \begin{questionBox}3{}
        \begin{flalign*}
            &
                \mathcal{R}eta\Vert\mathcal{P}lano
            \implies
                u|v=0:
                \left\{
                    \begin{aligned}
                    &    
                            u\in\mathcal{R}eta
                    \ldiv{} v\in\mathcal{P}lano
                    &
                    \end{aligned}
                \right\}
            \implies &\\&
            \implies
                \left.
                \left(
                    \begin{bmatrix}
                        2\\3\\0
                    \end{bmatrix}
                -   \begin{bmatrix}
                        2\\0\\-6
                    \end{bmatrix}
                \right)
                \right\vert
                \left(
                    \left(
                        \begin{bmatrix}
                            2\\2\\0
                        \end{bmatrix}
                    -   \begin{bmatrix}
                            1\\0\\-1
                        \end{bmatrix}
                    \right)
                \times
                    \left(
                        \begin{bmatrix}
                            1\\1\\1
                        \end{bmatrix}
                    -   \begin{bmatrix}
                            1\\0\\-1
                        \end{bmatrix}
                    \right)
                \right)
            = &\\&
            =
                \left.
                \left(
                    \begin{bmatrix}
                        0\\3\\6
                    \end{bmatrix}
                \right)
                \right\vert
                \left(
                    \left(
                        \begin{bmatrix}
                            1\\2\\1
                        \end{bmatrix}
                    \right)
                \times
                    \left(
                        \begin{bmatrix}
                            0\\1\\2
                        \end{bmatrix}
                    \right)
                \right)
            = &\\&
            =
                \left.
                \left(
                    \begin{bmatrix}
                        0\\3\\6
                    \end{bmatrix}
                \right)
                \right\vert
                \left(
                    \begin{bmatrix}
                        3\\-2\\1
                    \end{bmatrix}
                \right)
            = &\\&
            =
                -6+6 = 0
            &
        \end{flalign*}
    \end{questionBox}

    \begin{questionBox}3{}
        \begin{flalign*}
            &
                % \left\{    
                    \mathcal{R}eta\nsubseteq\mathcal{P}lano
                % 
            \implies
                \nexists\,p:p\in\mathcal{R}eta\land p\in\mathcal{P}lano
            \implies
                % 
            &
        \end{flalign*}
    \end{questionBox}
    
\end{questionBox}

\end{document}
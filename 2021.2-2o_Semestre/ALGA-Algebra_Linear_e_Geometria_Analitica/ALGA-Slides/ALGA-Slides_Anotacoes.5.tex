% !TEX root = ./ALGA-Slides_Anotacoes.5.tex
% !TEX root = ./ALGA-Slides_Anotacoes.tex
\providecommand\mainfilename{"./ALGA-Slides_Anotacoes.tex"}
\providecommand \subfilename{}
\renewcommand   \subfilename{"./ALGA-Slides_Anotacoes.5.tex"}
\documentclass[\mainfilename]{subfiles}

% \graphicspath{{\subfix{../images/}}}

\begin{document}

\mymakesubfile{5}[ALGA]
{Aplicações Lineares}
{Aplicações Lineares}

% \begin{sectionBox}1{Aplicação Linear}
    
\(f:E\to E'\) é aplicação linear (sobre \(\mathbb{K}\)) \(\iff\)

\begin{BM}
    \iff
        \begin{cases}
            f(u+v)=f(u)+f(v)
        \\  f(\alpha\,u)=\alpha\,f(u)
        \end{cases}
    \quad
        \begin{cases}
            \forall\,\{u,v\}\in E
        \\  \forall\,\alpha\in\mathbb{K}
        \end{cases}
\end{BM}

\begin{sectionBox}1{Extenção das condições}
    
    \begin{BM}
            f(\alpha\,u+\beta\,v)=\alpha\,f(u)+\beta\,f(v)
        \quad
            \begin{cases}
                \forall\,\{\alpha,\beta\}\in\mathbb{K}
            \\  \forall\,\{u,v\}\in E
            \end{cases}
    \end{BM}
    
\end{sectionBox}

% Dizemos que uma aplicação \(f:E\to E'\) é aplicação linear (sobre \(\mathbb{K}\)) se, para quaisquer \(\{u,v\}\in E\) e qualquer \(\alpha\in\mathbb{K}\), satisfaz as duas condições seguintes:
% \begin{itemize}
%     \item \(f(u+v)=f(u)+f(v)\)
%     \item \(\alpha\,u=\alpha\,f(u)\)
% \end{itemize}

\begin{sectionBox}1m{Categorização}
    
    \begin{sectionBox}*2{Homotetia}

        \begin{BM}
            f:
            \begin{cases}
                E\to E'
            \\  f(w)=\beta\,w
            \end{cases}
        \quad
            \begin{array}{l}
                \forall\,w\in E
            \\  \forall\,\beta\in\mathbb{K}
            \end{array}
        \end{BM}

        \(f\) é uma aplicação \textcolor{Emph}{homotetia} de razão \(\beta\)

    \end{sectionBox}

    \begin{sectionBox}*2{Aplicação Nula}
        
        \begin{BM}
                f:
                \begin{cases}
                    E\to E
                \\  f(u)=0_E
                \end{cases}
            \quad 
                \forall\,u\in E
        \end{BM}

        Uma aplicação \textcolor{Emph}{homotetia} de razão 0
        
    \end{sectionBox}

    \begin{sectionBox}*2{Aplicação Identidade}
        
        \begin{BM}
                \id_E
            \coloneqq 
                f
            :   \begin{cases}
                    E\to E
                \\  f(u)=u
                \end{cases}
            \quad
                \forall\,u\in E
        \end{BM}
        
    \end{sectionBox}

    \begin{sectionBox}*2{Derivada}
        
        \begin{BM}
            D
        \coloneqq
            f
        :   \begin{cases}
                \mathbb{R}_n[x]\to\mathbb{R}_n[x]
            \\  D
                \left(
                    \sum_{k=0}^n a_{k}\,x^{k}
                \right)
            =   \sum_{k=1}^n k\,a_k\,x^{k-1}
            \end{cases}
        \end{BM}
        
    \end{sectionBox}

\end{sectionBox}

\begin{sectionBox}1{Inclusão do termo nulo}
    
    \begin{BM}
        f(0_E)=0_{E'}
    \end{BM}

    \begin{flalign*}
        &
            f(u)+0_{E'} = f(u) = f(u+0_E) = f(u)+f(0_E)
        \implies
            0_{E'}=f(0_E)
        &
    \end{flalign*}
    
\end{sectionBox}

\begin{questionBox}1{}
    
    \begin{BM}
        f(-u)=-f(u)\quad\forall\,u\in E
    \end{BM}

    \begin{flalign*}
        &
        \implies
            0_{E'} 
        =   f(u) - f(u) 
        =   f(u) + f(-u) 
        =   f(u + (-u))
        =   f(0_E)
        &
    \end{flalign*}
    
\end{questionBox}

\section*{Operações com aplicações}

\begin{sectionBox}1{Aplicação Soma}
    
    \begin{BM}
            f+g
        :   \begin{cases}
                E\to E'
            \\  (f+g)(u)=f(u)+g(u)
            \end{cases}
        \quad
            \begin{array}{l}
                \forall\,f:E\to E'
            \\  \forall\,g:E\to E'
            \\  \forall\,u\in E
            \end{array}
    \end{BM}
    
\end{sectionBox}

\begin{sectionBox}1{Aplicação Produto}
    
    \begin{BM}
            \alpha\,f
        :   \begin{cases}
                E\to E'
            \\  (\alpha\,f)(u) = \alpha\,f(u)
            \end{cases}
        \quad
            \begin{array}{l}
                \forall\,f:E\to E'
            \\  \forall\,\alpha\in\mathbb{K}
            \\  \forall\,u\in E
            \end{array}
    \end{BM}
    
\end{sectionBox}

\begin{sectionBox}1{Aplicação Composta}
    
    \begin{BM}
            g\circ f
        :   \begin{cases}
                A\to C
            \\  (g\circ f)(a) = g(f(a))
            \end{cases}
        \quad
            \begin{array}{l}
                \forall\,f:A\to B
            \\  \forall\,g:B\to C
            \\  \forall\,a\in A
            \end{array}
    \end{BM}

    \paragraph{obs:}Tambem designada por ``\textit{g} \textcolor{Emph}{após} \textit{f}''
    
\end{sectionBox}

\begin{sectionBox}1{Potencia de Expoente \(k\)}
    
    \begin{BM}
            f^k
        :   \begin{cases}
                A\to A
            \\  f^k
            =   \begin{cases}
                    \id_A         &: k=0
                \\  f^{k-1}\circ f&: k\in\mathbb{N}
                \end{cases}
            \end{cases}
    \end{BM}
    
\end{sectionBox}

\begin{sectionBox}1{Imagem}
    
    \begin{BM}
        \Img F = \{f(a):a\in A\}\subseteq B
    \end{BM}
    
\end{sectionBox}

\begin{sectionBox}1{Núcleo}
    
    \begin{BM}
        \Nuc f = \Ker f = \{i\in E:f(u)=0_{E'}\}
    \end{BM}

    \begin{sectionBox}*2{\(\Nuc f\neq\emptyset\)}
        \begin{flalign*}
            &
                f(0_E)=0_E'\implies 0_E\in\Nuc f
            &
        \end{flalign*}
    \end{sectionBox}
    
\end{sectionBox}

\end{document}
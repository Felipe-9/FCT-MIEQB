% !TEX root = ./ALGA-Slides_Anotacoes.7.tex
% !TEX root = ./ALGA-Slides_Anotacoes.tex
\providecommand\mainfilename{"./ALGA-Slides_Anotacoes.tex"}
\providecommand \subfilename{}
\renewcommand   \subfilename{"./ALGA-Slides_Anotacoes.7.tex"}
\documentclass[\mainfilename]{subfiles}

% \graphicspath{{\subfix{../images/}}}

\begin{document}

\mymakesubfile{7}[ALGA]
{Produto Interno, Externo e Misto}
{Produto Interno, Externo e Misto}

\begin{sectionBox}1{Produto Interno}
    \begin{BM}
        a|b = \sum_{k=1}^{n}a_1\,b_1 = a\,b
    \end{BM}

    \begin{sectionBox}*2{Propriedades}
        \begin{align*}
            \bullet\quad & a|b = b|a                     & \bullet\quad & (\alpha a)|b = \alpha(a|b)
        \\  \bullet\quad & (a+b)|c = a|c + b|c           & \bullet\quad & a|a\geq 0
        \\  \bullet\quad & a|a=0\iff a=0_{\mathbb{R}^n}
        \end{align*}
    \end{sectionBox}
\end{sectionBox}

\begin{sectionBox}1{Norma}
    \begin{BM}
        \lVert{u}\rVert = \sqrt{u|u}
    \end{BM}

    \begin{sectionBox}*3{Propriedades}
        \begin{align*}
            \bullet\quad & \lVert{u}\rVert\geq0
        &   \bullet\quad & \lVert{u}\rVert = 0 \iff u=0_{\mathbb{R}^n}
        \\  \bullet\quad & ||\alpha\,u|| = |\alpha| \lVert{u}\rVert
        \end{align*}
    \end{sectionBox}

    

\end{sectionBox}

\begin{multicols}{2}
    
    \begin{sectionBox}3{Vetor unitário (versor)}
        \begin{BM}
            \hat{u} = u/\lVert{u}\rVert
        \end{BM}
    \end{sectionBox}
    
    \begin{sectionBox}3{Desigualdade de Schwarz}
        \begin{BM}
            \lvert{u|v}\lvert\leq\lVert{u}\rVert\,\lVert{v}\rVert
        \end{BM}
    \end{sectionBox}
    
    \begin{sectionBox}3{Desigualdede triangular}
        \begin{BM}
            \lVert{u+v}\rVert \leq \lVert{u}\rVert + \lVert{v}\rVert
        \end{BM}
    \end{sectionBox}
    
\end{multicols}

\begin{sectionBox}1{Angulo dentre vetores}
    
    \begin{BM}
        \measuredangle(u,v)=\arccos\frac{u|v}{\lVert{u}\rVert\,\lVert{v}\rVert}
    \end{BM}

    \begin{sectionBox}*2{Demonstração}
        \begin{flalign*}
            &
                |u|v|\leq\lVert{u}\rVert\,\lVert{v}\rVert:
                u\neq0_{\mathbb{R}^n}\neq v
            \implies
                \frac{ |u|v| }{\lVert{u}\rVert\,\lVert{v}\rVert}\leq1
            \implies &\\&
            \implies
                -1\leq\frac{ u|v }{\lVert{u}\rVert\,\lVert{v}\rVert}\leq1
            \implies
                \cos\measuredangle(u,v) = \frac{u|v}{\lVert{u}\rVert\,\lVert{v}\rVert}
            \implies &\\&
            \implies
                \measuredangle(u,v)=\arccos\frac{u|v}{\lVert{u}\rVert\,\lVert{v}\rVert}
            &
        \end{flalign*}
    \end{sectionBox}
    
    \begin{sectionBox}*2{Propríedades}
        \begin{BM}[align*]
           \bullet\quad& \measuredangle(u,v)=\measuredangle(v,u)
        &  \bullet\quad& u|v = \lVert{u}\rVert\lVert{v}\rVert\measuredangle(u,v)
        \end{BM}
    \end{sectionBox}

\end{sectionBox}

\begin{questionBox}1{}
    
    \begin{BM}
        \measuredangle((1,0,0),(0,0,1))
    \end{BM}

    \begin{flalign*}
        &
        =   \arccos
            \frac
                {(1,0,0)|(0,0,1)}
                {\lVert{(1,0,0)}\rVert\lVert{(0,0,1)}\rVert}
        =   \arccos(0)=\pi/2
        &
    \end{flalign*}
    
\end{questionBox}

\begin{questionBox}1{}
    
    \begin{BM}
        \measuredangle(u,v):
        \{u,v\}\subset\mathbb{R}^3\backslash\{0\}
    \land u=\alpha\,v:\alpha\in\mathbb{R}
    \end{BM}

    \begin{flalign*}
        &
            \measuredangle(u,v)
        =   \arccos\frac{u|v}{\lVert{u}\rVert\lVert{v}\rVert}
        =   \arccos\frac{\alpha\,v|v}{\lVert{\alpha\,v}\rVert\lVert{v}\rVert}
        =   \arccos\frac{\alpha}{\lvert{\alpha}\rvert}\frac{v|v}{\lVert{v}\rVert^2}
        =   \arccos\pm1
        =   \begin{cases}
                0\\\pi
            \end{cases}
        &
    \end{flalign*}

\end{questionBox}

\begin{sectionBox}2{Ortogonalidade}
    
    \begin{BM}
        \measuredangle(u,v)=0 \iff u|v=0
    \end{BM}
    
    \begin{sectionBox}3{Proprieades}
        \begin{BM}
            \measuredangle(u,v)=0\iff 
            \lVert{u+v}\rVert^2 
            = \lVert{u}\rVert^2+\lVert{v}\rVert^2
        \end{BM}
    \end{sectionBox}

\end{sectionBox}

\begin{sectionBox}3{Sequencia Ortogonal}
    
    \begin{BM}
        u\subset\mathbb{R}^n:u_i|u_j=0\quad\forall\,i\neq j
    \end{BM}
    
\end{sectionBox}

\begin{sectionBox}3{Sequencia Ortonormada}
    \begin{BM}
        u\subset\mathbb{R}^n:
        u_i|u_j=
        \begin{cases}
            0: & i\neq j
        \\  1: & i=j
        \end{cases}
    \end{BM}
\end{sectionBox}

\begin{sectionBox}3{Base ortonormada}
    \begin{BM}
        \left\{
            \begin{aligned}
               & u|v = \sum_{i=1}^{n}\alpha_i\,\beta_i
            \\ & \lVert{u}\rVert=\sqrt{\sum_{i=1}^{n}\alpha_i^2}
            \\ & \alpha_i=u|e_i\quad\forall\,i
            \iff 
            \\ & 
            \iff 
                u=\sum_{i=1}^{n}(u|e_i)\,e_i
            \end{aligned}
        \right\}:
        \left\{
            \begin{aligned}
            &
                    u=e\,\alpha
            \ldiv{} v=e\,\beta
            \ldiv{} e\text{ base ortonormada }
            \ldiv{} e\subset\mathbb{R}^n
            &
            \end{aligned}
        \right\}
    \end{BM}
\end{sectionBox}

\begin{questionBox}1{}
    
    Determinar a sequencia de coodenadas do vetor \textit{v} na base \(\mathcal{B}\)

    \begin{BM}
        v=(a,b,c)
    \\  \mathcal{B}=
        \left(
            \frac{(2,1,0)}{\sqrt{5}},\frac{(-1,2,0)}{\sqrt{5}},(0,0,1)
        \right)
    \end{BM}

    \begin{flalign*}
        &
            \left\{
            \begin{aligned}
                b_1 | b_1 &= 4/5 + 1/5 = 1
            \\  b_2 | b_2 &= 1/5 + 4/5 = 1
            \\  b_3 | b_3 &= 1
            \\  b_1 | b_2 &= -2/5 + 2/5 = 0
            \\  b_1 | b_3 &= 0
            \\  b_2 | b_3 &= 0
            \end{aligned}
            \right\};
        % &\\&
            (v|b_1,v|b_2,v|b_3)
        =   \left(
                \frac{2\,a+b}{\sqrt{5}}
            ,   \frac{2\,b-a}{\sqrt{5}}
            ,   c
            \right)
        &
    \end{flalign*}
    
\end{questionBox}

\begin{propositionBox}1{}
    
    \begin{BM}
        \left\{
            \begin{aligned}
            &
                    u\text{ é seq. orto.}
            \ldiv{} u_i\neq0\quad\forall\,i
            &
            \end{aligned}
        \right\}
    =   \left\{
            \begin{aligned}
            &
                    u_i|u_j=0\quad\forall\,i\neq j
            \ldiv{} u_i\neq 0\quad\forall\,i
            &
            \end{aligned}
        \right\}
    \implies \\
    \implies
        \left\{
            \begin{aligned}
            &
                u\text{ é l.i.}
            \ldiv{} \#u\leq n
            &
            \end{aligned}
        \right\}
    = \\
    =   \left\{
            \begin{aligned}
            &
                \nexists\,\alpha\in\mathbb{R}^n\backslash\{I_i\}:u_i=\alpha\,u\quad\forall\,i
            \ldiv{} \#u\leq n
            &
            \end{aligned}
        \right\}
    : \\
        \left\{
            \begin{aligned}
            &
                    u\subset\mathbb{R}^n
            \ldiv{} u_i\neq0_{\mathbb{R}^n}\quad\forall\,i
            &
            \end{aligned}
        \right\}
    \end{BM}
    
\end{propositionBox}

\begin{questionBox}1{}
    
    encontre uma base ortornomada para \(\mathbb{R}^3\) usando \textit{u}

    \begin{BM}
        u=
        \left(
            (2,1,0),
            (-2,4,0),
            (0,0,2)
        \right)
    \end{BM}

    \begin{flalign*}
        &
            \left\{
                \begin{aligned}
                &
                        u_1|u_2 = -4+4 = 0
                \ldiv{} u_1|u_3 = 0
                \ldiv{} u_2|u_3 = 0
                &
                \end{aligned}
            \right\}
        \implies
            \left\{
                \begin{aligned}
                &
                    u\text{ é l.i.}
                \ldiv{}
                    \#u = 3 = \dim{\mathbb{R}^3}
                &
                \end{aligned}
            \right\}
        \implies &\\&
        \implies
            \left(
                \bigg(\frac{u_1}{\lVert{u_1}\rVert}\bigg),
                \bigg(\frac{u_2}{\lVert{u_2}\rVert}\bigg),
                \bigg(\frac{u_3}{\lVert{u_3}\rVert}\bigg)
            \right)
        =   \left(
                \bigg(\frac{u_1}{\sqrt{5}} \bigg),
                \bigg(\frac{u_2}{\sqrt{20}}\bigg),
                \bigg(\frac{u_3}{2}        \bigg)
            \right)
        &
    \end{flalign*}
    
\end{questionBox}

\begin{sectionBox}1{Processo de Ortogonalização de Gram-Schmidt}
    \begin{BM}
        \left\{
            \begin{aligned}
            &
                    v_i = u_i - \sum_{j=2}^{i} \frac{u_i|v_j}{\lVert{v_j}\rVert^2}\,v_j
            \ldiv{} u\text{ é base de }\mathbb{R}^n
            &
            \end{aligned}
        \right\}
    \implies
        v\text{ é base ortog. de }\mathbb{R}^n
    \end{BM}

    \paragraph{Objetivo:} Processo para encontrar uma base ortogonal a partir de uma base arbritraria de qualquer subespaço não nulo de \(\mathbb{R}^n\)

\end{sectionBox}

\begin{questionBox}1{}
    
    Encontre a base ortogonal para o subespaço \(F\subset\mathbb{R}^n\) a partir de \textit{u}

    \begin{BM}
        u =
        \bigl(
            (1,1,1),
            (1,3,2)
        \bigr)
    \end{BM}

    \begin{flalign*}
        &
            v=
            \biggl(
                (1,1,1),
                (1,3,2) - \frac{(1,3,2)|(1,1,1)}{\lVert{(1,1,1)}\rVert}(1,1,1)
            \biggr)
        = &\\&
        =   \biggl(
                (1,1,1),
                (1,3,2) - \frac{6}{\sqrt{3}^2}(1,1,1)
            \biggr)
        =   \bigl(
                (1,1,1),
                (-1,1,0)
            \bigr)
        &
    \end{flalign*}
    
\end{questionBox}

\begin{questionBox}1{}
    
    Encontre uma base ortogonal para \(\mathbb{R}^3\) usando a base do exemplo anterior

    \begin{flalign*}
        &
            v
        =
            \left(
                \begin{bmatrix}
                    1\\1\\1
                \end{bmatrix},
                \begin{bmatrix}
                    -1\\1\\0
                \end{bmatrix},
                \begin{bmatrix}
                    0\\0\\1
                \end{bmatrix}
                - \frac{(0,0,1)|(1,1,1)}{\lVert{(1,1,1)}\rVert}
                \begin{bmatrix}
                    1\\1\\1
                \end{bmatrix}
                - \frac{(0,0,1)|(-1,1,0)}{\lVert{(-1,1,0)}\rVert}
                \begin{bmatrix}
                    -1\\1\\0
                \end{bmatrix}
            \right)
        = &\\&
        =
            \left(
                \begin{bmatrix}
                    1\\1\\1
                \end{bmatrix},
                \begin{bmatrix}
                    -1\\1\\0
                \end{bmatrix},
                \begin{bmatrix}
                    0\\0\\1
                \end{bmatrix}
                - \frac{1}{3}
                \begin{bmatrix}
                    1\\1\\1
                \end{bmatrix}
                - \frac{0}{\lVert{(-1,1,0)}\rVert}
                \begin{bmatrix}
                    -1\\1\\0
                \end{bmatrix}
            \right)
        = &\\&
        =
            \bigl(
                (1,1,1),
                (-1,1,0),
                (-1/3,-1/3,2/3)
            \bigr)
        &
    \end{flalign*}
    
\end{questionBox}

\begin{sectionBox}1{Complemento Ortogonal}
    
    \begin{BM}
        F^{\bot}=
        \bigl\{
            u\in\mathbb{R}^3:u|v=0\quad\forall\,v\in F
        \bigr\}
    \end{BM}
    
\end{sectionBox}

\begin{propositionBox}1{}
    \begin{BM}
        F\text{ subespaço de }\mathbb{R}^n\implies F^\bot\text{ sybespaço de } \mathbb{R}^n
    \end{BM}
\end{propositionBox}

\begin{questionBox}1{}
    
    \begin{multicols}{2}
        \begin{questionBox}3{}
            
            \begin{BM}
                F^\bot:F=\{(0,0,0)\}
            \end{BM}

            \begin{flalign*}
                &
                    u | (0,0,0) = 0 \quad\forall\,u
                \implies &\\&
                \implies
                    F^\bot = \mathbb{R}^3
                &
            \end{flalign*}
            
        \end{questionBox}

        \begin{questionBox}3{}
            
            \begin{BM}
                F^\bot:F=\mathbb{R}^3
            \end{BM}

            \begin{flalign*}
                &
                    u|v=0\quad\forall\,v\in\mathbb{R}^3
                \implies &\\&
                \implies
                    F^\bot=\{(0,0,0)\}
                &
            \end{flalign*}
            
        \end{questionBox}

        \begin{questionBox}3{}
            
            \begin{BM}
                F^\bot:\\F=
                \left\{
                    \begin{aligned}
                        (x,y,z)\in\mathbb{R}^3:
                    \\  x-2\,y+3\,z=0
                    \end{aligned}
                \right\}
            \end{BM}

            \begin{flalign*}
                &
                    u|(x,y,z) = 0 = x-2\,y+3\,z
                \implies &\\&
                \implies
                    F^\bot=\{(1,-2,3)\}
                &
            \end{flalign*}
            
        \end{questionBox}
    \end{multicols}
\end{questionBox}

\begin{propositionBox}1{}
    \begin{BM}
        w|u_i = 0\quad\forall\,i\implies
        w\in\langle{u}\rangle^\bot:
        \begin{cases}
            w\in\mathbb{R}^n
            u\subset\mathbb{R}^n
        \end{cases}
    \end{BM}
\end{propositionBox}

\begin{questionBox}1{}
    
    \begin{BM}
        F^\bot:F=\{(x,y,z)\in\mathbb{R}^3:x-2\,y+3\,z=0\}
    \end{BM}

    \begin{flalign*}
        &
            F
        =   \{(x,y,z)\in\mathbb{R}^3:x-2\,y+3\,z=0\}
        =   \{(2\,y-3\,z,y,z):\{y,z\}\subset\mathbb{R}\}
        = &\\&
        =   \langle{(2,1,0),(-3,0,1)}
        \implies &\\&
        \implies
            F^\bot
        =   \{(x,y,z)\in\mathbb{R}^3:
                (x,y,z)|(2,1,0)=0
            \land
                (x,y,z)|(-3,0,1)=0
            \}
        = &\\&
        =   \{(x,y,z)\in\mathbb{R}^3:
                y=-2\,x
            \land
                z=3\,x
            \}
        =   \langle{(1,-2,3)}\rangle
        &
    \end{flalign*}
    
\end{questionBox}

\begin{propositionBox}1{}
    \begin{BM}
        F\oplus F^\bot=\mathbb{R}^3\land\dim \mathbb{R}^3=\dim F + \dim F^\bot
    \end{BM}
\end{propositionBox}

\begin{propositionBox}1{}
    \begin{BM}
        F^\bot=\mathcal{L}(A):
    \\  F\text{ subespaço de }\mathbb{R}^n\text{ das soluções}
    \\  \text{de equações lineares homogêneo } AX=0
    \end{BM}
\end{propositionBox}

\begin{sectionBox}1{Projeção Ortogonal}
    
    \begin{BM}
        u = \proj_F{u} +\proj_{F^\bot}{u}:
        \begin{cases}
            \proj_{F\phantom{^\bot}}(u)\in F
        \\  \proj_{F^\bot}(u)\in F^\bot
        \end{cases}
    \end{BM}
    
\end{sectionBox}

\end{document}

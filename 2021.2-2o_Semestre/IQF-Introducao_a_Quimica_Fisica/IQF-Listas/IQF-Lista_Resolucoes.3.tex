% L3 Resolução
\part{}



% Q1.1
\begin{questionBox}1{}
    \begin{flalign*}
        &
            V
        =   \frac
                {\unit{\litre}}
                {0.17685\,\unit{\gram}}
        \,  10\,\unit{\gram}
        \cong
            \qty{56}{\liter}
        &
    \end{flalign*}
\end{questionBox}



% Q1.2
\begin{questionBox}1{}
    \ch{%
        C14H18N2O5 + 16 O2 <=> 14 CO2\gas + 9 H2O\lqd + N2\gas%
    }
\end{questionBox}



% Q1.3
\begin{questionBox}1{}
    \begin{flalign*}
        &
            m\,\unit{\gram\of{\ch{Fe}}}
        =
        &
    \end{flalign*}
\end{questionBox}



% Q1.3
\setcounter{question}{2}
\begin{questionBox}1{}

    % 20g H2O\gas * 1 mole H2O\gas * 40.7 kilo KJ
    %               18g    H2O\gas   1 mole H2O\gas
    % >
    % 20g gelo * 1 mole gelo * 6.01 KJ
    %          * 18g gelo    * 1 mole gelo

    % \begin{flalign*}
    %     &
    %
    %     &
    % \end{flalign*}

\end{questionBox}

% Q1.5
\setcounter{question}{4}
\begin{questionBox}1{}

    \begin{center}
        \ch{
            CH4\gas + 2 O2\gas -> 2 H2O\gas + CO2\gas
        }
    \end{center}

    % Pelo O2 ser o mais abundante n se conta a sua entalpia de formação

    % Q1.5.a)
    \begin{questionBox}2{}
        \begin{flalign*}
            &
                \Delta H_t
            =
            -   \Delta H_{\text{form}\,\ch{CH4}}
            +   \Delta H_{\text{form}\,\ch{H2O}}
            +   \Delta H_{\text{form}\,\ch{CO2}}
            =   \left(
                    \frac{74.9\,\unit{\kilo\joule}}
                         {      \unit{\mole\of{\ch{CH4}}}}
                -   \frac{286\,\unit{\kilo\joule}}
                         {   \unit{\mole\of{\ch{H2O}}}}
                *   \frac{2\,\unit{\mole\of{\ch{H2O}}}}
                         {   \unit{\mole\of{\ch{CH4}}}}
                -   \frac{394\,\unit{\kilo\joule}}
                         {     \unit{\mole\of{\ch{CO2}}}}
                *   \frac{\unit{\mole\of{\ch{CO2}}}}
                         {\unit{\mole\of{\ch{CH4}}}}
                \right)
                \frac{      \unit{\mole\of{\ch{CH4}}}}
                     {22.4\,\unit{\deci\metre^3\of{\ch{CH4}}}}
            *   1\,\unit{\metre^3\of{\ch{CH4}}}
            \cong &\\&
            \cong
                % \qty{...}{\kilo\joule}
            &
        \end{flalign*}
    \end{questionBox}

    % Q1.5.b)
    \begin{questionBox}2{}

        % \begin{flalign*}
        %     &
        %
        %     &
        % \end{flalign*}

    \end{questionBox}

\end{questionBox}

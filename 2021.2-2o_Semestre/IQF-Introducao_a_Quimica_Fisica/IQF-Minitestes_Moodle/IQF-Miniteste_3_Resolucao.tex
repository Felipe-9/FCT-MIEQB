% !TEX root = ./IQF - Minitestes.tex
% IQF 2021.2 Miniteste 3
\setcounter{part}{2}
\part{}

\begin{questionBox}1{}
    
    Titulou-se 12\,\unit{\milli\liter} de uma solução de 0.09\,\unit{\molar} de ibuprofeno (\ch{(CH3)2CHC6H5CH(CH3)COOH}, \(K_a= 3.72\,\mathrm{E}-5\)) (princípio activo anti-inflamatório) com uma solução 0.15\,\unit{\molar} de \ch{NaOH}, a 25\,\unit{\celsius}.

    % \ch{
    %         (CH3)2CHC6H5CH(CH3)COOH
    %     +   H2O
    %     <>[\num{3.72E-5}]
    %         (CH3)2CHC6H5CH(CH3)COO-
    %     +   H3O^+
    %     % +   H2O
    % }

    \begin{enumerate}
        \begin{multicols}{3}
            \item Titulado
            \item Titulante
            \item Solução final
        \end{multicols}
    \end{enumerate}

    \begin{questionBox}3m{O pH da solução de ibuprofeno antes de se iniciar a titulação é \line(1,0){2em}}

        % \begin{table}[H]\centering
        %     \begin{tabular}{c r r r r}
                
        %         \toprule

        %         &   \multicolumn{1}{          c}{\ch{HA\aq{1}}}
        %         &   \multicolumn{1}{@{\,+\,}  c}{\ch{H2O\lqd{1}}}
        %         &   \multicolumn{1}{@{\ch{<>}}c}{\ch{A^{-}\aq{1}}}
        %         &   \multicolumn{1}{@{\,+\,}  c}{\ch{H3O^{+}\aq{1}}}

        %         \\\midrule

        %             ini
        %         &   \ch{[HA\aq{1}]_i}
        %         &   --
        %         &   0
        %         &   0

        %         \\  
                    
        %             eq
        %         &   \ch{[HA\aq{1}]_i}-x
        %         &   --
        %         &   x
        %         &   x

        %         \\\bottomrule

        %     \end{tabular}
        % \end{table}

        \begin{flalign*}
            &
                \begin{aligned}
                    &
                    \ch{[H3O^+\aq{1}]} = x
                \,\land \\
                \land\, &
                    \frac{x^2}{\ch{[HA\aq{1}]_i}-x}
                =   K_a
                % % 
                %     x^2+K_a\,x-K_a\,\ch{[C13H18O2\aq{1}]_i}
                % =   0
                \end{aligned}
            \implies &\\&
            \implies
                \begin{aligned}
                    &
                        p\ch{H} = -\log(x)
                    \,\land \\
                    \land\, &
                        x^2+K_a\,x-K_a\,\ch{[HA\aq{1}]}_i = 0
                \end{aligned}
            \implies &\\&
            \implies
                p\mathrm{H}
            =   -\log
                \left(
                    \frac
                        {
                            -K_a\pm
                            \sqrt{
                                K_a^2
                            -   4
                            \,  (-K_a\,\ch{[HA\aq{1}]}_i)
                            }
                        }
                        {2}
                \right)
            = &\\& 
            =   -\log
                \left(
                    \frac
                        {
                            -3.72\E-5\pm
                            \sqrt{
                                (3.72\E-5)^2
                            -   4
                            \,  (-3.72\E-5*0.09)
                            }
                        }
                        {2}
                \right)
            \cong &\\&
            \cong
                \num{2.74202193320574}
            &
        \end{flalign*}
        
    \end{questionBox}

\end{questionBox}
\begin{questionBox}{}

    \begin{questionBox}3{Para que o p\ch{H} da solução seja igual a 4.43 temos que adicionar \line(1,0){2em}\,\unit{\milli\liter} de solução de \ch{NaOH\aq{}}.}

        % \begin{table}[H]\centering
        %     \begin{tabular}{c *{4}{l}}
                
        %         \toprule
                
        %         % &   \multicolumn{1}{c @{\,+}      }{\ch{HA}}
        %         % &   \multicolumn{1}{c @{\,\ch{<>}}}{\ch{H2O}}
        %         % &   \multicolumn{1}{c @{\,+}      }{\ch{A^-}}
        %         % &   \multicolumn{1}{c             }{\ch{H3O^+}}

        %         % \\\midrule
                
        %         %     t0
        %         % &   \(\ch{[HA]}_{t0}\)
        %         % &   --
        %         % &   0
        %         % &   0

        %         % \\\midrule
                
        %         %     t1
        %         % &   \(\ch{[HA]}_{t0}-x\)
        %         % &   --
        %         % &   0
        %         % &   0

        %         % \\\bottomrule

        %         % \multicolumn{5}{r}{
        %         %     Concentrações(\unit{\molar}) para Titulado
        %         % }

        %         % \\\toprule

        %         &   \multicolumn{1}{c @{\,+}      }{\ch{HA}}
        %         &   \multicolumn{1}{c @{\,\ch{<>}}}{\ch{OH^-}}
        %         &   \multicolumn{1}{c @{\,+}      }{\ch{A^-}}
        %         &   \multicolumn{1}{c             }{\ch{H2O}}
                
        %         \\\midrule
                
        %             \(t0\)
        %         &   \(\ch{[HA]}_{t0}\)
        %         &   \(\ch{[OH^-]}_{t0}\)
        %         &   \(0\)
        %         &   --
                
        %         \\  
                
        %             \(t1\)
        %         &   \(\ch{[HA]}_{t1}\)
        %         &   \(0\)
        %         &   \(\ch{[OH^-]}_{t0}\)
        %         &   --
                
        %         \\\toprule
                
        %         &   \multicolumn{1}{c @{\,+}      }{\ch{HA}}
        %         &   \multicolumn{1}{c @{\,\ch{<>}}}{\ch{H2O}}
        %         &   \multicolumn{1}{c @{\,+}      }{\ch{A^-}}
        %         &   \multicolumn{1}{c             }{\ch{H3O^+}}
                
        %         \\\midrule

        %             t1
        %         &   \(\ch{[HA]}_{t1}\)
        %         &   --
        %         &   \(\ch{[A^-]}_{t1}\)
        %         &   \(0\)

        %         \\

        %             t2
        %         &   \(\ch{[HA]}_{t1}-y\)
        %         &   --
        %         &   \(\ch{[A^-]}_{t1}+y\)
        %         &   \(y\)

        %         \\\bottomrule

        %         \multicolumn{5}{r}{
        %             Concentrações(\unit{\molar}) para solução final
        %         }

        %     \end{tabular}
        % \end{table}

        \begin{flalign*}
            &
                \begin{aligned}
                &
                    pKa
                =   -\log(K_a)
                =   \num{4.429457060118102}
                =   pH
                \,\land \\
                \land\, &
                    K_a\,\ch{[HA]}
                =   \ch{[A^-][H3O^+]}
                \end{aligned}
            \implies &\\&
            \implies
                Vol_2
            =   N_{\ch{NaOH\aq{2}}}
            /   2\,\ch{[NaOH\aq{2}]}
            =   \ch{[HA]}_{t0}\,Vol_1
            /   2\,\ch{[NaOH\aq{2}]}
            = &\\&
            =   \frac
                    {0.09*0.120}
                    {2*0.15}
            =   \qty{36}{\milli\liter\of{\ch{NaOH\aq{2}}}}
            &
        \end{flalign*}

    \end{questionBox}
    
\end{questionBox}
\begin{questionBox}{}

    \begin{questionBox}3{O pH da solução após a adição de 4.0\,\unit{\milli\liter} da solução de \ch{NaOH} é \line(1,0){2em}}

        \begin{table}[H]\centering
            \begin{tabular}{c *{4}{l}}
                
                \toprule
                
                &   \multicolumn{1}{c @{\,+}      }{\ch{HA}}
                &   \multicolumn{1}{c @{\,\ch{<>}}}{\ch{OH^-}}
                &   \multicolumn{1}{c @{\,+}      }{\ch{A^-}}
                &   \multicolumn{1}{c             }{\ch{H2O}}
                
                \\\midrule
                
                    \(t0\)
                &   \(\ch{[HA]}_{t0}\)
                &   \(\ch{[OH^-]}_{t0}\)
                &   \(0\)
                &   --
                
                \\
                
                    \(t1\)
                &   \(\ch{[HA]}_{t1}\)
                &   \(0\)
                &   \(\ch{[OH^-]}_{t0}\)
                &   --
                
                \\\toprule
                
                &   \multicolumn{1}{c @{\,+}      }{\ch{HA}}
                &   \multicolumn{1}{c @{\,\ch{<>}}}{\ch{H2O}}
                &   \multicolumn{1}{c @{\,+}      }{\ch{A^-}}
                &   \multicolumn{1}{c             }{\ch{H3O^+}}
                
                \\\midrule
                
                    \(t2\)
                &   \(\ch{[HA]}_{t1}\)
                &   --
                &   \(\ch{[OH^-]}_{t0}\)
                &   \(\ch{[H3O^+]}_{t2}\)
                
                
                \\\bottomrule
                
                \multicolumn{5}{r}{
                    Concentrações(\unit{\molar}) para solução final
                }
                
            \end{tabular}
        \end{table}

        \begin{flalign*}
            &
                pH
            =   -\log
                \left(
                    K_a
                \,  \frac
                        {\ch{[HA]}_{t0}-\ch{[OH^-]}_{t0}}
                        {\ch{[OH^-]}_{t0}}
                \right)
            \land &\\&
            \land
                \ch{[HA]}_{t0}
            =   \ch{[HA]}_1*Vol_1/Vol_3
            \land
                \ch{[OH^-]}_{t0}
            =   \ch{[OH^-]}_2*Vol_2/Vol_3
            \implies &\\&
            \implies
                pH
            =   -log
                \left(
                    K_a
                    \left(
                        \frac
                            {\ch{[HA]}_1*Vol_1/Vol3}
                            {\ch{[OH^-]}_2*Vol_2/Vol_3}
                    -   1
                    \right)
                \right)
            = &\\&
            =   -log
                \left(
                    3.72\E-5
                    \left(
                        \frac
                            {0.09*0.012}
                            {0.15*0.004}
                    -   1
                    \right)
                \right)
            \cong
                \num{4.526367073126159}
            &
        \end{flalign*}

    \end{questionBox}

\end{questionBox}
\begin{questionBox}{}

    \begin{questionBox}3{O pH da solução após a adição de 9.0\,\unit{\milli\liter} da solução de \ch{NaOH} é \line(1,0){2em}}

        \begin{table}[H]\centering
            \begin{tabular}{c *{4}{l}}
                
                \toprule
                
                &   \multicolumn{1}{c @{\,+}      }{\ch{HA}}
                &   \multicolumn{1}{c @{\,\ch{<>}}}{\ch{OH^-}}
                &   \multicolumn{1}{c @{\,+}      }{\ch{A^-}}
                &   \multicolumn{1}{c             }{\ch{H2O}}
                
                \\\midrule
                
                    \(t0\)
                &   \(\ch{[HA]}_{t0}\)
                &   \(\ch{[OH^-]}_{t0}\)
                &   \(0\)
                &   --
                
                \\
                
                    \(t1\)
                &   \(0\)
                &   \(\ch{[OH^-]}_{t1}\)
                &   \(\ch{[HA]}_{t0}\)
                &   --
                
                % \\\toprule
                
                % &   \multicolumn{1}{c @{\,+}      }{\ch{HA}}
                % &   \multicolumn{1}{c @{\,\ch{<>}}}{\ch{H2O}}
                % &   \multicolumn{1}{c @{\,+}      }{\ch{A^-}}
                % &   \multicolumn{1}{c             }{\ch{H3O^+}}
                
                % \\\midrule
                
                %     \(t1\)
                % &   \(\ch{[HA]}_{t1}\)
                % &   --
                % &   \(\ch{[A^-]}_{t1}\)
                % &   \(0\)
                
                % \\
                
                %     \(t2\)
                % &   0
                % &   --
                % &   \(\ch{[HA]}_{t0}\)
                % &   \(y\)
                
                \\\bottomrule
                
                \multicolumn{5}{r}{
                    Concentrações(\unit{\molar}) para solução final
                }
                
            \end{tabular}
        \end{table}

        \begin{flalign*}
            &
                pH
            =   14+\log
                \left( 
                    \ch{[OH^-]}_{t1}
                \right)
            = &\\&
            =   14+\log
                \left( 
                    \ch{[OH^-]}_{t0}-\ch{[HA]}_{t0}
                \right)
            = &\\&
            =   14+\log
                \left( 
                    \left(
                        \ch{[OH^-]}_2
                    \,  Vol_2/Vol_3
                    \right)
                -   \left(
                        \ch{[HA]}_1
                    \,  Vol_1/Vol_3
                    \right)
                \right)
            = &\\&
            =   14+\log
                \left( 
                    \frac
                        {
                            \ch{[OH^-]}_2
                        \,  Vol_2
                        -   \ch{[HA]}_1
                        \,  Vol_1
                        }
                        {Vol_1+Vol_2}
                \right)
            = &\\&
            =   14+\log
                \left(
                    \frac
                        {
                            0.15
                        *   0.0090
                        -   0.009
                        *   0.012
                        }
                        {0.012+0.0090}
                \right)
            \cong &\\&
            \cong
                \num{12.771902301106642}
            &
        \end{flalign*}

    \end{questionBox}

\end{questionBox}
\begin{questionBox}{}

    \begin{questionBox}3{No ponto de equivalência o pH da solução é \line(1,0){2em}}
        
        % \begin{table}[H]\centering
        %     \begin{tabular}{c *{4}{l}}
                
        %         \toprule
                
        %         &   \multicolumn{1}{c @{\,+}      }{\ch{HA}}
        %         &   \multicolumn{1}{c @{\,\ch{<>}}}{\ch{OH^-}}
        %         &   \multicolumn{1}{c @{\,+}      }{\ch{A^-}}
        %         &   \multicolumn{1}{c             }{\ch{H2O}}
                
        %         \\\midrule
                
        %             \(t0\)
        %         &   \(\ch{[HA]}_{t0}\)
        %         &   \(\ch{[OH^-]}_{t0}\)
        %         &   \(0\)
        %         &   --
                
        %         \\
                
        %             \(t1\)
        %         &   \(0\)
        %         &   \(0\)
        %         &   \(\ch{[A^-]}_{t1}\)
        %         &   --
                
        %         \\\toprule
                
        %         &   \multicolumn{1}{c @{\,+}      }{\ch{HA}}
        %         &   \multicolumn{1}{c @{\,\ch{<>}}}{\ch{H2O}}
        %         &   \multicolumn{1}{c @{\,+}      }{\ch{A^-}}
        %         &   \multicolumn{1}{c             }{\ch{H3O^+}}
                
        %         \\\midrule
                
        %             \(t1\)
        %         &   \(0\)
        %         &   --
        %         &   \(\ch{[A^-]}_{t1}\)
        %         &   \(\ch{[H3O^+]}_{t1}\)
                
        %         \\
                
        %             \(t2\)
        %         &   \(\ch{[HA]}_{t2}\)
        %         &   --
        %         &   \(\ch{[A^-]}_{t2}\)
        %         &   \(\ch{[H3O^+]}_{t2}\)
                
        %         \\\bottomrule
                
        %         \multicolumn{5}{r}{
        %             Concentrações(\unit{\molar}) para solução final
        %         }
                
        %     \end{tabular}
        % \end{table}

        \begin{table}[H]\centering
            \begin{tabular}{c *{4}{l}}
                
                \toprule
                
                &   \multicolumn{1}{c @{\,+}      }{\ch{HA}}
                &   \multicolumn{1}{c @{\,\ch{<>}}}{\ch{OH^-}}
                &   \multicolumn{1}{c @{\,+}      }{\ch{A^-}}
                &   \multicolumn{1}{c             }{\ch{H2O}}
                
                \\\midrule
                
                    \(t0\)
                &   \(\ch{[HA]}_{t0}\)
                &   \(\ch{[HA]}_{t0}\)
                &   \(0\)
                &   --
                
                \\
                
                    \(t1\)
                &   \(0\)
                &   \(0\)
                &   \(\ch{[HA]}_{t0}\)
                &   --
                
                \\
                
                    \(t2\)
                &   \(\ch{[HA]}_{t2}\)
                &   \(\ch{[HA]}_{t2}\)
                &   \(\ch{[A^-]}_{t2}\)
                &   --
                
                \\\bottomrule
                
                \multicolumn{5}{r}{
                    Concentrações(\unit{\molar}) para solução final
                }
                
            \end{tabular}\relax
        \end{table}\relax

        \vspace{-8ex}
        
        \begin{flalign*}
            &
                pH
            =   14+\log
                \left(
                    \ch{[HA]}_{t2}
                \right)
            \land &\\&
            \land
                \ch{[HA]}_{t2}
            =   \ch{[HA]}_{t0}
            -   \ch{[A^-]}_{t2}
            \land &\\&
            \land
                \frac
                    {\ch{[A^-]}_{t2}}
                    {\ch{[HA]}_{t2}^2}
            =   K_a
            \implies &\\&
            \implies
                pH
            =   14+\log
                \left(
                    \ch{[HA]}_{t2}
                \right)
            \land &\\&
            \land
                \frac
                    {\ch{[HA]}_{t0}-\ch{[HA]}_{t2}}
                    {\ch{[HA]}_{t2}^2}
            =   K_a
            \implies &\\&
            \implies
                \ch{[HA]}_{t2}
            =   \frac
                    {
                        -2\pm\sqrt{1-4*K_a*(-\ch{[HA]}_{t0})}
                    }
                    {2\,K_a}
            \land &\\&
            \land
                pH
            =   14+\log
                \left(
                    \frac
                    {
                        -1\pm\sqrt{1-4*K_a*(-\ch{[HA]}_{t0})}
                    }
                    {2\,K_a}
                \right)
            = &\\&
            =   14+\log
                \left(
                    \frac
                    {
                        -1\pm\sqrt{1-4*3.72\E-5*(-0.09)} % -1 \pm 1.000006695977582
                    }
                    {2*3.72\E-5}
                \right)
            \cong &\\&
            \cong
                \num{12.954241055432456}
            &
        \end{flalign*}
        
    \end{questionBox}

\end{questionBox}

% Q5
\setcounter{question}{4}
\begin{questionBox}1m{}
    
    Responda às seguintes questões com base nas constantes do produto de solubilidade e nos dados de potencial de redução padrão fornecidos

    \begin{itemize}
        \item \(K_{ps}(\ch{CuCl})=1.0\E-6\)
        \item \(E^\circ (\ch{Cu^+\aq{}/Cu\sld{}}) = 0.36\,\unit{\volt}\)
        \item \(E^\circ (\ch{Fe^{3+}/Fe^{2+}}) = 0.77\,\unit{\volt}\)
    \end{itemize}

    Considere a pilha constituída pelo acoplamento da semi-célula

    \begin{center}
        \ch{Fe^{3+}}(\(1.0\E-5\))\,\unit{\molar},
        \ch{Fe^{2+}}(\(1.0\E-3\))\,\unit{\molar} \big|\ch{Pt\sld{}} com
        \ch{Cu\sld{}/CuCl_{sat}}
    \end{center}

    \begin{center}
        \ch{
            Fe^{3+} + e^- <> Fe^{2+}
        \\  Cu\sld{} <> Cu^{+}\aq{} + e^-
        \\[2ex]
            Fe^{3+} + Cu\sld{} <> Fe^{2+} + Cu^{+}\aq{}
        }
    \end{center}

    \begin{questionBox}3{%
        A 25\,\unit{\celsius} o valor do potencial padrão de redução da pilha assim formada será \(E^\circ = \)\,\line(1,0){2em}\,\unit{\volt}%
    }
        
        \begin{flalign*}
            &
                E^\circ
            =   E^\circ_{\text{catodo}}
            -   E^\circ_{\text{anodo}}
            =   (0.77-0.36)\,\unit{\volt}
            =   0.41\,\unit{\volt}
            &
        \end{flalign*}
        
    \end{questionBox}

    \begin{questionBox}3{%
        Calcule o quociente da reação que ocorre na pilha, \(Q =\)\,\line(1,0){2em}%
    }
        
        \begin{flalign*}
            &
                Q
            =   \frac
                    {\ch{[Fe^{2+}]}\,\ch{[Cu^+]}}
                    {\ch{[Fe^{3+}]}}
            \land &\\&
            \land
                \ch{[Cu^+]\,[Cl^-]}
            =   \ch{[Cu^+]}^2
            =   K_{ps}
            \implies &\\&
            \implies
                Q
            =   \frac
                    {\ch{[Fe^{2+}]}\,\sqrt{K_{ps}}}
                    {\ch{[Fe^{3+}]}}
            = &\\&
            =   \frac
                    {1.0\E-3\,\sqrt{1.0\E-6}}
                    {1.0\E-5}
            = &\\&
            =   0.1
            &
        \end{flalign*}
        
    \end{questionBox}
    
    \begin{questionBox}3{%
        A pilha apresenta uma diferença de potencial \(E=\)\,\line(1,0){2em}%
    }
        
        \begin{flalign*}
            &
                E
            =   E^\circ
            -   0.0257
            \,  \ln(Q)
            /   n
            = &\\&
            =   0.41
            -   0.0257
            \,  \ln(0.1)
            \,  \unit{\volt}
            /   1
            \cong &\\&
            \cong
                \qty{0.469176436889947}{\volt}
            &
        \end{flalign*}
        
    \end{questionBox}

    \begin{questionBox}3{%
        A constante de equilíbrio da reação que ocorre na pilha é \(K =\)\,\line(1,0){2em}%
    }
        
        \begin{flalign*}
            &
                0.0257\,\ln{K}/n
            =   E^\circ
            \implies &\\&
            \implies
                K
            =   \exp
                \left(
                    \frac
                        {n\,E^\circ}
                        {0.0257}
                \right)
            =   &\\&
            =   \exp
                \left(
                    \frac
                        {1*0.41}
                        {0.0257}
                \right)
            =   &\\&
            =   \num{2.718281828459045}
            &
        \end{flalign*}
        
    \end{questionBox}

    Considere agora uma nova pilha onde se utilizou uma solução com \ch{[Cu^+]}\,\( = 0.01\)\,\unit{\molar} ao invés da solução saturada de \ch{CuCl} no elétrodo de cobre, mantendo todas as outras condições constantes.

    \begin{questionBox}3{%
        A nova pilha assim formada tem uma força eletromotriz de \(E = \)\,\line(1,0){2em}\,\unit{\volt}%
    }

        \begin{flalign*}
            &
                E
            =   E^\circ
            -   0.0257
            \,  \ln(Q_2)
            /   n
            \land &\\&
            \land
                Q_2
            =   \frac
                    {\ch{[Fe^{2+}]\,[Cu^{+}]}}
                    {\ch{[Fe^{3+}]}}
            \implies &\\&
            \implies
                E
            =   E^\circ
            -   0.0257
            \,  \ln
                \left(
                    \frac
                        {\ch{[Fe^{2+}]\,[Cu^{+}]}}
                        {\ch{[Fe^{3+}]}}
                \right)
            /   n
            = &\\&
            =   0.41\,\unit{\volt}
            -   0.0257
            \,  \ln
                \left(
                    \frac
                        {1.0\E-2*0.01}
                        {1.0\E-5}
                \right)
            /   1
            \cong &\\&
            \cong 
                \qty{0.350823563110053}{\volt}
            &
        \end{flalign*}
        
    \end{questionBox}

\end{questionBox}

% Q6
\begin{questionBox}1{}
    
    Para preparar 250\,\unit{\milli\liter} de uma solução de ácido nítrico 0.39\,\unit{\molar}, partindo de uma solução concentrada do ácido nítrico (70\,\unit{\percent(\gram/\gram)} e densidade = 1.420\,\unit{\kilo\gram\per\liter}) que volume de solução de \ch{HNO3} concentrada necessito?

    \sisetup{
        % scientific-notation = engineering,  % scientific/engineering/fixed/false
        % output-exponent-marker = {\,\mathrm{E}},
        round-precision     = 3,
        round-mode          = places,       % figures/places/none
        exponent-to-prefix  = false,        % 1000 g -> 1 kg
        % round-minimum       = 0.01,
        % fixed-exponent      = 0,
    }

    \begin{flalign*}
        &
            Vol_{\ch{HNO3}}
        =   \frac
                {\unit{\liter\of{\ch{HNO3\aq{1}}}}}
                {1.420\,\unit{\kilo\gram\of{\ch{HNO3\aq{1}}}}}
        \,  \frac
                {\unit{\gram\of{\ch{HNO3\aq{1}}}}}
                {0.7\,\unit{\gram\of{\ch{HNO3}}}}
        \,  \frac
                {63.01\,\unit{\gram\of{\ch{HNO3}}}}
                {\unit{\mole\of{HNO3}}}
        \,  \frac
                {0.39\,\unit{\mole\of{HNO3}}}
                {\unit{\liter\of{\ch{HNO3\aq{2}}}}}
        \,* &\\&
        *   250\,\unit{\milli\liter\of{\ch{HNO3\aq{2}}}}
        \cong
            \qty{6.180558350100604}{\milli\liter\of{\ch{HNO3\aq{1}}}}
        &
    \end{flalign*}

\end{questionBox}

% Q7
\begin{questionBox}1{}
    
    Qual a diferença de potencial de uma pilha constituida por um elétrodo de prata e um elétrodo de estanho mergulhados respectivamente numa solução \(9.09\E-2\)\,\unit{\molar} em \ch{Ag^+} e numa solução \(3.87\E-2\)\,\unit{\molar} em \ch{Sn^{2+}}, a 25\,\unit{\celsius}?

    \begin{itemize}
        \item \(E^\circ (\ch{Ag^+/Ag}) = 0.800\,\unit{\volt}\)
        \item \(E^\circ (\ch{Sn^{2+}/Sn}) = -0.136\,\unit{\volt}\)
    \end{itemize}

    
    \begin{center}
        \ch{
            2 Ag^+\aq{} + 2 e^- <> 2 Ag\sld{}
        \\  Sn\sld{} <> Sn^{2+}\aq{} + 2 e^-
        }
    \end{center}

    \begin{flalign*}
        &
            E
        =   E^\circ - 0.0257\,\ln(Q)/n
        = &\\&
        =   (0.800-(-0.0136))\,\unit{\volt}
        -   0.0257
        \,  \ln
            \left(
                \frac
                    {3.87\E-2}
                    {(9.09\E-2)^2}
            \right)
        \,  \unit{\volt}
        /   2
        \cong &\\&
        \cong
            \qty{0.824572877154748}{\volt}
        &
    \end{flalign*}
    
\end{questionBox}

% Q10
\setcounter{question}{9}
\begin{questionBox}1{}
    
    Qual a concentração de amónia aquosa (\ch{NH3}) em \unit{\mole\per\cubic\deci\meter} (\unit{\molar}) necessária para iniciar a precipitação de \ch{Fe(OH)2} de uma solução 0.0025\,\unit{\molar} em \ch{FeCl2}?
    
    \begin{itemize}
        \item \(K_{b}  (\ch{NH3}) = 1.8\E-5\)
        \item \(K_{ps} (\ch{Fe(OH)2}) = 1.6\E-14\)
    \end{itemize}

    \begin{table}[H]\centering
        \begin{tabular}{c *{4}{c}}
            
            \toprule
            
            &   \multicolumn{1}{c @{\,+}      }{\ch{NH3\aq{}}}
            &   \multicolumn{1}{c @{\,\ch{<>}}}{\ch{H2O\lqd{}}}
            &   \multicolumn{1}{c @{\,+}      }{\ch{NH4^+\aq{}}}
            &   \multicolumn{1}{c             }{\ch{OH^-\aq{}}}
            
            \\\midrule
            
                \(t0\)
            &   \(\ch{[NH3]}_{t0}\)
            &   --
            &   0
            &   0
            
            \\
            
                \(t1\)
            &   \(\ch{[NH3]}_{t1}\)
            &   --
            &   \(\ch{[NH4^+]}_{t1}\)
            &   \(\ch{[OH^-]}_{t1}\)
            
            \\\bottomrule
            
            &   \multicolumn{2}{c @{\,\ch{<>}}}{\ch{Fe(OH)2\sld{}}}
            &   \multicolumn{1}{c @{\,+}      }{\ch{Fe^{2+}\aq{}}}
            &   \multicolumn{1}{c             }{\ch{2 OH^-\aq{}}}
            
            \\\midrule
            
                \(t1\)
            &   \multicolumn{2}{c}{--}
            &   \(\ch{[Fe^{2+}]}_{t1}\)
            &   \(\ch{[OH^-]}_{t1}\)
            
            \\\bottomrule
            
            % \multicolumn{5}{r}{
            %     Concentrações(\unit{\molar}) para solução final
            % }
            
        \end{tabular}
    \end{table}

    \begin{flalign*}
        &
            \ch{[NH3]}_{t0}
        =   \ch{[NH3]}_{t1}
        +   \ch{[OH^-]}_{t1}
        \land &\\&
        \land
            \frac
                {\ch{[OH^-]}_{t1}^2}
                {\ch{[NH3]}_{t1}}
        =   K_b
        \land &\\&
        \land
            \ch{[Fe^{2+}]}_{t1}
        \,  \ch{[OH^-]}_{t1}^2
        =   K_{ps}
        \implies &\\&
        \implies
            \ch{[NH3]}_{t0}
        =   
            \frac
                { |K_{ps}/\ch{[Fe^{2+}]}_{t1}| }
                {K_b}
        +   \sqrt{K_{ps}/\ch{[Fe^{2+}]}_{t1}}
        = &\\&
        =
            \frac
                { |1.6\E-14/0.0025| }
                {1.8\E-5}
        +   \sqrt{1.6\E-14/0.0025}
        \cong &\\&
        \cong
            \qty{2.885377683690259E-6}{\molar}
        &
    \end{flalign*}
    
\end{questionBox}
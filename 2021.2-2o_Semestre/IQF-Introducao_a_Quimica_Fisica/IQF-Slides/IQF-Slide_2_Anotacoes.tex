% !TEX root = ./IQF - Slides Anotações.tex
% Slide 2 - Introdução à Química-Física

\setcounter{part}{1}
\part{Introdução à Química-Física}

% Gases reais
\begin{sectionBox}1m{Gases Reais}

    \begin{sectionBox}*2{Fator de compressibilidade}

        \begin{BM}
            Zvê q tem varias temperaturas apontando p cada linha, elas dizem onde tá a fronteira entre as fazes a cada temperatura
            =   \bar{V}_{\text{real}}/\bar{V}_{\text{ideal}}
            =   p\,V_m/R\,T
        \end{BM}

        \begin{BM}
            p\,\bar{V}=Z\,R\,T
            \quad
            \begin{cases}
                T\uparrow  \implies Z>1
            \\  T\downarrow\implies Z<1
            \end{cases}
        \end{BM}
    \end{sectionBox}

    \begin{sectionBox}*2{Expansão do Virial}
        \begin{BM}
            \frac{p\,V}{R\,T}=Z_{(T)}=1+B_{(T)}/\bar{V} + C_{(T)}/\bar{V}^2 + \dots
        \end{BM}

        \paragraph{Obs:} \(B,C,\dots\) são termos normalmente despresados

        \begin{BM}
            \lim\limits_{\footnotesize
                \begin{array}{l@{}l}
                    p       & \to0
                \\  \bar{V} & \to\infty
                \end{array}
            }
            Z_{(T)} = 1
        \end{BM}
    \end{sectionBox}

    \begin{sectionBox}*2{Equação de Estado de van der Walls}

        \begin{BM}
            \big( p+a\,n^2/\bar{V}^2 \big)(\bar V - n\,b) = n\,R\,T
        \end{BM}

        \begin{itemize}[leftmargin = 6em]
            \item[\( a\,n^2/\bar{V}^2 \rightarrow\)] forças atrativas
            \item[\( \bar V - n\,b    \rightarrow\)] forças repulsivas
        \end{itemize}

    \end{sectionBox}

\end{sectionBox}

\begin{sectionBox}1m{Equilíbrio das Fases}

    \paragraph{Equilíbrio Dinâmico}%
        Taxa de evaporação = Taxa de Condensação

    \paragraph{Pressão de vapor de equilíbrio}%
        Pressão de vapor medida em condições de equilíbrio dinâmico entre a condensação e a evaporação


    \begin{sectionBox}*2{Equação de Clausius-Clpaeyron}

        \begin{BM}
            \ln P = C - \Delta H_{\text{vap}}/R\,T
        \end{BM}

        \paragraph{Obs:} $P$ = Pressão do vapor no equilíbrio

    \end{sectionBox}

    \begin{sectionBox}*2{Pontos Críticos}
        \paragraph{T\textsubscript{c}} Temperatura a qual acima dela a substancia não possa se liquefazer independentemente da pressão
        \paragraph{P\textsubscript{c}} Pressão para liquefazer a substancia quando na temperatura crítica
    \end{sectionBox}

    \begin{sectionBox}*2{Diagrama de Fases P\,\times\,T}
        Informa com base em pressão\,\times\,temperatura o estado que se encontra a substancia

        \subsubsection*{Obs:}
        \begin{itemize}
            \item Possivel vilualizar o ponto triplo (trifásico)
            \item Sobre as fronteiras se pode encontrar multiplos estados da substancia
        \end{itemize}

        \begin{sectionBox}*3{Graus de liberdade}
            \begin{BM}
                V = \#_{\text{Componentes}} - \#_{\text{Fases}} + 2
            \end{BM}

            \subsubsection*{Obs:}
            \begin{itemize}
                \item Diferentes fases de uma substancia não conta como diferentes componentes
                \item Quantidade de variáveis que posso variar para manter a situação
                \item pode definir um ponto (\(V=0\)), uma curva (\(V=1\)) ou uma area (\(V=2\))
            \end{itemize}

        \end{sectionBox}

    \end{sectionBox}

    \begin{sectionBox}*2{Diagrama de Fases P\,\times\,V}
        Informa com base em pressão\,\times\,volume e curvas isotérmicas o estado que se encontra a substancia.

        \subsubsection*{Obs:}
        \begin{itemize}
            \item Possível se encontrar T\textsubscript{c} e P\textsubscript{c}
            \item Descontinuidade das curvas isotérmicas com T\,<\,T\textsubscript{c}
        \end{itemize}

    \end{sectionBox}


\end{sectionBox}

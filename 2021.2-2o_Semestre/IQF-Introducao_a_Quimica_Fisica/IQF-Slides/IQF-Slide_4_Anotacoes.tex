% !TEX root = ./IQF - Slides Anotações.tex
% IQF - Anotações do slide 4

\setcounter{part}{3}
\part{Variações de Entalpia}

\begin{sectionBox}1m{Capacidades Caloríficas}
    
    \begin{BM}
        C = Q/\adif T
    \end{BM}

    \begin{sectionBox}*2{A volume constante}
        
        \begin{BM}
            C_V = \adv{U}{T}
        \end{BM}
        
        \begin{flalign*}
            &
                \left.
                    \begin{aligned}
                    &   C = Q / \adif T
                    \,\land\\\land\,
                    &   Q + W = \adif U
                    \,\land\\\land\,
                    &   W = 0
                    \end{aligned}
                \right\}
            % \implies &\\&
            \implies 
                C = \adv{U}{T}
            &
        \end{flalign*}

    \end{sectionBox}

    \begin{sectionBox}*2{A pressão constante}
        
        \begin{BM}
            C_P = \adv{H}{T}
        \end{BM}
        
        \begin{flalign*}
            &
                \left.
                    \begin{aligned}
                    &   C = Q / \adif T
                    \,\land\\\land\,
                    &   Q + W = \adif U
                    \,\land\\\land\,
                    &   \adif U = \adif H + P\,\adif V
                    \,\land\\\land\,
                    &   W = P_{\text{ext}}\,\adif V
                    \,\land\\\land\,
                    &   P_{\text{ext}} = P
                    \end{aligned}
                \right\}
            % \implies &\\&
            \implies 
                C = \adv{H}{T}
            &
        \end{flalign*}

    \end{sectionBox}
    
    \begin{sectionBox}*2{Para gases ideais}
        
        \begin{BM}
            C_P = C_V + N\,R
        \end{BM}

        \begin{flalign*}
            &
                \left.
                    \begin{aligned}
                    &   C_P = \adv{H}{T}
                    \,\land\\\land\,
                    &   H \coloneqq U + P\,V
                    \,\land\\\land\,
                    &   P\,V = N\,R\,T
                    \end{aligned}
                \right\}
            \implies &\\& 
            \implies 
                C_P = \adv{U}{T} + N\,R
            =   C_V + N\,R
            &
        \end{flalign*}
        
    \end{sectionBox}

    \paragraph{Capacidade Calorífica molar} se obtem quando divide a respectiva capacidade calorífica pela quantidade da substancia, denotando por exemplo: \(C_{P\,m}\)

\end{sectionBox}

\begin{questionBox}1{}
    
    Calcule a temperatura final resultante da mistura de 20\,\unit{\gram} de vapor de água a 100\,\unit{\celsius} e de 20\,\unit{\gram} de gelo a 0\,\unit{\celsius}
    
    \begin{itemize}
        \begin{multicols}{2}
            \item \(\adif H^{\circ}_{\text{fus}} (\ch{H2O}) =  6.01\,\unit{\kilo\joule\per\mole}\)
            \item \(\adif H^{\circ}_{\text{vap}} (\ch{H2O}) = 40.7 \,\unit{\kilo\joule\per\mole}\)
            \item \(C_P (\ch{H2O}) = 4.2\,\unit{\joule\per\kelvin\per\gram}\)
        \end{multicols}
    \end{itemize}

    % M O = 15.9994
    % M H = 1.00794
    % M H2O = 18.01528 = 15.9994 + 2*1.00794

    \sisetup{
        % scientific-notation = engineering,  % scientific/engineering/fixed/false
        % output-exponent-marker = {\,\mathrm{E}},
        % round-precision     = 2,
        % round-mode          = places,       % figures/places/none
        % exponent-to-prefix  = false,        % 1000 g -> 1 kg
        % round-minimum       = 0.01,
        % fixed-exponent      = 0,
        per-mode = fraction
    }

    \begin{flalign*}
        &
            \adif H
        =   0
        =   \left(
            \begin{array}{ll}
                  &   \adif H_{20\,\unit{\gram},sld \to liq}
            \,+\\+&   \adif H_{20\,\unit{\gram},liq, (0\to100)\,\unit{\celsius}}
            \,+\\+&   \adif H_{20\,\unit{\gram},liq\to vap}
            \,+\\+&   \adif H_{40\,\unit{\gram},vap\to liq}
            \,+\\+&   \adif H_{40\,\unit{\gram},liq, (100\to t)\,\unit{\celsius}}
            \end{array}
            \right)
        = &\\&
        =   20\,\unit{\gram\of{\ch{H2O}}}
        \,  \left(
                \frac{\unit{\mole\of{\ch{H2O}}}}{\qty{18.01528}{\gram\of{\ch{H2O}}}}
                \left(
                    \frac{6.01\,\unit{\kilo\joule}}{\unit{\mole\of{\ch{H2O}}}}
                +   \frac{40.7\,\unit{\kilo\joule}}{\unit{\mole\of{\ch{H2O}}}}
                \right)
        %     \right.
        % &\\&
        %     \left.
            +   \frac{4.2\,\unit{\joule}}{\unit{\kelvin\kilo\gram\of{\ch{H2O}}}}
            \,  (100-0)\,\unit{\kelvin}
            \right)
        \,+ &\\&
        +   40\,\unit{\gram\of{\ch{H2O}}}
        \,  \left(
                \frac{\unit{\mole\of{\ch{H2O}}}}{\qty{18.01528}{\gram\of{\ch{H2O}}}}
            \,  \left(
                    \frac{-40.7\,\unit{\kilo\joule}}{\unit{\mole\of{\ch{H2O}}}}
                \right)
            +   \frac{4.2\,\unit{\joule}}{\unit{\kelvin\gram\of{\ch{H2O}}}}
            \,  (t-100)\,\unit{\kelvin}
            \right)
        \implies &\\&
        \implies 
            t\,\unit{\celsius}
        =   \left(
            -   20
            \,  \left(
                    \frac
                        {
                            6.01\,\text{k}
                        +   40.7\,\text{k}
                        }
                        {\num{18.01528}}
                +   (100+273)*4.2/\text{k}
                \right)
            \right.
        \,+ &\\&
            \left.
            +   40
            \,  \left(
                    \frac{40.7\,\text{k}}{\num{18.01528}}
                +   (100+273)*4.2/\text{k}
                \right)
            \right)
            (
                4.2
            *   40
            )^{-1}
        \,  \unit{\celsius}
        -   273\,\unit{\celsius}
        \cong &\\&
        % \cong
        %     \qty{}{\celsius}
        &
    \end{flalign*}

\end{questionBox}

\begin{questionBox}1{}
    
    Uma peça de ferro com 150\,\unit{\gram} foi aquecida até 500\,\unit{\celsius} é introduzida rapidamente num recipiente isolado com 25\,\unit{\gram} de gelo a -25\,\unit{\celsius} o qual foi imediatamente selado.

    \begin{itemize}
        \begin{multicols}{2}
            \item \(C_P(\ch{Fe}) = 25.1\,\unit{\joule\per\kelvin\per\mole}\)
            \item \(M  (\ch{Fe}) = 55.85\,\unit{\gram\per\mole}\)
            \item \(\adif H^{\circ}_{\text{fus}} (\ch{H2O}) =  6.0\,\unit{\kilo\joule\per\mole}\)
            \item \(\adif H^{\circ}_{\text{vap}} (\ch{H2O}) = 40.7\,\unit{\kilo\joule\per\mole}\)
            \item \(C_P(\ch{H2O\sld}) = 1.94\,\unit{\joule\per\kelvin\per\gram}\)
            \item \(C_P(\ch{H2O\lqd}) = 4.18\,\unit{\joule\per\kelvin\per\gram}\)
            \item \(C_P(\ch{H2O\gas}) = 2.01\,\unit{\joule\per\kelvin\per\gram}\)
        \end{multicols}
    \end{itemize}

    % \begin{flalign*}
    %     &
    %         \begin{array}{@{\bullet\enspace} l @{\quad\bullet\enspace} l @{}}
    %             C_P(\ch{Fe}) = 25.1\,\unit{\joule\per{\kelvin\mole}}
    %         &   C_P(\ch{H2O\sld}) = 1.94\,\unit{\joule\per{\kelvin\gram}}
    %         \\  M  (\ch{Fe}) = 55.85\,\unit{\gram\per\mole}
    %         &   C_P(\ch{H2O\lqd}) = 4.18\,\unit{\joule\per{\kelvin\gram}}
    %         \\  \adif H^{\circ}_{\text{fus}} (\ch{H2O}) =  6.0\,\unit{\kilo\joule\per\mole}
    %         &   C_P(\ch{H2O\gas}) = 2.01\,\unit{\joule\per{\kelvin\gram}}
    %         \\  \adif H^{\circ}_{\text{vap}} (\ch{H2O}) = 40.7\,\unit{\kilo\joule\per\mole}
    %         \end{array}
    %     &
    % \end{flalign*}

\end{questionBox}
\begin{questionBox}{}

    \begin{questionBox}2m{}
        
        Admitindo que não há percas de energia para o exterior, qual a temperatura final do sistema?
        

        \begin{flalign*}
            &
                \adif H
            =   0
            =   \left(
                    \begin{array}{ll}
                            &   \adif H_{\ch{H2O},sld       ,(-25\to  0)\unit{\celsius}}
                    \,+\\+\,&   \adif H_{\ch{H2O},sld\to lqd,0\,        \unit{\celsius}}
                    \,+\\+\,&   \adif H_{\ch{H2O},lqd       ,(  0\to100)\unit{\celsius}}
                    \,+\\+\,&   \adif H_{\ch{H2O},lqd\to gas,100\,      \unit{\celsius}}
                    % \,+\\+\,&   \adif H_{\ch{H2O},gas       ,(  0\to  t)\unit{\celsius}}
                    \,+\\+\,&   \adif H_{\ch{Fe}            ,(500\to100)\unit{\celsius}}
                    \end{array}
                \right)
            = &\\&
            =   25\,\unit{\gram\of{\ch{H2O}}}
            \,  \left(
                    \begin{array}{ll}
                            &    
                                \frac{1.94\,\unit{\joule}}{\unit{\kelvin\gram\of{\ch{H2O}}}}
                            \,  (0-(-25))\,\unit{\kelvin}
                    \,+\\+\,&   
                                \frac{4.18\,\unit{\joule}}{\unit{\kelvin\gram\of{\ch{H2O}}}}
                            \,  (100-0)\,\unit{\kelvin}
                    % \,+\\+\,&   
                    %             \frac{2.02\,\unit{\joule}}{\unit{\kelvin\gram\of{\ch{H2O}}}}
                    %         \,  ((t+273)-(100+273))\,\unit{\kelvin}
                    \,+\\+\,&   
                                \frac{\unit{\mole\of{\ch{H2O}}}}{\qty{18.01528}{\gram\of{\ch{H2O}}}}
                            \,  \frac{ 6.0\,\unit{\kilo\joule}}{\unit{\mole\of{\ch{H2O}}}}
                    \end{array}
                \right)
            +   m\,\unit{\gram\of{\ch{H2O}}}
            \,  \frac{\unit{\mole\of{\ch{H2O}}}}{\qty{18.01528}{\gram\of{\ch{H2O}}}}
            \,  \frac{40.7\,\unit{\kilo\joule}}{\unit{\mole\of{\ch{H2O}}}}
            \,+ &\\&
            +   150\,\unit{\gram\of{\ch{Fe}}}
            \,  \frac{\unit{\mole\of{\ch{Fe}}}}{55.85\,\unit{\gram\of{\ch{Fe}}}}
            \,  \frac{25.1\,\unit{\joule}}{\unit{\kelvin\mole\of{\ch{Fe}}}}
            \,  (100-500)\,\unit{\kelvin}
            \implies &\\&
            \implies
                x
            =   
                \left(
                    \begin{array}{ll}
                    -\,&
                        25
                        \,  \left(
                                \begin{array}{ll}
                                        &   1.94*25
                                \,+\\+\,&   4.18*100
                                \,+\\-\,&   2.02\,(100+273)
                                \,+\\+\,&   \frac{6.0\,\text{h}+40.7\,\text{h}}{\num{18.01528}}
                                \end{array}
                            \right)
                        % \,+ &\\&
                    \,+\\+\,&
                        \frac{150*25.1\,(500+273)}{55.85}
                    \end{array}
                \right)
            &\\&
                \left(
                    \begin{array}{ll}
                                &   2.02*25
                        \,+\\+\,&   \frac{25.1*150}{55.85}
                    \end{array}
                \right)^{-1}
            \,  \unit{\celsius}
            -   273\,\unit{\celsius}
            \cong &\\&
            \cong 
                \qty{}{\celsius}
            &
        \end{flalign*}

    \end{questionBox}

    \begin{questionBox}2{}
        
        No equílibrio, em que fases estará presente a água?

        \paragraph{Rs:} Liquido e gás a 100\,\unit{\celsius}
        
    \end{questionBox}
    
\end{questionBox}
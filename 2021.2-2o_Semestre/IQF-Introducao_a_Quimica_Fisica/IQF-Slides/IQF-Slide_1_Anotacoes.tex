Lorem ipsum dolor sit amet consectetur adipisicing elit. Dolore nam, repellat veniam quod in ad iure rem magni ullam? Rem voluptatum nesciunt quaerat sint fuga tenetur exercitationem saepe error ducimus.% Slide 1 - Introdução a Matéria

\setcounter{part}{0}
\part{Introdução a cadeira}



% Apresentação do programa da disciplina
    % Basicamente a gente vai aprender cinética química e equilíbrios

% teremos 3 testes

% em aulas práticas haverão 5 encontros
% 2 online 3 no laboratório

% testes:
    % 1/5 - nota prática
    % 1/5 - nota teórica-paratica
    % 3/5 - nota teórica

    % fazer testes do moodle


% \part{Introdução à Matéria}
%
% % Programa
% \begin{sectionBox}1m{Programa}
% % \begin{multicols}{2}
%
%     \begin{enumerate}[label=\arabic*\textsuperscript{o}, leftmargin=1em]
%
%         \setlist{
%             label      = \arabic{enumi}.\arabic{enumii},
%             leftmargin = 1em
%         }
%
%         % 1.
%         \item Introdução
%         \begin{enumerate}
%             \item Sólidos, líquidos e gases
%             \item Mudanças de fase
%             \item Gases ideais e gases reais
%             \item Equações de estado
%         \end{enumerate}
%
%         % 2.
%         \item Termoquímica Revisão
%         \begin{enumerate}
%             \item sistemas
%             \item trabalho
%             \item energia e calor
%             \item Calorimetria
%             \item Calor específico
%             \item 1ª Lei da Termodinâmica
%             \item Trocas de energia em reacções químicas
%             \item Entalpia
%         \end{enumerate}
%
%         % 3.
%         \item Entropia e energia de Gibbs
%         \begin{enumerate}
%             \item Processos espontâneos
%             \item 2ª lei da Termodinâmica
%             \item Variação total de entropia
%             \item Energia de Gibbs e equilíbrio
%         \end{enumerate}
%
%         % 4.
%         \item Equilíbrio físico
%         \begin{enumerate}
%             \item Diagramas de fase e transições de fase
%             \item Solubilidade
%             \item Propriedades coligativas
%             \item Misturas líquidas binárias
%         \end{enumerate}
%
%         % 5.
%         \item Equilíbrio químico
%         \begin{enumerate}
%             \item Factores que o afectam
%             \item Princípio de Le Châtelier
%         \end{enumerate}
%
%         % 6.
%         \item Equilíbrio ácido-base
%         \begin{enumerate}
%             \item Ácidos e bases conjugados
%             \item Auto-ionização da água e escala de pH
%             \item pH de soluções
%             \item Soluções tampão
%             \item Titulações
%         \end{enumerate}
%
%         % 7.
%         \item Equilíbrio de solubilidade
%         \begin{enumerate}
%             \item Produto de solubilidade
%             \item Efeito de ião comum
%             \item Aplicações
%         \end{enumerate}
%
%         % 8.
%         \item Electroquímica
%         \begin{enumerate}
%             \item Célula galvânica
%             \item Representação esquemática de células galvânicas
%             \item Potencial padrão de eléctrodo. Equação de Nernst
%             \item Célula electrolítica
%             \item Electrólise
%         \end{enumerate}
%
%         % 9.
%         \item Cinética Química
%         \begin{enumerate}
%             \item Velocidades de reacção
%             \item Leis de velocidade
%             \item Método integral e método diferencial
%             \item Lei de Arrhenius e energia de activação
%             \item Mecanismo reaccional
%             \item Reacções elementares
%             \item Molecularidade
%         \end{enumerate}
%
%         % 10.
%         \item Conclusão
%         \begin{enumerate}
%             \item Análise e avaliação da UC pelo docente e alunos.
%         \end{enumerate}
%
%     \end{enumerate}
%
% %  1. Introdução (2h).Sólidos, líquidos e gases. Mudanças de fase.Gases ideais e gases reais.Equações de estado
% %
% %  2. Termoquímica (3h) Revisão: sistemas, trabalho, energia e calor. Calorimetria. Calor específico.1ª Lei da Termodinâmica.Trocas de energia em reacções químicas. Entalpia.
% %
% %  3. Entropia e energia de Gibbs (3h). Processos espontâneos. 2ª lei da Termodinâmica. Variação total de entropia. Energia de Gibbs e equilíbrio.
% %
% %  4. Equilíbrio físico (3h). Diagramas de fase e transições de fase. Solubilidade. Propriedades coligativas. Misturas líquidas binárias.
% %
% %  5. Equilíbrio químico (3h). Factores que o afectam. Princípio de Le Châtelier.
% %
% %  6. Equilíbrio ácido-base (4h). Ácidos e bases conjugados. Auto-ionização da água e escala de pH. pH de soluções. Soluções tampão. Titulações.
% %
% %  7. Equilíbrio de solubilidade (3h). Produto de solubilidade. Efeito de ião comum.  Aplicações.
% %
% %  8. Electroquímica (3h). Célula galvânica. Representação esquemática de células galvânicas. Potencial padrão de eléctrodo. Equação de Nernst. Célula electrolítica. Electrólise.
% %
% %  9. Cinética Química (3h). Velocidades de reacção. Leis de velocidade. Método integral e método diferencial. Lei de Arrhenius e energia de activação. Mecanismo reaccional. Reacções elementares. Molecularidade.
% %
% % 10. Conclusão (1h). Análise e avaliação da UC pelo docente e alunos.
%
% % \end{multicols}
% \end{sectionBox}
%
%
%
% % Testes
% \begin{sectionBox}1{Testes}
% \begin{multicols}{2}
%
%     \subsection*{Testes t.}
%     \begin{enumerate}[label=\arabic*\textsuperscript{o} Teste, leftmargin=4em]
%         \item 2021/11/03 - 1 \to\ 4
%         \item 2021/11/29 - 5 \to\ 6
%         \item 2022/01/11 - 7 \to\ 9
%     \end{enumerate}
%
%     \subsection*{Minitestes tp.}
%     \begin{enumerate}[label=\arabic*\textsuperscript{o} Miniteste, leftmargin=6em]
%         \item 2021/10/(27\to29)
%         \item 2021/11/(24\to27)
%         \item 2021/12/(13\to15)
%         \item 2022/01/(05\to07)
%     \end{enumerate}
%
%     \subsection*{Trabalhos Práticos}
%     \begin{enumerate}[label=TL\arabic*, leftmargin=3em]
%         \item 2021/10/(16\to23)
%         \item 2021/11/(06\to13)
%         \item 2021/12/(23\to26)
%         \item 2021/12/30\to2022/01/03
%         \item 2022/01/(4\to7)
%     \end{enumerate}
%
% \end{multicols}
% \end{sectionBox}
%
% % Aulas Práticas
% \begin{sectionBox}1{Aulas Práticas}
%     \begin{enumerate}[label=\arabic*\textsuperscript{o} Lab., leftmargin=3em]
%         \item Ficha de estudo dirigido: Calorimetria
%         \item Ficha de estudo dirigido: Diagramas de fase
%         \item Preparação de aulas de laboratório 4 e 5
%         \item Titulações ácido-base. Constância do produto de Solubilidade
%         \item Cinética da reacção de corantes com o ião hidróxido
%     \end{enumerate}
% \end{sectionBox}
%
% % Bibliografia recomendada
% \begin{sectionBox}1{Bibliografia}
%     \begin{itemize}[leftmargin=1em]
%         \item \href{~/Library/Mobile\\ Documents/com\\~apple\\~CloudDocs/Library/Estudo\\ \\:\\ Tecnico/Química\\ \\:\\ Chemistry/Chemical\\ Principles\\ -\\ ed.7\\ -\\ Peter\\ Atkins\\ \\(author\\)\\,\\ Loretta\\ Jones\\ \\(author\\)\\,\\ Leroy\\ Laverman\\ -\\ \\(2016\\,\\ Kate\\ Ahr\\ Parker\\).pdf}{Chemical Principles - ed.7 - Peter Atkins (author), Loretta Jones (author), Leroy Laverman - (2016, Kate Ahr Parker)}
%         \item \href{~/Library/Mobile\\ Documents/com\\~apple\\~CloudDocs/Library/Estudo\\ \\:\\ Tecnico/Química\\ \\:\\ Chemistry/Chemical\\ Principles\\ Solutions\\ -\\ ed.5\\ -\\ John\\ Krenos\\,\\ Joseph\\ Potenza\\,\\ Laurence\\ Lavelle\\,\\ Yinfa\\ Ma\\,\\ Carl\\ Hoeger^À^Ù\\ -\\ \\(2010\\,\\ W.H.\\ Freeman\\ \\\&\\ Company\\).pdf}{Chemical Principles Solutions - ed.5 - John Krenos, Joseph Potenza, Laurence Lavelle, Yinfa Ma, Carl Hoeger€™ - (2010, W.H. Freeman \& Company)}
%         \item \href{~/Library/Mobile\\ Documents/com\\~apple\\~CloudDocs/Library/Estudo\\ \\:\\ Tecnico/Química\\ \\:\\ Chemistry/Química\\ -\\ ed.11\\ -\\ Raymond\\ Chang\\,\\ Kenneth\\ A.\\ Goldsby\\ -\\ \\(2013\\,\\ AMGH\\).pdf}{Química - ed.11 - Raymond Chang, Kenneth A. Goldsby - (2013, AMGH)}
%     \end{itemize}
% \end{sectionBox}
%
%
%
% \begin{sectionBox}1{Avaliação}
% \begin{multicols}{2}
%
%     \begin{sectionBox}2{Classificação final}
%         \begin{BM}
%             = (\text{p.} + \text{pt.} + 3\,\text{t.})/5
%         \end{BM}
%     \end{sectionBox}
%
%     \begin{sectionBox}2{Notas}
%         \paragraph{tp.} Média dos 4 minitestes do moodle
%         \paragraph{t.}  Média 3 testes de 90' >= 9.5
%         \paragraph{p.}  Trabalhos práticos
%     \end{sectionBox}
%
%     \begin{sectionBox}2{Frequencia}
%
%         \subsection*{Comp. Teórica}
%         \begin{itemize}
%             \item 4 Minitestes no moodle
%             \item \(50\%\) das Aulas t.
%             \item \(50\%\) das Aulas tp.
%         \end{itemize}
%
%         \subsection*{Comp. Prática}
%         \begin{itemize}
%             \item Todos os trabalhos práticos
%         \end{itemize}
%
%     \end{sectionBox}
%
% \end{multicols}
% \end{sectionBox}

% Natureza dos gazes
\section*{Natureza dos gazes}

% Pressão
\begin{sectionBox}1{Pressão}
    % Introdução á pressão

    \begin{BM}
        \text{Pressão} = \text{Força}/\text{Area}
    \end{BM}

    \begin{sectionBox}*2{Unidades}
        \begin{itemize}
            \item \( 1\,\unit{\Pa}  =   1    \,\unit{\N\per\metre^2} \)
            \item \( 1\,\unit{\atm} = 760    \,\unit{\mmHg} = 760\,\unit{\Torr} \)
            \item \( 1\,\unit{\atm} = 101.325\,\unit{\Pa} \)
        \end{itemize}
    \end{sectionBox}

\end{sectionBox}

\section*{Equação dos Gases Ideais}
\begin{multicols}{2}

    % Lei de boyle
    \begin{sectionBox}1{Lei de Boyle}

        \begin{BM} % substituir cong por 'proporcional'
            V \propto 1/p
        \end{BM}

        % Processo Isotérmico
        % Variação do volume a temperatura constante
        % relação constante entre pessão e o invérso do volume
        \section*{Gráfico em um processo isotérmico}

        % inserir grafico aqui

    \end{sectionBox}

    % Lei de Charles e Gay-Lussac
    \begin{sectionBox}1{Lei de Charles e Gay-Lussac}

        \begin{BM}
            \mathrm{V} \propto \mathrm{T}
        \\  \mathrm{P} \propto \mathrm{T}
        \end{BM}
        % Processo Isobárico (diferentes experimentos com pressões fixas diferentes)
        % Variação do volume c/ pressão constante
        % temperatura varia diretamente proprocional
    \end{sectionBox}

    % Lei de Avogadro
    \begin{sectionBox}1{Lei de Avogadro}

        \begin{BM}
            V \propto n
        \end{BM}
        % Processo Isocóricas
        % volume proporiconal com a quantidade de substancias
    \end{sectionBox}

    %
    \begin{sectionBox}1{Volume molar}
        \begin{BM}
            \cong 22.41\,\unit{\frac{\cubic\deci\meter}{\mole}}
            \begin{cases}
                0\unit{\celsius}
            \\  1.013\,\unit{\bar}
            \end{cases}
        \end{BM}
    \end{sectionBox}

\end{multicols}

% Equação dos gazes ideais
\begin{sectionBox}1{Equação dos gases ideais}
    % Summário das leis anteriores
    \begin{BM}
        P\,V=\,n\,R\,T
    \end{BM}

    \begin{BM}
        R\cong 8.20574*10^{-2}\,\unit{\litre\atm\per{\K\mole}}
    \end{BM}

    % Constante dos gazes perfeitos: R
\end{sectionBox}

% Teoria Cinética dos Gáses
\begin{sectionBox}1{Teoria Cinética do Gáses}

    % Sistema isolado = Energia constante
    \begin{BM}
        V_{\text{rms}} = \sqrt{3\,R\,T/M}
    \end{BM}

    \begin{sectionBox}{}
        \begin{itemize}
            \item Um gás é composto por um conjunto de moléculas em contínuo movimento aleatório. As moléculas movem-se em linha recta só alterando o percurso quando há colisões. As colisões são perfeitamente elásticas.
            \item As moléculas do gás são pontuais, (volume ocupado desprezável).
            \item Não existem forças repulsivas nem atractivas entre as moléculas do gás.
            \item A energia cinética média das moléculas é proporcional à temperatura do gás, em Kelvin.
        \end{itemize}
    \end{sectionBox}

\end{sectionBox}


\part*{Termodinâmica}

\begin{sectionBox}{}
    \begin{enumerate}[label=Lei \arabic{enumi}:, leftmargin=3em]\setcounter{enumi}{-1}
        \item Def. Temperatura
        \item Def. Energia
        \item Def. Entropia
        \item Atribui um valor numérico à entropia
    \end{enumerate}
\end{sectionBox}

\begin{sectionBox}1{Definições}

    É necessário dividir o espaço para que se possa estudar sua termodinâmica
    \begin{itemize}[leftmargin=6em]
        \item[Sistema:] Região de interesse
        \item[Vizinhança:] Resto do universo
        \item[Universo:] União do sistema e vizinhança
    \end{itemize}

\end{sectionBox}

\begin{sectionBox}1{Estado de um sistema}
    \begin{itemize}[leftmargin=4em]
        \item[Aberto:] Troca de energia e matéria com a vizinhança
        \item[Fechado:] Troca de energia com a vizinhança
        \item[Isolado:] Não troca nada com a vizinhança
    \end{itemize}
\end{sectionBox}


\begin{questionBox}1{}

    Uma amostra de gás natural contém \(8.24\,\unit{\mole}\) de \ch{CH4}, \(0.421\,\unit{\mole}\) de \ch{C2H6} e \(0.116\unit{\mole}\) de \ch{C3H8}. Se a pressão total dos gases for \(1.37\,\unit{\atm}\), qual é a pressão parcial do propano (\ch{C3H8})?

    \begin{flalign*}
        &
            p_{i}\,\unit{\atm\of{\ch{C3H8\gas}}}
        =   p\,\unit{\atm\of{sol}}
        *   \frac{n_i\,\unit{\mole\of{\ch{C3H8\gas}}}}{n\,\unit{\mole\of{sol}}}
        % = &\\&
        =   1.37\,\unit{\atm\of{sol}}
        *   \frac{0.116}{8.24+0.421+0.116}
        \cong &\\&
        \cong
            \qty{0.018106414492423}{\atm\of{\ch{C3H8\gas}}}
        &
    \end{flalign*}

\end{questionBox}

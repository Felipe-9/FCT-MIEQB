% !TEX root = ./F_IIQ-Anotações_Testes.1.tex
\providecommand\mainfilename{"./F_IIQ-Anotações_Testes.tex"}
\providecommand \subfilename{}
\renewcommand   \subfilename{"./F_IIQ-Anotações_Testes.1.tex"}
\documentclass[\mainfilename]{subfiles}

% \tikzset{external/force remake=true} % - remake all

\begin{document}

% \graphicspath{{\subfix{./.build/figures/F_IIQ-Anotações_Testes.1}}}

\mymakesubfile{1}
[F IIQ]
{Anotações: Campo Elétrico \& Potencial} % Subfile Title
{Campo Elétrico \& Potencial} % Part Title


\begin{sectionBox}1{The Electric Field I: Discrete Charge Distributions} % S1
    
    \subsection{Coulomb's Constant}
    \begin{BM}
        k = \qty{8.987551792314}{\newton\,\metre^2/\coulomb^2}
    \end{BM}

    \subsection{Coulomb's Law}
    \begin{BM}
        \vv{F}=\frac{
            k\,q_1\,q_2
        }{
            r_{1,2}^2}\hat{r_{1,2}
        }
    \end{BM}

    \subsection{Electric Field}
    \begin{BM}
        \vv{E} 
        = \frac{\vv{F}}{q_0}
        = \frac{k\,q}{r^2}\hat{r}
    \end{BM}

    \subsection{Dipole}
    \begin{BM}
        \vv{p} = q\,\vv{L}_{(-\to+)}
        \qquad
        \vv{\tau}=\vv{p}\times\vv{E}
        \\
        U = -\vv{p}\cdot\vv{E} + U_0
    \end{BM}
    
\end{sectionBox}

\begin{sectionBox}1{The Electric Field II: Continuous Charge Distributions} % S2
    
    \subsection{Fluxo Elétrico}
    \begin{BM}
        \phi
        = \lim_{\adif{A}_i\to0}{
            \sum_i{\vv{E_i}\cdot\hat{n}\adif{A}_i}
        }
        = \int_S{\vv{E}\cdot\hat{n}\,\odif{A}}
    \end{BM}

    \subsection{Electric Constant (Permissivity of Free Space)}
    \begin{BM}
        \varepsilon_0
        = (4\,\pi\,k)^{-1}
        = \qty{8.854187812747955e-12}{\coulomb^2/\newton\,\metre^2}
    \end{BM}

    \subsection{Gauss's Law}
    \begin{BM}
        \phi_{net} 
        = \oint_S{\vv{E}\cdot\hat{n}\,\odif{A}}
        = Q_{inside}/\varepsilon_0
    \end{BM}

    \subsection{Discontinuity of \(E_n\)}
    \begin{BM}
        E_{n+} - E_{n-}
        = \sigma/\varepsilon_0
    \end{BM}

    \subsection{\(\vec{E}\) just outside a Conductor}
    \begin{BM}
        E = \sigma/\varepsilon_0
    \end{BM}
    \paragraph*{Acts like an infinite place surface}

    \section*{Electric Fields for Selected Uniform Charge Distributions}
    
    \subsection{Of a \emph{line} charge of infinite lenght}
    \begin{BM}
        \vv{E}
        = 2\,k\,\lambda\,\hat{r}/R
    \end{BM}

    \subsection{On the axis of a charged \emph{ring}}
    \begin{BM}
        \vv{E}
        = k\,Q\,z
        \,\left(
            z^2+a^2
        \right)^{-3/2}
        \,\hat{z}
    \end{BM}

    \subsection{On the axis of a charged \emph{disk}}
    \begin{BM}
        \vv{E}
        = \frac{
            \sign{(z)}
            \,\sigma
            \,\hat{z}
        }{
            2\,\varepsilon_0
        }\,\left(
            1-\left(
                1+R^2/z^2
            \right)^{-1}
        \right)
    \end{BM}

    \subsection{Of a charged \emph{infinite plane}}
    \begin{BM}
        \vv{E} = \sign{(z)}\sigma\,\hat{z}/2\,\varepsilon_0
    \end{BM}

    \subsection{Of a charged thin \emph{spherical shell}}
    \begin{BM}
        \vv{E} = 
        \begin{cases}
            k\,Q\,\hat{r}/r &\quad r>0
            \\
            0 &\quad r<0
        \end{cases}
    \end{BM}
    
\end{sectionBox}

\begin{sectionBox}1{Electric Potential} % S3
    
    \subsection{Units and Constants}
    \begin{center}
        \begin{tabular}{l c}
            
            \\\toprule
            
                \(V\text{ and }\adif{V}\) 
                & \(1\,\unit{\volt} = 1\,\unit{\joule/\coulomb}\)
                \\  Electric Field 
                & \(1\,\unit{\newton/\coulomb}=1\,\unit{\volt/\metre}\)
                \\  Electron volt
                & \(1\,\unit{\electronvolt} = \qty{1.602176634e-19}{\coulomb\,\volt} = \qty{1.602176634e-19}{\joule}\)
                \\  Dielectric Strength
                & \(\max{E} \approx\qty{3e6}{\mega\volt/\metre}\)


            \\\bottomrule
            
        \end{tabular}
    \end{center}
    
    \subsection{Potential Energy of Two Point Charges}
    \begin{BM}
        U = q_0\,V = k\,q_0\,q/r
    \end{BM}

    \section*{Potential Functions}

    \subsection{On the axis of a uniformly charged \emph{ring}}
    \begin{BM}
        V = \frac{k\,Q}{\sqrt{z^2+a^2}}
    \end{BM}

    \subsection{On the axis of a uniformly charged \emph{disk}}
    \begin{BM}
        V = 2\,\pi\,k\,\sigma\,\myvert{z}\left(
            \sqrt{1+R^2/z^2}-1
        \right)
    \end{BM}

    \subsection{For an infinite \emph{plane} of charge}
    \begin{BM}
        V = V_0-2\,\pi\,k\,\sigma\,\myvert{x}
    \end{BM}

    \subsection{For a \emph{spherical shell} of charge}
    \begin{BM}
        V 
        % = k\,Q/\max(R,r)
        = \begin{cases}
            k\,Q/r &\quad r\geq R
            \\
            k\,Q/R &\quad r\geq R
        \end{cases}
    \end{BM}

    \subsection{For an infinite \emph{line} of charge}
    \begin{BM}
        V = 2\,k\,\lambda\,\ln\frac{R_{ref}}{R}
    \end{BM}

    \section*{Electrostatic Potential Energy}
    
    \subsection{Of \emph{point} charges}
    \begin{BM}
        U = \frac{1}{2}\sum_{i=1}^n{q_i\,V_i}
    \end{BM}

    \subsection{Of a \emph{conductor} with charge \textit{Q} at potential \textit{V}}
    \begin{BM}
        U = Q\,V/2
    \end{BM}

    \subsection{Of a \emph{system of conductors}}
    \begin{BM}
        U = \frac{1}{2}\sum_{i=1}^n{Q_i\,V_i}
    \end{BM}

\end{sectionBox}


\end{document}
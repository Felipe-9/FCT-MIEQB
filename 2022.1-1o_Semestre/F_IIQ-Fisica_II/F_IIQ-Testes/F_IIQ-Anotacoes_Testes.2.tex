% !TEX root = ./F_IIQ-Anotações_Testes.2.tex
\providecommand\mainfilename{"./F_IIQ-Anotações_Testes.tex"}
\providecommand \subfilename{}
\renewcommand   \subfilename{"./F_IIQ-Anotações_Testes.2.tex"}
\documentclass[\mainfilename]{subfiles}

% \tikzset{external/force remake=true} % - remake all

\begin{document}

% \graphicspath{{\subfix{./.build/figures/F_IIQ-Anotações_Testes.2}}}
% \tikzsetexternalprefix{./.build/figures/F_IIQ-Anotações_Testes.2/}

\mymakesubfile{2}
[Física 2]
{Anotações: Teste} % Subfile Title
{Magnetismo} % Part Title

\begin{sectionBox}1{Capacitance} % S1
    
    \section*{Capacitance}

    \subsection{Definition}
    \begin{BM}
        C=Q/V
    \end{BM}

    \subsection{Capacitance on different capacitors}

    \begin{multicols}{3}
        \subsubsection{Of an isolated \emph{spherical} conductor}
        \vspace{-3ex}
        \begin{BM}
            C = 4\,\pi\,\varepsilon_0\,R
        \end{BM}
    
        \subsubsection{Of a \emph{parallel-plate} capacitor}
        \vspace{-3ex}
        \begin{BM}
            C = \varepsilon_0\,A/d
        \end{BM}
    
        \subsubsection{Of a \emph{cylindrical} capacitor}
        \vspace{-3ex}
        \begin{BM}
            C = \frac{
                2\,\pi\,\varepsilon_0\,L
            }{
                \ln(R_2/R_1)
            }
        \end{BM}
    \end{multicols}


    \subsection{Energy stored in a capacitor}
    \begin{BM}
        U
        =\frac{Q\,V}{2}
        =\frac{Q^2}{2\,C}
        =\frac{C\,V^2}{2}
    \end{BM}

    \subsection{Energy density of an electric field}
    \begin{BM}
        u_e = \varepsilon_0\,E^2/2
    \end{BM}

    \subsection{\emph{Equivalent} Capacitance}
    \begin{multicols}{2}
        \subsubsection{Parallel}
        \vspace{-3ex}
        \begin{BM}
            C_{eq} = \sum_{i=1}^{n}{C_i}
        \end{BM}

        \subsubsection{Serie}
        \vspace{-3ex}
        \begin{BM}
            C_{eq}^{-1} = \sum_{i=1}^{n}{C_i}^{-1}
        \end{BM}
    \end{multicols}

    \subsection{Electric field \emph{inside}}
    \begin{BM}
        E = E_0/\kappa
    \end{BM}

    \subsection{Effect on capacitance}
    \begin{BM}
        C = \kappa\,C_0
    \end{BM}

    \subsection{Permittivity \emph{\(\varepsilon\)}}
    \begin{BM}
        \varepsilon=\kappa\,\varepsilon_0
    \end{BM}
    
\end{sectionBox}

\begin{sectionBox}1{Electric Current and Direct-Current Circuits} % S1
    
    \section*{Electric Current}

    \subsection{Definition}
    \begin{BM}
        I = \adv{Q}{T}
    \end{BM}

    \subsection{Drift Velocity}
    \begin{BM}
        I = q\,n\,A\,v_d
    \end{BM}

    \subsection{Current Density}
    \begin{BM}
        \vv{J}=q\,n\,\vv{v_d}
    \end{BM}

    \section*{Resistance}

    \subsection{Definition}
    \begin{BM}
        R = V/I
    \end{BM}

    \subsection{Resistivity \(\rho\)}
    \begin{BM}
        R = \rho\,L/A
    \end{BM}

    \subsection{Temperature coefficient of resitivity (\(\alpha\))}
    \begin{BM}
        \alpha
        = \frac{
            \rho/\rho_0-1
        }{
            T-T_0
        }
    \end{BM}

    \subsection{Ohm's Law}
    \begin{BM}
        V = I\,R
    \end{BM}

    \subsection{Power}

    \subsubsection{Suplied by a device or segment}
    \begin{BM}
        P = I\,V
    \end{BM}

    \subsubsection{Delivered to a resistor}
    \begin{BM}
        P=I\,V=I^2\,R=V^2/R
    \end{BM}

    \subsection{Emf}
    \begin{BM}
        P = I\,\mathcal{E}
    \end{BM}

    \section*{Battery}

    \subsection{Terminal Voltage}
    \begin{BM}
        V_a-V_b=\mathcal{E}-I\,r
    \end{BM}

    \subsection{Total energy stored}
    \begin{BM}
        E_{stored} = Q\,\mathcal{E}
    \end{BM}

    \subsubsection{Equivalent Resistance}
    \begin{multicols}{2}
        \subsubsection{Series}
        \vspace{-3ex}
        \begin{BM}
            R_{eq}=\sum_{i=1}^{n}{R_i}
        \end{BM}

        \subsubsection{Parallel}
        \vspace{-3ex}
        \begin{BM}
            R_{eq}^{-1}=\sum_{i=1}^{n}{R_i^{-1}}
        \end{BM}
    \end{multicols}

    \subsection{Kirchhoff's Rules}
    \begin{enumerate}
        \item When any closed loop is traversed, the algebraic sum of the changes in potential around the loop must equal zero.
        \item At any junction (branch point) in a circuit where the current can divide, the sum of the currents into the junction must equal the sum of the currents out of the junction.
    \end{enumerate}

    \section*{Disharging the Capacitor}

    \subsection{Charge the capacitor}
    \begin{BM}
        Q(t)
        = Q_0\,\exp(-t/R\,C)
        = Q_0\,\exp(-t/\tau)
    \end{BM}

    \subsection{Current in a circuit}
    \begin{BM}
        I
        = -\odv{Q}{t}
        = \frac{V_0}{R}\exp(-t/R\,C)
        = I_0\,\exp(-t/\tau)
    \end{BM}

    \subsection{Time constant}
    \begin{BM}
        \tau=R\,C
    \end{BM}

    \section*{Charging a Capacitor}

    \subsection{Charge on the capacitor}
    \begin{BM}
        Q 
        = C\,\mathcal{E}
        \,\left(
            1-\exp(-t/R\,C)
        \right)
        = Q_f
        \,\left(
            1-\exp(-t/\tau)
        \right)
    \end{BM}

\end{sectionBox}

\begin{sectionBox}1{The Magnetic Field} % S3
    
    \subsection{Magnetic Force}
    \begin{multicols}{2}

        \subsubsection{On a moving \emph{charge}}
        \vspace{-3ex}
        \begin{BM}
            \vv{F}
            =q\,\vv{v}\times\vv{B}
        \end{BM}

        \subsubsection{On a \emph{current} element}
        \vspace{-3ex}
        \begin{BM}
            \odif{\vv{F}}
            = I\,\odif{\vv{l}}\times\vv{B}
        \end{BM}
        
    \end{multicols}

    \subsection{Unit of the magnetic field (\emph{Tesla})}
    \begin{BM}
        \unit{\tesla} = 10^4\,\unit{\gauss}
    \end{BM}

    \section*{Motion of Point Charges}

    \subsection{Newton's Second Law}
    \begin{BM}
        q\,v\,B=m\,v^2/r
    \end{BM}

    \subsection{Cyclotron}
    \begin{multicols}{2}
        \subsubsection{Period}
        \begin{BM}
            T = \frac{2\,\pi\,m}{q\,B}
        \end{BM}
        
        \subsection{Frequency}
        \begin{BM}
            f=T^{-1}=\frac{q\,b}{2\,\pi\,m}
        \end{BM}
    \end{multicols}

    \subsection{Velocity Selector}
    \begin{BM}
        v=E/B
    \end{BM}
    A velocity selector consists of crossed electric and magnetic fields so that the electric and magnetic forces balance for a particle moving with speed \textit{v}.

    \section*{Current Loops}

    \subsection{Magnetic dipole moment}
    \begin{BM}
        \vv{\mu}=N\,I\,A\,\hat{n}
    \end{BM}

    \subsection{Torque}
    \begin{BM}
        \vv{\tau}=\vv{\mu}\times\vv{B}
    \end{BM}

    \subsection{Potential Energy of a magnetic dipole}
    \begin{BM}
        U = -\vv{\mu}\cdot\vv{B}
    \end{BM}

    \subsection{Net force}
    The net force on a current loop in a uniform magnetic field is zero.

    % \section*{The Hall Effect}
    % When a conducting strip carrying a current is placed in a magnetic field, the magnetic force on the charge carriers causes a separation of charge called the Hall effect.

    % \subsection{Hall voltage}
    % \begin{BM}
    %     V_H=E_H\,w=v_d\,B\,w=\frac{\myvert{I}}{nte}B
    % \end{BM}

    \subsection{Conventional von Klitzing constant (definition of ohm)}
    \begin{BM}
        R_{k-90}=25\,812.807\,6\,\unit{\ohm} \text{ (exact)}
    \end{BM}

\end{sectionBox}

\begin{sectionBox}1{Sources of the magnetic field} % S4
    
    \subsubsection{Magnetic Constant (\emph{\(\mu_0\)})}
    \begin{BM}
        \mu_0
        =4\,\pi\,10^{-7\,\unit{\tesla.\metre/\ampere}}
        =4\,\pi\,10^{-7\,\unit{\newton/\ampere^2}}
    \end{BM}

    \subsection{Magnetic Field (\(\vec{B}\))}
    
    \begin{multicols}{2}
        \subsubsection{Due to a moving \emph{point} charge}
        \begin{BM}
            \vv{B}
            = \frac{\mu_0}{4\,\pi}
            \,\frac{q\,\vv{v}\times\hat{r}}{r^2}
        \end{BM}

        \subsubsection{Due to a \emph{current} element (Biot-Savart law)}
        \begin{BM}
            \odif{\vv{B}}
            = \frac{\mu_0}{4\,\pi}
            \,\frac{I\,\odif{\vv{l}}\times\hat{r}}{r^2}
        \end{BM}

        \subsubsection{On the axis of a current \emph{loop}}
        \begin{BM}
            B_z
            = \frac{\mu_0}{4\,\pi}
            \,\frac{2\,\pi\,R^2\,I}{(z^2+R^2)^{3/2}}
        \end{BM}

        \subsubsection{Inside a long \emph{solenoid} (far from the edges)}
        \begin{BM}
            B_Z = \mu_0\,n\,I
        \end{BM}
        Where \textit{n} is the number of turns per unit length

        \subsubsection{Inside the tubs of a tight wound \emph{toroid}}
        \begin{BM}
            B
            = \frac{\mu_0}{2\,\pi}
            \,\frac{N\,I}{r}
        \end{BM}

        \subsubsection{Due to a \emph{straight wire} segment}
        \begin{BM}
            B
            = \frac{\mu_0}{4\,\pi}
            \,\frac{I}{R}\adif{\sin(\theta)}
        \end{BM}
        where \textit{R} is the perpendicular distance to the wire and \(\theta_1\) and \(\theta_2\) are the angles subtended at the field point by the ends of the wire.

    \end{multicols}

    \subsubsection{Due to a \emph{long straight wire}}
    \begin{BM}
        B
        = \frac{\mu_0}{4\,\pi}
        \,\frac{I}{R}\left(
            \sin\left(\frac{\pi}{2}\right)
            -\sin\left(-\frac{\pi}{2}\right)
        \right)
        = \frac{\mu_0}{4\,pi}
        \,\frac{I}{R}2
    \end{BM}

    \subsection{\emph{Gauss's Law} for Magnetism}
    \begin{BM}
        \phi_{m,net}
        = \oint_{S}{
            \vv{B}\cdot\hat{n}\,\odif{A}
        }
        = \oint_{S}{
            B_n\,\odif{A}
        }
        = 0
    \end{BM}

    % \subsection{Ampère's Law}
    % \begin{BM}
    %     \oint_{C}{
    %         \vv{B}\cdot\odif{\vv{l}}
    %     }
    %     = \oint_{C}{
    %         B_t\,\odif{l}
    %     }
    %     = \mu_0\,I_C
    % \end{BM}
    % \begin{itemize}
    %     \item Where \textit{C} is any closed curve
    %     \item Cant be used where \textit{I} varies
    % \end{itemize}

    \section*{\(\vec{B}\) in Magnetic Materials}
    \begin{BM}
        \vv{B}=\vv{B}_{app}+\mu_0\,\vv{M}
    \end{BM}

    \subsection{Magnetic susceptibility (\(\chi_m\))}
    \begin{BM}
        \vv{M}=\chi_m\,\frac{\vv{B}_{app}}{\mu_0}
    \end{BM}

    \subsection{Relative Permeability}
    \begin{BM}
        \vv{B}=K_m\,\vv{B}_{app}
        \quad
        \text{Where }K_m=1+\chi_m
    \end{BM}

    \section*{Atomic Magnetic Moments}
    \begin{BM}
        \vv{\mu}=\frac{q}{2\,m}\,\vv{L}
    \end{BM}
    Where \(\vv{L}\) is the angular momentum of the particle

    \subsection{Bohr magneton}
    \begin{BM}
        \mu_B
        = \frac{e\hslash}{2\,m_e}
        = 9.27*10^{-24}
        \,\unit{\ampere.\metre^2}
        = \\
        = 9.27*10^{-24}
        \,\unit{\joule/\tesla}
        = 5.79*10^{-5}
        \,\unit{e\volt/\tesla}
    \end{BM}

    \subsection{Due to orbital motion of an eletron}
    \begin{BM}
        \vv{\mu}_l=-\mu_B\frac{\vv{L}}{\hslash}
    \end{BM}
    \subsection{Due to electron spin}
    \begin{BM}
        \vv{\mu}_s=-2\,\mu_B\frac{\vv{S}}{\hslash}
    \end{BM}

    \section*{Paramagnetism}

    \subsection{Curie's Law (weak fields)}
    \begin{BM}
        M=\frac{\mu\,B_{app}}{3\,k\,T}M_s
    \end{BM}
    
\end{sectionBox}

\begin{sectionBox}1{Magnetic Induction} % S5
    
    \section*{Magnetic Flux \(\phi_m\)}
    \subsection{Units}
    \begin{BM}
        1\,\unit{\weber}=1\,\unit{\tesla.\metre^2}
    \end{BM}

    \subsection{General Definition}
    \begin{BM}
        \phi_m
        =\int_{S}{\vv{B}\cdot\hat{n}\,\odif{A}}
    \end{BM}

    \subsubsection{Uniform field, flat surface bounded by \emph{coil} of \textit{N} turns}
    \begin{BM}
        \phi_m=N\,B\,A\,\cos{\theta}
    \end{BM}
    \begin{itemize}
        \item \textit{A} Flat surface bounded by a single turn
    \end{itemize}

    \begin{multicols}{2}
        
        \subsection{Due to a current in \emph{a circuit}}
        \begin{BM}
            \phi_m=L\,I
        \end{BM}
        
        \subsection{Due to a current in \emph{two circuit}}
        \begin{BM}
            \phi_{m,1}=L_1\,I_1+M\,I_2
            \\
            \phi_{m,2}=L_2\,I_2+M\,I_1
        \end{BM}

    \end{multicols}

    \section*{EMF}

    \subsection{Faraday's Law}
    Includes both induction and motional emf
    \begin{BM}
        \mathcal{E}=-\odv{\phi_m}{t}
    \end{BM}

    \begin{multicols}{2}
        
        \subsubsection{Induction}
        Time-varying magnetic field, \textit{C} stationary
        \begin{BM}
            \mathcal{E}
            =\oint_{C}{\vv{E}\cdot\odif{\vv{l}}}
        \end{BM}
    
        \subsubsection{Rod moving perpendicular to both its length and \(\vec{B}\)}
        \begin{BM}
            \myvert{\mathcal{E}}
            = v\,B\,l
        \end{BM}
        
    \end{multicols}

    \subsubsection{Self inducted (back emf)}
    \begin{BM}
        \mathcal{E}
        = -L\,\odv{I}{t}
    \end{BM}

    \subsection{Lenz's Law}
    The induced emf and induced current are in such a direction as to oppose, or tend to oppose, the change that produces them.

    \section*{Inductance}

    \subsection{Units and constants}
    \begin{BM}
        1\,\unit{\henry}
        = 1\,\unit{\weber/\ampere}
        = 1\,\unit{\tesla.\metre^2/\ampere}
        \\
        \mu_0=4\,\pi*10^{-7}\,\unit{\henry/\metre}
    \end{BM}

    \subsection{Formulas}

    \begin{multicols}{2}
        
        \subsubsection{Self Inductance}
        \begin{BM}
            L=\phi_m/I
        \end{BM}
        
        \subsubsection{Self Inductance of a solenoid}
        \begin{BM}
            L=\mu_0\,n^2\,A\,l
        \end{BM}
        
        \subsubsection{Mutual inductance}
        \begin{BM}
            M=\phi_{m,2,1}/I_1=\phi_{m,1,2}/I_2
        \end{BM}

    \end{multicols}

    \section*{Magnetic Energy}
    \subsection{Energy stored in an inductor}
    \begin{BM}
        U_m = L\,I^2/2
    \end{BM}

    \subsection{Energy density in a magnetic field}
    \begin{BM}
        u_m=\frac{B^2}{2\,\mu_0}
    \end{BM}

    \section*{RL Circuits}

    \subsection{Potential difference across an inductor}
    \begin{BM}
        \adif{V}
        = \mathcal{E}
        -I\,r
        =-L\,\odv{I}{t}
        -I\,r
    \end{BM}
    
\end{sectionBox}

\end{document}
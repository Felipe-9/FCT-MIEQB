% !TEX root = ./F_IIQ-Testes_Resoluções.1.tex
\providecommand\mainfilename{"./F_IIQ-Testes_Resoluções.tex"}
\providecommand \subfilename{}
\renewcommand   \subfilename{"./F_IIQ-Testes_Resoluções.1.tex"}
\documentclass[\mainfilename]{subfiles}

% \tikzset{external/force remake=true} % - remake all

\begin{document}

% \graphicspath{{\subfix{./.build/figures/F_IIQ-Testes_Resoluções.1}}}

\mymakesubfile{1}
[F IIQ]
{Resolução Teste} % Subfile Title
{Resolução Teste} % Part Title

\begin{questionBox}1{ % Q1
    Cargas, onde pode por \(q=1\,\unit{\coulomb}\) para ter campo nulo
} % Q1
    \answer{A) zona 3}
\end{questionBox}

\begin{questionBox}1{ % Q2
    Esfera condutora neutra e varão negativo
} % Q2
    \answer{d) eletrons se afastam do varão}
\end{questionBox}

\begin{questionBox}1{ % Q3
    Carga pontual, superficie gausiana é um cubo, fluxo por uma face
} % Q3
    \begin{flalign*}
        &
            \phi/6 
            = (q/\varepsilon_0)/6
        &
    \end{flalign*}

    \answer{b)}
\end{questionBox}

\begin{questionBox}1{ % Q4
    Tendencia de uma carga negativa
} % Q4
    \answer{B) pot baixo\to{}elevado}
\end{questionBox}

\begin{questionBox}1{ % Q5
    2 Esferas, 2Q, 0, r, 3r, depois de ligadas qual a carga
} % Q5
    \begin{flalign*}
        &
            Q_1
            = E_1\,R_1/k
            = (k\,Q_2/R_2)\,R_1/k
            = Q_2\,(R_1/R_2)
            = (2Q-Q_1)\,(R_1/3\,R_1)
            = &\\&
            = (2Q-Q_1)/3
            \implies &\\&
            \implies
            Q_1 = 2Q/4 = Q/2
        &
    \end{flalign*}

    \answer{E)}
\end{questionBox}

\begin{questionBox}1{ % Q6
    Afirm sobre linhas de campo
} % Q6
    \answer{B)}
\end{questionBox}

\begin{questionBox}1{ % Q7
    Afirm falsa dipolo
} % Q7
    \answer{B) deslocase em um campo}
\end{questionBox}

\begin{questionBox}1{ % Q8
    Condutor com carga central
} % Q8
    \answer{C)}
\end{questionBox}

\begin{questionBox}1{ % Q9
    grafico
} % Q9
    \answer{A) potencial de esfera de raio R}
\end{questionBox}

\begin{questionBox}1{ % Q10
    Afirm verd
} % Q10
    \answer{E) sup condut é equipot}
\end{questionBox}

\begin{questionBox}1{ % Q11
    2 Grandes placas paralelas, \(L=12\,\unit{\centi\metre}\), \(Q_1=-Q_2\), \(F_e =6.5*10^{-16}\,\unit{\newton}\) atua sobre o eletron
} % Q11
    \begin{questionBox}2{ % Q11.1
        Campo no eletron
    } % Q11.1
        \begin{flalign*}
            &
                E 
                = F/q 
                \cong \frac{6.5*10^{-16}\,\unit{\newton}}{\num{1.602176634e-19}\,\unit{\coulomb}}
                \cong
                \qty{4056.980898399495695}{\newton/\coulomb}
            &
        \end{flalign*}
    \end{questionBox}

    \begin{questionBox}2{ % Q11.2
        \(\adif{V}\) entre as placas
    } % Q11.2
        \begin{flalign*}
            &
                \adif{V}
                = \adif{V}_2-\adif{V}_1
                = 2\,\pi\,k\,\sigma_2\,x_2-2\,\pi\,k\,\sigma_1\,x_1
                = 2\,\pi\,k\,(\sigma_2\,x_2-\sigma_1\,x_1)
                = &\\&
                = 2\,\pi\,k\,(\sigma\,x_2-\sigma\,x_1)
                = 2\,\pi\,k\,\sigma*(12)
                = 24\,\pi\,k\,\sigma
            &
        \end{flalign*}
        % \begin{flalign*}
        %     &
        %         \adif{V}
        %         = \lvert{\adif{V_2}-\adif{V_1}}\rvert
        %         = \lvert{
        %             (V_0-2\,\pi\,k\,\sigma_2\,x_2)
        %             -(V_0-2\,\pi\,k\,\sigma_1\,x_1)
        %         }\rvert
        %         = \lvert{
        %             2\,\pi\,k\left(
        %                 \sigma_1-\sigma_2
        %             \right)
        %         }\rvert
        %         = &\\&
        %         = \lvert{
        %             2\,\pi\,k\left(
        %                 2\,\sigma_1
        %             \right)
        %         }\rvert
        %         = 4\,\pi\,k\,\lvert{
        %             \sigma_1
        %         }\rvert
        %         = 4\,\pi\,k\,\lvert{
        %             2\,\varepsilon_0\,E_1
        %         }\rvert
        %         = 2\lvert{
        %             \num{4056.980898399495695}/2
        %         }\rvert
        %         = \num{4056.980898399495695}
        %     &
        % \end{flalign*}
    \end{questionBox}

    \begin{questionBox}2{ % Q11.3
        Trabalho do campo no eletron largardo na placa negativa
    } % Q11.3
        \begin{flalign*}
            &
                w 
                = \int_{0}^{0.12}{
                    \vv{E}\cdot\hat{i}\odif{x}
                }
                = \int_{0}^{0.12}{
                    -E\odif{x}
                }
                = -E*0.12
                \cong -\num{4056.980898399495695}*0.12
                \cong -\num{486.83770780793952}
            &
        \end{flalign*}
    \end{questionBox}

    \begin{questionBox}2{ % Q11.4
        Velocidade do eletron ao meio
    } % Q11.4
        \begin{flalign*}
            &
                U 
                = U_{k} + U_{E}
                = m\,v^2/2 + \num{486.83770780793952}/2
                = \num{486.83770780793952}
                \implies &\\&
                \implies
                v 
                \cong \sqrt{\num{486.83770780793952}/m_e}
                \cong \sqrt{\num{486.83770780793952}/{9*10^{-31}}}
                \cong \qty{2.325791879023715e16}{\metre/\second}
            &
        \end{flalign*}
        % jcs
    \end{questionBox}
\end{questionBox}

\end{document}
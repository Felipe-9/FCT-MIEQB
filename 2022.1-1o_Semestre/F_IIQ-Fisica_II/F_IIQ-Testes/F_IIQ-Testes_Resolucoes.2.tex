% !TEX root = ./F_IIQ-Testes_Resoluções.2.tex
\providecommand\mainfilename{"./F_IIQ-Testes_Resoluções.tex"}
\providecommand \subfilename{}
\renewcommand   \subfilename{"./F_IIQ-Testes_Resoluções.2.tex"}
\documentclass[\mainfilename]{subfiles}

% \tikzset{external/force remake=true} % - remake all

\begin{document}

% \graphicspath{{\subfix{./.build/figures/F_IIQ-Testes_Resoluções.2}}}
% \tikzsetexternalprefix{./.build/figures/F_IIQ-Testes_Resoluções.2/}

\mymakesubfile{2}
[Física 2]
{2 Teste Resolução} % Subfile Title
{Teste Resolução} % Part Title

\begin{questionBox}1{ % Q1
    Caracteristica Voltimetro e Amperimetro ideais
} % Q1
    \answer{D) Amperimetro nula, Volt infinita}
\end{questionBox}

\begin{questionBox}1{ % Q2
    Circuito, Medição amperimetro
} % Q2
    \begin{flalign*}
        &
            100.0
            - 25.0\,I 
            - 75.0\,I
            - 25.0\,I
            - 25.0\,I
            - 15.0\,I
            = 0
            \implies &\\&
            \implies
            I = \frac{100.0}{
                75.0+25.0*3+15.0
            }
            \cong \qty{0.606060606060606}{\ampere}
        &
    \end{flalign*}
    \answer{B}
\end{questionBox}

\begin{questionBox}1{ % Q3
    Partir imã em dois
} % Q3
    \answer{C}
\end{questionBox}

\begin{questionBox}1{ % Q4
    Campo em agua
} % Q4
    \answer{C) 19.2\,kC}
\end{questionBox}

\begin{questionBox}1{ % Q5
    O que acontece quando fecha o circuito
} % Q5
    \answer{D) A carga almenta até \(C/\varepsilon\)}
\end{questionBox}

\begin{questionBox}1{ % Q6
    Esquema campo magnético imã
} % Q6
    \answer{C)}
\end{questionBox}

\begin{questionBox}1{ % Q7
    Particula movendo num campo, quanto vale \(\vec{B}\cdot\vec{F}\)
} % Q7
    \begin{flalign*}
        &
            \vv{B}\cdot\vv{F}
            = \vv{B}\cdot\left(
                q\,\vv{v}\times\vv{B}
            \right)
            = \vv{B}\cdot\left(
                q\,\vv{v}\times\vv{B}
            \right)
            = 0: \vv{B}\perp\vv{v}\times\vv{B}
        &
    \end{flalign*}
    \answer{D)}
\end{questionBox}

\begin{questionBox}1{ % Q8
    Força de um feixe de protons num campo
} % Q8
    \begin{itemize}
        \item \(v = 2.0*10^{5}\,\unit{\metre/\second}\)
        \item \(B_z = 5.0\,\unit{\tesla}\)
        \item \(\vv{v}\angle\vv{B}=30^\circ\)
    \end{itemize}

    \begin{flalign*}
        &
            \vv{F}
            = q\,\vv{v}\times\vv{B}
            = -1.6*10^{-19}\,2.0*10^{5}\,5.0\,\sin{30^\circ}\,\hat{\jmath}\,\unit{\newton}
            \cong
            -8*10^{-14}\,\hat{\jmath}\,\unit{\newton}
        &
    \end{flalign*}

    \answer{A}
\end{questionBox}

\begin{questionBox}1{ % Q9
    Qualidade diferente de dois condensadores de diff separação
} % Q9
    \answer{D) Energia armazenada}
\end{questionBox}

\begin{questionBox}1{ % Q10
    Quando força mag altera a velocidade de uma partícula
} % Q10
    \answer{E) Não pode}
\end{questionBox}

\begin{questionBox}1{ % Q11
    Condensador
} % Q11

    \begin{itemize}
        \begin{multicols}{2}
            \item Placas circulares de cobre
            \item Diâmetro: \(D = 3.0\,\unit{\centi\metre}\)
            \item Distância: \(l = 1.2\,\unit{\milli\metre}\)
            \item Diff de pot aplicada: \(V_0=85\,\unit{\volt}\)
            \item Duração da aplicação: \(t=10\,\unit{\second}\)
        \end{multicols}
        \item Material dielétrico inserido: 
        \begin{itemize}
            % \begin{multicols}{2}
                \item Constante dielétrica: \(\kappa=3.5\)
                \item Espessura: \(l_d = 1.2\,\unit{\milli\metre}\)
                \item Rigiez dielétrica \(\max{V}=10\,\unit{\mega\volt/\metre}\)
            % \end{multicols}
        \end{itemize}
    \end{itemize}

    \begin{questionBox}2{ % Q11.1
        A capacidade do condensador antes e depois de ser introduzido o dielétrico
    } % Q11.1
        \begin{flalign*}
            &
                C_0
                = \varepsilon_0\,A/l
                = \varepsilon_0\,\pi\,(D/2)^2/l
                \cong 8.85*10^{-12}\,\pi\,(3.0\E-2/2)^2/1.2\E-3
                \cong &\\&
                \cong
                \num{52.130803095505632e-13}
            &\\[3ex]&
                C
                = \left(
                    (\varepsilon\,A/l_d)^{-1}
                    +(\varepsilon_0\,A/(l-l_d))^{-1}
                \right)^{-1}
                = \varepsilon\,A/l
                = \kappa\,\varepsilon_0\,A/l
                = \kappa\,\varepsilon_0\,\pi\,(D/2)^2/l
                = &\\&
                \cong 3.5
                *8.85*10^{-12}
                *\pi
                *(3.0\E-2/2)^2
                /1.2\E-3
                \cong
                \num{182.45781083426971e-13}
            &
        \end{flalign*}

        % \begin{BM}
        %     \varepsilon_0=8.85*10^{-12}
        % \end{BM}
    \end{questionBox}

    \begin{questionBox}2{ % Q11.2
        A carga acumulada em cada uma das placads do condensador antes e depois de ser introduzido o dielétrico.
    } % Q11.2
        \begin{flalign*}
            &
                Q_0=Q=C_0\,V_0
                \cong \num{52.130803095505632e-13}*85
                \cong \qty{4431.11826311797855e-13}{\coulomb}
            &
        \end{flalign*}
    \end{questionBox}

    \begin{questionBox}2{ % Q11.3
        A densidade superficial de carga introduzida no dielétrico
    } % Q11.3
        \begin{flalign*}
            &
                \sigma
                = 2\,E\,\kappa\,\varepsilon_0
                = 2\,\frac{Q}{2\,\pi\,(D/2)^2\,\varepsilon_0}\,\kappa\,\varepsilon_0
                = \frac{Q}{\pi\,(D/2)^2}\,\kappa
                \cong &\\&
                \cong 
                \frac{
                    \num{4431.11826311797855e-13}
                }{
                    \pi\,(3.0\E-2/2)^2
                }\,3.5
                % \cong &\\&
                \cong 
                \qty{2194.062500000000158e-8}{\coulomb/\metre^2}
            &
        \end{flalign*}
    \end{questionBox}

    \begin{questionBox}2{ % Q11.4
        A diferença de potencial máxima que pode ser aplicadada às placas do condensador sem que haja ruptura dielétrica
    } % Q11.4
        \begin{flalign*}
            &
                \max{V}
                = V:
                V_i=1.12*10^{-3}*10\,\unit{\mega\volt}
                =1.12*10^{-2}\,\unit{\mega\volt}
            &
        \end{flalign*}
    \end{questionBox}
\end{questionBox}

\end{document}
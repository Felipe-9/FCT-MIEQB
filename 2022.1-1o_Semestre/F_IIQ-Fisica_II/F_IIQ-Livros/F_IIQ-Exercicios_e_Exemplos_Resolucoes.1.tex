% !TEX root = ./F_IIQ-Exercicios_e_Exemplos_Resoluções.1.tex
\providecommand\mainfilename{"./F_IIQ-Exercicios_e_Exemplos_Resoluções.tex"}
\providecommand \subfilename{}
\renewcommand   \subfilename{"./F_IIQ-Exercicios_e_Exemplos_Resoluções.1.tex"}
\documentclass[\mainfilename]{subfiles}

% \tikzset{external/force remake=true} % - remake all

\begin{document}

% \graphicspath{{\subfix{./.build/figures/F_IIQ-Exercicios_e_Exemplos_Resoluções.1}}}

\mymakesubfile{1}
[F IIQ]
{The Electric Field I: Discrete Charge Distributions} % Subfile Title
{The Electric Field I: Discrete Charge Distributions} % Part Title

\part*{Exemplos}
\begin{exampleBox}1{ % E1
    } % E1
    The electrons have a total charge given by the number of electrons in the penny, Ne , multiplied by the charge of an electron, -e. The number of electrons in a copper atom is 29 (the atomic number of copper). So, the total charge of the electrons is 29 electrons multiplied by the number of copper atoms \(N_{at}\) in a penny. To find \(N_{at}\) , we use the fact that one mole of any substance has Avogadro's number (\(N_{A} = 6.02 * 10^{23}\)) of particles (molecules, atoms, or ions), and the number of grams in one mole of copper is the molar mass M, which is 63.5\,\unit{\gram/\mole} for copper.
    
    \begin{flalign*}
        &
            Q 
            = (Z\,N_{at})(-e)
            = 29
            *\left(
                \frac{3.10*6.02*10^{23}}{63.5}
            \right)
            *(\num{1.602176634e-19})
            \cong 
            -1.37*10^5
        &
    \end{flalign*}
    
\end{exampleBox}

\setcounter{example}{10}

\begin{exampleBox}1{ % E11
    } % E11
    An electron is projected into a uniform electric field \(\vec{E} = (1000\,\unit{\newton/\coulomb})\hat{i}\) with an initial velocity \(\vec{v} = (2.00*10^{6}\,\unit{\metre/\second})\hat{i}\) in the direction of the field (Figure 21-26). How far does the electron travel before it is brought momentarily to rest?
    
    \begin{flalign*}
        &
            v_x^2 
            = v_{x,0}^2 + 2\,a_x\,\adif{x}
            = v_{x,0}^2 + 2\,\frac{F_x}{m}\,\adif{x}
            = v_{x,0}^2 + 2\,\frac{-e\,E_x}{m}\,\adif{x}
            \implies
            \adif{x}
            = \frac{
                (v_x^2 - v_{x,0}^2)\,m
            }{
                -e\,E_x\,2
            }
        &
    \end{flalign*}
    
\end{exampleBox}

\end{document}
% !TEX root = ./IPEIO-Testes_Resolução.2023.1.tex
\providecommand\mainfilename{"./IPEIO-Testes_Resolução.tex"}
\providecommand \subfilename{}
\renewcommand   \subfilename{"./IPEIO-Testes_Resolução.2023.1.tex"}
\documentclass[\mainfilename]{subfiles}

% \tikzset{external/force remake=true} % - remake all

\begin{document}

% \graphicspath{{\subfix{./.build/figures/IPEIO-Testes_Resolução.2023.1}}}

\mymakesubfile{1}
[IPEIO]
{Teste 2023} % Subfile Title
{Teste 2023} % Part Title

\begin{questionBox}1{ % Q1
} % Q1
    \begin{itemize}
        \item \(P(D) = 0.05\)
        \item \(P(+\vert{}\overline{D})=0.08\)
        \item \(P(-\vert{}D)=0.12\)
    \end{itemize}

    \begin{questionBox}2{ % Q1.1
        Prob de teste negativo
    } % Q1.1
        \begin{flalign*}
            &
                P(-)
                = P(-\vert{}\overline{D})\,P(\overline{D})
                + P(-\vert{}D)\,P(D)
                = (1-0.08)(1-0.05)
                + 0.12*0.05
                = &\\&
                = 0.88
            &
        \end{flalign*}
        \answer{A}
    \end{questionBox}

    \begin{questionBox}2{ % Q1.2
        Prob de !D sabendo q foi negativo
    } % Q1.2
        \begin{flalign*}
            &
                P(\overline{D}\vert{-})
                = \frac{P(\overline{D}\cap-)}{P(-)}
                = \frac{P(-\vert\overline{D})P(\overline{D})}{P(-)}
                = \frac{(1-0.08)(1-0.05)}{0.88}
                = \frac{(0.92)(0.95)}{0.88}
                \cong &\\&
                \cong
                \num{0.993181818181818}
            &
        \end{flalign*}
        \answer{B}
    \end{questionBox}
\end{questionBox}

\begin{questionBox}1{ % Q2
    v.a X, \# gripe A sem entrada/h, 6/h
} % Q2
    \begin{questionBox}2{ % Q2.1
        prob 1/h
    } % Q2.1
        \begin{flalign*}
            &
                P(X=1)
                = \exp(-6)\,6^1/1!
                \cong
                \num{0.014872513059998}
            &
        \end{flalign*}
        \answer{B}
    \end{questionBox}

    \begin{questionBox}2{ % Q2.2
        Prob de 2 em 15m
    } % Q2.2
        \begin{flalign*}
            &
                P(X=2)
                = \exp(-15*6/60)(15*6/60)^2/2!
                = \num{0.251021430166984}
            &
        \end{flalign*}
        \answer{D}
    \end{questionBox}

    \begin{questionBox}2{ % Q2.3
        E(X2)
    } % Q2.3
        \begin{flalign*}
            &
                E(X^2)
                = V(X)+E^2(X)
                = 6+6^2
                = 42
            &
        \end{flalign*}
        \answer{D}
    \end{questionBox}

    \begin{questionBox}2{ % Q2.4
        T de chegadas consecultivas
    } % Q2.4
        \begin{flalign*}
            &
                1/6
            &
        \end{flalign*}
        \answer{B}
    \end{questionBox}
\end{questionBox}

\begin{questionBox}1{ % Q3
    Considere a v.a. X com f densidade
} % Q3
    \begin{BM}
        f(x) 
        = \begin{cases}
            (x+1)/2 \quad& -1<x<1
            \\
            0\quad& c.c.
        \end{cases}
    \end{BM}

    \begin{questionBox}2{ % Q3.1
        % \(P(X\geq0)\)
    } % Q3.1
        \begin{flalign*}
            &
                P(X\geq0)
                = 1-P(X\leq0)
                = 1-f(0)
                = 1-1/2
                = 1/2
            &
        \end{flalign*}
        \begin{flalign*}
            &
                P(X\geq0)
                =1-P(X\leq0)
                = 1-\int_{-1}^{0}{f(x)\,\odif{x}}
                = 1-\int_{-1}^{0}{(x+1)\odif{x}/2}
                = &\\&
                = 1-2^{-1}\int_{-1}^{0}{(x+1)\odif{(x+1)}}
                % = &\\&
                = 1-2^{-1}\adif{((x+1)^2/2)}\big\vert_{-1}^{0}
                = &\\&
                = 1-4^{-1}((0+1)^2-(-1+1)^2)
                % = &\\&
                = 1-4^{-1}(1)
                = 3/4
            &
        \end{flalign*}
        \answer{A}
    \end{questionBox}

    \begin{questionBox}2{ % Q3.2
        c: p(X<c)=1/4
    } % Q3.2
        \begin{flalign*}
            &
                c:P(X<c)=1/4
                \implies &\\&
                \implies
                P(X<c) 
                = \int_{0}^{c}{(x+1)\odif{x+1}/2}
                = 4^{-1}((c+1)^2-(0+1)^2)
                = 4^{-1}(c^2+2\,c)
                = c/4(c+2)
                = 1/4
                \implies &\\&
                \implies
                c=1\lor c=-3.5/2
            &
        \end{flalign*}
        \begin{flalign*}
            &
                c:P(X<c)=1/4
                \implies &\\&
                \implies
                P(X<c) 
                = f(c)
                = (c+1)/2
                = 1/4 
                \implies &\\&
                \implies
                c
                = \frac{2}{4}-1
                = -1/2
            &
        \end{flalign*}
        \answer{E}
    \end{questionBox}

    \begin{questionBox}2{ % Q3.3
    } % Q3.3
        \begin{flalign*}
            &
                E(2\,X+1/3)
                = 2\,E(X)+1/3
                = 1/3+2\,\int_{-1}^{1}{x\,f(x)\,\odif{x}}
                = &\\&
                = 1/3+2\,\int_{-1}^{1}{x\,(x+1)\,\odif{x}}
                = 1/3+2\,\int_{-1}^{1}{(x^2+x)\,\odif{x}}
                = &\\&
                = 1/3+2\,\left(
                    \adif{(x^3/3)}\big\vert_{-1}^{1}
                    +\adif{(x^2/2)}\big\vert_{-1}^{1}
                \right)
                = &\\&
                = 1/3+2\,\left(
                    (1^3-(-1)^3)/3
                    +((1^2-(-1)^2)/2)
                \right)
                = 1/3+2*2/3
                = 5/3
            &
        \end{flalign*}
        \answer{C}
    \end{questionBox}
\end{questionBox}

\begin{questionBox}1{ % Q4
    v.a X, peso, dist \(N(35,1.25^2)\)
} % Q4
    \begin{questionBox}2{ % Q4.1
        Prob de pesar mais de 34g
    } % Q4.1
        \begin{flalign*}
            &
                P(X>34)
                = 1-P(X\leq 34)
                = 1-P(Z\leq (34-35)/1.25)
                = 1-P(Z\leq -0.8)
                = 1-(1-P(Z\leq 0.8))
                = P(Z\leq 0.8)
                = 0.7881
            &
        \end{flalign*}
        \answer{A}
    \end{questionBox}

    \begin{questionBox}2{ % Q4.2
    } % Q4.2
        \begin{flalign*}
            &
                x:P(X\leq x) = 0.8770
                \implies &\\&
                \implies
                P(Z\leq (x-\mu)/\sigma) = 0.8770
                \implies &\\&
                \implies
                x 
                = 1.16*1.25+35
                = 36.45
            &
        \end{flalign*}
        \answer{B}
    \end{questionBox}

    \begin{questionBox}2{ % Q4.3
        Dist de Y
    } % Q4.3
        \begin{BM}
            X_1,X_2,X_3
            \quad
            X\sim(N(35,1.25^2))
            \\
            Y = 2\,X_1+X_2-2\,X_3
        \end{BM}

        \begin{flalign*}
            &
                Y\sim
                \normal\left(
                    \sum_{i=1}^{3}{a_i\,\mu_i},
                    \sum_{i=1}^{3}{a_i^2\,\sigma_i^2}
                \right)
                = \normal\left(
                    (2+1-2)\,35,
                    (2^2+1+(-2^2))\,1.25^2
                \right)
                = \normal\left(
                    35,
                    14.0625
                \right)
            &
        \end{flalign*}
        \answer{D}
    \end{questionBox}

    \begin{questionBox}2{ % Q4.4
        Dist do Numero de h infectados
    } % Q4.4
        \begin{itemize}
            \begin{multicols}{2}
                \item \(n=30\)
                \item \(x=10\)
                \item sem rep
                \item \(a=19\)
            \end{multicols}
        \end{itemize}

        \begin{flalign*}
            &
                X\sim
                \binomial(19,10/30)
                =\binomial(19,0.333)
            &
        \end{flalign*}
        \answer{B}
    \end{questionBox}
\end{questionBox}

\begin{questionBox}1{ % Q5
    verd ou falsa
} % Q5
    \begin{itemize}
        \item \((X_1,\dots,X_n)\)
        \item T* é um est centrado em \(\theta\)
        \item T não é um est centrado em \(\theta\)
        \item \(E(T)=\theta/2\)
    \end{itemize}

    \begin{questionBox}2{ % Q5.1
        Se \(V(T)=V(T*)\) então \textit{T} é melhor q \textit{T*}
    } % Q5.1
        \begin{flalign*}
            &
                V(T)
                = E(T^2)-E^2(T)
                = E(T^2)-(\theta/2)^2
            &
        \end{flalign*}
    \end{questionBox}
\end{questionBox}

\begin{questionBox}1{ % Q6
    tempo em min é v.a. X normal
} % Q6
    \begin{itemize}
        \item \(n=25\)
        \item \(X\sim\normal(\mu,\sigma^2)\)
        \item \(\sum_{i=1}^{25}{x_i}=1350\)
        \item \(\sum_{i=1}^{25}{(x_i-\overline{x})^2}=864\)
    \end{itemize}

    \begin{questionBox}2{ % Q6.1
        Indique um pont p sigma2
    } % Q6.1
        \begin{flalign*}
            &
                864/24
                =36
            &
        \end{flalign*}
        \answer{D}
    \end{questionBox}

    \begin{questionBox}2{ % Q6.2
        IC95
    } % Q6.2
        \begin{flalign*}
            &
                IC_{.95}
                =\myrange*{
                    \frac{1350}{25}-1.960\frac{6}{\sqrt{25}},
                    \frac{1350}{25}+1.960\frac{6}{\sqrt{25}}
                }
                =\myrange*{
                    51.648,
                    56.352
                }
            &
        \end{flalign*}
        \answer{D}
    \end{questionBox}

    \begin{questionBox}2{ % Q6.3
    } % Q6.3
        \begin{BM}
            \begin{cases}
                H_0:\sigma^2\leq10^2
                \\
                H_1:\sigma^2>10^2
            \end{cases}
        \end{BM}
        \answer{B}
    \end{questionBox}

    \begin{questionBox}2{ % Q6.4
        p-valor
    } % Q6.4
        \begin{flalign*}
            &
                P(\sigma^2=S^2\vert{H_0})
            &
        \end{flalign*}
        \answer{C}
    \end{questionBox}

    \begin{questionBox}2{ % Q6.5
    } % Q6.5
        \answer{D}
    \end{questionBox}
\end{questionBox}

\end{document}
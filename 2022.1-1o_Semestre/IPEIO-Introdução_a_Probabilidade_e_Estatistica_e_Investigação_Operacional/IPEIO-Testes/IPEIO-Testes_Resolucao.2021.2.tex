% !TEX root = ./IPEIO-Testes_Resolução.2021.2.tex
\providecommand\mainfilename{"./IPEIO-Testes_Resolução.tex"}
\providecommand \subfilename{}
\renewcommand   \subfilename{"./IPEIO-Testes_Resolução.2021.2.tex"}
\documentclass[\mainfilename]{subfiles}

% \tikzset{external/force remake=true} % - remake all

\begin{document}

\graphicspath{{\subfix{./.build/figures/IPEIO-Testes_Resolução.2021.2}}}
\tikzsetexternalprefix{./.build/figures/IPEIO-Testes_Resolução.2021.2/}

\mymakesubfile{2}
[IPEIO]
{Resolução Teste} % Subfile Title
{Resolução Teste} % Part Title

\setcounter{group}{1}
\group{}

\begin{questionBox}1{ % Q1
    Considere o problema de Programação Linear Q com variáveis \(x_1, x_2\text{ e }x_3\) e uma função objetivo de tipo máximo. O problema tem duas restrições. As variáveis de folga associadas à primeira e segunda restrições são respetivamente \(f_1\text{ e }f_2\). Considere o seguinte quadro do Simplex relativo ao problema Q.
} % Q1
    \begin{center}
        \vspace{1ex}
        \begin{tabular}{*{5}{C} | CC}
            \toprule
            
                x_1 & x_2 & x_3 & f_1 & f_2 & MDir & sba

            \\\midrule

                0 & A & B & 1 & 0 & 20 & z=20
                \\
                1 & 2 & C & 2 & 0 & 12 & x_1=12
                \\
                0 & -1 & 1 & 1 & 1 & 10 & f_2=10
            
            \\\bottomrule
        \end{tabular}
        \vspace{2ex}
    \end{center}

    Apresentam-se em seguida algumas afirmações sobre o quadro anterior. Para cada afirmação, indique se se trata de uma afirmação Verdadeira (V) ou Falsa (F) marcando um X no quadrado da coluna respetiva.
\end{questionBox}

\begin{questionBox}2{ % Q1.1
    No quadro, a variável \(f_1\) é não básica.
} % Q1.1
    Verdade, pois não é uma coluna pivot
\end{questionBox}

\begin{questionBox}2{ % Q1.2
    No quadro, a variável \(f_1\) tem valor 1.
} % Q1.2
    Falsa
\end{questionBox}

\begin{questionBox}2{ % Q1.3
    Considere \(A=B=C=2\). Nesta situação o quadro é ótimo.
} % Q1.3
    Sim, pois a solução maximizante é não positiva
\end{questionBox}

\begin{questionBox}2{ % Q1.4
    Quaisquer que sejam os valores de A, B e C, o problema Q tem sempre pelo menos uma solução básica admissível.
} % Q1.4
    Verdadeira
\end{questionBox}

\group{}

\setcounter{question}{1}

\begin{questionBox}1{ % Q2
    Considere o seguinte projeto onde as atividades têm duração aleatória. Junto a cada atividade indica-se o valor médio da sua duração (em dias) e sabe-se que o desvio padrão da duração de cada atividade é 10\% do respetivo valor médio.
} % Q2

    \tikzset{external/remake next=true} % remake next
    \begin{center}
    \begin{tikzpicture}[myGraphsStyle]
    
        % ===================== Vertices ===================== %
    
        % 1,2,3`'
        \pic (1) at (0, 0) {graphVertice={1/0/0/below/5mm}};
        \pic (2) at (2, 1) {graphVertice={2/0/0/below/5mm}};
        \pic (3) at (2,-1) {graphVertice={3/0/0/below/5mm}};
        \pic (4) at (4, 1) {graphVertice={4/0/0/below/5mm}};
        \pic (5) at (6, 0) {graphVertice={5/0/0/below/5mm}};
    
        % ======================= Edges ====================== %
    
        % % 1 -> 2 -> 4 -> 5
        \draw[->] (1-x) --node[sloped,above]{A 10} (2-x);
        \draw[->] (2-x) --node[sloped,above]{B 10} (4-x);
        \draw[->] (4-x) --node[sloped,above]{C 20} (5-x);
        % 1 -> 3 -> 5
        \draw[->] (1-x) --node[sloped,below]{D 20} (3-x);
        \draw[->] (3-x) --node[sloped,below]{E 10} (5-x);
    
        % ==================================================== %
    
    \end{tikzpicture}
    \end{center}

    De acordo com a técnica PERT, a probabilidade da duração do projeto ser superior a 42 dias é:

    \begin{answerBox}{} % RS 
        \tikzset{external/remake next=true} % remake next
        \begin{center}
        \begin{tikzpicture}[myGraphsStyle]
        
            % ===================== Vertices ===================== %
        
            % 1,2,3`'
            \pic (1) at (0, 0) {graphVertice={1/0/0/below/5mm}};
            \pic (2) at (2, 1) {graphVertice={2/10/10/below/5mm}};
            \pic (3) at (2,-1) {graphVertice={3/20/30/below/5mm}};
            \pic (4) at (4, 1) {graphVertice={4/20/20/below/5mm}};
            \pic (5) at (6, 0) {graphVertice={5/40/40/below/5mm}};
        
            % ======================= Edges ====================== %
        
            % % 1 -> 2 -> 4 -> 5
            \draw[->] (1-x) --node[sloped,above]{A 10} (2-x);
            \draw[->] (2-x) --node[sloped,above]{B 10} (4-x);
            \draw[->] (4-x) --node[sloped,above]{C 20} (5-x);
            % 1 -> 3 -> 5
            \draw[->] (1-x) --node[sloped,below]{D 20} (3-x);
            \draw[->] (3-x) --node[sloped,below]{E 10} (5-x);
        
            % ==================================================== %
        
        \end{tikzpicture}
        \end{center}

        \begin{flalign*}
            &
                \sigma
                = \sqrt{
                    (10*0.1)^2
                    +(10*0.1)^2
                    +(20*0.1)^2
                } = \sqrt{6};
                &\\&
                P(T>42)
                = 1-P(T\leq42)
                = &\\&
                = 1-P\left(z\leq\frac{42-40}{\sqrt{6}}\right)
                = 1-P(z\leq\num{0.816496580927726})
                \cong 1-0.7939
                = 0.2061
            &
        \end{flalign*}

    \end{answerBox}
\end{questionBox}


\end{document}
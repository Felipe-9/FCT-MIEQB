% !TEX root = ./IPEIO-Testes_Resolução.2018.1.tex
\providecommand\mainfilename{"./IPEIO-Testes_Resolução.tex"}
\providecommand \subfilename{}
\renewcommand   \subfilename{"./IPEIO-Testes_Resolução.2018.1.tex"}
\documentclass[\mainfilename]{subfiles}

% \tikzset{external/force remake=true} % - remake all

\begin{document}

% \graphicspath{{\subfix{./.build/figures/IPEIO-Testes_Resolução.2018.1}}}

\mymakesubfile{1}
[IPEIO]
{Teste 2018} % Subfile Title
{Teste 2018} % Part Title

\begin{questionBox}1{ % Q1
    Admita que \textit{A}, \textit{B} e \textit{C} são acontecimentos de um espaço de acontecimentos \((\Omega,\mathcal{F})\) e que: \(P(A) = 0.2\), \(P (B) = 0.4\), \(P (A\vert{}B) = 0.3\), \(P(A\cap{}C) = 0.1\) e os acontecimentos \textit{A} e \textit{C} são independentes.
} % Q1
    \begin{questionBox}2{ % Q1.1
        \(P(C) = 0.3\)
    } % Q1.1
        \begin{flalign*}
            &
                P(C)\,P(A)
                = P(A\cap{}C)
                % \implies &\\&
                \implies
                P(C) 
                = P(A\cap{}C)/P(A)
                = 0.1/0.2 
                = 0.5
                \neq 0.3
            &
        \end{flalign*}
    \end{questionBox}

    \begin{questionBox}2{ % Q1.2
        \(P(B\vert{}A) = 0.6\)
    } % Q1.2
        \begin{flalign*}
            &
                P(B\vert{}A)
                = P(A\cap{}B)/P(A)
                = P(A\vert{}B)\,P(B)/P(A)
                = 0.3*0.4/0.2
                = 0.6
            &
        \end{flalign*}
    \end{questionBox}

    \begin{questionBox}2{ % Q1.3
        \(P(A-C) = 0.1\)
    } % Q1.3
        \begin{flalign*}
            &
                P(A-C)
                = P(A)-P(A\cap{}C)
                = 0.2-0.1
                = 0.1
            &
        \end{flalign*}
    \end{questionBox}
\end{questionBox}

\begin{questionBox}1{ % Q2
    Numa determinada empresa de I\&D, 60\% dos seus colaboradores foram formados na FCT-NOVA. Sabe-se que quando um projecto é entregue a um colaborador formado na FCT-NOVA a probabilidade de ser concluido com sucesso é de 90\%, já quando é entregue a um qualquer outro colaborador essa probabilidade desce para 70\%.
} % Q2

    \begin{BM}
        P(FCT) = 60\%
        ;\quad
        P(1\vert{FCT}) = 90\%
        ;\quad
        P(1\vert{\overline{FCT}})=70\%
    \end{BM}

    \begin{questionBox}2{ % Q2.1
        Qual a probabilidade de um projecto ser concretizado com sucesso e por um colaborador formado na FCT-NOVA?
    } % Q2.1
        \begin{flalign*}
            &
                P(1\cap{FCT})
                = P(1\vert{FCT})\,P(FCT)
                = 0.9*0.6 = 0.54
            &
        \end{flalign*}
    \end{questionBox}

    \begin{questionBox}2{ % Q2.2
        Tendo sido distribuido de forma aleatória um novo projecto por entre os colaboradores da empresa, qual a probabilidade deste ser concretizado com sucesso?
    } % Q2.2
        \begin{flalign*}
            &
                P(1)
                = P(1\vert{FCT})\,P(FCT)
                + P(1\vert{\overline{FCT}})\,P(\overline{FCT})
                = 0.9*0.6 + 0.7*0.4
                = 0.82
            &
        \end{flalign*}
    \end{questionBox}

    \begin{questionBox}2{ % Q2.3
        Se um projecto tiver sido concluido com sucesso, qual a probabilidade de ter sido realizado por um colaborador formado na FCT-NOVA?
    } % Q2.3
        \begin{flalign*}
            &
                P(FCT\vert{1})
                = P(FCT\cap{}1)/P(1)
                = 0.54/0.82
                = 27/41
            &
        \end{flalign*}
    \end{questionBox}

    \begin{questionBox}2{ % Q2.4
        Se a empresa tiver um total de 100 colaboradores e 10 forem selecionados de forma aleatória para formarem uma equipa, o número de colaboradores formados na FCT-NOVA que participam nessa equipa é uma variável aleatória com distribuição:
    } % Q2.4
        Extrai 10 de 100 sem reposição de onde \(0.6*100=60\) são da FCT
        \begin{flalign*}
            &
                \hiperbolica{(100,60,10)}
            &
        \end{flalign*}
    \end{questionBox}
\end{questionBox}

\begin{questionBox}1{ % Q3
    Considere a seguinte fução
} % Q3
    \begin{BM}
        g(x)=
        \begin{cases}
            a/b\quad& x\in\myrange{0,b}
            \\
            0\quad& x\not\in\myrange{0,b}
        \end{cases}
    \end{BM}

    % \begin{questionBox}2{ % Q3.1
    %     \box{V} ou \box{F} para \(a=1\) e \(b=1\), a função \textit{g} é uma função densidade de probabilidade.
    % } % Q3.1
    %     body
    % \end{questionBox}

    % \begin{questionBox}2{ % Q3.2
    %     Considere \textit{X}, uma v.a. com função densidade \(f_X(x)=g(x)\) com \(a=1,b=2\)
    % } % Q3.2
    % \end{questionBox}
\end{questionBox}

\end{document}
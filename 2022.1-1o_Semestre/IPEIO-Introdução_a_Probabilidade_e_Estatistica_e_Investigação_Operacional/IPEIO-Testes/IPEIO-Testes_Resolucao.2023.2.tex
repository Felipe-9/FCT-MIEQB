% !TEX root = ./IPEIO-Testes_Resolução.2023.2.tex
\providecommand\mainfilename{"./IPEIO-Testes_Resolução.tex"}
\providecommand \subfilename{}
\renewcommand   \subfilename{"./IPEIO-Testes_Resolução.2023.2.tex"}
\documentclass[\mainfilename]{subfiles}

% \tikzset{external/force remake=true} % - remake all

\begin{document}

\graphicspath{{\subfix{./.build/figures/IPEIO-Testes_Resolução.2023.2}}}
\tikzsetexternalprefix{./.build/figures/IPEIO-Testes_Resolução.2023.2/}

\mymakesubfile{2}
[IPEIO]
{Teste Resolução} % Subfile Title
{Teste Resolução} % Part Title

\begin{questionBox}1{ % Q1
} % Q1
    \begin{center}
        \vspace{1ex}
        \begin{tabular}{c c}
            \toprule
            
                Periodo do dia
                & \begin{tabular}{c}
                    Numero mínimo
                    \\de consultores
                    \\necessários
                \end{tabular}
            
            \\\midrule
            
                    8h--12h & 4
                \\ 12h--16h & 8
                \\ 16h--20h & 10
                \\ 20h--24h & 6
            
            \\\bottomrule
        \end{tabular}
        \vspace{2ex}
    \end{center}

    \begin{itemize}
        \item Consultores tempo inteiro ou parcial
        \item Tempo inteiro=> 8h consecultivas:
        \begin{itemize}
            \item  8h--16h
            \item 12h--20h
            \item 16h--24h
        \end{itemize}
        \item Tempo inteiro são pagos 40EUR/h
        \item Tempo parcial, periodos na tabela
        \item Tempo parcial são pagos 30EUR/h
        \item min 2 Tempo inteiro a cada 1 parcial
    \end{itemize}

    % \answer{}

    \begin{description}[
        leftmargin=!,
        labelwidth=\widthof{\(C_x\)} % Longest item
    ]
        \item[\(C_{i,j}\)] Numero de consultores em tempo inteiro no horario \{
            1: 8h--16h, 
            2:12h--20h, 
            3:16h--24h
        \}
        \item[\(C_{p,j}\)] Numero de consultores em tempo parcial no horario \{
            1:  8h--12h,
            2: 12h--16h,
            3: 16h--20h,
            4: 20h--24h
        \}
    \end{description}

    \begin{BM}[align*]
        \text{Minimizar} \quad& 
        C
        =40*8\,\sum_{j=1}^{3}{c_{i,j}}
        +30*4\,\sum_{j=1}^{4}{c_{p,j}}
        \\\text{Sujeito a} \quad&
        \left\{
            \begin{aligned}
                   c_{i,1}/2           &\geq c_{p,1}
                \\ (c_{i,1}+c_{i,2})/2 &\geq c_{p,2}
                \\ (c_{i,2}+c_{i,3})/2 &\geq c_{p,3}
                \\ c_{i,3}/2           &\geq c_{p,4}
                \\c_{p,1}+c_{i,1}         &\geq 4
                \\c_{p,2}+c_{i,1}+c_{i,2} &\geq 8
                \\c_{p,3}+c_{i,2}+c_{i,3} &\geq 10
                \\c_{p,4}+c_{i,3}         &\geq 6
                \\c_{p,j},c_{i,j}\in\mathbb{N}
            \end{aligned}
        \right.
    \end{BM}

\end{questionBox}

\begin{questionBox}1{ % Q2
} % Q2
    \begin{BM}[align*]
        \max z &= 5\,x+3\,y
        \\
        s.a. 
        & \begin{cases}
               x&\leq 4
            \\ y&\leq6
            \\ 3\,x+2\,y&\leq18
            \\ x+y&\leq2
            \\ x,y&\leq0
        \end{cases}
    \end{BM}

    \begin{questionBox}2{ % Q2.1
        Região adm
    } % Q2.1
        E
    \end{questionBox}

    \begin{questionBox}2{ % Q2.2
        Vertice ótimo e sba
    } % Q2.2

        \begin{flalign*}
            &
                0=5\,x+3\,y
                \implies
                y=\frac{5}{3}x
            &
        \end{flalign*}

        Percebemos uma reta de declive crescente (5/3) que tende sempre a almentar, podemos ver que o vertice \((x,y)=(0,6)\) é solução por estar mais a cima e a esquerda possível
        \begin{itemize}
            \item Vértice: \((0,6)\)
            \item S.b.a: \((x,y)=(0,6)\)
        \end{itemize}
    \end{questionBox}

    \begin{questionBox}2{ % Q2.3
        Variavéis do vertice (2,6)
    } % Q2.3
        \begin{BM}
            (x,y,f_2,f_3)
        \end{BM}
    \end{questionBox}

    \begin{questionBox}2{ % Q2.4
        Nova função
    } % Q2.4
        \begin{BM}
            z'=c\,x+3\,y
        \end{BM}

        \begin{enumerate}
            \begin{multicols}{2}
                \item \(c=2000\): F, solução sempre limitada
                \item \(c=1\): V, declive positívo
                \item \(c=9/2\): F, se for positiva vai sempre dar (0,6)
                \item \(c<0\): F, existe uma margem q resulta em (4,y>0)
            \end{multicols}
        \end{enumerate}

        Resposta, b)
    \end{questionBox}

\end{questionBox}

\begin{questionBox}1{ % Q3
    Simplex
} % Q3
    \begin{center}
        \vspace{1ex}
        \begin{tabular}{*{5}{C}|C}
            \toprule
            
                x_1 & x_2 & x_3 & f_1 & f_2 & T.I
            
            \\\midrule
            
                -5+\alpha & 0 & 2-2\,\alpha & 0 & -3-\alpha & 18
                \\ \alpha-1 & 1 & 1 & 0 & -1 & 10-\alpha
                \\ 3 & 0 & 1 & 1 & 1 & 6-\alpha
            
            \\\bottomrule
        \end{tabular}
        \vspace{2ex}
    \end{center}

    \begin{questionBox}2{ % Q3.1
        \(\alpha=3\) e minimizar, sol otima é \((x_1^*,x_2^*,x_3^*)=(0,7,0)\)
    } % Q3.1
    \begin{center}
        \vspace{1ex}
        \begin{tabular}{*{5}{C}|C}
            \toprule
            
                x_1 & x_2 & x_3 & f_1 & f_2 & T.I
            
            \\\midrule
            
                -5+3 & 0 & 2-2*3 & 0 & -3-3 & 18
                \\ 3-1 & 1 & 1 & 0 & -1 & 10-3
                \\ 3 & 0 & 1 & 1 & 1 & 6-3
            
            \\\midrule
            
                  -2 & 0 & -4 & 0 & -6 & 18
                \\ 2 & 1 &  1 & 0 & -1 & 7
                \\ 3 & 0 &  1 & 1 & 1 & 3
            
            \\\bottomrule
        \end{tabular}
        \vspace{2ex}
    \end{center}

    Falsa
    \end{questionBox}

    \begin{questionBox}2{ % Q3.2
        \(\alpha=1\) e max, sol não ótima, \(x_1\) vira básica e \(f_1\) vira não básica
    } % Q3.2
    \begin{center}
        \vspace{1ex}
        \begin{tabular}{*{5}{C}|C}
            \toprule
            
                x_1 & x_2 & x_3 & f_1 & f_2 & T.I
            
            \\\midrule
            
                -5+-1 & 0 & 2-2*-1 & 0 & -3+1 & 18
                \\ -1-1 & 1 & 1 & 0 & -1 & 10+1
                \\ 3 & 0 & 1 & 1 & 1 & 6+1
            
            \\\midrule
            
                -6 & 0 & 4 & 0 & -2 & 18
                \\ -2 & 1 & 1 & 0 & -1 & 11
                \\ 3 & 0 & 1 & 1 & 1 & 7
            
            \\\bottomrule
        \end{tabular}
        \vspace{2ex}
    \end{center}
    Falsa
    \end{questionBox}

    \begin{questionBox}2{ % Q3.3
        \(\alpha=3\) e max, é sol ótima
    } % Q3.3
        \begin{center}
            \vspace{1ex}
            \begin{tabular}{*{5}{C}|C}
                \toprule
                
                    x_1 & x_2 & x_3 & f_1 & f_2 & T.I

                \\\midrule
                
                    -2 & 0 & -4 & 0 & -6 & 18
                    \\ 2 & 1 &  1 & 0 & -1 & 7
                    \\ 3 & 0 &  1 & 1 & 1 & 3
                
                \\\bottomrule
            \end{tabular}
            \vspace{2ex}
        \end{center}
        Verdadeira
    \end{questionBox}

    \begin{questionBox}2{ % Q3.4
        Max e \(\alpha=5\), mais q uma sol otima
    } % Q3.4
        \begin{center}
            \vspace{1ex}
            \begin{tabular}{*{5}{C}|C}
                \toprule
                
                    x_1 & x_2 & x_3 & f_1 & f_2 & T.I
                
                \\\midrule
                
                    -5+5 & 0 & 2-2*5 & 0 & -3-5 & 18
                    \\ 5-1 & 1 & 1 & 0 & -1 & 10-5
                    \\ 3 & 0 & 1 & 1 & 1 & 6-5
                
                \\\midrule
                
                       0 & 0 & -8 & 0 & -8 & 18
                    \\ 4 & 1 & 1 & 0 & -1 & 4
                    \\ 3 & 0 & 1 & 1 & 1 & 1
                
                \\\bottomrule
            \end{tabular}
            \vspace{2ex}
        \end{center}

        Verdadeira, anula \(x_1\)
    \end{questionBox}

    \begin{questionBox}2{ % Q3.5
        \(x_1=-5+\alpha\)
    } % Q3.5
        Falsa, esse é o escalar multiplicando o numero
    \end{questionBox}
\end{questionBox}

\begin{questionBox}1{ % Q4
    Projeto
} % Q4
    
    \begin{questionBox}2{ % Q4.1
        Desenho
    } % Q4.1
        % \tikzset{external/remake next=true} % remake next
        \begin{center}
        \begin{tikzpicture}[myGraphsStyle]

            % ===================== Vertices ===================== %

            % 1,2,5,6
            \pic (1) at (0, 0) {graphVertice={1/ 0/ 0/below/4mm}};
            \pic (3) at (2, 1) {graphVertice={3/24/24/above/4mm}};
            \pic (4) at (4, 1) {graphVertice={4/42/42/above/4mm}};
            \pic (6) at (6, 0) {graphVertice={6/60/60/below/4mm}};
            % 3,4
            \pic (2) at (2,-1) {graphVertice={2/12/12/below/4mm}};
            \pic (5) at (4,-1) {graphVertice={5/52/52/below/4mm}};

            % ======================= Edges ====================== %

            % 1 ->[A] 3 ->[D] 4 ->[H] 6
            \draw[->] (1-x) --node[sloped,above]{\(A_{14}\)} (3-x);
            \draw[->] (3-x) --node[sloped,above]{\(D_{18}\)} (4-x);
            \draw[->] (4-x) --node[sloped,above]{\(H_{16}\)} (6-x);
            % 1 ->[B] 3 ->[E] 5 ->[G] 6
            \draw[->] (1-x) --node[sloped,above]{\(B_{12}\)} (2-x);
            \draw[->] (2-x) --node[sloped,above]{\(E_{12}\)} (5-x);
            \draw[->] (5-x) --node[sloped,above]{\(G_{ 8}\)} (6-x);
            % 2 ->[C] 3
            \draw[->] (2-x) --node[sloped,above]{\(C_{12}\)} (3-x);
            % 4 -> 5
            \draw[->] (4-x) --node[sloped,above]{\(F_{10}\)} (5-x);
            % 3 - -> 5
            \draw[->,dashed] (3-x) -- (5-x);
            
            % ==================================================== %

        \end{tikzpicture}
        \end{center}
    \end{questionBox}

    \begin{questionBox}2{ % Q4.2
        Duração total e caminho crítico médio
    } % Q4.2
        \begin{itemize}
            \item caminho: B\rightarrow C\rightarrow D\rightarrow F\rightarrow G\\
            \item Tempo: 60 dias
        \end{itemize}
    \end{questionBox}

    \begin{questionBox}2{ % Q4.3
        Prob de exceder 52 dias
    } % Q4.3
        \begin{flalign*}
            &
                \sigma
                = \sqrt{0.8+0.6+1.2+0.8+0.6}
                = \sqrt{4} = 2
                &\\&
                P(x\geq 57)
                = 1-P(x\leq 57)
                = 1-P\left(z\leq \frac{60-57}{2}\right)
                = &\\&
                = 1-P(z\leq 1.5)
                \cong 1-0.9332
                = 0.0668
            &
        \end{flalign*}
    \end{questionBox}

\end{questionBox}

\begin{questionBox}1{ % Q5
    Prog lin inteira
} % Q5
    \begin{BM}[align*]
        ?? \quad &z=4\,x_1+8\,x_2
        \\
        s.a &
        \begin{cases}
               8\,x_1+3\,x_2 &\leq 52
            \\ 2\,x_1+7\,x_2 &\leq 46
            \\ x_1,x_2&\in\mathbb{N}
        \end{cases}
    \end{BM}

    \begin{enumerate}
        \item F
        \item F
        \item V
        \item F
        \item V
    \end{enumerate}

\end{questionBox}

% (52-3*y)/8=x
% (52-3*6)/8=x

\end{document}

% \tikzset{external/remake next=true} % remake next
% \begin{center}
% \begin{tikzpicture}[myGraphsStyle]

%     % ===================== Vertices ===================== %

%     % 1,2,5,6
%     \pic (1) at (0, 0) {graphVertice={1/ 0/ 0/below/4mm}};
%     \pic (2) at (2, 1) {graphVertice={2/ 6/10/above/4mm}};
%     \pic (5) at (4, 1) {graphVertice={5/14/14/above/4mm}};
%     \pic (6) at (6, 0) {graphVertice={6/24/24/below/4mm}};
%     % 3,4
%     \pic (3) at (2,-1) {graphVertice={3/ 8/ 8/below/4mm}};
%     \pic (4) at (4,-1) {graphVertice={4/14/14/below/4mm}};

%     % ======================= Edges ====================== %

%     % 1 ->[A] 2 ->[B] 5 ->[E] 6
%     \draw[->] (1-x) --node[sloped,above]{A  6} (2-x);
%     \draw[->] (2-x) --node[sloped,above]{B  4} (5-x);
%     \draw[->] (5-x) --node[sloped,above]{E 10} (6-x);
%     % 1 ->[C] 3 ->[D] 4 ->[F] 6
%     \draw[->] (1-x) --node[sloped,above]{C  8} (3-x);
%     \draw[->] (3-x) --node[sloped,above]{D  6} (4-x);
%     \draw[->] (4-x) --node[sloped,above]{F  8} (6-x);
%     % 4 -> 5
%     \draw[->,dashed] (4-x) -- (5-x);
    
%     % ==================================================== %

% \end{tikzpicture}
% \end{center}
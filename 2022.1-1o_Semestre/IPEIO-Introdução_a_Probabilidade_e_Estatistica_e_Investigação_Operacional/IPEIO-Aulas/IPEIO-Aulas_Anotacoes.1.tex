% !TEX root = ./IPEIO-Aulas_Anotações.1.tex
\providecommand\mainfilename{"./IPEIO-Aulas_Anotações.tex"}
\providecommand \subfilename{}
\renewcommand   \subfilename{"./IPEIO-Aulas_Anotações.1.tex"}
\documentclass[\mainfilename]{subfiles}

% \tikzset{external/force remake=true} % - remake all

\begin{document}

% \graphicspath{{\subfix{./.build/figures/}}}

\mymakesubfile{1}
[IPEIO]
{Anotações} % Subfile Title
{Anotações} % Part Title

\begin{questionBox}1{ % Q1
    question
} % Q1
    \begin{flalign*}
        &
            P(A\cup B\cup C)
            = P(A\cup (B\cup C))
            = P(A) + P(B\cup C) - P(A\cap(B\cup C))
            = &\\&
            = P(A) + P(B\cup C) - P(A\cap B) - P(A\cup C) + P(A\cap B\cap C)
            = &\\&
            = P(A) + P(B) + P(C) - P(A\cap B) - P(A\cup C) - P(B\cup C) + P(A\cap B\cap C)
            = &\\&
            = P(A) + P(B) + P(C) - P(A) - P(\emptyset) + P(\emptyset)
            = &\\&
            = 0.2 + 0.4 + 0.1 - 0.2 - 0 + 0
            = &\\&
            = 0.5
        &
    \end{flalign*}
\end{questionBox}

\begin{questionBox}1{ % Q2
    Para efeitos de controlo da poluição no rio Tejo, são recolhidas de forma periódica amostras de  ́agua em três localizações distintas L1, L2 e L3. Em L1 são recolhidas o dobro das amostras relativamente a qualquer uma das outras localizações (L2,L3). A percentagem de amostras com resultado positivo, para um certo tipo de poluente, é de 2\% em L1 e L2, enquanto que em L3 é de 4\%. Todas as amostras são guardadas e arquivadas num uníco lugar mas alguém eliminou todos os registos identificativos das amostras.
} % Q2
    \begin{questionBox}2{ % Q2.1
        Escolhendo uma amostra ao acaso de entre todas as que estão arquivadas, qual a probabilidade desta ser positiva para o poluente?
    } % Q2.1
        \begin{flalign*}
            &
                P(1)
                = P(1|L_1)\,P(L_1) 
                + P(1|L_2)\,P(L_2) 
                + P(1|L_3)\,P(L_3)
                = &\\&
                = 0.02*0.50
                + 0.02*0.25
                + 0.04*0.25
                = 0.025
            &
        \end{flalign*}
    \end{questionBox}

    \begin{questionBox}2{ % Q2.2
        Prob de ser A se for positiva
    } % Q2.2
        \begin{flalign*}
            &
                P(L_1|1)
                = \frac{P(L_1\cap 1)}{P(P)}
                = \frac{P(1|L_1)*P(L_1)}{P(1)}
                = \frac{0.02*0.5}{0.025}
                = 0.40
            &
        \end{flalign*}
    \end{questionBox}
\end{questionBox}

\part{15/03/2023}

\begin{questionBox}1{ % Q1
} % Q1
    \begin{flalign*}
        &
            P(X=x)
            = \frac{\binom{M}{x}\binom{N-M}{n-x}}{\binom{N}{n}}
            = \frac{\binom{5}{x}\binom{4}{3-x}}{\binom{9}{3}}
        &
    \end{flalign*}
\end{questionBox}

\begin{sectionBox}1{Distribuições} % S1
    
    \begin{sectionBox}2{Hipergeométrica} % S1.1
        
        \begin{BM}
            X\sim H(N,M,n)
            \implies
            P(X=x)
            = \frac{\binom{M}{x}\binom{N-M}{n-x}}{\binom{N}{n}}
        \end{BM}
        
    \end{sectionBox}

    \begin{sectionBox}2{Binomial} % S1.2
        
        \begin{BM}
            X\sim Bin(n,p)
            \implies
            P(X=x)
            = \binom{n}{x}
            \, p^x
            \, (1-p)^{n-x}
            \\
            E(X)=n\,p
            \qquad
            V(X)= n\,p\,(1-p)
        \end{BM}
        
    \end{sectionBox}
    
    \begin{sectionBox}2{Bernouli} % S1.3
        
        \begin{BM}
            X\sim Ber(p) = Bin(1,p)
        \end{BM}
        
    \end{sectionBox}

    \begin{sectionBox}2{Poisson} % S1.4
        
        \begin{BM}
            X\sim Poi(\lambda)
            \implies
            \begin{Bmatrix}
                P(X=k) = \frac{e^{-\lambda}\,\lambda^{k}}{k!}
                \quad k\in\mathbb{N}
                \\
                E(X) = V(X) = \lambda
            \end{Bmatrix}
        \end{BM}
        
    \end{sectionBox}

\end{sectionBox}


\end{document}
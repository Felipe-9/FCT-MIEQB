% !TEX root = ./IPEIO-Aula_P.01_Anotações.tex
% !TEX root = ./IPEIO-Aulas_Anotações.tex
\providecommand\mainfilename{"./IPEIO_-_Aulas_Anotações.tex"}
\providecommand \subfilename{}
\renewcommand   \subfilename{"./IPEIO_-_Aula_P.01_Anotações.tex"}
\documentclass[\mainfilename]{subfiles}

% \graphicspath{{\subfix{../images/}}}

\begin{document}

\mymakesubfile{1}
{IPEIO 03/10}
{IPEIO 03/10}

\begin{sectionBox}1{Arranjo}
    
    \begin{BM}
        ^nA_m = \frac{n!}{(n-m)!}
    \end{BM}
    
\end{sectionBox}

\begin{sectionBox}1{Combinação}
    
    \begin{BM}
        ^nC_m = \frac{n!}{(n-m)!\,m!}
    \end{BM}
    
\end{sectionBox}

\begin{sectionBox}1{Acontecimentos independentes}
    
    \begin{BM}
        \{A,B\} \text{ São Independentes} \iff P(A\cap B) = P(A)\,P(B)
    \end{BM}
    
\end{sectionBox}

\begin{sectionBox}1{Probabilidade condicional}
    
    \begin{BM}
        P(A|B) = \frac{P(A\cap B)}{P(B)}
    \end{BM}
    
\end{sectionBox}

\section*{Exerçicios da Lista}

% Q19
\setcounter{question}{18}
\begin{questionBox}1{}
    
    \begin{flalign*}
        &
            P(A\cup B\cup C) 
            = P(A\cup B) + P(C) - P((A\cup B)\cap C)
            = P(A) + P(B) - P(A\cap B) + P(C) - P((A\cap B)\cup (B\cap C))
            = P(A) + P(B) - P(A\cap B) + P(C) - (P(A\cap C) + P(B\cap C) - P(A\cap B\cap C))
            = P(A) + P(B) + P(C) - P(A\cap C) - P(A\cap B) - P(B\cap C) + P(A\cap B\cap C)
            = 3\frac{1}{4} - \frac{1/8}
            = 5/8
        &
    \end{flalign*}
    
\end{questionBox}

% Q20
\begin{questionBox}1{}
    
    \begin{BM}
        P(A)=\frac{2}{3},\ P(B)=\frac{1}{2},\ P(A\cap B) = \frac{1}{3}
    \end{BM}

    \begin{enumerate}[label=\roman{enumi}]
        \begin{multicols}{2}
            \item \(P(A-B)\)
            \item \(P(A\cup B) \)
            \item \(P(\bar{A}\cup \bar{B}) \)
            \item \(bar{A}\cap B \)
            \item \(A\cup \bar{B}\)
        \end{multicols}
    \end{enumerate}

    \begin{multicols}{2}

        \begin{questionBox}3{\(P(A-B)\)}
            \begin{flalign*}
                &
                    = P(A-B)=P(A)-P(A\cap B);
                &
            \end{flalign*}
        \end{questionBox}

        \begin{questionBox}3{\(P(\bar{A}\cup \bar{B})\)}
            \begin{flalign*}
                &
                    = P(\bar{A\cap B}) = 1-P(A\cap B)
                &
            \end{flalign*}
        \end{questionBox}

        \begin{questionBox}3{\(P(\bar{A}\cap B)\)}
            \begin{flalign*}
                &
                    = P(B)-P(A\cap B)
                &
            \end{flalign*}
        \end{questionBox}

        \begin{questionBox}3{\(P(A\cup \bar{B})\)}
            \begin{flalign*}
                &
                    = P(\bar{\bar{A}\cup B}) = 1-P(\bar{A}\cap B)
                &
            \end{flalign*}
        \end{questionBox}

    \end{multicols}
    
\end{questionBox}

% Q21
\begin{questionBox}1{}

    \begin{questionBox}3{\(P(A\cap B)\)}
        \begin{flalign*}
            &
                = P(A)\,P(B)
            &
        \end{flalign*}
    \end{questionBox}

    \begin{questionBox}3{\(P(A\cap\bar{B}) \)}
        \begin{flalign*}
            &
                = P(A)\,P(\bar{B})
            &
        \end{flalign*}
    \end{questionBox}

    \begin{questionBox}3{\(P(\bar{A}\cap B)\)}
        \begin{flalign*}
            &
                = P(\bar{A})\,P(B)
            &
        \end{flalign*}
    \end{questionBox}

    \begin{questionBox}3{\(P(A\cup B) - P(A\cap B)\)}
        \begin{flalign*}
            &
                = P(A) + P(B) - 2\,P(A\cap B) 
                = P(A) + P(B) - 2\,P(A)\,P(B)
            &
        \end{flalign*}
    \end{questionBox}
    
\end{questionBox}

% Q26
\setcounter{question}{25}
\begin{questionBox}1{}
    
    \begin{BM}
        \{A,B\} \text{ São independentes}
        \\
        \{A,\bar{B}\} \text{ São independentes?}
    \end{BM}
    
    \begin{flalign*}
        &
            P(A\cap \bar{B}) = P(A) - P(A\cap B)
            = P(A) - P(A)\,P(B)
            = P(A)(1-P(B))
            = P(A\cap \bar{B})
        &
    \end{flalign*}

\end{questionBox}

\begin{questionBox}*1{Exercício do teste}
    
    \begin{flalign*}
        &
            P(A\cup B\cup C)    
            = P(A) + P(B) + P(C) - (P(A\cap B) + P(B\cap C) + P(A\cap C)) + P(A\cup B\cup C)    
            = P(A) + P(B) + P(C) - (P(A\cap B) + P(A)P(C))
            = 0.2 + 0.4 + 0.1 - (0.1 +  0.2*0.1)
            = 0.58
        &
    \end{flalign*}
    
\end{questionBox}

\end{document}


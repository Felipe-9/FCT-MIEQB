% !TEX root = ./IPEIO_-_Aula_TP.02_Anotações.tex
% !TEX root = ./IPEIO_-_Aulas_Anotações.tex
\providecommand\mainfilename{"./IPEIO_-_Aulas_Anotações.tex"}
\providecommand \subfilename{}
\renewcommand   \subfilename{"./IPEIO_-_Aula_TP.02_Anotações.tex"}
\documentclass[\mainfilename]{subfiles}

% \graphicspath{{\subfix{../images/}}}
% DATA: 03/18

\begin{document}

\mymakesubfile{2}{IPEIO TP}{03/18}

\begin{sectionBox}1{Esperança}
    
    \begin{BM}
        \mu = E(x)
        =   \begin{cases}
                \sum_{i=1}^{\infty} x_i\,P(X=x_i) & :X\text{ é va. discreta}
            \\  \int_{-\infty}^{\infty} x\,f(x)\delta x & :X\text{ é va. continua}
            \end{cases}
    \end{BM}
    
\end{sectionBox}

\begin{questionBox}1{Numero de Caras no lançamento de duas moedas}
    
    \begin{BM}
        X:
        \begin{cases}
            \begin{matrix}
                0 & 1 & 2
            \\  1/4 & 1/2 & 1/4
            \end{matrix}
        \end{cases}
    \end{BM}

    \begin{flalign*}
        &
            \mu
            = \sum x_i(P(X=x_i))
            = 0*P(X=0)
            + 1*P(X=1)
            + 2*P(X=2)
            = 1/2 + 1/2
            = 1
        &
    \end{flalign*}
    
\end{questionBox}

\begin{questionBox}1{População de ratos infetados}
    
    \begin{BM}
        f_X(x)
        =   \begin{cases}
                2(1-x), & 0<x<1
            \\  0, & c.c
            \end{cases}
    \end{BM}

    \begin{flalign*}
        &
            \mu
            = \int_{\infty}^{\infty} x\,f(x)\delta x
            = \int_{0}^{1} x\,2\,(1-x)\delta x
            = 2\int_{0}^{1} x  \delta x
            - 2\int_{0}^{1} x^2\delta x
            = 2\,\Delta(x^2/2)\big\vert_0^1
            - 2\,\Delta(x^3/2)\big\vert_0^1
            = 1 - 2/3
            = 1/3
        &
    \end{flalign*}
    
\end{questionBox}

\begin{sectionBox}1{Teorema da Linearidade}
    
    \begin{BM}
        E(a\,X+b) = a\,E(X)+b:\exists\,E(X)
    \end{BM}
    
\end{sectionBox}

\begin{sectionBox}1{Momento de ordem \textit{k} em torno da origem}
    
    \begin{BM}
        m_k = E[X^k]
        =   \begin{cases}
                \sum_{i=1}^{\infty} x_i^k\,P(X=x_i), & X\text{ é uma va. discreta}
            \\  \int_{-\infty}^{\infty}x^k\,f(x)\delta x, & X\text{ é uma v.a. continua}
            \end{cases}
    \end{BM}
    
\end{sectionBox}

\begin{questionBox}1{Numero de Caras no lançamento de duas moedas}
    
    \begin{BM}
        X:
        \begin{cases}
            \begin{matrix}
                0 & 1 & 2
            \\  1/4 & 1/2 & 1/4
            \end{matrix}
        \end{cases}
    \end{BM}

    \begin{flalign*}
        &
            m_2
            = \sum_{i=1}^{\infty}x_i^2\,P(X=x_i)
            = 0
            + 1^2\,(1/2)
            + 2^2\,(1/4)
            = 1/2 + 1 
            = 3/2
        &
    \end{flalign*}
    
\end{questionBox}

\begin{questionBox}1{População de ratos infetados}
    
    \begin{BM}
        f_X(x)
        =   \begin{cases}
                2(1-x), & 0<x<1
            \\  0, & c.c
            \end{cases}
    \end{BM}

    \begin{flalign*}
        &
            m_2
            = \int_0^1 x^2\,2(1-x)\,\delta x
            = 2\int_0^1 (x^2-x^3)\,\delta x
            = 2\,\Delta(X^3/3 - X^4/4)\big\vert_0^1
        &\\&
            = 2\,(1/3-1/4)
            = 2/12 = 1/6
        &
    \end{flalign*}
    
\end{questionBox}

\begin{sectionBox}1{Momento Centra de Ordem \textit{k}}
    
    \begin{BM}
        \mu_k = E((1-\mu)^k)
    \end{BM}

\end{sectionBox}

\begin{sectionBox}2{Variancia}
        
    \begin{BM}
        \sigma^2 = V(X) = \mu_2 = E((1-\mu)^2)
    \end{BM}

    \begin{sectionBox}*3{Propríedades}
        

        \begin{enumerate} % [mark=(\roman*)]
            \begin{multicols}{2}
                \item \(V(a) = 0: a\in\mathbb{R}^2\)
                \item \(V(a\,X+b) = a^2\,V(X)\)
            \end{multicols}
        \end{enumerate}

        % \begin{sectionBox}3{}
            
        %     body
            
        % \end{sectionBox}
        
    \end{sectionBox}
    
\end{sectionBox}

\begin{sectionBox}2{Desvio Padrão}
    
    \begin{BM}
        \sigma = \sqrt{V(X)} = \sqrt{E((1-\mu)^2)}
    \end{BM}
    
\end{sectionBox}

\begin{questionBox}1{}
    
    \begin{BM}
        \sigma^2 = E(X^2)-E^2(X)
    \end{BM}
    
\end{questionBox}

\begin{questionBox}1{Moeda}
    
    \begin{BM}
        X:
        \begin{cases}
            \begin{matrix}
                0 & 1 & 2
            \\  1/4 & 1/2 & 1/4
            \end{matrix}
        \end{cases}
    \end{BM}
    
\end{questionBox}

\begin{questionBox}1{Ratos}
    
    \begin{BM}
        f_X(x)
        =   \begin{cases}
                2(1-x), & 0<x<1
            \\  0, & c.c
            \end{cases}
    \end{BM}

    \begin{flalign*}
        &
            \sigma^2
        &
    \end{flalign*}
    
\end{questionBox}

\begin{sectionBox}1{Proposição}
    
    \begin{BM}
        E(X\,Y) = E(X)\,E(Y) : \{X,Y\}\text{ v.a. independentes}
    \end{BM}
    
\end{sectionBox}

\begin{sectionBox}1{Proposição}
    
    \begin{BM}
        V(X+Y) = V(X) + V(Y) : \{X,Y\}\text{ v.a. independentes}
    \end{BM}
    
\end{sectionBox}

\begin{sectionBox}1{Mediana}
    
    \begin{BM}
        m_e\in\mathbb{R}:P(X\leq m_e)\geq 0.5 \land P(X\geq m_e)\leq 0.5
    \end{BM}

    Valor intermediário
    
\end{sectionBox}

\begin{sectionBox}1{Moda}
    
    \begin{BM}
        m_0
    \end{BM}

    Valor mais comum
    
\end{sectionBox}

\begin{sectionBox}1{Coeficiente de Variação}
    
    \begin{BM}
        c.v. = 100*\sigma/\mu
    \end{BM}
    
\end{sectionBox}

\begin{sectionBox}1{Coeficiente de Simetria}
    
    \begin{BM}
        \gamma_1 
        = \mu_3/\sigma^3
        = \frac{E((1-\mu)^4)}{E((1-\mu)^{4})}
    \end{BM}
    
\end{sectionBox}

\begin{sectionBox}1{Coeficiente de Achatamento (kurtosis)}
    
    \begin{BM}
        \gamma_2= \mu_4/\sigma^4-3
    \end{BM}
    
\end{sectionBox}

\begin{sectionBox}1{Distribuição Hipergeométrica}
    
    \begin{BM}
        X\sim H(N,M,n)
    \end{BM}

    \paragraph{Note:} Extrações sem reposição

    \begin{sectionBox}*3{Esperança}
        \begin{BM}
            E(X) = n\,M/N
        \end{BM}
    \end{sectionBox}

    \begin{sectionBox}*3{Variancia}
        
        \begin{BM}
            V(X) = \frac{n\,M\,(N-M)(N-n)}{N^2(N-1)}
        \end{BM}
        
    \end{sectionBox}
    
\end{sectionBox}

\begin{questionBox}1{Numero de Peixes saldaveis}
    
    \begin{BM}
        X
        =   \begin{cases}
                \text{Numero de peixes saldaveis ao retirar 3}
            \\
                P(X=x_i) = \frac{C_{x_i}^5\,C_{3-x_i}^4}{C_3^9}:x_i\in[0,3]
            \end{cases}
    \end{BM}

    \begin{BM}
        X\sim H(0,5,3)
    \end{BM}
    
\end{questionBox}

\begin{sectionBox}1{Prova de Bernoulli}
    
    \begin{BM}
        X\sim\bernoulli(1,p)\implies
        X=  \begin{cases}
                \begin{matrix}
                    1 & 0
                \\  p & 1-p
                \end{matrix}
            \end{cases}
    \end{BM}

\end{sectionBox}

\begin{sectionBox}1{Distribuição Binomial}
    
    \begin{BM}
        X\sim\binomial(n,p)
    \end{BM}

    \paragraph{Note:} Extrações com reposição

    \begin{sectionBox}*2{Esperança}
        \begin{BM}
            E(X) = n\,P
        \end{BM}
    \end{sectionBox}

    \begin{sectionBox}*2{Variancia}
        \begin{BM}
            V(X)  = n\,p(1-p)
        \end{BM}
    \end{sectionBox}
    
\end{sectionBox}

\begin{sectionBox}2{Somatório de distribuições binomiais}
    
    \begin{BM}
        \sum_i^{m} X_i \sim\binomial(\sum_i^m n, p):
        \\
        X_i\sim\binomial(n_i,p)\land \{X_i,X_j\}\text{ Independentes }\forall\,\{i,j\}
    \end{BM}
    
\end{sectionBox}

\begin{questionBox}1{}
    
    Probabilidade de um aluno acertar 2 perguntas de 5 tendo em conta que o aluno conhece 60\% da matéria

    \begin{BM}
        X\sim \binomial(5,0.6)
    \\  P(X=x_i) = C_{x_i}^5\,0.6^{x_i}*(1-0.6)^{5-x_i}
    \end{BM}

    \begin{flalign*}
        &
            P(X=2)
            =   C_2^5\,0.6^2*(1-0.6)^3
            =   \dots
        &
    \end{flalign*}
    
\end{questionBox}

\begin{sectionBox}1{Comparação de Distribuições}
    
    % \begin{table}[H]\centering
    %     \begin{tabular}{c}
            
    %         \\\toprule
            
    %             \multicolumn{1}{c}{}
            
    %         \\\midrule
            
                
            
    %         \\\bottomrule
            
    %     \end{tabular}
    % \end{table}
    
\end{sectionBox}

\begin{sectionBox}1{Distribuição de Poisson}
    
    \begin{BM}
        X\sim\poisson(\lambda)
        \implies \\
        \implies
        P(X=x_i) = e^{-\lambda}\,\lambda^x/x!:
        \begin{cases}
            x=[0,\infty)\land\lambda>0
        \end{cases}
    \end{BM}

    \paragraph{Notas:}
    \begin{itemize}
        \item[\(\lambda\):] Taxa de orcorrencias por unidade de tempo
    \end{itemize}

    \begin{sectionBox}*3{Esperança}
        \begin{BM}
            E(X)=\lambda
        \end{BM}
    \end{sectionBox}

    \begin{sectionBox}*3{Variancia}
        \begin{BM}
            V(X)=\lambda
        \end{BM}
    \end{sectionBox}
    
\end{sectionBox}

\begin{sectionBox}*2{Somatório de Distrib de Poisson}
    
    \begin{BM}
        \sum_i^m X_i \sim\poisson\bigg(\sum_i^m\lambda\bigg):
        \\
        X_i\sim\poisson(\lambda_i)
        \land
        \{X_i,X_j\}\text{ Independentes }\forall\,i\neq j
    \end{BM}
    
\end{sectionBox}

\end{document}
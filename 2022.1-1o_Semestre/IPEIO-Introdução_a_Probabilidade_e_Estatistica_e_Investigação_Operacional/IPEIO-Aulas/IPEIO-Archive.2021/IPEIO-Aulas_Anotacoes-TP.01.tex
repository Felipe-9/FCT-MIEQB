% !TEX root = ./IPEIO_-_Aulas_Anotações.tex
% !TEX root = ./IPEIO_-_Aula_TP.01_Anotações.tex
\providecommand\mainfilename{"./IPEIO_-_Aulas_Anotações.tex"}
\providecommand \subfilename{}
\renewcommand   \subfilename{"./IPEIO_-_Aula_TP.01_Anotações.tex"}
\documentclass[\mainfilename]{subfiles}

% \graphicspath{{\subfix{../images/}}}

\begin{document}
\mymakesubfile{1}
{IPEIO 03/11}
{IPEIO 03/11}

\part*{Variáveis Discrétas}

\begin{sectionBox}1{Partição}

    
    \begin{BM}
        X\text{ é partição} \implies 
        \begin{cases}
            X_i\cap X_j = 0\quad\forall\,i\neq j
        \\  \bigcup X_i = \Omega % todo o espaço?
        \end{cases}
    \end{BM}
    
\end{sectionBox}

\begin{questionBox}1{}
    
    \begin{BM}
        B\text{ é uma partição}; A\implies
        P(A) = \sum_{i}P(A|B_i)\,P(B_i)
    \end{BM}
    
    \begin{questionBox}2{Demonstração}
        \begin{flalign*}
            &
            P(A) 
            = P(A\cap \Omega) 
            = P\Bigl(A\cap \bigcup B_i\Bigr)
            = P\Bigl(\bigcup (A\cap B_i)\Bigr)
            = &\\&
            = \sum_i P(A\cap B_i)
            = \sum_i P(A|B_i)\,P(B_i)
            &
        \end{flalign*}
    \end{questionBox}

\end{questionBox}

\begin{sectionBox}2{}
    
    \begin{flalign*}
        &
        P(B_i|A) 
        = \frac{P(B_i\cap A)}{P(A)}
        = \frac{P(A|B_i)\,P(B)}{P(A)}
        \implies
        P(B_i|A)\,P(A) = P(A|B_i)\,P(B_i)
        &
    \end{flalign*}

\end{sectionBox}

\begin{questionBox}1{}
    
    \begin{itemize}
        \begin{multicols}{2}
            \item \( P(+ | L_1) = 2\% \)
            \item \( P(+ | L_2) = 2\% \)
            \item \( P(+ | L_3) = 4\% \)
            \item \( P(L_1) = 2\,P(L_2) = 2\,P(L_3) \)
            \item \( \sum P(L_i) = 1 \)
        \end{multicols}
    \end{itemize}
    
    \begin{questionBox}3{}
        \begin{flalign*}
            &
            P(+)
            = \sum P(+|L_i)\,(L_i)
            = 2\%\,(1/2) + 2\%\,(1/4) + 4\%\,(1/4)
            = &\\&
            = 1\% + 0.5\% + 1\% 
            = 2.5\%
            &
        \end{flalign*}
    \end{questionBox}

    \begin{questionBox}3{}
        \begin{flalign*}
            &
            P(L_1|+)
            = \frac
                {P(+|L_1)\,P(L_1)}
                {P(+)}
            = 2\%\,(1/2)/2.5\%
            &
        \end{flalign*}
    \end{questionBox}

\end{questionBox}

\begin{questionBox}1{}
    
    \begin{itemize}
        \begin{multicols}{2}
            \item \( P(A) = 15\% \)
            \item \( P(B) = 25\% \)
            \item \( P(C) = 65\% \)
            \item \( P(Def| A) = 5\% \)
            \item \( P(Def| B) = 7\% \)
            \item \( P(Def| C) = 4\% \)
        \end{multicols}
    \end{itemize}
    
    \begin{flalign*}
        &
            P(B\vert Def)
            = \frac{P(Def\vert B)P(B)}{P(Def)}
            = &\\&
            = \frac
                {P(Def\vert B)P(B)}
                {
                    P(Def\vert A)P(A)
                   +P(Def\vert B)P(B)
                   +P(Def\vert C)P(C)
                }
            = &\\&
            =   \Bigl(
                    \frac{
                        P(Def\vert C)P(C)
                       +P(Def\vert A)P(A)
                    }{
                        P(Def\vert B)P(B)
                    }
                   +1
                \Bigr)^{-1}
            = &\\&
            =   \Bigl(
                    \frac{
                        4\%*65\%
                       +5\%*15\%
                    }{
                        7\%*25\%
                    }
                   +1
                \Bigr)^{-1}
            \cong \num{34.313725490196078}\%
        &
    \end{flalign*}

\end{questionBox}


\section*{Funções de Probabilidade}

\begin{sectionBox}1{Funções Discretas}
    
\end{sectionBox}

\begin{questionBox}1{Lançamento de uma modea just 2 vezes}
    
    \begin{table}[H]\centering
        \begin{tabular}{c | *{4}{ c }}
        
        \hline
            w & FF & FC & CF & CC
     \\ \hline
            X(w) & 2 & 1 & 1 & 0
     \\ \hline
            \multicolumn{5}{c}{\textit{X}: Numero de caras ser igual a 2}
        
        \end{tabular}
    \end{table}
    
    \begin{flalign*}
        &
        \begin{array}{c}
            P(X=0) = P(CC) = (1/2)^2 = 1/4
        \\  P(X=2) = P(FF) = (1/2)^2 = 1/4
        \\  P(X=3) = P(CF) + P(FC) = 2*(1/4) = 1/2
        \end{array}
        &
    \end{flalign*}

\end{questionBox}

\begin{questionBox}2{Função de probabilidade da variável \textit{X}}
    
    \begin{flalign*}
        &
        X
        \begin{cases}
            \begin{matrix}
                0 & 1 & 2
            \\  1/4 & 1/2 & 1/4
            \end{matrix}
        \end{cases}
        &
    \end{flalign*}
    
\end{questionBox}

\begin{questionBox}3{}
    
    \begin{flalign*}
        &
        P(X>0) = 1-P(X\leq 0) = 1-1/4 = 3/4
        &
    \end{flalign*}
    
\end{questionBox}

\begin{questionBox}1{Lançamento de dado}
    
    \begin{table}[H]\centering
        \begin{tabular}{c *{5}{ c| }c}
        
        \hline
            w & 1 & 2 & 3 & 4 & 5 & 6
     \\ \hline
            Y(w) & 1/6 & 1/6 & 1/6 & 1/6 & 1/6 & 1/6
     \\ \hline
            \multicolumn{7}{c}{\textit{X}: Numero de pontos no lançamento de um dado}
        
        \end{tabular}
    \end{table}
    
\end{questionBox}

\begin{sectionBox}1{Função Distribuição}
    
    \begin{BM}
        F(x) = P(X\leq x)
    \end{BM}

    Probabilidades aculmuladas
    
    \begin{sectionBox}*2{Propríedades}
        
        \begin{itemize}
            \item \( \lim_{x\to \infty} F(x) = 1 \)
            \item \( \lim_{x\to-\infty} F(x) = 0 \)
            \item \( F(x) \) é continua a direita   
            \item \( F(x)\leq F(y) \quad\forall\,\{x,y\}: x\leq y \)
        \end{itemize}
        
    \end{sectionBox}

\end{sectionBox}

\begin{sectionBox}1{Função Probabilidade}
    
    \begin{BM}
        f(x_i) = P(X=x_i) = p_i \geq 0
     \\ \sum_i^{\infty} p_i = 1
    \end{BM}    
    
\end{sectionBox}

\part*{Variáveis continuas}

\begin{sectionBox}1{Função Densidade (Probabilidade)}
    
    \begin{BM}
        P(X\in I) = \int_I f(x)\,\delta x:
        \begin{cases}
            D = \{ a\in\mathbb{R}: P(X=a)>0 \}\neq\emptyset
        \end{cases}
    \end{BM}
    
    \begin{sectionBox}*2{Probabilidade de um Ponto}
        
        \begin{BM}
            P(X=a) = 0
        \end{BM}
        
    \end{sectionBox}
    
\end{sectionBox}

\begin{questionBox}1{}
    
    % \begin{tikz}
    %     \begin{axis}[]
    %         \plot { (0,1), (1,1), (1,0), (2,0) };
    %     \end{axis}
    % \end{tikz}
    
    \begin{flalign*}
        &
        P(1/2\leq X \leq 3/4) 
        = \int_{1/2}^{3/4}f(x)\,\delta x
        = 3/4-1/2 = 1/4
        \\
        P(1/2\leq X)
        = \int_{1/2}^{\infty} f(x)\,\delta x
        = \int_{1/2}^{1} f(x)\,\delta x
        = 1-1/2 = 1/2
        &
    \end{flalign*}
    
\end{questionBox}

\begin{questionBox}{}
    
    \begin{BM}
        f(x)=
        \begin{cases}
            k(1-x) & 0<x<1
        \\  0      & cc
        \end{cases}
    \end{BM}
    
    \begin{flalign*}
        &
        \int_{-\infty}^{\infty} f(x)\,\delta x
        = \int_{0}^{1} k(1-x)\,\delta x
        = k\int_{0}^{1} \delta x
        - k\int_{0}^{1} x\,\delta x
        = k(1-0) - k(1/2)
        = k/2 = 1
        \implies
        k = 2
        &
    \end{flalign*}

    \begin{flalign*}
        &
        P(X>0.3|X<0.5)
        = P(0.3<X<0.5) / P(X<0.5)
        = \frac
            {\int_{0.3}^{0.5} f(x)\,\delta x}
            {\int_{-\infty}^{0.5} f(x)\,\delta x}
        = \frac
            {\int_{0.3}^{0.5} 2(1-x)\,\delta x}
            {\int_{0}^{0.5} 2(1-x)\,\delta x}
        = \frac
            {
                2\int_{0.3}^{0.5} \delta x
            -   2\int_{0.3}^{0.5} x\,\delta x
            }
            {
                2\int_{0}^{0.5} \delta x
            -   2\int_{0}^{0.5} x\,\delta x
            }
        = \frac
            {
                2(0.5-0.3)
            -   2(0.25-0.09)
            }
            {
                2(0.5)
            -   2(0.25)
            }
        = 0.16
        &
    \end{flalign*}

\end{questionBox}

\end{document}

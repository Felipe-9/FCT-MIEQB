% !TEX root = ./IPEIO-Exercicios_Resolução.5.tex
\providecommand\mainfilename{"./IPEIO-Exercicios_Resolução.tex"}
\providecommand \subfilename{}
\renewcommand   \subfilename{"./IPEIO-Exercicios_Resolução.5.tex"}
\documentclass[\mainfilename]{subfiles}

% \tikzset{external/force remake=true} % - remake all

\begin{document}

% \graphicspath{{\subfix{./.build/figures/IPEIO-Exercicios_Resolução.5}}}

\mymakesubfile{5}
[IPEIO]
{Exercicios: Estimação Pontual} % Subfile Title
{Estimação Pontual} % Part Title

\setcounter{question}{1}
\begin{questionBox}1{ % Q2
    Considere que se seleccionou uma amostra aleatória (\(X_1, X_2,\dots, X_n\)) de uma população com valor médio \(\mu\) e variância \(\sigma^2\).
} % Q2
    \begin{questionBox}2{ % Q2.1
        Mostre que \(\overline{X}\) é estimador centrado e consistente da média populacional.
    } % Q2.1
        \begin{BM}
            \overline{X}=n^{-1}\sum_{i=1}^{n}{X_i}
        \end{BM}
        \begin{flalign*}
            &
                \esperanca(\overline{X})
                =\esperanca\left(
                    n^{-1}\sum_{i=1}^{n}{X_i}
                \right)
                =n^{-1}\esperanca\left(
                    \sum_{i=1}^{n}{X_i}
                \right)
                =n^{-1}\sum_{i=1}^{n}{\esperanca(X_i)}
                =n^{-1}n\mu
                =\mu
                ; &\\[3ex]&
                \variancia(\overline{X})
                = \variancia\left(n^{-1}\sum_{i=1}^{n}{X_i}\right)
                = n^{-2}\variancia\left(\sum_{i=1}^{n}{X_i}\right)
                % = &\\&
                = n^{-2}\sum_{i=1}^{n}{\variancia(X_i)}
                % = &\\&
                = n^{-2}\sum_{i=1}^{n}{\sigma^2}
                = &\\&
                = \sigma^2/n
                \implies &\\&
                \implies
                \lim_{n\to\infty}{EQB(\overline{X})}
                =\lim_{n\to\infty}{\variancia(\overline{X})}
                =\lim_{n\to\infty}{\sigma^2/n}=0
            &
        \end{flalign*}
        \(\overline{X}\) é consistente em média quadrática de \(\mu\)
    \end{questionBox}

    \begin{questionBox}2{ % Q2.2
        Mostre que \(\hat{\theta}_1,\ \hat{\theta_2}\) também são estimadores centrados de \(\mu\). Qual é melhor? São consistentes?
    } % Q2.2
        \begin{BM}
            \hat{\theta}_1
            = \frac{X_1+X_n}{2}
            \qquad
            \hat{\theta}_2
            = \frac{2\,X_1+3\,X_2+5\,X_3}{10}
        \end{BM}
        \begin{flalign*}
            &
                % --------------------------------- \theta_1 --------------------------------- %
                % Esperanca
                \esperanca(\hat{\theta}_1)
                =\esperanca\left(\frac{X_1+X_n}{2}\right)
                =2^{-1}\esperanca\left(X_1+X_n\right)
                =2^{-1}(\esperanca(X_1)+\esperanca(X_n))
                =2^{-1}(2\mu)
                = &\\&
                = \mu
                ;&\\[3ex]&
                % Variancia
                \variancia(\hat{\theta}_1)
                = \variancia\left(\frac{X_1+X_n}{2}\right)
                = 4^{-1}\variancia\left(X_1+X_n\right)
                = 4^{-1}(\variancia(X_1)+\variancia(X_n))
                = &\\&
                = 4^{-1}(\sigma^2+\sigma^2)
                = \sigma^2/2
                % EQM
                ;&\\[3ex]&
                \lim_{n\to\infty}{EQM(\hat{\theta}_1)}
                =\lim_{n\to\infty}{\variancia(\hat{\theta}_1)}
                =\lim_{n\to\infty}{\sigma^2/2}
                = \sigma^2/2
                \neq 0
                ;&\\[6ex]&
                % --------------------------------- \theta_2 --------------------------------- %
                % Esperanca
                \esperanca(\hat{\theta}_2)
                =\esperanca\left(
                    \frac{2\,X_1+3\,X_2+5\,X_3}{10}
                \right)
                =10^{-1}\esperanca\left(
                    2\,X_1+3\,X_2+5\,X_3
                \right)
                = &\\&
                =10^{-1}\left(
                     2\,\esperanca(X_1)
                    +3\,\esperanca(X_2)
                    +5\,\esperanca(X_3)
                \right)
                =10^{-1}\left(
                     2\,\mu
                    +3\,\mu
                    +5\,\mu
                \right)
                = &\\&
                =10^{-1}\left(
                    10\,\mu
                \right)
                = \mu
                ;&\\[3ex]&
                % Variancia
                \variancia(\hat{\theta}_2)
                = \variancia\left(
                    \frac{2\,X_1+3\,X_2+5\,X_3}{10}
                \right)
                = 10^{-2}\variancia\left(
                    2\,X_1
                    +3\,X_2
                    +5\,X_3
                \right)
                = &\\&
                = 10^{-2}\left(
                     2^2\,\variancia(X_1)
                    +3^2\,\variancia(X_2)
                    +5^2\,\variancia(X_3)
                \right)
                = 10^{-2}\left(
                     2^2\,\sigma^2
                    +3^2\,\sigma^2
                    +5^2\,\sigma^2
                \right)
                = &\\&
                = 0.38\,\sigma^2
                ;&\\[3ex]&
                % EQM
                \lim_{n\to\infty}{EQM(\hat{\theta}_2)}
                =\lim_{n\to\infty}{\variancia(\hat{\theta}_2)}
                =\lim_{n\to\infty}{0.38\,\sigma^2}
                = 0.38\,\sigma^2 \not 0
                ;&\\[6ex]&
                \therefore
                EQM(\hat{\theta}_2) < EQM(\hat{\theta}_1)
            &
        \end{flalign*}
        \(\hat{\theta}_2\) é mais preciso
    \end{questionBox}

    \begin{questionBox}2{ % Q2.3
        Mostre que \(\bar{X}^2\) não é estimador centrado de \(μ^2\).
    } % Q2.3
        \begin{flalign*}
            &
                E(\bar{X}^2)
                = V(\bar{X})+E^2(\bar{X})
                = \sigma^2/n+\mu^2
                \neq \mu
            &
        \end{flalign*}
    \end{questionBox}
\end{questionBox}

\begin{questionBox}1{ % Q3
    Suponha que seleccionou uma amostra aleatória (\(X_1,X_2,\dots,X_n\)) de uma população com distribuição \(U(0,\theta)\), isto é, com função densidade:
} % Q3
    \begin{BM}
        f(x)=\begin{cases}
            \theta^{-1}\qquad& 0\leq x\leq \theta
            \\
            0\qquad& c.c.
        \end{cases}
    \end{BM}

    \begin{flalign*}
        &
            E(X)
            = \int_{0}^{\theta}{x\,\theta^{-1}\odif{\theta}}
            = \theta^{-1}\,\theta^2/2
            = \theta/2
            ; &\\[1.5ex]&
            \variancia(X)
            = E(X^2) - E^2(X)
            = \int_{0}^{\theta}{x^2\,\theta^{-1}\odif{\theta}} 
            - \theta^2/4
            = \theta^{-1}\,\theta^3/3 
            - \theta^2/4
            = \theta^2/12
        &
    \end{flalign*}

    \begin{questionBox}2{ % Q3.1
        Verifique se o estimador \(2\,\overline{X}\) é centrado e consistente.
    } % Q3.1
        \begin{flalign*}
            &
                % Esperanca
                \esperanca(2\,\overline{X})
                = 2\,\esperanca(\overline{X})
                = 2\,\esperanca(X)
                = 2\,\theta/2
                = \theta
                ; &\\[3ex]&
                % EQM
                \lim_{n\to\infty}{EQM(2\,\overline{X})}
                = \lim_{n\to\infty}{V(2\,\overline{X})}
                % = &\\&
                = \lim_{n\to\infty}{4\,V(\overline{X})}
                = \lim_{n\to\infty}{4\,V(X)/n}
                = &\\&
                = \lim_{n\to\infty}{4\,(\theta^2/12)/n}
                = \lim_{n\to\infty}{\theta^2/3n}
                = 0
            &
        \end{flalign*}
        \(2\,\overline{X}\) é estimador consistente em média quadrática de \(\theta\)
    \end{questionBox}

    \begin{questionBox}2{ % Q3.2
        Dada a amostra (\(1.215, 1.580, 0.726, 2.843, 3.394, 0.612, 2.621, 1.181, 2.930, 0.317\)), estime o valor de \(\theta\). Nota:\( \sum{x_i} = 17.42\).
    } % Q3.2
        \begin{flalign*}
            &
                \hat{\theta}
                = 2\,\overline{x} 
                = 2*17.42/10
                = 3.484
            &
        \end{flalign*}
    \end{questionBox}
\end{questionBox}

\end{document}
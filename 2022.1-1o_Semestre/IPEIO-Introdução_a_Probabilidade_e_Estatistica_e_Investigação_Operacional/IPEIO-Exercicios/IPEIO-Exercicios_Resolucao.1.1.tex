% !TEX root = ./IPEIO-Exercicios_Resolução.1.tex
\providecommand\mainfilename{"./IPEIO-Exercicios_Resolução.tex"}
\providecommand \subfilename{}
\renewcommand   \subfilename{"./IPEIO-Exercicios_Resolução.1.tex"}
\documentclass[\mainfilename]{subfiles}

% \tikzset{external/force remake=true} % - remake all

\begin{document}

% \graphicspath{{\subfix{./.build/figures/}}}

\mymakesubfile{1}
[IPEIO]
{Exercicios PE - Introdução à Teoria da Probabilidade} % Subfile Title
{Introdução à Teoria da Probabilidade} % Part Title



\setcounter{question}{17}
\begin{questionBox}1{ % Q18
    Considere os acontecimentos A e B de um espaço de resultados tais que \(P(A\cup B)=0.8\), e \(P(A-B)=0.3\). Qual o valor da \(P(B)\)
} % Q18
    \begin{flalign*}
        &
            P(B) = P(B\cup A) - P(A-B) = 0.8-0.3 = 0.5
        &
    \end{flalign*}
\end{questionBox}

\setcounter{question}{26}
\begin{questionBox}1{ % Q27
    Um aluno conhece bem 60\% da matéria dada. Num exame com cinco perguntas, sorteadas ao acaso, sobre toda a matéria, qual a probabilidade de vir a responder correctamente a duas perguntas?
} % Q27
    \begin{flalign*}
        &
            P(X=2) 
            = \binom{5}{2}*0.6^n*(1-0.6)^{5-2}
        &
    \end{flalign*}
\end{questionBox}

\end{document}
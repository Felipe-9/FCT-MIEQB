% !TEX root = ./IPEIO-Exercicios_Resolução.2.tex
\providecommand\mainfilename{"./IPEIO-Exercicios_Resolução.tex"}
\providecommand \subfilename{}
\renewcommand   \subfilename{"./IPEIO-Exercicios_Resolução.2.tex"}
\documentclass[\mainfilename]{subfiles}

% \tikzset{external/force remake=true} % - remake all

\begin{document}

% \graphicspath{{\subfix{./.build/figures/}}}

\mymakesubfile{2}
[IPEIO]
{Exercicios PE - Variáveis Aleatórias} % Subfile Title
{Variáveis Aleatórias} % Part Title

% ---------------------------------------------------------------------------- %
%                             Por na lista anterior                            %
% ---------------------------------------------------------------------------- %
% \setcounter{question}{28}
% \begin{questionBox}1{ % Q1
%     (Teste de P.E. 2006/07) Uma urna tem oito moedas, seis honestas e duas viciadas. O resultado do lançamento de uma moeda viciada é sempre “cara”.
% } % Q1
%     \begin{questionBox}2{ % Q29.1
%         Escolhendo duas das oito moedas disponíveis, ao acaso e sem reposição, qual a probabilidade de seleccionar as duas moedas viciadas.
%     } % Q29.1
%         \begin{flalign*}
%             &
%                 \frac{2}{8}
%                 * \frac{1}{7}
%             &
%         \end{flalign*}
%     \end{questionBox}
    
%     \begin{questionBox}2{ % Q29.2
%         Escolhendo uma moeda ao acaso, qual a probabilidade de obter três caras em três lançamentos sucessivos dessa moeda?
%     } % Q29.1
%         \begin{flalign*}
%             &
%                 P(T)
%                 = P(3\cap H)
%                 + P(3\cap \bar{H})
%                 = P(3\vert H)\,P(H)
%                 + P(3\vert \bar{H})\,P(\bar{H})
%                 = (1/2)^3*6/8
%                 + 1*1/4
%                 = 11/32
%             &
%         \end{flalign*}
%     \end{questionBox}

%     \begin{questionBox}2{ % Q29.3
%         Se em três lançamentos, da mesma moeda, o resultado foi sempre “cara”, qual a probabilidade de ter escolhido a moeda viciada?
%     } % Q29.1
%         \begin{flalign*}
%             &
%                 P(\bar{H}\vert T)
%                 = \frac{P(T\cap \bar{H})}{P(T)}
%                 = \frac{P(T\vert \bar{H})\,P(\bar{H})}{P(T)}
%             &
%         \end{flalign*}
%     \end{questionBox}
% \end{questionBox}


\begin{questionBox}1{ % Q1
    A variável aleatória (v.a.) X representa o número de doentes com gripe que procuram, por dia, o Dr. Remédios. Em 50\% dos dias, pelo menos 2 pacientes com gripe procuram o Dr. Remédios. A sua função de probabilidade é dada por:
} % Q1
    % \begin{center}
        \begin{BM}
            x
            \left\{
                \begin{aligned}
                    0 \quad & 1 \quad & 2 \quad & 3
                    \\
                    p \quad & 0.2 \quad & q \quad & 0.3
                \end{aligned}
            \right\}
        \end{BM}
    % \end{center}

    \begin{questionBox}2{ % Q1.1
        Determine p e q.
    } % Q1.1
        \begin{flalign*}
            &
                q
                = P(X=2)
                = P(X\geq 3) - P(X=3)
                = 0.5 - 0.3
                = 0.2
                &\\&
                p
                = P(X=0)
                = 1-P(\bar{X=0})
                = 1-P(X>0)
                = &\\&
                = 1
                - P(X=1)
                - P(X=2)
                - P(X=3)
                = 1-0.2-0.2-0.3
                = 0.3
            &
        \end{flalign*}
    \end{questionBox}

    \begin{questionBox}2{ % Q1.2
        Determine a função de probabilidade das v.a.'s \(Y = 40\,X\) e \(W = \max(X, 1)\).
    } % Q1.2
        \begin{BM}
            Y
            = \left\{
                \begin{Bmatrix}
                    0 & 40 & 80 & 120
                    \\
                    0.3 & 0.2 & 0.2 & 0.3
                \end{Bmatrix}
            \right\}
            \qquad
            W
            = \begin{Bmatrix}
                1 & 2 & 3
                \\
                0.5 & 0.2 & 0.3
            \end{Bmatrix}
        \end{BM}
    \end{questionBox}
\end{questionBox}


\begin{questionBox}1{ % Q2
    A v.a. \textit{X} representa o número de pontos que saem no lançamento de um determinado dado. A sua função de distribuição segue-se:
} % Q2
    \begin{BM}
        F(x)
        =
        \begin{Bmatrix}
            0 & x<1
            \\ 1/6 & 1\leq x<2
            \\ 1/4 & 2\leq x<4
            \\ 1/2 & 4\leq x<5
            \\ 7/12 & 5\leq x<6
            \\ 1 & x\geq6
        \end{Bmatrix}
    \end{BM}

    \begin{questionBox}2{ % Q2.1
        Calcule as seguintes probabilidades, usando a função de distribuição:
    } % Q2.1
        \subsubquestion{A probabilidade de o número de pontos saídos ser no máximo 3.}
        \begin{flalign*}
            &
                P(X\leq 3) = F(3)
            &
        \end{flalign*}

        \subsubquestion{}
        \begin{flalign*}
            &
                P(1<X\leq 2)
                = P(X\leq 2)
                - P(X\leq 1)
                = F(2)-F(1)
            &
        \end{flalign*}

        \subsubquestion{}
        \begin{flalign*}
            &
                P(2\leq X<6)
                = P(X<6)
                - P(X<2)
                = F(6^-)
                - F(2^-)
                = 7/12
                - 1/6
                = 5/12
            &
        \end{flalign*}
    \end{questionBox}

    \begin{questionBox}2{ % Q2.2
        Determine a função de probabilidade de X e confirme os resultados acima obtidos.
    } % Q2.2

        \begin{flalign*}
            &
                P(X=x)
                = P(X\leq x) - P(X<x)
                = F(X) - F(X^-)
                \implies &\\&
                \implies
                \left\{
                    \begin{aligned}
                        &  P(X=1) = 1/6 - 0 =& 1/6
                        \\ & P(X=2) = 1/4 - 1/6 =& 1/12
                        \\ & P(X=3) = 1/4 - 1/4 =& 0
                        \\ & P(X=4) = 1/2 - 1/4 =& 1/4
                        \\ & P(X=5) = 7/12 - 1/2 =& 1/12
                        \\ & P(X=6) = 1 - 7/12 =& 5/12
                    \end{aligned}
                \right\}
            &
        \end{flalign*}

        \begin{BM}
            X
            = \begin{Bmatrix}
                1 & 2 & 4 & 5 & 6
                \\
                1/6 & 1/12 & 1/4 & 1/12 & 5/12
            \end{Bmatrix}
        \end{BM}
    \end{questionBox}

    \begin{questionBox}2{ % Q2.3
        Pode afirmar que o dado é equilibrado? Justifique.
    } % Q2.3
        N, os lados tem diff probs
    \end{questionBox}
    
    \begin{questionBox}2{ % Q2.4
        Sabendo que o número de pontos saído é pelo menos 4, calcule a probabilidade de saírem 6 pontos.
    } % Q2.4
        \begin{flalign*}
            &
                P(X=6\vert X\geq 4)
                = \frac{P(X=6\cap X\geq 4)}{P(X\geq 4)}
                = \frac{P(X=6)}{P(X\geq 4)}
                = \frac{F(6)-F(6')}{F(4^-)}
                = &\\&
                = \frac{1-7/12}{1/4}
                = 5/9
            &
        \end{flalign*}
    \end{questionBox}
\end{questionBox}

\setcounter{question}{5}

\begin{questionBox}1{ % Q6
    Seja X uma v.a. com a seguinte função densidade probabilidade:
} % Q6
    \begin{BM}
        f(x)
        = \begin{cases}
            4\,x &\quad 0<x<k
            \\ 0 &\quad c.c.
        \end{cases}
    \end{BM}

    \begin{questionBox}2{ % Q6.1
        k
    } % Q6.1
        \begin{flalign*}
            &
                \int_{-\infty}^{infty}{
                    f(x)\,\odif{x}
                }
                = \int_{0}^{k}{
                    f(x)\,\odif{x}
                }
                = \int_{0}^{k}{
                    4\,x\,\odif{x}
                }
                = 4\left(x^2\right)\big\vert_{0}^{k}
                = 4\,k^2
                = 1
                \implies &\\&
                \implies
                k = 1/\sqrt{2}
            &
        \end{flalign*}
    \end{questionBox}

    \begin{questionBox}2{ % Q6.2
        Calcule \(P(1/4 \leq X \leq 1/3)\), a mediana e o quantil de ordem 0.95.
    } % Q6.2
        \subsubquestion{\(P(1/4\leq X\leq 1/3)\)}
        \begin{flalign*}
            &
                P(1/4\leq X\leq 1/3)
                = \int_{1/4}^{1/3}{f(x)\odif{x}}
                = \int_{1/4}^{1/3}{4\,x\odif{x}}
                = 4\left(x^2\right)\big\vert_{1/4}^{1/3}
                = &\\&
                = 4\left(3^{-2}-4^{-2}\right)
                = 4\left(3^{-2}-4 ^{-2}\right)
                = \num{0.194444444444444}
            &
        \end{flalign*}
        \subsubquestion{Mediana}
        \begin{flalign*}
            &
                P(X\leq m_e)
                = \int_{0}^{m_e}{f(x)\odif{x}}
                = 4(m_e^2)
                =1/2
                \implies
                m_e 
                = \sqrt{(1/2)/4}
                = \num{0.353553390593274}
            &
        \end{flalign*}
        \subsubquestion{}
        \begin{flalign*}
            &
                P(X\leq q_{95})
                = \int_{0}^{q_{95}}{f(x)\odif{x}}
                = 0.95
            &
        \end{flalign*}
    \end{questionBox}

\end{questionBox}

\setcounter{question}{12}

\begin{questionBox}1{ % Q13
    Seja \textit{X} uma v.a. com a seguinte função de distribuição:
} % Q13
    \begin{BM}
        F(x)=
        \begin{cases}
            0, \qquad& x<0
            \\
            1-(x+1)\,e^{-x}, \qquad& x\geq0
        \end{cases}
    \end{BM}

    \begin{questionBox}2{ % Q1
        Calcule \(P(X<1)\)
    } % Q1
    \end{questionBox}

    \begin{questionBox}2{ % Q2
        Calcule \(P(X<2\vert X\geq 1)\)
    } % Q2
    \end{questionBox}

    \begin{questionBox}2{ % Q3
        Determine a função densidade probabilidade de \textit{X}.
    } % Q3
    \end{questionBox}

    \begin{questionBox}2{ % Q4
        Dado que \(\int_{0}^{\infty}{x^2\,e^{-x}\,\odif{x}} = 2\,\int_{0}^{\infty}{x\,e^{-x}\,\odif{x}}\) e \(\int_{0}^{\infty}{x^3\,e^{-x}\,\odif{x}} = 3\,\int_{0}^{\infty}{x^2\,e^{-x}\,\odif{x}}\), determine \textit{E(X)} e \textit{V(X)}.
    } % Q4
        \subsubquestion{\textit{E(X)}}
        \begin{flalign*}
            &
                E(X) 
                = \int_{\mathbb{R}}{x\,f(x)\,\odif{x}}
                = \int_{0}^{+\infty}{x\,x\,e^{-x}\,\odif{x}}
                = \int_{0}^{+\infty}{x^{2}\,e^{-x}\,\odif{x}}
                = &\\&
                = 2\,\int_{0}^{+\infty}{x\,e^{-x}\,\odif{x}}
                = 2\,\adif{x\,e^{-x}\,\odif{x}}\big\vert_{0}^{+\infty}
                = 2\,\left(
                    -0+\frac{0+1}{e^{0}}
                \right)
                = 2
            &
        \end{flalign*}

        \subsubquestion{\textit{V(X)}}
        \begin{flalign*}
            &
                V(X)
                = - E^2(X) + E(X^2)
                = - 2^2 + \int_{\mathbb{R}}{x^2\,f(x)\,\odif{x}}
                = - 2^2 + \int_{0}^{\infty}{x^3\,e^{-x}\,\odif{x}}
                = &\\&
                = - 2^2 + 3\,\int_{0}^{\infty}{x^2\,e^{-x}\,\odif{x}}
                = - 2^2 + 3*2
                = 2
            &
        \end{flalign*}
    \end{questionBox}

\end{questionBox}

\end{document}
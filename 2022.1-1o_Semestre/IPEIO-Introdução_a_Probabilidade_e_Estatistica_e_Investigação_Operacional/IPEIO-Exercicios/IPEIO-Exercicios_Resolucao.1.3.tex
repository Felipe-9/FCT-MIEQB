% !TEX root = ./IPEIO-Exercicios_Resolução.3.tex
\providecommand\mainfilename{"./IPEIO-Exercicios_Resolução.tex"}
\providecommand \subfilename{}
\renewcommand   \subfilename{"./IPEIO-Exercicios_Resolução.3.tex"}
\documentclass[\mainfilename]{subfiles}

% \tikzset{external/force remake=true} % - remake all

\begin{document}

% \graphicspath{{\subfix{./.build/figures/}}}

\mymakesubfile{3}
[IPEIO]
{Exercicios PE - Principais Distribuições} % Subfile Title
{Principais Distribuições} % Part Title

\setcounter{question}{2}

\begin{questionBox}1{ % Q3
    O senhor Sousa tem uma empresa que compra e vende selos e outros artigos de colecionismo. Ele guarda 20 selos dentro de uma bolsa preta, estando ainda cada um deles metido num envelope opaco. 6 destes selos valem 100 euros cada um e os restantes nada valem. O senhor Sousa, para promover a venda, cobra 20 euros por cada selo, mas não permitindo que o cliente veja o conteúdo do envelope. Suponha que um cliente compra 5 selos.
} % Q3
    \begin{questionBox}2{ % Q3.1
        Qual a probabilidade dos cinco selos nada valerem?
    } % Q3.1
        \begin{flalign*}
            &
                P(X=0)
                = \frac{
                    \binom{6}{0}
                    \binom{20-6}{5}
                }{
                    \binom{20}{5}
                }
                = \frac{
                    \frac{6!}{0!*6!}
                    \binom{20-6}{5}
                }{
                    \binom{20}{5}
                }
            &
        \end{flalign*}
    \end{questionBox}

    \begin{questionBox}2{ % Q3.2
        Qual a probabilidade do cliente não perder nem ganhar dinheiro com a compra?
    } % Q3.2
        \begin{flalign*}
            &
                P(X=1)
                = \frac{
                    \binom{6}{1}
                    \binom{14}{4}
                }{
                    \binom{20}{5}
                }
            &
        \end{flalign*}
    \end{questionBox}
\end{questionBox}

\setcounter{question}{4}

\begin{questionBox}1{ % Q5
    Determinado exame é constituído por 5 questões de escolha múltipla, em que cada questão tem 4 opções de resposta possíveis - apenas uma sendo a correcta. Supondo que um aluno que vai fazer o exame responde a tudo ao acaso
} % Q5
    \begin{questionBox}2{ % Q5.1
        qual é a probabilidade de ele acertar a mais de metade das questões?
    } % Q5.1
        \begin{flalign*}
            &
                P(X\geq3)
                = \sum_{n=3}^{5}{P(X=n)}
                = \sum_{n=3}^{5}{\binom{5}{n}\,0.25^{n}\,0.75^{5-n}}
            &
        \end{flalign*}
    \end{questionBox}

    \begin{questionBox}2{ % Q5.2
        Qual é o número médio de respostas correctas?
    } % Q5.2
        \begin{flalign*}
            &
                E(X)
                = n*p 
                = 5*0.25
                = 1.25
            &
        \end{flalign*}
    \end{questionBox}

    \begin{questionBox}2{ % Q5.3
        E o seu desvio padrão?
    } % Q5.3
        \begin{flalign*}
            &
                \rho
                = \sqrt{V(X)}
                = \sqrt{n*p*(1-p)}
                = 0.9682
            &
        \end{flalign*}
    \end{questionBox}
\end{questionBox}

\setcounter{question}{9}

\begin{questionBox}1{ % Q10
    Suponha que, o número de pessoas que utilizam uma caixa multibanco é um processo de Poisson de taxa \(\lambda = 10/\unit{\hour}\). Calcule;
} % Q10
    \begin{questionBox}2{ % Q10.1
        a probabilidade de não ir ninguém à caixa multibanco durante 1 hora.
    } % Q10.1
        \begin{flalign*}
            &
                P(X=0)
                = \frac{e^{-10}\,10^{0}}{0!}
                = e^{-10}
                \cong \num{4.539992976248485e-5}
            &
        \end{flalign*}
    \end{questionBox}

    \begin{questionBox}2{ % Q10.2
        a probabilidade de irem 20 pessoas à referida caixa durante 4 horas.
    } % Q10.2
        \begin{flalign*}
            &
                P(X=20)
                = \frac{e^{-40}\,40^{20}}{20!}
                = \num{1.919976590469239e-4}
            &
        \end{flalign*}
    \end{questionBox}

    \begin{questionBox}2{ % Q10.3
        O número médio de visitas à caixa multibanco durante 4 horas e o seu coeficiente de variação.
    } % Q10.3
    \end{questionBox}
\end{questionBox}

\end{document}
% !TEX root = ./IPEIO-Exercicios_Resolução.7.tex
\providecommand\mainfilename{"./IPEIO-Exercicios_Resolução.tex"}
\providecommand \subfilename{}
\renewcommand   \subfilename{"./IPEIO-Exercicios_Resolução.7.tex"}
\documentclass[\mainfilename]{subfiles}

% \tikzset{external/force remake=true} % - remake all

\begin{document}

% \graphicspath{{\subfix{./.build/figures/}}}

\mymakesubfile{7}
[IPEIO]
{Exercicios: Testes de Hipóteses} % Subfile Title
{Testes de Hipóteses} % Part Title

\begin{questionBox}1{ % Q1
    Uma fábrica de gelados afirma que a procura do gelado de chocolate no verão, por dia e em euros, é uma v.a. Normalmente distribuída com valor médio 200 EUR e desvio padrão 40 EUR. Numa amostra aleatória constituída por 10 dias seleccionados ao acaso do período de verão verificou-se que \(\bar{x}\cong 216\).
} % Q1
    \begin{questionBox}2{ % Q1.1
        Teste, ao nível de significância 5\%, se de facto o consumo médio de gelado de chocolate no verão é de 200 EUR por dia.
    } % Q1.1
        \begin{flalign*}
            &   
                Z = \frac{\bar{X}-\mu_0}{\sigma/\sqrt{n}}
                \sim N(0,1)
                \begin{cases}
                    H_0: \mu_0=200
                    \\
                    H_1: \mu_1\neq200
                \end{cases}
                &\\&
                \frac{\bar{x}-\mu}{\sigma/\sqrt{n}}
                =\frac{216-200}{40/\sqrt{10}}
                \cong
                \num{1.264911064067352}
                \not\in
                \left[
                    -z_{\alpha/2},
                    z_{\alpha/2}
                \right]^c
                = &\\&
                = \left[
                    -z_{5\%/2},
                    z_{5\%/2}
                \right]^c
                = \left[
                    -z_{0.025},
                    z_{0.025}
                \right]^c
                = \left[
                    -1.96,1.96
                \right]^c
            &
        \end{flalign*}
    \end{questionBox}

    \begin{questionBox}2{ % Q1.2
        Teste, ao ao nível de significância 5\%, se de facto o consumo médio de gelado de chocolate no verão é menor do que e200 por dia.
    } % Q1.2
        \begin{flalign*}
            &   
                Z = \frac{\bar{X}-\mu_0}{\sigma/\sqrt{n}}
                \sim N(0,1)
                \begin{cases}
                    H_0: \mu_0\geq 200
                    \\
                    H_1: \mu_1<200
                \end{cases}
                &\\&
                \frac{\bar{x}-\mu}{\sigma/\sqrt{n}}
                =\frac{216-200}{40/\sqrt{10}}
                \cong
                \num{1.264911064067352}
                \not\in
                \left[
                    -z_{\alpha},
                    \infty
                \right]^c
                = \left[
                    -z_{0.05},
                    \infty
                \right]^c
                = \left[
                    -1.645,
                    \infty
                \right]^c
            &
        \end{flalign*}
    \end{questionBox}
    \begin{questionBox}2{ % Q1.3
        Qual a potência do teste, da alínea anterior, se \(\mu = 190\).
    } % Q1.3
        \begin{flalign*}
            &
                1-\beta
                = P(\text{Rejeitar }H_0\vert \mu=190)
            &
        \end{flalign*}
    \end{questionBox}
    \begin{questionBox}2{ % Q1.4
        Resolva as duas primeiras alíneas usando o p-valor.
    } % Q1.4
        % ------------------------------------ (b) ----------------------------------- %
        \subsubquestion{(b)}
        \begin{flalign*}
            &
                \text{p-valor}
                = P(Z<z_{observado}\vert H_0) 
                = P(Z< 1.26 \vert H_0) 
                = &\\&
                = \Phi(1.26)\cong 0.8962 > 0.05 = \alpha
                &\\&
                \therefore \text{ não rejeita}
            &
        \end{flalign*}
        % ------------------------------------ (a) ----------------------------------- %
        \subsubquestion{(a)}
        \begin{flalign*}
            &
                \text{p-valor}
                = 2\,P(Z>z_{obs}\vert H_0)
                = 2\,P(Z>1.26\vert H_0)
                % = &\\&
                = 2\,(1-P(Z<1.26\vert H_0))
                = &\\&
                = 2\,(1-\Phi(1.26))
                = 2\,(1-0.8962)
                = 0.2076 
                = 0.05 > \alpha
                &\\&
                \therefore \text{ não rejeita}
            &
        \end{flalign*}
        \paragraph*{Nota:} Como em (a) o \alpha{} possui duas regiões de rejeição de \(\alpha/2\) fazemos o p-valor ser o dobro da probabilidade para comparar com o valor de \alpha{} (sem dividir por 2), e como se refere ao valor da direita pegamos o valor da complementar da tabela da normal.
    \end{questionBox}
\end{questionBox}

\end{document}
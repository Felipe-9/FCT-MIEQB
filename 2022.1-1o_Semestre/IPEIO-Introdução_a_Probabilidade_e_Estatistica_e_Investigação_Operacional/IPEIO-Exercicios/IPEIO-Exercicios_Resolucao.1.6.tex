% !TEX root = ./IPEIO-Exercicios_Resolução.6.tex
\providecommand\mainfilename{"./IPEIO-Exercicios_Resolução.tex"}
\providecommand \subfilename{}
\renewcommand   \subfilename{"./IPEIO-Exercicios_Resolução.6.tex"}
\documentclass[\mainfilename]{subfiles}

% \tikzset{external/force remake=true} % - remake all

\begin{document}

% \graphicspath{{\subfix{./.build/figures/IPEIO-Exercicios_Resolução.6}}}

\mymakesubfile{6}
[IPEIO]
{Exercicios: Estimação por intervalo de confiança} % Subfile Title
{Estimação por intervalo de confiança} % Part Title

\begin{questionBox}1{ % Q1
    Para avaliar o peso médio das maçãs produzidas por um determinado agricultor analisaram-se 20 maçãs seleccionadas ao acaso da produção. Estas resultaram num peso médio de \(\overline{x}=320\,\unit{\gram}\). Assuma que os pesos das maçãs têm distribuição Normal com desvio padrão \(\sigma = 20\,\unit{\gram}\).
} % Q1
    \begin{questionBox}2{ % Q1.1
        Construa um intervalo de confiança a 90\% para a média do peso.
    } % Q1.1
        \begin{flalign*}
            &
                Z = \frac{\overline{X}-\mu}{\sigma/\sqrt{n}}\sim\normal(0,1)
                ; &\\&
                z_{\alpha/2}
                = z_{0.05}
                = \Phi^{-1}(0.95) = 1.645
                ; &\\&
                IC_{0.9}(\mu) 
                \cong &\\&
                \cong 
                \myrange*{
                    \overline{x}-\Phi^{-1}(0.95)\frac{\sigma}{\sqrt{n}},
                    \overline{x}+\Phi^{-1}(0.95)\frac{\sigma}{\sqrt{n}}
                }
                \cong &\\&
                \cong \myrange*{
                    320-1.645\frac{20}{\sqrt{20}},
                    320+1.645\frac{20}{\sqrt{20}}
                }
                \cong &\\&
                \cong \myrange*{
                    \num{312.643336354025692},
                    \num{327.356663645974308}
                }
            &
        \end{flalign*}
    \end{questionBox}

    \begin{questionBox}2{ % Q1.2
        Qual deve ser o tamanho da amostra de forma a que a amplitude do correspondente intervalo de confiança a 90\% para a média seja de 1\,\unit{\gram}? E 5\,\unit{\gram}? Comente.
    } % Q1.2

        \subsubquestion{1\,\unit{\gram}}
        \begin{flalign*}
            &
                n:\Delta = 2\,z_{\alpha/2}\frac{\sigma}{\sqrt{n}} \leq 1
                \implies &\\&
                \implies
                n 
                \geq (2\,z_{\alpha/2}\,\sigma)^2
                \cong (2*1.645*20)^2
                = 4329.64
            &
        \end{flalign*}

        \subsubquestion{5\,\unit{\gram}}
        \begin{flalign*}
            &
                n:\Delta = 2\,z_{\alpha/2}\frac{\sigma}{\sqrt{n}} \leq 5
                \implies &\\&
                \implies
                n 
                \geq (2\,z_{\alpha/2}\,\sigma/5)^2
                \cong (2*1.645*20/5)^2
                = 173.1856
            &
        \end{flalign*}
    \end{questionBox}
\end{questionBox}

\setcounter{question}{6}

\begin{questionBox}1{ % Q7
    (Exame de P.E. D - 2008/09) A população das estaturas dos alunos da FCT, em metros, segue uma distribuição Normal. Recolheu-se a seguinte amostra aleatória de estaturas de 40 alunos desta faculdade:
} % Q7
    \begin{center}
        \setlength\tabcolsep{3mm}        % width
        \renewcommand\arraystretch{1.25} % height
        \begin{tabular}{*{10}{c}}
            
            \\\toprule
            
               1.79 & 1.80 & 1.72 & 1.82 & 1.57 & 1.78 & 1.78 & 1.66 & 1.78 & 1.80
            \\ 1.75 & 1.74 & 1.60 & 1.77 & 1.82 & 1.82 & 1.75 & 1.66 & 1.84 & 1.77
            \\ 1.78 & 1.78 & 1.69 & 1.78 & 1.52 & 1.72 & 1.84 & 1.65 & 1.71 & 1.79
            \\ 1.76 & 1.70 & 1.63 & 1.71 & 1.70 & 1.64 & 1.59 & 1.63 & 1.74 & 1.71
            
            \\\bottomrule
            
        \end{tabular}
    \end{center}

    correspondendo a uma média amostral de 1.73 e a um desvio padrão amostral de 0.08.

    \begin{questionBox}2{ % Q7.1
        Indique uma estimativa pontual, com base nesta amostra, para a verdadeira estatura média populacional.
    } % Q7.1
        \begin{flalign*}
            &
                \bar{x}
                = 40^{-1}\,\sum_{i=1}^{40}{x_i} 
                \cong \num{1.72725}
            &
        \end{flalign*}
    \end{questionBox}

    \begin{questionBox}2{ % Q7.2
        Deduza e calcule um intervalo de confiança a 92\% para a estatura média populacional.
    } % Q7.2
        \sisetup{
            % scientific / engineering / input / fixed
            exponent-mode           = engineering,
            exponent-to-prefix      = false,          % 1000 g -> 1 kg
            % exponent-product        = *,             % x * 10^y
            % fixed-exponent          = 0,
            round-mode              = places,        % figures/places/unsertanty/none
            round-precision         = 3,
            % round-minimum           = 0.01, % <x => 0
            % output-exponent-marker  = {\,\mathrm{E}},
        }
        \(\sigma^2 \text{ e } \mu \text{ desconhecido} \therefore t_{n-1}\)
        \begin{flalign*}
            &
                T 
                = \frac{\bar{x}-\mu}{\sigma/\sqrt{n}}
                \sim t_{n-1}
                &\\[1.5ex]&
                s = \left(
                    (40-1)^{-1}
                    \sum_{i=1}^{40}{(x_i-\bar{x})^2}
                \right)^{1/2}
                \cong
                \num{0.07854404284646078}
                &\\&
                t_{39,(1-0.92)/2}
                = t_{39,0.04}
                \cong
                1.685
                \implies &\\&
                \implies
                IC_{0.92}(\mu)
                \cong &\\&
                \cong \myrange*{
                    \mu-t_{39,0.04}\frac{\num{0.07854404284646078}}{\sqrt{40}},
                    \mu+t_{39,0.04}\frac{\num{0.07854404284646078}}{\sqrt{40}}
                }
                \cong &\\&
                \cong \myrange*{
                    1.73-1.685\frac{\num{0.07854404284646078}}{\sqrt{40}},
                    1.73+1.685\frac{\num{0.07854404284646078}}{\sqrt{40}}
                }
                \cong &\\&
                \cong \myrange*{
                    \num{1.709074147431247},
                    \num{1.750925852568753}
                }
            &
        \end{flalign*}
    \end{questionBox}
\end{questionBox}

\end{document}
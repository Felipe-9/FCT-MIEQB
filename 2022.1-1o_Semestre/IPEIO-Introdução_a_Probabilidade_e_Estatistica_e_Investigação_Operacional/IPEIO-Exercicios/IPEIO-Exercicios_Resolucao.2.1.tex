% !TEX root = ./IPEIO-Exercicios_Resolução.2.1.tex
\providecommand\mainfilename{"./IPEIO-Exercicios_Resolução.tex"}
\providecommand \subfilename{}
\renewcommand   \subfilename{"./IPEIO-Exercicios_Resolução.2.1.tex"}
\documentclass[\mainfilename]{subfiles}

% \tikzset{external/force remake=true} % - remake all
\pgfplotsset{height=7cm, width= .8\textwidth}

\begin{document}

\graphicspath{{\subfix{./.build/figures/IPEIO-Exercicios_Resolução.2.1/}}}
\tikzsetexternalprefix{./.build/figures/IPEIO-Exercicios_Resolução.2.1/}

\mymakesubfile{1}
[IO]
{Exercicios Resolução} % Subfile Title
{Exercicios Resolução} % Part Title

\begin{questionBox}1{ % Q1
    Numa quinta de criação de animais pretende-se determinar a quantidade diária de milho, trigo e alfafa que cada animal deve receber de modo a serem satisfeitas certas exigências nutricionais.
} % Q1

    Na tabela seguinte são indicadas as quantidades de nutrientes presentes em cada quilograma de milho, trigo e alfalfa.
    \begin{center}
        \vspace{1ex}
        \begin{tabular}{l *{3}{c}}
            
            \toprule
            
                \multicolumn{1}{c}{Nutrientes:}
                & \multicolumn{1}{c}{Milho\,\unit{\kilo\gram}}
                & \multicolumn{1}{c}{Trigo\,\unit{\kilo\gram}}
                & \multicolumn{1}{c}{Alfafa\,\unit{\kilo\gram}}
            
            \\\midrule
            
                Hidratos de Carbono (\unit{\gram})
                & 90 & 20 & 40
                \\ Proteinas (\unit{\gram})
                & 30 & 80 & 60
                \\ Vitaminas (\unit{\milli\gram})
                & 10 & 20 & 60
            \\\midrule
                Custo por \unit{\kilo\gram} (\unit{u.m.})
                & 42 & 36 & 30
            
            \\\bottomrule
            
        \end{tabular}
        \vspace{0ex}
    \end{center}

    As quantidades diárias que cada animal necessita de hidratos de carbono, proteínas e vitaminas são pelo menos de 200\,\unit{\gram}, 180\,\unit{\gram} e 150\,\unit{\milli\gram}, respetivamente.

    \begin{questionBox}2{ % Q1.1
        Sabendo que se pretende minimizar os custos da alimentação de cada animal, formule este problema como um modelo de Programação Linear.
    } % Q1.1
        \begin{flalign*}
            &
                \min{\text{Custo}}
                = 42\,x_1
                + 36\,x_2
                + 30\,x_3
            &
                x\in\mathbb{R}^3:
                \left\{
                    \begin{aligned}
                        &
                            x_i\geq0\,\forall\,x_i
                        \ldiv {}
                            90\,x_1
                            + 20\,x_2
                            + 40\,x_3
                        \ldiv {}
                            30\,x_1 
                            + 80\,x_2
                            + 60\,x_3
                        \ldiv {}
                            10\,x_1
                            + 20\,x_2
                            + 60\,x_3
                        &
                    \end{aligned}
                \right\}
            &
        \end{flalign*}
    \end{questionBox}

    \begin{questionBox}2{ % Q1.2
        Indique uma solução admissível para o problema e o respetivo custo.
    } % Q1.2
        \begin{flalign*}
            &
                (100,100,100)
            &
        \end{flalign*}
        qualquer solução que resolve x
    \end{questionBox}

    \begin{questionBox}2{ % Q1.3
        Resolva o problema recorrendo ao solver do Microsoft Excel. Com base na solução ótima determinada, indique a quantidade de hidratos de carbono, proteínas e vitaminas que cada animal recebe.
    } % Q1.3
        \begin{center}
            % \includegraphics[width=.8\textwidth]{Q8.1.png}
        \end{center}
    \end{questionBox}
        
\end{questionBox}

\setcounter{question}{2}

\begin{questionBox}1{ % Q3
    Uma empresa agrícola possui três terrenos onde pode fazer plantações. A empresa pretende plantar melão, batata-doce e tomate podendo cada um destes produtos ser plantado em mais do que um terreno. Sabendo que a empresa pretende saber que área deve plantar de cada tipo de plantação em cada terreno de modo a maximizar o lucro total, formule este problema como um modelo de Programação Linear.
} % Q3

    Na tabela seguinte apresenta-se a área de cada terreno disponível para as plantações e a quantidade de água disponível para regar que pode ser utilizada.
    \begin{center}
        \vspace{1ex}
        \begin{tabular}{*{3}{c}}
            
            \toprule
            
                \multicolumn{1}{c}{Terreno}
                &\multicolumn{1}{c}{Área Diponível (ha)}
                &\multicolumn{1}{c}{Agua Disponível (\unit{\metre^3/\day})}
            
            \\\midrule
            
                1    & 500 & 1600
                \\ 2 & 600 & 1800
                \\ 3 & 300 & 1000
            
            \\\bottomrule
            
        \end{tabular}
        \vspace{2ex}
    \end{center}

    Na tabela seguinte indica-se para cada tipo de plantação: a área total máxima que pode ser plantada, o consumo diário de água por cada hectare de plantação e o correspondente lucro.
    \begin{center}
        \vspace{-1ex}
        \setlength\tabcolsep{3mm}        % width
        \renewcommand\arraystretch{1.25} % height
        \begin{tabular}{l *{3}{c}}
            \toprule
            
                \multicolumn{1}{c}{
                    \begin{tabular}{c}
                        Tipo de\\Plantação
                    \end{tabular}
                }
                & \multicolumn{1}{c}{Area Máxima (ha)}
                & \multicolumn{1}{c}{
                    \begin{tabular}{c}
                        Consumo Diário\\de agua (\unit{\metre^3/ha})
                    \end{tabular}
                }
                & \multicolumn{1}{c}{
                    \begin{tabular}{c}
                        Lucro por\\ha (u.m.)
                    \end{tabular}
                }
            
            \\\midrule
            
                   Melão  & 700 & 5 & 7
                \\ Batata & 500 & 4 & 5
                \\ Tomate & 350 & 6 & 6
            
            \\\bottomrule
        \end{tabular}
        \vspace{1ex}
    \end{center}

    \begin{description}[
        leftmargin=!, 
        labelwidth=\widthof{\(a_{i,j}\)}
    ]
        \item[\(i\)] ( 1:Melão, 2:milho, 3:batata )
        \item[\(x_i\)] Terreno total ocupado pela plantação i
        \item[\(a_{i,j}\)] Area do terreno j em que está dedicado a plantação i
    \end{description}

    \vspace{-3ex}

    \begin{flalign*}
        &
            \max{\text{Lucro}}
            = 7*x_1
            + 5*x_2
            + 6*x_3
        &\\&
            \left\{
                \begin{aligned}
                    &
                        x_i = \sum_{j=1}^{3}{a_{i,j}}
                    \ldiv*{}
                        a_{i,j}\geq0\,\forall\,a_{i,j}
                    % Terreno Max
                    \ldiv{}
                        \sum_{i=1}^{3}{a_{i,1}}\leq 500
                    \ldiv*{}
                        \sum_{i=1}^{3}{a_{i,2}}\leq 600
                    \ldiv*{}
                        \sum_{i=1}^{3}{a_{i,3}}\leq 300
                    % Plantação Max
                    \ldiv{}
                        \sum_{i=1}^{3}{a_{1,i}}\leq 700
                    \ldiv*{}
                        \sum_{i=1}^{3}{a_{2,i}}\leq 500
                    \ldiv*{}
                        \sum_{i=1}^{3}{a_{3,i}}\leq 300
                    % Agua
                    \ldiv{}
                          a_{1,1}*5 
                        + a_{2,1}*4 
                        + a_{3,1}*6
                        \leq 1600
                    \ldiv{}
                          a_{1,2}*5 
                        + a_{2,2}*4 
                        + a_{3,2}*6
                        \leq 1800
                    \ldiv{}
                          a_{1,3}*5 
                        + a_{2,3}*4 
                        + a_{3,3}*6
                        \leq 1000
                    &
                \end{aligned}
            \right\}
        &
    \end{flalign*}
\end{questionBox}

\begin{questionBox}1{ % Q4
    Uma empresa possui duas fábricas produtoras de dois tipos de farinha. Sabendo que se pretende determinar o plano mais económico de abastecimento dos clientes, formule o problema em Programação Linear.
} % Q4

    A empresa deve garantir que a cada um dos seus três clientes chegam semanalmente pelo menos as quantidades de farinha (em toneladas) indicadas na tabela seguinte:
    \begin{center}
        \vspace{1ex}
        \begin{tabular}{c *{3}{C}}
            \toprule
            
                & \multicolumn{3}{c}{Clientes}
                \\
                Tipo de Farinha:
                & 1 & 2 & 3

            \\\midrule

                 A & 75 & 70 & 80
                \\ B & 60 & 85 & 90
            
            \\\bottomrule
        \end{tabular}
        \vspace{2ex}
    \end{center}

    A tabela seguinte contém a quantidade de farinha (em toneladas) produzida semanalmente em cada fábrica.

    \begin{center}
        \vspace{1ex}
        \begin{tabular}{c *{2}{C}}
            \toprule
            
                Tipo de Farinha
                & \multicolumn{1}{c}{Fábrica 1}
                & \multicolumn{1}{c}{Fábrica 2}
            
            \\\midrule
            
                A & 150 & 120
                \\ B & 160 & 130
            
            \\\bottomrule
        \end{tabular}
        \vspace{2ex}
    \end{center}

    \begin{itemize}
        \item A fábrica 1 não dispõe de espaço de armazenamento pelo que toda a farinha que produz deve ser distribuída pelos clientes. 
        \item A farinha produzida na fábrica 2 e que não é enviada para os clientes fica armazenada no armazém da fábrica. 
    \end{itemize}

    Os custos de transportar (em unidades monetárias) uma tonelada de farinha entre cada fábrica e cada cliente encontram-se registados na tabela seguinte:
    \begin{center}
        \vspace{1ex}
        \begin{tabular}{c *{3}{C}}
            \toprule
            
                & \multicolumn{3}{c}{Clientes}
                \\ Fábrica & 1 & 2 & 3
            
            \\\midrule
            
                1 & 20 & 30 & 60
                \\ 2 & 30 & 40 & 50
            
            \\\bottomrule
        \end{tabular}
        \vspace{2ex}
    \end{center}

    \begin{description}[
        leftmargin=!,
        labelwidth=\widthof{} % Longest item
    ]
        \begin{multicols}{2}
            \item[\textit{i}] Numero do Cliente \(i\in\myrange{1,3}\)
            
            \item[\textit{j}] Numero da Fábrica \(j\in\myrange{1,2}\)
            
            \item[\textit{k}] Tipo de farinha \(k=\{1:A,2:B\}\)
            
            \item[\(c_{i,j,k}\)] Quantidade de farinha \(k=\{1:A,2:B\}\) em toneladas entregue semanalmente ao cliente \(j\in\myrange{1,3}\) pela fabrica \textit{i}
    
            \item[\(C_{i,j}\)] Custo de farinha em tonelas semanalmente que cada fabrica \(i\in\myrange{1,2}\) tem em entregar para o cliente \(j\in\myrange{1,3}\)
    
            \item[\(M_{i,k}\)] A quantidade minima de farinha \(k=\{1:A,2:B\}\) em toneladas entregue semanalmente ao cliente \(i\in\myrange{1,3}\)

            \item[\(l_{j,k}\)] Quantidade de farinha produzida em toneladas semanalmente do tipo \(k=\{1:A,2:B\}\)
        \end{multicols}
       
    \end{description}

    \begin{BM}
        \min{\text{Custo}}
        = \sum_{i=1}^{2}{
            \sum_{j=1}^{3}{
                (c_{i,j,1}+c_{i,j,2})*C_{i,j}
            }
        }
    \end{BM}

    Sujeito a
    \begin{flalign*}
        &
            \sum_{j=1}^{3}{
                c_{i,j,k}
            }\geq M_{i,k}
            \forall\{i\in\myrange{1,3},k\in\{1,2\}\}
        &\\&
            \sum_{i=1}^{3}{
                c_{i,j,k}
            } = l_{j,k}
            \forall\{j\in\myrange{1,3},k\in\{1,2\}\}
        &
    \end{flalign*}

\end{questionBox}

\setcounter{question}{7}

\begin{questionBox}1{ % Q8
    Uma empresa produz componentes de tipo A, B e C. Sabe-se que a empresa pretende maximizar o lucro resultante da venda das peças. Formule este problema como um modelo de Programação Linear que pode incluir variáveis inteiras e/ou binárias.
} % Q8
    A tabela seguinte contém para cada componente: o lucro resultante da sua venda e o número de horas que deve ser processada em cada uma das máquinas.
    \begin{center}
        \begin{tabular}{l *{3}{c}}
            
            \\\toprule
            
                & \multicolumn{3}{c}{Componentes}
                \\
                & \multicolumn{1}{c}{A}
                & \multicolumn{1}{c}{B}
                & \multicolumn{1}{c}{C}
            
            \\\midrule
            
                Lucro & 10 & 50 & 100
                \\ M1 (\unit{\hour}) & 1 & 2 & 3
                \\ M2 (\unit{\hour}) & 2 & 1 & 1
            
            \\\bottomrule
            
        \end{tabular}
    \end{center}

    \begin{itemize}[left=0em]
        \item Por exemplo, o fabrico de uma componente B requer 2 horas na máquina 1 e 1 hora na máquina 2 sendo o seu preço de venda de 50.
        \item Cada componente deve ser obrigatoriamente processada em duas máquinas e sabe-se que cada máquina não trabalha mais do que 40 horas.
        \item Para fabricar uma componente B é necessário gastar uma componente A enquanto que, para fabricar uma componente C é necessário gastar uma componente B. Deste modo, as componentes A e B gastas no fabrico de outras componentes não podem ser vendidas.
    \end{itemize}

    \begin{description}[
        leftmargin=!, 
        labelwidth=\widthof{\(x_i\)}
    ]
        \item[\(i\)] (1:A, 2:B, 3:C)
        \item[\(x_i\)] numero de Componentes i vendidas
    \end{description}

    \begin{flalign*}
        &
            \text{Lucro} 
            = 10*x_1 
            + 50*x_2
            + 100*x_3
        &\\&
            \left\{
                \begin{aligned}
                    &
                        x_i\geq 0\,\forall\,x_i
                    \ldiv{}
                        (x_1+x_2+x_3)*1 + (x_2+x_3)*2 +x_3*3
                        \leq 40
                    \ldiv{}
                        (x_1+x_2+x_3)*2 + (x_2+x_3)*1 +x_3*1
                        \leq 40
                    &
                \end{aligned}
            \right\}
        &
    \end{flalign*}

\end{questionBox}

\setcounter{question}{5}

\begin{questionBox}1{ % Q6
    Uma fábrica que produz adubo. Sabendo que se pretende determinar o plano de produção de adubo para os meses de janeiro, fevereiro e março que minimiza os custos totais, formule o problema como um modelo de Programação Linear.
} % Q6

    tem que satisfazer as procuras registadas na tabela seguinte:
    \begin{center}
        \vspace{1ex}
        \begin{tabular}{l *{3}{C}}
            \toprule
            
                \multicolumn{1}{c}{Mês}
                & \multicolumn{1}{c}{Janeiro}
                & \multicolumn{1}{c}{Fevereiro}
                & \multicolumn{1}{c}{Março}
            
            \\\midrule
            
                Procura (\unit{\tonne})
                & 70 & 80 & 100
                \\ Custo de prod (\unit{u.m/\tonne})
                & 10 & 15 & 12
                \\ Maximo (\unit{\tonne})
                & 80 & 100 & 90
            
            \\\bottomrule
        \end{tabular}
        \vspace{2ex}
    \end{center}

    \begin{itemize}
        \item A fábrica pode produzir em janeiro, fevereiro e março até 80, 100 e 90 toneladas de adubo respetivamente.
        \item No início de janeiro existem 15 toneladas de adubo em stock que podem ser usadas para satisfazer a procura. Se após a satisfação da procura em janeiro e fevereiro sobrar adubo, este será armazenado para poder ser usado nos meses seguintes. No final de março não deve ficar qualquer adubo em stock.
        \item Os custos de produção por tonelada em janeiro, fevereiro e março são de 10, 15 e 12 unidades monetárias respetivamente. 
        \item No final de janeiro e fevereiro é necessário pagar o aluguer do armazém que é de 2 unidades monetárias por cada tonelada de adubo que fique em stock para o mês seguinte.
    \end{itemize}

    \begin{description}[
        leftmargin=!,
        labelwidth=\widthof{\(x_i\)} % Longest item
    ] 
        \item[\textit{i}] \{1:Janeiro,2:Fevereiro,3:Março\}
        \item[\(x_i\)] Quantidade de adubo produzida no mês
    \end{description}

    \begin{flalign*}
        &
            \min\left(
                \begin{aligned}
                    &
                        10\,x_1
                        + 15\,x_2
                        + 12\,x_3
                    &+\\+&
                        2\left(
                            x_1 + 15 - 70
                        \right)
                    &+\\+&
                        2\left(
                            (x_1 + 15 - 70)
                            + x_2 - 80
                        \right)
                    &
                \end{aligned}
            \right);
            &\\&
            \begin{aligned}
                    & x_1 \leq 80
                    ;x_2 \leq 100
                    ;x_3 \leq 90
                    \\&
                    x_i\geq0\,\forall\,x_i
            \end{aligned}
        &
    \end{flalign*}

\end{questionBox}

\setcounter{question}{11}

\begin{questionBox}1{ % Q12
    Num clube de natação, os quatro melhores nadadores são o Manuel, o Pedro, o Miguel e o João. Vai decorrer em breve um campeonato onde cada um dos nadadores referidos vai participar apenas numa prova.
} % Q12

    Na tabela seguinte encontra-se registado o tempo (em unidades de tempo) que cada um dos nadadores demora a completar cada uma das provas:
    \begin{center}
        \vspace{1ex}
        \begin{tabular}{l *{4}{C}}
            \toprule
            
                & \multicolumn{1}{c}{Bruços}
                & \multicolumn{1}{c}{Costas}
                & \multicolumn{1}{c}{Mariposa}
                & \multicolumn{1}{c}{Craw}
            
            \\\midrule
            
                Manuel
                & 62 & 61 & 61 & 56
                \\ Pedro
                & 61 & 59 & 63 & 58 
                \\ Miguel
                & 58 & 57 & 59 & 56
                \\ João
                & 59 & 58 & 61 & 57

            \\\bottomrule
        \end{tabular}
        \vspace{2ex}
    \end{center}


    \begin{description}[
        leftmargin=!,
        labelwidth=\widthof{\(x_{i,j}\)} % Longest item
    ]
        \item[\textit{i}] \{1: Manuel, 2:Pedro, 3: Miguel, 4: João\}
        \item[\textit{j}] \{1: Bruços, 2: Costas, 3: Mariposa, 4: Craw\}
        \item[\(x_{i,j}\)] \{1: Nadador \textit{i} participa da prova \textit{j}, 0: Caso contrário\}
        \item[\(t_{i,j}\)] Tempo que o nadador \textit{i} demora na prova \textit{j}
    \end{description}

    \begin{questionBox}2{ % Q12.1
        Sabendo que se pretende atribuir uma e uma só prova a cada um dos nadadores e um nadador a cada prova de modo que o tempo total das 4 provas seja minimizado, formule este problema como um modelo de Programação Linear.
    } % Q12.1
        \begin{flalign*}
            &
                \min\left(
                    \sum_{j=1}^{4}{
                        \sum_{i=1}^{4}{
                            t_{i,j}\,x_{i,j}
                        }
                    }
                \right)
            % ; \quad
            % ; &\\&
                \begin{cases}
                    \sum_{i=1}^{4}{x_{i,j}}
                    = 1\quad\forall\,j
                    \\ \sum_{j=1}^{4}{x_{i,j}}
                    = 1\quad\forall\,i
                    \\ x_{i,j}\in\{0,1\}\quad\forall\,i,j
                \end{cases}
            &
        \end{flalign*}
    \end{questionBox}

    \begin{questionBox}2{ % Q12.2
        Admita que nem o João nem o Miguel podem realizar a prova de mariposa. Que alterações devem ser introduzidas no modelo apresentado em a) de modo a contemplar esta situação?
    } % Q12.2
        \begin{flalign*}
            &
                x_{i,3}=0:i\in\{3,4\}
            &
        \end{flalign*}
    \end{questionBox}
\end{questionBox}

\begin{questionBox}1{ % Q13
    O João decidiu distribuir por alguns dos seus 10 melhores amigos bilhetes para o EURO2020. Sabendo que se pretende maximizar a satisfação total dos amigos do João, formule o problema em Programação Linear, podendo recorrer à utilização de variáveis inteiras e/ou binárias.
} % Q13
    A satisfação de cada amigo depende do bilhete que receber. Na tabela seguinte o João registou essa satisfação.
    \begin{center}
        \vspace{1ex}
        \setlength\tabcolsep{3mm}        % width
        % \renewcommand\arraystretch{1.25} % height
        \begin{tabular}{l *{10}{C}}
            \toprule
            
                & \multicolumn{10}{c}{Amigo}
                \\ \multicolumn{1}{c}{Jogo}
                & 1 & 2 & 3 & 4 & 5 & 6 & 7 & 8 & 9 & 10
            
            \\\midrule
            
                Itália--Bélgica
                & 10 &  12 &  11 &  10 &  12 &  12 &  11 &  13 &  14 &  10
                \\ Suíça--Espanha
                &  9 &  11 &  11 &  11 &  12 &  12 &  11 &  10 &  13 &  11
                \\ Final da EURO 2020
                & 10 &  11 &  12 &  12 &  13 &  12 &  10 &  12 &  14 &  10

            \\\bottomrule
        \end{tabular}
        \vspace{2ex}
    \end{center}

    O João também já decidiu que:
    \begin{itemize}
        \item {\color{blue\Light} vai distribuir 3 bilhetes para o jogo Itália-Bélgica, 2 bilhetes para o jogo Suíça-Espanha e um bilhete para a final.}
        \item {\color{red\Light} cada amigo pode receber, no máximo, um bilhete;}
        \item {\color{green\Light} após a distribuição dos bilhetes, a satisfação total do amigo 4 não deve ser inferior à satisfação total do amigo 7;}
        \item {\color{cyan\Light} se o amigo 2 receber um bilhete para o Suíça-Espanha então o amigo 9 deve receber um bilhete para a final}
    \end{itemize}

    \begin{description}[
        leftmargin=!,
        labelwidth=\widthof{\(x_{i,j}\)} % Longest item
    ]
        \begin{multicols}{2}
            \item[\textit{i}] Amigo \(\in\myrange{1,10}\)
            \item[\textit{j}] Jogo, 
            \(\begin{cases}
                1: \text{Itália--Belgica};
                \\ 2: \text{Suiça--Espanha};
                \\ 3: \text{Final da EURO 2020}
            \end{cases}\)
            \item[\(s_{i,j}\)] Satisfação do amigo \textit{i} no jogo \textit{j}, valores na tabela
            \item[\(x_{i,j}\)] 
            \(\begin{cases}
                1: \text{o amigo \textit{i} vai ao jogo \textit{j}}
                \\ 0: \text{Caso contrário}
            \end{cases}\)
        \end{multicols}
    \end{description}

    \begin{flalign*}
        &
            \max\left(
                \begin{aligned}
                    \sum_{i=1}^{10}{
                        \sum_{j=1}^{4}{
                            s_{i,j}\,x_{i,j}
                        }
                    }
                \end{aligned}
            \right)
            \begin{cases}
                x_{i,j}\in\{0,1\}
                \\ {\color{blue\Light} 
                    \sum_{i=1}^{10}{
                        x_{i,1}
                    }= 3
                    \quad \sum_{i=1}^{10}{
                        x_{i,2}
                    } = 2
                    \quad \sum_{i=1}^{10}{
                        x_{i,3}
                    } = 1
                }
                \\ {\color{red\Light}
                    \sum_{j=1}^{3}{x_{i,j}} = 1\quad\forall\,i
                }
                \\ {\color{green\Light}
                    \sum_{j=1}^{3}{
                        s_{4,j}\,x_{4,j}
                        -s_{7,j}\,x_{7,j}
                    }\leq 0
                }
                \\ {\color{cyan\Light}
                    x_{2,2}
                    \leq x_{3,9}
                }
            \end{cases}
        &
    \end{flalign*}
\end{questionBox}

\setcounter{question}{16}

\begin{questionBox}1{ % Q17
    Considere a região admissível \textit{S} de um problema de Programação Linear definida por
} % Q17
    \begin{BM}[align*]
        x+y &\leq 7
        \\ x+y &\geq 2
        \\ -x+y &\leq 2
        \\ 3\,x-2\,y &\leq 6
        \\ x,y &\geq 0
    \end{BM}

    
    \begin{center}

        % \tikzset{external/remake next=true} %   - remake next
        \begin{tikzpicture}
        \begin{axis}
            [
                % xmajorgrids = true,
                % legend pos  = north west
                axis lines = {center}, % 3D center/box/left/right
                axis on top,
                xmin=0,
                ymin=0,
            ]
            % Legends
            % \addlegendimage{empty legend}
            % \addlegendentry[Red]{\( x \)}
            
            % % Plot from equation
            % \addplot[
            %     smooth,
            %     thick,
            %     % Red,
            %     domain  = -2:2,
            %     samples = 0.4*\mysampledensity,
            % ]{  x };

            \addplot[no marks]{7-x};
            \addplot[no marks]{2-x};
            \addplot[no marks]{x+2};
            \addplot[no marks]{(3*x-6)/2};
            
        \end{axis}
        \end{tikzpicture}
    \end{center}
\end{questionBox}

\setcounter{question}{18}

\begin{questionBox}1{ % Q19
    Considere o seguinte problema (P) de Programação Linear:
} % Q19
    \begin{BM}[align*]
        \max{F} &= 4\,x+2\,y
        \\
        s.a.
        &\begin{cases}
            2\,x-y &\geq 1
            \\ 2\,x+y &\leq 10
            \\ -x+4\,y &\leq 7
            \\ x,y &\geq 0
        \end{cases}
    \end{BM}

    \begin{questionBox}2{ % Q19.1
        Recorrendo ao Método Gráfico, resolva o problema (P).
    } % Q19.1
        \begin{center}

            % \tikzset{external/remake next=true} %   - remake next

            \begin{tikzpicture}
            \begin{axis}
                [
                    % xmajorgrids = true,
                    % legend pos  = north west
                    axis lines = {center}, % 3D center/box/left/right
                    axis on top,
                    xmin=0,
                    ymin=0,
                    xmax=5,
                    ymax=6,
                ]
                % Legends
                % \addlegendimage{empty legend}
                % \addlegendentry[Red]{\( x \)}

                \addplot[no marks]{2*x-1};
                \addplot[no marks]{10-2*x};
                \addplot[no marks]{(7+x)/4};
                
            \end{axis}
            \end{tikzpicture}
        \end{center}
    \end{questionBox}

    \begin{questionBox}2{ % Q19.2
        Admita que o termo independente da terceira restrição passou a ser \(\theta\,(\theta\geq0)\). Resolva o problema nesta situação.
    } % Q19.2
    \end{questionBox}

    \begin{questionBox}2{ % Q19.3
        Admita que a função objetivo passou a ser \(\max{G}=\theta\,x + 2\,y\), com \(\theta\geq0\). Resolva o problema nesta situação
    } % Q19.3
    \end{questionBox}

    \begin{questionBox}2{ % Q19.4
        Admita que ao problema (P) foi adicionada a restrição: ``x e y devem ser inteiros''. Resolva o problema recorrendo ao Método Gráfico.
    } % Q19.4
        body
    \end{questionBox}

    \begin{questionBox}2{ % Q19.5
        Admita que ao problema (P) foi adicionada a restrição: ``y deve se inteiro''. Resolva o problema recorrendo ao Método Gráfico.
    } % Q19.5
        body
    \end{questionBox}
\end{questionBox}

\setcounter{question}{20}

\begin{questionBox}1{ % Q21
    Resolva os seguintes problemas de Programação Linear recorrendo ao Método do Simplex:
} % Q21

    \begin{questionBox}2{} % Q21.1
        \begin{BM}[align*]
            \max{F}&=3\,x+y
            \\ s.a.
            & \begin{cases}
                x+2\,y&\leq12
                \\x-y&\leq7
                \\x,y&\geq0
            \end{cases}
        \end{BM}

        \begin{center}
            \vspace{1ex}
            \begin{tabular}{*{4}{C} | C}
                \toprule
                
                    x & y & s_1 & s_2 & MD\mathbb{R}
                
                \\\midrule

                \rowcolor{Emph!20!background}
                    -3 & -1 & 0 & 0 & 0
                \\   1 &  2 & 1 & 0 & 12
                \\   1 & -1 & 0 & 1 & 7
                
                \\\midrule

                \rowcolor{Emph!20!background}
                     0 & -4 & 0 &  3 & 21
                \\   0 &  3 & 1 & -1 & 5
                \\   1 & -1 & 0 &  1 & 7
                
                \\\midrule

                \rowcolor{Emph!20!background}
                     0 &  0 & 4/3 &  5/3 & 83/3
                \\   0 &  1 & 1/3 & -1/3 &  5/3
                \\   1 &  0 & 1/3 &  2/3 & 16/3
                
                \\\bottomrule
            \end{tabular}
            \vspace{2ex}
        \end{center}
    \end{questionBox}

    \begin{questionBox}2{} % Q21.2
        \begin{BM}[align*]
            \max{F}&=4\,x+y+2\,z
            \\ s.a.
            & \begin{cases}
                2\,x+y-2\,z&\leq20
                \\3\,x+6\,y&\leq12
                \\x,y,z&\geq0
            \end{cases}
        \end{BM}
    \end{questionBox}

    \begin{questionBox}2{} % Q21.3
        \begin{BM}[align*]
            \min{W} &= -x-y
            \\ s.a.
            & \begin{cases}
                x+y&\leq 4
                \\-x+y &\leq 1
                \\x,y&\geq0
            \end{cases}
        \end{BM}

        \begin{center}
            \vspace{1ex}
            \begin{tabular}{C | *{4}{C} | C C}
                \toprule
                
                    & x & y & f_1 & f_2 & \multicolumn{1}{c}{T.I.}
                    & \Delta
                
                \\\midrule

                \rowcolor{Emph!20!background}
                    W   &  1 & 1 & 0 & 0 & 0 
                \\  f_1 &  1 & 1 & 1 & 0 & 4 & 4/1
                \\  f_2 & -1 & 1 & 0 & 1 & 1 & -
                
                \\\midrule

                \rowcolor{Emph!20!background}
                    W   &  0 & 0 & -1 &  0 & -4
                \\  x   &  1 & 1 &  1 &  0 &  4 & 4/1
                \\  f_2 &  0 & 2 &  1 &  1 &  5 & 5/2
                
                \\\midrule

                \rowcolor{Emph!20!background}
                    W   &  0 & 0 & -1   &  0   & -4
                \\  x   &  1 & 0 &  1/2 & -1/2 &  3/2
                \\  f_2 &  0 & 0 &  1/2 &  1/2 &  5/2
                
                \\\bottomrule
            \end{tabular}
            \vspace{2ex}
        \end{center}

        \begin{BM}
            (x^*,y^*) 
            = \lambda(4,0)
            + (1-\lambda)(3/2,5/2)
            \,\lambda\in\myrange{0,1}
        \end{BM}

    \end{questionBox}

\end{questionBox}

\begin{questionBox}1{ % Q22
    Resolva os seguintes problemas de Programação Linear Inteira pelo algoritmo Branch and Bound estudado. Resolva cada subproblema utilizando o Método Gráfico. Apresente as árvores de pesquisa correspondente à sua resolução.
} % Q22

    \begin{questionBox}2{ % Q22.1
    } % Q22.1
        \begin{BM}[align*]
            \max{G} &= \frac{1}{2}\,x_1+\frac{3}{2}\,x_2
            \\ s.a. &
            \begin{cases}
                x_1 + x_2 &\leq 1
                \\
                x_2 &\leq 5/2
                \\
                2\,x_1 + x_2 &\leq 7
                \\
                \{x_1,x_2\}\subset\mathbb{N}
            \end{cases}
        \end{BM}
    \end{questionBox}

    \begin{questionBox}2{ % Q22.2
    } % Q22.2
        \begin{BM}[align*]
            \max{G} &= 4\,x_1+3\,x_2
            \\ s.a. &
            \begin{cases}
                4\,x_1+9\,x_2 & \leq 26
                \\
                8\,x_1+5\,x_2 & \leq 20
                \\
                \{x_1,x_2\}\subset\mathbb{N}
            \end{cases}
        \end{BM}

        \begin{flalign*}
            &
                (x_1^*,x_2^*)
                : \begin{cases}
                    26/9-4\,x_1^*/9
                    = 20/5-8\,x_1^*/5
                    &\implies
                    x_1^*=0.96
                    \\
                    8\,x_1^*+5\,x_2^* = 20
                    &\implies
                    x_2^*=2.46
                \end{cases}
            &\\[3ex]&
                % ---------------------------------- PL 0   ---------------------------------- %
                \text{PL }0
                &\\&
                (x_1^*,x_2^*) = (0.96,2.46)
                \quad F^* = 11.22
            &\\[3ex]&
                % ---------------------------------- PL 0.1 ---------------------------------- %
                \text{PL }0.1\quad x_1 \leq 0
                &\\&
                x_1^*=0\land 4*0+9\,x_2^*=26 \implies x_2^*=26/9
                &\\&
                \therefore (x_1^*,x_2^*) = (0,26/9)
                \quad F^* = 8.66
            &\\[3ex]&
                % ---------------------------------- PL 0.2 ---------------------------------- %
                \text{PL }0.2\quad x_1 \geq 1
                &\\&
                x_1^*=0\land 4*0+9\,x_2^*=26 \implies x_2^*=26/9
                &\\&
                \therefore (x_1^*,x_2^*) = (0,26/9)
                \quad F^* = 8.66
            &
        \end{flalign*}

        \begin{center}

            % \tikzset{external/remake next=true} %   - remake next
            
            \begin{tikzpicture}
            \begin{axis}
                [
                    % xmajorgrids = true,
                    % legend pos  = north west
                    axis lines = {center}, % 3D center/box/left/right
                    axis on top,
                    xmin=0,
                    ymin=0,
                    xmax=3,
                    % ymax=6,
                ]
                % Legends
                % \addlegendimage{empty legend}
                % \addlegendentry[Red]{\( x \)}

                \addplot[no marks]{26/9-4*x/9};
                \addplot[no marks]{20/5-8*x/5};
                \addplot[no marks, dashed] coordinates {
                    (0.96,0)
                    (0.96,4)
                };
                
            \end{axis}
            \end{tikzpicture}
        \end{center}
    \end{questionBox}

    \begin{questionBox}2{} % Q22.3
        \begin{BM}[align*]
            \min{G} &= 2\,x_1+x_2
            \\ s.a. &
            \begin{cases}
                5\,x_1+6\,x_2 &\geq 20
                \\
                -5\,x_1+6\,x_2 & \leq 10
                \\
                \{x_1,x_2\}\subset\mathbb{N}
            \end{cases}
        \end{BM}

        \begin{center}

            % \tikzset{external/remake next=true} %   - remake next

            \begin{tikzpicture}
            \begin{axis}
                [
                    % xmajorgrids = true,
                    % legend pos  = north west
                    axis lines = {center}, % 3D center/box/left/right
                    axis on top,
                    xmin=0,
                    ymin=0,
                    xmax=5,
                    ymax=6,
                ]
                % Legends
                % \addlegendimage{empty legend}
                % \addlegendentry[Red]{\( x \)}

                \addplot[no marks]{20/6-5*x/6};
                \addplot[no marks]{10/6+5*x/6};
                
            \end{axis}
            \end{tikzpicture}
        \end{center}

    \end{questionBox}

\end{questionBox}

\setcounter{question}{23}

\begin{questionBox}1{ % Q24
    Considere o seguinte problema de Programação Linear Inteira
} % Q24
    \begin{BM}[align*]
        ??? &= -2\,x_1-3\,x_2
        \\
        s.a &
        \begin{cases}
            5\,x_1+2\,x_2 &\leq 25
            \\
            2\,x_1+5\,x_2 &\leq 25
            \\
            x_1+x_2 &\leq 7
            \\
            \{x_1,x_2\} \subset\mathbb{N}
        \end{cases}
    \end{BM}

    Começou a resolver-se o problema através do algoritmo Branch \& Bound estudado apresentando-se em seguida a árvore de pesquisa já obtida.

    \begin{center}
        \vspace{1ex}
        \includegraphics[width=.8\textwidth]{Q24.png}
    \end{center}

    \begin{questionBox}2{ % Q24.1
        Indique, justificando, se o problema que se está a resolver é de tipo máximo ou de tipo mínimo.
    } % Q24.1
        Minimizar pois avança no sentido de menor valor
    \end{questionBox}

    \begin{questionBox}2{ % Q24.2
        Com base na informação disponibilizada consegue identificar o valor ótimo para este problema? Em caso afirmativo indique qual é esse valor e a correspondente solução ótima. Em caso negativo, indique o melhor limite inferior e o melhor limite superior para o valor ótimo deste problema. Com vista à determinação do valor ótimo indique como se deveria prosseguir com o Algoritmo Branch \& Bound.
    } % Q24.2
        Ainda não temos solução ótima, tiramos o intervalo para \(Z^*\) de L3 e L5 por serem mais promissores
        \begin{BM}
            Z^*\in\myrange{-16,-15}
        \end{BM}
        A seguinte ramificação seria em L5 pondo \(5\leq x_2\geq 4\)
    \end{questionBox}
\end{questionBox}

\begin{questionBox}1{ % Q25
    Trace as redes correspondentes a cada um dos seguintes projetos. Determine o Caminho Crítico e a respetiva duração.
} % Q25

    \begin{questionBox}2{} % Q25.1
        \begin{center}
            \vspace{1ex}
            \begin{tabular}{c c C}
                \toprule
                
                    \multicolumn{1}{c}{Atividades}
                    & \multicolumn{1}{c}{Precendencias Imediatas}
                    & \multicolumn{1}{c}{Duração Meses}
                
                \\\midrule
                
                       A & -- & 3
                    \\ B & -- & 6
                    \\ C & -- & 5
                    \\ D &  C & 7
                    \\ E &  B & 4
                    \\ F &  A & 8
                
                \\\bottomrule
            \end{tabular}
            \vspace{2ex}
        \end{center}

        % \tikzset{external/remake next=true}
        % \begin{tikzpicture}
        %     \graph[
        %         nodes={circle,draw}
        %     ] {
        %         1 -> {2,3,4} -> 5;
        %     };
        % \end{tikzpicture}

    \end{questionBox}
\end{questionBox}

\begin{questionBox}1{ % Q26
    Considere um projeto cujas características se indicam na tabela seguinte:
} % Q26
    \begin{center}
        \setlength\tabcolsep{3mm}        % width
        % \renewcommand\arraystretch{1.25} % height
        \vspace{1ex}
        \begin{tabular}{*{4}{c}}
            \toprule
            
                Atividades
                & \begin{tabular}{c}
                    Pressedores\\Imediatos
                \end{tabular}
                & \begin{tabular}{c}
                    Duração\\Semanas
                \end{tabular}
                & \begin{tabular}{c}
                    Numero de\\funcionarios
                    \\necessários
                    \\por semana
                \end{tabular}
            
            \\\midrule
            
                   A & --  &  6 & 3
                \\ B & A   &  4 & 4
                \\ C & --  &  8 & 4
                \\ D & C   &  6 & 2
                \\ E & B,D & 10 & 3
                \\ F & D   &  8 & 1
            
            \\\bottomrule
        \end{tabular}
        \vspace{2ex}
    \end{center}

    \begin{questionBox}2{ % Q26.1
        Construa a rede que representa o projeto.
    } % Q26.1
        \tikzset{external/remake next=true} % remake next
        \begin{center}
        \begin{tikzpicture}[myGraphsStyle]
        
            % ===================== Vertices ===================== %
        
            % 1,2,5,6
            \pic (1) at (0, 0) {graphVertice={1/ 0/ 0/below/4mm}};
            \pic (2) at (2, 1) {graphVertice={2/ 6/10/above/4mm}};
            \pic (5) at (4, 1) {graphVertice={5/14/14/above/4mm}};
            \pic (6) at (6, 0) {graphVertice={6/24/24/below/4mm}};
            % 3,4
            \pic (3) at (2,-1) {graphVertice={3/ 8/ 8/below/4mm}};
            \pic (4) at (4,-1) {graphVertice={4/14/14/below/4mm}};
        
            % ======================= Edges ====================== %
        
            % 1 ->[A] 2 ->[B] 5 ->[E] 6
            \draw[->] (1-x) --node[sloped,above]{A  6} (2-x);
            \draw[->] (2-x) --node[sloped,above]{B  4} (5-x);
            \draw[->] (5-x) --node[sloped,above]{E 10} (6-x);
            % 1 ->[C] 3 ->[D] 4 ->[F] 6
            \draw[->] (1-x) --node[sloped,above]{C  8} (3-x);
            \draw[->] (3-x) --node[sloped,above]{D  6} (4-x);
            \draw[->] (4-x) --node[sloped,above]{F  8} (6-x);
            % 4 -> 5
            \draw[->,dashed] (4-x) -- (5-x);
            
            % ==================================================== %
        
        \end{tikzpicture}
        \end{center}
    \end{questionBox}

    \begin{questionBox}2{ % Q26.2
        Indique o caminho crítico e a sua duração
    } % Q26.2
        C,D,E, de duração 24 semanas
    \end{questionBox}

    \begin{questionBox}2{ % Q26.3
        Se a duração da atividade D passasse a ser de 7 semanas, haveria alteração na duração do projeto?
    } % Q26.3
        Sim, pois faz parte do caminho crítico
    \end{questionBox}

    \begin{questionBox}2{ % Q26.4
        Se a atividade B demorasse o dobro do tempo que alteração sofria a data de conclusão do projeto?
    } % Q26.4
        Não pois a nova duração não supassa o tempo de folga (10-6)
    \end{questionBox}

    \begin{questionBox}2{ % Q26.5
        Mostre que é possível realizar este projeto no tempo previsto sem necessitar de mais do que 7 funcionários.
    } % Q26.5 
        Desenho.
    \end{questionBox}
\end{questionBox}

\end{document}
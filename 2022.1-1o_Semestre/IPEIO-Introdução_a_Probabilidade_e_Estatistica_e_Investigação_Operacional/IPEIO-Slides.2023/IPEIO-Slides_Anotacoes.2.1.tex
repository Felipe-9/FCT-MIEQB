% !TEX root = ./IPEIO-Slides_Anotações.2.1.tex
\providecommand\mainfilename{"./IPEIO-Slides_Anotações.tex"}
\providecommand \subfilename{}
\renewcommand   \subfilename{"./IPEIO-Slides_Anotações.2.1.tex"}
\documentclass[\mainfilename]{subfiles}

% \tikzset{external/force remake=true} % - remake all

\begin{document}

\graphicspath{{\subfix{./.build/figures/IPEIO-Slides_Anotações.2.1}}}
\tikzsetexternalprefix{./.build/figures/IPEIO-Slides_Anotações.2.1/}

\mymakesubfile{1}
[IO]
{Slides Anotações} % Subfile Title
{Multiplas Soluções Optimas} % Part Title


\part*{Forma Standard}

\begin{exampleBox}1{} % E1
    
    \begin{BM}[align*]
        \max \,& z=x_1+2\,x_3
        \\
        \text{s.a.}\,
        & \left\{
        \begin{matrix}
            x_1&+&x_2&+&x_3 & \leq 4
            \\ x_1&& && &\leq 2
            \\ && &&x_3 &\leq 3
            \\ &3& x_2&+&x_3 &\leq 6
            \\ x_1&,&x_2&,&x_3 &\geq 0
        \end{matrix}
        \right.
    \end{BM}

    \begin{answerBox}{} % RS 
        \begin{flalign*}
            &
                \begin{pmatrix}
                       x_1&+&x_2&+&x_3 & \leq 4
                    \\ x_1&& && &\leq 2
                    \\ && &&x_3 &\leq 3
                    \\ &3& x_2&+&x_3 &\leq 6
                    \\ x_1&,&x_2&,&x_3 &\geq 0
                \end{pmatrix}
                = \begin{pmatrix}
                       x_1&+&x_2&+&x_3&+&x_4& = 4
                    \\ x_1&& && &+&x_5&= 2
                    \\ && &&x_3&+&x_6&= 3
                    \\ &3& x_2&+&x_3&+&x_7 &= 6
                    \\ x_1&,&x_2&,&x_3&,&x_4,x_5,x_6,x_7&\geq 0
                \end{pmatrix}
            &
        \end{flalign*}
    \end{answerBox}
    
\end{exampleBox}

\begin{exampleBox}{} % E2
    
    \begin{description}[
        leftmargin=!,
        % labelwidth=\widthof{} % Longest item
    ]
       \item[\(x_{i,j}\)] Quantidade de água, em milhões de \unit{\metre^3}, que a barragem \(B_i\) fornece à região \(R_j\), \(i\in\myrange{1,2}\) e \(j\in\myrange(1,3)\).
    \end{description}
    

    \begin{BM}[align*]
        \text{minimizar}\quad
        & z=\begin{cases}
            14\,x_{1,1}+10\,x_{1,2}+9\,x_{1,3}
            \\ + 13\,x_{2,1}+11\,x_{2,2}+12\,x_{2,3}
        \end{cases}
        \\
        \text{sujeito a}\quad
        & \begin{cases}
            x_{1,1}+x_{2,1} & \geq 10
            \\ x_{1,2}+x_{2,2} & \geq 5
            \\ x_{1,3}+x_{2,3} & \geq 15
            \\ x_{1,1}+x_{1,2}+x_{2,3} & \leq 14
            \\ x_{2,1}+x_{2,2}+x_{2,3} & \leq 16
            \\ x_{i,j}&\leq0\quad\forall\,\{i,j\}
        \end{cases}
    \end{BM}

    \begin{answerBox}{} % RS 
        \begin{flalign*}
            &
                \begin{pmatrix}
                    x_{1,1}+x_{2,1} &\geq& 10
                    \\ x_{1,2}+x_{2,2} &\geq& 5
                    \\ x_{1,3}+x_{2,3} &\geq& 15
                    \\ x_{1,1}+x_{1,2}+x_{2,3} &\leq& 14
                    \\ x_{2,1}+x_{2,2}+x_{2,3} &\leq& 16
                    \\ x_{i,j}&\geq&0
                \end{pmatrix}
                =\begin{pmatrix}
                    x_{1,1}+x_{2,1}+x_{3,1} &=& 10
                    \\ x_{1,2}+x_{2,2}+x_{3,2} &=& 5
                    \\ x_{1,3}+x_{2,3}+x_{3,4} &=& 15
                    \\ x_{1,1}+x_{1,2}+x_{2,3}-x_{3,5} &=& 14
                    \\ x_{2,1}+x_{2,2}+x_{2,3}-x_{3,6} &=& 16
                    \\ x_{i,j}&\geq&0
                \end{pmatrix}
            &
        \end{flalign*}
    \end{answerBox}

\end{exampleBox}

\part*{Algoritmo do Simplex}

\begin{exampleBox}{} % E3
    
    \begin{BM}[align*]
        \text{Maximizar}
        \quad& z=x_1+x_2
        \\
        \text{Sujeito a}
        \quad&
        \left\{
            \begin{matrix}
                    & x_1 &-& x_2 &\leq& 4
                \\-2& x_1 &-& x_2 &\leq& 2
                \\ x_i\geq0
            \end{matrix}
        \right.
    \end{BM}

    \begin{answerBox}{} % RS 
        \begin{center}
            \tikzset{external/remake next=true} %   - remake next
            \begin{tikzpicture}
            \begin{axis}
                [
                    % xmajorgrids = true,
                    % legend pos  = north west
                    axis lines = {center}, % 3D center/box/left/right
                    axis on top,
                    xmin=0,
                    ymin=0,
                    % xmax=10,
                    ymax=4,
                ]
                % Legends
                % \addlegendimage{empty legend}
                % \addlegendentry[Red]{\( x \)}

                \addplot[no marks]{x-4};
                \addplot[no marks]{2+2*x};
                % \addplot[no marks]{(3*x-6)/2};
                
            \end{axis}
            \end{tikzpicture}
        \end{center}
    \end{answerBox}
    
\end{exampleBox}

% \begin{exampleBox}1{ % E1
% } % E1
         
%     Utilizando o método do Simplex, decida sobre a existência de soluções ótimas múltiplas

%     \begin{BM}[align*]
%         \text{Minimizar}
%         \quad& 
%         z= 40\,x-20\,y
%         \\
%         \text{Sujeito a}
%         \quad& 
%         \begin{cases}
%                x + y \leq 80
%             \\ x \leq 40
%             \\ 2\,x+y \leq 100
%             \\ x,y \geq 0
%         \end{cases}
%     \end{BM}

%     \begin{center}
%         \vspace{1ex}
%         \begin{tabular}{*{5}{C} | C}
            
%             \toprule
            
%                 x 
%                 & y 
%                 & s_1 & s_2 & s_3 
%                 & \text{MD}\mathbb{R}
            
%             \\\midrule
            
%             \rowcolor{Emph!20!background}
%                 -40 & -20 & 0 & 0 & 0 & 0
%             \\  1 & 1 & 1 & 0 & 0 & 80
%             \\  1 & 0 & 0 & 1 & 0 & 40
%             \\  2 & 1 & 0 & 0 & 1 & 100
            
%             \\\midrule

%             \rowcolor{Emph!20!background}
%                 0 & -20 & 0 & 40 & 0 & 1600
%             \\  0 & 1 & 1 & -1 & 0 & 40
%             \\  1 & 0 & 0 & 1 & 0 & 40
%             \\  0 & 1 & 0 & -2 & 1 & 20
            
%             \\\midrule

%             \rowcolor{Emph!20!background}
%                 0 &   0 & 0 & 0 & 20 & 2000
%             \\  0 &   0 & 1 & 1 & -1 & 20
%             \\  1 &   0 & 0 & 1 & 0 & 40
%             \\  0 &   1 & 0 & -2 & 1 & 20
            
%             \\\midrule

%             \rowcolor{Emph!20!background}
%                 0 &   0 & 0 & 0 & 20 & 2000
%             \\  0 &   0 & 1 & 1 & -1 & 20
%             \\  1 &   0 & 0 & 0 & 1 & 20
%             \\  0 &   1 & 0 & 0 & -1 & 60
            
%             \\\midrule
%         \end{tabular}
%         \vspace{2ex}
%     \end{center}
    
% \end{exampleBox}

% \begin{exampleBox}1{ % E2
% } % E2
    
%     \begin{BM}[align*]
%         \text{Minimizar}
%         &\quad
%         z=x_1+x_2-x_3
%         \\
%         \text{Sujeito a}
%         &\quad
%         \begin{cases}
%             x_1+x_2+x_3 \leq 4
%             \\ x_1\leq 2
%             \\ x_3\leq 3
%             \\ 3\,x_2 + x_3\leq 6
%             \\ x_i\geq0
%         \end{cases}
%     \end{BM}

%     \begin{center}
%         \vspace{1ex}
%         \begin{tabular}{*{7}{C} | C}
%             \toprule
            
%                 x_1 & x_2 & x_3 & s_1 & s_2 & s_3 & s_4 & \text{MD}\mathbb{R}
            
%             \\\midrule
            
%             \rowcolor{Emph!20!background}
%                 -1 & -1 & 1 & 0 & 0 & 0 & 0 & 0
%             \\  1 & 1 & 1 & 1 & 0 & 0 & 0 & 4
%             \\  1 & 0 & 0 & 0 & 1 & 0 & 0 & 2
%             \\  0 & 0 & 1 & 0 & 0 & 1 & 0 & 3
%             \\  0 & 3 & 1 & 0 & 0 & 0 & 1 & 6
            
%             \\\midrule
            
%             \rowcolor{Emph!20!background}
%                 0 &-1 & 1 & 0 & 1 & 0 & 0 & 2
%             \\  0 & 1 & 1 & 1 &-1 & 0 & 0 & 2
%             \\  1 & 0 & 0 & 0 & 1 & 0 & 0 & 2
%             \\  0 & 0 & 1 & 0 & 0 & 1 & 0 & 3
%             \\  0 & 3 & 1 & 0 & 0 & 0 & 1 & 6
            
%             \\\midrule
            
%             \rowcolor{Emph!20!background}
%                 0 & 0 & 2 & 1 & 0 & 0 & 0 & 4
%             \\  0 & 1 & 1 & 1 &-1 & 0 & 0 & 2
%             \\  1 & 0 & 0 & 0 & 1 & 0 & 0 & 2
%             \\  0 & 0 & 1 & 0 & 0 & 1 & 0 & 3
%             \\  0 & 0 &-2 &-3 & 3 & 0 & 1 & 0
            
            
%             \\\midrule
            
%             \rowcolor{Emph!20!background}
%                 0 & 0 & 2   & 1 & 0 & 0 & 0   & 4
%             \\  0 & 1 & 1/3 & 0 & 0 & 0 & 1/3 & 2
%             \\  1 & 0 & 2/3 & 1 & 0 & 0 &-1/3 & 2
%             \\  0 & 0 & 1   & 0 & 0 & 1 & 0   & 3
%             \\  0 & 0 &-2/3 &-1 & 1 & 0 & 1/3 & 0
            
%             \\\midrule
%         \end{tabular}
%         \vspace{2ex}
%     \end{center}
    
% \end{exampleBox}


\end{document}
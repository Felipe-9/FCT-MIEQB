% !TEX root = ./BG.b-Slide_01_Anotações.tex
% !TEX root = ./BG.b-Slides_Anotações.tex
\providecommand\mainfilename{"./BG.b-Slides_Anotações.tex"}
\providecommand \subfilename{}
\renewcommand   \subfilename{"./BG.b-Slide_01_Anotações.tex"}
\documentclass[\mainfilename]{subfiles}

% \graphicspath{{\subfix{../images/}}}

\begin{document}

\mymakesubfile{1}
{BG.b Slide Introdução}
{BG.b Slide Introdução}

\begin{sectionBox}1{Macromoléculas Biológicas}
    
    \begin{enumerate}[label=(\roman*)]
        \begin{multicols}{2}
            \item \hyperref[Proteinas]       {Poteínas}
            \item \hyperref[Acidos_Nucleicos]{Ácidos Nucleicos}
            \item \hyperref[Polissacaridos]  {Polissacáridos}
            \item \hyperref[Lipidos]         {Lípidos}
        \end{multicols}
    \end{enumerate}
    
    \begin{sectionBox}3{Proteínas}
        
        Polímeros de L-Aminoácidos Ligados entre Si por ligações peptídicas
        
    \end{sectionBox}
    
    \begin{sectionBox}3{Ácidos Nucleicos}
        
        \begin{itemize}
            \begin{multicols}{2}
                \item DNA
                \item RNA
            \end{multicols}
        \end{itemize}
    
        São Polímeros de nucleótidos ligados entre si por ligações fosfodiéster (Monossacárido--Fosfato)
        
    \end{sectionBox}
    
    \begin{sectionBox}3{Polissacarídos}
        
        \begin{itemize}
            \begin{multicols}{2}
                \item Amido
                \item Celulose
            \end{multicols}
        \end{itemize}
    
        Polimeros de Açucares ligados entre si por ligações glicosídicas
        
    \end{sectionBox}
    
    \begin{sectionBox}3{Lípidos}
        
        Macromoléculas Constituídas por ácidos gosrdos (associados a outras moléculas) covalentemente ligados entre si.
        
    \end{sectionBox}

\end{sectionBox}

\end{document}
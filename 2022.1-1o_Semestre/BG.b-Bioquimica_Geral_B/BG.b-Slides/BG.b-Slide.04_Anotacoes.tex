% !TEX root = ./BG.b-Slide_04_Anotações.tex
% !TEX root = ./BG.b-Slides_Anotações.tex
\providecommand\mainfilename{"./BG.b-Slides_Anotações.tex"}
\providecommand \subfilename{}
\renewcommand   \subfilename{"./BG.b-Slide_04_Anotações.tex"}
\documentclass[\mainfilename]{subfiles}

% \graphicspath{{\subfix{../images/}}}

\begin{document}

\mymakesubfile{4}
{BG.b -- Metodos Analíticos}
{BG.b -- Metodos Analíticos}

\begin{sectionBox}1{Cromatografia de filtração em gel}
    
    Cromatografia de Exclusão molecular

    Separação de moleculas por \emph{forma} ou \emph{tamanho}. O gel cria um ambiente poroso por onde moleculas de diferentes tamanhos interagem diferentemente

    Fase estacionária consistem em polímeros insolúveis muito hidratados
    \begin{itemize}
        \item Agarose
        \item Dextrano
        \item Poliacrilamidas
    \end{itemize}
    
\end{sectionBox}

\begin{sectionBox}2{Exclusão Molecular}

    Cria uma interação mais forte com moleculas menores permitindo a passagem de molec maiores.

    Principio de gel poroso por onde moleculas menores entram criando uma dificuldade maior de difusão.
    
    \paragraph{Procedimento}
    \begin{enumerate}
        \item A amostra é aplicada ao topo da coluna
        \item O tampão (fase móvel) arrasta a amostra ao longo da coluna
        \item As moleculas difundem para dentro e fora dos poros da matriz (fase estacionária)
    \end{enumerate}

    \begin{sectionBox}*2{Limite de Exclusão}
        
        intervalo limite de massa molecular pelo qual as moleculas interagem diferente com o gel (caracteristico do gel)

        \paragraph{Interação levando em conta Massa Molecular (MM)}
        \begin{itemize}[left=3em]
            \item[\(MM<\)] Tem um volume de eluição \(v_e = v_0 + v_i\) (todo o volume lhes é acessível)
            \item[\(<MM<\)] Tem volume de eluição proporcional ao logarítimo da massa molecular
            \item[\(MM>\)] Não penetram os poros e são eluidas primeiro \(v_e = v_0\) (apenas o volume exterior é acessível)
        \end{itemize}
        
    \end{sectionBox}
    
\end{sectionBox}


\end{document}
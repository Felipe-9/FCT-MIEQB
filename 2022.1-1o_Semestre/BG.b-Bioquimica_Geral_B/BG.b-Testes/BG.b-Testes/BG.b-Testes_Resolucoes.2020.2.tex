% !TEX root = ./BG.b-Testes_Resoluções.2020.2.tex
% !TEX root = ./BG.b-Testes_Resoluções.tex
\providecommand\mainfilename{"./BG.b-Testes_Resoluções.tex"}
\providecommand \subfilename{}
\renewcommand   \subfilename{"./BG.b-Testes_Resoluções.2020.2.tex"}
\documentclass[\mainfilename]{subfiles}

% \graphicspath{{\subfix{../images/}}}

\begin{document}

\mymakesubfile{2}[BG.b]
{Teste 2 2020\,--\,2021}
{Teste 2 2020\,--\,2021}

\setcounter{question}{2}
\begin{questionBox}1{}
    
    Uma das reações na glicólise onde se dá origem a ATP a partir de ADP + Pi é catalizada pela enzima

    Escolha a resposta correta
    \begin{enumerate}
        \item Enolase
        \item Fosfofrutoquinase
        \item Quinase do Piruvato
        \item Hexoquinase
    \end{enumerate}

    \vspace{-5ex}

    \begin{questionBox}*3{}
        
        \paragraph{Enolase}
        é uma enzima homodimérica essencial que catalisa a desidratação reversível de 2-fosfo-D-glicerato em fosfoenolpiruvato como parte das vias glicolítica e gliconeogênese.\\

        \paragraph{Fosfofrutoquinase}
        A fosfofrutoquinase catalisa a fosforilação da frutose-6-fosfato em frutose-1,6-bisfosfato, uma etapa reguladora fundamental na via glicolítica.\\

        \paragraph{Quinase do Piruvato}
        A piruvato quinase é a enzima envolvida na última etapa da glicólise. Catalisa a transferência de um grupo fosfato do fosfoenolpiruvato (PEP) para o difosfato de adenosina (ADP), produzindo uma molécula de piruvato e uma molécula de ATP.\\

        \paragraph{Hexoquinase}
        Uma hexoquinase é uma enzima que fosforila hexoses (açúcares de seis carbonos), formando fosfato de hexose. Na maioria dos organismos, a glicose é o substrato mais importante para as hexoquinases, e a glicose-6-fosfato é o produto mais importante. A hexoquinase possui a capacidade de transferir um grupo fosfato inorgânico do ATP para um substrato.
        
    \end{questionBox}

    \paragraph{RS:} 3
    
\end{questionBox}

\begin{questionBox}1{}
    
    Dentre os lipideos com revelancia biológica, escolha a descrição correta para as moléculas de colesterol

    \begin{enumerate}
        \item Derivam a reação de um ácido gordo com um álcool gordo e estão envolvidos, entre outros, na proteção exterior e impermeabilização de folhas e frutos.
        \item Estão envolvidos na modulação da fluidez das membranas biológicas
        \item Servem essencialmente como combustível metabólico
    \end{enumerate}

    \vspace{-5ex}

    \begin{questionBox}*3{}
        
        \paragraph{Colesterol}
        O colesterol é qualquer uma de uma classe de certas moléculas orgânicas chamadas lipídios. É um esterol (ou esteróide modificado), um tipo de lipídio. O colesterol é biossintetizado por todas as células animais e é um componente estrutural essencial das membranas celulares animais.
        
    \end{questionBox}

    \paragraph{RS:}2
    
\end{questionBox}

\begin{questionBox}1{}
    
    Identifique a afirmação {\color{red\Light}falsa}

    \paragraph{As membranas biológicas:}
    \begin{enumerate}
        \item Pode conter lipidos e proteínas glicosiladas
        \item Podem conter preteínas extrinsecas e intrinsecas
        \item São livremente permeáveis a iões como o \ch{Na+}
        \item São assimétricas em termos de composição e de função
    \end{enumerate}

    \vspace{-5ex}

    \begin{questionBox}*3{}
        
        Uma das funções da membrana é manter uma diferença de potencial entre o meio externo e interno para converter a energia com a transferencia de ions.
        
    \end{questionBox}

    \paragraph{RS:}3
    
\end{questionBox}

\begin{questionBox}1{}
    
    Selecione uma opção de resposta.\\
    
    Um passo fermentativo é acoplado à glicólise em condições anóxicas porque:
    \begin{enumerate}
        \item A fermentação está acoplada á formação de ATP a partir de ADP + Piruvato
        \item A fermentação asegura a reoxidação do NADH a NAD+ necessário no passo de insersão de fosfato inorgânico no gliceraldeído-3-fosfato.
        \item A fermentação induz a formação de Acetil CoA a partir do produto final piruvato
        \item A fermentação estimula a liberação de água no complexo IV da cadeia de TE mitocondrial
    \end{enumerate}

    \vspace{-5ex}
    \begin{questionBox}*3{}
        
        \paragraph{Regeneração anóxica de \ch{NAD+}}
        Um método de fazer isso é simplesmente fazer com que o piruvato faça a oxidação; neste processo, o piruvato é convertido em lactato (a base conjugada do ácido láctico) em um processo chamado fermentação de ácido láctico:\\

        \ch{Pyruvate + NADH + H+ -> lactate + NAD+}
        
    \end{questionBox}

    \paragraph{RS:}2
    
\end{questionBox}

\begin{questionBox}1{}
    
    \paragraph{No ciclo de Krebs ocorre:}
    \begin{enumerate}
        \item A ligação do piruvato à coenzima A acoplada à eliminação de \ch{CO2} e à redução de \ch{NAD+} a NADH
        \item A descarboxilação completa do grupo acetil na acetil coenzima A
        \item A reação degradativa da glucose e de outras hexoses com formação de duas moléculas de piruvato
        \item A hidrólise do amido que leva à formação de várias moléculas de piruvato
        \item A redução de oxigénio molecular a água na cadeia respiratória mitocondrial
    \end{enumerate}

    \vspace{-5ex}
    
    \begin{questionBox}*3{}
        
        \paragraph{Passos do ciclo Krebs}
        \begin{enumerate}
            \item O ciclo TCA começa com uma reação enzimática de adição aldólica de acetil CoA ao oxaloacetato, formando citrato.

            \item O citrato é isomerizado por uma sequência de desidratação-hidratação para produzir (2R,3S)-isocitrato.
            
            \item Mais oxidação enzimática e descarboxilação dão 2-cetoglutarato.
            
            \item Após outra descarboxilação enzimática e oxidação, o 2-cetoglutarato é transformado em succinil-CoA.
            
            \item A hidrólise deste metabólito em succinato é acoplada à fosforilação do difosfato de guanosina (GDP) em trifosfato de guanosina (GTP).
            
            \item A dessaturação enzimática pela succinato desidrogenase dependente de flavina adenina dinucleotídeo (FAD) produz fumarato.
            
            \item Após a hidratação estereoespecífica, o fumarato catalisado pela fumarase é transformado em L-malato.
            
            \item A última etapa da oxidação acoplada ao NAD do L-malato ao oxaloacetato é catalisada pela malato desidrogenase e fecha o ciclo.
        \end{enumerate}

        \paragraph{Produtos do Ciclo de Krebs}
        \begin{itemize}
            \begin{multicols}{2}
                \item 1 GTP
                \item 3 NADH
                \item \ch{FADH2} (convertido em \ch{UQH2} na presença da coenzima Q (ubiquinona))
                \item \ch{2 CO2}
            \end{multicols}
        \end{itemize}

        
    \end{questionBox}

    \paragraph{RS:}2
    
\end{questionBox}

\end{document}
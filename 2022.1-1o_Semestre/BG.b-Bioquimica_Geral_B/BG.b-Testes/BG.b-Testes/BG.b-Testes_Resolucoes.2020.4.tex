% !TEX root = ./BG.b-Testes_Resoluções.2020.4.tex
% !TEX root = ./BG.b-Testes_Resoluções.tex
\providecommand\mainfilename{"./BG.b-Testes_Resoluções.tex"}
\providecommand \subfilename{}
\renewcommand   \subfilename{"./BG.b-Testes_Resoluções.2020.4.tex"}
\documentclass[\mainfilename]{subfiles}

% \graphicspath{{\subfix{../images/}}}

\begin{document}

\mymakesubfile{4}[BG.b]
{Teste 2020--2021}
{Teste 2020--2021}

\setcounter{question}{27}
\begin{questionBox}1{}
    
    Num estudo com duas pessoas verificou-se que ao igerir a mesma quantidade de uma pão rico em amilose, o voluntário A experimento u pico de glucose em sangue duas vezes maior que o voluntario B e a concentração de glucose em sangue demorou mais 3 horas em atingir os memso níveis que para o voluntario B.

    \begin{enumerate}
        \item O voluntario B tem um problema genético que lhe impede traslocar os GUT-1 até a membrana
        \item O voluntario A tem um problema genético que lhe impede traslocar os GUT-4 até a membrana
        \item O voluntario A é resistente a insulina
        \item B é resistente
    \end{enumerate}

    \begin{sectionBox}*3{Resposta}
        
        Como o A ficou com a glucose armazenada é porque a insluna não ativou o GUT-1, logo tem um problema gnético e é resistente a insulina

        \paragraph{RS:} 2 \& 4
        
    \end{sectionBox}
    
\end{questionBox}

\begin{questionBox}1{}
    
    Assinale a verdadeira. 
    \paragraph{Na mitocôndria}
    \begin{enumerate}
        \item O bombeio de protões do espaço inter-membranar a matriz requere ade energia fornecida com a movimentação do citocromo C.
        \item O NADH transfere seus eletrões no complexo II
        \item O NADH transfere seus eletrões no complexo I
    \end{enumerate}

    \begin{questionBox}*3{}
        
        \paragraph{Rs:} 4
        
    \end{questionBox}
    
\end{questionBox}

\begin{questionBox}1{}
    
    Assinale a verdadeira.
    \paragraph{A ATP-ase:}
    \begin{enumerate}
        \item Produz ATP ao transferir protões do citosol a membrana mitocondrial extena.
        \item Produz ATP ao transferir protões do espaço inter-membranar a parte interna da mitocôndria (matriz)
        \item Produz ATP ao transferir protões da mitocondria (matriz) ao espaço inter-membranar
        \item Produz ATP ao tranferir protões da membrana mitocondrial extena ao citosol
    \end{enumerate}

    \begin{questionBox}*3{Resposta}
        
        \paragraph{Rs:} 3
        
    \end{questionBox}
    
\end{questionBox}

\begin{sectionBox}1{}
    
    ATPase é uma proteina integral da membrana que produz ATP (explicito pelo nome) usando a energia da transferencia de cargas da parte externa da membrana para a interna
    
    Bastante comum na mitocondria
    
\end{sectionBox}

\setcounter{question}{32}
\begin{questionBox}1{}
    
    Glicogenio é a forma de reserva de energia (glucosa) em animais. O glicogénio tem um lado reativo e redutor e outro lado não redutor. Onde se produz a rápida mobiliação metabólica da glucose quando precisamos de energia?

    \begin{enumerate}[label={(\alph*)}]
        \item No lado reativo e redutor
        \item no lado não reativo
        \item na zona media do glicogenio
        \item nas cadeias laterais \chemalpha 1-4
    \end{enumerate}

    \begin{questionBox}*3{title}
        
        \paragraph{Rs:} b
        
    \end{questionBox}
    
\end{questionBox}

\begin{questionBox}1{}
    
    \paragraph{No ciclo de Krebs ocorre:}
    \begin{enumerate}
        \item A ligação do piruvato a coenzima A acoplada a eliminação de \ch{CO2} e a redução de \ch{NAD^+} a NADH
        \item A dascarboxilação completa do grupo acetil na acetil coenzima A
        \item A reacção degradativa da glucose e de outras hxoses com formação de duas moléculas de piruvato
        \item A hidrólize do amido que leva a formação de varias moléculas de glucose
        \item A redução de oxigenio molecular a agua na cadeia respiratória mitocondrial
    \end{enumerate}

    \begin{questionBox}*3{}
        
        \paragraph{Rs:} 2
        
    \end{questionBox}
    
\end{questionBox}

\begin{questionBox}1{}
    
    Assinale a verdadeira
    \paragraph{Dentro da mitocondria}
    O espaço inter-membranar tem\dots

    \begin{enumerate}
        \item Mais protões que a parte interna da mitocondria (matriz)
        \item Menos protões que a parte interna da mitocondria
        \item Menos protões que a parte interna da mitocondria (matriz membranar)
        \item tem um pH maior que a parte interna da mitocondria
    \end{enumerate}

    \begin{questionBox}*3{Resposta}
        
        \paragraph{Rs:} 1
        
    \end{questionBox}

\end{questionBox}

\end{document}
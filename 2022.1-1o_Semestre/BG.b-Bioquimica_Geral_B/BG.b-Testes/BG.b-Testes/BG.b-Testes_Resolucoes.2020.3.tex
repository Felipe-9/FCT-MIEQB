% !TEX root = ./BG.b-Testes_Resoluções.2020.3.tex
% !TEX root = ./BG.b-Testes_Resoluções.tex
\providecommand\mainfilename{"./BG.b-Testes_Resoluções.tex"}
\providecommand \subfilename{}
\renewcommand   \subfilename{"./BG.b-Testes_Resoluções.2020.3.tex"}
\documentclass[\mainfilename]{subfiles}

% \graphicspath{{\subfix{../images/}}}

\begin{document}

\mymakesubfile{3}[BG.b]
{Resolução Teste 2020-2021}
{Resolução Teste 2020-2021}

\setcounter{question}{3}
\begin{questionBox}1{}
    
    O par D-Glucosa e D-Mannosa são:
    \begin{enumerate}[label={(\alph*)}]
        \item Anomeros
        \item Epímeros
        \item Um par aldosa-cetosa
        \item Um par cetosa-cetosa
    \end{enumerate}

    \begin{questionBox}*3{Resposta}
        
        \paragraph{Epimeros:}
        \begin{itemize}
            \item Dois esterioisômeros que diferem em apenas um carbono quiral.
        \end{itemize}
        \paragraph{Anomeros:}
        \begin{itemize}
            \item na forma de anel um epímero é chamado de anômero
            \item Um tipo especial de epimero
            \item um dos dois tipos de estereoisomeros de um sacarídeo cíclico
        \end{itemize}
        
    \end{questionBox}

    \paragraph{Rs:} (a)
    
\end{questionBox}

\setcounter{question}{6}
\begin{questionBox}1{}
    
    Escolha a opção correta
    \begin{enumerate}
        \item Os D-monossacáridos são típicamente aldoses, enquanto os L-sacáridos são cetoses
        \item Os D-monossacáridos tem sempre cinco ou mais átomos de carbono
        \item Os D-monossacáridos como o carbono anomerio live são redutores
        \item Tosos os D-polissacaridos são polímero lineares de unidades sacáridos identicas
    \end{enumerate}

    \begin{questionBox}*3{}
        
        Os monossacaridos são os compostos mais simples de carboidrato, possuindo de 3 a 7 carbonos
    \end{questionBox}

    \paragraph{RS:} 3
    
\end{questionBox}

\setcounter{question}{9}
\begin{questionBox}1{}
    
    Glicogenio é a forma de reserva de energia (glucose em animais). O glicogenio tem um lado reativo e redutor e outro lado não redutor.
    
    
    Onde se produz a rapida mobilização metabólica da glucose quando precisamos de energia?
    \begin{enumerate}[label=(\alph*)]
        \item No lado reativo e redutor
        \item No lado não reativo
        \item Na zona média do glicogenio
        \item Nas cadeias laterais \chemalpha 1-4
    \end{enumerate}

    \paragraph{RS:} (b)
    
\end{questionBox}

\begin{questionBox}1{}
    
    A rotação específica dos anomeros puros alpha e beta da D-glucosa é \(+112^\circ\) e \(+18.7^\circ\) respectivamente. Quando um cristal puro de \chemalpha--D glucopyranose é dissolvido em agua a rotação específica diminui em \(112^\circ\) a um valor  de equilibrio de \(52.7^\circ\). Quais a proporçòes dos anomeros alpha e beta no equilíbrio.

    \begin{center}
        \begin{tabular}{c *{2}{r}}
            
            \toprule
            
                & \multicolumn{1}{c}{\(\alpha\)/\%}
                & \multicolumn{1}{c}{\(\beta\)/\%}
            
            \\\midrule
            
                (a) & 36 & 67
              \\(b) & 64 & 36
              \\(c) & 36 & 64
              \\(d) & 67 & 33
            
            \\\bottomrule
            
        \end{tabular}
    \end{center}



    \begin{questionBox}*3{}
        
        \begin{flalign*}
            &
                52.7
                = \lambda_{\alpha}\,112
                + \lambda_{\beta} \,18.7
                \implies
                0.36*112
                + 0.64* 18.7
                \cong
                \num{52.288}
            &
        \end{flalign*}
        
    \end{questionBox}

    \paragraph{RS:} (c)
    
\end{questionBox}

\setcounter{question}{12}
\begin{questionBox}1{}
    
    Diga qual a afirmação verdadeira
    O sacarído \chemalpha D (glucopiranósido)-1, 4--D (glucopiranósido)

    \begin{enumerate}
        \item É um monosacárido em que o carbono anomérico tem configuração alpha
        \item É um dissacárido de glucose em que a ligação glicsídica é alpha 1-4
        \item É um dissacárido de glucose e galactose em que a ligação envolve o carbono anomérico da unidade glucose e o átomo de carbono C4 da unidade galactose
        \item É um oligossacárido de glucose com quatro unidades constituintes
    \end{enumerate}

    \begin{questionBox}*3{}
        
        
        
    \end{questionBox}

    \paragraph{RS:} 2
    
\end{questionBox}

\begin{questionBox}1{}
    
    Suponha que uma proteína tem 3 sítios diferentes para fazer ligações glicosídicas tipo N. quantas prteínas diferentes poderemos ter?
    
    (não tenha em consideração o tipo de carbohidrato que se poderia ligar).

    \begin{questionBox}*3{}

        \paragraph{Ligações Glicosidicas}
        \begin{itemize}
            \item São ligações que ligam um grupo carbohidrato (açucar) a um outro grupo que pode ou não ser outro carbohidrato
        \end{itemize}
        
        \paragraph{N-Glícosidicas}
        \begin{itemize}
            \item É uma ligação glicosídica onde o Oxigenio é substuido por um Nitrogenio
            \item Substancias contendo esse tipo de ligação são conhecidas por glicosilaminas
        \end{itemize}

        \paragraph{Carbono Anomérico}
        \begin{itemize}
            \item Carbono proximo de algo que não é um carbono em uma cadeia organica que tem a tendencia a ligar a um novo grupo (rever definição)
        \end{itemize}
        
    \end{questionBox}
    
\end{questionBox}

\begin{questionBox}1{}
    
    Diga qual das afirmações é falsa:
    \begin{enumerate}
        \item Os carbohidratos tambem designados por sacáridos, são aldeídos ou caetona com multiplos grupos \ch{-OH}
        \item Os carbohidratos, tambem designados por sacaridos, tem capacidade para ciclização interna dando origem a furanosidos ou piranosidos
        \item Muitos carboxilos 
    \end{enumerate}
    
\end{questionBox}

\setcounter{question}{19}
\begin{questionBox}1{}
    
    Quantos fosfolipideos ha num \unit{\milli\metre^2} de uma bicamada lipidica?

    Suponha que cada fsofoípido ocupa 70\,\unit{\angstrom^2} (\(1\,\unit{\angstrom}=1\E-10\,\unit{\metre}\))

    \begin{questionBox}*3{}
        \begin{flalign*}
            &
                1\,\unit{\micro\metre^2}
                * \left(
                    \frac
                        {1\,\unit{\angstrom}}
                        {1\E-10\,\unit{\metre}}
                \right)^2
                * \frac
                    {\text{lipideos}}
                    {70\,\unit{\angstrom^2}}
                * 2
                = &\\&
                = (1\E-12*1\E+20*2/70)\text{lipideos}
                = \qty{
                    2857142.857142857142857
                }{lipideos}
            &
        \end{flalign*}
    \end{questionBox}
    
\end{questionBox}

\end{document}
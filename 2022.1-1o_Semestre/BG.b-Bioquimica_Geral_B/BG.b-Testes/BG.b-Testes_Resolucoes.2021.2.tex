% !TEX root = ./BG.b-Testes_Resoluções.2.tex
% !TEX root = ./BG.b-Testes_Resoluções.tex
\providecommand\mainfilename{"./BG.b-Testes_Resoluções.tex"}
\providecommand \subfilename{}
\renewcommand   \subfilename{"./BG.b-Testes_Resoluções.2.tex"}
\documentclass[\mainfilename]{subfiles}

% \graphicspath{{\subfix{../images/}}}

\begin{document}

\mymakesubfile{2}[BG.b]
{Teste Prático}
{Teste Prático}

\setcounter{question}{4}
\begin{questionBox}1{}
    
    Uma solução mãe de proteína BSA de concentração 0.25\,\unit{\milli\gram/\milli\litre} é utilizada para preparar uma serie de standards como indicado a continuação.\\

    Volumes de Solução Mãe. Volume de Água Volume de reagente C Volume de reagente D são dados em \unit{\milli\litre}

    \begin{center}

        \setlength\tabcolsep{2mm}        % width
        \renewcommand\arraystretch{1.25} % height

        \begin{tabular}{*{5}{c}}
            
            \toprule
            
                \multicolumn{1}{c}{Vol. Sol. Mae}
               &\multicolumn{1}{c}{Vol. \ch{H2O}}
               &\multicolumn{1}{c}{Vol. Reag.C}
               &\multicolumn{1}{c}{Vol. Reag.D}
               &\multicolumn{1}{c}{Abs}
            
            \\\midrule
            
               0   & 120 & 200 & 1400 & 0.020
            \\ 20  & 100 & 200 & 1400 & 0.102
            \\ 40  &  80 & 200 & 1400 & 0.214
            \\ 60  &  60 & 200 & 1400 & 0.320
            \\ 100 &  20 & 200 & 1400 & 0.433
            \\ 120 &   0 & 200 & 1400 & 0.539
            
            \\\bottomrule
            
        \end{tabular}
    \end{center}

        
        100\,\unit{\milli\gram} de plasma são liofilizados e depois o sólido formado e redissolvido em 1\,\unit{\milli\litre}  de buffer.\\
        
        20\,\unit{\milli\litre} desta solução são levados a 120\,\unit{\milli\litre} com um buffer.\\

        A estos 120\,\unit{\milli\litre} são adicionados o reagente C e o reagente D nos mesmos volumes que as soluções estândar (200\,\unit{\milli\litre} e 1400\,\unit{\milli\litre} respetivamente).\\


        As absorbências desta última solução de (\(120+200+1400\)) foram 0.370, 0.375 e 0.360, (branco descontado) calcule a concentração de proteína total nas 100\,\unit{\milli\gram} originais de Plasma em \unit{\milli\gram\of{prot}/\milli\gram} de plasma.\\

    \begin{questionBox}*3{Resolução}

        \begin{itemize}
            \item 0.519 Abs prot + plasma
            \item \num{0.368333333333333} Abs plasma processado
        \end{itemize}

        \begin{flalign*}
            &
                \ch{[Prot]}
                = &\\&
                = \frac{Abs_{prot}}{Abs_{plas}}
                = \frac{0.519 - Abs_{plas}}{Abs_{plas}}
                = \frac{0.519}{Abs_{plas}}
                - 1
                = \frac{0.519}{
                    \num{0.368333333333333}
                    \frac{
                        20
                    }{
                        120+200+1400
                    }
                }
                - 1
                \cong &\\&
                \cong
                    \num{-0.718190045248869}
                % = &\\&
                % = \frac{
                %     \frac{
                %         \frac{
                %             0.25\,\unit{\milli\gram\of{Prot}}
                %         }{
                %             \unit{\milli\litre\of{Sol.1}}
                %         }
                %         * 20\,\unit{\milli\litre\of{Sol.1}}
                %     }{
                %         120\,
                %     }
                    % 
                %     * 120\,\unit{\milli\litre\of{Sol.1}}
                % }{
                %     120 + 200 + 1400
                % }
                % = &\\&
                % = 
                % \frac{
                %     \frac{100\,\unit{\milli\gram\of{plasma}}}{1\,\unit{\milli\litre\of{1}}}
                %     * 20\,\unit{\milli\litre\of{2}}
                %     }{
                %     (
                %         120
                %         + 200
                %         + 1400
                %     )\,\unit{\milli\litre\of{3}}
                % }
            &
        \end{flalign*}
        
    \end{questionBox}
    
\end{questionBox}

\setcounter{question}{14}
\begin{questionBox}1{}
    
    Uma coluna cromatográfica de separação por tamanho tem um diâmetro de 1\,\unit{\centi\metre} e a altura do leito de resina e de 17\,\unit{\centi\metre}. Sendo que 5\,\unit{\milli\litre} é volume com o qual é extraída a molécula de tamanho superior ao tamanho máximo de separação da resina, calcule qual será o volume de extração para a molécula que marca o tamanho inferior da resina de separação.

    \begin{enumerate}[label=\alph*.]
        \begin{multicols}{2}
            \item  8.74\,\unit{\milli\liter}
            \item 49.98\,\unit{\milli\liter}
            \item  9.25\,\unit{\milli\liter}
            \item 54.97\,\unit{\milli\liter}
        \end{multicols}
    \end{enumerate}

    Tamanho inferior: \(V = V_i + V_0\)

    \begin{flalign*}
        &
            Vol 
            = Vol_i + Vol_0
            = Vol_i + 5\,\unit{\milli\liter};
        &\\&
            Vol_t
            = -Vol_{res}
            + \pi
            \,((1/2)\,\unit{\centi\metre})^2
            \,17\,\unit{\centi\metre}
            = -Vol_{res}
            + \qty{13.35}{\centi\metre^3}
            \implies &\\&
            \implies 
            Vol_i
            = \qty{13.35}{\centi\metre^3}
            - 5\,\unit{\centi\meter^3}
            - Vol_{res}
            = \qty{8.35}{\centi\metre^3}
            - Vol_{res}
        &
    \end{flalign*}
    
\end{questionBox}


\end{document}
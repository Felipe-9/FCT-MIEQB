% !TEX root = ./BG.b-Testes_Resoluções.3.tex
% !TEX root = ./BG.b-Testes_Resoluções.tex
\providecommand\mainfilename{"./BG.b-Testes_Resoluções.tex"}
\providecommand \subfilename{}
\renewcommand   \subfilename{"./BG.b-Testes_Resoluções.3.tex"}
\documentclass[\mainfilename]{subfiles}

% \graphicspath{{\subfix{../images/}}}

\begin{document}

\mymakesubfile{3}[BG.b]
{Teste 3 2021\,--\,2022 Resolução}
{Teste 3 2021\,--\,2022 Resolução}

\begin{questionBox}1{} % Q1
    
    Identifique a resposta correcta.

    \paragraph{A glicólise é:}
    \begin{enumerate}
        \item A ligação do  piruvato  à coenzima A  acoplada à eliminação de CO2 e à redução de NAD+ a NADH 
        \item A descarboxilação completa do grupo acetil na acetil coenzima A 
        \item A reacção degradativa da glucose e de outras hexoses com formação de duas moléculas de piruvato 
        \item A hidrólise do amido que leva à formação de várias  moléculas de glucose 
        \item A redução de oxigénio molecular a água  na cadeia respiratória mitocondrial
    \end{enumerate}

    \paragraph{RS:} 3
    
\end{questionBox}

\begin{questionBox}1{} % Q2
    
    Escolha a opção correcta:
    \begin{enumerate}
        \item Os D-monossacáridos  são típicamente aldoses, enquanto os L-sacáridos são cetoses
        \item Os D-monossacáridos  têm sempre cinco ou mais átomos de carbono 
        \item Os D-monossacáridos com o carbono anomérico livre são redutores 
        \item Todos os D-polissacáridos são polímeros lineares de unidades sacárido idênticas 
    \end{enumerate}

    \paragraph{RS:}
    
\end{questionBox}

\begin{questionBox}1{} % Q3
    
    Uma das reacções na glicólise onde se dá origem a ATP a partir de ADP+Pi é catalizada pela enzima\\
    
    Escolha a resposta certa
    \begin{enumerate}
        \item enolase
        \item fosfofrutocinase.
        \item cinase do piruvato.
        \item hexocinase.
    \end{enumerate}

    \paragraph{RS:} 3
    
\end{questionBox}

\begin{questionBox}1{} % Q4
    
    Assinale as afirmações verdadeiras
    \begin{enumerate}
        \item A energia nos organismos vivos e intercambiada COM RECURSO a forma de uma O-glicosilação.
        \item A energia nos organismos vivos e intercambiada COM RECURSO a forma de uma N-glicosilação.
        \item A energia nos organismos vivos e intercambiada COM RECURSO a uma adenosine diphospate. 
        \item A energia nos organismos vivos e intercambiada COM RECURSO a uma adenosine triphospate. 
        \item A energia nos organismos vivos e intercambiada COM RECURSO A enlace fosfato. 
    \end{enumerate}

    \paragraph{RS:}
    
\end{questionBox}

\begin{questionBox}1{} % Q5

    Considere a glicólise a partir da molécula glucose:\\
    
    Assinale verdadeira/s:
    \begin{enumerate}
        \item A glicólise produz 2 ATPs e consume 2 ATPS
        \item A glicólise produz 4 ATPs e consume 2 ATPS
        \item glicólise produz 2ATPs e consume 4 ATPS
        \item  glicólise produz  2 moléculas de piruvato
        \item A glicólise produz  4 moléculas de piruvato
        \item A glicólise produz  1 moléculas de piruvato
    \end{enumerate}

    \paragraph{RS:} 2 e 4
    
\end{questionBox}

\begin{questionBox}1{} % Q6
    
    Quantas moléculas de ATP (descontadas as consumidas) são formadas por degradação glicolítica de 20 moléculas de glucose, seguida de fermentação láctica?\\

    Selecione uma opção de resposta:
    \begin{enumerate}
        \begin{multicols}{4}
            \item 40
            \item 100
            \item 30
            \item 20
        \end{multicols}
    \end{enumerate}

    \vspace{-5ex}

    \begin{questionBox}*3{}
        
        \begin{flalign*}
            &
                20*4
            &
        \end{flalign*}
        
    \end{questionBox}

    \paragraph{RS:} 2
    
\end{questionBox}

\begin{questionBox}1{} % Q7
    
    Assinale a verdadeira: No caminho metabólico central, a formação de ATP catalizada por cinases (fosforilação a nível de substracto) dá-se\\

    Selecione uma opção de resposta:
    \begin{enumerate}
        \item Na reacção catalizada pela enzima piruvatocinase 
        \item Na reacção catalizada pela enzima fosfofrutocinase
        \item Na reacção catalizada pela enzima piruvato desidrogenase
        \item No complexo V da cadeia respiratória mitocondrial (F0F1 ATPase)
    \end{enumerate}

    \paragraph{RS:} 1
    
\end{questionBox}

\begin{questionBox}1{} % Q8
    
    Considere a temperatura de fusão (Tm, melting point) dos seguintes troços de DNA duplex.\\

    \begin{center}
        \begin{tabular}{c c}
            
            \toprule
            
                1 & \begin{tabular}{c}
                    GAAATTTC
                 \\ CTTTAAAG
                \end{tabular}
             \\ \midrule
                2 & \begin{tabular}{c}
                    GCCATGGC
                 \\ CGGTACCG
                \end{tabular}
             \\ \midrule
                3 & \begin{tabular}{c}
                    GCGCGCGC
                 \\ CGCGCGCG
                \end{tabular}
             \\ \midrule
                4 & \begin{tabular}{c}
                    GGAATTCC
                 \\ CCTTAAGG
                \end{tabular}
            
            \\\bottomrule
            
        \end{tabular}
    \end{center}

    Diga qual a resposta correcta:
    \begin{enumerate}[label={\alph*)}]
        \item \(Tm1>Tm2>Tm3>Tm4\)
        \item \(Tm3>Tm2>Tm4>Tm1 \)
        \item \(Tm2>Tm3>Tm4>Tm1\)
        \item \(Tm4>Tm3>Tm2>Tm1\)
    \end{enumerate}

    \begin{questionBox}*3{}
        
        \begin{flalign*}
            &
                \left.
                    \begin{aligned}
                        1 & 2
                    \\ 2 & 6
                    \\ 3 & 8
                    \\ 4 & 4
                    \end{aligned}
                \right\}
                \implies
                Tm3
                > Tm2
                > Tm4
                > Tm1
            &
        \end{flalign*}
        
    \end{questionBox}

    \paragraph{RS:} b
    
\end{questionBox}

% Q9
\begin{questionBox}1{}
    
    Qual dos seguintes compostos / moléculas / complexos não é necessário na síntese de proteínas:

    \begin{enumerate}[label={\alph*)}]
        \begin{multicols}{3}
            \item Ribossoma
            \item peptidil transferase
            \item spliceosoma
            \item tRNA 
            \item metionina
        \end{multicols}
    \end{enumerate}

    \vspace{-5ex}

    \begin{questionBox}*3{}
        
        % \paragraph{Ribossoma}
        
    \end{questionBox}

    \paragraph{RS:}
    
\end{questionBox}

\begin{questionBox}1{}
    
    Um passo fermentativo é acoplado à glicólise em condições anóxicas porque:

    \begin{enumerate}
        \item A fermentação está acoplada à formação de ATP a partir de ADP + Pi 
        \item A fermentação asegura a reoxidação  do NADH a NAD+  necessário  no passo de insersão de fosfato inorgânico  no gliceraldeído-3-fosfato.   
        \item A fermentação induz a formação de Acetil CoA a partir do produto final piruvato 
        \item A fermentação estimula a libertação de água no complexo IV da cadeia de TE mitocondrial
    \end{enumerate}

    \paragraph{RS:} 2
    
\end{questionBox}

\begin{questionBox}1{}
    
    A glicólise envolve:
    \begin{enumerate}
        \item 10 passos e 09 enzimas, finalizando em 2 moléculas de Piruvato.
        \item 10 passos e 10 enzimas, finalizando em 2 moléculas de Piruvato.
        \item 10 passos e 10 enzimas, finalizando em 2 moléculas de Lactato.
        \item 10 passos e 10 enzimas, finalizando em 2 moléculas de lactato.
    \end{enumerate}

    \paragraph{RS:} 2
    
\end{questionBox}

\begin{questionBox}1{}

    Selecione uma opção de resposta:\\

    O complexo de pré-iniciação de síntese de proteínas em procariontes e composto por:
    \begin{enumerate}[label={\alph*)}]
        \item Fatores de iniciação, mRNA, 30S subunit, 50S subunit, ATP
        \item Fatores de iniciação, mRNA, 30S subunit, GTP
        \item Fatores de iniciação, 30S subunit, 50S subunit, ATP
        \item Fatores de iniciação, mRNA, 50S subunit, GTP
        \item Fatores de iniciação, mRNA, 30S subunit, 50S subunit, GTP
        \item  sem resposta
    \end{enumerate}

    \paragraph{RS:}
    
\end{questionBox}

\begin{questionBox}1{}
    
    Considere o diagrama do fluxo de informação genética dos organismos vivos:
    \begin{center}
        \ch{3 DNA ->[1] RNA ->[2] Proteína}
    \end{center}

    Diga qual a resposta certa:
    \begin{center}[H]\centering
        \begin{tabular}{c *{3}{l}}
            
            \toprule
            
                & \multicolumn{1}{c}{Passo 1}
                & \multicolumn{1}{c}{Passo 2}
                & \multicolumn{1}{c}{Passo 3}
            
            \\\midrule
            
            1. & Replicação  & transcrição & tradução
         \\ 2. & transcrição & replicação  & tradução
         \\ 3. & Tradução    & transcrição & Replicação
         \\ 4. & Transcrição & tradução    & Replicação
         \\ 5. & Replicação  & tradução    & transcrição
            
            \\\bottomrule
            
        \end{tabular}
    \end{center}

    \paragraph{RS:} 4
    
\end{questionBox}

\begin{questionBox}1{}
    
    Selecione uma ou mais opções de resposta\\

    Assinale as afirmações verdadeiras:
    \begin{enumerate}
        \item A ATPsynthase e considerada constituída por duas aprtes principais a F0 integrada na membrana e a F1 integrada no espaço fora da membrana
        \item A ATPsynthase e considerada constituída por duas partes principais a F1 integrada na membrana e a F0 integrada no espaço de fora da membrana 
        \item A ATPsynthase faz fluir protões desde o espaço intermembranar para a matriz 
        \item A ATPsynthase faz fluir protões desde a matriz na direção do espaço intermembranar 
    \end{enumerate}

    \paragraph{RS:} 3 e (1 ou 2)
    
\end{questionBox}

\begin{questionBox}1{}
    
    Selecione uma opção de resposta\\

    Na figura seguinte apresenta-se um diagrama para o processo de transferência electrónica em  bactérias oxidantes de sulfureto.\\

    [Figura]\\

    Neste esquema, uma entidade com função semelhante ao complexo IV da cadeia respiratória mitocondrial
    \begin{enumerate}
        \item coresponde ao troço integrado no parêntesis 1
        \item coresponde ao troço integrado no parêntesis 3 
        \item coresponde ao troço integrado no parêntesis 4
        \item não existe
    \end{enumerate}

    \paragraph{RS:} 3 ou 4
    
\end{questionBox}

\begin{questionBox}1{}

    Selecione uma opção de resposta\\
    
    Assinale a afirmação verdadeira
    \begin{enumerate}
        \item A ATPsynthase produze aproximadamente 60 kg de ATP por pessoas por dia.
        \item A ATPsynthase produze aproximadamente 6 kg de ATP por pessoas por dia.
        \item A ATPsynthase produze aproximadamente 60 g de ATP por pessoas por dia.
        \item A ATPsynthase produze aproximadamente 6 g de ATP por pessoas por dia.
        \item A ATPsynthase produze aproximadamente 600 g de ATP por pessoas por dia.
    \end{enumerate}

    \begin{questionBox}*3{}
        
        \paragraph{ATPSintase} diariamente produz ATP em peso equivalente ao peso do individuo \(\cong 60\,\unit{\kilo\gram}\) por humano adulto
        
    \end{questionBox}

    \paragraph{RS:} 1
    
\end{questionBox}

\begin{questionBox}1{}
    
    Diga qual a reposta {\color{red\Light}errada}
    \begin{enumerate}
        \item A tradução resulta na síntese de proteína, ocorre nos ribossomas e envolve rRNA, mRNA e tRNA
        \item Na tradução a informação codificada em triades no mRNA  interage com tríades complementares  em  tRNAs originando a  síntese de proteína
        \item Na tradução a síntese de uma proteína envolve  tRNAs  ligados a diferentes aminoácidos 
        \item Na tradução a informação codificada nas triades no DNA é lida por tríades complementares  em   tRNAs  originando a síntese de proteína 
    \end{enumerate}


    
\end{questionBox}

\begin{questionBox}1{Nula}
    
    Cinco amostras de DNA duplex  isoladas de diferentes  estirpes (A-D) de bactérias apresentam as seguintes percentagens de guanina: 

    \begin{enumerate}[
        label={Estirpe \Alph*.}, 
        left=1em
    ]
        \begin{multicols}{2}
            \item 40\%
            \item 35\%
            \item 30\%
            \item 25\%
            \item 20\%
        \end{multicols}
    \end{enumerate}

    Indique a resposta falsa
    \begin{enumerate}
        \item A amostra  cujo DNA   tem  35\%  de resíduos adenina é a da  estirpe B
        \item A amostra  cujo DNA   tem  20\%  de resíduos adenina é da estirpe C
        \item A amostra  cujo DNA   tem  30\% de resíduos citosina  é a da  estirpe C
        \item A amostra com temperatura de fusão mais elevada é a amostra E 
    \end{enumerate}

    \dots
    
\end{questionBox}

\begin{questionBox}1{}
    
    O número de subunidades do ribossoma numa célula humana é de:
    \begin{enumerate}[label=\alph*)]
        \begin{multicols}{3}
            \item 1
            \item 2
            \item 3
            \item 4
            \item 5
        \end{multicols}
    \end{enumerate}

    \paragraph{RS:} b)
    
\end{questionBox}

% Q20
\begin{questionBox}1{}
    
    A formação do enlace peptídico entre aminoácidos numa cadeia polipeptídico no ribossoma em formação no ribossoma é catalisada pela:
    \begin{enumerate}[label={\alph*)}]
        \begin{multicols}{2}
            \item Peptidyl transferase
            \item Amino acyl-tRNA systhetase
            \item Peptide polymesase
            \item Peptidyl synthesase
            \item Peptidyl nuclease
            \item sem resposta
        \end{multicols}
    \end{enumerate}

    \paragraph{RS:}
    
\end{questionBox}

% Q21
\begin{questionBox}1{}
    
    Identifique a resposta correcta.\\

    Na glicólise em organismos aeróbicos, o piruvato é transportado para as mitocôndrias  e convertido em:
    \begin{enumerate}
        \begin{multicols}{2}
            \item Acetil CoA 
            \item Etanol
            \item Lactato
            \item Glucose
        \end{multicols}
    \end{enumerate}

    \paragraph{RS:} 3
    
\end{questionBox}

% Q22
\begin{questionBox}1{}
    
    Identifique a afirmação verdadeira\\

    ``No Ciclo de Krebs dá-se:''
    \begin{enumerate}
        \item A descarboxilação completa do grupo acetil na acetil coenzima A 
        \item A reacção degradativa da glucose e de outras hexoses com formação de duas moléculas de piruvato
        \item A hidrólise do amido que leva à formação de várias  moléculas de glucose 
        \item O transporte de electões que leva à redução final de oxigénio molecular a água 
    \end{enumerate}

    \paragraph{RS:} 2
    
\end{questionBox}

% Q23   
\begin{questionBox}1{}
    
    Que moléculas produzidas na glicólise são usadas na fermentação láctica? 
    \begin{enumerate}
        \begin{multicols}{2}
            \item glucose, ATP e  NAD+
            \item piruvato e  ATP
            \item acetil CoA e NADH
            \item piruvato e NADH 
            \item lactato, ATP e  CO2
        \end{multicols}
    \end{enumerate}

    \paragraph{RS:} 4
    
\end{questionBox}

% Q24
\begin{questionBox}1{}
    
    No ciclo de Krebs, o grupo acetil na  acetilCoA é totalmente convertida em
    \begin{enumerate}
        \begin{multicols}{2}
            \item Duas moléculas de CO2
            \item Duas moléculas de H2O
            \item Uma molécula de piruvato
            \item Uma molécula de oxaloacetato
            \item Dois protões e dois electrões
        \end{multicols}
    \end{enumerate}

    \paragraph{RS:} 1
    
\end{questionBox}

% Q25
\begin{questionBox}1{}
    
    Os aminoácidos de uma proteína podem-se determinar no DNA pela ordem de:
    \begin{enumerate}[label={\alph*)}]
        \begin{multicols}{2}
            \item rRNA
            \item tRNA
            \item Nucleotidos
            \item mRNA
            \item anticodões
            \item sem resposta
        \end{multicols}
    \end{enumerate}

    \paragraph{RS:} c)
    
\end{questionBox}

\begin{questionBox}1{}
    
    No caminho metabólico central, a libertação de CO2 dá-se:
    \begin{enumerate}
        \item No complexo IV da cadeia respiratória mitocondrial
        \item Em dois dos passos da glicólise 
        \item Em dois dos passos do ciclo de Krebs 
        \item Em um dos passos do ciclo de Krebs
    \end{enumerate}

    \paragraph{RS:} 4
    
\end{questionBox}

\begin{questionBox}1{}
    
    Diga qual das afirmações é {\color{red\Light}falsa}
    \begin{enumerate}
        \item Os  carbohidratos,  também designados por sacáridos, são aldeídos ou cetonas com múltiplos grupos –OH  
        \item Os  carbohidratos,  também designados por sacáridos, são todos aldoses 
        \item Muitos  carbohidratos,  também designados por sacáridos, têm capacidade para ciclização interna dando origem a furanósidos ou piranósidos 
        \item Muitos  carbohidratos,  também designados por sacáridos, têm fórmula bruta Cn(H2O)n
    \end{enumerate}

    \paragraph{RS:}
    
\end{questionBox}

\begin{questionBox}1{}
    
    Identifique a resposta correcta.\\

    Nos produtos finais da glicólise de uma hexose  contam-se duas moléculas de, duas moléculas de e duas moléculas de.
    \begin{enumerate}
        \item ATP, gliceraldeído 3-fosfato, piruvato
        \item Água , gliceraldeído 3-fosfato, piruvato
        \item ATP, NADH, piruvato 
        \item Água, dióxido de carbono, glucose 
    \end{enumerate}

    \paragraph{RS:} 3
    
\end{questionBox}

\begin{questionBox}1{}
    
    Identifique as ligações corretas:
    \begin{enumerate}
        \begin{multicols}{2}
            \item NADH e complexo I 
            \item NADH e complexo II 
            \item NADH e complexo III
            \item FADH e complexo I
            \item FADH e complexo II 
            \item FADH e complexo III
        \end{multicols}
    \end{enumerate}

    \paragraph{RS:} 1, 2 e 6
    
\end{questionBox}

\begin{questionBox}1{}
    
    Identifique as afirmações correctas:
    \begin{enumerate}
        \item Os complexos 1, 2, 3 bombam protões desde a membrana ao espaço intermembranar
        \item Os complexos 1, 3 e 4 bombam protões  desde a membrana ao espaço intermembranar
        \item O complexo 1 não bomba protões  desde a membrana ao espaço intermembranar
        \item O complexo 2 não bomba protões  desde a membrana ao espaço intermembranar
    \end{enumerate}

    \paragraph{RS:} 3 e 4
    
\end{questionBox}

\begin{questionBox}1{}
    
    Uma amostra de DNA contem 180 000 pares de bases, com um conteúdo em G+C de 32,5\%. Quantos grupos fosfato tem a amostra
    \begin{enumerate}[label={\alph*)}]
        \begin{multicols}{2}
            \item 180 000
            \item 260 000
            \item 90 000
            \item 720 000
            \item 360 000
            \item sem resposta
        \end{multicols}
    \end{enumerate}

    \begin{questionBox}*3{}
        
        \begin{flalign*}
            &
                2*180000= 360000
            &
        \end{flalign*}
        
    \end{questionBox}

    \paragraph{RS:} e)
    
\end{questionBox}

\begin{questionBox}1{}
    
    Duas amostras de DNA (A e B) foram hidrolisadas. As bases constituintes foram separadas por cromatografia em papel e eluídas separadamente com 10 ml de água. Mediu-se a absorvância A260 nm das quatro soluçõeseluídas, obtendo-se os resultados seguintes:\\
    
    Qual o numero de moles de adenina em cada amostra
    \begin{enumerate}
        \begin{multicols}{2}
            \item A: 1E-5 B: 7E-6
            \item A: 1E-6 B: 7E-7 
            \item A: 1E-7 B: 7E-8 
            \item A: 1E-9 B: 7E-9
            \item A: 1E-10 B: 7E-7
            \item Todas as respostas estão erradas
        \end{multicols}
    \end{enumerate}

    \paragraph{RS:} 1
    
\end{questionBox}

\begin{questionBox}1{}
    
    Assinale as afirmações verdadeiras
    \begin{enumerate}
        \item In ATPsynthase the catalytic unit is made of a dimer of subunits and there are three of these arranged in a ring.
        \item In ATPsynthase the catalysis (conversion of ATP in ADP + Pi) occurs at the interface between the dimmers.
        \item If complex 1 stops, the OXPHO cycle stops.
        \item If complex 2 stops, the OXPHO cycle stops.
        \item Water is formed in Complex III
        \item A healthy human produces about 7 litres of water per day.
        \item A lack of oxygen renders the OXPHO machinery jumping the electrons to COMPLEX V via cytochrome C.
    \end{enumerate}

    \paragraph{RS:} 6 e 5
    
\end{questionBox}

\begin{questionBox}1{}
    
    DNA Gyrase tem a função de:
    \begin{enumerate}[label={\alph*)}]
        \item Evitar que a doble hélice do DNA desestabilize.
        \item Catalisa a adição de novos nucleótidos.
        \item Ajuda a colocar a DNA polimerase no seu lugar durante a replicação.
        \item Colocar os primers no seu lugar para que a DNA polimerase inicie o a replicação.
        \item Ajuda a leading strand durante o processo de replicação.
        \item Sem resposta
    \end{enumerate}

    \paragraph{RS:} c)
    
\end{questionBox}

\begin{questionBox}1{}
    
    Identifique a afirmação incorrecta\\

    \paragraph{RS:} 2

    
\end{questionBox}

\begin{questionBox}1{}
    
    No ciclo de krebs ocorre:
    \paragraph{RS:} 2
    
\end{questionBox}

\begin{questionBox}1{}
    
    Mitocondria

    \paragraph{RS} 4
    
\end{questionBox}

\end{document}
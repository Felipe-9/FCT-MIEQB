% !TEX root = ./BG.b-Aulas_Anotações.11_Anotações.tex
% !TEX root = ./BG.b-Aulas_Anotações.tex
\providecommand\mainfilename{"./BG.b-Aulas_Anotações.tex"}
\providecommand \subfilename{}
\renewcommand   \subfilename{"./BG.b-Aulas_Anotações.T.11.tex"}
\documentclass[\mainfilename]{subfiles}

% \graphicspath{{\subfix{../images/}}}

\begin{document}

\mymakesubfile{11}
{BG.b 17/05/22}
{BG.b 17/05/22}

\begin{sectionBox}1{Krebs}
    
    \href
    {https://www.youtube.com/watch?v=rdF3mnyS1p0}
    {Electron transport chain}

    Parece com a glicolise

    Serie de reações 

    Libera GTP que será transformado em ATP

    Foco em unir a glucose com o ciclo de krebs para formar o atp

    \paragraph{Resultados:}
    \begin{itemize}
        \item 3x \ch{NAOH}
        \item 1x \ch{FADH2}
        \item 1x \ch{GTP}
        \item 2x \ch{CO2}
    \end{itemize}
    
\end{sectionBox}

\begin{sectionBox}2{Anfibolico}
    
    The term amphibolic is used to describe a biochemical pathway that involves both catabolism and anabolism. Catabolism is a degradative phase of metabolism in which large molecules are converted into smaller and simpler molecules, which involves two types of reactions
    
\end{sectionBox}

\begin{sectionBox}2{Anaperotico}
    
    body
    
\end{sectionBox}

\begin{sectionBox}1{Mitocondrias}
    
    Processo de energia gera ATP

    Acontece em uma cadeia de 5 proteínas que em armonia conseguem transportar eletrons do exterior para o interior da mitocondria produzindo agua e ATP
    
\end{sectionBox}

\end{document}
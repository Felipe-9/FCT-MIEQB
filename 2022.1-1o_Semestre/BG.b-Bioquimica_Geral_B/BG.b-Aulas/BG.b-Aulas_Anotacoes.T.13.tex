% !TEX root = ./BG.b-Aulas_Anotações.13_Anotações.tex
% !TEX root = ./BG.b-Aulas_Anotações.tex
\providecommand\mainfilename{"./BG.b-Aulas_Anotações.tex"}
\providecommand \subfilename{}
\renewcommand   \subfilename{"./BG.b-Aulas_Anotações.T.13.tex"}
\documentclass[\mainfilename]{subfiles}

% \graphicspath{{\subfix{../images/}}}

\begin{document}

\mymakesubfile{13}
{BG.b Aula 24/05/22}
{BG.b Aula 24/05/22}

\begin{sectionBox}1{Ciclo de Cori}
    
    coordenação entre figado, coração e musculos em estresse intenso como um ``sprint''
    
    \begin{multicols}{2}
    
        \begin{sectionBox}2{Piruvato}
            
            Formar muito pirubato no coração aculmula lactato e causa infarto
            
        \end{sectionBox}
    
        \begin{sectionBox}2{Coração}
            
            Ciclo não produz lactato
            
        \end{sectionBox}
    
        \begin{sectionBox}2{Figado}
            
            Lactato -> piruvato ou glucose que é retornado ao sangue
            
        \end{sectionBox}

    \end{multicols}

\end{sectionBox}    


\begin{sectionBox}1{Feedback inibihition}
    
    Regula a velocidade de uma cadeia de reação para encaixar com as necessidades da celula

    Receptores se ligam ao substrato de uma proteína evitando a continuação da reação da proteína, a quantidade de receptores regula o numero de proteínas afetadas e controla a velocidade do processo todo

    \begin{sectionBox}2{Kinases}
        
        Transmissaão de sinais entre celulas
    
        \begin{sectionBox}3{Phosphorylation}
            
            dos 3 phosfatos do ATP a terceira pode ser fosforilada (perde o terceiro fosfato)
            
        \end{sectionBox}
        
    \end{sectionBox}
    
\end{sectionBox}



\end{document}
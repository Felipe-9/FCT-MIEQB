% !TEX root = ./Main_File.tex
% !TEX root = ./Sub_File.tex
\providecommand\mainfilename{"././Main_File.tex"}
\providecommand \subfilename{}
\renewcommand   \subfilename{"./Sub_File.tex"}
\documentclass[\mainfilename]{subfiles}

% \graphicspath{{\subfix{../images/}}}

\begin{document}

\mymakesubfile{1}{Cytoescape}{Cytoescape}

\Part{Aminoácidos}

\begin{sectionBox}{Aminoácidos}
    
    \begin{itemize}
        \item Essenciais
        \item Não Essenciais
        \item Limitantes
    \end{itemize}

    \begin{sectionBox}*2{Representação}
        
        \begin{itemize}
            \item \ch{AH2^+}
            \item \ch{AH ^0}
            \item \ch{A  ^-}
        \end{itemize}
        
    \end{sectionBox}
    
\end{sectionBox}

\begin{sectionBox}1{Glicina}
    
    % desenhar glicina
    \ch{NH3CH2COO}

    \begin{sectionBox}2{}
        
        Em pH neutro a glicina se encontra em \ch{AH^0} e se encadeiam ligando grupos carboxilos (\ch{COO}) e aminas (\ch{NH3^+})
        
    \end{sectionBox}
    
\end{sectionBox}

\part*{Proteinas}

\begin{sectionBox}1{Formação}
    
    Conjunto de Aminoácidos em cadeia
    Se forma em risossomos dentro de celulas
    Ao se formar adquire forma terciária (compacta)

    
\end{sectionBox}

\begin{sectionBox}2{Formas}
    
    Prim, Sec, Ter
    
\end{sectionBox}

\begin{sectionBox}2{Desnaturação}
    
    Retorna proteina para forma secundaria onde uma enzima pode quebrar em pequenos pedaços, os pedaços são chamados de peptinas
    
\end{sectionBox}

\begin{sectionBox}1{Distinção}
    
    Espectometro de massa, se pesa os pedaços e compara os pesos com uma base de dados de peptideos com massas conhecidas
    
\end{sectionBox}

\begin{questionBox}*1{Análize de Urina}
    
    Se usa a análize da quantidade de proteinas na urina, para diferenciar passientes com infeccoes caso estes sejam do tipo recorrentes e não recorrentes.

    tudo com base na quantidade de proteinas esperadas de infecções presentes na urina.
    
\end{questionBox}

\part*{Inicio da aula Prática}

\begin{sectionBox}1{Proteinas}
    
    Pesam serca sempre \(>10\,\unit{\kilo\dalton}\), e os espectrometros de massa não conseguem medir massas tão altas e por isso é necessário esperdaçalas em peptideos.

    Desnaturação lineariza uma proteína facilitando o corte dos peptideos

    Pode se fazer a desnaturação usando temperatura.
    
\end{sectionBox}

\begin{sectionBox}1{Introdução ao Protocolo prático}
    
    Estudo da interação de um metal (Cadmio) com ratos.

    O estudo será feito sobre as proteínas presentes em figados de ratos de dois grupos (controle e experimento)
    
\end{sectionBox}

\begin{sectionBox}*2{Volcano Plot}
    
    Curva usada para apresentar o nível de significancia,

    Diferenciando as proteínas que são afetadas dentre o grupo controle e experimento

    y: \(-log(p)\), onde \textit{p} é a significancia de geralmente (\(<5\E-2\))
    x: Diff

    Categoriza as proteínas pelas:
    \begin{itemize}
        \item Almentadas pela exposição a cadmio
        \item Diminuidas pela exposição a cadmio
    \end{itemize}
    
\end{sectionBox}

\begin{sectionBox}*2{ClueGO Plot}
    
    Ao se plotar as proteinas que possuem almento e diminuição podemos verificar os processos biologicos em que as proteínas dos dados coletados participam, e apresentar as relações entre os Processos comparando as semelhancas entre as proteínas participadoras.

    
\end{sectionBox}

\begin{sectionBox}*2{ROS}
    
    Reactive Oxidation Stress-induced

    Moleculas altamente perigosas apara celulas, ativa auto destrução celular.

    Afeta as enzimas que degradam as moleculas intermediárias do metabolizmo de mitocondrias, processo que transforma O e \ch{H2O2}, acumulando esses compostos que consequentimente induzem morte celular.

    Similarmente não é ideal usar agua oxigeáda(peroxido de hidrogênio), melhor alternativa iodo beta-ativo
    
\end{sectionBox}

\begin{sectionBox}*2{Cellular Hormone Metabolic Process}
    
    Relacionado com a glandulas que controlam series de orgãos que produzem hormonos,

    Metais pesados são conhecidos por pertubarem os eixos hormonais
    
\end{sectionBox}

\end{document}
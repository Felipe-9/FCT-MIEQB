% !TEX root = ./BG.b-Aulas_AnotaçõesP.01_Anotações.tex
% !TEX root = ./BG.b-Aulas_Anotações.tex
\providecommand\mainfilename{"./BG.b-Aulas_Anotações.tex"}
\providecommand \subfilename{}
\renewcommand   \subfilename{"./BG.b-Aulas_Anotações.P.01.tex"}
\documentclass[\mainfilename]{subfiles}

% \graphicspath{{\subfix{../images/}}}

\begin{document}

\mymakesubfile{1}{BG.b 03/08}{BG.b 03/08}

% !TEX root = ./Aula mar:08.tex

% Aula tp

% testes 3:
% - 80% 2T
% - 20% 1P


\part{}

\section{aminoácios}

% Insert graph representation here
trang - Circ (- exagono) - quadrado

Grupo amino(triang): \ch{NH3^+}
Grupo Quadrado: \ch{COOH}
grupo circ: CH (grupo assimétrico = quatro substituintes diff)
grupo hex: R (diff p cada aminoacido)

grupo amino: N possui um H a mais por isso possui carga ligeiramente +
grupo Quadrado: O OH pode perder o H e ficar negativo

os Hs dos grupos laterais podem perder ou ganhar Hs baseado no pH
pH extremo positivo remove todos os Hs

\subsection{grupo assimétrico}
Caracteristicas: polariza luz

\subsection{limitante}

\subsection{essenciais}

\subsection{Aminoácidos com cadeias polarizadas}
contem elementos da ultima coluna da tab period

\begin{sectionBox}2{Reação com Agua}
    
    \ch{ R*-COOH + H2O <> R*-COOH + H3O^+}

    \begin{BM}
        pH = pK_a + \log\frac{\ch{R^*-COO^-}}{\ch{R*-COOH}}
    \end{BM}
    
\end{sectionBox}

\begin{sectionBox}1{Cargas Formais}
    
    \ch{ AH_n <> AH_{n-1} + H3O^+ }
    \begin{BM}
        pH = pK_a + \log\frac{\ch{AH_{n-1}}}{\ch{AH_n}}
    \end{BM}
    
\end{sectionBox}

\begin{questionBox}1{}
    \begin{flalign*}
        &
            \frac{AH_2}{AH_3} = \exp(+2.90) = 7.9433\E 2
            \frac{AH_1}{AH_2} = \exp(+0.93) = 8.5114\E 0
            \frac{A   }{AH_1} = \exp(-4.47) = 3.3884\E-5
            \implies
                AH_2 = AH_3(7.9433\E 2)/(AH_3(1 + (7.9433\E 2) + (8.5114\E 0)(7.9433\E 2) + (7.9433\E 2)(8.5114\E 0)(3.3884\E-5)))
                AH_3 = \frac{100}{2.53451} + 1 = 1.257\E-1\%
                AH_1 = 25.312\%
                A    = 8.5767\E-4\%
        &
    \end{flalign*}
\end{questionBox}

\end{document}

% !TEX root = ./BG.b-Aulas_Anotações.08_Anotações.tex
% !TEX root = ./BG.b-Aulas_Anotações.tex
\providecommand\mainfilename{"./BG.b-Aulas_Anotações.tex"}
\providecommand \subfilename{}
\renewcommand   \subfilename{"./BG.b-Aulas_Anotações.T.08.tex"}
\documentclass[\mainfilename]{subfiles}

% \graphicspath{{\subfix{../images/}}}

\begin{document}

\mymakesubfile{8}
{BG.b 03/31 - Proteínas}
{BG.b 03/31 - Proteínas}

% Slide 3.20

\begin{sectionBox}1{Estruturas (Prim,Sec,Ter,Quat)}
    
    
    
\end{sectionBox}

\begin{sectionBox}2{Estrutura Terciária}
    
    
    
\end{sectionBox}

\begin{sectionBox}1{Experiência de Anfinsen}
    
    Tense uma enzima e uma 
    nas posições apontadas tem uma histeina 

    Usou ureia e \chembeta-Mercaptoethanol para desnaturar a proteína

    Forma então pontes \ch{S-S} gerando novas pontes de higrogênio diferentes da original, então remove a ureia e verificase que a proteína se reverte para a original voltando a ser ativa
    
\end{sectionBox}

\begin{sectionBox}1{Função de Aminoácidos e Proteínas}
    
    Proteínas (em seu estado terciário e Quartenário?) agem como catalizadores para reações quimicas, varias proteínas geram uma cadeia de reações de ordem imensamente superior de velocidade e constante de Reação.

    Cada grupo \textit{R} pertencente ao grupo ativo interage com os reagentes da reação forcandoa que ocorra.

    Mutações em proteínas podem mudar fortemente a estrutura da proteína gerando mudanças indiretas e bastante complexas de prever.
    
\end{sectionBox}


\end{document}
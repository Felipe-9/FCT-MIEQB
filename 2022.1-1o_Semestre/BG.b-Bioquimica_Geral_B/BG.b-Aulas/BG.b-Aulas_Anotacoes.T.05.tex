% !TEX root = ./BG.b-Aulas_Anotações.05_Anotações.tex
% !TEX root = ./BG.b-Aulas_Anotações.tex
\providecommand\mainfilename{"./BG.b-Aulas_Anotações.tex"}
\providecommand \subfilename{}
\renewcommand   \subfilename{"./BG.b-Aulas_Anotações.T.05.tex"}
\documentclass[\mainfilename]{subfiles}

% \graphicspath{{\subfix{../images/}}}

\begin{document}

\mymakesubfile{5}{03/22}{03/22}

\begin{sectionBox}1{Focagem Isoelétrica}
    
    \paragraph{Separação de proteínas por carga elétrica:}
    Posicionando proteínas em um tubo tipo ampola de cruger onde se permite inferir uma variação de pH entre os polos da ampola alem de uma diferença de potencial.
    
    As proteínas carregadas podem transladar até as variações de pH fizer modificar as posições de protões e as proteínas se tornarem neutras.

    \vspace{2ex}

    \paragraph{Note:} Slides possuem erro: cargas devem se direcionar para polos de carga opósta

    \begin{sectionBox}*2{Propríedades}
        Tira vantagem da capacidade das proteínas de variar quantidade de protões 
    \end{sectionBox}
    
\end{sectionBox}

\begin{sectionBox}1{Isolamento de Proteínas de celulas}
    
    \paragraph{Centrifugação} permite separação de uma amostra em diferentes densidades, se fáz a centrifugação em diferentes etapas onde em cada se remova a fase sólida e em cada uma se adquire diferentes compostos.

    \vspace{2ex}

    \paragraph{Solução tampão} é usada para manter o pH constante para evitar desnaturação de proteínas e danificação da amostra
    
\end{sectionBox}


\end{document}
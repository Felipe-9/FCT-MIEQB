% !TEX root = ./BG.b-Aulas_Anotações.07_Anotações.tex
% !TEX root = ./BG.b-Aulas_Anotações.tex
\providecommand\mainfilename{"./BG.b-Aulas_Anotações.tex"}
\providecommand \subfilename{}
\renewcommand   \subfilename{"./BG.b-Aulas_Anotações.T.07.tex"}
\documentclass[\mainfilename]{subfiles}

% \graphicspath{{\subfix{../images/}}}

\begin{document}

\mymakesubfile{7}
{BG.b 03/29 - Proteinas}
{BG.b 03/29 - Proteinas}

\begin{sectionBox}*1{}
    
    Sequencias de aminoácidos
    
\end{sectionBox}

\begin{sectionBox}2{Caracterização}
    
    % \begin{itemize}
    %     \begin{multicols}{2}
    %         \item 1
    %     \end{multicols}
    % \end{itemize}
    
\end{sectionBox}

\begin{sectionBox}3{Fibrosas}
    
    Proteção?
    
\end{sectionBox}

\begin{sectionBox}3{Globulares}
    
    Recebem o nome por terem em geral forma de ``globo'', são o tipo de proteínas mais comuns.

    Transporte, soluveis, se movimentam pelo organizmo
    
\end{sectionBox}

\begin{sectionBox}1{Estruturas das Proteína}
    
    Após sair do ribossoma, a proteína interage com o citoplasma, daí pode gerar dois comportamentos, um deles é torcer-se em si formando a estrutura secundária e então se contontorse em si formando uma estrutura terciária, tendo as estruturas apolares no centro e polares/hidrofilicas para fora
    
\end{sectionBox}

\begin{sectionBox}2{Estrutura Primária}
    
    Apenas a sequencia linear, determinada pela sequência de bases do DNA do gene qye a codifíca.


    
    
\end{sectionBox}

\begin{sectionBox}3{Proteínas Homologas}
    
    Proteínas relacionadas de um ponto de vista evolutivo

    \paragraph{Aminoácidos Conservativas} Aminoácidos que me mantem em todas as especies/mutações

    \paragraph{Aminoácidos Subistituidos Semelhantes} os que mudam em cada mutação/especie
    
\end{sectionBox}

\begin{sectionBox}2{Estrutura Secundária}
    
    Quando a proteína se torse em si própria em forma de helice, possui dois tipos específicos de estruturas:

    \begin{itemize}
        \item Hélice \chemalpha
        \item Folha  \chembeta
    \end{itemize}

    Em forma geral parte externa polar, interna apolar
    
\end{sectionBox}

\begin{sectionBox}3{Hélice \chemalpha}
    
    Estrutura característica do DNA

    Possue esturutras paralelas e antiparalelas

    Paralelas, conexão O e N
    
\end{sectionBox}

\begin{sectionBox}3{Folha \chembeta}
    
    
    
\end{sectionBox}

\begin{sectionBox}2{Estrutura Terciária}
    
    
    
\end{sectionBox}

\begin{sectionBox}2{Estrutura Quartenária}
    
    Estrutura tridimensional
    
\end{sectionBox}

\begin{sectionBox}1{Diagram de Ramchan}
    
    Mostra que parte do aminoácido é a mais estável

    \begin{itemize}
        \item Branco H
        \item Preto C
        \item Azul N
        \item Vermelho O
        \item Verde grupo R
    \end{itemize}
    
\end{sectionBox}

\end{document}
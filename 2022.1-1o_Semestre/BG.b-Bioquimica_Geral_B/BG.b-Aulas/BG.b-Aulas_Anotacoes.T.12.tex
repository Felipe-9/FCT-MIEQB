% !TEX root = ./BG.b-Aulas_Anotações.12_Anotações.tex
% !TEX root = ./BG.b-Aulas_Anotações.tex
\providecommand\mainfilename{"./BG.b-Aulas_Anotações.tex"}
\providecommand \subfilename{}
\renewcommand   \subfilename{"./BG.b-Aulas_Anotações.T.12.tex"}
\documentclass[\mainfilename]{subfiles}

% \graphicspath{{\subfix{../images/}}}

\begin{document}

\mymakesubfile{12}
{BG.b 19/05/22 -- Respiração Resumida}
{BG.b 19/05/22 -- Respiração Resumida}

\begin{sectionBox}1{Acidose Latica? (Doença)}
    
    Doença quando a prot que transforma o piruvato em acetyl-co-enz-A
    
\end{sectionBox}

\begin{questionBox}1{Acidose Latica? (Doença)}
    
    O mal funcionamento do Piruvato-hidroxilase acumula piruvato e estanca o funcionamento do ciclo de krebs.

    Sem a produção de ATP, a celula é forcada a ativar ciclos diferentes, anaeróbicos que produz lactase.
    
    Esse mecanismo naturalmente acontece em atividade muscular intensa quando a necessidade energética supera o fornecimento de oxigênio, o que é nocivo se acontecer em grande intencidade.

    Com essa doenca esse mecanismo é constantemente ativo causando grandes problemas para a pessoa
    
\end{questionBox}

\begin{sectionBox}1{Lactato (fermentação aerobica dos musculos)}
    
    Ciclo de core \to\ produz açucar
    
\end{sectionBox}

\end{document}
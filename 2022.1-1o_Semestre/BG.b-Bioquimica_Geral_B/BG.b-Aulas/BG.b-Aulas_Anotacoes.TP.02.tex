% !TEX root = ./BG.b-Aulas_Anotações.tex
% !TEX root = ./BG.b-Aulas_AnotaçõesP.02_Anotações.tex
\providecommand\mainfilename{"./BG.b-Aulas_Anotações.tex"}
\providecommand \subfilename{}
\renewcommand   \subfilename{"./BG.b-Aulas_Anotações.P.02.tex"}
\documentclass[\mainfilename]{subfiles}

% \graphicspath{{\subfix{../images/}}}
\renewcommand\thequestion{Questão \arabic{list}.\arabic{question}}

\begin{document}

\setcounter{list}{1}
\mymakesubfile{2}{BG.b 03/22 - Lista \arabic{list}}{BG.b 03/22 - Lista \arabic{list}}

\part*{Lista \arabic{list}}

\setcounter{question}{6}

\question{}

\begin{questionBox}2{}
    
    \begin{flalign*}
        &
            \left.
            \begin{aligned}
                pK_{a\,1}
               =& pH - \log\frac{\ch{[A^-]}}{\ch{[AH]}}
                \cong
                2.2
            \\
                pK_{a\,2}
               =& pH - \log\frac{\ch{[A^-]}}{\ch{[AH]}}
                \cong
                4.1
            \\
                pK_{a\,3}
               =& pH - \log\frac{\ch{[A^-]}}{\ch{[AH]}}
                \cong
                9.5
            \end{aligned}
            \right\}
        \therefore
            \text{Glutamic Acid (Glu) (E)}
        &
    \end{flalign*}
    
\end{questionBox}

\setcounter{question}{10}

\begin{questionBox}1{}
    
    \begin{flalign*}
        &
            \left.
            \begin{aligned}
                pK_{a\,1}
               =& pH - \log\frac{\ch{[A^-]}}{\ch{[AH]}}
                \cong
                2.2
            \\
                pK_{a\,2}
               =& pH - \log\frac{\ch{[A^-]}}{\ch{[AH]}}
                \cong
                4.5
            \\
                pK_{a\,3}
               =& pH - \log\frac{\ch{[A^-]}}{\ch{[AH]}}
                \cong
                5.9
            \\
                pK_{a\,4}
               =& pH - \log\frac{\ch{[A^-]}}{\ch{[AH]}}
                \cong
                6.3
            \\
                pK_{a\,5}
               =& pH - \log\frac{\ch{[A^-]}}{\ch{[AH]}}
                \cong
                9.7
            \end{aligned}
            \right\}
        \therefore
            \text{Arginina (Arg) (R)}
        &
    \end{flalign*}
    
\end{questionBox}

\setcounter{list}{2}
\setcounter{question}{0}
\part*{Lista \arabic{list}}

\begin{sectionBox}1{Electroforese em Papel}
    
    Separação de aminoácidos ou peptídeos sobre uma base de papel e dois banhos carregados
    
\end{sectionBox}

\begin{questionBox}1{}
    
    Considere a seguinte mistura de aminoácidos sujeita a electroforese em papel.

    \begin{questionBox}{}
        \begin{itemize}
            \begin{multicols}{3}
                \item Ala
                \item Ser
                \item Fen
                \item Arg
                \item Asp
                \item His
            \end{multicols}
        \end{itemize}
    \end{questionBox}

    Indique a direção de migração de cada aminoácido a \(pH = 3.9\).
    Esboçe a distribuição dos aminoácidos revelados com ninidrina na tira de papel após a experiência

    \vspace{2ex}
    
    pH muito próximo do equilíbrio \ch{AH <> AH2^+} dessa forma seram despresadas as quantidades dos outros compostos

    \begin{enumerate}
        \begin{multicols}{2}
            \item Alanina
        \end{multicols}
    \end{enumerate}

\end{questionBox}

\begin{questionBox}3{Alanina}
    \begin{flalign*}
        &
            pH 
           = pK_a + \log\frac{\ch{[AH]}}{\ch{[AH2^+]}}
           \implies
           10^(3.9 - 2.35)\ch{[AH2^+]}
          =2\,\ch{[AH]}
        &
    \end{flalign*}
\end{questionBox}

\begin{questionBox}3{Aginina (Arg)}

    \begin{flalign*}
        &
            \left.
            \begin{aligned}
                &
                    \ch{[AH]} 
                   =\frac{\ch{[AH]}}{\ch{[AH]} + \ch{[AH2^+]}}
                \ldiv{}
                    \ch{[AH]} = \ch{[AH2^+]}\exp(pH - pK_a)
                &
            \end{aligned}
            \right\}
            \implies &\\&
            \implies
                \ch{[AH]}
               =\frac{\exp(3.9 - 1.82)\ch{[AH2^+]}}{\exp(3.9 - 1.82)\ch{[AH2^+]} + \ch{[AH2^+]}}
               =\frac{\exp(3.9 - 1.82)}{\exp(3.9 - 1.82) + 1}
               =\num{99.175097469294645}
        &
    \end{flalign*}
\end{questionBox}

\begin{sectionBox}1{Eletroforense}
    
    Ha dois tipos: 
    \begin{itemize}
        \item Nitrato de prata (fica cor preta e branca) e é mais sensivel, melhor para quantidade de proteinas mais pequena.
        \item Gel de cor azul.
    \end{itemize}

    separação por tamanho porem puxadas por potencial elétrico
    
\end{sectionBox}

\begin{questionBox}1{}
    
    Na eletroforese em gel de poliacrimlamida (SDS-PAGE), as proteínas são sujeitas a um tratamento com dodoceil sulfato de sódio, tratamento esse que desnatura as proteínas e lhes confere a mesma densidade de carga negamtiva. Por consequencia, na eletroforese todas as protínas se deslocam no sentido do anodo. Os resultados de um ensaio SDS-PAGE são apresnetados na figura senguinte.

    \begin{questionBox}2{}
        
        O SDS-PAGE permite separar e ordenar as proteínas em função do seu volume molecular porque\dots

        \paragraph{Resposta:} 1.
        
    \end{questionBox}

    \begin{questionBox}2{}
        
        A proteína padrão G:

        \paragraph{Resposta:} 1. Dimensão diminui verticalmente em ordem de deslocação e respectivamente de tamanho.
        
    \end{questionBox}

    \begin{questionBox}2{}
        
        Quanto a composição das amostras:
        
        \paragraph{Respostas:} 1. Se pode ver um borrão a baixo do padrão de meno tamanho.
        
    \end{questionBox}

    
\end{questionBox}

\begin{questionBox}2{}
    
    \begin{multicols}{2}

        \begin{enumerate}
            \item 2.5
            \item 4.3
            \item 5
            \item 7.4
            \item 8.7
            \item 9.9
            \item 10.6
        \end{enumerate}
        
    \end{multicols}

    \begin{itemize}
        \item 8.0
    \end{itemize}
        
    % \begin{center}
    %     \pgfplotsset{height=7cm, width= .6\textwidth}
    %     \begin{tikzpicture}
    %     \begin{axis}
    %         [
    %             % extra x ticks= { 0.5 },
    %             % extra y ticks= { 0.236798 },
    %             % Number formating
    %             x tick label style={
    %                 /pgf/number format/.cd,
    %                     fixed,
    %                     fixed zerofill,
    %                     precision=3,
    %                 /tikz/.cd
    %             },
    %             y tick label style={
    %                 /pgf/number format/.cd,
    %                     fixed,
    %                     fixed zerofill,
    %                     precision=3,
    %                 /tikz/.cd
    %             },
    %             % Label titles
    %             xlabel= {d},
    %             ylabel= {\(\log m/\unit{\centi\meter}\)},
    %             % MIN/MAX
    %             % xmin= 250,    xmax= 1100,
    %             % ymin= 0,        ymax= ,
    %             % Ticks
    %             % minor x tick num= 3,
    %             minor x tick num= 1,
    %             % Grids Major
    %             xmajorgrids= true,
    %             ymajorgrids= false,
    %             % Grids Minor
    %             % xminorgrids= false,
    %             % yminorgrids= false,
    %         ]
            
    %         % Legends
    %         % \addlegendimage{empty legend}
    %         % \addlegendentry[LegendColor\Light]{  }
            
    %         % % Plot from csv file
    %         \addplot[smooth, thick] % mesh for colormap
    %         table {
    %             x       y
    %             2.5     6.301029995663981
    %             4.3     5.065392961561992
    %             5       4.988558956878616
    %             7.4     4.8208579894397
    %             8.7     4.653212513775344
    %             9.9     4.491361693834273
    %             10.6    4.15836249209525
    %         };
            
    %         % % Plot from equation
    %         % \addplot[
    %         %     smooth,
    %         %     thick,
    %         %     % Red,
    %         %     domain  = -2:2,
    %         %     samples = 0.4*\mysampledensity,
    %         % ]{  x };
            
    %     \end{axis}
    %     \end{tikzpicture}
    % \end{center}
    
\end{questionBox}

\begin{sectionBox}1{Solubilidade}
    
    A solubilidade da proteína depende da concentração de sais dissolvidos

    Pode se separar proteínas por decandação aplicando sais que movem a solubilidade das proteínas forçãndoas a decantar seletivamente
    
\end{sectionBox}

\begin{questionBox}1{}
    
    
\end{questionBox}

\begin{sectionBox}1{Ponto isoelétrico}
    
    pH em quem a proteína possui carga nula, aponta quando uma proteína se encontra negatíva ou positíva
    
\end{sectionBox}

\begin{questionBox}1{}
    
    
    
\end{questionBox}

\part*{Tabela dos aminoácidos}

% \begin{table}[H]\centering
%     \begin{tabular}{l l *{5}{r}}
        
%         \\\toprule
        
%             \multicolumn{1}{c}{Nome}
%            &\multicolumn{1}{c}{Molecula}
%            &\multicolumn{1}{c}{Residula}
        
%         \\\midrule
        
            
        
%         \\\bottomrule
        
%     \end{tabular}
% \end{table}

\end{document}
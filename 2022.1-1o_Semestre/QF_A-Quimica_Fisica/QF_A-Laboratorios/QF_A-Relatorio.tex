% \documentclass{article}
\documentclass[
    class=article,
    multi={minipage},
    border={6mm},
]{standalone}
% -------------------------------------------------------- %
%                          Colors                          %
% -------------------------------------------------------- %
\usepackage[hyperref]{xcolor} % Options: gray - dark and light
\usepackage{mypallete}
\colorlet{foreground}{light}
\colorlet{background}{dark}
% Palette Secondaries
\colorlet{Emph}  {DarkGreen!70!foreground}
\colorlet{Link} {LightGreen!25!foreground}
\colorlet{Comment}   {foreground!60!background}
\colorlet{Background}{foreground!27!background}
% ======================= setcolor ======================= %
\pagecolor{background}
\color    {foreground}
% -------------------------------------------------------- %
%                        Essentials                        %
% -------------------------------------------------------- %
\usepackage{luacode}
\usepackage{xparse} % xparse - multiple optional arguments
\usepackage{calc}
% % -------------------------------------------------------- %
% %                         Geometry                         %
% % -------------------------------------------------------- %
\setlength\textwidth{160mm}
% \usepackage{geometry}
% \geometry{
%     % papersize = {300mm, 400mm}, % ( 4:3 ) SVGA x 0.5
%     % papersize = {240mm, 320mm}, % ( 4:3 ) SVGA x 0.4
%     % papersize = {120mm, 640mm}, % ( 2*4:3/2 )
%     % papersize = {180mm, 240mm}, % ( 4:3 ) SVGA x 0.3
%     % papersize = {229mm, 305mm}, % ( 4:3 ) ArchA/Arch1
%     % papersize = {320mm, 512mm}, % (16:10)
%     % papersize = {280mm, 448mm}, % (16:10)
%     % papersize = {240mm, 384mm}, % (16:10)
%     % a4paper,  % {210mm, 297mm}, % (√2:1 ) A4
%     papersize = {180mm, 1m}, % LONG ( 4:3 ) SVGA x 0.3
%     top       = 21mm,
%     bottom    = 21mm,
%     left      = .06\paperwidth,
%     right     = .06\paperwidth,
%     portrait  = true,
% }
% -------------------------------------------------------- %
%                           Fonts                          %
% -------------------------------------------------------- %
\usepackage[fontsize=12pt]{fontsize}
\usepackage[T1]{fontenc}
\usepackage{fontspec}
% \usepackage[sfdefault, ultralight]{FiraSans}
% \renewcommand{\familydefault}{\sfdefault} % set sans serif as defalt
% % ==================== Set font family =================== %
\setmainfont[
    Ligatures = TeX,
    UprightFont    = *-Light,
    ItalicFont     = *-LightIt,
    SmallCapsFont  = *SmText-Light,
    BoldFont       = *-Regular,
    BoldItalicFont = *-It
]{
    % ==================== Serif fonts =================== %
    SourceSerif4
    % ================= Handwritten fonts ================ %
    % DancingScript
    % Apple Chancery
    % Luminari
    % PetitFormalScript
    % Merienda
    % LaBelleAurore
    % Handlee
        % BadScript
        % Calligraffitti
}
% % ======================= Monofont ======================= %
% \setmonofont[Contextuals = {ligatures}]{FiraCode}
% \makeatletter
% \def\verbatim@nolig@list{}
% \makeatother
% -------------------------------------------------------- %
%                         Linguagem                        %
% -------------------------------------------------------- %
\usepackage[portuguese]{babel} % Babel
%\usepackage{polyglossia}      % Polyglossia
%\setdefaultlanguage[variant=brazilian]{portuguese}
% % -------------------------------------------------------- %
% %                         Graphics                         %
% % -------------------------------------------------------- %
% \usepackage[draft, final]{graphics, graphicx}
\usepackage{graphicx}
\graphicspath{{./.build/figures/QF_A-Relatorio/}}
% -------------------------------------------------------- %
%                tocloft - Table of contents               %
% -------------------------------------------------------- %
\usepackage{tocloft}
\setcounter{tocdepth}{2}    % remove subsubsection from toc
% ========================= part ========================= %
\renewcommand\cftpartfont{\bfseries}
%\renewcommand\cftpartafterpnum{\vspace{0mm}}
\setlength\cftbeforepartskip{1ex}
% ========================== sec ========================= %
\renewcommand\cftsecfont{} % Font
\renewcommand\cftsecpagefont{} % page number font
\renewcommand\cftsecleader{\cftdotfill{\cftdotsep}} % Dots
\setlength\cftbeforesecskip{0.5ex}
\setlength\cftsecindent{0mm}
%\setlength{\cftsecnumwidth}{25mm} % Fix section width
% ======================== subsec ======================== %
\setlength\cftsubsecindent{0mm}
%\setlength{\cftsubsecnumwidth}{15mm}
% ====================== tab (table) ===================== %
\setlength\cfttabindent{0mm}
% % -------------------------------------------------------- %
% %                       filecontents                       %
% % -------------------------------------------------------- %
%\usepackage{scontents}     % the better filecontents
%\usepackage{filecontents}  % Create files
% % -------------------------------------------------------- %
% %                         Multicols                        %
% % -------------------------------------------------------- %
\usepackage{multicol}
\setlength{\columnsep}{.05\textwidth}
\multicoltolerance = 200
% \renewenvironment{multicols}[1]{}{\relax} % toggle multicols  on/off
% -------------------------------------------------------- %
%                         enumitem                         %
% -------------------------------------------------------- %
\usepackage{enumitem}  % modify enumerate index
\setlist[description]{
    format={\color{Emph}}
}
% -------------------------------------------------------- %
%                         titlesec                         %
% -------------------------------------------------------- %
\usepackage{titlesec}
% ================= Reset section on part ================ %
\counterwithin*{section}{part}
% ======================== Spacing ======================= %
% 	itlespacing*{<left>}{<before>}{<after>}[<right>]
\titlespacing*{\part}      {0pt}{ 0pt}{0pt}
\titlespacing*{\section}   {0pt}{10mm}{0pt}
\titlespacing*{\subsection}{0pt}{ 5mm}{0pt}

% Part customization
\titleclass{\part}{straight}
\titleformat{\part}
    [block]                         % shape
    {\Large\bfseries\color{Emph}}   % format
    {\thepart\hspace{.5em}{--}}     % label
    % {\thepart}                      % label without --
    {.5em}                          % sep
    {\Large\bfseries}               % before-code
    [\vspace{1em}]                  % after  code

% % Chapter customization
% \titleclass{\chapter}{straight}
% \titleformat{\chapter}
%    [block]                         % shape
%    {\huge\bfseries\color{Emph}}    % format
%    {\thepart\hspace{5mm}{\(|\)}}   % label
%    {5mm}                           % sep
%    {\huge\bfseries}                % before-code
%    [\vspace{0.5mm}]                % after-code
% % -------------------------------------------------------- %
% %                         Appendix                         %
% % -------------------------------------------------------- %
% \usepackage{appendix}
% -------------------------------------------------------- %
%                     siunix: SI units                     %
% -------------------------------------------------------- %
\usepackage{siunitx}
\sisetup{
    % scientific / engineering / false / fixed
    scientific-notation    = engineering,
    exponent-to-prefix     = false, % 1000 g -> 1 kg
    % exponent-product       = *, % x * 10^y
    round-mode             = places, % figures/places/none
    round-precision        = 2,
    output-exponent-marker = {\,\mathrm{E}},
}
% ==================== Declaring units =================== %
\DeclareSIUnit\atm{atm}
\DeclareSIUnit\calorie{cal}
\DeclareSIUnit\Torr{Torr}
\DeclareSIUnit\bar{bar}
\DeclareSIUnit\mmHg{mmHg}
\DeclareSIUnit\molar{M}
\DeclareSIUnit\M{M}
% % -------------------------------------------------------- %
% %                           Maths                          %
% % -------------------------------------------------------- %
\usepackage{amsmath, amssymb, bm}
% \usepackage{derivative} % Derivative
% \usepackage{mathrsfs}   % more symbols \mathscr{} (Hamiltonian)
% % ====================== Math fonts ====================== %
% \usepackage[math-style=ISO]{unicode-math} % change math font
% \usepackage{firamath-otf}
% % ======================================================== %
\usepackage{mathtools}
% % ======================================================== %
% ============== MathOperators Declarations ============== %
\DeclareMathOperator\E{\,E}
\DeclareMathOperator\e{\,E}
% % ========= Missing trigonometric math operators ========= %
% \DeclareMathOperator\sech   {sech}
% \DeclareMathOperator\csch   {csch}
% \DeclareMathOperator\arcsec {arcsec}
% \DeclareMathOperator\arccot {arccot}
% \DeclareMathOperator\arccsc {arccsc}
% \DeclareMathOperator\arccosh{arccosh}
% \DeclareMathOperator\arcsinh{arcsinh}
% % ================== Calculus operators ================== %
% \DeclareMathOperator\fronteira {fr}
% \DeclareMathOperator\interior  {int}
% \DeclareMathOperator\exterior  {ext}
% \DeclareMathOperator\grafico   {Graf}
% \DeclareMathOperator\dominio   {D}
% \DeclareMathOperator\visinhanca{\mathcal{V}}
% % =================== Algebra operators ================== %
% \DeclareMathOperator\adj{adj}
% \DeclareMathOperator\id{id}
% \DeclareMathOperator\Img{Im}
% \DeclareMathOperator\Nuc{Nuc}
% \DeclareMathOperator\Ker{Ker}
% % ================== Statistics Operator ================= %
% \DeclareMathOperator\esperanca{E}
% \DeclareMathOperator\variancia{V}
% \DeclareMathOperator\bernoulli{Ber}
% \DeclareMathOperator\hiperbolica{H}
% \DeclareMathOperator\binomial{Bin}
% \DeclareMathOperator\poisson{P}
% \DeclareMathOperator\normal{N}
% % ======================================================== %
% % ======================================================== %
% =================== My math functions ================== %
\usepackage{mathBM}
\usepackage{myLogicOperators}
% \lxor, \lnand
% \ldiv
% [<*>line break]
% [<l>left column]
% [<r>right column]
% [<division>]
% [<space>]
% [<before column>]
\usepackage{mathEnclosings} 
% \myvert[*scale]{<content>}
% \myVert[*scale]{<content>}
% \myrange[<*>invert all][<l>invert left][<r>invert right]{<content>}
% % -------------------------------------------------------- %
% %                          Vectors                         %
% % -------------------------------------------------------- %
% \usepackage{esvect}      % Vector over-arrow
% \NewCommandCopy\vec\vv



% Tikz
\usepackage{tikz}
    \usetikzlibrary{
%         perspective,
%         3d,
        external,
    }

% External remake commands
% 	ikzset{external/force remake=false} - remake all
% 	ikzset{external/remake next=true}   - remake next


% External lib
\tikzexternalize[
    up to date check = {simple}, % faster check
    figure list = true, % generate list of figures file
    % optimize  = false,
    % optimize command away = { % commands to be ignored when optimizing
    %     \BM, \endBM,
    %     \ldiv,
    %     \question, \subquestion, \subsubquestion,
    %     % \sobpart,
    %     \questionBox, \endquestionBox,
    %     \sectionBox,  \endsectionBox,
    %     % \codeBox,     \endcodeBox
    %     \mymaketitle
    % },
    prefix = ./.build/figures/QF_A-Relatorio/graphics/
] % turn externalization on/off
\tikzsetfigurename{figure_\arabic{part}.\arabic{section}.} % set figure names

\tikzset{external/system call={%
    lualatex \tikzexternalcheckshellescape
    --halt-on-error
    --shell-escape
    --interaction=batchmode
    --jobname "\image" "\texsource"
    % --jobname "\image" \subfilename
}}

\NewCommandCopy\oldtikzpicture\tikzpicture
\renewcommand\tikzpicture{\oldtikzpicture\nopagecolor}



% \usepackage{varwidth}   % List inside TikzPicture

% Gradient text color
% \newcommand\fadingtext[3][]{%
%     \begin{tikzfadingfrompicture}[name=fading letter]
%         \node[text=transparent!0,inner xsep=0pt,outer xsep=0pt,#1] {#3};
%     \end{tikzfadingfrompicture}%
%     \begin{tikzpicture}[baseline=(textnode.base)]
%         \node[inner sep=0pt, outer sep=0pt,#1](textnode){\phantom{#3}};
%         \shade[path fading=fading letter,#2,fit fading=false]%
%             (textnode.south west) rectangle (textnode.north east);%
%     \end{tikzpicture}%
% }

% \newcommand\fadingtext[2][]{%
%   \setbox0=\hbox{{\special{pdf:literal 7 Tr }#2}}%
%   \tikz[baseline=0]\path [#1] \pgfextra{\rlap{\copy0}} (0,-\dp0) rectangle (\wd0,\ht0);%
% }
%
% \NewCommandCopy\oldsection\section
% \renewcommand\section[1]{
%     \oldsection{\fadingtext{
%         left  color   = DarkGreen!00!foreground,
%         right color   = DarkGreen!30!foreground,
%         shading angle = 150 % 90 + 60
%     }{ \relax\parbox[b]{\linewidth}{#1} }}
% }
%
% \NewCommandCopy\oldpart\part
% \renewcommand\part[1]{
%     \oldpart{\fadingtext{
%         left  color   = Emph,
%         right color   = Emph!50!LightGreen,
%         shading angle = 170 % 90 + 60
%     }{ #1 }}
% }

% pgf
    \usepackage{pgf}
    % pgfmath
    % \usepackage{pgfmath}    % calculations

    % pgfplots
    \usepackage{pgfplots}
    % \usepgfplotslibrary{fillbetween}
% ========================= Setup ======================== %
    % 1e4 too much
    % 1e3 fancy
    % 1e2 simple
    % 1e1 draft
    \newcommand\mysampledensityDouble{2}
    \newcommand\mysampledensitySimple{1e1}
    \newcommand\mysampledensityFancy{1e2}
    \pgfplotsset{
        compat       = newest,
        width        = .95\linewidth,   % width
        height       = .22\textheight,  % height
        samples      = \mysampledensityDouble,
        % Color Map
        colormap = {cool}{
            color=(DarkGreen\Light!20!background);
            % color=(DarkGreen\Light!60!background);
            color=(DarkGreen\Light!70!background);
            % color=(DarkGreen\Light!80!background);
            color=(DarkGreen\Light!90!background)
        },
        % Preset plot style
        every axis plot/.append style = {
            thick,
            % mark=*,
            mark options={
                fill=Emph,
                fill opacity=1
            },
            color=Emph,
            % fill=Emph,
            fill opacity=0.1,
        },
        % Legend
        legend style = {
            draw         = none,
            fill         = foreground\Dark,
            fill opacity = 0.3,
            text opacity = 1,
        },
        % Grid
        major grid style = {
            very thin,
            color= foreground!15!background
        },
        % minor Grid
        minor grid style = {
            very thin,
            color= foreground!7!background
        },
        % Tick Label
        ticklabel style = {
            /pgf/number format/.cd,
            set thousands separator={},
            tick style = {
                color= foreground!60!background
            },
        },
        % Extra ticks
        every extra x tick/.style = {
            tick style       = {draw=none},
            major grid style = {
                draw, thin,
                color = foreground!60!background,
            },
            ticklabel pos = top,
        },
        every extra y tick/.style = {
            tick style       = {draw=none},
            major grid style = {
                draw, thin,
                color= foreground!90!background
            },
            ticklabel pos    = right,
        },
    }

    % % pgfplotstable
    % \usepackage{pgfplotstable}


% Tabular
\usepackage{multirow}
\usepackage{float}  % table position H(ere)
    \restylefloat{table}
% \usepackage{longtable}
%
\setlength\tabcolsep{6mm}        % width
\renewcommand\arraystretch{1.25} % height

% booktabs
\usepackage{booktabs}
\setlength\heavyrulewidth{.75pt} % Top and bottom rule
\setlength\lightrulewidth{.50pt} % Middle rule
% \usepackage{colortbl}            % Colored Cells

% Mathtable
\usepackage{array} % for \newcolumntype macro
\newcolumntype{C}{>{$}c<{$}} % math-mode version of c column type
\newcolumntype{L}{>{$}l<{$}} % math-mode version of l column type
\newcolumntype{R}{>{$}r<{$}} % math-mode version of r column type


\newcounter{tablecounter}
% \newcommand\thetable{Tabela \arabic{tablecounter}:}
\NewDocumentCommand\tablecaption{o m}{
    \stepcounter{table}
    Tabela \thetable{}: #2
    \ifblank{#2}{}{\label{table:#1}}
    \addcontentsline{lot}{table}{Tabela \thetable{}: #2}
}

% % Chem
\usepackage{chemformula} % formulas quimicas
\usepackage{chemfig}     % Estruturas quimicas
% \usepackage{modiagram}   % Molecular orbital diagram
% \setmodiagram {
%     names,           % Display names
%     labels,          % Display labels
%     labels-fs=\tiny, % label font
% }
% \newlength\AtomVScale    \setlength\AtomVScale{1cm}
% \newlength\MoleculeVScale\setlength\MoleculeVScale{1cm}
\usepackage{chemmacros}
\chemsetup[phases]{pos=sub}
% \newcommand{\mol}[1]{ \unit{\mole\of{\ch{ #1 }}} } % mol



% % Biology
% \usepackage{mydnaseq}



% Constants
% \usepackage{physconst, physunits}



% Code
% % Run on terminal: lualatex --shell-escape [file[.tex]]
% \usepackage{shellesc, minted}
% \setminted {
%     linenos,     % line number
%     autogobble,  % line trim
%     tabsize = 4, % tab size
%     obeytabs,    % tab alignment
%     breaklines,  % break lines
%     % python3,     % Python lexer or idk
% }
% \usemintedstyle{stata-dark}
% % \usemintedstyle{fruity}
% % \usemintedstyle{paraiso}
% % \usemintedstyle{rainbow_dash}
% % \usemintedstyle{solarized-dark}
% % \usemintedstyle{native}



% tcolorbox
\usepackage{tcolorbox}
\tcbuselibrary{
    breakable,                % allow page break
    % minted, xparse, listings, % code minted
}
\tcbset{ every box/.style = {
    coltext      = foreground,           % text  color
    % coltitle     = foreground,           % title color
    % fonttitle    = \bfseries,       % title font
    notitle,                        % Remove title
    opacityfill  = 0.1,             % background opacity
    opacityframe = 0,               % frame      opacity
    colback      = Background,      % background color
    colframe     = Background,      % border     color
    arc          = 3mm,             % Curvature
    width        = \linewidth,      % Width
    top          = 3mm,             % Space between text and top
    bottom       = 3mm,             % Space between text and bottom
    before upper = {\parindent2ex}, % Paragraph indentation
    before skip  = 0mm,             % Set vspace before box
}}



% mytitle and myauthor
\newcommand\mytitle   {{Cinética da redução do corante azul de toluidina pelo ião sulfito}}
\newcommand\myauthor  {{Felipe B. Pinto 61387 -- MIEQB}}
\newcommand\mycreator {{Felipe B. Pinto}}
\newcommand\mysubject {{Relatório Laboratorial}}
\newcommand\mykeywords{{{QF A}, {Relatório}}}

% title, author and date
\title{\huge\bfseries\color{Emph}\mytitle}
\author{\Large\myauthor}
\date{\Large\today}



% % fancyhdr - Header and Footer customization
% \usepackage{fancyhdr}
% \pagestyle{fancy}
% \fancyhf{} % Clear
% \fancyhead[R]{\normalsize\thepart}
% \fancyfoot[L]{\normalsize\myauthor}
% \fancyfoot[R]{\thepage}
% \renewcommand\footrulewidth{.5pt}
% % Marks 
% % \renewcommand{\partmark}[1]{\markboth{}{\thepart#1}}



% Subfiles
\usepackage{subfiles}



% hyperref
\usepackage{hyperref}
\hypersetup{
    % Links customization
    % hidelinks   = true,
    colorlinks  = true,
    linkcolor   = Link,
    anchorcolor = Link,
    urlcolor    = Link,
    % Metadata
    pdfinfo = {
        Title    = \mytitle,
        Author   = \myauthor,
        Creator  = \mycreator,
        Subject  = \mysubject,
        Keywords = \mykeywords,
    },
    % PDF display customization
    pdfpagelayout      = {OneColumn},
    pdfstartview       = {FitH},
    pdfremotestartview = {FitH}
    pdfdisplaydoctitle = true,
    % pdfpagetransition  = Glitter, % wtf
}
% Fix links when reseting section on part
% \renewcommand\theHpart{\theHsobpart.\arabic{part}}
\renewcommand\theHsection{\theHpart.\arabic{section}}

% get parttitle command
% \NewCommandCopy\oldpart{\part}
% \newcommand\parttitle{}
% \renewcommand{part}[1]{\oldpart{#1}\renewcommand{\parttitle}{#1}}

% Color targets
\NewCommandCopy\oldhypertarget\hypertarget
\renewcommand\hypertarget[3][Link]{\oldhypertarget{#2}{\textcolor{#1}{#3}}}



\usepackage{sections}
% \usepackage{questions}
% \usepackage{answers}
% \usepackage{examples}
% \usepackage{definitions}



% /* ------------------------- Divisions customization ------------------------ */
\renewcommand\thesubsubsection{(\roman{subsubsection})}
% \renewcommand\thepart{Part \arabic{part}}


\usepackage{mytitle}    % \mymaketitle
\usepackage{mysubfile}  % \mymakesubfile

\begin{document}

% \mymaketitle

\noindent\begin{minipage}{\linewidth}
    
    \vspace{3ex}

    \begin{center}\Large
        UNIVERSIDADE NOVA DE LISBOA\\
        NOVA SCHOOL OF SCIENCE AND TECHNOLOGY\\
        \vspace*{1ex}
        \includegraphics[width=.4\textwidth]{nova.png}
        \vspace*{3ex}

        \large
        Departamento de química\\
        Química Física A\\
        \vspace{3ex}

        {\bfseries\huge\color{Emph}
            \mytitle
        }
    \end{center}

    \begin{center}
        \vspace{-1ex}
        \begin{tabular}{l c c}
            \multicolumn{3}{l}{Ano Letivo 2022--2023}
            
            \\\toprule
              Felipe Pinto & 61387 & MIEQB
            \\Francisco Duarte & 63754 & LEQB
            \\Lunara Maciel & 54768 & MIEQB
            \\Sebastião Carvalhal & 60823 & MIEQB
            
            \\\bottomrule

        \end{tabular}
        % \vspace{2ex}
    \end{center}
    
    \vspace{-3ex}

    % Table of Contents
    % \renewcommand\contentsname{} % remove title
    
    % \begin{multicols}{2}[\section*{Conteúdo}]
        \tableofcontents
    % \end{multicols}

    \vspace{3ex}

\end{minipage}

\begin{sectionBox}*1{Lista de Figuras} % S
    
    \vspace{-6ex}
    \renewcommand\listfigurename{}
    \listoffigures
    
\end{sectionBox}

\begin{sectionBox}*1{Lista de Tabelas} % S
    
    \vspace{-6ex}
    \renewcommand\listtablename{}
    \listoftables
    
\end{sectionBox}


% \tikzset{external/force remake=true}



% 88888888ba
% 88      "8b
% 88      ,8P
% 88aaaaaa8P'   ,adPPYba,  ,adPPYba,  88       88  88,dPYba,,adPYba,    ,adPPYba,
% 88""""88'    a8P_____88  I8[    ""  88       88  88P'   "88"    "8a  a8"     "8a
% 88    `8b    8PP"""""""   `"Y8ba,   88       88  88      88      88  8b       d8
% 88     `8b   "8b,   ,aa  aa    ]8I  "8a,   ,a88  88      88      88  "8a,   ,a8"
% 88      `8b   `"Ybbd8"'  `"YbbdP"'   `"YbbdP'Y8  88      88      88   `"YbbdP"'


\begin{sectionBox}1{Resumo} % S
    
    A realização desta atividade experimental teve como objetivo estudar as propriedades e a cinética do corante azul de toluidina durante o processo de redução pelo ião sulfito.\newline

    Preparamos 5 soluções de concentrações diferentes e com a sua força iónica constante e igual a 0.49\,\unit{\M} medindo-se a absorvância usando um espectrofotómetro. Com os valores obtidos, representamos graficamente o logaritmo neperiano da absorvância em função do tempo e também calculamos a pseudo-constante da velocidade (\textit{k'}) para cada solução.\newline
    
    Com os resultados obtidos contruiu-se um gráfico de \textit{k'} em função da concentração de \(\ch{[SO3^{2-}]}_0\), onde constatamos que o valor da constante cinética da reação (\textit{K}) é igual a 0.037\,\unit{\mole^{-1}.\second^{-1}}. Sendo que o valor obtido é semelhante aos valores obtidos pelos demais grupos podemos concluir que os resultados foram os esperados.
    
\end{sectionBox}


% 88
% 88                 ,d
% 88                 88
% 88  8b,dPPYba,   MM88MMM  8b,dPPYba,   ,adPPYba,
% 88  88P'   `"8a    88     88P'   "Y8  a8"     "8a
% 88  88       88    88     88          8b       d8
% 88  88       88    88,    88          "8a,   ,a8"
% 88  88       88    "Y888  88           `"YbbdP"'



\begin{sectionBox}1{Introdução} % S
    
    No estudo da reação de redução do azul de toluidina \ch{TB^+} pelo ião sulfito \ch{SO3^{2-}} descrita pela equação:
    \begin{center}
        % \tikzset{external/remake next=true}
        % \schemestart
            % \chemfig[angle increment=30]{
            %     *6(
            %         (-[7]\ch{H2N})
            %         -
            %         =*6( 
            %             -S^+
            %             =*6(
            %                 -
            %                 =(-[-1]N(-[-3])(-[1]))
            %                 -=-
            %             )
            %             -=N-
            %         )
            %         -=
            %         -(-[5])
            %         =
            %     )
            % }
            % \+
            % \chemfig{SO3^{-2}}
            % \+
            % \chemfig{H2O}
            % \arrow(.mid east--.mid west)
        % \schemestop


        \includegraphics[width=.8\textwidth]{reacao.png}
    \end{center}

    O azul de toluidina \ch{TB^+} é um corante caraterizado por \(\chemlambda_{\max} = 596\,\unit{\nano\metre}\) e por \(\chemepsilon (596\,\unit{\nano\metre}) = 24000\,\unit{\M^{-1}.\centi\metre^{-1}}\) em solução aquosa.\cite{:protocolo} O ião sulfito reduz o azul de toluidina em branco de toluidina \ch{TBH}. Esta reação segue uma cinética de 2ª ordem, dado que velocidade da reação é diretamente proporcional ao produto das concentrações dos reagentes, isto é, a velocidade da reação aumenta exponencialmente à medida que as concentrações dos reagentes aumentam:

    \begin{BM}
        v=k\,\ch{[SO3^{-2}]_0[TB^+]}
    \end{BM}

    Como sabemos que o ião sulfito se encontra em excesso em comparação com o azul de toluidina, a sua concentração não irá variar significativamente durante a reação, o que nos permite estudar a reação como sendo de pseudo 1ª ordem. Assim a velocidade da reação depende apenas da concentração do reagente em menor quantidade, podendo ser aproximada a uma cinética de 1ª ordem

    \begin{BM}
        v=k'\,\ch{[TB^+]}
        \qquad
        k'=k\,\ch{[SO3^{2-}]_0}
    \end{BM}
    \cite{:protocolo}

    Partindo da nova equação de velocidade, se integrarmos obtemos:

    \begin{BM}
        \ln\ch{[TB^+]}
        =\ln\ch{[TB^+]}_0-k'\,t
        \iff
        \ln{A}=\ln{A_0}-k'\,t
    \end{BM}
    \cite{:protocolo}

    Através da lei de Lambert-Beer, \(A=\chemepsilon\,b\ch{[TB^+]}\), sabemos que a \ch{TB+} é diretamente proporcial à absorvância, (\textit{A}),  medida experimentalmente pelo espectrofotómetro. Assim conseguimos obter o valor de  \(k'\) pelo declive da reta \(\ln{(A)}\) \times{} tempo. Através da ordenada na origem da mesma reta consegue-se obter o valor da absorvância inicial do azul de toluidina, que se deve encontrar entre 310 a 340\,\unit{\nano\metre}.\newline

    Utilizando diferentes concentrações de \ch{[SO3^{2-}]_0} para diferentes soluções, construimos um gráfico dos respetivos \textit{k'} em função da concentração \ch{[SO3^{2-}]_0}  cujo declive da reta formada equivale ao valor do \textit{k}, constante cinética da reação em estudo.

\end{sectionBox}


% 88888888ba                           88                       88
% 88      "8b                          88                       88
% 88      ,8P                          88                       88
% 88aaaaaa8P'  8b,dPPYba,   ,adPPYba,  88           ,adPPYYba,  88,dPPYba,
% 88""""""'    88P'   "Y8  a8P_____88  88           ""     `Y8  88P'    "8a
% 88           88          8PP"""""""  88           ,adPPPPP88  88       d8
% 88           88          "8b,   ,aa  88           88,    ,88  88b,   ,a8"
% 88           88           `"Ybbd8"'  88888888888  `"8bbdP"Y8  8Y"Ybbd8"'

\begin{sectionBox}1{Cálculos e Resultados}

    \subsection{Cálculos Pré-laboratoriais}
    
    \subsection*{\ch{TB^+}}
    \begin{flalign*}
        &
            V_{Mae}
            = \frac{
                \unit{\milli\litre\of{Mae}}
            }{
                2.0*10^{-4}\,\unit{\mole\of{\ch{TB^+}}}
            }
            \,\frac{
                2.0*10^{-5}
                \,\unit{\mole\of{\ch{TB^+}}}
            }{
                \unit{\milli\litre\of{Sol}}
            }
            \,20\,\unit{\milli\litre\of{Sol}}
            = 2.0\,\unit{\milli\litre\of{Mae}}
        &
    \end{flalign*}

    \subsection*{\ch{Na2SO3}}
    \begin{flalign*}
        &
            V_{Mae}
            = \frac{
                \unit{\milli\litre\of{Mae}}
            }{
                0.20\,\unit{\mole\of{\ch{Na2SO3}}}
            }
            \,\frac{
                x\,\unit{\mole\of{\ch{Na2SO3}}}
            }{
                \unit{\milli\litre\of{Sol}}
            }
            \,20\,\unit{\milli\litre\of{Sol}}
            = 100\,x\,\unit{\milli\litre\of{Mae}}
        &
    \end{flalign*}
    \begin{center}
        \vspace{-1ex}
        \begin{tabular}{l *{5}{C}}
            \toprule
            
                \unit{\molar\of{\ch{Na2SO3}}}
                & 0.02 & 0.04 & 0.06 & 0.08 & 0.10
                \\\unit{\milli\litre\of{Mae}}
                & 2 & 4 & 6 & 8 & 10
            
            \\\bottomrule
        \end{tabular}
        \vspace{2ex}
    \end{center}

    \subsection*{\ch{NaCl}}
    \begin{BM}
        V_{Mae}=\frac{0.48999 - 3\,c_{\ch{Na2SO3}}}{0.03}
        \,\unit{\milli\litre\of{Mae}}
    \end{BM}
    \begin{flalign*}
        &
            V_{Mae}
            = \frac{\unit{\milli\liter\of{Mae}}}{
                0.60\,\unit{\mole\of{\ch{NaCl}}}
            }
            \,\frac{
                \ch{[NaCl]}\unit{\mole\of{\ch{NaCl}}}
            }{
                \unit{\milli\litre\of{Sol}}
            }
            \,20\,\unit{\milli\litre\of{Sol}}
            = \frac{\ch{[NaCl]}}{0.03}
            \,\unit{\milli\litre\of{Mae}}
        &\\[2ex]&
            I
            = 0.49
            = \frac{1}{2}
            \sum_{i=1}^{n}{c_n\,z_n^2}
            = \frac{1}{2}
            \left(
                \begin{aligned}
                    &
                        2.0*10^{-5} & *\,(+1)^2 
                    &+\\+&
                        \ch{[Na2SO3]}*2 & *\,(+1)^2 
                    &+\\+&
                        \ch{[Na2SO3]} & *\,(-2)^2 
                    &+\\+&
                        \ch{[NaCl]} & *\,(+1)^2 
                    &+\\+&
                        \ch{[NaCl]} & *\,(-1)^2 
                    &
                \end{aligned}
            \right)
            \begin{aligned}
                (\ch{TB^+})
                \\ (\ch{Na^{2+}})
                \\ (\ch{So^{2-}})
                \\ (\ch{Na^{1+}})
                \\ (\ch{Cl^{1-}})
            \end{aligned}
            \implies &\\[2ex]&
            \implies
            \ch{[NaCl]}
            = 0.48999 - 3\,\ch{[Na2SO3]}
            % &\\[3ex]&
            \qquad
            \therefore
            \frac{0.48999 - 3\,\ch{[Na2SO3]}}{0.03}
            \,\unit{\milli\litre\of{Mae}}
        &
    \end{flalign*}
    \begin{center}
        \vspace{-1ex}

        \begin{tabular}{l *{5}{C}}
            \toprule
            
                \unit{\molar\of{\ch{Na2SO3}}}
                & 0.02 & 0.04 & 0.06 & 0.08 & 0.10
                \\\unit{\milli\litre\of{Mae}}
                & \num{14.333}
                & \num{12.333}
                & \num{10.333}
                & \num{8.333}
                & \num{6.332999999999998}
            
            \\\bottomrule
        \end{tabular}
        \vspace{2ex}
    \end{center}
    
    \subsection*{Volumes usados}
    \begin{center}
        \vspace{1ex}
        \setlength\tabcolsep{3mm}        % width
        % \renewcommand\arraystretch{1.25} % height
        \begin{tabular}{*{5}{C}}
            \toprule
            
                \multicolumn{1}{c}{Solução}
                & \ch{TB^+}/\unit{\milli\litre}
                & \ch{Na2SO3}/\unit{\milli\litre}
                & \ch{NaCl}/\unit{\milli\litre}
                & \ch{H2O}/\unit{\milli\litre}
            
            \\\midrule
            
                   1 & 2 &  2 & 14 & 2
                \\ 2 & 2 &  4 & 12 & 2
                \\ 3 & 2 &  6 & 10 & 2
                \\ 4 & 2 &  8 &  8 & 2
                \\ 5 & 2 & 10 &  6 & 2
            
            \\\bottomrule

            \multicolumn{5}{r}{Volume Total: \(20\)\,\unit{\milli\litre}}
        \end{tabular}\\
        \tablecaption{Volumes usados em cada solução}
        \vspace{2ex}
    \end{center}

\end{sectionBox}


% 88888888ba                                       88                                88
% 88      "8b                                      88    ,d                          88
% 88      ,8P                                      88    88                          88
% 88aaaaaa8P'   ,adPPYba,  ,adPPYba,  88       88  88  MM88MMM  ,adPPYYba,   ,adPPYb,88   ,adPPYba,   ,adPPYba,
% 88""""88'    a8P_____88  I8[    ""  88       88  88    88     ""     `Y8  a8"    `Y88  a8"     "8a  I8[    ""
% 88    `8b    8PP"""""""   `"Y8ba,   88       88  88    88     ,adPPPPP88  8b       88  8b       d8   `"Y8ba,
% 88     `8b   "8b,   ,aa  aa    ]8I  "8a,   ,a88  88    88,    88,    ,88  "8a,   ,d88  "8a,   ,a8"  aa    ]8I
% 88      `8b   `"Ybbd8"'  `"YbbdP"'   `"YbbdP'Y8  88    "Y888  `"8bbdP"Y8   `"8bbdP"Y8   `"YbbdP"'   `"YbbdP"'


    %   ,ad8888ba,                              ad88  88
    %  d8"'    `"8b                            d8"    ""
    % d8'                                      88
    % 88             8b,dPPYba,  ,adPPYYba,  MM88MMM  88   ,adPPYba,   ,adPPYba,   ,adPPYba,
    % 88      88888  88P'   "Y8  ""     `Y8    88     88  a8"     ""  a8"     "8a  I8[    ""
    % Y8,        88  88          ,adPPPPP88    88     88  8b          8b       d8   `"Y8ba,
    %  Y8a.    .a88  88          88,    ,88    88     88  "8a,   ,aa  "8a,   ,a8"  aa    ]8I
    %   `"Y88888P"   88          `"8bbdP"Y8    88     88   `"Ybbd8"'   `"YbbdP"'   `"YbbdP"'

    %  ad88888ba                88
    % d8"     "8b               88
    % Y8,                       88
    % `Y8aaaaa,     ,adPPYba,   88
    %   `"""""8b,  a8"     "8a  88
    %         `8b  8b       d8  88
    % Y8a     a8P  "8a,   ,a8"  88
    %  "Y88888P"    `"YbbdP"'   88

\begin{sectionBox}2{Resultados} % S
    
    % \tikzset{external/force remake=true}

    \subsection*{Graficos das \textit{k'} das soluções}

    % \begin{multicols}{2}
    \begin{center}

        % \tikzset{external/force remake=true}

        \pgfplotsset{height=7cm, width=1\textwidth}

        % ---------------------------------------------------------------------------- %
        %                                   Solução 1                                  %
        % ---------------------------------------------------------------------------- %

        % \tikzset{external/remake next=true}
        \begin{figure}\centering
        {\Large\bfseries{Solução 1}}\par\medskip
        \begin{tikzpicture}
        \begin{axis}
            [
                xmajorgrids=true,
                ymajorgrids=true,
                minor tick num=3,
                xminorgrids=true,
                yminorgrids=true,
                % legend pos  = north west
                % axis on top,
                ylabel={\(\ln{Abs}\)},
                xlabel={Tempo/\unit{\second}},
            ]
            % Legends
            \addlegendimage{empty legend}
            \addlegendentry[Emph]{\( y=-0.0004\,x-1.1773 \)}
            \addlegendimage{empty legend}
            \addlegendentry[Emph]{\( R^2=0.9921 \)}

            \addplot[
                no marks,
                dashed
            ] expression[
                domain=30:1139
            ] {
                -0.0004*x-1.1773
            };

            \addplot[
                mark=*,
                only marks,
            ] coordinates {
                (  30, -1.184170177)
                (  64, -1.158362293)
                (  83, -1.187443502)
                ( 113, -1.231001477)
                ( 150, -1.244794799)
                ( 197, -1.262308381)
                ( 221, -1.272965676)
                ( 240, -1.283737773)
                ( 271, -1.294627173)
                ( 317, -1.305636458)
                ( 335, -1.30933332 )
                ( 367, -1.313043899)
                ( 393, -1.328025453)
                ( 419, -1.32425897 )
                ( 452, -1.339410775)
                ( 483, -1.350927217)
                ( 507, -1.370421012)
                ( 539, -1.37436579 )
                ( 570, -1.394326533)
                ( 607, -1.406497068)
                ( 635, -1.418817553)
                ( 663, -1.439695138)
                ( 695, -1.443923474)
                ( 725, -1.452434164)
                ( 754, -1.456716825)
                ( 783, -1.474033275)
                ( 813, -1.48722028 )
                ( 841, -1.482805262)
                ( 872, -1.500583508)
                ( 901, -1.509592577)
                ( 932, -1.523260216)
                ( 962, -1.527857925)
                ( 989, -1.551169004)
                (1022, -1.565421027)
                (1050, -1.57987911 )
                (1078, -1.570217199)
                (1111, -1.5945493  )
                (1139, -1.609437912)
            };
            
        \end{axis}
        \end{tikzpicture}
        \caption{Grafico apresentando o logaritimo neperiano em função do tempo em segundos para a primeira solução}
        \end{figure}

        % ---------------------------------------------------------------------------- %
        %                                   Solução 2                                  %
        % ---------------------------------------------------------------------------- %

        % \tikzset{external/remake next=true}
        \begin{figure}\centering
        {\Large\bfseries{Solução 2}}\par\medskip
        \begin{tikzpicture}
        \begin{axis}
            [
                xmajorgrids=true,
                ymajorgrids=true,
                minor tick num=3,
                xminorgrids=true,
                yminorgrids=true,
                % legend pos  = north west
                % axis on top,
                ylabel={\(\ln{Abs}\)},
                xlabel={Tempo/\unit{\second}},
            ]
            % Legends
            \addlegendimage{empty legend}
            \addlegendentry[Emph]{\( y=-0.0005\,x-1.1086 \)}
            \addlegendimage{empty legend}
            \addlegendentry[Emph]{\( R^2=0.9922 \)}

            \addplot[
                no marks,
                dashed
            ] expression[
                domain=32:1123
            ] {
                -0.0005*x-1.1086
            };

            \addplot[
                mark=*,
                only marks,
            ] coordinates {
                (32,	-1.087672349)
                (60,	-1.108662625)
                (91,	-1.127011763)
                (119,	-1.142564176)
                (147,	-1.180907531)
                (181,	-1.190727578)
                (207,	-1.220779923)
                (239,	-1.234432012)
                (270,	-1.255266099)
                (298,	-1.265848208)
                (330,	-1.276543497)
                (357,	-1.287354413)
                (388,	-1.30933332)
                (415,	-1.328025453)
                (445,	-1.335601247)
                (474,	-1.358679194)
                (504,	-1.366491734)
                (544,	-1.386294361)
                (563,	-1.394326533)
                (593,	-1.402423743)
                (622,	-1.427116356)
                (653,	-1.435484605)
                (682,	-1.443923474)
                (715,	-1.452434164)
                (744,	-1.482805262)
                (771,	-1.491654877)
                (801,	-1.505077897)
                (831,	-1.523260216)
                (858,	-1.537117251)
                (888,	-1.537117251)
                (918,	-1.551169004)
                (950,	-1.560647748)
                (978,	-1.570217199)
                (1006,	-1.5847453)
                (1034,	-1.599487582)
                (1061,	-1.619488248)
                (1092,	-1.62455155)
                (1123,	-1.64506509)
            };
            
        \end{axis}
        \end{tikzpicture}
        \caption{Grafico apresentando o logaritimo neperiano em função do tempo em segundos para a segunda solução}
        \end{figure}

        % ---------------------------------------------------------------------------- %
        %                                   Solução 3                                  %
        % ---------------------------------------------------------------------------- %

        % \tikzset{external/remake next=true}
        \begin{figure}\centering
        {\Large\bfseries{Solução 3}}\par\medskip
        \begin{tikzpicture}
        \begin{axis}
            [
                xmajorgrids=true,
                ymajorgrids=true,
                minor tick num=3,
                xminorgrids=true,
                yminorgrids=true,
                % legend pos  = north west
                % axis on top,
                ylabel={\(\ln{Abs}\)},
                xlabel={Tempo/\unit{\second}},
            ]
            % Legends
            \addlegendimage{empty legend}
            \addlegendentry[Emph]{\( y=-0.0012\,x-1.0484 \)}
            \addlegendimage{empty legend}
            \addlegendentry[Emph]{\( R^2=0.9996 \)}

            \addplot[
                no marks,
                dashed
            ] expression[
                domain=30:1142
            ] {
                -0.0012*x-1.0484
            };

            \addplot[
                mark=*,
                only marks,
            ] coordinates {
                (30,	-1.090644119)
                (59,	-1.127011763)
                (90,	-1.161552088)
                (121,	-1.200645014)
                (149,	-1.22758267)
                (180,	-1.258781041)
                (212,	-1.301953213)
                (239,	-1.331806176)
                (270,	-1.366491734)
                (298,	-1.398366942)
                (330,	-1.439695138)
                (359,	-1.46967597)
                (389,	-1.514127733)
                (420,	-1.541779264)
                (452,	-1.5945493)
                (479,	-1.619488248)
                (509,	-1.655481851)
                (539,	-1.692819521)
                (568,	-1.737271284)
                (601,	-1.766091722)
                (629,	-1.795767491)
                (660,	-1.838851077)
                (690,	-1.870802677)
                (720,	-1.917322692)
                (750,	-1.944910649)
                (779,	-1.980501594)
                (810,	-2.009915479)
                (840,	-2.063568193)
                (869,	-2.103734234)
                (898,	-2.128631786)
                (929,	-2.162823151)
                (959,	-2.207274913)
                (990,	-2.244316185)
                (1020,	-2.282782466)
                (1052,	-2.302585093)
                (1079,	-2.3330443)
                (1110,	-2.353878387)
                (1142,	-2.419118909)
            };
            
        \end{axis}
        \end{tikzpicture}
        \caption{Grafico apresentando o logaritimo neperiano em função do tempo em segundos para a terceira solução}
        \end{figure}

        % ---------------------------------------------------------------------------- %
        %                                   Solução 4                                  %
        % ---------------------------------------------------------------------------- %

        % \tikzset{external/remake next=true}
        \begin{figure}\centering
        {\Large\bfseries{Solução 4}}\par\medskip
        \begin{tikzpicture}
        \begin{axis}
            [
                xmajorgrids=true,
                ymajorgrids=true,
                minor tick num=3,
                xminorgrids=true,
                yminorgrids=true,
                % legend pos  = north west
                % axis on top,
                ylabel={\(\ln{Abs}\)},
                xlabel={Tempo/\unit{\second}},
            ]
            % Legends
            \addlegendimage{empty legend}
            \addlegendentry[Emph]{\( y=-0.0027\,x-1.0666 \)}
            \addlegendimage{empty legend}
            \addlegendentry[Emph]{\( R^2=0.9963 \)}

            \addplot[
                no marks,
                dashed
            ] expression[
                domain=29:708
            ] {
                -0.0027*x-1.0666
            };

            \addplot[
                mark=*,
                only marks,
            ] coordinates {
                (29,	-1.148853505)
                (60,	-1.22758267)
                (92,	-1.30933332)
                (120,	-1.378326191)
                (148,	-1.46967597)
                (179,	-1.541779264)
                (206,	-1.63475572)
                (239,	-1.698269126)
                (265,	-1.795767491)
                (297,	-1.870802677)
                (324,	-1.951928221)
                (354,	-2.009915479)
                (382,	-2.095570924)
                (412,	-2.18036746)
                (441,	-2.225624052)
                (471,	-2.3330443)
                (503,	-2.419118909)
                (531,	-2.364460497)
                (562,	-2.63108916)
                (589,	-2.70306266)
                (619,	-2.764620553)
                (649,	-2.830217835)
                (678,	-2.882403588)
                (708,	-2.937463365)
            };
            
        \end{axis}
        \end{tikzpicture}
        \caption{Grafico apresentando o logaritimo neperiano em função do tempo em segundos para a quarta solução}
        \end{figure}

        % ---------------------------------------------------------------------------- %
        %                                   Solução 5                                  %
        % ---------------------------------------------------------------------------- %

        % \tikzset{external/remake next=true}
        \begin{figure}\centering
        {\Large\bfseries{Solução 5}}\par\medskip
        \begin{tikzpicture}
        \begin{axis}
            [
                xmajorgrids=true,
                ymajorgrids=true,
                minor tick num=3,
                xminorgrids=true,
                yminorgrids=true,
                % legend pos  = north west
                % axis on top,
                ylabel={\(\ln{Abs}\)},
                xlabel={Tempo/\unit{\second}},
            ]
            % Legends
            \addlegendimage{empty legend}
            \addlegendentry[Emph]{\( y=-0.003\,x-1.073 \)}
            \addlegendimage{empty legend}
            \addlegendentry[Emph]{\( R^2=0.9949 \)}

            \addplot[
                no marks,
                dashed
            ] expression[
                domain=32:622
            ] {
                -0.003*x-1.073
            };

            \addplot[
                mark=*,
                only marks,
            ] coordinates {
                (32,	-1.158362293)
                (64,	-1.241328591)
                (91,	-1.335601247)
                (121,	-1.431291727)
                (149,	-1.527857925)
                (181,	-1.63989712)
                (210,	-1.692819521)
                (239,	-1.666008264)
                (269,	-1.910543005)
                (302,	-1.951928221)
                (326,	-2.087473713)
                (361,	-2.171556831)
                (386,	-2.26336438)
                (417,	-2.343407088)
                (446,	-2.465104022)
                (475,	-2.501036032)
                (517,	-2.603690186)
                (534,	-2.645075402)
                (568,	-2.733368009)
                (593,	-2.813410717)
                (622,	-2.882403588)
            };
            
        \end{axis}
        \end{tikzpicture}
        \caption{Grafico apresentando o logaritimo neperiano em função do tempo em segundos para a quinta solução}
        \end{figure}

        
    \end{center}
    % \end{multicols}

\end{sectionBox}

% 88888888ba,    88
% 88      `"8b   ""
% 88        `8b
% 88         88  88  ,adPPYba,   ,adPPYba,  88       88  ,adPPYba,  ,adPPYba,  ,adPPYYba,   ,adPPYba,
% 88         88  88  I8[    ""  a8"     ""  88       88  I8[    ""  I8[    ""  ""     `Y8  a8"     "8a
% 88         8P  88   `"Y8ba,   8b          88       88   `"Y8ba,    `"Y8ba,   ,adPPPPP88  8b       d8
% 88      .a8P   88  aa    ]8I  "8a,   ,aa  "8a,   ,a88  aa    ]8I  aa    ]8I  88,    ,88  "8a,   ,a8"
% 88888888Y"'    88  `"YbbdP"'   `"Ybbd8"'   `"YbbdP'Y8  `"YbbdP"'  `"YbbdP"'  `"8bbdP"Y8   `"YbbdP"'

\begin{sectionBox}1{Discussão} % S
    
    \subsection{Análise da absorvância em função do tempo.}

    Como podemos observar nas tabelas das diferentes cinco soluções, ao longo do tempo os valores da absorvancia vao diminuindo e tonalidade da solução vai aclareando. Isto acontece pois à medida que decorre a reação o azul toluidina está a ser reduzido pelo sulfito, transformando se em branco de toluidina que absorve menos radiação. Com o aumento da concentração de \ch{Na2SO3} a velocidade da reação aumenta proporcionalmente, justificado pela equacação da velocidade da reação. Com este aumento, nota se uma maior descrepância  entre os valores da absorvância no memso intervalo de tempo.\newline

    \subsection{Análise dos gráficos de \textit{k'} e \textit{K}.}
    
    Na criação dos gráficos que seguem  a equação  \(\ln{A}=\ln{A}_0-k'\,t\), a linearidade dos pontos é fulcral para a obtenção de um valor preciso de \textit{k'}. Conseguimos observar que nas duas primeiras soluções as medicões das absorvâncias não foram executadas corretamnete o que causa uma perturbação na linearidade dos pontos da reta. Pelo contrário nas três ultimas soluções, as medicões tiveram uma melhor execução o que se reflete na proximidade ao valor 1 do coeficiente de determinação. A razão por de trás da má qualidade das primeiras medições em comparação com as últimas pode dever se ao facto, do número de medições efectuadas ou à contaminação do exterior da célula usada. Como consequência da falta de precisão nos valores de \textit{k'} obtivemos uma reta através da equação \(k'=k\,\ch{[SO3^{-2}]}_0\) cujo coeficente de determinação indica a falta de linearidade dos pontos. Assim concluimos teoricamente que o nosso valor de \textit{k} não seja aceitável.

\end{sectionBox}

\begin{sectionBox}2{Arrhenius (Tabela)} % S

    Através da equação \(k'=k\,\ch{[SO3^{-2}]}_0\), e dos dados disonibilizados para estudo, retiramos para cada concentração inicial de sulfito um valor de \textit{k} a uma temperatura específica. Com estes valores de \textit{k} às suas respetivas temperaturas, criámos gráficos de \(\ln(k)\) \times{} 1/Temperatura.

    \begin{center}

        \sisetup{
            % scientific / engineering / input / fixed
            exponent-mode           = scientific,
            exponent-to-prefix      = false,          % 1000 g -> 1 kg
            % exponent-product        = *,             % x * 10^y
            % fixed-exponent          = 0,
            round-mode              = places,        % figures/places/unsertanty/none
            round-precision         = 3,
            % round-minimum           = 0.01, % <x => 0
            % output-exponent-marker  = {\,\mathrm{E}},
        }

        \setlength\tabcolsep{9mm}        % width
        % \renewcommand\arraystretch{1.25} % height

        \vspace{1ex}
        \begin{tabular}{*{4}{C}}
            
            % --------------------------------- Para 0.02 -------------------------------- %

            \multicolumn{4}{l}{Para \ch{[SO3^{-2}]} 0.02}
            \\\toprule
            
                k/\unit{\M^{-1}.\second^{-1}} 
                & \ln{k}/\unit{\M^{-1}.\second^{-1}} 
                & T/\unit{\celsius}
                & T^{-1}/\unit{\celsius^{-1}}
            
            \\\midrule

                   2.88E-02 & \num{-3.549117512} & 20.0 & \num{0.05}
                \\ 3.68E-02 & \num{-3.303617053} & 22.5 & \num{0.044444444}
                \\ 4.08E-02 & \num{-3.199073198} & 25.0 & \num{0.04}
                \\ 4.59E-02 & \num{-3.082380080} & 30.0 & \num{0.033333333}
                \\ 6.00E-02 & \num{-2.813410717} & 31.0 & \num{0.032258065}
                \\ 2.65E-02 & \num{-3.632499121} & 20.0 & \num{0.05}
                \\ 2.73E-02 & \num{-3.600868577} & 21.9 & \num{0.0456621}
                \\ 5.85E-02 & \num{-2.838728525} & 35.0 & \num{0.028571429}
            
            \\\bottomrule

            % --------------------------------- Para 0.04 -------------------------------- %

            \multicolumn{4}{l}{Para \ch{[SO3^{-2}]} 0.04}
            \\\toprule
            
                k/\unit{\M^{-1}.\second^{-1}} 
                & \ln{k}/\unit{\M^{-1}.\second^{-1}} 
                & T/\unit{\celsius}
                & T^{-1}/\unit{\celsius^{-1}}
            
            \\\midrule

               2.75E-02 & \num{-3.593569274} & 20 & \num{0.05}
            \\ 4.05E-02 & \num{-3.206453305} & 22.5 & \num{0.044444444}
            \\ 3.75E-02 & \num{-3.283414346} & 25 & \num{0.04}
            \\ 5.13E-02 & \num{-2.971039661} & 31 & \num{0.032258065}
            \\ 2.43E-02 & \num{-3.718308265} & 20 & \num{0.05}
            \\ 1.83E-02 & \num{-4.002221273} & 21.9 & \num{0.0456621}
            \\ 6.60E-02 & \num{-2.718100537} & 35 & \num{0.028571429}
            
            \\\bottomrule

            % --------------------------------- Para 0.06 -------------------------------- %

            \multicolumn{4}{l}{Para \ch{[SO3^{-2}]} 0.06}
            \\\toprule
            
                k/\unit{\M^{-1}.\second^{-1}} 
                & \ln{k}/\unit{\M^{-1}.\second^{-1}} 
                & T/\unit{\celsius}
                & T^{-1}/\unit{\celsius^{-1}}
            
            \\\midrule

               2.55E-02 & \num{-3.669076827} & 20 & \num{0.05}
            \\ 5.02E-02 & \num{-2.992404483} & 22.5 & \num{0.044444444}
            \\ 3.07E-02 & \num{-3.484578991} & 25 & \num{0.04}
            \\ 6.43E-02 & \num{-2.743677379} & 30 & \num{0.033333333}
            \\ 6.88E-02 & \num{-2.676067155} & 31 & \num{0.032258065}
            \\ 2.40E-02 & \num{-3.729701449} & 20 & \num{0.05}
            \\ 1.73E-02 & \num{-4.055123849} & 21.9 & \num{0.0456621}
            \\ 6.10E-02 & \num{-2.796881415} & 35 & \num{0.028571429}
            
            \\\bottomrule

            % --------------------------------- Para 0.08 -------------------------------- %

            \multicolumn{4}{l}{Para \ch{[SO3^{-2}]} 0.08}
            \\\toprule
            
                k/\unit{\M^{-1}.\second^{-1}} 
                & \ln{k}/\unit{\M^{-1}.\second^{-1}} 
                & T/\unit{\celsius}
                & T^{-1}/\unit{\celsius^{-1}}
            
            \\\midrule

               2.64E-02 & \num{-3.635338687} & 20 & \num{0.05}
            \\ 2.86E-02 & \num{-3.553474817} & 22.5 & \num{0.044444444}
            \\ 3.00E-02 & \num{-3.506557897} & 25 & \num{0.04}
            \\ 7.16E-02 & \num{-2.636311104} & 30 & \num{0.033333333}
            \\ 6.04E-02 & \num{-2.807180167} & 31 & \num{0.032258065}
            \\ 2.19E-02 & \num{-3.822410847} & 20 & \num{0.05}
            \\ 1.68E-02 & \num{-4.089357021} & 21.9 & \num{0.0456621}
            \\ 6.35E-02 & \num{-2.756715373} & 35 & \num{0.028571429}
            
            \\\bottomrule

            % --------------------------------- Para 0.10 -------------------------------- %

            \multicolumn{4}{l}{Para \ch{[SO3^{-2}]} 0.10}
            \\\toprule
            
                k/\unit{\M^{-1}.\second^{-1}} 
                & \ln{k}/\unit{\M^{-1}.\second^{-1}} 
                & T/\unit{\celsius}
                & T^{-1}/\unit{\celsius^{-1}}
            
            \\\midrule

               2.82E-02 & \num{-3.568433301} & 20 & \num{0.05}
            \\ 3.87E-02 & \num{-3.251915679} & 22.5 & \num{0.044444444}
            \\ 3.33E-02 & \num{-3.402197882} & 25 & \num{0.04}
            \\ 2.01E-02 & \num{-3.907035464} & 20 & \num{0.05}
            \\ 1.69E-02 & \num{-4.080441657} & 21.9 & \num{0.0456621}
            \\ 6.09E-02 & \num{-2.798522104} & 35 & \num{0.028571429}
            
            \\\bottomrule
        \end{tabular}\\
        \tablecaption[lnKxt]{Dados para discussão}
        \vspace{2ex}
    \end{center}
    
\end{sectionBox}

\begin{sectionBox}2{Arrenius (Graficos)\cite{toluina}} % S


    % \tikzset{external/force remake=true}

    \pgfplotsset{height=7cm, width=.9\textwidth}

    % ---------------------------------------------------------------------------- %
    %                                   Sulfito 0.02                                  %
    % ---------------------------------------------------------------------------- %

    % \tikzset{external/remake next=true}
    \begin{figure}\centering
    {\Large\bfseries{\(\ch{[SO3^{-2}]_0}=0.02\,\unit{\M}\)}}\par\medskip
    \begin{tikzpicture}
    \begin{axis}
        [
            xmajorgrids=true,
            ymajorgrids=true,
            minor tick num=3,
            xminorgrids=true,
            yminorgrids=true,
            % axis on top,
            ylabel={\(\ln{k}/\unit{\M^{-1}.\second^{-1}}\)},
            xlabel={\(T^{-1}*10^2/\unit{\second^{-1}}\)},
        ]
        % Legends
        \addlegendimage{empty legend}
        \addlegendentry[Emph]{\(y=-37.621\,x-1.7276 \)}
        \addlegendimage{empty legend}
        \addlegendentry[Emph]{\(R^2=0.9117\)}

        \addplot[
            no marks,
            dashed
        ] expression[
            domain=2.85:5
        ] {
            -37.621e-2*x-1.7276
        };

        \addplot[
            mark=*,
            only marks,
        ] coordinates {
            (5.,        -3.549117512)
            (4.4444444, -3.303617053)
            (4.,        -3.199073198)
            (3.3333333, -3.08238008)
            (3.2258065, -2.813410717)
            (5.,        -3.632499121)
            (4.56621,   -3.600868577)
            (2.8571429, -2.838728525)
        };
        
    \end{axis}
    \end{tikzpicture}
    \caption{\(\ln\) constante da velocidade \times{} inverso do tempo para a amostra concentração de sulfito de 0.02, valores da tabela na seção: \ref{table:lnKxt}}
    \end{figure}

    % ---------------------------------------------------------------------------- %
    %                                   Sulfito 0.04                                  %
    % ---------------------------------------------------------------------------- %

    % \tikzset{external/remake next=true}
    \begin{figure}\centering
    {\Large\bfseries{\(\ch{[SO3^{-2}]_0}=0.04\,\unit{\M}\)}}\par\medskip
    \begin{tikzpicture}
    \begin{axis}
        [
            xmajorgrids=true,
            ymajorgrids=true,
            minor tick num=3,
            xminorgrids=true,
            yminorgrids=true,
            % axis on top,
            ylabel={\(\ln{k}/\unit{\M^{-1}.\second^{-1}}\)},
            xlabel={\(T^{-1}*10^2/\unit{\second^{-1}}\)},
        ]
        % Legends
        \addlegendimage{empty legend}
        \addlegendentry[Emph]{\(y=-45.293\,x-1.4737 \)}
        \addlegendimage{empty legend}
        \addlegendentry[Emph]{\(R^2=0.7343\)}

        \addplot[
            no marks,
            dashed
        ] expression[
            domain=2.85:5
        ] {
            -45.293e-2*x-1.4737
        };

        \addplot[
            mark=*,
            only marks,
        ] coordinates {
            (5.,        -3.593569274)
            (4.4444444, -3.206453305)
            (4.,        -3.283414346)   
            (3.2258065, -2.971039661)
            (5.,        -3.718308265)
            (4.56621,   -4.002221273)
            (2.8571429, -2.718100537)
        };
        
    \end{axis}
    \end{tikzpicture}
    \caption{\(\ln\) constante da velocidade \times{} inverso do tempo para a amostra concentração de sulfito de 0.04, valores da tabela na seção: \ref{table:lnKxt}}
    \end{figure}

    % ---------------------------------------------------------------------------- %
    %                                   Sulfito 0.06                                  %
    % ---------------------------------------------------------------------------- %

    % \tikzset{external/remake next=true}
    \begin{figure}\centering
    {\Large\bfseries{\(\ch{[SO3^{-2}]_0}=0.06\,\unit{\M}\)}}\par\medskip
    \begin{tikzpicture}
    \begin{axis}
        [
            xmajorgrids=true,
            ymajorgrids=true,
            minor tick num=3,
            xminorgrids=true,
            yminorgrids=true,
            % axis on top,
            ylabel={\(\ln{k}/\unit{\M^{-1}.\second^{-1}}\)},
            xlabel={\(T^{-1}*10^2/\unit{\second^{-1}}\)},
        ]
        % Legends
        \addlegendimage{empty legend}
        \addlegendentry[Emph]{\(y=-52.257\,x-1.1503\)}
        \addlegendimage{empty legend}
        \addlegendentry[Emph]{\(R^2=0.6738\)}

        \addplot[
            no marks,
            dashed
        ] expression[
            domain=2.85:5
        ] {
            -52.257e-2*x-1.1503
        };

        \addplot[
            mark=*,
            only marks,
        ] coordinates {
            (5.00000, -3.669076827)
            (4.44444, -2.992404483)
            (4.00000, -3.484578991)
            (3.33333, -2.743677379)
            (3.22581, -2.676067155)
            (5.00000, -3.729701449)
            (4.56621, -4.055123849)
            (2.85714, -2.796881415)
        };
        
    \end{axis}
    \end{tikzpicture}
    \caption{\(\ln\) constante da velocidade \times{} inverso do tempo para a amostra concentração de sulfito de 0.06, valores da tabela na seção: \ref{table:lnKxt}}
    \end{figure}

    % ---------------------------------------------------------------------------- %
    %                                   Sulfito 0.08                                  %
    % ---------------------------------------------------------------------------- %

    % \tikzset{external/remake next=true}
    \begin{figure}\centering
    {\Large\bfseries{\(\ch{[SO3^{-2}]_0}=0.08\,\unit{\M}\)}}\par\medskip
    \begin{tikzpicture}
    \begin{axis}
        [
            xmajorgrids=true,
            ymajorgrids=true,
            minor tick num=3,
            xminorgrids=true,
            yminorgrids=true,
            % axis on top,
            ylabel={\(\ln{k}/\unit{\M^{-1}.\second^{-1}}\)},
            xlabel={\(T^{-1}*10^2/\unit{\second}^{-1}\)},
        ]
        % Legends
        \addlegendimage{empty legend}
        \addlegendentry[Emph]{\(y=-58.624\,x-0.9747 \)}
        \addlegendimage{empty legend}
        \addlegendentry[Emph]{\(R^2=0.8032\)}

        \addplot[
            no marks,
            dashed
        ] expression[
            domain=2.85:5
        ] {
            -58.624e-2*x-0.9747
        };

        \addplot[
            mark=*,
            only marks,
        ] coordinates {
            (5.00000, -3.635338687)
            (4.44444, -3.553474817)
            (4.00000, -3.506557897)
            (3.33333, -2.636311104)
            (3.22581, -2.807180167)
            (5.00000, -3.822410847)
            (4.56621, -4.089357021)
            (2.85714, -2.756715373)
        };
        
    \end{axis}
    \end{tikzpicture}
    \caption{\(\ln\) constante da velocidade \times{} inverso do tempo para a amostra concentração de sulfito de 0.08, valores da tabela na seção: \ref{table:lnKxt}}
    \end{figure}

    % ---------------------------------------------------------------------------- %
    %                                   Sulfito 0.10                                  %
    % ---------------------------------------------------------------------------- %

    % \tikzset{external/remake next=true}
    \begin{figure}\centering
    {\Large\bfseries{\(\ch{[SO3^{-2}]_0}=0.10\,\unit{\M}\)}}\par\medskip
    \begin{tikzpicture}
    \begin{axis}
        [
            xmajorgrids=true,
            ymajorgrids=true,
            minor tick num=3,
            xminorgrids=true,
            yminorgrids=true,
            % axis on top,
            ylabel={\(\ln{k}/\unit{\M^{-1}.\second^{-1}}\)},
            xlabel={\(T^{-1}*10^2/\unit{\second}^{-1}\)},
        ]
        % Legends
        \addlegendimage{empty legend}
        \addlegendentry[Emph]{\(y=-48.037\,x-1.5166\)}
        \addlegendimage{empty legend}
        \addlegendentry[Emph]{\(R^2=0.6421\)}

        \addplot[
            no marks,
            dashed
        ] expression[
            domain=2.85:5
        ] {
            -46.037e-2*x-1.5166
        };

        \addplot[
            mark=*,
            only marks,
        ] coordinates {
            (5.000, -3.568433301)
            (4.444, -3.251915679)
            (4.000, -3.402197882)
            (5.000, -3.907035464)
            (4.566, -4.080441657)
            (2.857, -2.798522104)
        };
        
    \end{axis}
    \end{tikzpicture}
    \caption{\(\ln\) constante da velocidade \times{} inverso do tempo para a amostra concentração de sulfito de 0.10, valores da tabela na seção: \ref{table:lnKxt}}
    \end{figure}
    
    Em cada gráfico obtivemos uma reta que segue a lei de arrhenius, \(\ln{k}=\ln{A}-E_a/(RT)\), cujo declive representa \(-E_a/R\). Assim conseguimos obter a energia de ativação, a energia cinética mínima que os reagentes devem ter para formar produtos, para cada concentração de Sulfito.

    \begin{center}
        \vspace{1ex}
        \sisetup{
            % scientific / engineering / input / fixed
            exponent-mode           = scientific,
            exponent-to-prefix      = false,          % 1000 g -> 1 kg
            % exponent-product        = *,             % x * 10^y
            % fixed-exponent          = 0,
            round-mode              = places,        % figures/places/unsertanty/none
            round-precision         = 5,
            % round-minimum           = 0.01, % <x => 0
            % output-exponent-marker  = {\,\mathrm{E}},
        }
        \begin{tabular}{*{3}{C}}
            \toprule
            
                \ch{[SO3^{-2}]_0}/\unit{\M}
                & m & E_a/\unit{\joule.\mole^{-1}}
            
            \\\midrule
            
               0.02 & -37.621 & \num{312780.994}
            \\ 0.04 & -45.293 & \num{376566.002}
            \\ 0.06 & -52.257 & \num{434464.698}
            \\ 0.08 & -58.624 & \num{487399.936}
            \\ 0.10 & -46.037 & \num{382751.618}
            
            \\\bottomrule
        \end{tabular}
        \vspace{2ex}
    \end{center}

\end{sectionBox}

%   ,ad8888ba,                              ad88  88
%  d8"'    `"8b                            d8"    ""
% d8'                                      88
% 88             8b,dPPYba,  ,adPPYYba,  MM88MMM  88   ,adPPYba,   ,adPPYba,
% 88      88888  88P'   "Y8  ""     `Y8    88     88  a8"     ""  a8"     "8a
% Y8,        88  88          ,adPPPPP88    88     88  8b          8b       d8
%  Y8a.    .a88  88          88,    ,88    88     88  "8a,   ,aa  "8a,   ,a8"
%   `"Y88888P"   88          `"8bbdP"Y8    88     88   `"Ybbd8"'   `"YbbdP"'

%        db
%       d88b                                                      ,d
%      d8'`8b                                                     88
%     d8'  `8b      88,dPYba,,adPYba,    ,adPPYba,   ,adPPYba,  MM88MMM  8b,dPPYba,  ,adPPYYba,
%    d8YaaaaY8b     88P'   "88"    "8a  a8"     "8a  I8[    ""    88     88P'   "Y8  ""     `Y8
%   d8""""""""8b    88      88      88  8b       d8   `"Y8ba,     88     88          ,adPPPPP88
%  d8'        `8b   88      88      88  "8a,   ,a8"  aa    ]8I    88,    88          88,    ,88
% d8'          `8b  88      88      88   `"YbbdP"'   `"YbbdP"'    "Y888  88          `"8bbdP"Y8

\begin{sectionBox}2{Força iônica variável} % S
    % \tikzset{external/force remake=true}

    \pgfplotsset{height=7cm, width=.9\textwidth}

    % ---------------------------------------------------------------------------- %
    %                                   Solução 9                                  %
    % ---------------------------------------------------------------------------- %

    % \tikzset{external/force remake=true}

    % \tikzset{external/remake next=true}
    \begin{figure}\centering
    {\Large\bfseries{Amostra 9}}\par\medskip
    \begin{tikzpicture}
    \begin{axis}
        [
            xmajorgrids=true,
            ymajorgrids=true,
            minor tick num=3,
            xminorgrids=true,
            yminorgrids=true,
            legend pos=north east,
            % axis on top,
            ylabel={\(\ln{k'}\)},
            xlabel={\(\sqrt{I}\)},
        ]
        % Legends
        \addlegendimage{empty legend}
        \addlegendentry[Emph]{\( y=0.0614\,x+4*10^{-5} \)}
        \addlegendimage{empty legend}
        \addlegendentry[Emph]{\( R^2=0.9961 \)}

        \addplot[
            no marks,
            dashed
        ] expression[
            domain=2.4:4.9
        ] {
            -1.4199e-1*x-2.6159
        };

        \addplot[
            mark=*,
            only marks,
        ] coordinates {
            (2.44948974, -2.974694135)
            (3.00000000, -3.080921908)
            (3.46410162, -3.079354999)
            (3.87298335, -3.106793247)
            (4.89897949, -3.348721986)
        };
        
    \end{axis}
    \end{tikzpicture}
    \caption{\(\ln\) da pseudo constante da velocidade \times{} \(\sqrt{\text{força iônica}}\) para a amostra 9, valores da tabela na seção: \ref{table:tabKCComp2}}
    \end{figure}

    % ---------------------------------------------------------------------------- %
    %                                   Solução 10                                 %
    % ---------------------------------------------------------------------------- %

    % \tikzset{external/remake next=true}
    \begin{figure}\centering
    {\Large\bfseries{Amostra 10}}\par\medskip
    \begin{tikzpicture}
    \begin{axis}
        [
            xmajorgrids=true,
            ymajorgrids=true,
            minor tick num=3,
            xminorgrids=true,
            yminorgrids=true,
            legend pos=north east,
            % axis on top,
            ylabel={\(\ln{k'}\)},
            xlabel={\(\sqrt{I}\)},
        ]
        % Legends
        \addlegendimage{empty legend}
        \addlegendentry[Emph]{\( y=-0.6638\,x-2.8165 \)}
        \addlegendimage{empty legend}
        \addlegendentry[Emph]{\( R^2=0.9618 \)}

        \addplot[
            no marks,
            dashed
        ] expression[
            domain=2.4:4.9
        ] {
            -0.6638e-1*x-2.8165
        };

        \addplot[
            mark=*,
            only marks,
        ] coordinates {
            (2.44948974, -2.978810701)
            (3.00000000, -3.006123085)
            (3.46410162, -3.044793462)
            (3.87298335, -3.094204120)
            (4.89897949, -3.132532512)
        };
        
    \end{axis}
    \end{tikzpicture}
    \caption{\(\ln\) da pseudo constante da velocidade \times{} \(\sqrt{\text{força iônica}}\) para a amostra 10, valores da tabela na seção: \ref{table:tabKCComp2}}
    \end{figure}

    % ---------------------------------------------------------------------------- %
    %                                   Solução 11                                 %
    % ---------------------------------------------------------------------------- %

    \tikzset{external/remake next=true}
    \begin{figure}\centering
    {\Large\bfseries{Amostra 11}}\par\medskip
    \begin{tikzpicture}
    \begin{axis}
        [
            xmajorgrids=true,
            ymajorgrids=true,
            minor tick num=3,
            xminorgrids=true,
            yminorgrids=true,
            legend pos=north east,
            % axis on top,
            ylabel={\(\ln{k'}\)},
            xlabel={\(\sqrt{I}\)},
        ]
        % Legends
        \addlegendimage{empty legend}
        \addlegendentry[Emph]{\( y=-0.4195\,x-2.7854 \)}
        \addlegendimage{empty legend}
        \addlegendentry[Emph]{\( R^2=0.6854 \)}

        \addplot[
            no marks,
            dashed
        ] expression[
            domain=2.4:4.9
        ] {
            -0.4195e-1*x-2.7854
        };

        \addplot[
            mark=*,
            only marks,
        ] coordinates {
            (2.44948974, -2.889410290)
            (3.00000000, -2.896196279)
            (3.46410162, -2.970616222)
            (3.87298335, -2.917214630)
            (4.89897949, -2.995678626)
        };
        
    \end{axis}
    \end{tikzpicture}
    \caption{\(\ln\) da pseudo constante da velocidade \times{} \(\sqrt{\text{força iônica}}\) para a amostra 11, valores da tabela na seção: \ref{table:tabKCComp2}}
    \end{figure}

\end{sectionBox}

% % 888888888888              88                       88
% %      88                   88                       88
% %      88                   88                       88
% %      88       ,adPPYYba,  88,dPPYba,    ,adPPYba,  88  ,adPPYYba,  ,adPPYba,
% %      88       ""     `Y8  88P'    "8a  a8P_____88  88  ""     `Y8  I8[    ""
% %      88       ,adPPPPP88  88       d8  8PP"""""""  88  ,adPPPPP88   `"Y8ba,
% %      88       88,    ,88  88b,   ,a8"  "8b,   ,aa  88  88,    ,88  aa    ]8I
% %      88       `"8bbdP"Y8  8Y"Ybbd8"'    `"Ybbd8"'  88  `"8bbdP"Y8  `"YbbdP"'

% %  ad88888ba                88
% % d8"     "8b               88
% % Y8,                       88
% % `Y8aaaaa,     ,adPPYba,   88
% %   `"""""8b,  a8"     "8a  88
% %         `8b  8b       d8  88
% % Y8a     a8P  "8a,   ,a8"  88
% %  "Y88888P"    `"YbbdP"'   88

\begin{sectionBox}*{} % S
    
    \subsection*{Calculando a constante cinética}

    \begin{center}

        \setlength\tabcolsep{6mm}        % width
        % \renewcommand\arraystretch{1.25} % height

            
        % \vspace{1ex}
        \begin{tabular}{*{4}{C}}
            \toprule
            
                \multicolumn{1}{c}{Solução}
                & k'/\unit{\second^{-1}}
                & \ch{[SO3^{-2}]}_0/\unit{\M}
                & I/\unit{\M}
            
            \\\midrule

               1 & 0.0004 & 0.02 &
            \multirow{5}{*}{0.49}
            \\ 2 & 0.0005 & 0.04 &
            \\ 3 & 0.0012 & 0.06 &
            \\ 4 & 0.0027 & 0.08 &
            \\ 5 & 0.0030 & 0.10 &

            \\\bottomrule
        \end{tabular}\\
        \tablecaption{Declive das soluções sobre a concentração de \ch{[SO3^{-2}]}}
        \vspace*{1ex}

        % \tikzset{external/remake next=true}
        \begin{figure}\centering
            % {\Large\bfseries{Solução 3}}\par\medskip
            
            \pgfplotsset{height=7cm, width=.9\textwidth}
            \begin{tikzpicture}
            \begin{axis}
                [
                    xmajorgrids=true,
                    ymajorgrids=true,
                    minor tick num=3,
                    xminorgrids=true,
                    yminorgrids=true,
                    legend pos=south east,
                    % axis on top,
                    ylabel={\(k'*10^3\)},
                    xlabel={\(\ch{[SO3^{-2}]_0}*10\)},
                ]
                % Legends
                \addlegendimage{empty legend}
                \addlegendentry[Emph]{\( y=0.037\,x-0.0007 \)}
                \addlegendimage{empty legend}
                \addlegendentry[Emph]{\( R^2=0.9169 \)}
    
                \addplot[
                    no marks,
                    dashed
                ] expression[
                    domain=0.2:1
                ] {
                    0.037e2*x-0.0007e3
                };
    
                \addplot[
                    mark=*,
                    only marks,
                ] coordinates {
                    (0.2, 0.4)
                    (0.4, 0.5)
                    (0.6, 1.2)
                    (0.8, 2.7)
                    (1.0, 3.0)
                };
                
            \end{axis}
            \end{tikzpicture}
            \caption{Grafico cruzando a constante cinética aparente com a concentração de \ch{SO3^{-2}}, seu declive revela a constante cinética real da reação}
            \end{figure}


            \begin{BM}\huge
                \therefore k_{\text{reação}}=0.037\,\unit{\second^{-1}.\mole^{-1}}
            \end{BM}
        % \includegraphics[width=.9\textwidth]{kxc.png}

    \end{center}
    

\end{sectionBox}


% % 88888888ba,                         88
% % 88      `"8b                        88
% % 88        `8b                       88
% % 88         88  ,adPPYYba,   ,adPPYb,88   ,adPPYba,   ,adPPYba,
% % 88         88  ""     `Y8  a8"    `Y88  a8"     "8a  I8[    ""
% % 88         8P  ,adPPPPP88  8b       88  8b       d8   `"Y8ba,
% % 88      .a8P   88,    ,88  "8a,   ,d88  "8a,   ,a8"  aa    ]8I
% % 88888888Y"'    `"8bbdP"Y8   `"8bbdP"Y8   `"YbbdP"'   `"YbbdP"'

\begin{sectionBox}1{Conclusão} % S
    
    Após a realização da atividade experimental, podemos concluir que conseguimos obter os resultados previstos. De uma maneira geral, conseguimos preparar de forma adequada as soluções e os valores obtidos foram de acordo aos valores teóricos que nos foram dados.\newline

    Apesar de alguns erros associados, sendo estes: o mau manuseamento da célula, a má colocação da célula no espetrofotómetro, a abertura constante da tampa para a retirada da célula, e algumas condições alheias a nossa vontade podem ter interferido na obtenção de valores mais exatos e assim feito com que não tenhamos obtido um valor de \textit{K} tão próximo ao valor tabelado.\newline

    Por fim, concluímos que o método usado para calcular a constante cinética das reações de redução do corante azul de toluidina pelo ião sulfito é um processo eficaz.
    
\end{sectionBox}


%   ,ad8888ba,                                       88
%  d8"'    `"8b                                      88
% d8'                                                88
% 88              ,adPPYba,  8b,dPPYba,  ,adPPYYba,  88
% 88      88888  a8P_____88  88P'   "Y8  ""     `Y8  88
% Y8,        88  8PP"""""""  88          ,adPPPPP88  88
%  Y8a.    .a88  "8b,   ,aa  88          88,    ,88  88
%   `"Y88888P"    `"Ybbd8"'  88          `"8bbdP"Y8  88

\begin{sectionBox}1{Dados} % S
    
    \subsection{Dados Para Estudo}
    
    \begin{center}
    % \vspace{1ex}
    \sisetup{
        % scientific / engineering / input / fixed
        exponent-mode           = scientific,
        exponent-to-prefix      = false,          % 1000 g -> 1 kg
        % exponent-product        = *,             % x * 10^y
        % fixed-exponent          = 0,
        round-mode              = figures,        % figures/places/unsertanty/none
        round-precision         = 3,
        % round-minimum           = 0.01, % <x => 0
        % output-exponent-marker  = {\,\mathrm{E}},
    }
    \setlength\tabcolsep{5mm}        % width
    % \renewcommand\arraystretch{1.25} % height
    \begin{tabular}{*{6}{C}}
        \toprule
        
            \text{Amostra}
            & T/\unit{\celsius}
            & \multicolumn{1}{c}{Solução}
            & k'/\unit{\second^{-1}}
            & \ch{[SO3^{-2}]}_0/\unit{\M}
            & I/\unit{\M}
        
        \\\midrule

            \multirow{5}{*}{1}
            & \multirow{5}{*}{20}
                & 1 & \num{5.75E-04} & 0.02
            & \multirow{5}{*}{0.3}
            \\ && 2 & \num{1.10E-03} & 0.04 &
            \\ && 3 & \num{1.53E-03} & 0.06 &
            \\ && 4 & \num{2.11E-03} & 0.08 &
            \\ && 5 & \num{2.82E-03} & 0.10 &
            \\\midrule
            \multirow{5}{*}{2}
            & \multirow{5}{*}{22.5}
                & 1 & \num{7.35E-04} & 0.02
            & \multirow{5}{*}{0.3}
            \\ && 2 & \num{1.62E-03} & 0.04 &
            \\ && 3 & \num{3.01E-03} & 0.06 &
            \\ && 4 & \num{2.29E-03} & 0.08 &
            \\ && 5 & \num{3.87E-03} & 0.10 &
            \\\midrule
            \multirow{5}{*}{3}
            & \multirow{5}{*}{25}
                & 1 & \num{8.16E-04} & 0.02
            & \multirow{5}{*}{0.3}
            \\ && 2 & \num{1.50E-03} & 0.04 &
            \\ && 3 & \num{1.84E-03} & 0.06 &
            \\ && 4 & \num{2.40E-03} & 0.08 &
            \\ && 5 & \num{3.33E-03} & 0.10 &
            \\\midrule
            \multirow{3}{*}{4}
            & \multirow{3}{*}{30}
                & 1 & \num{9.17E-04} & 0.02 
            & \multirow{3}{*}{0.3}
            \\ && 2 & \num{3.86E-03} & 0.06 &
            \\ && 3 & \num{5.73E-03} & 0.08 &
            \\\midrule
            \multirow{4}{*}{5}
            & \multirow{4}{*}{31}
                & 1 & \num{1.20E-03} & 0.02 
            & \multirow{4}{*}{0.3}
            \\ && 2 & \num{2.05E-03} & 0.04 &
            \\ && 3 & \num{4.13E-03} & 0.06 &
            \\ && 4 & \num{4.83E-03} & 0.08 &
            \\\midrule
            \multirow{5}{*}{6}
            & \multirow{5}{*}{20}
                & 1 & \num{5.29E-04} & 0.02
            & \multirow{5}{*}{0.39}
            \\ && 2 & \num{9.71E-04} & 0.04 &
            \\ && 3 & \num{1.44E-03} & 0.06 &
            \\ && 4 & \num{1.75E-03} & 0.08 &
            \\ && 5 & \num{2.01E-03} & 0.10 &
            \\\midrule
            \multirow{5}{*}{7}
            & \multirow{5}{*}{21.9}
                & 1 & \num{5.46E-04} & 0.02
            & \multirow{5}{*}{0.48}
            \\ && 2 & \num{7.31E-04} & 0.04 &
            \\ && 3 & \num{1.04E-03} & 0.06 &
            \\ && 4 & \num{1.34E-03} & 0.08 &
            \\ && 5 & \num{1.69E-03} & 0.10 &
            \\\midrule
            \multirow{5}{*}{8}
            & \multirow{5}{*}{35}
                & 1 & \num{1.17E-03} & 0.02
            & \multirow{5}{*}{0.48}
            \\ && 2 & \num{2.64E-03} & 0.04 &
            \\ && 3 & \num{3.66E-03} & 0.06 &
            \\ && 4 & \num{5.08E-03} & 0.08 &
            \\ && 5 & \num{6.09E-03} & 0.10 &
            \\\bottomrule
        \end{tabular}\\
        \tablecaption[tabKCComp1]{Temperatura, solução, \(k'\) concentração de \(\ch{SO3^{-2}}\) e força ionica (\textit{I}) das amostras 1 a 8}

        \vspace*{1ex}
        \setlength\tabcolsep{2mm}        % width
        % \renewcommand\arraystretch{1.25} % height
        \begin{tabular}{*{8}{C}}
            \toprule
            
                \text{Amostra}
                & T/\unit{\celsius}
                & \multicolumn{1}{c}{Solução}
                & k'/\unit{\second^{-1}}
                & \ln{k'}/\unit{\second^{-1}}
                & \ch{[SO3^{-2}]}_0/\unit{\M}
                & I/\unit{\M}
                & \sqrt{I}/\unit{\M}
            
            \\\midrule
            \multirow{5}{*}{9}
            & \multirow{5}{*}{21}
                & 1 & \num{1.06E-03} & \num{-2.974694135} & 0.02 & 0.06 & \num{0.244948974}
            \\ && 2 & \num{8.30E-04} & \num{-3.080921908} & 0.02 & 0.09 & \num{0.300000000}
            \\ && 3 & \num{8.33E-04} & \num{-3.079354999} & 0.02 & 0.12 & \num{0.346410162}
            \\ && 4 & \num{7.82E-04} & \num{-3.106793247} & 0.02 & 0.15 & \num{0.387298335}
            \\ && 5 & \num{4.48E-04} & \num{-3.348721986} & 0.02 & 0.24 & \num{0.489897949}
            \\\midrule
            \multirow{5}{*}{10}
            & \multirow{5}{*}{21}
                & 1 & \num{1.05E-03} & \num{-2.978810701} & 0.02 & 0.06 & \num{0.244948974}
            \\ && 2 & \num{9.86E-04} & \num{-3.006123085} & 0.02 & 0.09 & \num{0.300000000}
            \\ && 3 & \num{9.02E-04} & \num{-3.044793462} & 0.02 & 0.12 & \num{0.346410162}
            \\ && 4 & \num{8.05E-04} & \num{-3.094204120} & 0.02 & 0.15 & \num{0.387298335}
            \\ && 5 & \num{7.37E-04} & \num{-3.132532512} & 0.02 & 0.24 & \num{0.489897949}
            \\\midrule
            \multirow{5}{*}{11}
            & \multirow{5}{*}{24}
                & 1 & \num{1.29E-03} & \num{-2.889410290} & 0.02 & 0.06 & \num{0.244948974}
            \\ && 2 & \num{1.27E-03} & \num{-2.896196279} & 0.02 & 0.09 & \num{0.300000000}
            \\ && 3 & \num{1.07E-03} & \num{-2.970616222} & 0.02 & 0.12 & \num{0.346410162}
            \\ && 4 & \num{1.21E-03} & \num{-2.917214630} & 0.02 & 0.15 & \num{0.387298335}
            \\ && 5 & \num{1.01E-03} & \num{-2.995678626} & 0.02 & 0.24 & \num{0.489897949}

        \\\bottomrule
    \end{tabular}\\
    \tablecaption[tabKCComp2]{Temperatura, solução, \(k'\) concentração de \(\ch{SO3^{-2}}\) e força ionica (\textit{I}) das amostras 9 a 11}
    \vspace*{1ex}
    \end{center}

\end{sectionBox}

%  ad88888ba                88
% d8"     "8b               88
% Y8,                       88
% `Y8aaaaa,     ,adPPYba,   88
%   `"""""8b,  a8"     "8a  88
%         `8b  8b       d8  88
% Y8a     a8P  "8a,   ,a8"  88
%  "Y88888P"    `"YbbdP"'   88

%        db         88
%       d88b        88
%      d8'`8b       88
%     d8'  `8b      88,dPPYba,   ,adPPYba,
%    d8YaaaaY8b     88P'    "8a  I8[    ""
%   d8""""""""8b    88       d8   `"Y8ba,
%  d8'        `8b   88b,   ,a8"  aa    ]8I
% d8'          `8b  8Y"Ybbd8"'   `"YbbdP"'

\begin{sectionBox}*{} % S

    \begin{multicols}{2}

        \sisetup{
            % scientific / engineering / input / fixed
            exponent-mode           = engineering,
            exponent-to-prefix      = false,          % 1000 g -> 1 kg
            % exponent-product        = *,             % x * 10^y
            % fixed-exponent          = 0,
            round-mode              = places,        % figures/places/unsertanty/none
            round-precision         = 5,
            % round-minimum           = 0.01, % <x => 0
            % output-exponent-marker  = {\,\mathrm{E}},
        }

        \setlength\tabcolsep{3mm}        % width
        \renewcommand\arraystretch{1.25} % height


        % ---------------------------------------------------------------------------- %
        %                                   Solução 1                                  %
        % ---------------------------------------------------------------------------- %

        \begin{center}
            \vspace{1ex}
            \begin{tabular}{*{3}{C}}

                \multicolumn{3}{l}{Solução 1}
                \\\toprule
                
                    Abs
                    &\ln(Abs)
                    &\multicolumn{1}{c}{Tempo/\unit{\second}}
                
                \\\midrule
                
                  0.306 & \num{-1.184170177} & 30
                \\0.314 & \num{-1.158362293} & 64
                \\0.305 & \num{-1.187443502} & 83
                \\0.292 & \num{-1.231001477} & 113
                \\0.288 & \num{-1.244794799} & 150
                \\0.283 & \num{-1.262308381} & 197
                \\0.280 & \num{-1.272965676} & 221
                \\0.277 & \num{-1.283737773} & 240
                \\0.274 & \num{-1.294627173} & 271
                \\0.271 & \num{-1.305636458} & 317
                \\0.270 & \num{-1.30933332}  & 335
                \\0.269 & \num{-1.313043899} & 367
                \\0.265 & \num{-1.328025453} & 393
                \\0.266 & \num{-1.32425897}  & 419
                \\0.262 & \num{-1.339410775} & 452
                \\0.259 & \num{-1.350927217} & 483
                \\0.254 & \num{-1.370421012} & 507
                \\0.253 & \num{-1.37436579}  & 539
                \\0.248 & \num{-1.394326533} & 570
                \\0.245 & \num{-1.406497068} & 607
                \\0.242 & \num{-1.418817553} & 635
                \\0.237 & \num{-1.439695138} & 663
                \\0.236 & \num{-1.443923474} & 695
                \\0.234 & \num{-1.452434164} & 725
                \\0.233 & \num{-1.456716825} & 754
                \\0.229 & \num{-1.474033275} & 783
                \\0.226 & \num{-1.48722028}  & 813
                \\0.227 & \num{-1.482805262} & 841
                \\0.223 & \num{-1.500583508} & 872
                \\0.221 & \num{-1.509592577} & 901
                \\0.218 & \num{-1.523260216} & 932
                \\0.217 & \num{-1.527857925} & 962
                \\0.212 & \num{-1.551169004} & 989
                \\0.209 & \num{-1.565421027} & 1022
                \\0.206 & \num{-1.57987911}  & 1050
                \\0.208 & \num{-1.570217199} & 1078
                \\0.203 & \num{-1.5945493}   & 1111
                \\0.200 & \num{-1.609437912} & 1139
                
                \\\bottomrule
            \end{tabular}\\
            \tablecaption{Absorvancia e Tempo da solução 1}
            \vspace{1ex}

        % ---------------------------------------------------------------------------- %
        %                                   Solução 2                                  %
        % ---------------------------------------------------------------------------- %
            \vspace{1ex}
            \begin{tabular}{*{3}{C}}

                \multicolumn{3}{l}{Solução 2}
                \\\toprule
                
                    Abs
                    &\ln(Abs)
                    &\multicolumn{1}{c}{Tempo/\unit{\second}}
                
                \\\midrule
                
                  0.337 & \num{-1.087672349} & 32
                \\0.330 & \num{-1.108662625} & 60
                \\0.324 & \num{-1.127011763} & 91
                \\0.319 & \num{-1.142564176} & 119
                \\0.307 & \num{-1.180907531} & 147
                \\0.304 & \num{-1.190727578} & 181
                \\0.295 & \num{-1.220779923} & 207
                \\0.291 & \num{-1.234432012} & 239
                \\0.285 & \num{-1.255266099} & 270
                \\0.282 & \num{-1.265848208} & 298
                \\0.279 & \num{-1.276543497} & 330
                \\0.276 & \num{-1.287354413} & 357
                \\0.270 & \num{-1.30933332}  & 388
                \\0.265 & \num{-1.328025453} & 415
                \\0.263 & \num{-1.335601247} & 445
                \\0.257 & \num{-1.358679194} & 474
                \\0.255 & \num{-1.366491734} & 504
                \\0.250 & \num{-1.386294361} & 544
                \\0.248 & \num{-1.394326533} & 563
                \\0.246 & \num{-1.402423743} & 593
                \\0.240 & \num{-1.427116356} & 622
                \\0.238 & \num{-1.435484605} & 653
                \\0.236 & \num{-1.443923474} & 682
                \\0.234 & \num{-1.452434164} & 715
                \\0.227 & \num{-1.482805262} & 744
                \\0.225 & \num{-1.491654877} & 771
                \\0.222 & \num{-1.505077897} & 801
                \\0.218 & \num{-1.523260216} & 831
                \\0.215 & \num{-1.537117251} & 858
                \\0.215 & \num{-1.537117251} & 888
                \\0.212 & \num{-1.551169004} & 918
                \\0.210 & \num{-1.560647748} & 950
                \\0.208 & \num{-1.570217199} & 978
                \\0.205 & \num{-1.5847453}   & 1006
                \\0.202 & \num{-1.599487582} & 1034
                \\0.198 & \num{-1.619488248} & 1061
                \\0.197 & \num{-1.62455155}  & 1092
                \\0.193 & \num{-1.64506509}  & 1123
                
                \\\bottomrule
            \end{tabular}\\
            \tablecaption{Absorvancia e Tempo da solução 2}
            \vspace{1ex}
        \end{center}

        % ---------------------------------------------------------------------------- %
        %                                   Solução 3                                  %
        % ---------------------------------------------------------------------------- %

        \begin{center}
            \vspace{1ex}
            \begin{tabular}{*{3}{C}}

                \multicolumn{3}{l}{Solução 3}
                \\\toprule
                
                    Abs
                    &\ln(Abs)
                    &\multicolumn{1}{c}{Tempo/\unit{\second}}
                
                \\\midrule
                
                  0.336 & \num{-1.090644119} & 30
                \\0.324 & \num{-1.127011763} & 59
                \\0.313 & \num{-1.161552088} & 90
                \\0.301 & \num{-1.200645014} & 121
                \\0.293 & \num{-1.22758267}  & 149
                \\0.284 & \num{-1.258781041} & 180
                \\0.272 & \num{-1.301953213} & 212
                \\0.264 & \num{-1.331806176} & 239
                \\0.255 & \num{-1.366491734} & 270
                \\0.247 & \num{-1.398366942} & 298
                \\0.237 & \num{-1.439695138} & 330
                \\0.230 & \num{-1.46967597}  & 359
                \\0.22  & \num{-1.514127733} & 389
                \\0.214 & \num{-1.541779264} & 420
                \\0.203 & \num{-1.5945493}   & 452
                \\0.198 & \num{-1.619488248} & 479
                \\0.191 & \num{-1.655481851} & 509
                \\0.184 & \num{-1.692819521} & 539
                \\0.176 & \num{-1.737271284} & 568
                \\0.171 & \num{-1.766091722} & 601
                \\0.166 & \num{-1.795767491} & 629
                \\0.159 & \num{-1.838851077} & 660
                \\0.154 & \num{-1.870802677} & 690
                \\0.147 & \num{-1.917322692} & 720
                \\0.143 & \num{-1.944910649} & 750
                \\0.138 & \num{-1.980501594} & 779
                \\0.134 & \num{-2.009915479} & 810
                \\0.127 & \num{-2.063568193} & 840
                \\0.122 & \num{-2.103734234} & 869
                \\0.119 & \num{-2.128631786} & 898
                \\0.115 & \num{-2.162823151} & 929
                \\0.110 & \num{-2.207274913} & 959
                \\0.106 & \num{-2.244316185} & 990
                \\0.102 & \num{-2.282782466} & 1020
                \\0.100 & \num{-2.302585093} & 1052
                \\0.097 & \num{-2.3330443}   & 1079
                \\0.095 & \num{-2.353878387} & 1110
                \\0.089 & \num{-2.419118909} & 1142
                
                \\\bottomrule
            \end{tabular}\\
            \tablecaption{Absorvancia e Tempo da solução 3}
            \vspace{1ex}

        % ---------------------------------------------------------------------------- %
        %                                   Solução 4                                  %
        % ---------------------------------------------------------------------------- %

            \vspace{1ex}
            \begin{tabular}{*{3}{C}}

                \multicolumn{3}{l}{Solução 4}
                \\\toprule
                
                    Abs
                    &\ln(Abs)
                    &\multicolumn{1}{c}{Tempo/\unit{\second}}
                
                \\\midrule
                
                  0.317 & \num{-1.148853505} & 29
                \\0.293 & \num{-1.22758267}  & 60
                \\0.270 & \num{-1.30933332}  & 92
                \\0.252 & \num{-1.378326191} & 120
                \\0.230 & \num{-1.46967597}  & 148
                \\0.214 & \num{-1.541779264} & 179
                \\0.195 & \num{-1.63475572}  & 206
                \\0.183 & \num{-1.698269126} & 239
                \\0.166 & \num{-1.795767491} & 265
                \\0.154 & \num{-1.870802677} & 297
                \\0.142 & \num{-1.951928221} & 324
                \\0.134 & \num{-2.009915479} & 354
                \\0.123 & \num{-2.095570924} & 382
                \\0.113 & \num{-2.18036746}  & 412
                \\0.108 & \num{-2.225624052} & 441
                \\0.097 & \num{-2.3330443}   & 471
                \\0.089 & \num{-2.419118909} & 503
                \\0.094 & \num{-2.364460497} & 531
                \\0.072 & \num{-2.63108916}  & 562
                \\0.067 & \num{-2.70306266}  & 589
                \\0.063 & \num{-2.764620553} & 619
                \\0.059 & \num{-2.830217835} & 649
                \\0.056 & \num{-2.882403588} & 678
                \\0.053 & \num{-2.937463365} & 708
                
                \\\bottomrule
            \end{tabular}\\
            \tablecaption{Absorvancia e Tempo da solução 4}
            \vspace{1ex}

        % ---------------------------------------------------------------------------- %
        %                                   Solução 5                                  %
        % ---------------------------------------------------------------------------- %

            \vspace{1ex}
            \begin{tabular}{*{3}{C}}

                \multicolumn{3}{l}{Solução 5}
                \\\toprule
                
                    Abs
                    &\ln(Abs)
                    &\multicolumn{1}{c}{Tempo/\unit{\second}}
                
                \\\midrule
                
                  0.314 & \num{-1.158362293} &  32
                \\0.289 & \num{-1.241328591} &  64
                \\0.263 & \num{-1.335601247} &  91
                \\0.239 & \num{-1.431291727} & 121
                \\0.217 & \num{-1.527857925} & 149
                \\0.194 & \num{-1.63989712}  & 181
                \\0.184 & \num{-1.692819521} & 210
                \\0.189 & \num{-1.666008264} & 239
                \\0.148 & \num{-1.910543005} & 269
                \\0.142 & \num{-1.951928221} & 302
                \\0.124 & \num{-2.087473713} & 326
                \\0.114 & \num{-2.171556831} & 361
                \\0.104 & \num{-2.26336438}  & 386
                \\0.096 & \num{-2.343407088} & 417
                \\0.085 & \num{-2.465104022} & 446
                \\0.082 & \num{-2.501036032} & 475
                \\0.074 & \num{-2.603690186} & 517
                \\0.071 & \num{-2.645075402} & 534
                \\0.065 & \num{-2.733368009} & 568
                \\0.060 & \num{-2.813410717} & 593
                \\0.056 & \num{-2.882403588} & 622
                
                \\\bottomrule
            \end{tabular}\\
            \tablecaption{Absorvancia e Tempo da solução 5}
            \vspace{1ex}
        \end{center}

    \end{multicols}
    
\end{sectionBox}


% 88888888ba   88  88
% 88      "8b  ""  88
% 88      ,8P      88
% 88aaaaaa8P'  88  88,dPPYba,
% 88""""""8b,  88  88P'    "8a
% 88      `8b  88  88       d8
% 88      a8P  88  88b,   ,a8"
% 88888888P"   88  8Y"Ybbd8"'



\begin{sectionBox}1{Referencias} % S
    
    % \printbibliography[heading=none]{}
    \bibliographystyle{plain}
    \bibliography{.build/libraries/QF_A-Relatorio.bib}
    
\end{sectionBox}

\end{document}
% !TEX root = ./QF_A-Lab.3.tex
\providecommand\mainfilename{"./QF_A-Lab.tex"}
\providecommand \subfilename{}
\renewcommand   \subfilename{"./QF_A-Lab.3.tex"}
\documentclass[\mainfilename]{subfiles}

% \tikzset{external/force remake=true} % - remake all

\begin{document}

\graphicspath{{\subfix{./.build/figures/QF_A-Lab.3}}}

\mymakesubfile{3}
[QF A]
{Lab: Cinética da redução do corante azul de toluidina pelo ião sulfito} % Subfile Title
{Cinética da redução do corante azul de toluidina pelo ião sulfito} % Part Title

\part*{Introdução}

\begin{sectionBox}1{Reação} % S
    
    \subsection*{Azul de toluina (\ch{TB^+})}
    \begin{itemize}
        \item \(\lambda_{\max}=596\,\unit{\nano\metre}\)
        \item \(\varepsilon_{596\,\unit{\nano\metre}} = 24000\,\unit{M^{-1}\,\centi\metre^{-1}}\)
    \end{itemize}

    \begin{center}
        \includegraphics[width=.8\textwidth]{reacao.png}
    \end{center}

    \vspace{-3ex}
    \begin{BM}
        v
        =k\,\ch{[SO3^{2-}]_0[TB^+]}
        \cong k'\,\ch{[TB^+]}
        \\
        A = \varepsilon\,b\,\ch{[TB^+]}
        \\[2ex]
        \implies
        \ln{A}=\ln{A}_0-k'\,t
    \end{BM}

    \begin{itemize}
        \item Na equação da velocidade \(\ch{[SO3^{-2}]}=\ch{[SO3^{-2}]}_0,\,\therefore k\,\ch{[SO3]}_0=k'\)
        \item Relacionando a equação da velocidade com a da absorvancia temos a e.q. final que podemos receber os valores experimentalmente e retirar \textit{k'} do declive do gráfico
    \end{itemize}
    
\end{sectionBox}

\end{document}
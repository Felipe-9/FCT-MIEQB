% !TEX root = ./QF_A-Minitestes_Resoluções.2020.1.tex
\providecommand\mainfilename{"./QF_A-Minitestes_Resoluções.tex"}
\providecommand \subfilename{}
\renewcommand   \subfilename{"./QF_A-Minitestes_Resoluções.2020.1.tex"}
\documentclass[\mainfilename]{subfiles}

% \tikzset{external/force remake=true} % - remake all

\begin{document}

% \graphicspath{{\subfix{./.build/figures/QF_A-Minitestes_Resoluções.2020.1}}}
% \tikzsetexternalprefix{./.build/figures/QF_A-Minitestes_Resoluções.2020.1/}

\mymakesubfile{1}
[QF A]
{2020 Miniteste Resolução} % Subfile Title
{2020 Miniteste Resolução} % Part Title

\begin{sectionBox}{} % S
    
    Seja o seu no de aluno \textit{abcde} (5 algarismos). \(Y=a+b+c+d+e\). Por exemplo, para o no de aluno 56432, \(Y=20,\,(d+e)=5,\,(d-e)=1,\,d*e=6,(b+d)=9\), etc.

    \vspace{-3ex}
    \begin{BM}
        Y = 6+1+3+8+7 = 25
        \begin{cases}
            a=6\\b=1\\c=3\\d=8\\e=7
        \end{cases}
    \end{BM}
    
\end{sectionBox}

\begin{questionBox}1{ % Q1
    A decomposição da fosfina foi seguida através de medidas de pressão, a \textit{Y}\,\unit{\celsius}.
} % Q1
    
    \begin{center}
        {\large
            \ch{4 PH3\gas{} -> P4\gas{} + 6 H2\gas{}}
        }
        \vspace{1ex}
        \begin{tabular}{*{3}{C}}
            \toprule
            
                P/\unit{\bar}
                & Y/150 & Y/100
                \\
                t/\unit{\min}
                & 0 & 80
            
            \\\bottomrule
        \end{tabular}
        % \vspace{2ex}
    \end{center}
    
    Sabendo que a reação é de 1ª ordem em relação à fosfina, calcule a constante cinética a esta temperatura. Não se esqueça de indicar o resultado em unidades de \unit{\min^{-1}}. Insira só o valor numérico.

\end{questionBox}

\end{document}
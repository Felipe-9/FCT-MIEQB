% !TEX root = ./QF_A-Listas_Notas.5.tex
\providecommand\mainfilename{"./QF_A-Listas_Notas.tex"}
\providecommand \subfilename{}
\renewcommand   \subfilename{"./QF_A-Listas_Notas.5.tex"}
\documentclass[\mainfilename]{subfiles}

% \tikzset{external/force remake=true} % - remake all

\begin{document}

% \graphicspath{{\subfix{./.build/figures/QF_A-Listas_Notas.5}}}
% \tikzsetexternalprefix{./.build/figures/QF_A-Listas_Notas.5/}

\mymakesubfile{5}
[QF A]
{Superficies} % Subfile Title
{Superficies} % Part Title

\part*{Tensão Superficial}

\begin{sectionBox}1{Young-Laplace} % S
    
    \begin{BM}
        \adif{P}=2\,\gamma/r
    \end{BM}
    \begin{description}[
        leftmargin=!,
        labelwidth=\widthof{\(r=a/\cos\theta\)} % Longest item
    ]
       \item[\(\gamma\)] Tensão superficial (\unit{\newton/\metre})
       \item[\(r=a/\cos\theta\)] Raio de curvatura, onde \textit{a} é o raio interno do tubo capilar e \(\theta\) o angulo de contato
    \end{description}
    
\end{sectionBox}

\begin{sectionBox}1{Isotérmica de Gibbs} % S2
    
    \begin{BM}
        \Gamma
        =\frac{-1}{R\,T}
        \,\odv{\gamma}{\ln{C}}
    \end{BM}
    \begin{description}[
        leftmargin=!,
        labelwidth=\widthof{} % Longest item
    ]
       \item[Dominio] Só se aplica ao ramo decrescente
    \end{description}

    Plotando \(\gamma\times\ln{C}\) seu declive (\(\odv{\gamma}{\ln{C}}\)) pode ser aplicado na equação
    
\end{sectionBox}

\begin{sectionBox}1{Isotérmica de Langmuir} % S3
    
    \begin{BM}
        \theta
        =\left(
            1+(\alpha\,p)^{-1}
        \right)^{-1}
        =\frac{\alpha\,p}{1+\alpha\,p}
        \iff
        \frac{1}{\theta}
        =1+\frac{1}{\alpha\,p}
        \\ (\theta^{-1}\times p^{-1})
        \\
        \alpha=\frac{k_a}{k_d}
    \end{BM}

    \begin{description}[
        leftmargin=!,
        labelwidth=\widthof{} % Longest item
    ]
       \item[\(\theta\)] Fração da area em que ocorre a absorção
    \end{description}

    \begin{flalign*}
        &
            \odv{\theta}{t}
            =k_a\,p\,N(1-\theta)
            =-k_d\,N\,\theta
            % \implies
            \underset{equilibrium}{\implies}
            k_a\,p\,N(1-\theta)
            +k_d\,N\,\theta
            =0
            \implies &\\&
            \implies
            \theta
            = \frac{k_a\,p}{k_a\,p+k_d}
            = \frac{(k_a/k_d)\,p}{(k_a/k_d)\,p+(k_d/k_d)}
            = \frac{\alpha\,p}{\alpha\,p+1}
        &
    \end{flalign*}
    
\end{sectionBox}

\end{document}
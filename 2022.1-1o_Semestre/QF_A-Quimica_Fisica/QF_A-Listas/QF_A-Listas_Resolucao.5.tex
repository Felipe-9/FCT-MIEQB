% !TEX root = ./QF_A-Listas_Resolucao.5.tex
\providecommand\mainfilename{"./QF_A-Listas_Resolucao.tex"}
\providecommand \subfilename{}
\renewcommand   \subfilename{"./QF_A-Listas_Resolucao.5.tex"}
\documentclass[\mainfilename]{subfiles}

% \tikzset{external/force remake=true} % - remake all

\begin{document}

\graphicspath{{\subfix{./.build/figures/QF_A-Listas_Resolucao.5}}}
\tikzsetexternalprefix{./.build/figures/QF_A-Listas_Resolucao.5/}

\mymakesubfile{5}
[QF A]
{Lista Superficies} % Subfile Title
{Lista Superficies} % Part Title

\begin{questionBox}1{ % Q1
    A tensão superficial da água é de 72\,\unit{\milli\newton.\metre^{-1}} a 298\,\unit{\kelvin}. Calcule a elevação capilar num tubo de 0.5\,\unit{\milli\metre} de diâmetro para um ângulo de contacto de 30\unit{\degree}. 
    (\(\rho_{\ch{H2O}}=1.0\,\unit{\gram.\centi\metre^{-3}}\))
} % Q1
    \answer{}
    \begin{flalign*}
        &
            \rho\,g\,h
            =\adif{P}
            =\frac{2\,\gamma}{r}
            =\frac{2\,\gamma}{a/\cos\theta}
            \implies &\\&
            \implies
            h
            =\frac{
                2\,\gamma\,\cos(\theta)
            }{
                a\,\rho\,g
            }
            =\frac{
                2
                *(72\E-3)
                *\cos(30^\circ)
                % 124.707658144959165
            }{
                (0.5\E-3/2)
                *(1.0\E3)
                *\num{9.780327}
                % 2.44508175
            }
            \cong &\\&
            \cong
            \qty{51.003471824596116e-3}{\metre}
        &
    \end{flalign*}
\end{questionBox}

\begin{questionBox}1{} % Q2
    Medidas de tensão superficial de soluções aquosas do tensoativo CTAB16, a 20\unit{\celsius}, deram os seguintes resultados:

    \begin{center}
        \vspace{1ex}
        \begin{tabular}{L *{6}{C}}
            \toprule
            
                C*10^4/\unit{\M}
                & 1 & 2 & 5 & 10 & 50 & 100
            \\  \gamma/\unit{\milli\newton.\metre^{-1}}
                & 68 & 60 & 48 & 39 & 36 & 36
            
            \\\bottomrule
        \end{tabular}
        \vspace{2ex}
    \end{center}

    \begin{questionBox}2{ % Q2.1
        Usando a isotérmica de adsorção de Gibbs, calcule a área ocupada por molécula de CTAB16 à superfície da solução
    } % Q2.1
        \answer{}

        \begin{center}
            \setlength\tabcolsep{3mm}        % width
            % \renewcommand\arraystretch{1.25} % height
            \vspace{1ex}
            \begin{tabular}{L *{6}{C}}
                \toprule
                
                    \ln C/\unit{\M}
                    & \num{-9.210340371976182}
                    & \num{-8.517193191416238}
                    & \num{-7.600902459542082}
                    & \num{-6.907755278982137}
                    & \num{-5.298317366548036}
                    & \num{-4.605170185988091}
                \\  \gamma/\unit{\newton.\metre^{-1}}
                    & 0.068
                    & 0.06
                    & 0.048
                    & 0.039
                    & 0.036
                    & 0.036

                    % 0.068   ,-9.210340371976182
                    % 0.06    ,-8.517193191416238
                    % 0.048   ,-7.600902459542082
                    % 0.039   ,-6.907755278982137
                    % 0.036   ,-5.298317366548036
                    % 0.036   ,-4.605170185988091
            
                \\\bottomrule
            \end{tabular}
            \vspace{2ex}
        \end{center}

        \begin{center}
            % \tikzset{external/remake next=true}
            % \pgfplotsset{height=7cm, width= .6\textwidth}
            \begin{tikzpicture}
            \begin{axis}
                [
                    % xmajorgrids = true,
                    % legend pos  = north west
                    xlabel={\(\ln C/\unit{\M}\)},
                    ylabel={\(\gamma/\unit{\newton.\metre^{-1}}\)},
                    extra x ticks={-6.637795275590551}
                ]
                % Legends
                \addlegendimage{empty legend}
                \addlegendentry[Graph]{\(\gamma=-0.0127\ln{C}-0.0483\)}
                \addlegendimage{empty legend}
                \addlegendentry[GraphC]{\(\gamma=0.036\)}
                
                \addplot[
                    % smooth,
                    domain={-9.22:-6},
                    dashed,
                    opacity=.5,
                    samples=\mysampledensityDouble,
                ] { -0.0127*x-0.0483 };

                \addplot[
                    mark=*,
                    only marks,
                ] coordinates {
                    (-9.210340371976182, 0.068)
                    (-8.517193191416238, 0.060)
                    (-7.600902459542082, 0.048)
                    (-6.907755278982137, 0.039)
                    % (-5.298317366548036, 0.036)
                    % (-4.605170185988091, 0.036)
                };

                \addplot[
                    % smooth,
                    mark=*,
                    only marks,
                    GraphC,
                    mark options={
                        fill=GraphC,
                        fill opacity=1
                    },
                    % domain  = -2:2,
                    % samples = \mysampledensityDouble,
                ] coordinates {
                    % (-9.210340371976182, 0.068)
                    % (-8.517193191416238, 0.060)
                    % (-7.600902459542082, 0.048)
                    % (-6.907755278982137, 0.039)
                    (-5.298317366548036, 0.036)
                    (-4.605170185988091, 0.036)
                };

                \addplot[
                    domain={-7:-4.62},
                    dashed,
                    GraphC,
                    opacity=.5,
                    samples=\mysampledensityDouble,
                ] { 0.036 };
                
            \end{axis}
            \end{tikzpicture}
        \end{center}

        \begin{flalign*}
            &
                \Gamma
                =\frac{-1}{R\,T}
                \,\odv{\gamma}{\ln{C}}
                \cong\frac{-1}{
                    \num{8.314462618}
                    *(20+273.15)
                }
                (-0.0127)
                \cong
                \qty{5.210502845e-6}{\mole.\metre^{-2}}
                \implies &\\&
                \implies
                A
                \cong\frac{\unit{\metre^2}}{\qty{5.210502845e-6}{\mole}}
                \,\frac{1\,\unit{\mole}}{\num{6.02214076e23}\unit{moleculas}}
                \cong
                \qty{3.1869075146313e-19}{\metre^2.molecule^{-1}}
            &
        \end{flalign*}

    \end{questionBox}

    \begin{questionBox}2{ % Q2.2
        Calcule a concentração micelar crítica
    } % Q2.2
        \begin{flalign*}
            &
                -0.0127\ln(CMC)-0.0483=0.036
                \implies &\\&
                \implies
                CMC
                \cong\exp{\left(
                    -\frac{0.036+0.0483}{0.0127}
                \right)}
                % \cong &\\&
                \cong
                \qty{1.309912057646e-3}{\M}
            &
        \end{flalign*}
    \end{questionBox}
    
\end{questionBox}

\begin{questionBox}1{} % Q3
    A tabela seguinte dá o volume de azoto (a 0\unit{\celsius} e 1\unit{.\bar}) adsorvido por grama de carvão ativado a diferentes pressões:

    \begin{center}
        \vspace{1ex}
        \begin{tabular}{L *{5}{C}}
            \toprule
            
                p/\unit{\milli\bar}
                &  5.17
                & 17.08
                & 30.18
                & 44.75
                & 73.99
            \\  V/\unit{\centi\metre^3.\gram^{-1}}
                & 0.987
                & 3.04
                & 5.08
                & 7.04
                & 10.31
            
            \\\bottomrule
        \end{tabular}
        \vspace{2ex}
    \end{center}

    \begin{questionBox}2{ % Q3.1
        Construa um gráfico de forma a verificar a aplicabilidade da isotérmica de Langmuir
    } % Q3.1
        \answer{}

        \begin{flalign*}
            &
                n^{-1}
                =\frac{R\,T}{P\,V}
                \cong\frac{\num{8.314462618}*273.15}{1\E5*V*10^{-2}}
                \cong\frac{\num{2271.0954641067e-3}}{V}
            &
        \end{flalign*}

        \begin{center}
            \vspace{1ex}
            \setlength\tabcolsep{1mm}        % width
            % \renewcommand\arraystretch{1.25} % height
            \begin{tabular}{L *{5}{C}}
                \toprule
                
                    p^{-1}/\unit{\pascal^{-1}}
                    & \num{19342359.767891683}
                    & \num{5854800.936768151}
                    & \num{3313452.617627568}
                    & \num{2234636.87150838}
                    & \num{1351533.9910798757}
                \\  n^{-1}/\unit{\mole^{-1}.\gram}
                    & \num{2.301008575589362}
                    & \num{0.7470708763508882}
                    & \num{0.4470660362414764}
                    & \num{0.32259878751515625}
                    & \num{0.22028084035952472}
                
                \\\bottomrule
            \end{tabular}
            \vspace{2ex}
        \end{center}

        \begin{center}
            % \tikzset{external/remake next=true}
            \pgfplotsset{height=7cm, width= .8\textwidth}
            \begin{tikzpicture}
            \begin{axis}
                [
                    % xmajorgrids = true,
                    legend pos=south east,
                    % domain=0:4,
                    xlabel={\(p^{-1}/\unit{\pascal^{-1}}\)},
                    ylabel={\(n^{-1}/\unit{\mole^{-1}.\gram}\)},
                    extra y ticks={0.0653},
                ]
                % Legends
                \addlegendimage{empty legend}
                \addlegendentry[Graph]{\( n^{-1}=1.1564\E-7*p^{-1}+0.0653\)}

                \addplot[
                    Graph,dashed,
                    domain={0:19342359.767891683},
                    samples=\mysampledensityDouble,
                    opacity=0.5,
                ]{ 1.1564e-7*x+0.0653 };
                
                % Plot from equation
                \addplot[
                    mark=*,
                    only marks,
                    Graph,
                    % domain  = -2:2,
                    % samples = \mysampledensityDouble,
                ] coordinates {
                    (19342359.767891683, 2.301008575589362)
                    (5854800.936768151,  0.7470708763508882)
                    (3313452.617627568,  0.4470660362414764)
                    (2234636.87150838,   0.32259878751515625)
                    (1351533.9910798757, 0.22028084035952472)
                };
                
            \end{axis}
            \end{tikzpicture}
        \end{center}

    \end{questionBox}

    \begin{questionBox}2{ % Q3.2
        Determine a área superficial por grama de carvão, admitindo que a área ocupada por molecula de azoto é 16\unit{\AA^2} \((\AA^2=10^{-20}\))
    } % Q3.2
        \begin{flalign*}
            &
                A
                =\frac{N_{av}}{n_{\max}}
                \,16*10^{-20}
                \cong\frac{\num{6.02214076e23}}{0.0653}
                \,16*10^{-20}
                \cong
                \num{147556.282021439509954e-3}
            &
        \end{flalign*}
    \end{questionBox}

\end{questionBox}

\begin{questionBox}1{} % Q4
    Mediu-se a adsorção de metano em carvão ativado, a 20\unit{\celsius}, obtendo-se os seguintes resultados:
    \begin{center}
        \vspace{1ex}
        \begin{tabular}{*{2}{C}}
            \toprule
            
                n_{a}/\unit{\mole.\gram^{-1}}
                & p/\unit{\bar}
            
            \\\midrule
        
                   4.20\E-4 & 0.133 
                \\ 6.38\E-4 & 0.267 
                \\ 8.01\E-4 & 0.400 
                \\ 9.25\E-4 & 0.533
                
            \\\bottomrule
        \end{tabular}
        \vspace{2ex}
    \end{center}
    Utilizando a isotérmica de adsorção de Langmuir, calcule a fração de área ocupada pelo metano, \(q\ch{CH4}\), à pressão de 0.4\,\unit{\bar}.

    \answer{}
    \begin{center}
        \vspace{1ex}
        \begin{tabular}{*{2}{C}}
            \toprule
            
                n_{a}^{-1}/\unit{\mole^{-1}.\gram}
                & p^{-1}/\unit{\bar^{-1}}
            
            \\\midrule
        
                   \num{0.23809523809523808e4} & \num{7.518796992481203}
                \\ \num{0.15673981191222572e4} & \num{3.745318352059925}
                \\ \num{0.12484394506866417e4} & \num{2.5}
                \\ \num{0.10810810810810811e4} & \num{1.8761726078799248 }

                \\\bottomrule
        \end{tabular}
        \vspace{2ex}
    \end{center}

    \begin{center}
        \tikzset{external/remake next=true}
        % \pgfplotsset{height=7cm, width= .6\textwidth}
        \begin{tikzpicture}
        \begin{axis}
            [
                % xmajorgrids = true,
                % legend pos  = north west
                % domain=0:4,
                % xlabel={},
                % ylabel={},
            ]
            % Legends
            % \addlegendimage{empty legend}
            % \addlegendentry[Red]{\( x \)}

            % Plot from equation
            \addplot[
                only marks,mark=*,
                % smooth,
                % thick,
                Graph,
                % domain  = -2:2,
                samples = \mysampledensityDouble,
            ] coordinates {
                (7.518796992481203,  0.23809523809523808e4)
                (3.745318352059925,  0.15673981191222572e4)
                (2.5,                0.12484394506866417e4)
                (1.8761726078799248, 0.10810810810810811e4)
            };
            
        \end{axis}
        \end{tikzpicture}
    \end{center}

\end{questionBox}

\end{document}
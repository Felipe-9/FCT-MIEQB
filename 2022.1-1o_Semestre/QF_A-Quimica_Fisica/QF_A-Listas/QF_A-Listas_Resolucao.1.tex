% !TEX root = ./QF_A-Listas_Resolução.1.tex
\providecommand\mainfilename{"./QF_A-Listas_Resolução.tex"}
\providecommand \subfilename{}
\renewcommand   \subfilename{"./QF_A-Listas_Resolução.1.tex"}
\documentclass[\mainfilename]{subfiles}

% \tikzset{external/force remake=true} % - remake all

\begin{document}

\graphicspath{{\subfix{./.build/figures/QF_A-Listas_Resolução.1/}}}

\mymakesubfile{1}
[QF A]
{Exercicios Cinética} % Subfile Title
{Cinética} % Part Title

\setcounter{question}{1}

\begin{questionBox}1{ % Q2
    The cis-trans isomerization of 1,2-dimethylcyclopropane at 450\,\unit{\celsius} is a reversible 1st order reaction. The composition of the mixture in percent molar is given in the following table:
} % Q2
    \begin{center}
        \setlength\tabcolsep{3mm}        % width
        \renewcommand\arraystretch{1.25} % height
        \begin{tabular}{c *{8}{r}}
            
            \\\toprule
            
                t(\unit{\second})
                & 0
                & 45
                & 90
                & 225
                & 270
                & 360
                & 495
                & \(\infty\)

            \\

                \% trans
                & 0
                & 10.8
                & 18.9
                & 37.7
                & 41.8
                & 49.3
                & 56.5
                & 70.0

            \\\bottomrule
            
        \end{tabular}
    \end{center}

    Calculate the equilibrium constant and the kinetic constants of the forward and reverse reaction.
    
    \begin{flalign*}
        &
            y
            = -\ln\frac{
                \ch{[A]}-\ch{[A]}_{eq}
            }{
                \ch{[B]}_{eq}
            }
            = -\ln\frac{
                \ch{[A]}-0.3
            }{
                0.7
            }
            = -(k_{+1}+k_{-1})\,t
            \cong -(3.3\E{-3})\,t
            \quad R^2 \cong 0.999
        &
    \end{flalign*}
    
\end{questionBox}

\setcounter{question}{3}

\begin{questionBox}1{ % Q4
} % Q4
\end{questionBox}

\setcounter{question}{6}

\begin{questionBox}1{ % Q7
    The \ch{2 A -> 3 B} gas phase reaction composition was followed by measuring the total pressure as a function of time, giving the following results:
} % Q7
    \begin{center}
        \begin{tabular}{l *{6}{r}}
            
            \\\toprule
            
                t/\unit{\minute}
                & 0
                & 4
                & 8
                & 12
                & 16
                & 20

            \\

                p/\unit{\bar}
                & 1.250
                & 1.298
                & 1.342
                & 1.381
                & 1.416
                & 1.448

            \\\bottomrule
            
        \end{tabular}
    \end{center}

    \begin{center}
        \begin{tabular}{c c c c}
            
            \\\toprule
            
                t
                & 2\,A
                & \rightarrow
                & 3\,B
            
            \\\midrule
            
                0
                & 1 && 0.25
                \\
                t
                & \(1-2\,x\) && \(0.25 + 3\,x\)
            
            \\\bottomrule
            
        \end{tabular}
    \end{center}

    \begin{flalign*}
        &
            P_{A}
            = 1-2\,x
            = P_{t}-P_{B}
            = P_{t} - 0.25 + 3\,x
            \implies
            P_{A}
            = 1-2\,\left(
                P_t-1.25
            \right)
            % = &\\&
            = 3.5-2\,P_t
        &
    \end{flalign*}

    \begin{center}
        \setlength\tabcolsep{3mm}        % width
        \renewcommand\arraystretch{1.25} % height
        \begin{tabular}{r r r}
            
            \\\toprule
            
                \multicolumn{1}{c}{t}
                & \multicolumn{1}{c}{\(P_t\)}
                & \multicolumn{1}{c}{\(P_A\)}
            
            \\\midrule
            
                    0 & 1.250 & 1.000
                \\  4 & 1.298 & 0.904
                \\  8 & 1.342 & 0.816
                \\ 12 & 1.381 & 0.738
                \\ 16 & 1.416 & 0.668
                \\ 20 & 1.448 & 0.604
            
            \\\bottomrule
            
        \end{tabular}

        \vspace{1ex}
        \includegraphics[width=.8\textwidth]{Screenshot 2023-03-17 at 15.30.24.png}
    \end{center}

    \dots

\end{questionBox}

\setcounter{question}{8}

\begin{questionBox}1{ % Q9
    The reaction
} % Q9
    \begin{center}\bfseries
        \ch{2 NO\gas{} + O2\gas{} -> 2 NO2\gas{}}
    \end{center}

    is 1st order with respect to \ch{N2O2}. Derive an expression for the variation of the \ch{NO} partial pressure as a function of time.

    \begin{enumerate}
        \item \ch{2 NO ->[k{ 1}] N2O2}
        \item \ch{N2O2 ->[k{-1}] 2 NO}
        \item \ch{N2O2 + O2 ->[k_{2}] 2 NO2} (lento)
    \end{enumerate}
\end{questionBox}

\end{document}
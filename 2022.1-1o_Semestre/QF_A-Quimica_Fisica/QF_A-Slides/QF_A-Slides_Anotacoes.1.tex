% !TEX root = ./QF_A-Slides_Anotações.1.tex
\providecommand\mainfilename{"./QF_A-Slides_Anotações.tex"}
\providecommand \subfilename{}
\renewcommand   \subfilename{"./QF_A-Slides_Anotações.1.tex"}
\documentclass[\mainfilename]{subfiles}

% \tikzset{external/force remake=true} % - remake all

\begin{document}

\graphicspath{{\subfix{./.build/figures/QF_A-Slides_Anota-ções.1}}}

\mymakesubfile{1}
[QF A]
{Cinética Química} % Subfile Title
{Cinética Química} % Part Title

\part*{The rates of reactions}

\begin{sectionBox}1{The definition of Rate} % S1
    
    \begin{BM}
        \xi
        = \frac{n_j-n_{j,0}}{v_j}
        = V^{-1}\odv{\xi}{t}
    \end{BM}
    
    \vspace{-3ex}

    \begin{exampleBox}*3{} % E1
        
        \begin{center}\Large
            \ch{A + 2 B -> 3 C + D}
        \end{center}

        \begin{BM}
            \odv{\ch{[D]}}{t}
            = 3^{-1}\odv{\ch{[C]}}{t}
            = -1^{-1}\odv{\ch{[A]}}{t}
            = -2^{-1}\odv{\ch{[B]}}{t}
        \end{BM}
        
    \end{exampleBox}
    
\end{sectionBox}

\begin{exampleBox}1{ % E1
    Predict how the total pressure varies during the gas-phase decomposition
} % E1
    
    \begin{center}\Large
        \ch{ 2 N2O5\gas{} -> 4 NO2\gas{} + O2\gas{}}
    \end{center}

    \begin{center}
        \begin{tabular}{c c c c c}
            
            \\\toprule
            
                \multicolumn{1}{c}{t}
                & \multicolumn{1}{c @{\rightarrow}}
                {\ch{2 N2O5}}
                & \multicolumn{1}{c}{\ch{4 NO2\gas{}}}
                & \multicolumn{1}{@{\,+\,} c}
                {\ch{O2\gas{}}}
                & \multicolumn{1}{c}{Total}
            
            \\\midrule
            
                0 & n & 0 & 0 & 0
                \\
                \chemalpha
                & \(n(1-\chemalpha)\)
                & \(2\,n\,\chemalpha\)
                & \(n\,\chemalpha/2\)
                & \(n(1+\chemalpha\,3/2)\)
                \\
                1 & 0 & 2\,n & n/2 & n\,5/2
            
            \\\bottomrule
            
        \end{tabular}

        \begin{flalign*}
            &
                p_1
                = \frac{n_1\,R\,T}{V}
                = n_0(1+3/2)\frac{R\,T}{V}
                = p_0\,5/2
            &
        \end{flalign*}
    \end{center}
    
\end{exampleBox}

\begin{sectionBox}1{Rate laws and rate constants} % S2
    
    
\end{sectionBox}

\begin{sectionBox}1{Determination of Rate Law (Isolation Method)} % S3
    
    Have a reagenct in large exceess so its concentration acts as a constant characterizing the reaction as psudo first-order rate law.

    \begin{BM}
        \lim_{\ch{[B]}\gg\ch{[A]}}{v} 
        = k\,\ch{[A]}\ch{[B]}_0
        = k'\,\ch{[A]}
    \end{BM}
    
\end{sectionBox}

\part*{Integrated Rate Laws}

\begin{sectionBox}1{First-order reactions} % S4
    
    \begin{BM}
        \odv{\ch{[A]}}{t} = -k\,\ch{[A]};
        \qquad
        \ln{\frac{\ch{[A]}}{\ch{[A]}_0}}
        = -k\,t;
        \qquad
        \ch{[A]}=\ch{[A]}_0\,\exp(-k\,t)
    \end{BM}
    
\end{sectionBox}

\begin{exampleBox}1{ % E2
    The variation in the partial pressure of azomethane with time was followed at 600\,\unit{\kelvin}, with the results given bellow. Confirm that the decomposition
} % E2
    
    \begin{center}\Large
        \ch{CH3N2CH3\gas{} -> CH3CH3\gas{} + N2\gas{}}
    \end{center}

    is first-order in azomethane, and find the rate constant at 600\,\unit{\kelvin}

    \begin{center}
        \begin{tabular}{c *{5}{c}}
            
            \\\toprule
            
                \(t/\unit{\second}\)
                & 0 & 1000 & 2000 & 3000 & 4000
                \\
                \(p/\unit{\pascal}\)
                & 10.9 & 7.63 & 5.32 & 3.71 & 2.59
            
            \\\bottomrule
            
        \end{tabular}
    \end{center}
    
    \begin{answerBox}1{} % RS 

        \begin{flalign*}
            &
                \ln{\frac{\ch{[A]}}{\ch{[A]}_0}}
                = \ln{\frac{p_{\ch{A}}/(R\,T)}{p_{\ch{A},0}/(R\,T)}}
                = \ln{\frac{p_{\ch{A}}}{p_{\ch{A},0}}}
            &
        \end{flalign*}

        \begin{center}
            \setlength\tabcolsep{3mm}        % width
            \renewcommand\arraystretch{1.25} % height
            \sisetup{
                % scientific / engineering / input / fixed
                exponent-mode           = fixed,
                exponent-to-prefix      = false,          % 1000 g -> 1 kg
                % exponent-product        = *,             % x * 10^y
                % fixed-exponent          = 0,
                round-mode              = places,        % figures/places/unsertanty/none
                round-precision         = 4,
                % round-minimum           = 0.01, % <x => 0
                % output-exponent-marker  = {\,\mathrm{E}},
            }
            \begin{tabular}{c *{5}{c}}
                
                \\\toprule
                
                    \(t/\unit{\second}\)
                    & 0 
                    & 1000 
                    & 2000 
                    & 3000 
                    & 4000
                    \\
                    \(\ln{(p/p_0)}\)
                    & 0
                    & \num{-0.35667494393873245}
                    & \num{-0.717289485881545}
                    & \num{-1.0777309126157544}
                    & \num{-1.4371049135236518}
                
                    % 0, 0 
                    % -0.35667494393873245, 1000 
                    % -0.717289485881545, 2000 
                    % -1.0777309126157544, 3000 
                    % -1.4371049135236518, 4000
    
                \\\bottomrule
                
            \end{tabular}

            \begin{figure}\centering
                \includegraphics[width=\textwidth]{e2}
                \caption{\(\ln{(p/p_0)}=-3.595*10^{-4}\,t+0.0013\)}
            \end{figure}
        \end{center}

        \begin{flalign*}
            &
                k = 3.595\e-4
            &
        \end{flalign*}
    \end{answerBox}

\end{exampleBox}

\begin{sectionBox}1{Halflife and time constants} % S5
    
    \begin{BM}
        t_{1/2} = \ln{2}/k
    \end{BM}

    \vspace{-3ex}

    \begin{flalign*}
        &
            \ln{(\ch{[A]}/\ch{[A]}_0)}
            =-k\,t
            % \implies &\\&
            \implies
            \ln{(2^{-1}\ch{[A]}_0/\ch{[A]}_0)}
            = \ln{2^{-1}}
            =-k\,t_{1/2}
            % \implies &\\&
            \implies
            t_{1/2}=\ln{2}/k
        &
    \end{flalign*}

    \begin{sectionBox}2{Time constant} % S4.1
        
        \begin{BM}
            \tau = t_{1/e} = k^{-1}
        \end{BM}
        
        \vspace{-3ex}

        \begin{flalign*}
            &
                -k\,\tau
                = -k\,t_{e^{-1}}
                = \ln{\frac{e^{-1}\ch{[A]}_0}{\ch{[A]}_0}}
                = \ln{e^{-1}}
                = -1
                \implies
                \tau = k^{-1}
            &
        \end{flalign*}
        
    \end{sectionBox}
    
\end{sectionBox}

\begin{sectionBox}1{Second-order reactions} % S6
    
    \begin{center}\Large
        \ch{A^2 -> P}
    \end{center}
    
    \vspace{-3ex}

    \begin{BM}
        \odv{\ch{[A]}}{t}
        =-k\,\ch{[A]}^{2}
        ; \qquad
        k\,t
        = \ch{[A]}^{-1}
        - \ch{[A]}_{0}^{-1}
        ; \qquad
        \ch{[A]}
        = \frac{\ch{[A]}_0}{1+k\,t\ch{[A]}_0}
        \\
        t_{1/2} = (k\,\ch{[A]}_0)^{-1}
    \end{BM}

    \begin{sectionBox}2{nth-order reaction half life} % S
        
        \begin{center}\Large
            \ch{A^n -> P}
        \end{center}
        
        \vspace{-3ex}

        \begin{BM}
            t_{1/2} = (k\,\ch{[A]}_0^{n-1})^{-1}
        \end{BM}
        
    \end{sectionBox}
    
\end{sectionBox}

\begin{sectionBox}1{Other kind of second-order reactions} % S7
    
    \begin{center}
        {\Large
            \ch{A + B -> P}
        }

        \begin{tabular}{c c c c}
            
            \\\toprule
            
                t
                & \multicolumn{1}{c@{\,+\,}}{A}
                & \multicolumn{1}{c@{\rightarrow}}{B}
                & \multicolumn{1}{c}{P}
            
            \\\midrule
            
                0 & \(\ch{[A]_0}\) & \(\ch{[B_0]}\) & \(\ch{[P]_0}\)
                \\
                t & \(\ch{[A]_0}-x\) & \(\ch{[B_0]}-x\) & \(\ch{[P]_0}+x\)
            
            \\\bottomrule
            
        \end{tabular}
    \end{center}

    \begin{flalign*}
        &
            \odv{\ch{[A]}}{t}
            = - \odv{x}{t}
            = &\\&
            = - k\,\ch{[A][B]}
            = - k\,(\ch{[A]}_0-x)(\ch{[B]}_0-x)
            \implies &\\&
            \implies
            \int_{0}^{x}{
                \frac{\odif{x}}{
                    (\ch{[A]}_0-x)(\ch{[B]}_0-x)
                }
            }
            = &\\&
            = \int_{0}^{x}{
                \frac{\odif{x}}{
                    \ch{[B]}_0-\ch{[A]}_0
                }
                \left(
                    (\ch{[A]}_0-x)^{-1}
                    -(\ch{[B]}_0-x)^{-1}
                \right)
            }
            = &\\&
            = \frac{1}{\ch{[B]}_0-\ch{[A]}_0}
            \left(
                \int_{0}^{x}{
                    \frac{\odif{x}}{\ch{[A]}_0-x}
                }
                -\int_{0}^{x}{
                    \frac{\odif{x}}{\ch{[B]}_0-x}
                }
            \right)
            = &\\&
            = \frac{1}{\ch{[B]}_0-\ch{[A]}_0}
            \left(
                -\int_{0}^{x}{
                    \frac{\odif{(\ch{[A]}_0-x)}}{\ch{[A]}_0-x}
                }
                +\int_{0}^{x}{
                    \frac{\odif{(\ch{[B]}_0-x)}}{\ch{[B]}_0-x}
                }
            \right)
            = &\\&
            = \frac{1}{\ch{[B]}_0-\ch{[A]}_0}
            \left(
                -\ln{\frac{\ch{[A]}_0-x}{\ch{[A]}_0}}
                +\ln{\frac{\ch{[B]}_0-x}{\ch{[B]}_0}}
            \right)
            = &\\&
            = \frac{1}{\ch{[B]}_0-\ch{[A]}_0}
            \left(
                -\ln{\frac{\ch{[A]}}{\ch{[A]}_0}}
                +\ln{\frac{\ch{[B]}}{\ch{[B]}_0}}
            \right)
            = &\\&
            = \frac{1}{\ch{[B]}_0-\ch{[A]}_0}
            \left(
                \ln\frac{
                    \ch{[B]}/\ch{[B]}_0
                }{
                    \ch{[A]}/\ch{[A]}_0
                }
            \right)
            = &\\&
            = k\int_{0}^{t}{\odif{t}}
            = k\,t
            \implies &\\&
            \implies
            \ln\frac{
                \ch{[B]}/\ch{[B]}_0
            }{
                \ch{[A]}/\ch{[A]}_0
            }
            = \frac{k\,t}{\ch{[B]}_0-\ch{[A]}_0}
        &
    \end{flalign*}

    
\end{sectionBox}

\part*{Mecanismo de uma reação}
\begin{sectionBox}*1{} % S
    
    Sucessão de paços elementares

    \begin{description}
       \item[Ordem:] Molecularidade num passo elementar
       \item[Passo mais lento] Controla a velocidade
       \item[Estado estacionário] de produtos elementares 
    \end{description}
    
\end{sectionBox}

\part*{Reactions approaching equilibrium}

\begin{sectionBox}1{First-order reactions close to equilibrium} % S8
    
    \begin{center}\Large
        \ch{A <>[k+1][k-1] B}

        \vspace{-3ex}

        \begin{BM} 
            v_{+1} = k_{+1}\,\ch{[A]};
            \qquad
            v_{-1} = k_{-1}\,\ch{[B]}
            \\
            \odv{\ch{[A]}}{t}
            = v_{+1}-v_{-1}
            = k_{+1}\,\ch{[A]}-k_{-1}\,\ch{[B]}
        \end{BM}
    \end{center}
    
    \begin{sectionBox}2{ % S8.1
    } % S8.1
        
        \begin{BM}
            \ch{[A]}+\ch{[B]}=\ch{[A]}_0
            : \ch{[A]}_0\neq0
            \land \ch{[B]}_0=0
            \\
            \therefore
            \begin{cases}
                \ch{[A]}
                =\cfrac{
                    k_{-1}+k_{+1}\exp{(-(k_{+1}+k_{-1})t)}
                }{
                    k_{+1}+k_{-1}
                }\\
                \ch{[A]}_{eq}
                = \cfrac{k_{-1}\,\ch{[A]}_0}{k_{+1}+k_{-1}}
                \\
                \ch{[B]}_{eq}
                = \cfrac{k_{+1}\,\ch{[A]}_0}{k_{+1}+k_{-1}}
            \end{cases}
        \end{BM}

        \begin{flalign*}
            &
                \odv{\ch{[A]}}{t}
                = k_{+1}\,\ch{[A]}-k_{-1}\,\ch{[B]}
                = k_{+1}\,\ch{[A]}-k_{-1}\,(\ch{[A]}_0-\ch{[A]})
                = &\\&
                =\ch{[A]}_0\,k_{-1}
                -\ch{[A]}(k_{+1}+k_{-1})
            &
        \end{flalign*}
        
    \end{sectionBox}

\end{sectionBox}

\begin{exampleBox}1{ % E2
} % E2
    
    The equilibrium constant for the autoprotolysus of water, \ch{H2O <> H^+\aq{} + OH^{-}\aq{}}, is \(k_w = a_{\ch{H^+}}\,a_{\ch{OH^-}}=1.008*10^{-14}\) at 298\,\unit{\kelvin}. After a temperature-jump, the reaction returns to equilibrium with a relaxation time of 37\unit{\micro\second} at 298\,\unit{\kelvin} and \(pH\approx 7\). Given that the forward reaction is first-order and the reverse is second-order overall, calculate the rate constants for the forward and reerse reactions

    \begin{center}
        {\Large
            \ch{H2O <> H^+\aq{} + OH^{-}\aq{}}
        }
        
        \begin{tabular}{*{4}{c}}
            
            \\\toprule
            
                \multicolumn{1}{c}{t}
                & \multicolumn{1}{c@{\hspace{1em}\rightleftharpoons}}
                {\ch{H2O}}
                & \multicolumn{1}{c@{\,+\,}}
                {\ch{H^+\aq{}}}
                & \multicolumn{1}{c}{\ch{OH^-\aq{}}}
            
            \\\midrule
            
                0 & \(\ch{[H2O]}_0\) & 0 & 0
                \\
                t & \(\ch{[H2O]}_0-x\) & x & x
            
            \\\bottomrule
            
        \end{tabular}
    \end{center}

    \begin{exampleBox}2{ % E.1
        Forward
    } % E.1
        
        \begin{flalign*}
            &
                \odv{\ch{[H2O]}}{t}
                = k_{-1}\ch{[H^+][OH^{-}]}
                - k_{+1}\ch{[H2O]}
                \implies &\\&
                \implies
                \odv{x}{t}
                = \left(
                    \begin{aligned}
                        -&
                        x\,\left(
                            k_{+1}+k_{-1}(\ch{[H^+]}_{eq}+\ch{[OH^-]}_{eq})
                        \right)
                        & +\\- &
                            k_{+1}\,\ch{[H2O]}_{eq}
                        & +\\+ &
                            k_{-1}\,\ch{[H^+][OH^-]}
                        & +\\+ &
                            k_{-1}\,x^2
                        &
                    \end{aligned}
                \right)
                = &\\&
                = 
            &
        \end{flalign*}
        
    \end{exampleBox}
    
\end{exampleBox}

\end{document}
% !TEX root = ./QF_A-Aulas.1.tex
\providecommand\mainfilename{"./QF_A-Aulas.tex"}
\providecommand \subfilename{}
\renewcommand   \subfilename{"./QF_A-Aulas.1.tex"}
\documentclass[\mainfilename]{subfiles}

% \tikzset{external/force remake=true} % - remake all

\begin{document}

% \graphicspath{{\subfix{./.build/figures/}}}

\mymakesubfile{1}
[QF A]
{Anotações} % Subfile Title
{Anotações} % Part Title

\begin{questionBox}1{ % Q1
    The enzyme carbonic anhydrase catalyses the hydration of \ch{CO2} in red blood cells to give bicarbonate (hydrogencarbonate) ion:
} % Q1
    % \begin{center}
    %     \ch{CO2\gas{} + H2O\liq{} -> HCO3^-\aq{} + H^+\aq{}}
    % \end{center}

    The following data were obtained for the reaction at \(pH = 7.1, 273.5\,\unit{\kelvin}\), and an enzyme concentration of 2.3\,\unit{\nano\mole\,\deci\metre^{-3}}:

    % \begin{table}[H]\centering
    %     \begin{tabular}{l r r r r}
            
    %         % \\\toprule
            
    %             % \ch{[CO2]} / \unit{\mili\mole\,\deci\metre^{-3}}
    %             & 1.25 & 2.50 & 5.00 & 20.00

    %         % \\
                
    %             % rate / \unit{\mili\mole\,\deci\metre^{-3}\,\second^{-1}}
    %             % & \(2.78\E-2\)
    %             % & \(5.00\E-2\)
    %             % & \(8.33\E-2\)
    %             % & \(1.67\E-1\)
            
    %         % \\\bottomrule
            
    %     \end{tabular}
    % \end{table}

\end{questionBox}

\begin{sectionBox}1{???} % S
    
    \begin{flalign*}
        &
            \int_{\ch{[A]}_0}^{\ch{[A]}}{
                \frac{
                    \odif{\ch{[A]}}
                }{
                    \ch{[A]}-\ch{[A]}_{eq}
                }
            }
            = \int_{0}^{t}{
                1
            }
        &
    \end{flalign*}
    
\end{sectionBox}



\end{document}
% !TEX root = ./QF_A-Aulas_Exercicios.1.tex
\providecommand\mainfilename{"./QF_A-Aulas_Exercicios.tex"}
\providecommand \subfilename{}
\renewcommand   \subfilename{"./QF_A-Aulas_Exercicios.1.tex"}
\documentclass[\mainfilename]{subfiles}

% \tikzset{external/force remake=true} % - remake all

\begin{document}

% \graphicspath{{\subfix{./.build/figures/QF_A-Aulas_Exercicios.1}}}
% \tikzsetexternalprefix{./.build/figures/QF_A-Aulas_Exercicios.1/graphics/}

\mymakesubfile{1}
[QF A]
{Exercicios Cinética} % Subfile Title
{Exercicios Cinética} % Part Title


%   ,ad8888ba,        88
%  d8"'    `"8b     ,d88
% d8'        `8b  888888
% 88          88      88
% 88          88      88
% Y8,    "88,,8P      88
%  Y8a.    Y88P       88
%   `"Y8888Y"Y8a      88

\begin{questionBox}1{ % Q1
    Para a reação obtiveram os seguintes dados cinéticos:
} % Q1
    \begin{center}\large\bfseries
    \ch{2 ICl\gas{} + H2\gas{} -> I2\gas{} + 2 HCl\gas{}}
    \end{center}

    \begin{center}
        \vspace{1ex}
        \setlength\tabcolsep{3mm}        % width
        % \renewcommand\arraystretch{1.25} % height
        \begin{tabular}{*{3}{C} R}
            \toprule
            
                \multicolumn{1}{c}{Experiencia}
                & [\ch{ICl\gas{}}]_0/\unit{\milli\mole.\litre^{-1}}
                & [\ch{H2\gas{}}]_0/\unit{\milli\mole.\litre^{-1}}
                & v_0/\unit{\mole.\litre^{-1}.\second^{-1}}

            \\\midrule
            
               1 & 1.4 & 1.5 &  3.7\E{-7}
            \\ 2 & 3.0 & 1.5 &  3.7\E{-7}
            \\ 3 & 3.0 & 4.5 & 33.2\E{-7}
            \\ 4 & 4.7 & 2.7 & \text{?}
            
            \\\bottomrule
        \end{tabular}
        \vspace{2ex}
    \end{center}

    \begin{questionBox}2{ % Q1.1
        Determine a lei de velocidade da reação.
    } % Q1.1
        \answer{}
        \begin{flalign*}
            &
                v=k\,\ch{[ICl]}^{\alpha}\,\ch{[H2]}^{\beta}
                ; &\\[3ex]&
                \frac{v_1}{v_2}
                = \frac
                {3.7\E-4}
                {3.7\E-4}
                = 1
                = &\\&
                = \frac
                {k\,\ch{[ICl]}_{1}^{\alpha}\,\ch{[H2]}_{1}^{\beta}}
                {k\,\ch{[ICl]}_{2}^{\alpha}\,\ch{[H2]}_{2}^{\beta}}
                = \frac
                {1.4^{\alpha}*1.5^{\beta}}
                {3.0^{\alpha}*1.5^{\beta}}
                =
                \left(
                    \frac{1.4}{3.0}
                \right)^{\alpha}
                \implies &\\&
                \implies
                \alpha
                = \log_{(1.4/3.0)}{1}
                = 0
                ; &\\[3ex]&
                \frac{v_2}{v_3}
                = \frac{3.7\E-4}{33.2\E-4}
                = \frac{3.7}{33.2}
                = &\\&
                = \frac
                {k\,\ch{[ICl]}_{2}^{\alpha}\,\ch{[H2]}_{2}^{\beta}}
                {k\,\ch{[ICl]}_{3}^{\alpha}\,\ch{[H2]}_{3}^{\beta}}
                = \left(
                    \frac{1.5}{4.5}
                \right)^{\beta}
                = \left(
                    \frac{1}{3}
                \right)^{\beta}
                \implies &\\&
                \implies
                \beta 
                = \log_{(1/3)}{\frac{3.7}{33.2}}
                \cong \num{1.997262436352681}
            &
        \end{flalign*}
        \begin{BM}
            \therefore
            v\cong k\,\ch{[H2]}^{2}
        \end{BM}
    \end{questionBox}

    \begin{questionBox}2{ % Q1.2
        Calcule a constante de velocidade da reação.
    } % Q1.2
        \answer{}
        \begin{flalign*}
            &
                \bar{k} 
                = 3^{-1}\sum_{i=1}^{3}{k_i}
            &
        \end{flalign*}
        \begin{center}
            \vspace{1ex}
            \setlength\tabcolsep{3mm}        % width
            % \renewcommand\arraystretch{1.25} % height
            \begin{tabular}{*{4}{C}}
                \toprule
                
                    \multicolumn{1}{c}{Experiencia}
                    & \ch{[H2\gas{}]}_0/\unit{\milli\mole.\litre^{-1}}
                    & v_0/\unit{\milli\mole.\litre^{-1}.\second^{-1}}
                    & k_i/\unit{\milli\mole^{-1}.\litre.\second^{-1}}
                
                \\\midrule
                
                       1 & 1.5 &  3.7\E{-4} & \num{1.6444444444444446e-4}
                    \\ 2 & 1.5 &  3.7\E{-4} & \num{1.6444444444444446e-4}
                    \\ 3 & 4.5 & 33.2\E{-4} & \num{1.6395061728395064e-4}

                \\\bottomrule
            \end{tabular}
            % \vspace{2ex}
        \end{center}
        \begin{flalign*}
            &
                \therefore \bar{k} 
                \cong 
                \qty{1.6427983539094653e-4}{\milli\mole^{-1}.\litre.\second^{-1}}
            &
        \end{flalign*}
    \end{questionBox}

    \begin{questionBox}2{ % Q1.3
        A partir dos dados experimentais estime a velocidade inicial para a experiência 4.
    } % Q1.3
        \begin{flalign*}
            &
                v_4
                \cong k\,\ch{[H2]}_{4}^{2}
                \cong 
                \num{1.6427983539094653e-4}
                *{2.7}^{2}
                \cong
                \qty{11.976e-4}{\milli\mole.\litre^{-1}.\second^{-1}}
            &
        \end{flalign*}
    \end{questionBox}

\end{questionBox}

%   ,ad8888ba,     ad888888b,
%  d8"'    `"8b   d8"     "88
% d8'        `8b          a8P
% 88          88       aad8"
% 88          88       ""Y8,
% Y8,    "88,,8P          "8b
%  Y8a.    Y88P   Y8,     a88
%   `"Y8888Y"Y8a   "Y888888P'

\setcounter{question}{2}
\begin{questionBox}1{ % Q3
    A velocidade de uma dada reação aumenta de um fator de 1000 na presença de um catalisador a 25\,\unit{\celsius}. A energia de ativação pelo mecanismo reacional na ausencia de catalisador é 98\,\unit{\kilo\joule.\mole^{-1}}. Qual será a nova energia de ativação na presença de um catalisador, mantendo todos os outros fatores constantes?
} % Q3
    \answer{}
    \begin{flalign*}
        &
            k_{cat}
            = A\,\exp(-Ea_{cat}/R\,T)
            = 1000\,k
            = 1000\,A\,\exp(-Ea/R\,T)
            \implies &\\&
            \implies
            \frac
            {\exp(-Ea_{cat}/R\,T)}
            {\exp(-Ea/R\,T)}
            = \exp\left(
                -\frac{Ea_{cat}}{R\,T}+\frac{Ea}{R\,T}
                \right)
            = \exp\left(
                \frac{Ea-Ea_{cat}}{R\,T}
            \right)
            =1000
            \implies &\\&
            \implies
            Ea_{cat}
            = Ea-R\,T\,\ln{1000}
            \cong &\\&
            \cong 98\E3
            -\num{8.314462618}
            *(25+273.15)
            *\ln{1000}
            \cong &\\&
            \cong
            \qty{80.875971492709828026}{\kilo\joule.\mole^{-1}}
        &
    \end{flalign*}
\end{questionBox}

%   ,ad8888ba,    8888888888
%  d8"'    `"8b   88
% d8'        `8b  88  ____
% 88          88  88a8PPPP8b,
% 88          88  PP"     `8b
% Y8,    "88,,8P           d8
%  Y8a.    Y88P   Y8a     a8P
%   `"Y8888Y"Y8a   "Y88888P"

\setcounter{question}{4}
\begin{questionBox}1{ % Q5
    Para uma dada reação do tipo \ch{A -> P} registrou-se, 42\,\unit{\celsius}, a seguinte evolução:
} % Q5
    \begin{center}
        \vspace{1ex}
        \begin{tabular}{*2{C}}
            \toprule
            
                \ch{[A]}/\unit{\molar}
                & t/\unit{\hour}
            
            \\\midrule
            
                   0.500 & 0
                \\ 0.469 & 1
                \\ 0.440 & 2
                \\ 0.340 & 6
            
            \\\bottomrule
        \end{tabular}
        \vspace{2ex}
    \end{center}

    \begin{questionBox}2{ % Q5.1
        Determine a lei da velocidade da reação.
    } % Q5.1
        \answer{}
        \begin{flalign*}
            &
                \odv{\ch{[A]}}{t} = -k\,\ch{[A]}^n
                \left\{
                    \begin{aligned}
                        0: & \adif{\ch{[A]}} &= -k\,\adif{t}
                        \\
                        1: & \adif{\ln\ch{[A]}} &=-k\,\adif{t}
                        \\
                        2: & \adif{-\ch{[A]}^{-1}} &= -k\,\adif{t}
                    \end{aligned}
                \right\}
            &
        \end{flalign*}
        \begin{center}
            \vspace{1ex}
            \begin{tabular}{C | *{3}{C}}
                \toprule
                
                    \adif{t}/\unit{\hour}
                    & -\adv{\ch{[A]}}{t}/\unit{\M.\hour^{-1}}
                    & -\adv{\ln\ch{[A]}}{t}/\unit{\hour^{-1}}
                    & -\adv{-\ch{[A]}^{-1}}{t}/\unit{\M^{-1}.\hour^{-1}}
                
                \\\midrule
                
                       1 & 0.031                     & \num{0.06400532997591243} & \num{0.13219616204690832}
                    \\ 2 & 0.03                      & \num{0.06391668575494244} & \num{0.13636363636363646}
                    \\ 6 & \num{0.02666666666666666} & \num{0.06427708013533076} & \num{0.15686274509803919}
                \\\bottomrule
            \end{tabular}
            \vspace{2ex}
        \end{center}
        \begin{flalign*}
            &
                \therefore
                \bar{k}
                = 3^{-1}\sum_{i=1}^{3}{k_i}
                = \qty{0.06406636528872856}{\hour^{-1}}
            &
        \end{flalign*}
    \end{questionBox}

    \begin{questionBox}2{ % Q5.2
        Sabendo que a energia de ativação da reação é de 87\,\unit{\kilo\joule.\mole^{-1}}, calcule a velocidade da reação a 50\,\unit{\celsius} para uma concentração de \ch{A} de 0.457\,\unit{\molar}.
    } % Q5.2
        \answer{}
        \begin{flalign*}
            &
                v
                =k\,\ch{[A]}
                ; &\\[3ex]&
                k=A\,\exp(-E_a/R\,T)
                \implies &\\&
                \implies
                \frac{k_1}{k_2}
                = \frac
                {\exp(-E_a/R\,T_1)}
                {\exp(-E_a/R\,T_2)}
                =\exp\left(
                    \frac{E_a}{R}\,(T_2^{-1}-T_1^{-1})
                \right)
                \implies &\\&
                \implies
                k_1
                =k_2\,\exp\left(
                    \frac{E_a}{R}\,(T_2^{-1}-T_1^{-1})
                \right)
                \implies &\\[3ex]&
                \implies
                v
                =k_2\,\exp\left(
                    \frac{E_a}{R}\,(T_2^{-1}-T_1^{-1})
                \right)
                \,\ch{[A]}
                \cong &\\&
                \cong
                \num{0.06406636528872856}
                \,\exp\left(
                    \frac{87\E3}{\num{8.314462618}}
                    \,((42+273.15)^{-1}-(50+273.15)^{-1})
                \right)
                % 2.274966864957952
                \,0.457
                \cong &\\&
                \cong
                \qty{0.066607228192898}{\M.\hour^{-1}}
            &
        \end{flalign*}
    \end{questionBox}
\end{questionBox}

\end{document}
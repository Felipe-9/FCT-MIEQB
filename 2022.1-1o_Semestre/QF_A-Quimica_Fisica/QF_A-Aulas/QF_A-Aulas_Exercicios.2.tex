% !TEX root = ./QF_A-Aulas_Exercicios.2.tex
\providecommand\mainfilename{"./QF_A-Aulas_Exercicios.tex"}
\providecommand \subfilename{}
\renewcommand   \subfilename{"./QF_A-Aulas_Exercicios.2.tex"}
\documentclass[\mainfilename]{subfiles}

% \tikzset{external/force remake=true} % - remake all

\begin{document}

% \graphicspath{{\subfix{./.build/figures/QF_A-Aulas_Exercicios.2}}}
% \tikzsetexternalprefix{./.build/figures/QF_A-Aulas_Exercicios.2/graphics/}

\mymakesubfile{2}
[QF A]
{Exercicios} % Subfile Title
{Exercicios} % Part Title

\begin{questionBox}1{ % Q1
    Prove que a reação é de primeira ordem em relação a \ch{N2O2} sabendo que no instante inicial \(t=0\) já existem 0.25\,\unit{\bar\of{\ch{NO}}} no reator. A pressão total varia da seguinte maneira em função do tempo:
} % Q1
    \begin{center}\large\bfseries
        \ch{N2O2\gas{} -> 2 NO\gas{}}
    \end{center}
    \begin{center}
        \vspace{1ex}
        \begin{tabular}{L *{6}{C}}
            \toprule
            
                t/\unit{\min}
                & 1 & 2 & 3 & 5 & 20 & 100
                \\ 
                p_t/\unit{\bar}
                & 2.30 & 2.62 & 2.85 & 3.14 & 3.45 & 3.45

            \\\bottomrule
        \end{tabular}
        \vspace{2ex}
    \end{center}

    \answer{}
    \begin{center}
        \vspace{1ex}
        \begin{tabular}{C C C | C}
            \toprule
            
                t/\unit{\minute}
                & p_{\ch{N2O2\gas{}}}/\unit{\bar}
                & p_{\ch{2 NO2\gas{}}}/\unit{\bar}
                & p_t
            
            \\\midrule
            
                   0 & p_0 & 0.25 & p_0+0.25
                \\ t & p_0-x & 0.25+2\,x & p_0+0.25+x
                \\ \infty & 0 & 0.25+2\,p_0 & 0.25+2\,p_0
            
            \\\bottomrule
        \end{tabular}
        \vspace{2ex}
    \end{center}
    \begin{flalign*}
        &
            3.45=0.25+2\,p_0
            \implies 
            p_0
            \cong (3.45-0.25)/2
            = 1.6
            ; &\\[3ex]&
            x 
            = p_t-p_0-0.25
            = p_t-1.85
            = p_t-1.85
            ; &\\[3ex]&
            p_{\ch{N2O2\gas{}}}
            = p_0-x = 1.6-x
            ; &\\[3ex]&
            \adif{\ln\ch{[A]}}=-k\,\adif{t}
            \implies
            k=-\adv{\ln{p_{\ch{A}}}}{t}
        &
    \end{flalign*}
    \begin{center}
        \vspace{1ex}
        % \setlength\tabcolsep{2mm}        % width
        % \renewcommand\arraystretch{1.25} % height
        \begin{tabular}{R *{3}{C}}
            \toprule
            
                t/\unit{\min}
                & x/\unit{\bar}
                & \adif{\ln p_{\ch{N2O2\gas}}}
                & k/\unit{\minute^{-1}}

                \\ 0   &      & 0 
                \\ 1   & 0.45 & \num{-0.33024168687057653} & \num{0.33024168687057653}
                \\ 2   & 0.77 & \num{-0.6563332074372289}  & \num{0.32816660371861445}
                \\ 3   & 1    & \num{-0.9808292530117261}  & \num{0.326943084337242}
                \\ 5   & 1.29 & \num{-1.6411866107486806}  & \num{0.3282373221497361}
                \\ 20  & 1.6  &
                \\ 100 & 1.6  &

            \\\bottomrule
        \end{tabular}
        \vspace{2ex}
    \end{center}
    \begin{flalign*}
        &
            \therefore k \cong\num{0.3283971742690423}
        &
    \end{flalign*}

\end{questionBox}

\end{document}
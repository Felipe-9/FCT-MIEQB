% !TEX root = ./QF_A-Bib_Notes.17.1.tex
\providecommand\mainfilename{"./QF_A-Bib_Notes.tex"}
\providecommand \subfilename{}
\renewcommand   \subfilename{"./QF_A-Bib_Notes.17.1.tex"}
\documentclass[\mainfilename]{subfiles}

% \tikzset{external/force remake=true} % - remake all

\begin{document}

\graphicspath{{\subfix{./.build/figures/QF_A-Bib_Notes.17.1}}}
\tikzsetexternalprefix{./.build/figures/QF_A-Bib_Notes.17.1/}


\mymakesubfile{1}
[QF A]
{Rates of Chemical reactions} % Subfile Title
{Rates of Chemical reactions} % Part Title

\section{Monitoring the progress of a reaction}

\begin{exampleBox}1{ % E1
    Self test 17A.1
} % E1
    
    \begin{center}\large
        \ch{
            2 NOBr\gas{} -> 2 NO\gas{} + Br2\gas{}
        }
    \end{center}

    \answer{}

    \begin{center}
        \vspace{1ex}
        \begin{tabular}{C | *{2}{C} | C}
            \toprule
            
                p(\ch{NOBr\gas{}})
                & p(\ch{NO\gas{}})
                & p(\ch{Br2\gas{}})
                & p
            
            \\\midrule
            
                p_0 & 0 & 0 & p_0
            \\  p_0-2\,\adif{p} 
                & 2\,\adif{p} 
                & \adif{p}
                & p_0+\adif{p}
            \\  3\,p_0-2\,p
                & 2(p-p_0)
                & p-p_0
                
            \\\bottomrule
            \multicolumn{4}{R}{\adif{p}=p-p_0}
        \end{tabular}
        \vspace{2ex}
    \end{center}
    
\end{exampleBox}

\begin{sectionBox}1{Definition of Rate} % S1
    
    \begin{center}\Large\bfseries
        \ch{A + 2 B -> 3 C + D}
    \end{center}

    Rate for each substance
    \begin{BM}
        v
        = \odv{\ch{[D]}}{t}
        = \frac{1}{3}\odv{\ch{[C]}}{t}
        = -\odv{\ch{[A]}}{t}
        = -\frac{1}{2}\odv{\ch{[B]}}{t}
    \end{BM}

    \subsection{Extend of reaction (\chemxi{})}
    Define a universal rate of the reaction
    \begin{BM}
        \odif{n}_J=v_J\,\odif{\chemxi}
    \end{BM}

    \begin{description}
       \item[\(v_J\)] Is the stoichiometric number of the species
    \end{description}

    \subsection{Rate of reaction}
    \begin{BM}
        v
        = \frac{1}{V}\,\odv{\chemxi}{t}
        = \frac{1}{V\,v_J}\,\odv{n_J}{t}
        \underset{\text{const }V}{=}
        \frac{1}{v_J}\,\odv{\ch{[J]}}{t}
    \end{BM}

    \begin{description}
       \item[\textit{V}] Volume of the system
    \end{description}

    \subsubsection{Heterogeneous reaction}
    \begin{BM}
        v=\frac{1}{v_J}\odv{\sigma_J}{t}
        \qquad \sigma_J=\frac{n_J}{A}
    \end{BM}
    \begin{description}
       \item[\(A\)] Surface area
    \end{description}
    
\end{sectionBox}

\end{document}
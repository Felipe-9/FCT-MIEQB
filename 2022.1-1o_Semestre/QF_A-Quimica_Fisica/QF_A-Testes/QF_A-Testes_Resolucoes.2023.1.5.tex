% !TEX root = ./QF_A-Testes_Resolucoes.2023.1.5.tex
\providecommand\mainfilename{"./QF_A-Testes_Resolucoes.tex"}
\providecommand \subfilename{}
\renewcommand   \subfilename{"./QF_A-Testes_Resolucoes.2023.1.5.tex"}
\documentclass[\mainfilename]{subfiles}

% \tikzset{external/force remake=true} % - remake all

\begin{document}

\graphicspath{{\subfix{./.build/figures/QF_A-Testes_Resolucoes.2023.1.5}}}
\tikzsetexternalprefix{./.build/figures/QF_A-Testes_Resolucoes.2023.1.5/graphics/}

\mymakesubfile{5}
[QF A]
{Exame epoca especial} % Subfile Title
{Exame epoca especial} % Part Title


%   ,ad8888ba,        88
%  d8"'    `"8b     ,d88
% d8'        `8b  888888
% 88          88      88
% 88          88      88
% Y8,    "88,,8P      88
%  Y8a.    Y88P       88
%   `"Y8888Y"Y8a      88

\begin{questionBox}1{ % Q1
    A reação
} % Q1
    \begin{center}\large\bfseries
        \ch{N2O2\gas{} -> 2 NO\gas{}}
    \end{center}

    É de primeira ordem em relação a \ch{N2O2}, derive uma expressão para a variação da pressão parcial de \ch{N2O2} em função do tempo e outra para a variação da pressão parcial de \ch{NO} em função do tempo

    \answer{}

    \begin{center}
        \vspace{1ex}
        \begin{tabular}{C C}
            \toprule
            
                p_{\ch{N2O2}}
                & p_{\ch{NO2}}
            
            \\\midrule
            
                \ch{[N2O2]}_0 & 0
                \\ 
                \ch{[N2O2]}_0-x & 2\,x
            
            \\\bottomrule
        \end{tabular}
        \vspace{2ex}
    \end{center}

    \paragraph*{\ch{N2O2\gas{}}}
    \begin{flalign*}
        &
            \odv{\ch{[N2O2]}}{t} = -k\,\ch{[N2O2]}^{1}
            \implies &\\[3ex]&
            \implies
            \adif{\ln\ch{[N2O2]}}
            = \ln\frac
            {\ch{[N2O2]}}
            {\ch{[N2O2]}_0}
            = \ln\frac
            {p_{\ch{N2O2}}/R\,T}
            {p_{\ch{N2O2},0}/R\,T}
            = \ln\frac
            {p_{\ch{N2O2}}}
            {p_{\ch{N2O2},0}}
            = &\\&
            =-k\,\adif{t}
            =-k\,t
            \implies &\\[3ex]&
            \implies
            p_{\ch{N2O2}}
            = p_{\ch{N2O2},0}\,\exp(-k\,t)
        &
    \end{flalign*}

    \paragraph*{\ch{NO2}\gas{}}
    \begin{flalign*}
        &
            \ch{[NO2]}
            = p_{\ch{NO2}}/R\,T
            = &\\[2ex]&
            = 2\,x
            = 2\,\left(
                \ch{[N2O2]}_0
                -\ch{[N2O2]}
            \right)
            = 2\,\left(
                \frac{p_{\ch{N2O2},0}}{R\,T}
                -\frac{p_{\ch{N2O2}}}{R\,T}
            \right)
            &\\&
            = \frac{2}{R\,T}\,\left(
                p_{\ch{N2O2},0}
                -p_{\ch{N2O2},0}\,\exp(-k\,t)
            \right)
            = \frac{2\,p_{\ch{N2O2},0}}{R\,T}\,\left(
                1-\exp(-k\,t)
            \right)
            \implies &\\[3ex]&
            \implies
            p_{\ch{NO2}}
            = 2\,p_{\ch{N2O2},0}\left(1-\exp(-k\,t)\right)
        &
    \end{flalign*}

\end{questionBox}

\setcounter{question}{2}

%   ,ad8888ba,     ad888888b,        ad888888b,
%  d8"'    `"8b   d8"     "88       d8"     "88
% d8'        `8b          a8P               a8P
% 88          88       ,d8P"             ,d8P"
% 88          88     a8P"              a8P"
% Y8,    "88,,8P   a8P'              a8P'
%  Y8a.    Y88P   d8"          888  d8"
%   `"Y8888Y"Y8a  88888888888  888  88888888888

\begin{questionBox}2{ % Q2.1
    A decomposição de fosfina foi seguida através de medidaes de pressão, a 950\,\unit{\kelvin}
} % Q2.1
    \begin{center}
        {\large\bfseries
        \ch{4 PH3\gas{} -> P4\gas{} + 6 H2\gas{}}}

        \vspace{1ex}
        \begin{tabular}{L *{3}{C}}
            \toprule
            
                p/\unit{\bar}
                & 0.134 & 0.200 & 0.222
                \\
                t/\unit{\min}
                & 0 & 40 & 80
            
            \\\bottomrule
        \end{tabular}
        \vspace{2ex}
    \end{center}

    Calcule a ordem global da reação e a constante cinética a esta temperatura.

    \answer{}
    \begin{center}
        \setlength\tabcolsep{3mm}        % width
        % \renewcommand\arraystretch{1.25} % height
        \vspace{1ex}
        \begin{tabular}{*{4}{C} L}
            \toprule
            
                t/\unit{\min}
                & p_{\ch{PH3}}/\unit{\bar}
                & p_{\ch{P4}}/\unit{\bar}
                & p_{\ch{H2}}/\unit{\bar}
                & \multicolumn{1}{C}{p_{t}/\unit{\bar}}
            
            \\\midrule
            
                0
                & p_{\ch{PH3},0}
                & 0 & 0 & 0.134=p_{\ch{PH3},0}
                \\
                t
                & p_{\ch{PH3},0}-4\,x
                & x & 6\,x 
                & p_t = p_{\ch{PH3},0}+3\,x
            
            \\\bottomrule
        \end{tabular}
        \vspace{2ex}
    \end{center}
    \begin{flalign*}
        &
            p_{\ch{[PH3]}}
            = p_{\ch{[PH3]},0}-4\,x
            = p_{\ch{[PH3]},0}-4\,\left(
                (p_t-p_{\ch{[PH3]},0})/3
            \right)
            = p_{\ch{[PH3]},0}\,7/3-p_t\,4/3
            ; &\\[6ex]&
            \odv{{\ch{[PH3]}}}{t}
            =-k\,{\ch{[PH3]}}^n
            &\\&
            \left\{
                \begin{aligned}
                    0: & \adif{\ch{[PH3]}} &= -k\,\adif{t}
                    \\
                    1: & \adif{\ln\ch{[PH3]}} &= -k\,\adif{t}
                    \\
                    2: & \adif{-\ch{[PH3]}^{-1}} &= -k\,\adif{t}
                \end{aligned}
            \right\}
            = \left\{
                \begin{aligned}
                    0: & \frac{\adif{p_{\ch{PH3}}}}{R\,T} &= -k\,\adif{t}
                    \\
                    1: & \adif{\ln{p_{\ch{PH3}}}} &= -k\,\adif{t}
                    \\
                    2: & R\,T\,\adif{-\ch{[PH3]}^{-1}} &= -k\,\adif{t}
                \end{aligned}
            \right\}
        &
    \end{flalign*}
    \begin{center}
        \vspace{1ex}
        \setlength\tabcolsep{2mm}        % width
        % \renewcommand\arraystretch{1.25} % height
        \begin{tabular}{R | *{4}{C}}
            \toprule
            
                t/\unit{\min}
                & 0 
                & 40 
                & 80
                \\
                p_{\ch{PH3}}/\unit{\bar}
                & 0.134
                & 0.046
                & 0.0167
                \\ -\frac{\adif{p_{\ch{PH3}}}}{R\,T}/\unit{\M}
                & 0
                & \num{0.0011141018151326198}
                & \num{0.001485469086843493}
                \\ -\adif{\ln{p_{\ch{PH3}}}}
                & 0
                & \num{1.0691984034618156}
                & \num{2.084429083190872}
                \\ -R\,T\adif{-(p_{\ch{PH3}})^{-1}}/\unit{\M^{-1}}
                & 0
                & \num{1127.6591091460252}
                & \num{4149.785521657364}
            
            \\\bottomrule
        \end{tabular}
        \vspace{2ex}
    \end{center}

    % \tikzset{external/remake next=true}
    % {\Large\bfseries{Solução 1}}\par\medskip
    \begin{tikzpicture}
    \begin{axis}
    [
        xmajorgrids=true,
        ymajorgrids=true,
        minor tick num=3,
        xminorgrids=true,
        yminorgrids=true,
        legend pos={south east},
        % axis on top,
        % ylabel={}
        % xlabel={}
    ]
        
        % Legends
        \addlegendimage{empty legend}
        \addlegendentry[Graph]{\(
                -\adif{\ln{p_{\ch{PH3}}}}
                =0.0261\,t+0.009
                ;\quad 
                R =1
        \)}

        % ================= Interpolation ================ %
        \addplot[
            no marks,
            dashed,
            samples=2,
        ] expression [
            domain=0:80
        ] {
            0.0261*x+0.009
        };
        % ==================== 1 Order =================== %
        \addplot[
            mark=*,
            only marks,
        ] coordinates {
            ( 0,  0)
            (40, 1.0691984034618156)
            (80, 2.084429083190872)
        };
    
    \end{axis}
    \end{tikzpicture}

    % Fazer graficos.

    \begin{flalign*}
        &
            \therefore
            \begin{cases}
                \text{Ordem global:} 1
                \\
                \bar{k} \cong \qty{0.0261}{\min^{-1}}
            \end{cases}
        &
    \end{flalign*}
    
\end{questionBox}


%   ,ad8888ba,     ad888888b,        ad888888b,
%  d8"'    `"8b   d8"     "88       d8"     "88
% d8'        `8b          a8P               a8P
% 88          88       ,d8P"             ,d8P"
% 88          88     a8P"              a8P"
% Y8,    "88,,8P   a8P'              a8P'
%  Y8a.    Y88P   d8"          888  d8"
%   `"Y8888Y"Y8a  88888888888  888  88888888888

\begin{questionBox}2{ % Q2.2
    Os valores obtidos para a constante cinética desta reação foram os seguintes:
} % Q2.2
    \begin{center}
        \vspace{1ex}
        \begin{tabular}{*{2}{C}}
            \toprule
            
                T/\unit{\celsius}
                & k/\unit{\min^{-1}}
            
            \\\midrule
            
                   267 & 0.005
                \\ 297 & 0.020
                \\ 327 & 0.072
                \\ 357 & 0.225
            
            \\\bottomrule
        \end{tabular}
        \vspace{2ex}
    \end{center}

    Calcule o valor de \(\adif{G}^{\ddagger}\) a 350\,\unit{\celsius}. Comente o resultado obtido

    \answer{}
    \begin{flalign*}
        &
            \adif{G}^{\ddagger}
            = \adif{H}^{\ddagger}
            - T\,\adif{S}^{\ddagger}
            = \left(
                Ea-R\,T
            \right)
            - T\,\left(
                R\left(
                    \ln\frac{A}{B}-1
                \right)
            \right)
            = Ea-R\,T\,\ln\frac{A}{\left(
                \frac{k_b\,T}{R}
            \right)}
            = &\\&
            = Ea-R\,T\,\ln\frac{A\,R}{k_b\,T}
            ; &\\[3ex]&
            \ln{k} = \ln{A} - \frac{Ea}{R}\,T^{-1}
        &
    \end{flalign*}

    \begin{center}
        \vspace{1ex}
        \begin{tabular}{*{2}{C}}
            \toprule
            
                T^{-1}/\unit{\kelvin^{-1}}
                & \ln{k}/\unit{\second^{-1}}
            
            \\\midrule
            
               \num{0.0018513375914097937} & \num{-9.392661928770137}
            \\ \num{0.0017539244058581076} & \num{-8.006367567650246}
            \\ \num{0.0016662501041406316} & \num{-6.725433722188183}
            \\ \num{0.0015869237483138936} & \num{-5.585999438999817}
            
            % 0.0018513375914097937,-9.392661928770137
            % 0.0017539244058581076,-8.006367567650246
            % 0.0016662501041406316,-6.725433722188183
            % 0.0015869237483138936,-5.585999438999817
            
            \\\bottomrule
        \end{tabular}
        \vspace{2ex}
        % \begin{center}
        %     \includegraphics[width=.8\textwidth]{Screenshot 2023-07-24 at 11.57.07.png}
        % \end{center}
    \end{center}

    % \tikzset{external/remake next=true}
    % {\Large\bfseries{Solução 1}}\par\medskip
    \begin{tikzpicture}
    \begin{axis}
    [
        xmajorgrids=true,
        ymajorgrids=true,
        minor tick num=3,
        xminorgrids=true,
        yminorgrids=true,
        % legend pos={north west}
        % axis on top,
        ylabel={\(\ln{k}/\unit{\second^{-1}}\)},
        xlabel={\(T^{-1}/\unit{\kelvin^{-1}}\)},
    ]
        
        % % Legends
        \addlegendimage{empty legend}
        \addlegendentry[Graph]{\(\ln{k}=17.2904-1.4416e4*T^{-1}\)}
        % ================= Interpolation ================ %
        \addplot[
            no marks,
            dashed,
            samples=2,
        ] expression [
            domain=0.001587:0.00185
        ] {
            -1.4416e4*x+17.2904
        };
        % ==================== Points ==================== %
        \addplot[
            mark=*,
            only marks,
        ] coordinates {
            (0.0018513375914097937,-9.392661928770137)
            (0.0017539244058581076,-8.006367567650246)
            (0.0016662501041406316,-6.725433722188183)
            (0.0015869237483138936,-5.585999438999817)
        };
        
    \end{axis}
    \end{tikzpicture}

    \begin{flalign*}
        &
            \begin{cases}
                A 
                &\cong \exp(17.2904) 
                \cong \num{32294257.951819125410651}
                \\
                Ea 
                & \cong 1.4416\E4*\num{0.0831446261815324}
                \cong \num{1198.612931032971078}
            \end{cases}
            &\\[3ex]&
            \therefore
            \adif{G}^{\ddagger}
            = Ea-R\,T\,\ln\frac{A\,R}{k_b\,T}
            \cong &\\&
            \cong 
            \num{1198.612931032971078}
            -\num{0.0831446261815324}
            *(350+273.15)
            \ln\frac{
                \num{32294257.951819125410651}
                * \num{0.0831446261815324}
            }{
                \num{1.380649e-23}
                *(350+273.15)
            }
            % -44.913571781042906
            % 2327.042839180654102
            \cong &\\&
            \cong 
            \num{3525.655770213625102}
        &
    \end{flalign*}

    \(\adif{G}^{\ddagger}>0\implies\) reação não endotérmica e não expontânea

    % \begin{BM}
    %     \adif{G}^{\ddagger}
    %     \begin{cases}
    %         <0: & \text{Reação exotermica, processo expontâneo e estável}
    %         \\
    %         =0: & \text{Equilíbrio, taxa de form= a reversar}
    %         \\
    %         >0: & \text{Reação endotermica, não expontanea e instavel}
    %     \end{cases}
    % \end{BM}
    
\end{questionBox}


%   ,ad8888ba,     ad888888b,
%  d8"'    `"8b   d8"     "88
% d8'        `8b          a8P
% 88          88       aad8"
% 88          88       ""Y8,
% Y8,    "88,,8P          "8b
%  Y8a.    Y88P   Y8,     a88
%   `"Y8888Y"Y8a   "Y888888P'



\begin{questionBox}1{ % Q3
    Um exemplo da chamada dependencia anti-Arrhenius da temperatura é o que se observa na reação entre o oxido nitrico e oxigênio molecular, sendo a cinética de 3ª ordem global.
} % Q3
    \begin{center}\large\bfseries
        \ch{2 NO\gas{} + O2\gas{} -> 2 NO2\gas{}}
    \end{center}
    
    Um dos mecanismos propostos para esta reação foi o seguinte:

    % \begin{center}
        \begin{BM}[align*]
            \ch{
                2 NO ->[k1] N2O2
                \\ N2O2 ->[k-1] 2 NO
                \\ N2O2 + O2 ->[k2] 2 NO2 &\quad \text{(lento)}
            }
        \end{BM}
    % \end{center}
    Sabendo
    \begin{itemize}
        \item \(E_{1}=79.5\,\unit{\kilo\joule.\mole^{-1}}\)
        \item \(E_{-1}=205\,\unit{\kilo\joule.\mole^{-1}}\)
        \item \(E_{2}=84\,\unit{\kilo\joule.\mole^{-1}}\)
    \end{itemize}

    Calc a \textit{Ea} global segundo o mec, just
    \answer{}
    \begin{flalign*}
        &
            k_g
            = A_g\,\exp(-Ea_g/R\,T)
            ; &\\[3ex]&
            v_g
            = v_2
            = k_2\,\ch{[O2]}\,\ch{[N2O2]}
            ; &\\[3ex]&
            \odv{\ch{[N2O2]}}{t}
            = 0
            = \left\{
                \begin{aligned}
                    &
                        v_1
                    &+\\-&
                        v_{-1}
                    &+\\-&
                        v_{2}
                    &
                \end{aligned}
            \right\}
            = \left\{
                \begin{aligned}
                    &
                        k_{1}\,\ch{[NO]}^2
                    &+\\-&
                        k_{-1}\,\ch{[N2O2]}
                    &+\\-&
                        k_{2}\,\ch{[O2]}\ch{[N2O2]}
                    &
                \end{aligned}
            \right\}
            = &\\&
            = -(k_{-1}+k_2\,\ch{[O2]})\ch{[N2O2]}
            + k_{1}\,\ch{[NO]}^2
            \implies &\\&
            \implies
            \ch{[N2O2]}
            = \frac{k_1\,\ch{[NO]}^2}{k_{-1}+k_2\,\ch{[O2]}}
            % ============================================ %
            \implies &\\[3ex]&
            \implies
            v_g
            = k_2\,\ch{[O2]}\,\ch{[N2O2]}
            = k_2\,\ch{[O2]}\,\left(
                \frac{k_1\,\ch{[NO]}^2}{k_{-1}+k_2\,\ch{[O2]}}
            \right)
            = \frac{
                k_2\,k_1
            }{
                k_{-1}+k_2\,\ch{[O2]}
            }
            \,\ch{[O2]}
            \,\ch{[NO]}^2
            = &\\&
            = \frac{k_2\,k_1}{k_{-1}}
            \,\ch{[O2]}
            \,\ch{[NO]}^2
            = k_g
            \,\ch{[O2]}
            \,\ch{[NO]}^2
            % ============================================ %
            \implies &\\[3ex]&
            \implies
            A_g\,\exp(-Ea_g/R\,T)
            = k_g
            = \frac{k_2\,k_1}{k_{-1}}
            = &\\&
            = \frac{
                  A_2\,\exp(-Ea_2/R\,T)
                \,A_1\,\exp(-Ea_1/R\,T)
            }{
                A_{-1}\,\exp(-Ea_{-1}/R\,T)
            }
            = &\\&
            = \frac{A_2\,A_1}{A_{-1}}
            \exp\left(
                -(Ea_2+Ea_1-Ea_{-1})/R\,T
            \right)
            % ============================================ %
            \implies &\\[3ex]&
            \implies
            Ea_g
            = Ea_2+Ea_1-Ea_{-1}
            = 84+79.5-205
            = -41.5\,\unit{\kilo\joule.\mole^{-1}}
        &
    \end{flalign*}
\end{questionBox}


%   ,ad8888ba,            ,d8
%  d8"'    `"8b         ,d888
% d8'        `8b      ,d8" 88
% 88          88    ,d8"   88
% 88          88  ,d8"     88
% Y8,    "88,,8P  8888888888888
%  Y8a.    Y88P            88
%   `"Y8888Y"Y8a           88

\begin{questionBox}1{ % Q4
    Se a t.sup de uma sol aq sal dep da C de tensoativo (Tens) como indicado. \(\dim(\ch{Tens})=\unit{\deci\metre^3}\) e a t.sup em \unit{\newton.\metre^{-1}}, calc a A.sup ocupada por uma molec a 25\,\unit{\celsius} para C de 0.001\,\unit{\M} de tens
} % Q4
    \begin{BM}
        \gamma = 0.085-5.88\,\ch{[Tens]}
    \end{BM}
    \answer{}
    \begin{flalign*}
        &  
            \Gamma
            = -\frac{\ch{[Tens]}}{R\,T}
            \,\odv{\gamma}{\ch{[Tens]}}
            = -\frac{\ch{[Tens]}}{R\,T}
            \,\odv{(0.085-5.88\,\ch{[Tens]})}{\ch{[Tens]}}
            = -\frac{\ch{[Tens]}}{R\,T}\,5.88
            = &\\&
            = -\frac{0.001}{
                \num{8.314462618}
                *(25+273.15)
            }\,(-5.88)
            \cong
            \qty{2.371965278095801e-6}{\mole.\metre^{-2}}
            ; &\\[3ex]&
            A \cong
            \frac{\unit{\metre^2}}{\qty{2.371965278095801e-6}{\mole}}
            \,\frac{\unit{\mole}}{\qty{6.02214076e23}{molec}}
            \,\frac{10^{20}\unit{\angstrom^2}}{\unit{\metre}^{2}}
            = \qty{0.070006887643268e3}{\angstrom^2.molec^{-1}}
        &
    \end{flalign*}
\end{questionBox}


%   ,ad8888ba,    8888888888
%  d8"'    `"8b   88
% d8'        `8b  88  ____
% 88          88  88a8PPPP8b,
% 88          88  PP"     `8b
% Y8,    "88,,8P           d8
%  Y8a.    Y88P   Y8a     a8P
%   `"Y8888Y"Y8a   "Y88888P"

\begin{questionBox}1{ % Q5
    Tabela forn os val obs de vol de az (racalculado para 0\,\unit{\celsius} e 1.013\,\unit{\bar} de press) ads numa amo de carv ativ com cerca de 1.7789\,\unit{\gram} a 77\,\unit{\kelvin} para uma série de p aplicadas:
} % Q5
    \begin{center}
        \vspace{1ex}
        \begin{tabular}{C | *{5}{C}}
            \toprule
                p/\unit{\pascal}
                & 524
                & 1731
                & 3058
                & 4534
                & 7497
                \\
                V/\unit{\centi\metre^3}
                & 1.351
                & 4.161
                & 6.954
                & 9.637
                & 14.113
            
            \\\bottomrule
        \end{tabular}
        \vspace{2ex}
    \end{center}

    Sabendo que cada molécula de azoto ocupa 16\,\unit{\angstrom^2} calcule a área de superfície de carvão ativado usada para formar uma monocamad, usando a teoria de Langmuir.

    \answer{}
    \begin{flalign*}
        &
            n^{-1}
            = \left(
                \frac{p\,V}{R\,T}
            \right)^{-1}
            = \frac{R\,T}{p\,V}
            = n_{max}^{-1}
            + (k\,n_{max}\,p)^{-1}
            % \qty{0.0831446261815324e8}{10^3\centi\metre^3.\pascal.\mole^{-1}.\kelvin^{-1}}
        &
    \end{flalign*}
    \begin{center}
        \vspace{1ex}
        \setlength\tabcolsep{2.5mm}        % width
        % \renewcommand\arraystretch{1.25} % height
        \begin{tabular}{C | *{5}{C}}
            \toprule
                
                \frac{R\,T}{V\,p}/\unit{\mole^{-1}}
                & \num{904353.6051861491}
                & \num{88885.3376602994}
                & \num{30105.978199531495}
                & \num{14652.156052481876}
                & \num{6050.873280158796}
                \\
                p_a^{-1}/\unit{\pascal^{-1}}
                & \num{0.0019083969465648854}
                & \num{0.0005777007510109763}
                & \num{0.0003270111183780249}
                & \num{0.00022055580061755624}
                & \num{0.00013338668800853674}
            \\\bottomrule
        \end{tabular}
        \vspace{2ex}
    \end{center}

    \tikzset{external/remake next=true}
    % {\Large\bfseries{Solução 1}}\par\medskip
    \begin{tikzpicture}
    \begin{axis}
    [
        xmajorgrids=true,
        ymajorgrids=true,
        minor tick num=3,
        xminorgrids=true,
        yminorgrids=true,
        legend pos={south east},
        % axis on top,
        % ylabel={},
        % xlabel={},
    ]
        
        % Legends
        \addlegendimage{empty legend}
        \addlegendentry[Graph]{\(n^{-1}=5.2719e8*p^{-1}-1.2512e5\)}
        % ================= Interpolation ================ %
        \addplot[
            no marks,
            dashed,
            samples=2,
        ] expression [
            domain=1.9e-3:1.3e-4
        ] {
            5.2719e8*x-1.2512e5
        };
        % ==================== Points ==================== %
        \addplot[
            mark=*,
            only marks,
        ] coordinates {
            (0.0019083969465648854,  904353.6051861491)
            (0.0005777007510109763, 88885.3376602994)
            (0.0003270111183780249, 30105.978199531495)
            (0.00022055580061755624, 14652.156052481876)
            (0.00013338668800853674, 6050.873280158796)
        };
        
    \end{axis}
    \end{tikzpicture}
    \begin{flalign*}
        &
            \theta
            =\frac{n}{n_{max}}
            =(1+(K\,p)^{-1})^{-1}
            \implies &\\&
            \implies
            =n^{-1}
            =\frac{1}{n_{max}}
            +\frac{1}{n_{max}\,K\,p}
            &\\[3ex]&
            n_{max} 
            = (-1.2512\E5)^{-1}\,\unit{\mole}
            \,\frac{\num{6.02214076e23}\unit{molec}}{\unit{\mole}}
            \,\frac{16\,\unit{\angstrom^2}}{\unit{molec}}
        &
    \end{flalign*}
\end{questionBox}

%   ,ad8888ba,      ad8888ba,
%  d8"'    `"8b    8P'    "Y8
% d8'        `8b  d8
% 88          88  88,dd888bb,
% 88          88  88P'    `8b
% Y8,    "88,,8P  88       d8
%  Y8a.    Y88P   88a     a8P
%   `"Y8888Y"Y8a   "Y88888P"

\begin{questionBox}1{ % Q6
    A cinética de reação entre \ch{CO} e \ch{O2} catalizada por platina ou quartzo segue uma cinemática tal que a velocidade é diretamente proporcional a \(p_{\ch{O2}}\) e inversamente prop a \(p_{\ch{CO}}\). proponha um mecanismo para essa reação
} % Q6
    \answer{}
    \begin{center}\large\bfseries
        \ch{2 CO + O2 -> 2 CO2}
        \\[3ex] \ch{
            2 CO ->[k1] 2 C + O2
            \\
            2 C + O2 ->[k-1] 2 CO
            \\
            C + O2 ->[k2] CO2
        }
    \end{center}
\end{questionBox}


%   ,ad8888ba,    888888888888
%  d8"'    `"8b           ,8P'
% d8'        `8b         d8"
% 88          88       ,8P'
% 88          88      d8"
% Y8,    "88,,8P    ,8P'
%  Y8a.    Y88P    d8"
%   `"Y8888Y"Y8a  8P'

\begin{questionBox}1{ % Q7
    Considere a monécula de ciclobutadieno e as combinaçòes lineares das orbitais atómicas 2pz dos carbonos que dão origem às orbitais moleculares \chempi
} % Q7
\end{questionBox}

\begin{questionBox}2{ % Q7.1
    Escreva a equação secular na forma de matriz para a molécula.
} % Q7.1
    \answer{}
    \begin{flalign*}
        &
            % forma matriz
            \begin{bmatrix}
                   x & 1 & 0 & 1
                \\ 1 & x & 1 & 0
                \\ 0 & 1 & x & 1
                \\ 1 & 0 & 1 & x
            \end{bmatrix}
            % eq secular na forma matricial
            = \beta^{-1}\begin{bmatrix}
                   \alpha-E & \beta & 0 & \beta
                \\ \beta & \alpha-E & \beta & 0
                \\ 0 & \beta & \alpha-E & \beta
                \\ \beta & 0 & \beta & \alpha-E
            \end{bmatrix}
            % alpha: integral de culomb, referente a um átomo
            % beta: integral de ressonancia, referencia a interação entre dois átomos diff
            % E: energia do nível
        &
    \end{flalign*}
\end{questionBox}

\begin{questionBox}2{ % Q7.2q
    question
} % Q7.2
    \begin{flalign*}
        &
            x_i = (\alpha-E_i)/\beta
            \implies
            E_i = \alpha-\beta\,x_i
        &
    \end{flalign*}
\end{questionBox}



\begin{questionBox}1{ % Q8
    O dióxido de enxofre \ch{SO4} é um poluente com origem em fontes antropogénicas que envolvam processos de combustão de combustíveis contendo enxofre (\ch{S}), como a produção de eletricidade ou a combustão de suporte a processos fabris, comercial e residencial.
} % Q8
    \begin{center}
        \tikzset{external/remake next=true}
        \chemfig[angle increment=30]{
            S
            (=[-1]O)
            (=[7]O)
        }
    \end{center}
\end{questionBox}

\begin{questionBox}2{ % Q8.1
    Quantos modos vibracionais normais tem a molécula de dióxido de enxofre?
} % Q8.1
    \answer{}
    \begin{flalign*}
        &
            \begin{cases}
                \text{Linear:} & 3*n-5
                \\
                \lnot\text{Linear:} & 3*n-6
            \end{cases}
            &\\&
            \implies 
            3*3-6=3
        &
    \end{flalign*}
\end{questionBox}

\begin{questionBox}2{ % Q8.2
    question
} % Q8.2
    body
\end{questionBox}

\begin{questionBox}2{ % Q8.3
    Sabendo que o estiramento simétrico do \ch{SO2} aparece 1152\,\unit{\centi\metre^{-1}} (\(3.45\E13\,\unit{\second^{-1}}\)), qual a constante de força (em \unit{\newton.\metre^{-1}}) da ligação \unit{SO}?
} % Q8.3
    \answer{}
    \begin{flalign*}
        &
            \nu
            = \frac{1}{2\,\pi}\sqrt{k/\mu}
            ; &\\&
            \mu 
            = \frac{m_1\,m_2}{m_1+m_2}
            = \frac{
                1
            }{
                m_1^{-1}
                +m_2^{-1}
            }
            = \frac{
                1
            }{
                \left(
                    32\,\unit{\gram.\mole^{-1}}
                    \,\frac{\unit{\mole}}{\qty{6.02214076e23}{molec}}
                \right)^{-1}
                +\left(
                    16\,\unit{\gram.\mole^{-1}}
                    \,\frac{\unit{\mole}}{\qty{6.02214076e23}{molec}}
                \right)^{-1}
            }
            = \frac{
                1
            }{
                \left(
                    32\,\unit{\gram.\mole^{-1}}
                    \,\frac{\unit{\mole}}{\qty{6.02214076e23}{molec}}
                \right)^{-1}
                +\left(
                    16\,\unit{\gram.\mole^{-1}}
                    \,\frac{\unit{\mole}}{\qty{6.02214076e23}{molec}}
                \right)^{-1}
            }
            \cong
            \qty{1.771241671652103e-26}{\kilo\gram}
            &\\[3ex]&
            k
            = \mu\,(\nu*2\,\pi)^2
            \cong 
            \num{1.771241671652103e-26}
            \,(
                3.45\E13
                *2\,\pi
            )^2
            % 469.891865535864364
            \cong
            \num{832.292053407469601}
        &
    \end{flalign*}
\end{questionBox}

\end{document}
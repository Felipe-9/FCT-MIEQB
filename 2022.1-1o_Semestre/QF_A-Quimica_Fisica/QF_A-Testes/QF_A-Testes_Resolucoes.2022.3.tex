% !TEX root = ./QF_A-Testes_Resolucoes.2022.3.tex
\providecommand\mainfilename{"./QF_A-Testes_Resolucoes.tex"}
\providecommand \subfilename{}
\renewcommand   \subfilename{"./QF_A-Testes_Resolucoes.2022.3.tex"}
\documentclass[\mainfilename]{subfiles}

% \tikzset{external/force remake=true} % - remake all

\begin{document}

% \graphicspath{{\subfix{./.build/figures/QF_A-Testes_Resolucoes.2022.3}}}
% \tikzsetexternalprefix{./.build/figures/QF_A-Testes_Resolucoes.2022.3/graphics/}

\mymakesubfile{3}
[QF A]
{Teste Cinética Química} % Subfile Title
{Teste Cinética Química} % Part Title

\begin{questionBox}1{ % Q1
    A seguinte reação pode ser feita sobre vários catalisadores sólidos
} % Q1
    \begin{center}\bfseries\large
        \ch{CO\gas{} + 1/2 O2\gas{} -> CO2\gas{}}
    \end{center}

    Explique detalhadamente a razão destas observações seguintes, deduzindo as expressões, definindo \(k_1, k_2,k_3,k_4\) e avançando um mecanismo para cada caso.
    \vspace{3ex}
    \answer{}
    \begin{questionBox}{} % Q
        Seguindo a reação, esperamos a equação de velocidade:
        \begin{BM}
            v=k\,\frac{
                p_{\ch{O2}}^{1/2}\,p_{\ch{CO}}
            }{
                p_{\ch{CO2}}
            }
        \end{BM}
    \end{questionBox}

    \begin{questionBox}2{ % Q1.1
        Quando a reação se processa sobre \emph{platina}, a velocidade é dada por:
    } % Q1.1
        \begin{BM}
            v=k_1\frac{p_{\ch{O2}}^{1/2}}{p_{\ch{CO}}}
        \end{BM}
        \answer{}
        \begin{itemize}
            \item Por não encontrarmos a influencia da concentração de \ch{CO2} podemos supor que a platina esteja o adsorvendo fortemente
        \end{itemize}
    \end{questionBox}

    \begin{questionBox}2{ % Q1.2
        Quando a reação se passa por \emph{níquel}, a velocidade é dada por:
    } % Q1.2
        \begin{BM}
            v=k_2
            \,\frac{
                  p_{\ch{CO}}
                \,p_{\ch{O2}}^{1/2}
            }{
                p_{\ch{CO2}}^2
            }
        \end{BM}
    \end{questionBox}

    \begin{questionBox}2{ % Q1.3
        Quando a reação se passa por \emph{ródio}, a velocidade é dada por:
    } % Q1.3
        \begin{BM}
            v=k_3
            \,\frac{
                p_{\ch{CO}}
                \,p_{\ch{O2}}
            }{
                p_{\ch{CO2}}
            }
        \end{BM}
    \end{questionBox}

    \begin{questionBox}2{ % Q1.4
        Quando a reação se passa por \emph{tungsténio}, a velocidade é dada por:
    } % Q1.4
        \begin{BM}
            v=k_4
            \,p_{\ch{O2}}
        \end{BM}
    \end{questionBox}
\end{questionBox}

\end{document}
% !TEX root = ./QF_A-Testes_Resoluções.2023.1.1.tex
\providecommand\mainfilename{"./QF_A-Testes_Resoluções.tex"}
\providecommand \subfilename{}
\renewcommand   \subfilename{"./QF_A-Testes_Resoluções.2023.1.1.tex"}
\documentclass[\mainfilename]{subfiles}

% \tikzset{external/force remake=true} % - remake all

\begin{document}

\graphicspath{{\subfix{./.build/figures/QF_A-Testes_Resoluções.2023.1.1}}}

\mymakesubfile{1}
[QF A]
{Teste 1} % Subfile Title
{Teste 1} % Part Title

\begin{questionBox}1{ % Q1
    A reação de decomposição do pentóxido de azoto, \ch{2 N2O5\gas{} -> 4 NO2\gas{} + O2\gas{}} foi seguida a 67\,\unit{\celsius}:
} % Q1
    \begin{itemize}
        \item \ch{N2O5}\,=\,\chemalpha
    \end{itemize}

    \begin{center}
        \setlength\tabcolsep{1.5mm}        % width
        \renewcommand\arraystretch{1.25} % height
        \sisetup{
            % scientific / engineering / input / fixed
            exponent-mode           = fixed,
            exponent-to-prefix      = false,          % 1000 g -> 1 kg
            % exponent-product        = *,             % x * 10^y
            % fixed-exponent          = 0,
            round-mode              = places,        % figures/places/unsertanty/none
            round-precision         = 4,
            % round-minimum           = 0.01, % <x => 0
            % output-exponent-marker  = {\,\mathrm{E}},
        }
        \begin{tabular}{c *{6}{c}}
            
            \\\toprule
            
                \multicolumn{1}{c}{t/\unit{\minute}}
                & 0 & 1 & 2 & 3 & 4 & 5
                \\
                \multicolumn{1}{c}{\ch{[\chemalpha]}/\unit{\M}}
                & 1.000 & 0.705 & 0.497 & 0.349 & 0.246 & 0.173
            
            \\\midrule
            
                \multicolumn{1}{c}{t/\unit{\second}}
                & 0
                & 60
                & 120
                & 180
                & 240
                & 300
                \\
                \multicolumn{1}{c}{\(\ln{{[\chemalpha]}/{[\chemalpha]}_0}\)}
                & 0
                & \num{-0.34955747616986843}
                & \num{-0.6991652528855083}
                & \num{-1.0526833567797098}
                & \num{-1.4024237430497744}
                & \num{-1.7544636844843582}

                % -0.34955747616986843, 60
                % -0.6991652528855083, 120
                % -1.0526833567797098, 180
                % -1.4024237430497744, 240
                % -1.7544636844843582, 300

            \\\bottomrule
            
        \end{tabular}

        \begin{tabular}{*{4}{c}}
            
            \\\toprule
            
                \multicolumn{1}{c}{t}
                & \multicolumn{1}{c@{\rightarrow}}
                {\ch{2 N2O5\gas{}}}
                & \multicolumn{1}{c@{\,+\,}}
                {\ch{4 NO2\gas{}}}
                & \multicolumn{1}{c}{\ch{O2\gas{}}}
            
            \\\midrule
            
                0 & \(x_0\) & \(0\) & \(0\)
                \\
                t & \(x_0-2\,x\) & \(4\,x\) & \(x\)
            
            \\\bottomrule
            
        \end{tabular}
    \end{center}

    \begin{questionBox}3{ % Q1.1
        Ordem da reação
    } % Q1.1
        \begin{flalign*}
            &
                \frac{
                    \ln{{[\chemalpha]}_2/{[\chemalpha]}_0}
                }{
                    \ln{{[\chemalpha]}_1/{[\chemalpha]}_0}
                }
                \cong 2
                \neq 
                \frac{
                    \ln{{[\chemalpha]}_5/{[\chemalpha]}_0}
                }{
                    \ln{{[\chemalpha]}_4/{[\chemalpha]}_0}
                }
                \cong 1.25
            &
        \end{flalign*}
        Não é linear portanto 2a Ordem
    \end{questionBox}

    \begin{questionBox}3{ % Q1.2
        Constante cinética
    } % Q1.2
        \begin{flalign*}
            &
                \ln\frac{{[\chemalpha]}}{{[\chemalpha]}_0}
                =-k\,t
                \implies
                k 
                = -\ln\frac{{[\chemalpha]}}{{[\chemalpha]}_0}/t
                \cong -(-0.3496)/60
                = \num{5.826666666666667e-3}
            &
        \end{flalign*}
    \end{questionBox}

    \begin{questionBox}3{ % Q1.3
        Tempo de meia vida
    } % Q1.3
        \begin{flalign*}
            &
                t_{1/2}
                = (k\,{[\chemalpha]}_0)^{-1}
                \cong
                ((\num{5.826666666666667e-3})*1.00)^{-1}
                \cong
                \num{171.624713958810059}
            &
        \end{flalign*}
    \end{questionBox}

\end{questionBox}

\begin{questionBox}1{ % Q2
} % Q2
    \begin{center}
        {\Large
            \ch{HI\gas{} -> 1/2 H2\gas{} + 1/2 I2\gas{}}
        }
        \begin{tabular}{c *{3}{c}}
            
            \\\toprule
            
                \multicolumn{1}{c}{T/\unit{\kelvin}}
                & 558 
                & 723 
                & 781
                \\
                \multicolumn{1}{c}{\(\ln{k_2}\)/\unit{\second^{-1}}}
                & -13.8155 
                & -4.60517 
                & -2.30259

            \\\midrule

                \multicolumn{1}{c}{T\(^{-1}\)/\unit{\kelvin}}
                & \num{0.0017921146953405018}
                & \num{0.0013831258644536654}
                & \num{0.0012804097311139564}

                % 0.0017921146953405018 ,-13.8155 
                % 0.0013831258644536654 ,-4.60517 
                % 0.0012804097311139564,-2.30259
            
            \\\bottomrule
            
        \end{tabular}
        \begin{figure}\centering
            \includegraphics[width=1\textwidth]{q2}
            \caption{\(\ln{k_2} = -2.2505\e-4/t + 26.517\)}
        \end{figure}
    \end{center}

    \begin{questionBox}2{ % Q2.1
        Energia de Ativação da reação
    } % Q2.1
        \begin{flalign*}
            &
                E_a 
                = k\,R
                = 2.2505\e-4*\num{8.314462618}
                = \num{18.711698121809e-4}
            &
        \end{flalign*}
    \end{questionBox}

    \begin{questionBox}2{ % Q2.2
        Prove que é de 2a Ordem
    } % Q2.2
        \begin{center}
            \sisetup{
                % scientific / engineering / input / fixed
                exponent-mode           = fixed,
                exponent-to-prefix      = false,          % 1000 g -> 1 kg
                % exponent-product        = *,             % x * 10^y
                % fixed-exponent          = 0,
                round-mode              = places,        % figures/places/unsertanty/none
                round-precision         = 4,
                % round-minimum           = 0.01, % <x => 0
                % output-exponent-marker  = {\,\mathrm{E}},
            }
            \begin{tabular}{c *{4}{c}}
                
                \\\toprule
                
                    \multicolumn{1}{c}{t/\unit{\hour}}
                    & 0 & 48 & 96 & 144
                    \\
                    \multicolumn{1}{c}{\(p\)/\unit{\milli\bar}}
                    & 100 & 93 & 87 & 82

                \\\midrule

                    \multicolumn{1}{c}{t/\unit{\second}}
                    & 0 & 172800 & 345600 & 518400
                    \\
                    \multicolumn{1}{c}{\(\ln{(p/p_0)}\)}
                    & 0
                    & \num{-0.07257069283483537}
                    & \num{-0.13926206733350766}
                    & \num{-0.19845093872383818}

                    %  0, 0 
                    %  -0.07257069283483537 , 172800 
                    %  -0.13926206733350766 , 345600 
                    %  -0.19845093872383818, 518400
                    
                \\\bottomrule
                \multicolumn{5}{r}{\(T=645\,\unit{\kelvin}\)}
                
            \end{tabular}
            \begin{figure}\centering
                \includegraphics[width=1\textwidth]{q2.2}
                \caption{\(\ln{(p/p_0)} = 3.8313\e-7*t - 0.0033\)}
            \end{figure}
        \end{center}
    \end{questionBox}
\end{questionBox}

\begin{questionBox}1{ % Q3
    Explique o grafico
} % Q3
    O grafico apresenta a relação energia e avanço da reação \ch{A <> B} Podemos perceber que uma quantidade de energia \(b\) é liberada ao decorrer da reação e A é necessário receber \(a\) de energia para que a reação seja efetivada, caso \ch{B -> A}, B deveria adquirir \(a+b\) de energia para efetivar.
    Quando no topo do pico \(a+b\) a reação se encontra no complexo-ativado.
\end{questionBox}

\begin{questionBox}1{ % Q4
} % Q4
    \begin{center}\Large
        \ch{
            A + A ->[K+1] A_* + A
            \\
            A_* + A ->[K-1] A + A
            \\
            A_* ->[K2] P\quad (Lento)
        }
    \end{center}

    \begin{questionBox}2{ % Q4.1
        Aplique o método do estado estacionário p mostrar que é de 1a ordem quando [A] elevado e 2a quando baixo
    } % Q4.1
        \begin{flalign*}
            &
                \odv{\ch{[A_*]}}{t}
                = k_{+1}\ch{[A]}^2
                - k_{-1}\ch{[A][A_*]}
                - k_{2}\ch{[A_*]}
                = 0
                \implies &\\&
                \implies
                \ch{[A_*]}\left(
                    k_{-1}\ch{[A]}
                    +k_{2}
                \right)
                = k_{+1}\ch{[A]}^2
                \implies &\\&
                \implies
                \ch{[A_*]}
                = \frac{
                    k_{+1}\ch{[A]}^2
                }{
                    k_{-1}\ch{[A]}
                    +k_{2}
                }
                \implies &\\&
                \implies 
                \lim_{\ch{[A]}\gg}{\ch{[A_*]}}
                = \frac{
                    k_{+1}\ch{[A]}^2
                }{
                    k_{-1}\ch{[A]}
                }
                = \ch{[A]}\frac{k_{+1}}{k_{-1}}
                &\\[3ex]&
                \therefore
                \ch{[A_*]}
                = \begin{cases}
                    \ch{[A]}\frac{k_{+1}}{k_{-1}} &\quad \ch{[A]}\gg
                    \\
                    \ch{[A]}^2\,\frac{
                        k_{+1}
                    }{
                        k_{2}
                    }
                    & \quad \ch{[A]}\ll
                \end{cases}
            &
        \end{flalign*}
    \end{questionBox}
\end{questionBox}

\end{document}
% !TEX root = ./QF_A-Testes_Resolucoes.2023.1.4.tex
\providecommand\mainfilename{"./QF_A-Testes_Resolucoes.tex"}
\providecommand \subfilename{}
\renewcommand   \subfilename{"./QF_A-Testes_Resolucoes.2023.1.4.tex"}
\documentclass[\mainfilename]{subfiles}

% \tikzset{external/force remake=true} % - remake all

\begin{document}

\graphicspath{{\subfix{./.build/figures/QF_A-Testes_Resolucoes.2023.1.4}}}
\tikzsetexternalprefix{./.build/figures/QF_A-Testes_Resolucoes.2023.1.4/}

\mymakesubfile{4}
[QF A]
{Exame Resolução} % Subfile Title
{Exame Resolução} % Part Title


%   ,ad8888ba,        88
%  d8"'    `"8b     ,d88
% d8'        `8b  888888
% 88          88      88
% 88          88      88
% Y8,    "88,,8P      88
%  Y8a.    Y88P       88
%   `"Y8888Y"Y8a      88



\begin{questionBox}1{} % Q1
    {\centering\huge\color{Emph}
        \ch{CO\gas{} + Cl2\gas{} -> COCl2\gas{}}
        \begin{BM}
            25\,\unit{\celsius}
        \end{BM}
    }

    \paragraph*{Experiencia 1} \(p(\ch{Cl2})_{1,0}=400\,\unit{\torr}\text{ e }p(\ch{CO})_{1,0}=4\,\unit{\torr}\)
    \begin{center}
        \vspace{1ex}
        \begin{tabular}{L *{4}{C}}
            \toprule
            
                t/\unit{\min}
                & 34.5 & 69.0 & 138 & \infty
            
            \\  p(\ch{COCl2})
                & 2.0 & 3.0 & 3.75 & 4.0
                
            
            \\\bottomrule
        \end{tabular}
        \vspace{2ex}
    \end{center}

    \paragraph*{Experiencia 2} \(p(\ch{Cl2})_{2,0}=1600\,\unit{\torr}\text{ e }p(\ch{CO})_{2,0}=4\,\unit{\torr}\)
    \begin{center}
        \vspace{1ex}
        \begin{tabular}{L *{4}{C}}
            \toprule
            
                t/\unit{\min}
                & 34.5 & 69.0 & 138 & \infty
            
            \\  p(\ch{COCl2})
                & 3.0 & 3.75 & - & 4.0
                
            
            \\\bottomrule
        \end{tabular}
        \vspace{2ex}
    \end{center}

    Ordem de cada reagente e a constante de velocidade

    \answer{}
    \begin{center}
        \vspace{1ex}
        \begin{tabular}{C | C C | C}
            \toprule
            
                t/\unit{\second}
                & p(\ch{CO})
                & p(\ch{Cl2})
                & p(\ch{COCl2})
            
            \\\midrule
            
                    0   & 400 & 4    & 0
                \\ 34.5 & 398 & 2    & 2
                \\ 69   & 397 & 1    & 3
                \\ 138  & 396.25 & 0.25 & 3.75

            \\\midrule
            
                    0   & 1600    & 4    & 0   
                \\ 34.5 & 1597    & 1    & 3   
                \\ 69   & 1596.25 & 0.25 & 3.75
            
            \\\bottomrule
        \end{tabular}
        \vspace{2ex}
    \end{center}
    Pela grande quantidade de \ch{CO} nos experimentos podemos ignorar sua influencia na velocidade quando analizando individualmente o experimento, assim podemos ver que o \ch{Cl2} possui tempo de meia vida para o exp1 de 34.5, caracteristica de reação de primeira ordem.
    Comparando os dois experimentos podemos notar que ao quadruplicar a quantidade de \ch{CO} o tempo de meia vida do \ch{Cl2} reduz pela metade, o que nos leva a concluir que a influencia de \ch{CO} é de meia ordem (1/2).

    \begin{BM}
        v = k_r\,p(\ch{CO})^{1/2}p(\ch{Cl2})
    \end{BM}

    \begin{flalign*}
        &
            t_{1/2}
            = 34.5
            =\frac{\ln{2}}{k'_r}
            =\frac{\ln{2}}{k_r*400}
            \implies &\\&
            \implies
            k_r
            \cong
            \frac{\ln{2}}{34.5*400}
            \cong
            \num{5.0228056562e-5}
        &
    \end{flalign*}
    % \begin{flalign*}
    %     &
    %         v
    %         = \odv{p(\ch{COCl2})}{t}
    %         = k_r\,p(\ch{CO})p(\ch{Cl2})
    %         \implies &\\[3ex]&
    %         \implies
    %         \frac{1}{p(\ch{CO})_0+p(\ch{Cl2})_0}
    %         \left(
    %             \ln\frac{{p(\ch{CO})}-\ln{p(\ch{CO})_0}}{\ln{p(\ch{CO})_0}}
    %         \right)
    %         &\\[6ex]&
    %         \adif{(p(\ch{Cl2})/p(\ch{Cl2})_0)}
    %         =(2/4-4/4)
    %         = -1/2
    %         =-k'_r\,(34.5)
    %     &
    % \end{flalign*}

\end{questionBox}

\setcounter{question}{2}

\begin{questionBox}1{ % Q3
    Calcule a area superficial oc por uma molecula de ácido valérico para \(C=0.035\,\unit{\molar}\)
} % Q3
    \begin{itemize}
        \item Solução mãe \ch{C4H9COOH} 
        \begin{itemize}
            \begin{multicols}{3}
                \item \(M=102.13\,\unit{\gram.mole^{-1}}\)
                \item \(C=0.15\,\unit{\molar}\)
                \item \(V=250\,\unit{\centi\metre^3}\)
            \end{multicols}
        \end{itemize}
        \item Temperatura 45\,\unit{\celsius}
        \item Balões
        \begin{enumerate}
            \begin{multicols}{3}
                \item 25\,\unit{\centi\metre^3}
                \item 500\,\unit{\centi\metre^3}
                \item 10\,\unit{\centi\metre^3}
            \end{multicols}
        \end{enumerate}
    \end{itemize}

    \begin{center}
        \vspace{1ex}
        \begin{tabular}{L *{3}{C}}
            \toprule
            
                & 1 & 2 & 3
            
            \\\midrule
            
                V_{\ch{C4H9COOH}}/\unit{\centi\metre^3}
                & 15 & 125 & 1.5
                \\ V_{tot}/\unit{\centi\metre^3}
                & 25 & 500 & 10
                \\ \gamma/\unit{\milli\newton.\metre^{-1}}
                & 43.1 & 60.2 & 64.9
            
            \\\bottomrule
        \end{tabular}
        \vspace{2ex}
    \end{center}

    \answer{}

    \begin{flalign*}
        &
            C_i
            = \frac{0.15\,\unit{\mole\of{Ac}}}{\unit{\deci\metre^3\of{Mãe}}}
            * \frac{V_{i,mae}\,\unit{\centi\metre^3\of{mae}}}{1}
            * \frac{1}{V_{i,balao}\,\unit{\centi\metre^3\of{balao}}}
            = 0.15\frac{V_{i,mae}}{V_{i,balao}}\unit{\molar}
        &
    \end{flalign*}

    \begin{center}
        \vspace{1ex}
        \begin{tabular}{L *{3}{C}}
            \toprule
            
                & 1 & 2 & 3
            
            \\\midrule

                \ln{C}/\unit{\molar}
                & \num{-2.407945608651872}
                & \num{-3.283414346005772}
                & \num{-3.794239969771763}
                \\ 
                \gamma/\unit{\newton.\metre^{-1}}
                & 0.0431
                & 0.0602
                & 0.0649
            
            \\\bottomrule
        \end{tabular}
        \vspace{2ex}
    \end{center}

    \begin{center}
        % \tikzset{external/remake next=true}
        % \pgfplotsset{height=7cm, width= .6\textwidth}
        \begin{tikzpicture}
        \begin{axis}
            [
                % xmajorgrids = true,
                % legend pos  = north west
                % domain=0:4,
                xlabel={\(\ln{C}/\unit{\molar}\)},
                ylabel={\(\gamma/\unit{\newton.\metre^{-1}}\)},
            ]
            % Legends
            \addlegendimage{empty legend}
            \addlegendentry[Graph]{\( -0.0161\,\ln{C}+0.005 \)}

            % Plot from equation
            \addplot[
                only marks,
                mark=*,
                % smooth,
                % Graph,
                % samples = \mysampledensityDouble,
            ] coordinates {
                (-2.407945608651872, 0.0431)
                (-3.283414346005772, 0.0602)
                (-3.794239969771763, 0.0649)
            };

            \addplot[
                Graph,dashed,opacity=0.5,
                domain={-3.8:-2.4},
            ]{-0.0161*x+0.005};
            
        \end{axis}
        \end{tikzpicture}
    \end{center}

    \begin{flalign*}
        &
            \Gamma
            =\frac{-1}{R\,T}
            \,\frac{\gamma}{\ln{C}}
            \cong\frac{-1}{
                \num{8.314462618}
                (45+273.15)
            }
            \,\frac{-0.0161\,\ln{0.035}+0.005}{\ln{0.035}}
            % -0.017591465587447
            \cong
            \qty{6.650218434e-6}{\mole.\metre^{-2}}
            \implies &\\&
            \implies
            A_{0.035}
            \cong 
              \frac{\unit{\metre^2}}{\qty{6.650218434e-6}{\mole}}
            \,\frac{\unit{\mole}}{\num{6.02214076e23}\unit{molec}}
            \cong
            \qty{0.024969692103407e-17}{\metre^2.molec^{-1}}
        &
    \end{flalign*}

\end{questionBox}

\begin{questionBox}1{ % Q4
    Ads de \ch{N2} sobre 3.8624\unit{\gram} de dioxido de titanio anatase a 77\,\unit{\kelvin}, a essa temp a p de sat do azoto é 1021\,\unit{\milli\bar}
} % Q4
    \begin{center}
        \vspace{1ex}
        \begin{tabular}{L *{5}{C}}
            \toprule
            
                p/\unit{\milli\bar}
                & 47 & 117 & 187 & 257 & 319
                \\
                n/\unit{\milli\mole}
                & 160 & 174 & 191 & 212 & 236
            
            \\\bottomrule
        \end{tabular}
        \vspace{2ex}
    \end{center}

    \begin{questionBox}2{ % Q4.1
        Area disp p ads do n2 por grama de adsorv prevista por Langmuir. Area da molec \ch{N2} 16.2\unit{\AA^2}
    } % Q4.1
        \answer{}

        \begin{center}
            \vspace{1ex}
            \begin{tabular}{L *{5}{C}}
                \toprule
                
                    p^{-1}/\unit{\bar^{-1}}
                    & \num{21.27659574468085}
                    & \num{8.547008547008547}
                    & \num{5.347593582887701}
                    & \num{3.8910505836575875}
                    & \num{3.134796238244514}
                    \\
                    n^{-1}/\unit{\mole^{-1}.\gram}
                    & \num{24.14}
                    & \num{22.19770114942529}
                    & \num{20.221989528795813}
                    & \num{18.218867924528304}
                    & \num{16.366101694915255}

                \\\bottomrule
            \end{tabular}
            \vspace{2ex}
        \end{center}

        \begin{center}
            % \tikzset{external/remake next=true}
            % \pgfplotsset{height=7cm, width= .6\textwidth}
            \begin{tikzpicture}
            \begin{axis}
                [
                    % xmajorgrids = true,
                    % legend pos  = north west
                    % domain=0:4,
                    xlabel={\(n^{-1}/\unit{\mole^{-1}.\gram}\)},
                    ylabel={\(p^{-1}/\unit{\bar^{-1}}\)},
                ]
                Legends
                \addlegendimage{empty legend}
                \addlegendentry[Graph]{\(n^{-1}=0.2589\,p^{-1}-2.5043\e-14\)}
                
                % Plot from equation
                \addplot[
                    only marks, mark=*,
                    Graph,
                    % smooth,
                    % thick,
                    % domain  = -2:2,
                    % samples = \mysampledensityDouble,
                ] coordinates { 
                    (24.14,  6.25)
                    (22.19770114942529,  5.747126436781609)
                    (20.221989528795813,  5.2356020942408374)
                    (18.218867924528304, 4.716981132075472)
                    (16.366101694915255,  4.237288135593221)
                };

                \addplot[
                    domain={16.36:24.14},
                    Graph,dashed,opacity=0.5,
                ]{
                    0.2589*x-2.5043e-14
                };
                
            \end{axis}
            \end{tikzpicture}
        \end{center}

        \begin{flalign*}
            &
                A
                \cong 16.2*10^{-20}
                \,\frac{\num{6.02214076e23}}{0.2589}
                \cong
                \num{376.8199316801854e3}
            &
        \end{flalign*}

    \end{questionBox}

\end{questionBox}

\end{document}
\documentclass[12pt]{report}

% Linguagem
\usepackage[portuguese]{babel}

% Clickable Table of contents
\usepackage{hyperref}
\hypersetup{
	hidelinks=true,
	colorlinks=true,
	linkcolor=DarkGreen!20!LightGreen!25!
}

% Table of contents
\usepackage{tocloft}
\setlength{\cftsubsecnumwidth}{3em} % Fix subsection width
% Fix space between subsection items on toc
\renewcommand\cftsubsecafterpnum{\vskip2pt}
\renewcommand\cftsecafterpnum{\vskip2pt}
\setlength{\cftbeforesecskip}{3pt}

% Multicols
\usepackage{multicol}

% Appendix
\usepackage{appendix}

% Maths
%\usepackage{amssymb} 		
%\usepackage{amsmath} 
%\usepackage[utf8]{inputenc} %useful to type directly diacritic characters

%\newcommand{\bm}[1]{{\boldmath{\large{\begin{align*} #1 \end{align*}}}}}

% Vectors
%\usepackage{esvect} 	% Vector over-arrow
%\usepackage{tikz}		% Vector diagrams
%\usetikzlibrary{calc}  % Vector calculations
%\usepackage{varwidth}  % List inside TikzPicture

%\renewcommand{\vec}{\vv} % Vecto over-arrow

% Chem
\usepackage{chemformula} 	% formulas quimicas
\usepackage{chemfig} 		% Estruturas quimicas
										
% Colors
\usepackage{xcolor}

\definecolor{DarkBlue}	{HTML}{252A36}
\definecolor{LightGreen}{HTML}{7CCC6C}
\definecolor{DarkGreen}	{HTML}{008675}

\definecolor{Red}  {hsb}{0  ,.6,1}
\definecolor{Blue} {hsb}{0.6,.6,1}
\definecolor{Green}{hsb}{0.3,.6,1}

\pagecolor{DarkBlue!110}
\color{DarkGreen!20!}

% Resolução de listas
%\renewcommand\thesection{Questão \arabic{section} }
%\renewcommand\thesubsection{\arabic{section}-\alph{subsection}) }

% Fix sections, sub_sections and sub_sub_sections
%\setcounter{secnumdepth}{3}
\renewcommand \thesection {\arabic{section}}

% Customize Chapter
\usepackage{titlesec}

\titleformat{\chapter}[hang]{\Huge\bfseries\color{DarkGreen!75!}}{\thechapter\hspace{20pt}{$|$}\hspace{20pt}}{0pt}{\Huge\bfseries}


\setcounter{chapter}{-1}

\begin{document}

\title{\bfseries\color{DarkGreen!75!} Química Inorgânica I}
\author{Felipe Pinto - 61387}
\date{$1^o$ Semestre - $2021.1$}

\maketitle
\tableofcontents

\chapter{Introdução}

\section{Programa}
\begin{enumerate}
\item Interesse da Química Inorgânica. 
\item Definições
	\begin{itemize}
 	\item 
		\hyperlink{composto de coordenacao}%
					 {Composto de Coordenação}
	\item \hyperlink{elemento central}{Elemento Central}
	\item \hyperlink{ligando}{Ligando}
	\item 
		\hyperlink{numero de coordenacao}{Número de coordenação}
	\item 
		\hyperlink{esfera de coordenacao}{Esfera de coordenação}
	\end{itemize}
\item \hyperlink{tipos de ligandos}{Tipos de Ligandos}.
\item 
	\hyperlink{nomenclatura complexos de coordenacao}%
	{Regras de nomenclatura dos Complexos de Coordenação}
\item Afinidade de metais para ligandos.
\item Classificação de HSAB.
\item Estabilidade de compostos de coordenação.
\item Efeito de quelação.
\item Números de coordenação mais prováveis em compostos de coordenação.
\item Isomeria.
\item Teorias de ligação química em compostos de coordenação.
\item Teoria do Enlace de Valência.
\item Teoria do Campo Cristalino.
\item Interpretação de propriedades
	\begin{itemize}
	\item Magnéticas, 
	\item Espectros electrónicos 
	\item Propriedades termodinâmicas.
	\end{itemize}
\item Diagramas de Orgel e Tanabe-Sugano.
\item Reactividade Química de Complexos.
\end{enumerate}

% Interesse da Química Inorgânica.
% Definições - composto de coordenação, elemento central, ligando, número e esfera de coordenação.
% Tipos de ligandos.
% Regras de nomenclatura dos compostos de coordenação.
% Afinidade de metais para ligandos.
% Classificação de HSAB.
% Estabilidade de compostos de coordenação.
% Efeito de quelação.
% Números de coordenação mais prováveis em compostos de coordenação.
% Isomeria.
% Teorias de ligação química em compostos de coordenação; Teoria do Enlace de Valência; Teoria do Campo Cristalino.
% Interpretação de propriedades 
% 	magnéticas, 
% 	espectros electrónicos 
% 	propriedades termodinâmicas.
% Diagramas de Orgel e Tanabe-Sugano.
% Reactividade Química de Complexos. 

\section{Química Inorgânica}
A química se divide em dois ramos, química orgânica e inorgânica.
Química inorgânica compreendem todos os compostos que não possuem ligações de carbono do tipo \ch{C-H}

\begin{appendices}
\setcounter{chapter}{1}
\chapter{Background}

\hypertarget{lewis}{}%
\section{Reações Ácidos-Base de Lewis}

\hypertarget{ligacoes covalentes coordenadas}{}%
\section{Ligações Covalentes Coordenadas}

\hypertarget{ligacao sigma}{}
\section{Ligações $\sigma$}

\end{appendices}

\chapter{Complexos e Compostos de Coordenação}

\section{Complexos de Coordenação}%
%
São produtos de \hyperlink{lewis}{reações acido-base de Lewis}  onde um ou mais \hyperlink{elemento central}{átomos / elementos centrais}, geralmente átomos metálicos, se ligam a um arranjo de \hyperlink{ligandos}{ligandos} por \hyperlink{ligacoes covalentes coordenadas}{\textbf{ligações covalentes coordenadas}}.\\\\
Para ser considerado um complexo de coordenação, se deve ter o \hyperlink{indice de coordenacao}{índice de coordenação} superior a seu \hyperlink{estadodeoxidacao}{estado de oxidação}

\hypertarget{composto de coordeancao}{}%
\subsection{Compostos de Coordenação}

Compostos que possuem complexos de coordenação.\\
Compostos de coordenação e complexos são especies químicas distintas, suas propriedades e comportamentos químicos são diferentes dos seus componentes

\subsection{Estrutura de um Complexo de Coordenação}
\begin{center}
	\huge \ch{[A\,B]^C}
\end{center}

\renewcommand{\theenumi}{\Alph{enumi}}

\begin{enumerate}
\item 
	\hyperlink{elemento central}{\bf{Átomo / Elemento Central}}
\item \hyperlink{ligandos}{\bf{Ligandos}}
\item \textbf{Carga} do complexo de coordenação
\end{enumerate}

%\begin{enumerate}
%\item \hypertarget{ElementoCentral}{\textbf{Atomos/Ions metálicos centrais}}, são ácidos de Lewis
%\item \textbf{moléculas ou ions} ligados aos átomos centrais por \hyperlink{LigacoesCovalentesCoordenadas}{ligações covalentes coordenadas}, \hypertarget{Ligando}{\textbf{Ligandos}}, são bases de Lewis
%\item \textbf{Carga} do complexo de coordenação
%\end{enumerate}

\subsection{Caracteristicas}

\hypertarget{elemento central}{}%
\subsubsection{Átomo / Elemento Central}

Elemento metálico que ocupa o centro do complexo, considerado \hyperlink{Lewis}{ácido de Lewis}, existem complexos com \hyperlink{complexo polinuclear}{multiplos átomos / elementos centrais}.

\hypertarget{ligando}{}%
\subsubsection{Ligandos}

Elementos diretamente ligado ao \hyperlink{elemento central}{elemento central}, de natureza ionica ou molecular, são considerados \hyperlink{lewis}{bases de lewis}

\hypertarget{indice de coordenacao}{}%
\subsubsection{Índice / Numero de Coordenação}

É o numero de \hyperlink{ligacao sigma}{ligações $\sigma$} entre os \hyperlink{ligando}{ligandos} e o \hyperlink{elemento central}{elemento central}

\hypertarget{estado de oxidacao}{}%
\subsubsection{Estado de Oxidação}

Carga do \hyperlink{elemento central}{elemento central} representado em numeros romanos.\\
É definido como a carga que o átomo central teria se todos seus \hyperlink{ligando}{ligandos} e pares eletrônicos compartilhados fossem removidos

\hypertarget{quelacao}{}%
\subsubsection{Quelação e Ligandos Polidentados}

\hyperlink{ligando}{Ligandos} podem formar mais de uma ligação com o átomo central, formando um, ou mais anéis quelatos.\\
O numero de átomos de um ligando que se ligam ao mesmo tempo com o \hyperlink{elemento central}{átomo central} é indicado pelos adjetivos:
\textbf{Bi}dentado;
\textbf{Tri}dentado;
\textbf{Tetra}dentado;
\textbf{Penta}dentado;
$\cdots$
\paragraph{Exemplos de ligandos polidentados:}

\begin{itemize}

% Bidentados
	
	\item \ch{SO4^{-2}}: Bidentado instável\\
			Sulfato
	
	\item \ch{C2H4(NH2)2}: Bidentado\\
			Etilenodiamina $\to$ Etilenodiamino (en)
	
	\item \ch{C2O4^{2-}} = \ch{(CO2)2^{2-}}: Bidentado\\
			Oxalato $\to$ Oxalato
	
	\item \ch{C10H8N2}: Bidentado\\
			Dipiridina $\to$ Bipiridino (bipy)

	\item \ch{CH3COCHCOCH3^-}: Bidentado\\
			Acetilacetonato $\to$ Acetilacetonato (ACAC)
		
	\item \ch{OCC(CH3)3CHCOC3F7}: Bidentado\\
			Heptafluoro\,dimetil\,octanedionato (fod)
		
	\item \ch{C12H8N2}: Bidentado\\
			Fenatrolina $\to$ Fenatrolina (phen)\\
			Garante grande estabilidade, tambem é conhecido por 			emitir luz
		
% Tridentados
		
% Tetradentados

% Hexadentado

	\item \ch{diaza - 18 - crown - 6} = \ch{C12H26N2O4}:
			Hexadentado\\
			diaza-18-crown-6\\
			Forma um anél que rodeia todo o perímetro do átomo 			central o garantindo grande estabilidade
	
	\item \ch{C10H16N2O8} = \ch{(C2OOH)2NC2N(C2OOH)2}:
			Hexadentado\\
			etileno\,diamino\,tetra\,acetato $\to$
			etileno\,diamino\,tetra\,acetato (EDTA)
	
% Octadentado
	
	\item
		\ch{C14H23N2O10^{5-}} =
		\ch{(C2OOH)N(C2N(C2OOH)2)2^{5-}}: Octadentado\\
		dietileno\,triamino\,penta\,acetato $\to$
		dietileno\,triamino\,penta\,acetato (DTPA)
		
\end{itemize}

\paragraph{Ligandos Ambidentados}
Alguns ligandos polidentados podem fazer apenas algumas de suas ligações por vês com o \hyperlink{elemento central}{elemento central}, esses são caracterisados como ambidentados. por possuirem essa característica, sua presença em um complexo o atribui \hyperlink{isomeria de ligacao}{isomeria de ligação}

%\vspace{0.5cm}\noindent\begin{minipage}{\textwidth}
\paragraph{Exemplos de Ligandos Ambidentados}%
\begin{itemize}

	\item \ch{SCN^-} / \ch{NCS^-}: Tiocianato / Isotiocianato

\begin{multicols}{2}

	\item \ch{NO2^-} / \ch{ONO^-}: Nitro / Nitrito

\end{multicols}

\end{itemize}
%\end{minipage}


%\begin{itemize}
%\item \textbf{Complexos Iónicos} (ou sais complexos) são complexos de coordenação que possuem carga
%\item \textbf{Adultos} são complexos que apresentam carga elétrica nula
%\item \textbf{Átomo doador} é o átomo do ligando que está diretamente ligado ao átomo central
%\item \hypertarget{NumerodeCoordenacao}{\textbf{Índice/Numero de Coordenação}} é a quantidade de átomos doadores do complexo
%\item \hypertarget{EsferadeCoordenacao}{\textbf{Esfera de Coordenação}} é o conjunto formado pelos elementos centrais e seus ligandos
%\end{itemize}

\subsection{Exemplos:}

\begin{itemize}
\begin{multicols}{3}

	\item \ch{ [Ag(NH3)2]^{+ } }
	\item \ch{ [Fe(CO)5] }
	\item \ch{ [Pt(CN)4]^{-2} }
	\item \ch{ [Cl3BNH3] }
	\item \ch{ [Co(NH3)6]^{+3} }
	\item \ch{ [Ni(CO)4] }

\end{multicols}
\end{itemize}

\hypertarget{NomenclaturaComplexosdeCoordenacao}{}%
\section{Regras de Nomenclatura: Complexos de Coordenação}

\subsection{Estrutura}
\begin{center}
	\huge \ch
{ 
	[ \textcolor{Red}{A}\,\textcolor{Blue}{B} ] 
}: 
\textcolor{Blue}{B}\,\textcolor{Red}{A}
\end{center}

\begin{enumerate}
\color{Red} \item \color{Red!20} Nome do elemento central
\color{Blue}\item \color{Blue!20}Nome dos ligandos
\end{enumerate}

\subsection{Regras}

\renewcommand{\theenumi}{\arabic{enumi}}
\noindent\begin{multicols}{2}
	
	\begin{enumerate}
	\item \hyperlink{regras1}{Complexos Aniônicos}
	\item \hyperlink{regras2}{Multiplos ligandos}
	\item \hyperlink{regras3}%
		{Prefixos numéricos (di, tri, tetra, $\cdots$)}
	\item \hyperlink{regras4}{Regra geral para sufixos}
	\item \hyperlink{regras5}{Aniões terminados em ``eto''}
	\item \hyperlink{regras6}{Aniões terminados em ``ido''}
	\item \hyperlink{regras7}{Aniões terminados em ``ato''}
	\item \hyperlink{regras8}%
		{Ligandos radicais derivados de hidrocarbonetos}
	\item \hyperlink{regras9}%
		{Ligandos moleculares terminados em ``a''}
	\item \hyperlink{regras10}{Ligandos com nomes específicos}
	\item \hyperlink{regras11}%
		{Ligando que atua entre dois centros de coordenação}
	\end{enumerate}
	
\end{multicols}

\hypertarget{regras1}{}%
\subsubsection{Complexos Aniônicos*}

recebem a terminação ``ato'' seguido do seu numero de oxidação em formato romano entre parenteses\\
* aniônico: portador de carga negativa

\hypertarget{regras2}{}%
\subsubsection{Multiplos ligandos}

são nomeados na seguinte ordem:\\
neutros (moleculares), aniônicos (negativos), catiônicos (positivos).\\
ligantes de mesma natureza se ordena alfabeticamente.

\hypertarget{regras3}{}%
\subsubsection{Prefixos numéricos (di, tri, tetra, $\cdots$)}

indica a quantidade de ligandos iguais.\\
variação dos prefixos numericos para: ``dis, tris, tetrakis, pentakis, $\cdots$'' em ligandos que possuem prefixos numéricos em seu nome (ex: ligandos orgânicos, dipiridina, trifosfina)\\

\noindent\begin{minipage}{\linewidth}

\paragraph{Exemplos:}\hfill\vspace{0.3cm}
\begin{multicols}{2}

	\begin{itemize}
	
	\item \ch{[CoF\textcolor{LightGreen}{6}]^{3-}}:
		\textcolor{LightGreen}{Hex}fluorocobaltato(III)
		
	\item \ch{[Ag(NH3)\textcolor{LightGreen}{2}]^+}:
		\textcolor{LightGreen}{Di}aminprata(I)
		
	\item \ch{[Ni(CO)\textcolor{LightGreen}{4}]^+}:
		\textcolor{LightGreen}{Tetra}carbonilniquel(0)
		
	\item \ch{[Mn(CO)\textcolor{LightGreen}{5}]^+}:
		\hbox{%
			\textcolor{LightGreen}{Penta}carbonilmanganato(-I)
		}
	\end{itemize}
	
\end{multicols}

\end{minipage}

\hypertarget{regras4}{}%
\subsubsection{Regra geral para sufixos}

\begin{multicols}{2}
	\begin{itemize}
	\item 
		``\textcolor{LightGreen}{eto}'' 
		$\to$ ``\textcolor{LightGreen}{o}''
	\item 
		``\textcolor{LightGreen}{ito}''
		$\to$ ``\textcolor{LightGreen}{o}''
	\item 
		``\textcolor{LightGreen}{ido}''
		$\to$ ``\textcolor{LightGreen}{o}''
	\item 
		``\textcolor{LightGreen}{ato}''
		$\to$ ``\textcolor{LightGreen}{ato}''
	\end{itemize}
\end{multicols}

\hypertarget{regras5}{}%
\subsubsection{Aniões terminados em ``eto''}

\begin{multicols}{2}
	
	\begin{itemize}
	\item 
		Fluor\textcolor{LightGreen}{eto} (\ch{F^-})
		$\to$ Fluor\textcolor{LightGreen}{o}
 	\item 
		Clor\textcolor{LightGreen}{eto} (\ch{Cl^-})
		$\to$ Clor\textcolor{LightGreen}{o}
	\item 
		Brom\textcolor{LightGreen}{eto} (\ch{Br^-})
		$\to$ Brom\textcolor{LightGreen}{o}
	\item
		Iod\textcolor{LightGreen}{eto} (\ch{I^-})
		$\to$ Iod\textcolor{LightGreen}{o}
	\item
		Cian\textcolor{LightGreen}{eto} (\ch{CN^-})
		$\to$ Cian\textcolor{LightGreen}{o}
	\item
		Amid\textcolor{LightGreen}{eto} (\ch{NH2^-})
		$\to$ Amid\textcolor{LightGreen}{o}
	\end{itemize}

\end{multicols}

Com exceção do \textbf{hidreto (\ch{H^-})} que ou não sofre alteração ou se usa hidr\textcolor{LightGreen}{o}

\hypertarget{regras6}{}%
\subsubsection{Aniões terminados em ``ido''}

\begin{multicols}{2}
	
	\begin{itemize}
	\item 
		Hidróx\textcolor{LightGreen}{ido} (\ch{OH^-})
		$\to$ Hidrox\textcolor{LightGreen}{o}
	\item 
		Óx\textcolor{LightGreen}{ido} (\ch{O^{2-}})
		$\to$ Ox\textcolor{LightGreen}{o}
	\item
		Peróx\textcolor{LightGreen}{ido} (\ch{O2^{-2}})
		$\to$ Perox\textcolor{LightGreen}{o}
	\end{itemize}

\end{multicols}

\hypertarget{regras7}{}%
\subsubsection{Aniões terminados em ``ato''}

\begin{itemize}
	\begin{multicols}{2}
	\item
		Carbon\textcolor{LightGreen}{ato} (\ch{CO3^{2-}})
		$\to$ Carbon\textcolor{LightGreen}{ato}
	\item
		Nitr\textcolor{LightGreen}{ato} (\ch{NO3^-})
		$\to$ Nitr\textcolor{LightGreen}{ato}
	\item 
		Sulf\textcolor{LightGreen}{ato} (\ch{SO4^{2-}})
		$\to$ Sulf\textcolor{LightGreen}{ato}
	\item
		Oxal\textcolor{LightGreen}{ato} (\ch{C2O4^{2-}})
		$\to$ Oxal\textcolor{LightGreen}{ato}
	\end{multicols}\vspace{-0.5cm}
	\item
		Acet\textcolor{LightGreen}{ato} (\ch{CH3COO^-})
		$\to$ Acet\textcolor{LightGreen}{ato}
	\item
		Acetilaceton\textcolor{LightGreen}{ato} (\ch{CH3COCHCOCH3^-})
		$\to$ Acetilaceton\textcolor{LightGreen}{ato} (ACAC)
	\end{itemize}

\hypertarget{regras8}{}%
\subsubsection{Ligandos radicais derivados de hidrocarbonetos}

matem o nome do radical

\begin{itemize}
	\begin{multicols}{2}
 	\item Metil (\ch{CH3}) $\to$ Metil (Me)
	\item Etil (\ch{C2H5}) $\to$ Etil (et)
	\item Fenil (\ch{C6H5}) $\to$ Fenil (Ph ou \O)
	\end{multicols} \vspace{-0.5cm}
	\item 
		Ciclopentadienil (\ch{C5H5})
		$\to$ Ciclopentadienil (Cp)
\end{itemize}

\vspace{0.5cm}\noindent%
\begin{minipage}{\textwidth}

\hypertarget{regras9}{}%
\subsubsection{Ligandos moleculares terminados em ``a''}%

o substituem por ``o''
\vspace{0.25cm}

\begin{itemize}
 	\item 
		Piridin\textcolor{LightGreen}{a} (\ch{C5H5N})
		$\to$ Piridin\textcolor{LightGreen}{o} (Py)
	\item
		Dipiridin\textcolor{LightGreen}{a} (\ch{C10H5N})
		$\to$ \textcolor{LightGreen}{B}%
				ipiridin%
				\textcolor{LightGreen}{o}
				(Bipy)
	\item
		Etilenodiamin\textcolor{LightGreen}{a} (\ch{C2H8N2})
		$\to$	Etilenodiamin\textcolor{LightGreen}{o} (En)
	\item
		Trifenilfosfin\textcolor{LightGreen}{a} (\ch{P(C6H5)3})
		$\to$ Trifenilfosfin\textcolor{LightGreen}{o} 
				(PPh$_3$ ou P\O$_3$)
\end{itemize}

\end{minipage}

\hypertarget{regras10}{}%
\subsubsection{Ligandos com nomes específicos}

\begin{itemize}
	
	\item Água (\ch{H2O}) $\to$ Aquo
	\item Amónia (\ch{NH3}) $\to$ Amino ou Amin
	\item Monóxiodo de carbono (\ch{CO}) $\to$ Carbonilo
	\item Monóxido de azoto (\ch{NO}) $\to$ Nitrosil
	\item 
		Oxigénio (molecular) (\ch{O2})\hyperlink{o2n2}{*}
		$\to$ Dioxigénio
	\item 
		Azoto(molecular) (\ch{N2})\hyperlink{o2n2}{*}
		$\to$ Diazoto
	\item Hidreto (\ch{H^-}) $\to$ Hidro ou Hidreto
\end{itemize}

\hypertarget{o2n2}{*} Nas formulas esses ligandos devem ser escritos entre parênteses

\hypertarget{regras11}{}%
\subsubsection{Ligando que atua entre dois centros de coordenação}

\hyperlink{compolinucleares}{complexos polinucleares} quando o ligado atua entre os dois centros de coordenação, recebe a letra $\mu$ separada por hífens

\subsubsection{Exemplo Complexos de coordenação:}

\begin{multicols}{2}

	\begin{itemize}
	\item \ch{[Co\textcolor{LightGreen}{F}6]^{3-}}:
		Hex\textcolor{LightGreen}{fluoro}cobaltato(III)
	\item \ch{[Fe(S\textcolor{LightGreen}{CN})6]^{3-}}:
		Hexatio\textcolor{LightGreen}{cianato}ferrato(III)
	\end{itemize}
	
\end{multicols}








\end{document}










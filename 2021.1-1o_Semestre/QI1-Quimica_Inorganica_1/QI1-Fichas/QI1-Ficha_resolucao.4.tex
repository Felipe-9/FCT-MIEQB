\documentclass[12pt]{article}

% Linguagem
\usepackage[portuguese]{babel}

% Clickable Table of contents
\usepackage{hyperref}
\hypersetup{
	hidelinks=true,
	colorlinks=true,
	linkcolor=DarkGreen!20!LightGreen!25!
}

% Table of contents
\usepackage{tocloft}
\setlength{\cftsubsecnumwidth}{2em} % Fix subsection width
% Fix space between subsection items on toc
\renewcommand\cftsubsecafterpnum{\vskip5pt}

% Multicols
\usepackage{multicol}

% Customize Chapter
%\usepackage{titlesec}
%\titleformat{\chapter}[hang]{\Huge\bfseries\color{DarkGreen!75!}}{\thechapter\hspace{20pt}{$|$}\hspace{20pt}}{0pt}{\Huge\bfseries}

% Appendix
%\usepackage{appendix}

% Maths
%\usepackage{amssymb} 		
%\usepackage{amsmath} 
%\usepackage[utf8]{inputenc} %useful to type directly diacritic characters

%\newcommand{\bm}[1]{{\boldmath{\large{\begin{align*} #1 \end{align*}}}}}

% Vectors
%\usepackage{esvect} 	% Vector over-arrow
%\renewcommand{\vec}{\vv} % Vecto over-arrow

% Tikz
%\usepackage{tikz}		% Vector diagrams
%\usetikzlibrary{calc}  % Vector calculations
%\usepackage{varwidth}  % List inside TikzPicture

% Chem
\usepackage{chemformula} 	% formulas quimicas
\usepackage{chemfig} 		% Estruturas quimicas

% Tabular
%\usepackage{multirow}
%\usepackage{dcolumn}
%\newcolumntype{d}[1]{D{.}{\cdot}{#1} } % Align on decimal

% Colors
\usepackage{xcolor}

\definecolor{DarkBlue}	{HTML}{252A36}
\definecolor{LightGreen}{HTML}{7CCC6C}
\definecolor{DarkGreen}	{HTML}{008675}

\pagecolor{DarkBlue!110!}
\color{DarkGreen!20!}

%\definecolor{Red}  {hsb}{0  ,.6,1}
%\definecolor{Blue} {hsb}{0.6,.6,1}
%\definecolor{Green}{hsb}{0.3,.6,1}

% Counters
%\counterwithin*{section}{part} % Reset section on part

% Section and Subsection Customization
%\renewcommand\thesection{Questão \arabic{section} }
%\renewcommand\thesubsection{\arabic{section}-\alph{subsection}) }

\begin{document}

\title{\bfseries\color{DarkGreen!75!}%
	Química Inorgânica 1 - Ficha IV\\Teoria de Pearson%
}
\author{Felipe Pinto - 61387}

\maketitle
\tableofcontents
\break

% Q1
\section{}

% Q1 - a)
\subsection{\ch{K2[CdI4]}}
\begin{itemize}

	\item Ligandos:
	
	\item Átomos Doadores dos Ligandos:
	
	\item Contraião:
	
	\item Numero de Coordenação do Metal:
	
	\item Configuração Eletrônica do Metal:
	
	\item Natureza do Átomo Central e Ligandos:
	
	\item Isómeros:

\end{itemize}

\break

% Q1 - b)
\subsection{\ch{ [Ir(NH3)5(SO2)]Cl3 }}
\begin{itemize}

	\item Ligandos:
	
	\item Átomos Doadores dos Ligandos:
	
	\item Contraião:
	
	\item Numero de Coordenação do Metal:
	
	\item Configuração Eletrônica do Metal:
	
	\item Natureza do Átomo Central e Ligandos:
	
	\item Isómeros:

\end{itemize}

\break

% Q1 - c)
\subsection{\ch{ [Pd(en)2]SO4 }}
\begin{itemize}

	\item Ligandos:
	
	\item Átomos Doadores dos Ligandos:
	
	\item Contraião:
	
	\item Numero de Coordenação do Metal:
	
	\item Configuração Eletrônica do Metal:
	
	\item Natureza do Átomo Central e Ligandos:
	
	\item Isómeros:

\end{itemize}

\break

% Q1 - d)
\subsection{\ch{ [NiI2(PPh3)2] }}
\begin{itemize}

	\item Ligandos:
	
	\item Átomos Doadores dos Ligandos:
	
	\item Contraião:
	
	\item Numero de Coordenação do Metal:
	
	\item Configuração Eletrônica do Metal:
	
	\item Natureza do Átomo Central e Ligandos:
	
	\item Isómeros:

\end{itemize}

\break

% Q1 - e)
\subsection{\ch{ K[Ag(SCN)2] }}
\begin{itemize}

	\item Ligandos:
	
	\item Átomos Doadores dos Ligandos:
	
	\item Contraião:
	
	\item Numero de Coordenação do Metal:
	
	\item Configuração Eletrônica do Metal:
	
	\item Natureza do Átomo Central e Ligandos:
	
	\item Isómeros:

\end{itemize}

\break

% Q1 - f)
\subsection{\ch{ [La(NH3)4(OH2)2]F2 }}
\begin{itemize}

	\item Ligandos:
	
	\item Átomos Doadores dos Ligandos:
	
	\item Contraião:
	
	\item Numero de Coordenação do Metal:
	
	\item Configuração Eletrônica do Metal:
	
	\item Natureza do Átomo Central e Ligandos:
	
	\item Isómeros:

\end{itemize}

\break

% Q1 - g)
\subsection{\ch{ [Hg(SH2(CH2)2NH2)3]F2 }}
\begin{itemize}

	\item Ligandos:
	
	\item Átomos Doadores dos Ligandos:
	
	\item Contraião:
	
	\item Numero de Coordenação do Metal:
	
	\item Configuração Eletrônica do Metal:
	
	\item Natureza do Átomo Central e Ligandos:
	
	\item Isómeros:

\end{itemize}

\break

% Q1 - h)
\subsection{\ch{ [PtCl2(TeO)2] }}
\begin{itemize}

	\item Ligandos:
	
	\item Átomos Doadores dos Ligandos:
	
	\item Contraião:
	
	\item Numero de Coordenação do Metal:
	
	\item Configuração Eletrônica do Metal:
	
	\item Natureza do Átomo Central e Ligandos:
	
	\item Isómeros:

\end{itemize}

\break

% Q1 - i)
\subsection{\ch{ [Fe(bpy)3](ClO4)3 }}
\begin{itemize}

	\item Ligandos:
	
	\item Átomos Doadores dos Ligandos:
	
	\item Contraião:
	
	\item Numero de Coordenação do Metal:
	
	\item Configuração Eletrônica do Metal:
	
	\item Natureza do Átomo Central e Ligandos:
	
	\item Isómeros:

\end{itemize}

\break

% Q1 - j)
\subsection{\ch{ [Co(NH3)2I2]Br3 }}
\begin{itemize}

	\item Ligandos:
	
	\item Átomos Doadores dos Ligandos:
	
	\item Contraião:
	
	\item Numero de Coordenação do Metal:
	
	\item Configuração Eletrônica do Metal:
	
	\item Natureza do Átomo Central e Ligandos:
	
	\item Isómeros:

\end{itemize}

\break

\section{}

% Q2 - a)
\subsection{\ch{K2[CdI4]}}

% Q2 - b)
\subsection{\ch{ [Ir(NH3)5(SO2)]Cl3 }}

% Q2 - c)
\subsection{\ch{ [Pd(en)2]SO4 }}

% Q2 - d)
\subsection{\ch{ [NiI2(PPh3)2] }}

% Q2 - e)
\subsection{\ch{ K[Ag(SCN)2] }}

% Q2 - f)
\subsection{\ch{ [La(NH3)4(OH2)2]F2 }}

% Q2 - g)
\subsection{\ch{ [Hg(SH2(CH2)2NH2)3]F2 }}

% Q2 - h)
\subsection{\ch{ [PtCl2(TeO)2] }}

% Q2 - i)
\subsection{\ch{ [Fe(bpy)3](ClO4)3 }}

% Q2 - j)
\subsection{\ch{ [Co(NH3)2I2]Br3 }}

\break

% Q3
\section{}

% Q3 - a)
\subsection{\ch{ Au(I),\ SH2,\ H2O,\ NO3^-,\ CN }}

\begin{itemize}
\begin{multicols}{2}
	
	\item 1
	 
\end{multicols}
\end{itemize}

\begin{itemize}

	\item \ch{  }
	
	\begin{itemize}
 		
		\item Natureza do Elemento central:
		
		\item Natureza do Ligandos:
		
		\item Estrutura:
		
	\end{itemize}

\end{itemize}

% Q3 - b)
\subsection{\ch{ Ru(II),\ CN,\ SH2(CH2)2NH2,\ H2O,\ BF4^- }}

\begin{itemize}
\begin{multicols}{2}
	
	\item \ch{Ru(II)}: 			Intermédio (Mole)
	\item \ch{CN^-}:			Base Mole
	\item \ch{SH2(CH2)2NH2}:	Base
	\item \ch{H2O}:			Acido/Base Dura
	\item \ch{BF4^-}: 			Base Dura
	 
\end{multicols}
\end{itemize}

\begin{itemize}

	\item \ch{ Ru(H2O)4(BF4)2 } (Octaédrico)\\
		Apenas trans por causa do \ch{BF4}
		ocupar muito espaço
		
	\item Isomeros de ligação do anterior
	
	\item \ch{ Ru(CN)6^{4-} }
	
	\item Isomeros ionicos anterior com agua
	
	\item \ch{ Ru(SH2(CH2)2NH2)3 }
	
	\item Isomeros Opticos do anterior
	\item Isomeros fac mer do anterior

\end{itemize}

\break

% Q3 - c)
\subsection{\ch{ Cu(II),\ acac,\ NO2,\ Cl^- }}
\begin{itemize}
 
 	\item \ch{Cu(II)}:	Ácido Intermédio (D9: Complexos Octaedricos)
	\item \ch{acac^{2-}}:	Base Duro
	\item \ch{NO2^+}:		Ácido Duro
	\item \ch{Cl^-}:		Base Intermédia (dura)
	
\end{itemize}
\vspace{2mm}
\begin{itemize}

	\item \ch{ Cu(acac)3 }
	\item Intermédios \ch{ [Cu(NO2)_i(Cl)_j]^k } onde $i+j = 6$ e $k = 2-j$
	
	\item \ch{ Cu(Cl)4 }
	
\end{itemize}


% Q3 - d)
\subsection{\ch{ Ba(II),\ H2O,\ EDTA^-,\ ClO4^- }}
\begin{itemize}
	
	\item \ch{Ba(II)}: Ácido Duro
	\item \ch{H2O}: Base
	
\end{itemize}

\vspace{2mm}

\begin{itemize}
	
	\item \ch{Ba(EDTA)}
	
\end{itemize}

\break

% Q3 - e)
\subsection{\ch{ Fe(III),\ CO2,\ phen,\ en,\ F^- }}
\begin{itemize}
	
	\item \ch{Fe(III)}: Ácido Duro
	\item \ch{CO2}:	Base/Ácido Duro
	\item \ch{phen}:	Base Intermédia
	\item \ch{en}:		Base Intermédia (dura)
	\item \ch{F^-}:	Base Dura
	
\end{itemize}

\vspace{2mm}

\begin{itemize}
	
	\item \ch{Fe(phen)3}\\
		3 aneis quelatos
		
	\item \ch{ Fe(en)3 }
	
	\item \ch{Fe(F)6}
	
	\item \ch{Fe(CO2)6}
	
\end{itemize}

\break

% Q3 - f)
\subsection{\ch{ Hg(II),\ tu,\ F^-,\ SCN^- }}
\begin{itemize}
	
	\item \ch{Hg(III)}: Ácido Duro
	\item \ch{tu}:		Base\\
		ambidentado por 3, Ocupa espaço e não forma anél quelato então é instável
	\item \ch{F^-}:	Base Dura
	\item \ch{SCN^-}:	Base Duro/Intermédio
	
\end{itemize}

\vspace{2mm}

\begin{itemize}
	
	\item \ch{Hg(Tu)6}
	
	\item \ch{Hg(SCN)6}
	
	\item \ch{Hg(Tu)_i(SCN)_j(NCS)_k}\\
		onde $i+j+k = 6$\\
		quanto menos tiureia mais estável\\
		quanto mais SCN mais estável
	
\end{itemize}

%\begin{itemize}
%\begin{multicols}{2}
%	
%	\item 1
%	 
%\end{multicols}
%\end{itemize}
%
%\begin{itemize}
%
%	\item \ch{  }
%	
%	\begin{itemize}
% 		
%		\item Natureza do Elemento central:
%		
%		\item Natureza do Ligandos:
%		
%		\item Estrutura:
%		
%	\end{itemize}
%
%\end{itemize}







\end{document}











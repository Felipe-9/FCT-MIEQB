\documentclass[12pt]{article}

% Linguagem
\usepackage[portuguese]{babel}

% Clickable Table of contents
\usepackage{hyperref}
\hypersetup{
	hidelinks=true,
	colorlinks=true,
	linkcolor=DarkGreen!20!LightGreen!25
}

% Table of contents
\usepackage{tocloft}
\setlength{\cftsubsecnumwidth}{3em} % Fix subsection width
% Fix space between subsection items on toc
\renewcommand\cftsubsecafterpnum{\vskip5pt}
 
% Multicols
%\usepackage{multicol}

% Maths
%\usepackage{amssymb} 		
%\usepackage{amsmath} 
%\usepackage[utf8]{inputenc} %useful to type directly diacritic characters

%\newcommand{\bm}[1]{{\boldmath{\large{\begin{align*} #1 \end{align*}}}}}

% Vectors
%\usepackage{esvect} 	% Vector over-arrow
%\usepackage{tikz}		% Vector diagrams
%\usetikzlibrary{calc}  % Vector calculations
%\usepackage{varwidth}  % List inside TikzPicture

%\renewcommand{\vec}{\vv} % Vecto over-arrow

% Chem
\usepackage{chemformula} 	% formulas quimicas
\usepackage{chemfig} 		% Estruturas quimicas

% Colors
\usepackage{xcolor}

\definecolor{DarkBlue}	{HTML}{252A36}
\definecolor{LightGreen}{HTML}{7CCC6C}
\definecolor{DarkGreen}	{HTML}{008675}

\pagecolor{DarkBlue!110!}
\color{DarkGreen!20!}

% Counters
\counterwithin*{section}{part} % Reset section on part

% Resolução de listas
\renewcommand\thesection{A\,\Roman{section}}
\renewcommand\thesubsection{\thesection.\arabic{subsection}}

\begin{document}

\title{ \bfseries\color{DarkGreen!75}
	  QI I 
	- Ficha II
	\\Complexos Metálicos
}
\author{Felipe Pinto - 61387}

\maketitle
\tableofcontents
\break

\section{}

% A1.1
\noindent\begin{minipage}{\textwidth}
	
	\subsection{\ch{[CrCl2(en)2]Cl}}
	\begin{itemize}
   
   \item Índice de Coordenação:
   	6
   
   \item Número de Ligandos:
   	4	
	
   \item Ligandos:
   	2 cloretos e 2 etilenodiaminos
   	
   \item Átomos Doadores:
   	2 cloros e 4 azotos
   
   \item Átomo Metálico Central:
   	Cromo (\ch{Cr})
   
   \item Configuração Eletrónica:
   	\ch{[Ar]}$\, 3d^4$
   
   \item Estado de Oxidação:
   	+2
   
   \item Contraião:
   	Cloreto (\ch{Cl^-})
   
   \item Nome:
   	cloreto de dietilenodiamino\,dicloro\,cromo(II)
   
   \item Geometria:
   	octaedrica\vspace{0.3cm}\\
		\chemfig[angle increment = 15]%
		{%
			Cr?[cr]%
				(-[6,0.9]Cl)%
				(-[-6,0.9]Cl)%
				(
					-[1 ,1.2]N
					-[-1,0.8]C
					-[-4,0.6]C
					-[-11]N?[cr]
				)
				(
					-[11,1.2]N
					-[13,0.8]C
					-[16,0.6]C
					-[-1 ]N?[cr]
				)%	
		}
		
   
   \item Isomeria:
		Cis e trans
	\item Caracteristicas:
		estavel :)
	\end{itemize}
	
\end{minipage}

% A1.2
\noindent\begin{minipage}{\textwidth}
	
	\subsection{\ch{[Co(ONO)(NH3)5]SO4}}
	\begin{itemize}
   
   \item Índice de Coordenação:
   
   \item Número de Ligandos:
   
   \item Ligandos:
   
   \item Átomos Doadores:
   
   \item Átomo Metálico Central:
   
   \item Configuração Eletrónica:
   
   \item Estado de Oxidação:
   
   \item Contraião:
   
   \item Nome:
   
   \item Geometria:
   
   \item Isomeria:

	\item Caracteristicas:

	\end{itemize}
	
\end{minipage}

% A1.3
\noindent\begin{minipage}{\textwidth}
	
	\subsection{\ch{[Cr(H2O)4Cl2]Cl}}
	\begin{itemize}
   
   \item Índice de Coordenação:
   
   \item Número de Ligandos:
   
   \item Ligandos:
   
   \item Átomos Doadores:
   
   \item Átomo Metálico Central:
   
   \item Configuração Eletrónica:
   
   \item Estado de Oxidação:
   
   \item Contraião:
   
   \item Nome:
   
   \item Geometria:
   
   \item Isomeria:

	\item Caracteristicas:

	\end{itemize}
	
\end{minipage}

% A1.4
\noindent\begin{minipage}{\textwidth}
	
	\subsection{\ch{Na3[Co(NO2)6]}}
	\begin{itemize}
   
   \item Índice de Coordenação:
   
   \item Número de Ligandos:
   
   \item Ligandos:
   
   \item Átomos Doadores:
   
   \item Átomo Metálico Central:
   
   \item Configuração Eletrónica:
   
   \item Estado de Oxidação:
   
   \item Contraião:
   
   \item Nome:
   
   \item Geometria:
   
   \item Isomeria:

	\item Caracteristicas:

	\end{itemize}
	
\end{minipage}

% A1.5
\noindent\begin{minipage}{\textwidth}
	
	\subsection{\ch{[(NH3)5-Cr-OH-Cr(NH3)5]Cl5}}
	\begin{itemize}
   
   \item Índice de Coordenação:
   
   \item Número de Ligandos:
   
   \item Ligandos:
   
   \item Átomos Doadores:
   
   \item Átomo Metálico Central:
   
   \item Configuração Eletrónica:
   
   \item Estado de Oxidação:
   
   \item Contraião:
   
   \item Nome:
   
   \item Geometria:
   
   \item Isomeria:

	\item Caracteristicas:

	\end{itemize}
	
\end{minipage}

% A1.6
\noindent\begin{minipage}{\textwidth}
	
	\subsection{\ch{[Co(NH3)2(H2O)]Br2}}
	\begin{itemize}
   
   \item Índice de Coordenação:
   
   \item Número de Ligandos:
   
   \item Ligandos:
   
   \item Átomos Doadores:
   
   \item Átomo Metálico Central:
   
   \item Configuração Eletrónica:
   
   \item Estado de Oxidação:
   
   \item Contraião:
   
   \item Nome:
   
   \item Geometria:
   
   \item Isomeria:

	\item Caracteristicas:

	\end{itemize}
	
\end{minipage}

% A1.7
\noindent\begin{minipage}{\textwidth}
	
	\subsection{\ch{[Co(NH3)5Cl]Cl2}}
	\begin{itemize}
   
   \item Índice de Coordenação:
   
   \item Número de Ligandos:
   
   \item Ligandos:
   
   \item Átomos Doadores:
   
   \item Átomo Metálico Central:
   
   \item Configuração Eletrónica:
   
   \item Estado de Oxidação:
   
   \item Contraião:
   
   \item Nome:
   
   \item Geometria:
   
   \item Isomeria:

	\item Caracteristicas:

	\end{itemize}
	
\end{minipage}

% A1.8
\noindent\begin{minipage}{\textwidth}
	
	\subsection{\ch{K[PtCl3(C2H4)]}}
	\begin{itemize}
   
   \item Índice de Coordenação:
   
   \item Número de Ligandos:
   
   \item Ligandos:
   
   \item Átomos Doadores:
   
   \item Átomo Metálico Central:
   
   \item Configuração Eletrónica:
   
   \item Estado de Oxidação:
   
   \item Contraião:
   
   \item Nome:
   
   \item Geometria:
   
   \item Isomeria:

	\item Caracteristicas:

	\end{itemize}
	
\end{minipage}

% A1.9
\noindent\begin{minipage}{\textwidth}
	
	\subsection{\ch{[Co(NH3)4(NCS)Cl]NO3}}
	\begin{itemize}
   
   \item Índice de Coordenação:
   
   \item Número de Ligandos:
   
   \item Ligandos:
   
   \item Átomos Doadores:
   
   \item Átomo Metálico Central:
   
   \item Configuração Eletrónica:
   
   \item Estado de Oxidação:
   
   \item Contraião:
   
   \item Nome:
   
   \item Geometria:
   
   \item Isomeria:

	\item Caracteristicas:

	\end{itemize}
	
\end{minipage}

% A1.10
\noindent\begin{minipage}{\textwidth}
	
	\subsection{\ch{[Pt(en)Cl4]}}
	\begin{itemize}
   
   \item Índice de Coordenação:
   
   \item Número de Ligandos:
   
   \item Ligandos:
   
   \item Átomos Doadores:
   
   \item Átomo Metálico Central:
   
   \item Configuração Eletrónica:
   
   \item Estado de Oxidação:
   
   \item Contraião:
   
   \item Nome:
   
   \item Geometria:
   
   \item Isomeria:

	\item Caracteristicas:

	\end{itemize}
	
\end{minipage}

% A1.11
\noindent\begin{minipage}{\textwidth}
	
	\subsection{\ch{Na2[Ni(EDTA)]}}
	\begin{itemize}
   
   \item Índice de Coordenação:
   
   \item Número de Ligandos:
   
   \item Ligandos:
   
   \item Átomos Doadores:
   
   \item Átomo Metálico Central:
   
   \item Configuração Eletrónica:
   
   \item Estado de Oxidação:
   
   \item Contraião:
   
   \item Nome:
   
   \item Geometria:
   
   \item Isomeria:

	\item Caracteristicas:

	\end{itemize}
	
\end{minipage}

% A1.12
\noindent\begin{minipage}{\textwidth}
	
	\subsection{\ch{[Mn(bpy)3]. 2 H2O.F3}}
	\begin{itemize}
   
   \item Índice de Coordenação:
   	6
   
   \item Número de Ligandos:
   	3	
	
   \item Ligandos:
   	3 bipiridinas \ch{ (C5H4N)2 }
	
   \item Átomos Doadores:
   	6 azotos ou nenhum
   
   \item Átomo Metálico Central:
   	Manganês \ch{Mn}
   
   \item Configuração Eletrónica:
   	[Ag]: $3d^5$
   
   \item Estado de Oxidação:
   	II+
   
   \item Contraião:
   	Não ha
   
   \item Nome:
   	tris-bipiridina\,Manganês(II) di\,hidratado fluorado
	
   \item Geometria:
   	octaedrica\vspace{0.5cm}\\
		\chemfig[angle increment=15]{%
			Mn?[Mn]%
				(
					-[6,1.3]N?[bpy1.1]
					=[10,0.7]
					-[6,0.7]
					=[2,0.7]
					-[-2,0.7]
					=[-6,0.7]?[bpy1.1C]
					?[bpy1.1]
				)%
				(
					-[1,1.3]N?[bpy1.2]
					=[5,0.7]?[bpy1.1C]
					-[1,0.7]
					=[-3,0.7]
					-[-7,0.7]
					=[-11,0.7]
					?[bpy1.2]
				)
				(
					-[-1,1.8]N?[bpy2.1]
					=[2,0.7]
					-[-1,0.7]
					=[-5,0.7]
					-[-9,0.7]
					=[11,0.9]?[bpy2.1C]
					?[bpy2.1]
				)
				(
					-[-6,1.3]N?[bpy2.2]
					=[-1,1]?[bpy2.1C]
					-[-5,0.7]
					=[-9,0.7]
					-[11,0.7]
					=[8,0.7]
					?[bpy2.2]
				)
				(
					-[-11,1.7]N?[bpy3.1]
					=[-11,0.7]
					-[12,0.7]
					=[11,0.7]
					-[1,0.7]
					=[0,0.8]?[bpy3.1C]
					?[bpy3.1]
				)
				(
					-[11,1.1]N?[bpy3.2]
					=[12,0.9]?[bpy3.1C]
					-[11,0.7]
					=[8,0.6]
					-[0,0.8]
					=[-1,0.8]
					?[bpy3.2]
				)
		}
   
   \item Isomeria:
   	de espelho destro e estro

	\item Caracteristicas:
		super estavel :)
		
	\end{itemize}
	
\end{minipage}
\paragraph{*}
bpy: bipiridina \ch{ (C5H4N)2 }

% A1.13
\noindent\begin{minipage}{\textwidth}
	
	\subsection{\ch{[Eu(fod)3]^{3+}.H2O}}
	\paragraph{fod =} \ch{OCC(CH3)3CHCOC3F7}
	
	\begin{itemize}
   
   \item Índice de Coordenação:
   	6
	
   \item Número de Ligandos:
   	3
	
   \item Ligandos:
   	acetil acetonado
   
   \item Átomos Doadores:
   	6 oxigênios
   
   \item Átomo Metálico Central:
   	Euforium (\ch{Eu})
   
   \item Configuração Eletrónica:
   	[Kr]: $5s^2\,4d^{10}\,5p^5\,4e^7$
   
   \item Estado de Oxidação:
   	III+
   
   \item Contraião:
   	não ha
   
   \item Nome:
   	hepta fuoro tri acetil acetonado euphorium(III) hidratado errado
   
   \item Geometria:
   	octaedrico
   
   \item Isomeria:
   	optico destro e lestro
		alem da kiralidade dos ligandos

	\item Caracteristicas:
		hidratado e estavel?
		3 aneis quelatos

	\end{itemize}
	
\end{minipage}

\paragraph{*}
fod: acetil acetonado \ch{OCC(CH3)3CHCOC3F7} (bidentado pelos oxigênios)

% A1.14
\noindent\begin{minipage}{\textwidth}
	
	\subsection{\ch{[Zn(en)2Br2]}}
	\begin{itemize}
   
   \item Índice de Coordenação: 6
   
   \item Número de Ligandos: 4
   
   \item Ligandos: 
   	2 etileno diamina (en) bidentado quelatos
		e dois brometos \ch{Br^-}
   
   \item Átomos Doadores:
   	dois bromos
   
   \item Átomo Metálico Central:
   	Zinco
   
   \item Configuração Eletrónica:
		[Ar]: $3d^9$
		   
   \item Estado de Oxidação:
   	II+
   
   \item Contraião:
   	não ha
   
   \item Nome:
   	dietileno\,diamin\,dibrometo\,zinco(II+)
   
   \item Geometria:
   	octaedrico
   
   \item Isomeria:
		optica cis e triz
		o cis tem kiralidade
		
	\item Caracteristicas:
		2 aneis quelatos

	\end{itemize}
	
\end{minipage}

% A1.15
\noindent\begin{minipage}{\textwidth}
	
	\subsection{\ch{[Pd(diaza - 18 - crown - 6 )]I2}}
	\begin{itemize}
   
   \item Índice de Coordenação:
   	6
   
   \item Número de Ligandos:
   	1
   
   \item Ligandos:
   	diaza - 18 - crown - 6 
   
   \item Átomos Doadores:
   	nenhum
   
   \item Átomo Metálico Central:
   	Chumbo
   
   \item Configuração Eletrónica:
   	[Xe]: $6s^2\,5d^{10}$	
	
   \item Estado de Oxidação:
   	II+
	
   \item Contraião:
   	2 Iodetos \ch{I^-}
   
   \item Nome:
   	Iodeto de diaza-18-crown-6\,paladio(II)
   
   \item Geometria:
   	octaedrico com os dois Iodetos a posição axial
   
   \item Isomeria:
   	não ha

	\item Caracteristicas:
		super estavel, um anel quelato que envolve o atomo central

	\end{itemize}
	
\end{minipage}

% A1.16
\noindent\begin{minipage}{\textwidth}
	
	\subsection{\ch{K5[Cu(ClO4)2(SCN)4]}}
	\begin{itemize}
   
   \item Índice de Coordenação:6
   
   \item Número de Ligandos:6
   
   \item Ligandos: 
   	2 ions perclorato 
		4 tioscianato bidentadios ambidentados
   
   \item Átomos Doadores:
   	2 oxigênios e 2 (enxofres ou azotos)
   
   \item Átomo Metálico Central:
   	Cobre
   
   \item Configuração Eletrónica:
   	[Ar]: $3d^{10}$
   
   \item Estado de Oxidação:
   	I+
   
   \item Contraião:
   	5 potácios
   
   \item Nome:
   	diperclorato\,tetratiocianato\,cobrico de potácio
   
   \item Geometria:
   	Octaedrico + 1
   	
   \item Isomeria:
   	Combinação de tiocianados em relação aos atomos ligantes
		alem de cis e trans em relação com a posição dos percloretos

	\item Caracteristicas:
		Metal pequeno e 7 ligações? bastante instavel presumo eu
		

	\end{itemize}
	
\end{minipage}

% A1.17
\noindent\begin{minipage}{\textwidth}
	
	\subsection{\ch{[Ru(Phen)3]Cl2}}
	\begin{itemize}
   
   \item Índice de Coordenação:
   	6
   
   \item Número de Ligandos:
   	3
   
   \item Ligandos:
   	3 fenatrolinas bidentados quelatos
   
   \item Átomos Doadores:
   	6 azotos
   
   \item Átomo Metálico Central:
   	Ruthenium 44
   
   \item Configuração Eletrónica:
   	[Kr]: $4d^6$
   
   \item Estado de Oxidação:
   	II+
   
   \item Contraião:
   	2 Cloretos \ch{Cl^-}
   
   \item Nome:
   	Cloreto de trifenatrolina\,ruthenium(II)
   
   \item Geometria:
   	octaedrico
   
   \item Isomeria:
		optica: levo e destro

	\item Caracteristicas:
		bastante grande e estavel por formar 3 fortes aneis quelatos com a fenatrolina

	\end{itemize}
	
\end{minipage}

\paragraph{*}
Phen: Fenatrolina \ch{C12H8N2} se liga pelos azotos Conhecido por formar fortes complexos, conhecido por emitir luz

% A1.18
\noindent\begin{minipage}{\textwidth}
	
	\subsection{\ch{[Cu(CNCH3)4]Br}}
	\begin{itemize}
   
   \item Índice de Coordenação:
   	4
   
   \item Número de Ligandos:
   	4
   
   \item Ligandos:
   	4 etilcianato
   
   \item Átomos Doadores:
   	4 azotos
   
   \item Átomo Metálico Central:
   	Cobre 27
   
   \item Configuração Eletrónica:
   	[Kr]: $3d^10$
   
   \item Estado de Oxidação:
   	I+
   
   \item Contraião:
   	Brometo \ch{Br^-}
   
   \item Nome:
   	brometo de tetraacetrolina\,cobre\,(I+)
   
   \item Geometria:
   	bipiramidal
   
   \item Isomeria:
   	não ha

	\item Caracteristicas:	
		5 ligantes em um atomo pequeno é instável, somando isso com a toxidade da acetrolina esse complexo é um veneno

	\end{itemize}
	
\end{minipage}

\paragraph{*}
\ch{CNCH3}: Acetrolina, altamente toxico

% A1.19
\noindent\begin{minipage}{\textwidth}
	
	\subsection{\ch{Na[V(H2O)3(NO3)3]}}
	\begin{itemize}
   
   \item Índice de Coordenação:
   	6
   
   \item Número de Ligandos:
   	6
   
   \item Ligandos:
   	3 aguas 3 nitratos
   
   \item Átomos Doadores:
   	3 oxigênios
   
   \item Átomo Metálico Central:
   	Vanadium V 23
   
   \item Configuração Eletrónica:
   	[Ar]: $ 3d1 $
   
   \item Estado de Oxidação:
   	IV+
   
   \item Contraião:
   	Sódio Na+
   
   \item Nome:
   	triaquo\,trinitrato\,vanadato de sódio
   
   \item Geometria:
   	octaedrico
   
   \item Isomeria:
		fac merid

	\item Caracteristicas:
		ta faltando um sódio?

	\end{itemize}
	
\end{minipage}

% A1.20
\noindent\begin{minipage}{\textwidth}
	
	\subsection{\ch{ K3[La2(Cl)6(Ph3P)4][Sm(NO3)6] }}
	\begin{itemize}
   
   \item Índice de Coordenação:
   	10 no primeiro 
		6 no segundo
   
   \item Número de Ligandos:
   	10 no primeiro 
		6 no segundo
   
   \item Ligandos:
   	6 cloretos e 4 trifenilfosfina no primeiro
		6 nitratos no segundo
   
   \item Átomos Doadores:
   	6 cloros \ch{Cl^-} no primeiro
		6 azotos \ch{N^-} no segundo
   
   \item Átomo Metálico Central:
   	2 latanium 57 no primeiro
		1 samarium 62 no segundo
   
   \item Configuração Eletrónica:\\
   	 \ch{[Kr]}: $ 5s^2\,4d^{10}\,5p^5\,4e^1 $
	 	 no primeiro\\
		 \ch{[Kr]}: $ 5s^2\,4d^{10}\,5p^5\,4e^6 $
		 no segundo
   
   \item Estado de Oxidação:
   	III+ no primeiro e segundo
   
   \item Contraião:
   	3 potácios e 1 no primeiro
		3 potácios e 1 no segundo
   
   \item Nome:
   	hexanitro\,samarato(3)
		de dis tetratrifenilfosfina\,di-$\mu$-clorp[tetraacloro]
		de potácio
   
   \item Geometria:
   
   \item Isomeria:

	\item Caracteristicas:

	\end{itemize}
	
\end{minipage}

\paragraph{*}
Ph: fenil C6H5

% A1.21
\noindent\begin{minipage}{\textwidth}
	
	\subsection{\ch{[Cu(NH3)4][PtCl4]}}
	\begin{itemize}
   
   \item Índice de Coordenação:
   
   \item Número de Ligandos:
   
   \item Ligandos:
   
   \item Átomos Doadores:
   
   \item Átomo Metálico Central:
   
   \item Configuração Eletrónica:
   
   \item Estado de Oxidação:
   
   \item Contraião:
   
   \item Nome:
   
   \item Geometria:
   
   \item Isomeria:

	\item Caracteristicas:

	\end{itemize}
	
\end{minipage}


% A1.22
\noindent\begin{minipage}{\textwidth}
	
	\subsection{\ch{ [Ag(tu)2]I }}
	\begin{itemize}
   
   \item Índice de Coordenação:
   
   \item Número de Ligandos:
   
   \item Ligandos:
   
   \item Átomos Doadores:
   
   \item Átomo Metálico Central:
   
   \item Configuração Eletrónica:
   
   \item Estado de Oxidação:
   
   \item Contraião:
   
   \item Nome:
   
   \item Geometria:
   
   \item Isomeria:

	\item Caracteristicas:

	\end{itemize}
	
\end{minipage}


\paragraph{*}
tu: Tiureia \ch{CH4N2S}

% A1.23
\noindent\begin{minipage}{\textwidth}
	
	\subsection{\ch{ [Pt(dmg)2]^{2+} }}
	\begin{itemize}
   
   \item Índice de Coordenação:
   
   \item Número de Ligandos:
   
   \item Ligandos:
   
   \item Átomos Doadores:
   
   \item Átomo Metálico Central:
   
   \item Configuração Eletrónica:
   
   \item Estado de Oxidação:
   
   \item Contraião:
   
   \item Nome:
   
   \item Geometria:
   
   \item Isomeria:

	\item Caracteristicas:

	\end{itemize}
	
\end{minipage}

\paragraph{*}
dmg: Dimetilglioxima \ch{CH3C(NOH)C(NOH)CH3}, importante para retirar metais, liga ou pelos oxigenios ou azotos, bidentado ambidentado

% A1.24
\noindent\begin{minipage}{\textwidth}
	
	\subsection{\ch{ [Co(acac)3]Br3 }}
	\begin{itemize}
   
   \item Índice de Coordenação:
   
   \item Número de Ligandos:
   
   \item Ligandos:
   
   \item Átomos Doadores:
   
   \item Átomo Metálico Central:
   
   \item Configuração Eletrónica:
   
   \item Estado de Oxidação:
   
   \item Contraião:
   
   \item Nome:
   
   \item Geometria:
   
   \item Isomeria:

	\item Caracteristicas:

	\end{itemize}
	
\end{minipage}

% A1.21
\noindent\begin{minipage}{\textwidth}
	
	\subsection{\ch{ K[Au(CN)2] }}
	\begin{itemize}
   
   \item Índice de Coordenação:
   
   \item Número de Ligandos:
   
   \item Ligandos:
   
   \item Átomos Doadores:
   
   \item Átomo Metálico Central:
   
   \item Configuração Eletrónica:
   
   \item Estado de Oxidação:
   
   \item Contraião:
   
   \item Nome:
   
   \item Geometria:
   
   \item Isomeria:

	\item Caracteristicas:

	\end{itemize}
	
\end{minipage}

% A2
\section{}

\end{document}

















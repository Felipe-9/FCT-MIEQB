\documentclass[12pt]{article}

% Linguagem
\usepackage[portuguese]{babel}

% Clickable Table of contents
\usepackage{hyperref}
\hypersetup{
	hidelinks=true,
	colorlinks=true,
	linkcolor=DarkGreen!20!LightGreen!25
}

% Table of contents
\usepackage{tocloft}
\setlength{\cftsubsecnumwidth}{3em} % Fix subsection width
% Fix space between subsection items on toc
\renewcommand\cftsubsecafterpnum{\vskip5pt}
 
% Multicols
%\usepackage{multicol}

% Maths
%\usepackage{amssymb} 		
%\usepackage{amsmath} 
%\usepackage[utf8]{inputenc} %useful to type directly diacritic characters

%\newcommand{\bm}[1]{{\boldmath{\large{\begin{align*} #1 \end{align*}}}}}

% Vectors
%\usepackage{esvect} 	% Vector over-arrow
%\usepackage{tikz}		% Vector diagrams
%\usetikzlibrary{calc}  % Vector calculations
%\usepackage{varwidth}  % List inside TikzPicture

%\renewcommand{\vec}{\vv} % Vecto over-arrow

% Chem
\usepackage{chemformula} 	% formulas quimicas
\usepackage{chemfig} 		% Estruturas quimicas

% Colors
\usepackage{xcolor}

\definecolor{DarkBlue}	{HTML}{252A36}
\definecolor{LightGreen}{HTML}{7CCC6C}
\definecolor{DarkGreen}	{HTML}{008675}

\pagecolor{DarkBlue!110!}
\color{DarkGreen!20!}

% Counters
\counterwithin*{section}{part} % Reset section on part

% Resolução de listas
\renewcommand\thesection{A\,\Roman{section}}
\renewcommand\thesubsection{\thesection.\arabic{subsection}}

\begin{document}

\title{ \bfseries\color{DarkGreen!75}
	  QI I 
	- Ficha III
	\\Complexos Metálicos
	\\Tentativa 2
}
\author{Felipe Pinto - 61387}

\maketitle
\tableofcontents
\break

\section{}

% A1.1

	\subsection{\ch{[CrCl2(en)2]Cl}}
	\begin{itemize}
   
   \item Índice de Coordenação: 6: 2 Cloros e 4 Azotos
   
   \item Ligandos: 4:
   	2 Cloretos \ch{Cl^-}
	 	e 2 etilenodiaminos (en) bidentadas
   	
%   \item Átomos Doadores: 6: 2 Cloros e 4 Azotos
   
   \item Átomo Metálico Central: Cromo Cr 24
   
   \item Configuração Eletrónica:
   	[Ar]: $ 3d^3 $
   
   \item Estado de Oxidação: III+
   
   \item Contraião: Cloreto \ch{Cl^-}
   
   \item Nome: 
   	cloreto de bis\,etilen\,diamin\,dicloro\,cromio(III)
   
   \item Geometria: Octaédrica
   
   \item Isomeria:\\
   	geometrica (cis-trans),\\
		quiral / optica (levo-destro): no isomero cis

	\item Caracteristicas: 
		2 anéis quelatos garantindo moderada estabilidade

	\end{itemize}

\break

% A1.2
	
	\subsection{\ch{[Co(ONO)(NH3)5]SO4}}
	\begin{itemize}
   
   \item Índice de Coordenação: 6:
   	1 oxigênio \ch{O^-} e
		5 azotos \ch{N}
   
%   \item Número de Ligandos:
   
   \item Ligandos: 6:
   	1 Nitrito \ch{ONO^-} e
		5 amins \ch{NH3}

   
%   \item Átomos Doadores:
   
   \item Átomo Metálico Central: Cobalto Co 27
   
   \item Configuração Eletrónica: [Ar]: $3d6$
   
   \item Estado de Oxidação: III+
   
   \item Contraião: Sulfato \ch{SO4^{2-}}
   
   \item Nome: Sulfato de Pentamin\,nitrito\,cobalto(III)
   
   \item Geometria: Octaédrica
   
   \item Isomeria: Não ha

	\item Caracteristicas:
%		Instável pelo tamanho do átomo central e numero de ligandos

	\end{itemize}

\break

% A1.3

	
	\subsection{\ch{[Cr(H2O)4Cl2]Cl}}
	\begin{itemize}
   
%   \item Número de Ligandos:
   
   \item Ligandos: 6:
   	2 Cloretos \ch{Cl^-} e
		4 Aquos \ch{H2O}
   
%   \item Átomos Doadores:
   
   \item Índice de Coordenação: 6:
   	2 Cloros Cl e
		4 Oxigênios O
   
   \item Átomo Metálico Central: Cromo Cr 24
   
   \item Configuração Eletrónica: [Ar]: $ 3d^3 $
   
   \item Estado de Oxidação: III+
   
   \item Contraião: Cloreto \ch{Cl^-}
   
   \item Nome: Cloreto de Tetraquo\,dicloro\,cromio(III)
   
   \item Geometria: Octaédrica
   
   \item Isomeria:
   	Geométrica (cis-trans) com um ou dois cloretos na posição axial;

	\item Caracteristicas:
		Átomo central pequeno com 6 ligandos caracterizando uma moderada instabilidade

	\end{itemize}
	
\break

% A1.4
	
	\subsection{\ch{Na3[Co(NO2)6]}}
	\begin{itemize}
   
%   \item Número de Ligandos:
   
   \item Ligandos: 6: 6 Nitros \ch{NO2^-}
   
%   \item Índice de Coordenação:
   
   \item Átomos Doadores: 6: 6 Azotos N
   
   \item Átomo Metálico Central: Cobalto Co 27
   
   \item Configuração Eletrónica: [Ar]: $ 3d^6 $
   
   \item Estado de Oxidação: III+
   
   \item Contraião: 3 Sódios \ch{Na^+}
   
   \item Nome: Hexanitro\,cobaltato de sódio
   
   \item Geometria: Octaédrica
   
   \item Isomeria:\\
   	De ligação pelos ligandos nitros que são ambidentados,
		seus isomeros tem forma \ch{[Co(NO2)_i(ONO)_j]^{3+}} onde 		$i+j=6$\\
		Geométrica (cis-trans) nos isomeros \{i=2, j=4\} e
		\{i=4, j=2\}\\
		Geométrica (fac-mer) nos isomeros \{i=3, j=3\}
		

	\item Caracteristicas:\\
		leve instabilidade por ter 6 ligantes em um atomo metálico central pequeno;\\
		Complexo Aniônico

	\end{itemize}
	
\break

% A1.5
	
	\subsection{\ch{[(NH3)5-Cr-OH-Cr(NH3)5]Cl5}}
	\begin{itemize}
   
%   \item Número de Ligandos:
   
   \item Ligandos: 6:
   	5 amins \ch{NH3} e uma hidroxido \ch{OH^-}
   
   \item Átomos Doadores: 6:
   	5 azotos e
		1 oxigênio
   
%   \item Índice de Coordenação:
      
   \item Átomo Metálico Central: Ambos Cromos 24
   
   \item Configuração Eletrónica: Ambos [Ar]: $ 3d^3 $
   
   \item Estado de Oxidação: Ambos III+
   
   \item Contraião: 5 Cloretos \ch{Cl^-}
   
   \item Nome:
   	Pentacloreto $\mu$-hidroxido-bis[pentamin\,cromo(III)]
   
   \item Geometria: Ambos Octaédrica
   
   \item Isomeria: Não ha

	\item Caracteristicas: Complexo polinuclear

	\end{itemize}
	
\break

% A1.6
	
	\subsection{\ch{[Co(NH3)2(H2O)]Br2}}
	\begin{itemize}
   
%   \item Número de Ligandos:
   
   \item Ligandos: 3: 1 Aquos \ch{H2O} e 2 Amin
   
%   \item Índice de Coordenação:
   
   \item Átomos Doadores: 3: 1 Oxigênios O e 2 Azoto N
   
   \item Átomo Metálico Central: Cobalto Co 27
   
   \item Configuração Eletrónica: [Ar]: $ 3d^7 $
   
   \item Estado de Oxidação: II+
   
   \item Contraião: 2 Brometos \ch{Br^-}
   
   \item Nome: Brometo de Diamin\,aquo\,cobalto(II)
   
   \item Geometria: Trigonal
   
   \item Isomeria: Não ha

	\item Caracteristicas: Estado de oxidação do Co é igual a carga do complexo pois seus ligandos são todos moleculares

	\end{itemize}
	
\break

% A1.7

	
	\subsection{\ch{[Co(NH3)5Cl]Cl2}}
	\begin{itemize}
   
%   \item Número de Ligandos:
   
   \item Ligandos: 6:
   	5 Amins \ch{NH3} e 1 Cloreto \ch{Cl^-}
   
%   \item Índice de Coordenação:
   
   \item Átomos Doadores:6:
   	5 Azotos N e 1 Cloro Cl
   
   \item Átomo Metálico Central: Cobalto Co 27
   
   \item Configuração Eletrónica: [Ar]: $ 3d^6 $
   
   \item Estado de Oxidação: III+
   
   \item Contraião: 2 Cloretos \ch{Cl^-}
   
   \item Nome: Cloreto de Pentamin\,cloro\,cobalto(III)
   
   \item Geometria: Octaédrica
   
   \item Isomeria: Não ha

	\item Caracteristicas:

	\end{itemize}
	
\break

% A1.8

	
	\subsection{\ch{K[PtCl3(C2H4)]}}
	\begin{itemize}
   
%   \item Número de Ligandos:
   
   \item Ligandos: 4: 3 Cloretos \ch{Cl^-} e 1 Etileno \ch{C2H4}
   
%   \item Índice de Coordenação:
   
   \item Átomos Doadores: 4: 3 Cloros Cl e a ligação dupla do Etileno
   
   \item Átomo Metálico Central: Platina Pt 78
   
   \item Configuração Eletrónica:
   	[Xe]: $ 4f^{14}\,5d^8 $
   
   \item Estado de Oxidação: II+
   
   \item Contraião: Potássio \ch{K^+}
   
   \item Nome: Etilene\,tricloro\,platinato(II) de Potássio
   
   \item Geometria: Tetragonal*
   
   \item Isomeria: Não há

	\item Caracteristicas:\\
		O ligando Etileno está ligado ao átomo central pela sua ligação dupla\\
		Geometria tetragonal por possuir config eletronica $d^8$, é um complexo tenso e de geometria pouco comum

	\end{itemize}
	
\break

% A1.9

	
	\subsection{\ch{[Co(NH3)4(NCS)Cl]NO3}}
	\begin{itemize}
   
%   \item Número de Ligandos:
   
   \item Ligandos: 6:
   	4 Aminos \ch{NH3},
		1 isotiocianato \ch{NCS^-} e 
		1 cloreto \ch{Cl^-}
   
%   \item Índice de Coordenação:
   
   \item Átomos Doadores: 6:
   	5 Azotos N e
		1 Cloro Cl
   
   \item Átomo Metálico Central: Cobalto Co 27
   
   \item Configuração Eletrónica: [Ar]: $ 3d^6 $
   
   \item Estado de Oxidação: III+
   
   \item Contraião: Nitrato \ch{NO3^-}
   
   \item Nome: 
   	Nitrato de Tetramin\,cloro\,-N-isotiocianato\,cobalto(III)
   
   \item Geometria: Octaédrica
   
   \item Isomeria:\\
   	de Ligação pelo isotiocianato ambidentado,\\
		Geométrica (Cis-Trans) com a posição dos Ligandos diferentes dos aminos em ambos ou apenas um eixo axial\\
		Iônica pela troca do cloreto pelo nitrato

	\item Caracteristicas:

	\end{itemize}
	
\break

% A1.10

	
	\subsection{\ch{[Pt(en)Cl4]}}
	\begin{itemize}
   
%   \item Número de Ligandos:
   
   \item Ligandos: 5:
   	1 etilenodiamino en e 
		4 cloretos \ch{Cl^-}
   
%   \item Índice de Coordenação:
   
   \item Átomos Doadores: 6:
   	2 Azotos N e
		4 Cloros Cl
   
   \item Átomo Metálico Central: Platina Pt 78
   
   \item Configuração Eletrónica: 
   	[Xe]: $ 4f^{14}\,3d^6 $
   
   \item Estado de Oxidação: IV+
   
   \item Contraião: Não há
   
   \item Nome: Etilenodiamin\,tetracloro\,platina(IV)
   
   \item Geometria: Octaédrica
   
   \item Isomeria: Não ha

	\item Caracteristicas: Possui um anel quelato

	\end{itemize}
	
\break

% A1.11

	\subsection{\ch{Na2[Ni(EDTA)]}}
	\begin{itemize}
   
%   \item Número de Ligandos:
   
   \item Ligandos: 1 Hexadentado Etileno\,diamino\,tetracetato
   
%   \item Índice de Coordenação:
   
   \item Átomos Doadores: 6: 4 Oxigênios e 2 Azotos
   
   \item Átomo Metálico Central: Nikel Ni 28
   
   \item Configuração Eletrónica: [Ar]: $ 3d^8 $
   
   \item Estado de Oxidação: II+
   
   \item Contraião: 2 Sódios \ch{Na^+}
   
   \item Nome: Etileno\,diamino\,tetracetato\,nikelato(II) de Sódio
   
   \item Geometria: Octaédrico
   
   \item Isomeria: Óptica (levo-dextro)

	\item Caracteristicas: EDTA engloba o elemento central, ocupando todas as ligações feitas com ele

	\end{itemize}
	
	\paragraph{EDTA} Etileno\,diamino\,tetracetato
	
\break

% A1.12

	
	\subsection{\ch{Li[Mn(bpy)3]. 2 H2O}}
	\begin{itemize}
   
%   \item Número de Ligandos:
	
   \item Ligandos: 3 bipiridinas bpy
	
%   \item Índice de Coordenação:
   
   \item Átomos Doadores: 6 azotos N
   
   \item Átomo Metálico Central: Manganês Mn 25
   
   \item Configuração Eletrónica: [Ar]: $ 4s^2\,3d^6 $
   
   \item Estado de Oxidação: I-
   
   \item Contraião: Litio
   
   \item Nome: trixbipiridino\,manganato(I-)\,dihidratado de litio
	
   \item Geometria: Octaédrica
   
   \item Isomeria: Optica (levo-destro)

	\item Características:
		Complexo impossível de ser aniônico pois só possui 			ligandos moleculares, se existisse:\\
		seria bastante estável por possuir 3 anéis quelatos,\\
		estaria hidratada como indicado pelas moleculas de agua implicando que sua massa diminui quando aquecido.
			
	\end{itemize}
	
\paragraph{bpy:} bipiridino \ch{ (C5H4N)2 }

\break

% A1.13

	
	\subsection{\ch{[Eu(fod)3]^{3+}.H2O}}
	\paragraph{fod =} \ch{OCC(CH3)3CHCOC3F7}
	
	\begin{itemize}
   
%   \item Número de Ligandos:
	
   \item Ligandos: 
   	3 bidentados Heptafluoro\,dimetil\,octanedionato
   
%   \item Índice de Coordenação:
	
   \item Átomos Doadores: 6 Oxigênios
   
   \item Átomo Metálico Central: Europium Eu 63
   
   \item Configuração Eletrónica: 
   	[Kr]: $ 5s^2\,4d^{10}\,5p^5\,4f^7 $
   
   \item Estado de Oxidação: III+
   
   \item Contraião: Não há
   
   \item Nome:
   	Heptafluoro\,dimetil\,octanedionato\,europio(III)
	 	Hidratado
   
   \item Geometria: Octaédrico
   
   \item Isomeria: Óptica (levo-Destro)

	\item Caracteristicas:

	\end{itemize}
		
\paragraph{*}
fod: acetil acetonado \ch{OCC(CH3)3CHCOC3F7} (bidentado pelos oxigênios)
		
\break

% A1.14
	
	\subsection{\ch{[Zn(en)2Br2]}}
	\begin{itemize}
   
%   \item Número de Ligandos:
   
   \item Ligandos: 4: 
   	2 bidentado etileno\,diamina en e
		2 Brometos \ch{Br^-}
   
%   \item Índice de Coordenação:
   
   \item Átomos Doadores: 6:
   	4 Azotos N e
		2 Bromos Br
   
   \item Átomo Metálico Central: Zinco Zn 30
   
   \item Configuração Eletrónica: [Ar]: $ 3d^{10} $
		   
   \item Estado de Oxidação: II+
   
   \item Contraião: Nenhum
   
   \item Nome: bisetileno\,diamina\,dibromo\,zinco(II)
   
   \item Geometria: Octaédrico
   
   \item Isomeria:\\
   	Geometrica (axial-meridional),\\
		Óptica (levo-destro) no isomero Meridional
		
	\item Caracteristicas:

	\end{itemize}
	
\break

% A1.15

	
	\subsection{\ch{[Pd(diaza - 18 - crown - 6 )]I2}}
	\begin{itemize}
   
%   \item Número de Ligandos:
   
   \item Ligandos: 1 hexadentado diaza-18-crown-6
   
%   \item Índice de Coordenação:
   
   \item Átomos Doadores: 6: 4 Oxigênios O e 2 azotos N
   
   \item Átomo Metálico Central: Paladio 46
   
   \item Configuração Eletrónica: 
   	[Kr]: $ 4d^{8} $
	
   \item Estado de Oxidação: II+
	
   \item Contraião: 2 Iodetos \ch{I^-}
   
   \item Nome: Iodeto de diaza-18-crown-6\,paladio(II)
   
   \item Geometria: Octaédrica
   
   \item Isomeria: Óptica (levo-destro) pela posição dos azotos

	\item Caracteristicas: O ligando forma um anél ao redor do elemento central, permitindo grande estabilidade

	\end{itemize}
	
\break

% A1.16
	
	\subsection{\ch{K5[Cu(ClO4)2(SCN)4]}}
	\begin{itemize}
   
%   \item Número de Ligandos:
   
   \item Ligandos: 6: 2 percloratos e 4 ambidentados tiocianatos
   
%   \item Índice de Coordenação:
   
   \item Átomos Doadores: 6: 2 Oxigênios e 4 Enxofres
   
   \item Átomo Metálico Central: Cobre Cu 29
   
   \item Configuração Eletrónica: [Ar]: $ 3d^10 $
   
   \item Estado de Oxidação: I+
   
   \item Contraião: 5 Potássios \ch{K^+}
   
   \item Nome: diperclorato\,tetratiocianato\,cobrico(I) de potácio
   
   \item Geometria: Octaédrica
   	
   \item Isomeria:\\
   	De ligação pelos ligandos ambidentados tiocianatos da forma \ch{[Cu(ClO4)2(SCN)_i(NCS)_j]} onde $i+j=4$\\
		Geométrica (cis-trans) pela posição dos percloratos\\
		Geométrica (cis-trans) nos isomeros \{i=2,j=2\} pela posição dos tiocianatos\\
		Geométrica (fac-mer) nos isomeros \{i=1, j=3\} e \{i=3, j=1\} agrupando os 3 tiocianatos/isotiocianatos.

	\item Caracteristicas:

	\end{itemize}
	
\break

% A1.17

	
	\subsection{\ch{[Ru(Phen)3]Cl2}}
	\begin{itemize}
   
%   \item Número de Ligandos:
   
   \item Ligandos: 3 fenatrolinas Phen
   
%   \item Índice de Coordenação:
   
   \item Átomos Doadores: 6 azotos N
   
   \item Átomo Metálico Central: Rutenio Ru 44
   
   \item Configuração Eletrónica: [Kr]: $ 3d^6 $
   
   \item Estado de Oxidação: II+
   
   \item Contraião: 2 Cloretos \ch{Cl^-}
   
   \item Nome: Cloreto de trisfenatrolina\,rutenio(II)
   
   \item Geometria: Octaédrica
   
   \item Isomeria: Óptica (levo-destro)

	\item Caracteristicas: Emite luz

	\end{itemize}
	
\paragraph{Phen:}
Fenatrolina \ch{C12H8N2} se liga pelos azotos Conhecido por formar fortes complexos e por emitir luz
	
\break

% A1.18

	\subsection{\ch{[Cu(CNCH3)4]Br}}
	\begin{itemize}
   
%   \item Número de Ligandos:
   
   \item Ligandos: 4 Etanonitrilas
   
%   \item Índice de Coordenação:
   
   \item Átomos Doadores: 4 Azotos N
   
   \item Átomo Metálico Central: Cobre Cu 29
   
   \item Configuração Eletrónica: [Ar]: $ 3d^{10} $
   
   \item Estado de Oxidação: I+
    
   \item Contraião: Brometo \ch{Br^-}
   
   \item Nome: Brometo de tetraetanonitril\,cobre(I)
   
   \item Geometria: Tetraédrica
   
   \item Isomeria: Não há

	\item Caracteristicas: Seus ligandos são Etanonitrilas, substancia toxica, este composto complexo é um veneno
	
	\end{itemize}

\paragraph{\ch{CNCH3}:} Etanonitrila, altamente toxico
	
\break

% A1.19

	
	\subsection{\ch{Na[V(H2O)3(NO3)3]}}
	\begin{itemize}
   
%   \item Número de Ligandos:
   
   \item Ligandos: 6: 3 Aquos \ch{H2O} e 3 Nitros \ch{NO3}
   
%   \item Índice de Coordenação: 
   
   \item Átomos Doadores: 6: 3 Oxigênios O e 3 Azotos N
   
   \item Átomo Metálico Central: Vanadio V 23
   
   \item Configuração Eletrónica: [Ar]: $ 3d^3 $
   
   \item Estado de Oxidação: II+
   
   \item Contraião: Sódio \ch{Na^+}
   
   \item Nome: Triaquo\,trinitro\,vanadeto(II) de sódio
   
   \item Geometria: Octaédrico
   
   \item Isomeria:\\
   	de Ligação pelo ligante ambidentado Nitro \ch{NO2} e Nitrila \ch{ONO} de formula \ch{[V(H2O)3(NO3)_i(ONO)_j]^{1-}} onde $i+j=3$\\
		Geométrica (fac-mer) nos isomeros $\{i=3, j=0\}$ e $\{i=0, j=3\}$\\
		Geométrica (cis-trans) nos isomeros $\{i=2,j=1\}$ e $\{i=1,j=2\}$

	\item Caracteristicas:

	\end{itemize}
	
\break

\setcounter{subsection}{18}

% A1.19

	\subsection{\ch{ [V(NO3)4]I3 }}
	\begin{itemize}
   
%   \item Número de Ligandos:
   
   \item Ligandos: 4 Nitros \ch{NO3}
   
%   \item Índice de Coordenação: 
   
   \item Átomos Doadores: 4 Azotos N
   
   \item Átomo Metálico Central: Vanadio V 23
   
   \item Estado de Oxidação: VII+
   
   \item Configuração Eletrónica: 
   	[Ne]: $ 3s^2\,3p^4 $
   
   \item Contraião: Sódio \ch{Na^+}
   
   \item Nome: Iodeto de Trinitro\,vanadio(VII)
   
   \item Geometria: Tetraédrico
   
   \item Isomeria:\\
   	de Ligação pelo ligante ambidentado Nitro \ch{NO2} e Nitrila \ch{ONO} de formula \ch{[V(NO3)_i(ONO)_j]^{3+}} onde $i+j=4$

	\item Caracteristicas: Complexo não existe, maior grau de oxidação do vanadio é V

	\end{itemize}
	
\break


% A1.20

	\subsection{\ch{ K3[La2(Cl)6(Ph3P)4][Sm(NO3)6] }}
	\begin{itemize}
   
%   \item Número de Ligandos:
   
   \item Ligandos:\\
   	10: 6 Cloretos \ch{Cl^-} e 4 trifenil\,fosfina \ch{Ph3P}\\
		6 Nitros \ch{NO3}
   
%   \item Índice de Coordenação:
   
   \item Átomos Doadores:\\
   	10: 6 Cloros Cl e 4 Fosforos P\\
		6 Azotos N
   
   \item Átomo Metálico Central:
   	Latanio La 57; 
		Samário Sm 62

   \item Estado de Oxidação:
   	III+;
		III+
   
   \item Configuração Eletrónica:
   	\ch{[Xe]};
   	\ch{[Xe]}: $ 4f^5 $
   
   \item Contraião:
   
   \item Nome: 
   	Hexanitro\,samarato(III) 
		dis-$\mu$-cloro-dis[tetracloro\,tetrixtrifenilfosfina\,Latano(III)]
		De potácio
   
   \item Geometria: 
   	Octaédrica em ambos núcleos;
		Octaédrica
   
   \item Isomeria:
   	Não há em nenhum

	\item Caracteristicas:
		Não há isomeria no primeiro por possuir dois ligantes trifenilfosfia que ocupam um espaço muito grande assim sendo limitado a posição axial;\\
		A formação de ambos os complexos libera energia em forma de luz

	\end{itemize}
	
\paragraph{ph:} fenil C6H5
	
\break

% A1.21

	
	\subsection{\ch{[Cu(NH3)4][PtCl4]}}
	\begin{itemize}
   
   \item Índice de Coordenação:
   
   \item Número de Ligandos:
   
   \item Ligandos:
   
   \item Átomos Doadores:
   
   \item Átomo Metálico Central:
   
   \item Configuração Eletrónica:
   
   \item Estado de Oxidação:
   
   \item Contraião:
   
   \item Nome:
   
   \item Geometria:
   
   \item Isomeria:

	\item Caracteristicas:

	\end{itemize}
	
\break


% A1.22

	
	\subsection{\ch{ [Ag(tu)2]I }}
	\begin{itemize}
   
   \item Índice de Coordenação:
   
   \item Número de Ligandos:
   
   \item Ligandos:
   
   \item Átomos Doadores:
   
   \item Átomo Metálico Central:
   
   \item Configuração Eletrónica:
   
   \item Estado de Oxidação:
   
   \item Contraião:
   
   \item Nome:
   
   \item Geometria:
   
   \item Isomeria:

	\item Caracteristicas:

	\end{itemize}
	
\break


\paragraph{*}
tu: Tiureia \ch{CH4N2S}

% A1.23

	
	\subsection{\ch{ [Pt(dmg)2]^{2+} }}
	\begin{itemize}
   
   \item Índice de Coordenação:
   
   \item Número de Ligandos:
   
   \item Ligandos:
   
   \item Átomos Doadores:
   
   \item Átomo Metálico Central:
   
   \item Configuração Eletrónica:
   
   \item Estado de Oxidação:
   
   \item Contraião:
   
   \item Nome:
   
   \item Geometria:
   
   \item Isomeria:

	\item Caracteristicas:

	\end{itemize}
	
\break

\paragraph{*}
dmg: Dimetilglioxima \ch{CH3C(NOH)C(NOH)CH3}, importante para retirar metais, liga ou pelos oxigenios ou azotos, bidentado ambidentado

% A1.24

	
	\subsection{\ch{ [Co(acac)3]Br3 }}
	\begin{itemize}
   
   \item Índice de Coordenação:
   
   \item Número de Ligandos:
   
   \item Ligandos:
   
   \item Átomos Doadores:
   
   \item Átomo Metálico Central:
   
   \item Configuração Eletrónica:
   
   \item Estado de Oxidação:
   
   \item Contraião:
   
   \item Nome:
   
   \item Geometria:
   
   \item Isomeria:

	\item Caracteristicas:

	\end{itemize}
	
\break

% A1.21

	
	\subsection{\ch{ K[Au(CN)2] }}
	\begin{itemize}
   
   \item Índice de Coordenação:
   
   \item Número de Ligandos:
   
   \item Ligandos:
   
   \item Átomos Doadores:
   
   \item Átomo Metálico Central:
   
   \item Configuração Eletrónica:
   
   \item Estado de Oxidação:
   
   \item Contraião:
   
   \item Nome:
   
   \item Geometria:
   
   \item Isomeria:

	\item Caracteristicas:

	\end{itemize}
	
\break

% A2
\section{}




















\end{document}












\documentclass[12pt]{article}

\usepackage{chemformula} % formulas quimicas
\usepackage{chemfig} % Estruturas quimicas

\usepackage{color}
\pagecolor{black}
\color{white}

\begin{document}

\title{QI 1 - Química Inorgânica 1 - PO}
\maketitle
\tableofcontents

\paragraph{Quartetos: }
cada familia de metais

\section{Grupo 12: Metais pós transacionais}
os elementos do grupo 12 dos metais possuem ultimo orbital completo, então possuem caracteristicas unicas, assim considerados um grupo separado

\section{Metais do Bloco D: Grupo dos Metais Pesados}
4 primeiros grupos
\subsection{caracteristicas}
Estruturas de cristais, com exceção do mercurio \\
sempre mais pequenos que os equivalentes do grupo seguinte\\
tamanhos variam pouco\\ 
diminuem ao longo dos lantanios\\
Pouca Reatividade\\

22, 29 sabados

\section{Complexos}
\subsection{Caracteristicas}
\begin{itemize}
\item Elemento(s) Centrais
\item Ligandos $\to$ Elementos Ligados ao atomo Central
\item Numero de Coordenação
\item Esfera de Coordenação primária
\end{itemize}

\subsection{Para se Considerar um complexo}
O numero de ligações simples que se forma nos ligandos tem que ser superior ao estado de oxidação do elemento central

\subsection{Ligandos}
\paragraph{se caracterisa por numero de ligaçõe alem da que faz com o atomo central}
\begin{itemize}
\item Monodentado
\item Bidentados
\item Polidentado
\item Ambidentado: Pode ter mais de uma ligação mas não a está fazendo no momento
\end{itemize}

\subsection{Atomos que podem se coordenar diretamente co os átomos oi iões metalicos}
\begin{itemize}
\item mon-negativos: Hidretos e Halogênios
\item di-negativos: \ch{O^{-2},\ S^{-2},\ Se^{-2},\ Te^{-2}}
\end{itemize}



\section{Grupo da Platina}

\section{Metais de Transição}



\section{Random}

\subsection{ligandos polidentados}
possuem pouco espaco

\subsection{Pridina \ch{Pi}}
Anel aromatico com um nitrogenio no lugar do carbono

\subsection{EN: Etileno Diamina \ch{NH2-CH2-CH2-NH2}} 
Se liga pelos N

\subsection{\ch{CO, CN}}
São monodentados e ambidentados, pois ambos os atomos podem se coordenar com o átomo metalico central

\subsection{Quelatos: Aneis ao redor do atomo principal}

\end{document}

\documentclass[12pt]{article}

% Linguagem
\usepackage[portuguese]{babel}

% Clickable Table of contents
\usepackage{hyperref}
\hypersetup{
	hidelinks=true,
	colorlinks=true,
	linkcolor=DarkGreen!20!LightGreen!25!
}

% Table of contents
%\usepackage{tocloft}
%\setlength{\cftsubsecnumwidth}{3em} % Fix subsection width
% Fix space between subsection items on toc
%\renewcommand\cftsubsecafterpnum{\vskip5pt}

% Multicols
%\usepackage{multicol}

% Customize Chapter
%\usepackage{titlesec}
%\titleformat{\chapter}[hang]{\Huge\bfseries\color{DarkGreen!75!}}{\thechapter\hspace{20pt}{$|$}\hspace{20pt}}{0pt}{\Huge\bfseries}

% Appendix
%\usepackage{appendix}

% Maths
%\usepackage{amssymb} 		
%\usepackage{amsmath} 
%\usepackage[utf8]{inputenc} %useful to type directly diacritic characters

%\newcommand{\bm}[1]{{\boldmath{\large{\begin{align*} #1 \end{align*}}}}}

% Vectors
%\usepackage{esvect} 	% Vector over-arrow
%\renewcommand{\vec}{\vv} % Vecto over-arrow

% Tikz
%\usepackage{tikz}		
%\usepackage{pgfmath}  	% calculations
%\usepackage{varwidth}  % List inside TikzPicture

% Chem
\usepackage{chemformula} 	% formulas quimicas
%\usepackage{chemfig} 		% Estruturas quimicas

\newcommand{\mol}[2][]{ \text{mol}_{\ch{ #2 }\,#1} } % mol

% Tabular
%\usepackage{multirow}
%\usepackage{siunitx} % Column S: align on decimal


% Colors
\usepackage{xcolor}

\definecolor{DarkBlue}	{HTML}{252A36}
\definecolor{LightGreen}{HTML}{7CCC6C}
\definecolor{DarkGreen}	{HTML}{008675}

\colorlet{White}{DarkGreen!20!}
\colorlet{Black}{DarkBlue!110!}

\pagecolor{Black}
\color{White}

%\definecolor{Red}  {HTML}{FF7E79}
%\definecolor{Blue} {HTML}{6666FF}
%\definecolor{Green}{HTML}{66FF66}

% Counters
%\counterwithin*{section}{part} % Reset section on part

% Section and Subsection Customization
%\renewcommand\thesection{Questão \arabic{section} }
%\renewcommand\thesubsection{\arabic{section}-\alph{subsection}) }

\begin{document}

\title{\bfseries\color{DarkGreen!75!}%
	QI 1 - Ligação em compostos de ligação%
}
\author{Felipe Pinto - 61387}
\date{22/04 - 2021.1}

\maketitle
\tableofcontents
\break



\section{Teoria de campo ligando}
Limitada\\
\subsection{Valencia primaria e secundária}
Estado de oxidação e Numero de coordenação

\section{Teoria campo cristalino}
Metais de transição tem 5 orbitais D\\
orbitais iso degenerados

\subsection{Orbitais Octaédricos ``t'' e ``e''}


\subsection{$\Delta_{oct}$}
Diferença energética entre os orbitais ``t'' e ``e''
Define a cor que é absorvida pelo complexo


\subsection{$\Delta P$}
Energia de ligação do ligando com o atomo central

\subsection{Spin eletronico baixo - alto}
Soma dos numeros de spin, mais eletrons desemparelhados em orbitais tem maior numero de spin que quando existem eletrons eparelhados anulando seu spin








\end{document}











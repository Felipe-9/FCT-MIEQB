\documentclass[12pt]{article}

% Linguagem
\usepackage[portuguese]{babel}

% Clickable Table of contents
\usepackage{hyperref}
\hypersetup{
	hidelinks=true,
	colorlinks=true,
	linkcolor=DarkGreen!20!LightGreen!25!
}

% Table of contents
%\usepackage{tocloft}
%\setlength{\cftsubsecnumwidth}{3em} % Fix subsection width
% Fix space between subsection items on toc
%\renewcommand\cftsubsecafterpnum{\vskip5pt}

% Multicols
\usepackage{multicol}

% Customize Chapter
%\usepackage{titlesec}
%\titleformat{\chapter}[hang]{\Huge\bfseries\color{DarkGreen!75!}}{\thechapter\hspace{20pt}{$|$}\hspace{20pt}}{0pt}{\Huge\bfseries}

% Appendix
%\usepackage{appendix}

% Maths
%\usepackage{amssymb} 		
%\usepackage{amsmath} 
%\usepackage[utf8]{inputenc} %useful to type directly diacritic characters

%\newcommand{\bm}[1]{{\boldmath{\large{\begin{align*} #1 \end{align*}}}}}

% Vectors
%\usepackage{esvect} 	% Vector over-arrow
%\renewcommand{\vec}{\vv} % Vecto over-arrow

% Tikz
%\usepackage{tikz}		% Vector diagrams
%\usetikzlibrary{calc}  % Vector calculations
%\usepackage{varwidth}  % List inside TikzPicture

% Chem
\usepackage{chemformula} 	% formulas quimicas
%\usepackage{chemfig} 		% Estruturas quimicas

% Tabular
%\usepackage{multirow}
%\usepackage{dcolumn}
%\newcolumntype{d}[1]{D{.}{\cdot}{#1} } % Align on decimal

% Colors
\usepackage{xcolor}

\definecolor{DarkBlue}	{HTML}{252A36}
\definecolor{LightGreen}{HTML}{7CCC6C}
\definecolor{DarkGreen}	{HTML}{008675}

\pagecolor{DarkBlue!110!}
\color{DarkGreen!20!}

%\definecolor{Red}  {hsb}{0  ,.6,1}
%\definecolor{Blue} {hsb}{0.6,.6,1}
%\definecolor{Green}{hsb}{0.3,.6,1}

% Counters
%\counterwithin*{section}{part} % Reset section on part

% Section and Subsection Customization
%\renewcommand\thesection{Exercicio \arabic{section} }
\renewcommand\thesubsection{E\arabic{section} - \alph{subsection}) }

\begin{document}

\title{\bfseries\color{DarkGreen!75!}%
	QI 1 - Teoria de Pearson%
}
\author{Felipe Pinto - 61387}
\date{15/04 - 2021.1}

\maketitle
%\tableofcontents
\break

% E1
\section{Exercicio: Escrever complexos possíveis de se formar}

% E1 - a)
\subsection{\ch{ Ag(I),\ SH2,\ H2O,\ NO3^-,\ CN^- }}
\begin{itemize}
\begin{multicols}{2}
 
 	\item \ch{Ag(I)}: Acido Mole
	\item \ch{SH2}: Acido Mole?
	\item \ch{H2O}: Acido/Base Duro
	\item \ch{NO3^-}: Base Dura
	\item \ch{CN^-}: Base Mole

\end{multicols}
\end{itemize}

\begin{itemize}

	\item \ch{ AgCN }
	
	\begin{itemize}
 		
		\item Natureza Elemento Central:
			Acido Mole
	
		\item Natureza Ligandos:
			Base Mole
			
		\item Estruturas:
			Linear
		
	\end{itemize}
	
	\item \ch{ HNO3 }
	
	\begin{itemize}
 		
		\item Natureza Elemento Central:
			Acido Duro
			
		\item Natureza Ligandos:
			Base Dura
			
		\item Estruturas: Linear, Triangular Plana
			
	\end{itemize}
	
	\item \ch{ AgSH2^+ }
	
	\begin{itemize}
 		
		\item Natureza Elemento Central:
			Acido Mole
			
		\item Natureza Ligandos:
			Base Mole
			
		\item Estruturas: Linear
		
	\end{itemize}
	
\end{itemize}

\break

% E1 - b)
\subsection{\ch{ Au(I),\ CN^-,\ SH2(CH2)2NH2,\ H2O,\ BF4^- }}
\begin{itemize}
	
	\item \ch{Au(I)}: Acido Mole
	\item \ch{CN^-}: Base Mole
	\item \ch{SH2(CH2)2NH2}: ????
	\item \ch{H2O}: Base/Acido Duro
	
\end{itemize}

\begin{itemize}
	
	\item \ch{  }
	
	\begin{itemize}
 		
		\item Natureza Elemento Central:
	
		\item Natureza Ligandos:
	
		\item Estruturas:
		
	\end{itemize}



%	\item \ch{  }
%	
%	\begin{itemize}
% 		
%		\item Natureza Elemento Central:
%	
%		\item Natureza Ligandos:
%	
%		\item Estruturas:
%		
%	\end{itemize}



\end{itemize}



% E1 - c)
\subsection{\ch{ Cu(II),\ acac,\ NO2^-,\ Cl^- }}
\begin{itemize}

	\item Formula: \ch{  }
	
	\item Natureza Elemento Central:
	
	\item Natureza Ligandos:
	
	\item Estruturas:

\end{itemize}



% E1 - d)
\subsection{\ch{ Ca(II),\ H2O,\ EDTA^{4-},\ ClO4^- }}
\begin{itemize}

	\item Formula: \ch{  }
	
	\item Natureza Elemento Central:
	
	\item Natureza Ligandos:
	
	\item Estruturas:

\end{itemize}



% E1 - e)
\subsection{\ch{ Fe(III),\ CO2,\ Phen,\ en,\ F^- }}
\begin{itemize}

	\item Formula: \ch{  }
	
	\item Natureza Elemento Central:
	
	\item Natureza Ligandos:
	
	\item Estruturas:

\end{itemize}



% E1 - f)
\subsection{\ch{ Hg(II),\ tu,\ F^-,\ SCN^- }}
\begin{itemize}

	\item Formula: \ch{  }
	
	\item Natureza Elemento Central:
	
	\item Natureza Ligandos:
	
	\item Estruturas:

\end{itemize}



% E1 - g)
\subsection{\ch{ Pt(II),\ PPh3,\ Br^-,\ NH2(CH2)3NH2 }}
\begin{itemize}

	\item Formula: \ch{  }
	
	\item Natureza Elemento Central:
	
	\item Natureza Ligandos:
	
	\item Estruturas:

\end{itemize}




\section{Random:}

\subsection{}



\end{document}











\documentclass[12pt]{article}

% Multicols
%\usepackage{multicol}

% Maths
%\usepackage{amssymb} 		
%\usepackage{amsmath} 
%\usepackage[utf8]{inputenc} %useful to type directly diacritic characters

%\newcommand{\bm}[1]{{\boldmath{\large{\begin{align*} #1 \end{align*}}}}}

% Vectors
%\usepackage{esvect} 	% Vector over-arrow
%\usepackage{tikz}		% Vector diagrams

%\renewcommand{\vec}{\vv} % Vecto over-arrow

% Chem
\usepackage{chemformula} 	% formulas quimicas
\usepackage{chemfig} 		% Estruturas quimicas

% Colors
\usepackage{xcolor}

\definecolor{DarkBlue}	{HTML}{252A36}
\definecolor{LightGreen}	{HTML}{7CCC6C}
\definecolor{DarkGreen}	{HTML}{008675}

\pagecolor{DarkBlue!110!}
\color{DarkGreen!20!}

% Resolução de listas
%\renewcommand\thesection{Questão \arabic{section} }
%\renewcommand\thesubsection{\arabic{section}-\alph{subsection}) }

\begin{document}

\title{QI - Caracteristicas de Complexos}
\date{30/03}

\maketitle
\tableofcontents
\break

\section{Cores em complexos}
pela quantidade de elentrons existentes nos orbitais do elemento central, a cor do composto do complexo muda
\subsection{exemplos}
\begin{itemize}
\item \ch{[Cr(H2O)6]^{+2}} de cor azul do céu
\item \ch{[Mn(H2O)6]^{+2}} de cor rosa pálido
\end{itemize}

\subsection{caracteristicas verificadas pela cor}
dependendo do atomo central a cor será mais forte ou pálida


\section{caracterização por numero de elementos centrais}

\begin{itemize}
\item Mononucleares
\item Polinucleares
\end{itemize}

\subsection{Caracterização em relação aos atomos centrais}
\begin{itemize}
\item Homogeneo
\item Heterogenio
\end{itemize}

\section{Paramagnetismo}
indica se o composto possui propriedades magneticas
basicamente verificar a existencia de pelomenos um eletron desemparelhado

Pela existencia de mais de um elemento central existe uma Reatividade de transferencia de Ligados

\section{Formação de Complexos}
\ch{ [Co(H2O)6]^{2+}+4Cl^- -> [CoCl4]^-\, 6H2O }
se sabe quando a reação está acontecendo pela mudança de cor

\subsection{Estado de oxidação}
é muito dificil saber o estado de ligação quando o ligando não possui eletrons doadores.

Fazer os estados de oxidação dos metais para compreender melhor os estados de oxidação estáveis

Relacionar os grupos de metais por linhas

\subsection{Principio de Electroneutralidade}
A carga de qualquer átomo estará entre -1 e +1??

\subsection{Número de coordenação / Indice de coordenação}

\subsection{Modelo de Kepert}
Basicamente, o numero de ligandos define a geometria do complexo

\subsection{teorema da distorção de jahn-teller (N de coord. 6)}
forçando as ligações distorce a geometria dos complexos

\section{Isomerismo em Complexos Inorgânicos}

\subsection{Isomeria de Hidratação}
Quantidade de Moléculas de agua como ligando para anião





\section{Random}

\paragraph{Complexos em estado sólido} são geralmente neutros


\paragraph{15-crow-n} Ligantes que formam um anél ao redor do elemento central


\section{Para o teste 1}
\begin{itemize}
\item Ficha 1
\item Ficha 2
\item Ficha 3
\end{itemize}

\end{document}

\documentclass[12pt]{article}

% Linguagem
\usepackage[portuguese]{babel}

% Clickable Table of contents
\usepackage{hyperref}
\hypersetup{
	hidelinks=true,
	colorlinks=true,
	linkcolor=DarkGreen!20!LightGreen!25!
}

% Multicols
\usepackage{multicol}

% Maths
%\usepackage{amssymb} 		
%\usepackage{amsmath} 
%\usepackage[utf8]{inputenc} %useful to type directly diacritic characters

%\newcommand{\bm}[1]{{\boldmath{\large{\begin{align*} #1 \end{align*}}}}}

% Vectors
%\usepackage{esvect} 	% Vector over-arrow
%\usepackage{tikz}		% Vector diagrams
%\usetikzlibrary{calc}  % Vector calculations
%\usepackage{varwidth}  % List inside TikzPicture

%\renewcommand{\vec}{\vv} % Vecto over-arrow

% Chem
\usepackage{chemformula} 	% formulas quimicas
\usepackage{chemfig} 		% Estruturas quimicas

% Colors
\usepackage{xcolor}

\definecolor{DarkBlue}	{HTML}{252A36}
\definecolor{LightGreen}{HTML}{7CCC6C}
\definecolor{DarkGreen}	{HTML}{008675}

\pagecolor{DarkBlue!110!}
\color{DarkGreen!20!}

% Counters
\counterwithin*{section}{part} % Reset section on part

% Resolução de listas
\renewcommand\thesection{A\arabic{section}}
\renewcommand\thesubsection{\thesection.\arabic{subsection}}

\begin{document}

\title{Resolução Ficha 3}
\author{Felipe Pinto - 61387}
\date{06/04 2021.1}

\maketitle
\tableofcontents
\break

\section{}

% A1.1
\noindent\begin{minipage}{\textwidth}

   \subsection{\ch{[CoCl2(en)2]Cl}}
   \noindent\begin{itemize}
   \item Nome: 
   \item Átomo Metálico Central:
   \item Configuração Eletrónica:
   \item Estado de Oxidação:
   \item Ligandos:
   \item Contraion:
   \item Geometria:
   \item Índice de Coordenação
   \item Número de Ligandos:
   \item $\cdots$
   \end{itemize}

\end{minipage}

% A1.2
\noindent\begin{minipage}{\textwidth}

   \subsection{\ch{Co(ONO)(NH3)5}SO4}
   \noindent\begin{itemize}
   \item Nome:	
   	Sulfato de penta\,amin\,nitrito\,cobaltato
   
   \item Átomo Metálico Central: 
   	Cobalto \ch{Co}
   
   \item Configuração Eletrónica: 
   	\ch{[Ar]}: $3d^7\cdots$
   	
   \item Estado de Oxidação:
   
   \item Ligandos: 
   	Amin (\ch{NH3}) e Nitrato (\ch{ONO})
   
   \item Contraion: 
   	Sulfato (\ch{SO4})
   
   \item Geometria: 
   	Octaedrico
   
   \item Índice de Coordenação: 
   	6
   
   \item Número de Ligandos: 
   	6 (1 Nitrito e 5 Amin)
   
   \item Átomos Doadores: 
   	6 (1 oxigênio (O) e 5 azotos (N))
   
   \item Isomeria:
   	O Complexo não possui isomeria
   
   \end{itemize}
   
\end{minipage}

% A1.3
\noindent\begin{minipage}{\textwidth}

   \subsection{\ch{[Cr(H2O)4Cl2]Cl}}
   \noindent\begin{itemize}
   \item Nome: Cloreto de tetraquo\,dicloro\,cromio(III)
   	
   
   \item Átomo Metálico Central: Cromo
   	
   	
   \item Configuração Eletrónica:
   
   
   \item Estado de Oxidação: 
   
   
   \item Ligandos:
   
   
   \item Contraion:
   
   
   \item Geometria:
   
   
   \item Índice de Coordenação: 
   
   
   \item Número de Ligandos:
   
   
   \item Átomos Doadores:
   
   
   \item Isomeria:
   	Geometrica (cis e trans)
   
   
   \end{itemize}
   
\end{minipage}

% A1.4
\noindent\begin{minipage}{\textwidth}
	
	\subsection{\ch{Na3[Co(NO2)6]}}
	\begin{itemize}
   \item Nome: Exanitro\,Cobaltato(III) 
   	
   
   \item Átomo Metálico Central:
   	
   	
   \item Configuração Eletrónica:
   
   
   \item Estado de Oxidação:
   
   
   \item Ligandos:
   
   
   \item Contraion:
   
   
   \item Geometria:
   
   
   \item Índice de Coordenação: 
   
   
   \item Número de Ligandos:
   
   
   \item Átomos Doadores:
   
   
   \item Isomeria:
   	Duas isomerias
		(ONO)
	\end{itemize}
	
\end{minipage}

% A1.5
\noindent\begin{minipage}{\textwidth}
	
	\subsection{\ch{[(NH3)5-Cr-OH-Cr(NH3)5]Cl5}}
	\begin{itemize}
   \item Nome:
   	Cloreto de $\mu$-hidroxido-bis (Pentamin\,dicromo(III))
   
   \item Átomo Metálico Central:
   	Cromo (\ch{Cr})
   	
   \item Configuração Eletrónica:
   	\ch{[Ar]}: $3d^3$
   
   \item Estado de Oxidação:
   	III+
   
   \item Ligandos:
   	1 hidroxido (\ch{OH}) e 5 amin
   
   \item Contraion:
   	Cloreto (\ch{Cr^-})
   
   \item Geometria:
   	Octaedrica em ambos complexos
   
   
   \item Índice de Coordenação: 
   	6
   	
   \item Número de Ligandos:
   	6
   
   \item Átomos Doadores:
   	6 (5 azotos, 1 oxigênio)
   
   \item Isomeria:
   	
   
	\end{itemize}
	
\end{minipage}

% A1.6
\noindent\begin{minipage}{\textwidth}
	
	\subsection{\ch{[Co(NH3)2(H2O)]Br2}}
	\begin{itemize}
   \item Nome:
   	Brometo de diamin aqua cobalto (II)
   	
   
   \item Átomo Metálico Central:
   	Cobalto
   	
   \item Configuração Eletrónica:
   	\ch{[Ar]}: $3d^7\ 4s^2$
   
   \item Estado de Oxidação:
   	II+
   
   \item Ligandos:
   	Amin (\ch{NH3}), Agua (\ch{H2O})
   
   \item Contraion:
   	Brometo (\ch{Br^-})
   
   \item Geometria:
   	Trigonal Planar
   
   \item Índice de Coordenação: 
   
   
   \item Número de Ligandos:
   
   
   \item Átomos Doadores:
   
   
   \item Isomeria:
   	
   
   \item Características:
   	Átomo central não tende formar geometria trigonal, e por isso é instável
   
	\end{itemize}
	
\end{minipage}

% A1.7
\noindent\begin{minipage}{\textwidth}
	
	\subsection{\ch{}}
	\begin{itemize}
   \item Nome:
   	
   
   \item Átomo Metálico Central:
   	
   	
   \item Configuração Eletrónica:
   	D6
   
   \item Estado de Oxidação:
   	
   
   \item Ligandos:
   	6
   
   \item Contraion:
   	Cloreto (\ch{Cl^-})
   
   \item Geometria:
   	
   
   \item Índice de Coordenação: 
   
   
   \item Número de Ligandos:
   
   
   \item Átomos Doadores:
   
   
   \item Isomeria:
   	Nenhuma
   
	\end{itemize}
	
\end{minipage}

% A1.8
\noindent\begin{minipage}{\textwidth}
	
	\subsection{\ch{K[PtCl3(C2H4)]}}
	\begin{itemize}
   \item Nome:
   	Etileno tricloro platinado potácil
   
   \item Átomo Metálico Central:
   	Platina (\ch{Pl})
   	
   \item Configuração Eletrónica:
   	4f 14 5 d8
   
   
   \item Estado de Oxidação:
   	+2
   
   \item Ligandos:
   	cloro e etileno
   
   \item Contraion:
   	k+
   
   \item Geometria:
   	
   
   \item Índice de Coordenação: 
   
   
   \item Número de Ligandos:
   	4
   
   \item Átomos Doadores:
   	4 (1 etileno (\ch{C=C}) e 3 cloreto (\ch{Cl^-}))\\
		etileno coordena com o atomo central pela ligação dupla
   
   \item Isomeria:
   	Não possui isomeria
   	
   
   \item Caracteristica:
   	instável
   
	\end{itemize}
	
\end{minipage}

% A1.9
\noindent\begin{minipage}{\textwidth}
	
	\subsection{\ch{}}
	\begin{itemize}
	
   \item Nome:
   	Nitrato de tetramin\,cloro-N-isotiocianato cobalto(III)
		\\ -N- para identificar quem está se ligando do bidentado
   
   \item Átomo Metálico Central:
   	
   	
   \item Configuração Eletrónica:
   	1d6
   
   \item Estado de Oxidação:
   
   
   \item Ligandos:
   	nh3 no3 scn (isso tio cianato)
   
   \item Contraion:
   	nitrato no3-
   
   \item Geometria:
   	octaedrica
   
   \item Índice de Coordenação: 
   	6
   
   \item Número de Ligandos:
   	6
   
   \item Átomos Doadores:
   	6 (5azotos 1 cloro)
   
   \item Isomeria:
   	3 isomeros (bidentado)\\
		mediante ao prata, mediante ao infrav, mediante ao  
   
   \item Caracteristica:
   	se identifica bem por infravermelho\\
		tambem se adiciona prata para poder seprarar os isomeros
   
	\end{itemize}
	
\end{minipage}

% A1.10
\noindent\begin{minipage}{\textwidth}
	
	\subsection{\ch{Pt(en)Cl4}}
	\begin{itemize}
	
   \item Nome:
   	etilenodiamin tetracloro  platica (IV)
   
   \item Átomo Metálico Central:
   	Platina
   	
   \item Configuração Eletrónica:
   	4f14 5d6
   
   \item Estado de Oxidação:
   	6
   
   \item Ligandos:
   	5\\
		en: bidentado quelato
   
   \item Contraion:
   	
   
   \item Geometria:
   	octaedrica
   
   \item Índice de Coordenação: 
   
   
   \item Número de Ligandos:
   
   
   \item Átomos Doadores:
   
   
   \item Isomeria:
   	isomeria destrogeno e levogeno
	
	\item Caracteristica:
		não se precipita nada com prata
		
	\item 
	
	\end{itemize}
	
\end{minipage}

% A1.11
\noindent\begin{minipage}{\textwidth}
	
	\subsection{\ch{}}
	\begin{itemize}
   \item Nome:
   	etileno diamin tetracetato nikelato(II) de sódio
   
   \item Átomo Metálico Central:
   	Ni
   	
   \item Configuração Eletrónica:
   
   
   \item Estado de Oxidação:
   	
   
   \item Ligandos:
   	tetracetato
   
   \item Contraion:
   	sódio
   
   \item Geometria:
   	octaedrica
   
   \item Índice de Coordenação: 
   	6
   
   \item Número de Ligandos:
   
   
   \item Átomos Doadores:
   
   
   \item Isomeria:
   	destro e levo
	
	
	\end{itemize}
	
\end{minipage}

% A1.12
\noindent\begin{minipage}{\textwidth}
	
	\subsection{\ch{}}
	\begin{itemize}
   \item Nome:
   	
   
   \item Átomo Metálico Central:
   	
   	
   \item Configuração Eletrónica:
   
   
   \item Estado de Oxidação:
   
   
   \item Ligandos:
   
   
   \item Contraion:
   
   
   \item Geometria:
   
   
   \item Índice de Coordenação: 
   
   
   \item Número de Ligandos:
   
   
   \item Átomos Doadores:
   
   
   \item Isomeria:
	\end{itemize}
	
\end{minipage}


% A1.14
\noindent\begin{minipage}{\textwidth}
	
	\subsection{\ch{[Zn(en)2Br2]}}
	\begin{itemize}
   \item Nome:
   	etilenodiamin dibromo zinco(II)
   
   \item Átomo Metálico Central:
   	zinco
   	
   \item Configuração Eletrónica:
   	II+
   
   \item Estado de Oxidação:
   	[ar] 3d9
   
   \item Ligandos:
   	etilenodiamin
   
   \item Contraion:
   	não possui
   
   \item Geometria:
   	
   
   \item Índice de Coordenação: 
   
   
   \item Número de Ligandos:
   
   
   \item Átomos Doadores:
   	
   
   \item Isomeria:
	\end{itemize}
	
\end{minipage}

%\noindent\begin{minipage}{\textwidth}
%	
%	\subsection{\ch{}}
%	\begin{itemize}
%   \item Nome:
%   	
%   
%   \item Átomo Metálico Central:
%   	
%   	
%   \item Configuração Eletrónica:
%   
%   
%   \item Estado de Oxidação:
%   
%   
%   \item Ligandos:
%   
%   
%   \item Contraion:
%   
%   
%   \item Geometria:
%   
%   
%   \item Índice de Coordenação: 
%   
%   
%   \item Número de Ligandos:
%   
%   
%   \item Átomos Doadores:
%   
%   
%   \item Isomeria:
%	\end{itemize}
%	
%\end{minipage}

\break
\section{Random}

\hypertarget{en}{}
\subsection{\ch{en}: etileno diamina}

\subsection{Isomeros de polidentados}
indicada por analize infravermelho

\subsection{\ch{Na3[Co(NO2)3(ONO)3]} Isomeria FAC(ial) e isomeria MER(idional)}

\end{document}












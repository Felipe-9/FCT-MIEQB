\documentclass[12pt]{article}

\usepackage{chemformula} % formulas quimicas
%\usepackage{chemfig} % Estruturas quimicas

\usepackage{color}
\pagecolor{black}
\color{white}

\begin{document}

\title{QI 1 - Química Inorgânica 1 - OP}
\maketitle

\section{}

\begin{itemize}
\item Agua: \ch{H2O}
\item Acido Sulfirico: \ch{H2SO4}
\item Acido Sulfirico: \ch{H2S}
\item Nitrado de Tetraaminodiaquoferro(III): \ch{[Fe(NH3)4(H2O)2]N}
\item Cloreto de Hexaaquocobalto(II): \ch{ [Co(H2O)6]Cl }
\item Óxido de Potácio: \ch{K2O}
\item Peróxido de Potácio: \ch{KO}
\item Fosfina: \ch{PH3} % Fosfinas organicas
\item Ácido Clorídrico: \ch{HCl}
\item Nitrado de Sódio: \ch {NaNo3}
\item Cloreto de Aluminio: \ch{AlCl3}
\item Óxido de Cálcio: \ch{ CaO }
\item Ácido Perclórico
\item Brometo de Cobre(II): \ch{BrCu2}
\item Óxido de Cromo(III): \ch{Cr2O3}
\item Ião de Amônio: \ch{NH4^-}
\end{itemize}

\section{}

\begin{itemize}
\item \ch{Na2[Co(SCN)3]CO}: tris-tiocianato-carbonila-cobaltato(I) de sodio?
\item \ch{Fe2O3}: Óxido de Ferro 3
\item \ch{[Zn(H2O)6](NO3)2}: Nitrado de Hexaquozinco
\item \ch{D2O}: Agua deuterada % D = deuterio do H
\item \ch{HIO4}: Ácido Per-iodico
\item \ch{FeCl3}: Cloreto de Ferro (III)
\item \ch{K2[Zn(EDTA)]}: etileno de amin tetra acetato zinco de potássio % Ligandos Polidentados
\item \ch{KOH}: Hidroxido de Potássio
\item \ch{K2[PtCl4]}: Tetra cloro platinato de potássio (IV)
\item \ch{BaO2}: Peroxido de Bário
\item \ch{HClO}: Ácido Hipocloroso
\item \ch{ClO^-}: Hipoclorito
\item \ch{(NH4)2CrO4}: Cromato de amonio 
\item \ch{NHO3}: Acido Nitrico
\item \ch{Na3PO4}: Fosfato tri sódico
\item \ch{Cr(OH)2}: Hidroxido de Cromo (2)
\item \ch{[CoCl2(en)2]Cl}: Cloreto de dis Etileno di amin de cloro cobalto (III)
\item \ch{Ca3(PO4)2}: fosfato calcico ou Fosfato de tri calcico
\item \ch{KMnO4}: Permanganato de potássio
\item \ch{CuBr}: Brometo de Cobre
\end{itemize}

\section{Random}
\subsection{quelatoterapia}
EDTA: 
usa substancia EDTA para descontaminar pois ele complexa bastante com metais
ao tomar um copo de EDTA a pessoa pode ficar com

\subsection{en: Etileno Diamina: \ch{(NH3)2} }


\end{document}

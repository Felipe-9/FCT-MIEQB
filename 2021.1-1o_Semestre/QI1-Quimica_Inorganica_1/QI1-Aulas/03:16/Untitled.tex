\documentclass[12pt]{report}

\usepackage{xcolor}

\pagecolor{black}
\color{white}

\begin{document}

\title{QI 1 - Quimica Inorganica 1}
\author{}
\date{16/03/2021}

\maketitle

\tableofcontents

\newpage

\section{Introdução}

\begin{itemize}
\item Catalizador de Grubs
\item Catalizador de Schrock
\item Catalizador de algo???
\end{itemize}


\paragraph{Máquinas Moleculares: Química Supramolecular\\ }
Molecula A + Molecula B = Molecula C com propriedades diferentes das originais

\paragraph{Qímica Bio-Inorganica (me aguarde)\\ }

\paragraph{53 Cisplatina\\ }
Impede a replicação de DNA, isómero -\textit{trans} é inativo

\paragraph{"Complexos Inorgânicos" em Biologia\\ }
\begin{itemize}
\item Nitrogenase (Mo,Fe)
\item Citocromo C
\item Ferrintina (Homen)
\item Clorofila a
\end{itemize}

\break

\section{Nomenclatura e Formulação de Complexos Inorgânicos}

\subsection{Introdução ao Complexos de Transição}

Ligações metálicas mais fortes

\paragraph{[complexo]Ligandos}
\begin{itemize}
\item complexo: estrutura com átomo metálico central
\item Ligandos: moléculas ou átomos externos(não ligados) ao complexo
\end{itemize}

\subsection{Compostos de Coordenação}
\paragraph{Adulto:} Espécie coordenada que apresenta carga elétrica nula;

\paragraph{Sais Complexos ou Complexos Iónicos: } Espécie coordenada que apresenta carga negativa ou positiva

\subsection{Regras de Nomenclatura}
\begin{itemize}
\item Nome do Anion + Nome do Cátion
\item Se dois neutros ou dois iônicos: ordem alfabética
\item Ligandos mudam de nome quando entram no "parêntese"
\item Complexo negativo: ato
\item Complexo positivo: Nina
\item Atomo de enxofre: tio
\item Neutros antes de Negativo
\end{itemize}

\paragraph{indice de coordenação: } atomos de ligantes no atomo de ferro

\subsection{Regras de Nomenclatura relacionada com numeros}




\end{document}

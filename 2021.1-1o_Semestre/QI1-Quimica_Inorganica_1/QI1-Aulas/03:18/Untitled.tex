\documentclass[12pt]{report}

\usepackage{color}
\pagecolor{black}
\color{white}



\begin{document}

\title{QI 1 - Química Inorgânica 1}
\date{18/03}

\maketitle

\tableofcontents

\break

\chapter{Nomenclatura e Formulação de Complexos Inorgânicos}

\section{Introdução aos elementos: Complexos e Compostos de Coordenação}

\begin{itemize}
\item Complexos: Representado assim $ [\cdots] $
\item Compostos de Coordenação:  Anodo$\,$Catodo
\end{itemize}

\subsection{}

\break

complexo é uma formula química que possui no mínimo dois componentes: anodo e catodo, sendo um desses um metal

amônio dentro dos parênteses se chama "amin" ou "amino"

\subsection{regras de formulação}

se começa a dar nome pela parte negativa (anion)

Sequencia de Prioridade no nome Neutros + Aniônicos + Catiônicos + estádo de oxidação do átomo central
caso empate: ordem alfabetica

\subsection {Ligante: } De ion para ligante
\begin{itemize}
\item eto, ito e ido $\to$ o
\item ato $\to$ ato
\end{itemize}

\subsection{Ligandos Moleculares: } atomos que formam pontes
Se adiciona $\mu$ antes -$\mu$-"atomo"

\subsection{Complexo aniônico: } nome se termina em ato

\begin{itemize}
\item ido $\to$ oxo
\item ato $\to$ ato
\end{itemize}



\end{document}

\documentclass[12pt]{article}

% Linguagem
\usepackage[portuguese]{babel}

% Clickable Table of contents
\usepackage[hidelinks]{hyperref}

% Multicols
%\usepackage{multicol}

% Maths
%\usepackage{amssymb} 		
%\usepackage{amsmath} 
%\usepackage[utf8]{inputenc} %useful to type directly diacritic characters

%\newcommand{\bm}[1]{{\boldmath{\large{\begin{align*} #1 \end{align*}}}}}

% Vectors
\usepackage{esvect} 	% Vector over-arrow
\usepackage{tikz}		% Vector diagrams
\usetikzlibrary{calc}% Vector calculations
\usepackage{varwidth}% List inside TikzPicture

\renewcommand{\vec}{\vv} % Vecto over-arrow

% Chem
%\usepackage{chemformula} 	% formulas quimicas
%\usepackage{chemfig} 		% Estruturas quimicas

% Colors
\usepackage{xcolor}

\definecolor{DarkBlue}	{HTML}{252A36}
\definecolor{LightGreen}{HTML}{7CCC6C}
\definecolor{DarkGreen}	{HTML}{008675}

\pagecolor{DarkBlue!110!}
\color{DarkGreen!20!}

% Resolução de listas
\renewcommand\thesection{P10 - a)}
%\renewcommand\thesubsection{\arabic{section}-\alph{subsection}) }

\begin{document}

\section{Tabela 2x2}

\begin{tikzpicture}[scale = 1.5, font=\small, line width = 1]
	
	\clip (-4.5,-9.3) rectangle (9,2.5);
	
	\draw[DarkBlue!90!] (-10,0) -- (10,0);
	\draw[DarkBlue!90!] (0,-10) -- (0, 10);
	
	\coordinate (0) at (0,0);
	
	\node at (-2.5, 1) [label=above:\bf{Factos/Dados}] 
	{ 
		\begin{varwidth}{\linewidth}\begin{itemize}
      \item $\vec a_{J}$: aceleração de Joanna
      \item $\vec a_{S}$: aceleração de Sofia
      \item $t_1=5\,s$: momento especifico
  		\end{itemize}\end{varwidth}
	};
	\node at ( 2, 1.8) [label=above:\bf{Diagrama}] {};
	
	\begin{scope}
	[
		scale=2, shift={(0.8,0.4)},
		every node/.append style={minimum size=1, inner sep=1}
	]
			
			\def\maxX { 0.5};
			\def\maxY { 0.5};
			\def\minX {-0.15};
			\def\minY {-0.25}
						
			\clip (\minX,\minY) rectangle (\maxX,\maxY);
			
			\coordinate (MinXY) at (\minX,\minY);
			
			\draw[ultra thin, DarkBlue!85! , step=0.1 ] 
			(MinXY) grid (\maxX+1,\maxY+1);
			
			\draw[->, thin] (0,\minY) -- (0,\maxY)
			node[below left]{$y$};
			\draw[->, thin] (\minX,0) -- (\maxX,0)
			node[above left]{$x$};
			
			\coordinate (0) at (0,0);

			\draw[->, green!60!] 
			(0) to (0.3,-0.2) node[above right]{$\vec a_J$};
			
			\draw[->, cyan!60!] 
			(0) to (0.1,+0.3) node[above      ]{$\vec a_S$};
			
		\end{scope};
	
	\node at (-2.5,-1.5) [label=above:\bf{Questões}] 
	{
		\begin{varwidth}{\linewidth}\begin{itemize}
      	\item $ \vec v_J-\vec v_S =? $
      	\item $ \| \vec P_{J(t_1)} - \vec P_{S(t_1)} \| = ?$
      	\item $ \vec a_J - \vec a_S = ? $
	  		\end{itemize}\end{varwidth}
	};
	\node at ( 3,-3.5) [label=above:\bf{Passos}] 
	{
		\begin{varwidth}{\linewidth}\begin{itemize}
			\item a)
		
			\begin{itemize}
 			\item Encontre a equação de velocidade\\de cada carro e subtraia vetorialmente
			\end{itemize}
      	
			\item b)
	
			\begin{itemize}
 			\item Encontre a posição de cada\\carro no momento pedido
			\item Encontre o modulo da diferença\\dos vetores de posição
			\end{itemize}
			
			\item c)
 			\begin{itemize}
 			\item Subtraia os vetores de\\ aceleração de cada carro
			\end{itemize}
			
	  	\end{itemize}\end{varwidth}
	};
	
\end{tikzpicture}


\end{document}

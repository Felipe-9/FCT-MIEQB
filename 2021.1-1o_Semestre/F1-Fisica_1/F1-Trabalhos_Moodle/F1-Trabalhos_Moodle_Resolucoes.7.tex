\documentclass[12pt]{article}

% Geometry
\usepackage{geometry}
\geometry{
	total={},
	top = 25mm,
	bottom = 0mm,
	left = 5mm,
	right = 5mm,
}

% Linguagem
\usepackage[portuguese]{babel}

% Clickable Table of contents
\usepackage{hyperref}
\hypersetup{
	hidelinks=true,
	colorlinks=true,
	linkcolor=DarkGreen!20!LightGreen!25!
}

% Table of contents
%\usepackage{tocloft}
%\setlength{\cftsubsecnumwidth}{3em} % Fix subsection width
% Fix space between subsection items on toc
%\renewcommand\cftsubsecafterpnum{\vskip5pt}

% Multicols
%\usepackage{multicol}

% Customize Chapter
%\usepackage{titlesec}

%\titleformat{\chapter}[hang]{\Huge\bfseries\color{DarkGreen!75!}}{\thechapter\hspace{20pt}{$|$}\hspace{20pt}}{0pt}{\Huge\bfseries}

% Appendix
%\usepackage{appendix}

% Maths
\usepackage{amssymb} 		
\usepackage{amsmath} 
\usepackage[utf8]{inputenc} %useful to type directly diacritic characters

\newcommand{\bm}[1]{{\boldmath{\large{\begin{align*} #1 \end{align*}}}}}

% Vectors
\usepackage{esvect} 	% Vector over-arrow
\renewcommand{\vec}{\vv} % Vecto over-arrow

% Tikz
\usepackage{tikz}		% Vector diagrams
\usetikzlibrary{calc}  % Vector calculations
\usepackage{varwidth}  % List inside TikzPicture

% Chem
%\usepackage{chemformula} 	% formulas quimicas
%\usepackage{chemfig} 		% Estruturas quimicas

% Colors
\usepackage{xcolor}

\definecolor{DarkBlue}	{HTML}{252A36}
\definecolor{LightGreen}{HTML}{7CCC6C}
\definecolor{DarkGreen}	{HTML}{008675}

\pagecolor{DarkBlue!110!}
\color{DarkGreen!20!}


\definecolor{Red}  {hsb}{0  ,.6,1}
\definecolor{Blue} {hsb}{0.6,.6,1}
\definecolor{Green}{hsb}{0.3,.6,1}

% Counters
%\counterwithin*{section}{part} % Reset section on part

% Resolução de listas
%\renewcommand\thesection{Questão \arabic{section} }
%\renewcommand\thesubsection{\arabic{section}-\alph{subsection}) }

\begin{document}

\title{\bfseries\color{DarkGreen!75!}%
Física 1 - T7\\Tabela 2x2 P4 Lista S4\\Força e movimento I%
}
\author{Felipe Pinto - 61387}

\maketitle
\break

\begin{tikzpicture}[
	scale = 1.5, font=\small, line width = 1
]
	
	\clip (-6.5,-4.3) rectangle (6,2.5);
	
	\draw[DarkBlue!90!] (-10,0) -- (10,0);
	\draw[DarkBlue!90!] (0,-10) -- (0, 10);
	
	\coordinate (0) at (0,0);
	
% Factos / Dados
	\node at (-3.5, 1) [label=above:\bf{Factos/Dados}] 
	{ 
	\begin{varwidth}{\linewidth}
	\begin{itemize}
      	\item $\theta=15^o$:
		Angulo de inclinação da superficie
		
      	\item $\vec v_i = \vec 0$:
		Velocidade inicial é nula
		
      	\item $S = 2.0m$:
		Comprimento do plano e posição final
  	\end{itemize}
	\end{varwidth}
	};
	
% Diagrama
	\node at ( 2, 1.8) [label=above:\bf{Diagrama}] {};
	
	\begin{scope}
	[
		scale=2, shift={(0.8,0.4)},
		every node/.append style={
			minimum size=1, inner sep=1
		}
	]
			
			\def\maxX { 0.5};
			\def\maxY { 0.5};
			\def\minX {-0.3};
			\def\minY {-0.3};
						
			\clip (\minX,\minY) rectangle (\maxX,\maxY);
			
			\coordinate (MinXY) at (\minX,\minY);
			
			\fill[rotate=-15, fill=DarkGreen!110!]
			(-0.1,-0.1) rectangle (0.1,0.1);
			
			\draw[ultra thin, DarkBlue!85! , step=0.1 ] 
			(MinXY) grid (\maxX+1,\maxY+1);
			
			\draw[->, thin] (0,\minY) -- (0,\maxY)
			node[below left]{$y$};
			\draw[->, thin] (\minX,0) -- (\maxX,0)
			node[above left]{$x$};
			
			\draw[thin, yshift=-2.9]	 (-15:-1) -- (-15:1);
			\fill[fill opacity=0.3, yshift=-2.9]
			(-15:-1) -- (-15:1) -- +(0,-10);
			
			\coordinate (0) at (0,0);
			
			\draw[->, green!60!] 
			(0) -- (0,-0.2)
			node[left]{$\vec F_g$};
			
			\draw[->, red!60!]
			(0) -- (75:0.1932)
			node[above right]{$\vec F_n$};
			
		\end{scope};

% Questões
	\node at (-3.5,-1) [label=above:\bf{Questões}] 
	{
		\begin{varwidth}{\linewidth}
		\begin{itemize}
		
      	\item $ |\vec a| =? $
	
	  	\end{itemize}
		\end{varwidth}
	};
	
% Passos
	\node at ( 3,-2) [label=above:\bf{Passos}] 
	{
		\begin{varwidth}{\linewidth}
		\begin{itemize}
		
		\item Defina o módulo de $\vec a$
			 como a raiz da soma\\
			 de suas componentes x e y 
			 ao quadrado
		\item Encontre a força resultante\\
			 que rege cada componente
		\item Resolva as equações
		
	  	\end{itemize}
		\end{varwidth}
	};
	
\end{tikzpicture}






%\begin{tikzpicture}[scale = 1.5, font=\small, line width = 1]
%	
%	\clip (-4.5,-9.3) rectangle (9,2.5);
%	
%	\draw[DarkBlue!90!] (-10,0) -- (10,0);
%	\draw[DarkBlue!90!] (0,-10) -- (0, 10);
%	
%	\coordinate (0) at (0,0);
%	
%	\node at (-2.5, 1) [label=above:\bf{Factos/Dados}] 
%	{ 
%		\begin{varwidth}{\linewidth}\begin{itemize}
%      \item $\vec a_{J}$: aceleração de Joanna
%      \item $\vec a_{S}$: aceleração de Sofia
%      \item $t_1=5\,s$: momento especifico
%  		\end{itemize}\end{varwidth}
%	};
%	\node at ( 2, 1.8) [label=above:\bf{Diagrama}] {};
%	
%	\begin{scope}
%	[
%		scale=2, shift={(0.8,0.4)},
%		every node/.append style={minimum size=1, inner sep=1}
%	]
%			
%			\def\maxX { 0.5};
%			\def\maxY { 0.5};
%			\def\minX {-0.15};
%			\def\minY {-0.25}
%						
%			\clip (\minX,\minY) rectangle (\maxX,\maxY);
%			
%			\coordinate (MinXY) at (\minX,\minY);
%			
%			\draw[ultra thin, DarkBlue!85! , step=0.1 ] 
%			(MinXY) grid (\maxX+1,\maxY+1);
%			
%			\draw[->, thin] (0,\minY) -- (0,\maxY)
%			node[below left]{$y$};
%			\draw[->, thin] (\minX,0) -- (\maxX,0)
%			node[above left]{$x$};
%			
%			\coordinate (0) at (0,0);
%
%			\draw[->, green!60!] 
%			(0) to (0.3,-0.2) node[above right]{$\vec a_J$};
%			
%			\draw[->, cyan!60!] 
%			(0) to (0.1,+0.3) node[above      ]{$\vec a_S$};
%			
%		\end{scope};
%	
%	\node at (-2.5,-1.5) [label=above:\bf{Questões}] 
%	{
%		\begin{varwidth}{\linewidth}\begin{itemize}
%      	\item $ \vec v_J-\vec v_S =? $
%      	\item $ \| \vec P_{J(t_1)} - \vec P_{S(t_1)} \| = ?$
%      	\item $ \vec a_J - \vec a_S = ? $
%	  		\end{itemize}\end{varwidth}
%	};
%	\node at ( 3,-3.5) [label=above:\bf{Passos}] 
%	{
%		\begin{varwidth}{\linewidth}\begin{itemize}
%			\item a)
%		
%			\begin{itemize}
% 			\item Encontre a equação de velocidade\\de cada carro e subtraia vetorialmente
%			\end{itemize}
%      	
%			\item b)
%	
%			\begin{itemize}
% 			\item Encontre a posição de cada\\carro no momento pedido
%			\item Encontre o modulo da diferença\\dos vetores de posição
%			\end{itemize}
%			
%			\item c)
% 			\begin{itemize}
% 			\item Subtraia os vetores de\\ aceleração de cada carro
%			\end{itemize}
%			
%	  	\end{itemize}\end{varwidth}
%	};
%	
%\end{tikzpicture}

\end{document}









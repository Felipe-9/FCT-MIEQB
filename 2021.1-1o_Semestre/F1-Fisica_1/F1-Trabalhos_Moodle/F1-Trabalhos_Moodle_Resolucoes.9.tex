\documentclass[12pt]{article}

% Geometry
\usepackage{geometry}
\geometry{
	a4paper, % 210mm por 297mm
	top=15mm,
	left=15mm,
	right=15mm
}

% Linguagem
\usepackage[portuguese]{babel}	% Babel
%\usepackage{polyglossia}		% Polyglossia
%\setdefaultlanguage[variant=brazilian]{portuguese}

% Graphics
%\usepackage{graphics}

% calc
\usepackage{calc}

% Clickable Table of contents
\usepackage{hyperref}
\hypersetup{
	hidelinks=true,
	colorlinks=true,
	linkcolor=DarkGreen!20!LightGreen!25!
}

% Table of contents
%\usepackage{tocloft}
%\setlength{\cftsecnumwidth}{25mm}		% Fix section width
%\setlength{\cftsubsecnumwidth}{15mm}
%% Fix space between subsection items on toc
%\renewcommand\cftsubsecafterpnum{\vskip5pt}

% Multicols
\usepackage{multicol}

% Customize Chapter
%\usepackage{titlesec}
%\titleformat{\chapter}[hang]{\Huge\bfseries\color{DarkGreen!75!}}{\thechapter\hspace{20pt}{$|$}\hspace{20pt}}{0pt}{\Huge\bfseries}

% Appendix
%\usepackage{appendix}

% siunix: SI units
\usepackage{siunitx}
\sisetup{
	scientific-notation = engineering,
	exponent-to-prefix = true,
	exponent-product = *,
	round-mode = places,
	round-precision = 2,
%	round-minimum = 0.01
}

% Maths
\usepackage{amssymb} 		
\usepackage{amsmath} 

\newcommand{\bm}[1]{{\boldmath{\large{\begin{align*} #1 \end{align*}}}}}

% Vectors
\usepackage{esvect} 	% Vector over-arrow
\renewcommand{\vec}{\vv} % Vecto over-arrow

% Tikz
\usepackage{tikz}
\usetikzlibrary{positioning} % Advanced positioning
\usepackage{pgfmath}  	% calculations
%\usepackage{varwidth}  % List inside TikzPicture

% Chem
%\usepackage{chemformula} 	% formulas quimicas
%\usepackage{chemfig} 		% Estruturas quimicas

%\newcommand{\mol}[1]{ \text{mol}_{\ch{ #1 }} } % mol

% Tabular
%\usepackage{multirow}

% Colors
\usepackage{xcolor}

\definecolor{DarkBlue}	{HTML}{252A36}
\definecolor{LightGreen}{HTML}{7CCC6C}
\definecolor{DarkGreen}	{HTML}{008675}

\colorlet{White}{DarkGreen!20!}
\colorlet{Black}{DarkBlue!110!}

\pagecolor{Black}
\color{White}

%\definecolor{Red}  {HTML}{FF7E79}
%\definecolor{Blue} {HTML}{6666FF}
%\definecolor{Green}{HTML}{66FF66}

% Counters
%\counterwithin*{section}{part} % Reset section on part

% Section and Subsection Customization
%\renewcommand\thesection{Questão \arabic{section}}
%\renewcommand\thesubsection{%
%	\arabic{section} - \alph{subsection})%
%}
%\renewcommand\thesubsubsection{(\roman{subsubsection})}

\begin{document}

\title{\bfseries\color{DarkGreen!75!}%
	Física 1 - T9
\\	Diagrama de fluxo - Corpo num loop sem atrito%
}
\author{Felipe Pinto - 61387}


\begin{maketitle}
 
\end{maketitle}

\begin{center}
\begin{tikzpicture}%
[
	scale = 1, font=\small, line width = 1,
	cell/.style = {
		rectangle, draw, fill=DarkGreen!15!Black,
		inner sep=3mm
	}
]
	
	\node (1) [cell] 
	{$v_a = \sqrt{2\,\Delta E_{k}/m} - v_0$};
	
	\node (2) [cell, right=of 1]
	{$v_0 = 0$};
	
	\node (3) [cell, right=of 2] 
	{$\Delta E_{k} = \Delta E_{g} = m\,g\,\Delta h$};
	
	\node (4) [cell, below=of 1] {$\implies$};
	
	\draw (1) -- (4);
	\draw (2) -- (4);
	\draw (3) -- (4);
	
	\node (5) [cell, below=of 4]
	{$
		v_{a}
	=	\sqrt{2\,m\,g\,\Delta h/m}
	$};
	
	\draw[->] (4) -- (5);
	
	\node (6) [cell, right=of 5]
	{$ \Delta h = (3.5-2)\,R $};
	
	\node (7) [cell, below=of 5]{$\implies$};
	\draw (6) -- (7);
	\draw (5) -- (7);
	
	\node (8) [cell, below=of 7]
	{$ v_{a} = \sqrt{3\,g\,R} $};
	
	\draw[->] (7)--(8);
	
	\node (9) [cell, right=of 8]
	{$ \vv{F}_{N} = \vv{a}\,m - \vv{F_{g}} $};
	
	\node (10) [cell, right=of 9]
	{$ a = v^2/R $};
	
	\node (11) [cell, below=of 8]{$\implies$};
	
	\draw (8) -- (11) -- (9) (10) -- (11);
	
	\node (12) [cell, below=of 11]
	{$ 
		\vv{F_{N}} 
	= 	(m\,(\sqrt{3\,g\,R})^2/R-m\,g)\,(-\hat\jmath )
	=	-2\,m\,g\,\hat\jmath
	$};
	
	\draw[->] (11) -- (12);
	
	\node (13) [cell, right=of 12]
	{$ m = 5.00\unit{\,\gram} $};
	
	\node (14) [cell, right=of 13]
	{$ g = \num{9.80665} $};
	
	\node (15) [cell, below=of 12]{$\implies$};
	
	\draw (12) -- (15) (13) -- (15) (14) -- (15);
	
	\node (16) [cell, below=of 15]
	{$ \vv{F_{N}} = \qty{-0.0980665}{\newton}\,\hat\jmath $};
	
	\draw[->] (15) -- (16);
	

\end{tikzpicture}
\end{center}











\end{document}











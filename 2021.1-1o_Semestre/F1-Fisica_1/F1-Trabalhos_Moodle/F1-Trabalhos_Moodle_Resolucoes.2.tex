\documentclass[12pt]{article}

% Linguagem
\usepackage[portuguese]{babel}

% Clickable Table of contents
\usepackage[hidelinks]{hyperref}

% Multicols
%\usepackage{multicol}

% Maths
\usepackage{amssymb} 		
\usepackage{amsmath} 
\usepackage[utf8]{inputenc} %useful to type directly diacritic characters

\newcommand{\bm}[1]{{\boldmath{\large{\begin{align*} #1 \end{align*}}}}}

% Vectors
\usepackage{esvect} 	% Vector over-arrow
\usepackage{tikz}		% Vector diagrams
\usetikzlibrary{calc}  % Vector calculations
\usepackage{varwidth}  % List inside TikzPicture

\renewcommand{\vec}{\vv} % Vecto over-arrow

% Chem
%\usepackage{chemformula} 	% formulas quimicas
%\usepackage{chemfig} 		% Estruturas quimicas

% Colors
\usepackage{xcolor}

\definecolor{DarkBlue}	{HTML}{252A36}
\definecolor{LightGreen}{HTML}{7CCC6C}
\definecolor{DarkGreen}	{HTML}{008675}

\pagecolor{DarkBlue!110!}
\color{DarkGreen!20!}

% Resolução de listas
%\renewcommand\thesection{Questão \arabic{section} }
%\renewcommand\thesubsection{\arabic{section}-\alph{subsection}) }

\begin{document}

\title{Diagrama de fluxo\\(visualização) da construção da tabela 2x2}
\author{Felipe Pinto - 61387}

\maketitle
\break

\begin{tikzpicture}
[
	scale = 1, font=\small, line width = 1
]
		
	\node[rectangle, draw=DarkGreen!30!              ] (1) 
	{Ler questão};
	
	\node[rectangle, draw=DarkGreen!30!, below of = 1] (2) 
	{Anotar Factos/Dados};
	
	\node[rectangle, draw=DarkGreen!30!, below of = 2] (3) 
	{Anotar Pedido};
	
	\node[rectangle, draw=DarkGreen!30!, below of = 3] (4) 
	{Desenhar diagrama da situação};
	
	\node[rectangle, draw=DarkGreen!30!, below of = 4] (5) 
	{Equacionar situação em função do pedido};
	
	\node[rectangle, draw=DarkGreen!30!, below of = 5] (6) 
	{Encontrar formulas/relações matemáticas 
	 que relacionem Pedido e dado 
	 e anotar em paços};
	
	\foreach \i [evaluate=\i as \j using int(\i+1)] in {1,...,5}{
		
		\draw[->] (\i) to (\j);
			
	}
	
	
\end{tikzpicture}


\end{document}

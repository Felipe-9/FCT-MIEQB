\documentclass[12pt]{article}

% Linguagem
\usepackage[portuguese]{babel}

% Clickable Table of contents
\usepackage[hidelinks]{hyperref}

% Bibliografia
%\usepackage[nottoc]{tocbibind}

%\usepackage[style=numeric]{biblatex}
%\addbibresource{Untitled.bib}


% Multicols
%\usepackage{multicol}

% Maths
%\usepackage{amssymb} 		
%\usepackage{amsmath} 
%\usepackage[utf8]{inputenc} %useful to type directly diacritic characters

%\newcommand{\bm}[1]{{\boldmath{\large{\begin{align*} #1 \end{align*}}}}}

% Vectors
%\usepackage{esvect} 	% Vector over-arrow
%\usepackage{tikz}		% Vector diagrams

%\renewcommand{\vec}{\vv} % Vecto over-arrow

% Chem
%\usepackage{chemformula} 	% formulas quimicas
%\usepackage{chemfig} 		% Estruturas quimicas

% Colors
\usepackage{xcolor}

\definecolor{DarkBlue}	{HTML}{252A36}
\definecolor{LightGreen}{HTML}{7CCC6C}
\definecolor{DarkGreen}	{HTML}{008675}

\pagecolor{DarkBlue!110!}
\color{DarkGreen!20!}

% Resolução de listas
%\renewcommand\thesection{Questão \arabic{section} }
%\renewcommand\thesubsection{\arabic{section}-\alph{subsection}) }

\begin{document}

\title{Como é que obtive $20$ valores a Física I}
\author{Felipe Pinto - 61387}

\maketitle
\tableofcontents
\break

\section{Introdução}
\quad Um eficiente aprendizado de alguma matéria necessita de boa execução de três momentos: Aulas com professores, Estudo teórico, Estudo Prático. Assim se aumenta a possibilidade de se conseguir 20 valores na cadeira de Física I.

\section{Conduta durante aulas}
\quad	Durante a aula de um professor, o aluno deve extrair o máximo de informações do conteúdo apresentado, finalizar a aula com duvidas e não tratar das mesmas no momento tem grande potencial de trazer atrasos na matéria. Dessa forma é necessário focar em cada segundo da aula, inclusive durante exemplos e duvidas de outros alunos.

\section{Estudo Teórico}
\quad	Prescindindo o estudo prático, o estudo teórico se resume a leitura e resolução de exemplos contidos no livro da matéria, a escolha de um bom livro irá afetar bastante o andamento e o conteúdo abrangido pelo estudo teórico, dessa forma deve-se pedir ao professor uma recomendação bibliográfica, no meu caso usei Physics for Scientists and Engineers Extended Version do Tipler, P.A. e Mosca, G.. Se varia bastante o tempo em que cada matéria é absorvida, porem se deve ter em mente um estudo que dure $70\%$ da quantidade de aulas da cadeira.

\section{Estudo Prático}
\quad Findando o estudo teórico, se inicia a o momento prático, que se resume a resolução de lista de exercícios e exercícios do livro, é de interesse do aluno manter um documento com as resoluções dos exercícios para que futuramente uma revisita ao conteúdo seja facilitada, dessa forma as listas de exercícios devem ser documentadas, e os exercícios extra feitos do livro devem ser rascunhadas, assim aumentando o numero de questões que o aluno pode fazer. Esse momento dura no mínimo 30\% da quantidade de aulas da cadeira, porem a medida que o aluno se encontra com tempo livre, pode se permitir a extender o período.

\section{Conclusão}
\quad A principio, o estudo teórico será suficiente para a compreensão do conteúdo, o estudo prático permite a fixação do conteúdo, criação de um documento com resoluções, auto avaliação alem de proporcionar experiencia para variados casos de problemas. Seguindo essas orientações um tem grandes chances de adquirir 20 valores na nota de qualquer matéria.


\end{document}















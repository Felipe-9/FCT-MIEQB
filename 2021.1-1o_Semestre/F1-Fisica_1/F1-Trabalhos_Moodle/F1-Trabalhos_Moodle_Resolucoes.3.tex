\documentclass[12pt]{article}

% Linguagem
\usepackage[portuguese]{babel}

% Clickable Table of contents
\usepackage[hidelinks]{hyperref}

% Multicols
%\usepackage{multicol}

% Maths
%\usepackage{amssymb} 		
%\usepackage{amsmath} 
%\usepackage[utf8]{inputenc} %useful to type directly diacritic characters

%\newcommand{\bm}[1]{{\boldmath{\large{\begin{align*} #1 \end{align*}}}}}

% Vectors
%\usepackage{esvect} 	% Vector over-arrow
%\usepackage{tikz}		% Vector diagrams

%\renewcommand{\vec}{\vv} % Vecto over-arrow

% Chem
%\usepackage{chemformula} 	% formulas quimicas
%\usepackage{chemfig} 		% Estruturas quimicas

% Colors
\usepackage{xcolor}

\definecolor{DarkBlue}	{HTML}{252A36}
\definecolor{LightGreen}{HTML}{7CCC6C}
\definecolor{DarkGreen}	{HTML}{008675}

\pagecolor{DarkBlue!110!}
\color{	  DarkGreen!20!}

% Resolução de listas
%\renewcommand\thesection{Questão \arabic{section} }
%\renewcommand\thesubsection{\arabic{section}-\alph{subsection}) }

\begin{document}

\title{Reflexão sobre Vectores}
\author{Felipe Pinto - 61387}

\maketitle
\tableofcontents
\break

% Sucintamente, indique o que aprendeu ao realizar os exercícios;
% Indique quantos exercícios resolveu;
% O que fez para superar as suas dificuldades;
% Use até 100 palavras;
% Entregue esta tarefa até ao prazo indicado.

\section{Introdução}
\quad Vetores são uma ferramenta matemática extremamente util e aplicável em situações físicas, e adicionam uma característica não usual a numero alem de seu valor e unidade.

\section{Resolução de Exercícios }
\quad 31 exercícios, poder acomodar vetores em termos de versões criados de outro da uma nova perspectiva a relatividade de um problema. também há de escrever vetores em diferentes sistemas de coordenadas, um sistema que se referencia por modulo e angulo facilitam bastante em situações de comportamento circular, porem somar vetores nesse sistema proporciona uma nova dificuldade vem sido melhor trabalhada com estudo teórico.









\end{document}










\documentclass[12pt]{article}

% Geometry
\usepackage{geometry}
\geometry{
	a4paper, % 210mm por 297mm
	top=15mm,
	left=15mm,
	right=15mm
}

% Linguagem
\usepackage[portuguese]{babel}	% Babel
%\usepackage{polyglossia}		% Polyglossia
%\setdefaultlanguage[variant=brazilian]{portuguese}

% Graphics
%\usepackage{graphics}

% calc
\usepackage{calc}

% Clickable Table of contents
\usepackage{hyperref}
\hypersetup{
	hidelinks=true,
	colorlinks=true,
	linkcolor=DarkGreen!20!LightGreen!25!
}

% Table of contents
%\usepackage{tocloft}
%\setlength{\cftsecnumwidth}{25mm}		% Fix section width
%\setlength{\cftsubsecnumwidth}{15mm}
%% Fix space between subsection items on toc
%\renewcommand\cftsubsecafterpnum{\vskip5pt}

% Multicols
\usepackage{multicol}

% Customize Chapter
%\usepackage{titlesec}
%\titleformat{\chapter}[hang]{\Huge\bfseries\color{DarkGreen!75!}}{\thechapter\hspace{20pt}{$|$}\hspace{20pt}}{0pt}{\Huge\bfseries}

% Appendix
%\usepackage{appendix}

% siunix: SI units
%\usepackage{siunitx}
%\sisetup{
%	scientific-notation = engineering,
%	exponent-to-prefix = true,
%	exponent-product = *,
%	round-mode = figures,
%%	round-precision = 3,
%%	round-minimum = 0.01
%}

% Maths
%\usepackage{amssymb} 		
%\usepackage{amsmath} 

%\newcommand{\bm}[1]{{\boldmath{\large{\begin{align*} #1 \end{align*}}}}}

% Vectors
%\usepackage{esvect} 	% Vector over-arrow
%\renewcommand{\vec}{\vv} % Vecto over-arrow

% Tikz
%\usepackage{tikz}		
%\usepackage{pgfmath}  	% calculations
%\usepackage{varwidth}  % List inside TikzPicture

% Chem
%\usepackage{chemformula} 	% formulas quimicas
%\usepackage{chemfig} 		% Estruturas quimicas

%\newcommand{\mol}[1]{ \text{mol}_{\ch{ #1 }} } % mol

% Tabular
%\usepackage{multirow}
%\usepackage{booktabs}

% Colors
\usepackage{xcolor}

\definecolor{DarkBlue}	{HTML}{252A36}
\definecolor{LightGreen}{HTML}{7CCC6C}
\definecolor{DarkGreen}	{HTML}{008675}

\colorlet{White}{DarkGreen!20!}
\colorlet{Black}{DarkBlue!110!}

\pagecolor{Black}
\color{White}

%\definecolor{Red}  {HTML}{FF7E79}
%\definecolor{Blue} {HTML}{6666FF}
%\definecolor{Green}{HTML}{66FF66}

% Counters
%\counterwithin*{section}{part} % Reset section on part

% Section and Subsection Customization
%\renewcommand\thesection{Questão \arabic{section}}
%\renewcommand\thesubsection{%
%	\arabic{section} - \alph{subsection})%
%}
%\renewcommand\thesubsubsection{(\roman{subsubsection})}

\begin{document}

\title{\bfseries\color{DarkGreen!75!}%
	Física 1 - T8
\\	Comparações significativas
	(1$^a$ actividade de um conjunto de duas)%
}
\author{Felipe Pinto - 61387}


\maketitle


\begin{center}
\begin{tabular}{| c | p{7cm} | p{7cm} | c |}
	
	\cline{2-3}
	
	\multicolumn{1}{c}{}
& 	\multicolumn{2}{| c |}{\textbf{Diferenças}}
& 	\multicolumn{1}{c}{}
	
	\\ \cline{2-3}
	
	\multicolumn{1}{c |}{}
&	Características únicas da velocidade
	
& 	Características únicas do momento linear
	
& 	\multicolumn{1}{ c}{}

	\\ \cline{2-3}
	
	
	\multicolumn{1}{c |}{}
	
&	\begin{itemize}
		
		\item mede a variação de espaço no tempo
		
	\end{itemize}
	
& 	\begin{itemize}
		
		\item mede a variação da energia cinética com a velocidade
		
	\end{itemize}
	
& 	\multicolumn{1}{ c}{}

	\\ \hline
	
	\multicolumn{4}{|c|}{\textbf{Semelhanças}}
	
	\\ \hline
	
	\multicolumn{4}{| p{14cm} |}%
	{
	\begin{itemize}
		
		\item Grandezas Vetoriais
		\item Ambas são informações relacionadas a cinética do objeto
		\item Ambas podem ser usadas para determinar a variação de energia cinética de um sistema
		\item são diretamente proporcionais uma com a outra
		\item Sensíveis a aplicação de força
		\item possuem modulos iguais a zero na ausencia de movimento
		\item Ambas variam com referenciais em movimento
		
	\end{itemize}
	}
	
	\\ \hline
	
\end{tabular}
\end{center}









\end{document}











\documentclass[12pt]{article}

% Linguagem
\usepackage[portuguese]{babel}

% Clickable Table of contents
\usepackage{hyperref}
\hypersetup{
	hidelinks=true,
	colorlinks=true,
	linkcolor=DarkGreen!20!LightGreen!25!
}

% Table of contents
\usepackage{tocloft}
\setlength{\cftsubsecnumwidth}{3em} % Fix subsection width
% Fix space between subsection items on toc
\renewcommand\cftsubsecafterpnum{\vskip5pt}

% Multicols
%\usepackage{multicol}

% Customize Chapter
\usepackage{titlesec}

\titleformat{\chapter}[hang]{\Huge\bfseries\color{DarkGreen!75!}}{\thechapter\hspace{20pt}{$|$}\hspace{20pt}}{0pt}{\Huge\bfseries}

% Appendix
%\usepackage{appendix}

% Maths
\usepackage{amssymb} 		
\usepackage{amsmath} 
\usepackage[utf8]{inputenc} %useful to type directly diacritic characters

\newcommand{\bm}[1]{{\boldmath{\large{\begin{align*} #1 \end{align*}}}}}

% Vectors
\usepackage{esvect} 	% Vector over-arrow
\renewcommand{\vec}{\vv} % Vecto over-arrow

% Tikz
\usepackage{tikz}		% Vector diagrams
\usetikzlibrary{calc}  % Vector calculations
\usepackage{varwidth}  % List inside TikzPicture

% Chem
%\usepackage{chemformula} 	% formulas quimicas
%\usepackage{chemfig} 		% Estruturas quimicas

% Colors
\usepackage{xcolor}

\definecolor{DarkBlue}	{HTML}{252A36}
\definecolor{LightGreen}{HTML}{7CCC6C}
\definecolor{DarkGreen}	{HTML}{008675}

\pagecolor{DarkBlue!110!}
\color{DarkGreen!20!}

%\definecolor{Red}  {hsb}{0  ,.6,1}
%\definecolor{Blue} {hsb}{0.6,.6,1}
%\definecolor{Green}{hsb}{0.3,.6,1}

% Counters
%\counterwithin*{section}{part} % Reset section on part

% Resolução de listas
%\renewcommand\thepart{\arabic{part} }
\renewcommand\thesubsection{\thesection\ - \alph{subsection}) }

\begin{document}

\title{\bfseries\color{DarkGreen!75!}%
	F1 - Lista S5\\%
	Força e Movimento II%
}
\author{Felipe Pinto - 61387}

\maketitle
\tableofcontents
\break

% Questões

{\bfseries\color{DarkGreen!75!}	
	\part{Questões}
}
\renewcommand\thesection{Q\arabic{section} }

% Q1
\section{}
O ponto mais baixo pois sua normal será máxima igual a centrípeta mais gravidade

% Q2
\section{}
Para girar o balde de agua de forma que ele percorra o circulo sua velocidade deve ser tal que a aceleração centrípeta seja no mínimo de mesmo modulo que a da gravidade, dessa forma no ponto mais alto sua velocidade vai ser tangencial seguindo o percurso do balde.



% Problemas

{\bfseries\color{DarkGreen!75!}	
	\part{Problemas}
}
\renewcommand{\thesection}{P\arabic{section}}

% P1
\section{}

\subsection{}
\begin{flalign*}
&
		\vec a_{1} = a\,\hat\jmath
	;\ 
		\vec a_{2} = a\,\hat\imath
	;\
		\vec a_{3} = -a\,\hat\jmath
	;\ a\,m_2\,\hat\imath 
	= 
		g\,m_2 - g\,m_1 - \mu\,(g\,m_2 - g\,m_1)
&
\end{flalign*}

\setcounter{section}{8}

% P9
\section{}

% P9 - a)
\subsection{}
\begin{flalign*}
&
	\text{D}(|\vec{v}|) 
	=
		[v_{min},v_{max}]
	; &\\& \left\{
		v_{min}^2\,\hat{r}/R = \vec a_c
	;\
		\left| \sum \vec F \right|
	=
		m\,a_c\,\cos(\theta)
		+\mu\,(
			m\,g\,\cos(\theta)
			+m\,a_c\,\sin(\theta)
		) + \right. &\\& \left. -
		m\,g\,\sin{\theta}
	= 
		0
		\quad\forall\,\theta\in (0,90)
	\implies
		v_{min}
	=
		\sqrt{R\,
			\frac{m\,g\,\sin(\theta)
				 -\mu\,m\,g\,\cos(\theta)}
				{m\,\cos(\theta)-\mu\,m\,\sin(\theta)}
		}
	= \right. &\\& \left. =
		\sqrt{R\,g\,
			\frac{\sin(\theta)-\mu\,\cos(\theta)}
				{\cos(\theta)-\mu\,\sin(\theta)}
		}
		\quad\forall\,\theta\in (0,90)
	\right\};
&\\\\& 
	\left\{
		v_{max}^2\,\hat r/R=\vec a_c
	;\
		\left| \sum \vec F \right|
	=
		m\,a_c\,\cos(\theta)
		-\mu\,(
			m\,g\,\cos(\theta)
			+m\,a_c\,\sin(\theta)
	+ \right. &\\& \left. -
		m\,g\,\sin(\theta)
	=
		0
		\quad\forall\,\theta\in (0,90)
	\implies
		v_{max}
	=
		\sqrt{R\,
			\frac{m\,g\,\sin(\theta)+\mu\,m\,g\cos(\theta)}
				{m\,\cos(\theta)-\mu\,m\,\sin(\theta)}
		}
	= \right. &\\& \left. =
		\sqrt{R\,g\,
			\frac{\sin(\theta)+\mu\,\cos(\theta)}
				{\cos(\theta)-\mu\,\sin(\theta)}
		}
		\quad\forall\,\theta\in (0,90)
	\right\} 
	\implies &\\& \implies
		\text{D}(|\vec v|) 
	=
	\left\{ 
		v\in\mathbb{R}: 
		\sqrt{R\,g\,
			\frac{\sin(\theta)-\mu\,\cos(\theta)}
				{\cos(\theta)+\mu\,\sin(\theta)}
		}
		< v
		< 
			\sqrt{R\,g\,
			\frac{\sin(\theta)+\mu\,\cos(\theta)}
				{\cos(\theta)-\mu\,\sin(\theta)}
		}
	\right. &\\& \left.
		\forall\,\theta\in (0,90)
	\right\}
&
\end{flalign*}

% P9 - b)
\subsection{}
\begin{flalign*}
&
	\sqrt{R\,g\,
			\frac{\sin(\theta)-\mu\,\cos(\theta)}
				{\cos(\theta)+\mu\,\sin(\theta)}
	} = v_{min} = 0
	\implies
		\mu = \sin(\theta)/\cos(\theta)=\tan(\theta)
&
\end{flalign*}

\subsection{$
	R=100\,m \quad \theta = 10^o \quad \mu=0.10	
$}
\begin{flalign*}
&
	\text{D}(|\vec v|) = [v_{min},v_{max}]
	; &\\&
	\left\{
		v_{min} 
		= 
			\sqrt{R\,g\,
				\frac{\sin(\theta)-\mu\,\cos(\theta)}
					{\cos(\theta)+\mu\,\sin(\theta)}
			}
		\cong 8.6\,m/s
	\right\} &\\& \left\{
		v_{max}
		=
			\sqrt{R\,g\,
				\frac{\sin(\theta)+\mu\,\cos(\theta)}
					{\cos(\theta)-\mu\,\sin(\theta)}
			}
		\cong 17\,m/s
	\right\} &\\& \implies
		\text{D}(|\vec v|)=(8.6,16.6)
&
\end{flalign*}

\end{document}












\documentclass[12pt]{article}

% Linguagem
\usepackage[portuguese]{babel}

% Clickable Table of contents
\usepackage{hyperref}
\hypersetup{
	hidelinks=true,
	colorlinks=true,
	linkcolor=DarkGreen!20!LightGreen!25
}

% Table of contents
%\usepackage{tocloft}
%\setlength{\cftsubsecnumwidth}{3em} % Fix subsection width
% Fix space between subsection items on toc
%\renewcommand\cftsubsecafterpnum{\vskip5pt}

% Multicols
\usepackage{multicol}

% Customize Chapter
%\usepackage{titlesec}
%\titleformat{\chapter}[hang]{\Huge\bfseries\color{DarkGreen!75!}}{\thechapter\hspace{20pt}{$|$}\hspace{20pt}}{0pt}{\Huge\bfseries}

% Appendix
%\usepackage{appendix}

% Maths
\usepackage{amssymb} 		
\usepackage{amsmath} 
\usepackage[utf8]{inputenc} %useful to type directly diacritic characters

\newcommand{\bm}[1]{{\boldmath{\large{\begin{align*} #1 \end{align*}}}}}

% Vectors
\usepackage{esvect} 	% Vector over-arrow
\renewcommand{\vec}{\vv} % Vecto over-arrow

% Tikz
\usepackage{tikz}
\usetikzlibrary{calc}	% Coordinate calc
%\usepackage{varwidth}	% List inside TikzPicture

% Chem
%\usepackage{chemformula} 	% formulas quimicas
%\usepackage{chemfig} 		% Estruturas quimicas

%\newcommand{\mol}[1]{ \text{mol}_{\ch{ #1 }} } % mol

% Tabular
%\usepackage{multirow}
%\usepackage{siunitx} % Column S: align on decimal


% Colors
\usepackage{xcolor}

\definecolor{DarkBlue}	{HTML}{252A36}
\definecolor{LightGreen} {HTML}{7CCC6C}
\definecolor{DarkGreen}  {HTML}{008675}

\colorlet{White}{DarkGreen!20!}
\colorlet{Black}{DarkBlue!110!}

\pagecolor{Black}
\color{White}

\definecolor{Red}  {HTML}{FF7E79}
\definecolor{Blue} {HTML}{6666FF}
\definecolor{Green}{HTML}{66FF66}

% Counters
\counterwithin*{section}{part} % Reset section on part


\begin{document}

\title{\bfseries\color{DarkGreen!75!}%
	Física 1 - Ficha 3\\
	Movimento bi e tri dimensional%
}
\author{Felipe Pinto - 61387}
\date{}

\maketitle
\tableofcontents
\break

\part{Questões}

% Questões
\renewcommand\thesection{Questão \arabic{section} }
\renewcommand\thesubsection%
{%
	Q\arabic{section} - \alph{subsection}) %
}

\begin{multicols}{2}

% Q1
\section{}

\begin{tikzpicture}[
	scale=1.4, font=\small, align=center, line width=1,
	every node/.append style={minimum size=1, inner sep=2}
]
	
	\clip (-1.5,-0.9) rectangle (2.5,2.7);
	\draw[ultra thin, Black!85!] (-2,-1) grid (3,3);
	
	\node (0) at (0  ,1.732)
	[circle, fill, label=above:$ 0 $]{};
	
	\node (1) at (-1 ,0    )
	[circle, fill, label=below:$ 1 $]{};
	
	\node (2) at (0.8,0.3  )
	[circle, fill, label=below:$ 2 $]{};

	\draw[->, Green]
	(0) to node[above left]{$\vec v_0$} (1);
	
	\draw[->, Blue ]
	(1) to node[below]{$\vec v_1$} (2);
	
	\draw[->, Red  ] 
	(1) -- node[below right]{$\vec v_1-\vec v_0$}
	($ (1) + (2)-(1) + (0)-(1) $);
	
\end{tikzpicture}

% Q2
\section{}
\begin{tikzpicture}[
	scale=8.0, font=\small, align=center, line width=1
]
	
	\def\maxX { 0.35};
	\def\maxY { 0.35};
	\def\minX {-0.35};
	\def\minY {-0.35};
	
	\clip (\minX,\minY) rectangle (\maxX,\maxY);
	
	\coordinate (MinXY) at (\minX,\minY);
	
	\draw[ultra thin, Black!85! , step=0.1]
	(\minX,\minY) grid (\maxX,\maxY);
	
	\draw[-,thick]
	(\minX,0.05) -- (-0.05,0.05) -- (-0.05,\maxY);
	
	\draw[-,thick]
	(\minX,-0.05) -- (-0.05,-0.05) -- (-0.05,\minY);
	
	\draw[-,thick]
	(\maxX,0.05) -- (0.05,0.05) -- (0.05,\maxY);
	
	\draw[-,thick]
	(\maxX,-0.05) -- (0.05,-0.05) -- (0.05,\minY);
	
	\node at (0,\maxY-0.05){N};
	\node at (\maxY-0.05,0){E};
	
	% v0
	\draw[->, Blue]
	(0,-0.1) node[circle,fill=White]{}
	-- 
	+(0,0.16) node[above]{$\vec v_0$};
	
	% v1
	\draw[->, Red]
	(-0.1,0) node[circle,fill=White]{}
	--
	+(-0.12,0) node[left]{$\vec v_1$};
 	
	% delta v
	\coordinate (ref) at (0.3,0.1);
	
	\draw[->, Blue, shift={(ref)}]
	(0,0.16) -- (0,0) node[above left]{$-\vec v_0$};
	
	\draw[->, Red, shift={(ref)}]
	(0,0) -- +(-0.12,0) node[left]{$\vec v_1$};
	
	\draw[->, Green,shift={(ref)}]
	(0,0.16) 
	-- node[sloped,above]{$\vec v_1 - \vec v_0$}
	(-0.12,0);
	
\end{tikzpicture}


\end{multicols}

% Q3
\section{}

\begin{tikzpicture}[
	scale=8.0, font=\small, align=center, line width=1,
	every node/.append style={minimum size=1, inner sep=2}
]
	
	\def\maxX { 0.85};
	\def\maxY { 0.35};
	\def\minX {-0.85};
	\def\minY {-0.35};
	
	\clip (\minX,\minY) rectangle (\maxX,\maxY);
	
	\coordinate (MinXY) at (\minX,\minY);
	
	\draw[ultra thin, Black!85! , step=0.1]
	(\minX,\minY) grid (\maxX,\maxY);

	
	\draw
	(-0.8,0.3) 
	.. 
	controls (-0.1,-0.6)
	and (0.25,0)
	..
	(0.4,0)
	--
	(0.8,0);
	
	\node at (-0.7,0.25)
	[circle, fill, inner sep=6.5]{};
	
	\node (A) at (-0.35,-0.05)
	[circle, fill, label=above:$A$]{};
	
	\node (B) at (-0.1,-0.15)
	[circle, fill, label=left:$B$]{};
	
	\node (C) at (0.1,-0.1)
	[circle, fill, label=above:$C$]{};
	
	\node (D) at (0.4,0.05)
	[circle, fill, label=above:$D$]{};
	
	\node (E) at (0.6,0.05)
	[circle, fill, label=above:$E$]{};
	
	\draw[->, Blue]
	(A) -- +(0.1,-0.06);
	
	\draw[->, Blue]
	(B) -- +(0,0.1);
	
	\draw[->, Blue]
	(C) -- +(-0.08,-0.04);
	
	\node[left=5, color=Blue] at (D){$\vec a = \vec 0$};
	
	\node[right=5, color=Blue] at (E){$\vec a = \vec 0$};


\end{tikzpicture}

% Q4
\section{}
\begin{tikzpicture}[
	scale=1.0, font=\small, align=center, line width=1,
%	rotate = -90
]

	\def\maxX { 3.5};
	\def\maxY { 4.5};
	\def\minX {-3.5};
	\def\minY {-4.5};
	
	\clip (\minX,\minY) rectangle (\maxX,\maxY);
	
	\coordinate (MinXY) at (\minX,\minY);
	
	\draw[ultra thin, Black!85! , step=1]
	(\minX,\minY) grid (\maxX,\maxY);

	

	\node (A) at (-2,3)
	[circle, fill]{};
	
	\node (B) at (0,2.3)
	[circle, fill]{};
	
	\node (C) at (1.5,1)
	[circle, fill]{};
	
	\node (D) at (2,0)
	[circle, fill]{};
	
	\node (E) at (1.5,-1)
	[circle, fill]{};
	
	\node (F) at (0,-2.3)
	[circle, fill]{};
	
	\node (G) at (-2,-3)
	[circle, fill]{};
	
	\draw[->, Blue] (A) -- +(-15 :1.0);
	\draw[->, Blue] (B) -- +(-30 :0.8);
	\draw[->, Blue] (C) -- +(-50 :0.6);
	\draw[->, Blue] (D) -- +(-90 :0.5);
	\draw[->, Blue] (E) -- +(-130:0.6);
	\draw[->, Blue] (F) -- +(-150:0.8);
	\draw[->, Blue] (G) -- +(-165:1.0);
	
\end{tikzpicture}

\setcounter{section}{6}

% Q7
\section{}
Sim pois ambos possuem a mesma velocidade no plano xy.

% Q8
\section{}

\begin{multicols}{2}

% Q8 - a)
\subsection{}
O lançado na lua, pois seu campo gravitacional é mais fraco assim o projetil lançado sobre ela possui uma menor aceleração.

% Q8 - b)
\subsection{}
O lançado na lua, pelos mesmos motivos.

\end{multicols}

% Q9
\section{}

\begin{multicols}{2}

% Q9 - a)
\subsection{}
Não, a aceleração sempre é vertical para baixo enquanto a sua velocidade possui uma componente horizontal.

% Q9 - c)
\subsection{}
$v_x, a_x, a_y$

% Q9 - b)
\subsection{}
Sim, no ponto mais alto de sua trajetória a componente vertical da velocidade é zero lhe garantindo direção horizontal que é perpendicular a aceleração.

\end{multicols}

\begin{multicols}{2}

% Q10
\section{}
\begin{flalign*}
&
	3000\,rpm=3000\,\frac{r}{m}\,\frac{1\,m}{60\,s}=50\,rps
&
\end{flalign*}

\setcounter{section}{11}

% Q12
\section{}

\setcounter{subsection}{1}

% Q12 - b)
\subsection{}
\begin{flalign*}
&
	v_3>v_2=v_1
&
\end{flalign*}

\end{multicols}

% Q13
\section{}

\begin{multicols}{2}

% Q13 - a)
\subsection{}
\begin{flalign*}
&
	\vec v_{p\,a} = \vec v_{p\,b} + \vec v_{b\,a} 
&
\end{flalign*}

% Q13 - b)
\subsection{}
\begin{flalign*}
&
	\vec v_{p\,a}=(5+2)\,\hat v = 7\,\hat v
&
\end{flalign*}

\end{multicols}

\setcounter{section}{14}

% Q15
\section{}

\begin{multicols}{2}

% Q15 - a)
\subsection{}
\begin{flalign*}
&
	v_{a\,R} > v_{a\,S} > v_{a\,T}
&
\end{flalign*}

% Q15 - b)
\subsection{}
\begin{flalign*}
&
	a_{a\,R} = a_{a\,S} = a_{a\,T}
&
\end{flalign*}

\end{multicols}

% Q16
\section{}

\begin{multicols}{2}

% Q16 - a)
\subsection{}
\begin{flalign*}
&
	a_{c} = m\,g
&
\end{flalign*}

% Q16 - b)
\subsection{}
\begin{flalign*}
&
	a_{s} = m\,g
&
\end{flalign*}

\end{multicols}

% Problemas
\part{Problemas}
\renewcommand{\thesection}{Problema \arabic{section} }
\renewcommand\thesubsection%
{%
	P\arabic{section} - \alph{subsection}) %
}

% P1
\section{}

% P1 - a)
\subsection{}
\begin{flalign*}
&
	\Delta S_i\,\hat = v_c\,\Delta t
	;\
		a_g\,(\Delta t)^2/2=\Delta S_j
	\implies
		\Delta S_i 
	= 
		v_c\,\sqrt{\frac{2\,\Delta S_j}{a_g}}
	=
		10.0\,\sqrt{\frac{2*3.00}{9.80665}}
	\cong
		7.82\,m
&
\end{flalign*}

% P1 - b)
\subsection{}
\begin{flalign*}
&
	v_c\,\Delta t = \Delta S_i\,\hat
	\implies
		\Delta t 
	= 
		\Delta S_i/v_c
	=
		7.82/10.0
	\cong 0.782\,s
&
\end{flalign*}

% P2
\section{}

% P2 - a)
\subsection{}
\begin{flalign*}
&
	\vec a\,\Delta t = \Delta \vec v
	\implies
		\vec a 
	= 
		(\vec v - \vec v_0)/\Delta t
	=
		(
		20.0\,\hat\imath 
		- 5.0\,\hat\jmath
		- 4.0\,\hat\imath 
		- 1.0\,\hat\jmath
		)/20.0
	\cong &\\& \cong
		(0.8\,\hat\imath - 0.3\,\hat\jmath)\,m/s^2
&
\end{flalign*}

% P2 - b)
\subsection{}
\begin{flalign*}
&
	\sqrt{a_i^2+a_j^2}\,\cos(\theta) = a_i
	\implies
		\theta 
	= 
		\arccos\left( \frac{a_i}{\sqrt{a_i^2+a_j^2}} \right)
	=
		\arccos\left( \frac{0.8}{\sqrt{0.8^2+0.3^2}} \right)
	\cong &\\& \cong
		21^o
&
\end{flalign*}

% P2 - c)
\subsection{}
\begin{flalign*}
&
	\vec r_{(25)} 
	= 
		\vec r_{(0)} 
		+ \vec v_0\,\Delta t_1 
		+ \vec a_0\,(\Delta t_1)^2/2
		+ \vec v_1\,\Delta t_2
	;\
		\Delta t_2 = \Delta t_{Total} - \Delta t_1
	\implies &\\& \implies
		\vec r_{(25)}
	=
		1.0\,\hat\imath - 4.0\,\hat\jmath
		+ (4.0\,\hat\imath + 1.0\,\hat\jmath)\,20.0
		+ (0.8\,\hat\imath - 0.3\,\hat\jmath)\,(20.0)^2/2
		+ &\\& + 
		(20.0\,\hat\imath - 5.0\,\hat\jmath)\,5.0
	\cong (341\,\hat\imath - 69\,\hat\jmath)\,m
&
\end{flalign*}

% P3
\section{}
\bm{ 
	r_{(t)} = (
		- 5.0\,\sin(t)			\,\hat\imath
		+ (4.0 - 5.0 \,\cos(t))	\,\hat\jmath
	)\,m
}

% P3 - a)
\subsection{}
\begin{flalign*}
&
	v_0 = d\,r_{(0)}/d\,t 
	=
		(
		- 5.0\,\cos(0)	\,\hat\imath
		+ 5.0\,\sin(0)	\,\hat\jmath
		)
	=
		(
		- 5.0	\,\hat\imath
		+ 0.0	\,\hat\jmath
		)
&\\\\&
	a_0 = d^2\,r_{(0)}/(d\,t)^2
	=
		(
		+ 5.0\,\sin(0)	\,\hat\imath
		+ 5.0\,\cos(0)	\,\hat\jmath
		)
	=
		(
		+ 0.0	\,\hat\imath
		+ 5.0	\,\hat\jmath
		)
&
\end{flalign*}

% P3 - b)
\subsection{}
\begin{flalign*}
&
	r_{(t)} = (
		- 5.0\,\sin(t)			\,\hat\imath
		+ (4.0 - 5.0 \,\cos(t))	\,\hat\jmath
	)\,m
&\\\\&
	v_{(t)} = d\,r_{(t)}/d\,t
	=
		(
		- 5.0\,\cos(t)	\,\hat\imath
		+ 5.0\,\sin(t)	\,\hat\jmath
		)
&\\\\&
	a_{(t)} = d^2\,r_{(t)}/(d\,t)^2
	=
		(
		+ 5.0\,\sin(t)	\,\hat\imath
		+ 5.0\,\cos(t)	\,\hat\jmath
		)
&
\end{flalign*}

% P4
\section{}
\begin{flalign*}
&
	v = v_j/\sin(50^o) = v_i/\cos(50^0)
	;\
		\sqrt{(0.4\,S_i)^2 + S_i^2} = 30
	;\ &\\&
		\Delta \vec S
	=
		(
			v_i\,\Delta t 
		)\,\hat\imath
		+ (
			v_j\,\Delta t
			- a_g\,(\Delta t)^2/2
		)\,\hat\jmath
	\implies &\\& \implies 
		0.4*30/\sqrt{1.16}	
	=	
		v\,\sin(50^o)\frac{30/\sqrt{1.16}}{v\,\cos(50^o)}
		- \frac{a_g}{2} 
		\left( 
			\frac{30}{v\,\cos(50^o)} 
		\right)^2
	\implies &\\& \implies
		v 
	=	
		\frac{(30/\sqrt{1.16})/\cos(50)}
		{
      		\sqrt{
      		\frac{2}{9.80665}\left(
      			\frac{30}	     {\sqrt{1.16}}
				\frac{\sin(50^o)}{\cos(50^o)}
      			- \frac{0.4*30}{\sqrt{1.16}}
      		\right)
      		}
		}
	\cong 20.4\,m/s
&
\end{flalign*}

% P5
\section{}
\begin{flalign*}
&
	a_g\,(\Delta t)^2/2 = 240
\implies
	\Delta t = \sqrt{2*240/9.80665} 
\cong 
	7.00\,s
&\\\\&
	v_{med} = 1.00\,K\,m /\Delta t
=
	\frac{1000}{\sqrt{2*240/9.80665}}
\cong
	143\,m/s
&
\end{flalign*}

% P6
\section{}
\begin{flalign*}
&
	v_0\,\Delta t_0 = \Delta S_i
;\	
	\Delta S = v_{som}\,\Delta t_1
;\
	\Delta t_1 + \Delta t_0 = 3.0\,s
;\
	a_g\,(\Delta t_0)^2/2 = \Delta S_j
\implies &\\& \implies
	v_0 
= 
	\frac {\sqrt{
		(v_{som}\, (3.0-\sqrt{2*\Delta S_j/a_g}))^2 
		- (\Delta S_j)^2 
	} }
	{ \sqrt{2*\Delta S_j/a_g} }
= &\\& =
	\frac {\sqrt{
		(343\, (3.0-\sqrt{2*40.0/9.80665}))^2 
		- (40.0)^2 
	} }
	{ \sqrt{2*40.0/9.80665} }
\cong
	10.11\,m/s
&\\\\& =
	\frac {\sqrt{
		(343\, (3.0-\sqrt{2*40.0/10}))^2 
		- (40.0)^2 
	} }
	{ \sqrt{2*40.0/10} }	
\cong 15.3\,m/s
&
\end{flalign*}

% P7
\section{}

% P7 - a)
\subsection{}
\begin{flalign*}
&
	\Delta S_{(\max\,i)} = v_{(\max\,i)}\,\Delta t_0
;\
	v_{(\max\,i)}\,\cos(45^o)
	= v_{(\max\,j)}\,\sin(45^o)
	= v_{\max}
;\ &\\&
	v_{\max}\,\Delta t_1 - a_g\,(\Delta t_1)^2/2 = h
;\ 
	v_{\max} = a_g\,\Delta t_1
;\
	v_{(\max\,j)} = a_g\,\Delta t_0
\implies &\\& \implies
	v_{max} = \sqrt{2\,a_g\,h}
;\
	\Delta s_{(\max\,i)}
=
	\frac{\sqrt{2\,a_g\,h}}{\cos(45^o)}\,
	\frac{\sqrt{2\,a_g\,h}/\sin(45^o)}{a_g}
= &\\& =
	\frac{2\,h}{\cos(45^o)\,\sin(45^o)}
&
\end{flalign*}

% P7 - b)
\subsection{}
\begin{flalign*}
&
	\Delta t_0 = v_{(\max\,j)}/a_g
	= \frac{\sqrt{2\,a_g\,h}/\sin(45^o)}{a_g}
	= 2\,\sqrt{h/a_g}
&\\&
	\Delta t_1 = 2\,v_{\max}/a_g
&
\end{flalign*}


\setcounter{section}{9}

% P10
\section{}
\bm{
	\vec a_J &= (3.0\,\hat\imath-2.0\,\hat\jmath)m/s^2\\
	\vec a_S &= (1.0\,\hat\imath+3.0\,\hat\jmath)m/s^2\\
	t_1	   &= 5\,s
}

% P10 - a)
\subsection{}
\begin{flalign*}
&	\vec v_{J(t_1)} - \vec v_{S(t_1)}
	= \vec a_j\,t_1 - \vec a_s\,t_1 
	= ( 
		 ( 3.0-1.0)\,\hat\imath
		+(-2.0-3.0)\,\hat\jmath 
	)\,(m/s^2)\,(5s) = &\\
&	= (
		10\,\hat\imath-25\,\hat\jmath
	)m/s &
\end{flalign*}

% P10 - b)
\subsection{}
\begin{flalign*}
&	  \| \vec P_{J(t_1)}-\vec P_{S(t_1)} \|
	= \left\| \vec a_J\,t_1^2/2-\vec a_S\,t_1^2/2 \right\| = &\\
&	= \left\| 
	(
		 ( 3.0-1.0)\,\hat\imath
		+(-2.0-3.0)\,\hat\jmath
	)(m/s^2)(5s)^2/2 \right\|
	= |
		 25 \,\hat\imath
		-62.5\,\hat\jmath
	|m
	\cong 67\,m &
\end{flalign*}

% P10 - c)
\subsection{}
\begin{flalign*}
&	\vec a_J-\vec a_S = ( 2.0\,\hat\imath-5.0\,\hat\jmath )m/s^2 &
\end{flalign*}

% Folha de Cálculo
\part{Folha de Cálculo}
\renewcommand\thesection{S\arabic{section}}
\renewcommand\thesubsection%
{%
	S\arabic{section} - \alph{subsection})%
}


\end{document}











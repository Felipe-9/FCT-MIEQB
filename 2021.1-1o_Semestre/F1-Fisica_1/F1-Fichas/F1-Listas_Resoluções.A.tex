\documentclass[12pt]{article}

% Geometry
\usepackage{geometry}
\geometry{
	a4paper, % 210mm por 297mm
	top=15mm,
	left=15mm,
	right=15mm
}

% Linguagem
\usepackage[portuguese]{babel}	% Babel
%\usepackage{polyglossia}		% Polyglossia
%\setdefaultlanguage[variant=brazilian]{portuguese}

% Graphics
%\usepackage{graphics}

% calc
\usepackage{calc}

% Clickable Table of contents
\usepackage{hyperref}
\hypersetup{
	hidelinks=true,
	colorlinks=true,
	linkcolor=DarkGreen!20!LightGreen!25!
}

% Table of contents
\usepackage{tocloft}
\setlength{\cftsecnumwidth}{28mm}		% Fix section width
\setlength{\cftsubsecnumwidth}{15mm}
\setlength{\cftsubsubsecnumwidth}{6mm}
%% Fix space between subsection items on toc
%\renewcommand\cftsubsecafterpnum{\vskip5pt}

% Multicols
\usepackage{multicol}

% Customize Chapter
%\usepackage{titlesec}
%\titleformat{\chapter}[hang]{\Huge\bfseries\color{DarkGreen!75!}}{\thechapter\hspace{20pt}{$|$}\hspace{20pt}}{0pt}{\Huge\bfseries}

% Appendix
%\usepackage{appendix}

% siunix: SI units
\usepackage{siunitx}
\sisetup{
	scientific-notation = engineering,
	exponent-to-prefix = true,
	exponent-product = *,
	round-mode = figures,
%	round-precision = 3,
%	round-minimum = 0.01
}

% Maths
\usepackage{amssymb} 		
\usepackage{amsmath} 

\newcommand{\bm}[1]{{\boldmath{\large{\begin{align*} #1 \end{align*}}}}}

% Vectors
\usepackage{esvect} 	% Vector over-arrow
%\renewcommand{\vec}{\vv} % Vector over-arrow

% Tikz
%\usepackage{tikz}		
%\usepackage{pgfmath}  	% calculations
%\usepackage{varwidth}  % List inside TikzPicture

% Chem
%\usepackage{chemformula} 	% formulas quimicas
%\usepackage{chemfig} 		% Estruturas quimicas

%\newcommand{\mol}[1]{ \text{mol}_{\ch{ #1 }} } % mol

% Tabular
%\usepackage{multirow}
%\usepackage{booktabs}

% Colors
\usepackage{xcolor}

\definecolor{DarkBlue}	{HTML}{252A36}
\definecolor{LightGreen}{HTML}{7CCC6C}
\definecolor{DarkGreen}	{HTML}{008675}

\colorlet{White}{DarkGreen!20!}
\colorlet{Black}{DarkBlue!110!}

\pagecolor{Black}
\color{White}

%\definecolor{Red}  {HTML}{FF7E79}
%\definecolor{Blue} {HTML}{6666FF}
%\definecolor{Green}{HTML}{66FF66}

% Counters
\counterwithin*{section}{part} % Reset section on part

% Section and Subsection Customization
\renewcommand\thesubsubsection{(\roman{subsubsection})}

\begin{document}

\title{\bfseries\color{DarkGreen!75!}%
	F1 - Ficha SA
\\	Energia%
}
\author{Felipe Pinto - 61387}

\newgeometry{left=25mm, right=25mm}

\maketitle

% Remove contents title inside of \tableofcontents 
\section*{Conteúdo}
\renewcommand{\contentsname}{}

\begin{multicols}{2} \tableofcontents \end{multicols}

\restoregeometry

% Problemas
{\bfseries\color{DarkGreen!75}
	\part{Problemas}
}

\renewcommand\thesection{Problema \arabic{section}}
\renewcommand\thesubsection{%
	P\arabic{section} - \alph{subsection})%
}



% P12
\setcounter{section}{11}
\section{}

\sisetup{round-mode=places, round-precision=2}

\begin{multicols}{2}


% P12 - (i)
\subsubsection{$ v_{a} $}
\begin{flalign*}
&
=	\sqrt{2\,\Delta E_{k}/m} - v_0
;\	
	v_0 = 0
;\	&\\&
	\Delta E_{k} = \Delta E_{g} = m\,g\,\Delta h
\implies &\\& \implies
	v_{a}
=	\sqrt{2\,m\,g\,\Delta h/m}
=	&\\&
=	\sqrt{2*g*(3.5-2)R}
=	\sqrt{3\,g\,R}
&
\end{flalign*}


% P12 - (ii)
\subsubsection{$ \vec{F}_{N} $}
\begin{flalign*}
&
=	\vv{a}\,m - \vv{F_{g}}
;\	a = v^2/R
\implies &\\& \implies
	\vv{F_{N}}
=	(m\,(\sqrt{3\,g\,R})^2/R-m\,g)\,-\hat\jmath
=	&\\&
=	-2\,m\,g\,\hat\jmath
=	-2*5.00*\num{9.80665}
\cong
	\qty{-0.0980665}{\newton}\,\hat\jmath
&
\end{flalign*}



\end{multicols}

% g = \num{9.80665}

% P13
\section{$\Delta S_2$}

\sisetup{round-mode=figures, round-precision=3}

\begin{flalign*}
&
=	\Delta U_2/F_{\text{2 atrito}}
;\	
	F_{\text{2 atrito}} = \mu\,m\,g
;\	
	\Delta U_2 
= 	
	\Delta E_{\text{Gravidade}} 
- 	W_{\text{1 Atrito}}
=	
	m\,g\,\Delta h 
- 	\frac{F_{\text{1 atrito}}\,\Delta h}{\sin(30^\circ)}
\implies &\\& \implies
	\Delta S_2
=	
	\frac%
	{
		m\,g\,\Delta h 
	- 	\mu\,m\,g\,\cos(30^\circ)\,\Delta h/\sin(30^\circ)
	}
	{\mu\,m\,g}
=
	\left(
	\frac{1}{\mu}
-
	\frac{1}{\tan(\theta)}
	\right)
	\Delta h	
= 	&\\& 
=	\left(
	\frac{1}{0.20}
-
	\frac{1}{\tan(30^\circ)}
	\right)
	60\unit{\,\cm}
\cong
	\qty{196.076951545867362}{\cm}
&
\end{flalign*}







\end{document}











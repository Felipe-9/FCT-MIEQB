\documentclass[12pt]{article}

% Linguagem
\usepackage[portuguese]{babel}

% Clickable Table of contents
\usepackage{hyperref}
\hypersetup{
	hidelinks=true,
	colorlinks=true,
	linkcolor=DarkGreen!20!LightGreen!25!
}

% Table of contents
%\usepackage{tocloft}
%\setlength{\cftsubsecnumwidth}{3em} % Fix subsection width
% Fix space between subsection items on toc
%\renewcommand\cftsubsecafterpnum{\vskip5pt}

% Multicols
\usepackage{multicol}

% Customize Chapter
%\usepackage{titlesec}
%\titleformat{\chapter}[hang]{\Huge\bfseries\color{DarkGreen!75!}}{\thechapter\hspace{20pt}{$|$}\hspace{20pt}}{0pt}{\Huge\bfseries}

% Appendix
\usepackage{appendix}

% Maths
\usepackage{amssymb} 		
\usepackage{amsmath} 
\usepackage[utf8]{inputenc} %useful to type directly diacritic characters

\newcommand{\bm}[1]{{\boldmath{\large{\begin{align*} #1 \end{align*}}}}}

% Vectors
\usepackage{esvect} 	% Vector over-arrow
\renewcommand{\vec}{\vv} % Vecto over-arrow

% Tikz
\usepackage{tikz}		
\usepackage{pgfmath}  	% calculations
%\usepackage{varwidth}   % List inside TikzPicture

% Chem
%\usepackage{chemformula} 	% formulas quimicas
%\usepackage{chemfig} 		% Estruturas quimicas

%\newcommand{\mol}[1]{ \text{mol}_{\ch{ #1 }} } % mol

% Tabular
\usepackage{multirow}
\usepackage{siunitx} % Column S: align on decimal
%\usepackage{longtable}


% Colors
\usepackage{xcolor}

\definecolor{DarkBlue}	{HTML}{252A36}
\definecolor{LightGreen}{HTML}{7CCC6C}
\definecolor{DarkGreen}	{HTML}{008675}

\colorlet{White}{DarkGreen!20!}
\colorlet{Black}{DarkBlue!110!}

\pagecolor{Black}
\color{White}

\definecolor{Red}  {HTML}{FF7E79}
\definecolor{Blue} {HTML}{6666FF}
\definecolor{Green}{HTML}{66FF66}

% Counters
\counterwithin*{section}{part} % Reset section on part

% Section and Subsection Customization
%\renewcommand\thesection{Questão \arabic{section} }
%\renewcommand\thesubsection{\arabic{section}-\alph{subsection}) }

\begin{document}

\title{\bfseries\color{DarkGreen!75!}%
	Física 1 - Ficha S2\\
	Movimento Unidimensional%
}
\author{Felipe Pinto - 61387}

\maketitle
\tableofcontents
\break

% Questões
\part{Questões}
\renewcommand\thesection{Questão \arabic{section}}
\renewcommand\thesubsection%
{%
	Q\arabic{section} - \alph{subsection})%
}

% Q1
\section{}

\begin{multicols}{2}

% Q1 - a)
\subsection{}
\begin{tikzpicture}[
	scale=0.25, font=\small, align=center, line width=1,
	rotate=270
]
	
	\def\minX {-2.5};
	\def\maxX { 3.5};
	\def\minY {-5.5};
	\def\maxY { 16.5};
	
	\clip (\minX,\minY) rectangle (\maxX,\maxY);
	
	\draw[ultra thin, Black!75! , step=1]
	(\minX,\minY) grid (\maxX,\maxY);
	
	\draw
	(-1,\minY) -- (-1,\maxY)
	(1, \maxY) -- (1, \minY);
	
	\node at (2,-4)
	[above, circle, fill=Red, inner sep=2]{};
	\node at (2,-4)
	[below, circle, fill=Green, inner sep=2]{};
	
	\foreach \t in {0,1,2,3}
		\draw[fill] 
		(0,{0.5*\t*\t-4}) 
		circle 
		(0.1) node[above=6, font=\tiny]{\t};
		
	\foreach \t in {4,5,6}
		\draw[fill] 
		(0,{0.5*\t*\t-4})
		circle 
		(0.15) node[above=6, font=\tiny]{\t};
	
	\draw[thin] 
	(0.7,4) 
	-- 
	++(0.9,0) node[below=-4, font=\tiny]{$\vec a = \vec 0$};
		
\end{tikzpicture}

% Q1 - b)
\subsection{}
\begin{tikzpicture}[
	scale=0.25, font=\small, align=center, line width=1,
	rotate=270
]
	
	\def\minX {-2.5};
	\def\maxX { 3.5};
	\def\minY {-5.5};
	\def\maxY { 16.5};
	
	\clip (\minX,\minY) rectangle (\maxX,\maxY);
	
	\draw[ultra thin, Black!75! , step=1]
	(\minX,\minY) grid (\maxX,\maxY);
	
	\draw[->] 
	(\minX+0.5,\minY) 
	-- 
	(\minX+0.5,\maxY) node[below left]{$y$};
	
	\draw
	(0,\minY) -- (0,\maxY)
	(2, \maxY) -- (2, \minY);
	
	\draw[thin] (0.3,\maxY-0.5) rectangle (1.7,\maxY-2.5);
	\draw (1,\maxY-0.5) -- +(0,0.5);
	
	\def\a{-1}
	\def\0{\maxY-1.5}
	
	\foreach \t in {0,1,2,3}
		\draw[fill] 
		(1,{0.5*\a*\t*\t+\0}) 
		circle 
		(0.1) node[below=6, font=\tiny]{\t};
	
	\def\1{0.5*\a*3*3+\0}
	
	\foreach \t in {4,5,6}
		\draw[fill] 
		(1,{-3*(\t-3)+\1})
		circle 
		(0.1) node[below=6, font=\tiny]{\t};
	
	\def\2{-3*(6-3)+\1}

	\foreach \t in {7,8,9}
		\draw[fill] 
		(1,{-\a*0.5*(\t-6)*(\t-6)-3*(\t-6)+\2}) 
		circle 
		(0.1) node[below=6, font=\tiny]{\t};
	

\end{tikzpicture}

\end{multicols}

\begin{multicols}{2}

% Q1 - c)
\subsection{}
\begin{tikzpicture}[
	scale=0.25, font=\small, align=center, line width=1
]
	
	\def\minX {-0.5};
	\def\maxX { 21.5};
	\def\minY {-0.5};
	\def\maxY { 10.5};
	
	\clip (\minX,\minY) rectangle (\maxX,\maxY);
	
	\draw[ultra thin, Black!75! , step=1]
	(\minX,\minY) grid (\maxX,\maxY);

	\draw[thick]
	(2,-1) -- (30:100);
	
	\fill[opacity=0.3]
	(2,-1) -- (30:100) -| (\maxX,\minY);
	
	\draw[thin]
	(3.5,0)
	-- (7.5,0)
	   (6.5,0)
	arc (0:30:3) node[font=\tiny, below right]{$30^o$};
	
	\def\a{1}
	
	\foreach \t in {0,1,2,3,4,5,6}
		\draw[fill, rotate=30] 
		(20-0.5*\a*\t*\t,-1)
		circle 
		(0.1) node[above left, font=\tiny]{\t};


\end{tikzpicture}

% Q1 - d)
\subsection{}
\begin{tikzpicture}[
	scale=0.25, font=\small, align=center, line width=1
]
	
	\def\minX {-10.5};
	\def\maxX { 10.5};
	\def\minY {-5.5};
	\def\maxY { 5.5};
	
	\clip (\minX,\minY) rectangle (\maxX,\maxY);
	
	\draw[ultra thin, Black!75! , step=1]
	(\minX,\minY) grid (\maxX,\maxY);
	
	\def\ang{0.0027777777}

	\draw(0,0) circle (4);
	\fill[opacity=0.6](0,0) circle (4);
	
	\foreach \t in {0,1,2,7,8}
		\draw[fill]
		(0,0)
		(\t*360/9:4.5)
		circle
		(0.1) node[right,font=\tiny]{\t};
		
		
	\foreach \t in {3,...,6}
		\draw[fill]
		(0,0)
		(\t*360/9:4.5)
		circle
		(0.1) node[left,font=\tiny]{\t};
		
	\node at (7,4)[font=\tiny]{$\Delta t=10m$};

\end{tikzpicture}

\end{multicols}

% Q2
\section{}

\begin{multicols}{2}

\subsection{}
Movimento sobre superficie com atrito
(movimento retilíneo com aceleração inversa ao movimento)

\subsection{}
Queda livre (movimento retilíneo acelerado para baixo)

\subsection{}
Movimento retilíneo uniforma para cima,\\
Movimento circular desacelerado,\\
movimento circular acelerado,\\
movimento retilíneo uniforme para esquerda.

\end{multicols}

\setcounter{section}{3}

% Q4
\section{}

\begin{multicols}{2}

% Q4 - a)
\subsection{}

\begin{tabular}{| r| *{4}{c} |}
	
	\hline
	
	t & 2 & 4 & 6 & 10
	
	\\ \hline
	
	x & 10 & 20 & 20 & 40
	
	\\ \hline

\end{tabular}

% Q4 - b)
\subsection{}
\begin{flalign*}
&
	v=\Delta S/\Delta t
	= &\\& =
	(20-0)\,m/(4-0)\,s = 5\,m/s
&
\end{flalign*}

% Q4 - c)
\subsection{}
\begin{flalign*}
&
	v=\Delta S/\Delta t
	= &\\& =
	(20-20)\,m/(6-4)\,s = 0\,m/s
&
\end{flalign*}

% Q4 - d)
\subsection{}
\begin{flalign*}
&
	v=\Delta S/\Delta t 
	= &\\& = 
	(40-20)\,m/(10-6)\,s = 5\,m/s
&
\end{flalign*}

% Q4 - e)
\subsection{}
\begin{flalign*}
&
	v=\Delta S/\Delta t
	= &\\& =
	(20-40)\,m/(12-10)\,s = -10\,m/s
&
\end{flalign*}

\end{multicols}

% Problemas
\part{Problemas}
\renewcommand\thesection{Problema \arabic{section}}
\renewcommand\thesubsection%
{%
	P\arabic{section} - \alph{subsection})%
}

% P1
\section{}

\begin{multicols}{2}

% P1 - a)
\subsection{}
\begin{tikzpicture}[
	scale=0.25, font=\small, align=center, line width=1
]
	
	\def\minX {-3.5};
	\def\maxX {12.5};
	\def\minY {-1.5};
	\def\maxY {10.5};
	
	\clip (\minX,\minY) rectangle (\maxX,\maxY);
	
	\draw[ultra thin, Black!75! , step=1]
	(\minX,\minY) grid (\maxX,\maxY);

	\draw[->] 
	(0,\minY) -- 
	(0,\maxY) node[below right, font=\tiny]{$y(m)$};
	\draw[->] 
	(\minX,0) -- 
	(\maxX,0) node[below=6, left=-4, font=\tiny]{$x(min)$};
	
	\def\yscale{ 1/2.5*20/12 }
	
	\foreach \y in {3,6,...,12}
		\draw[thin]
		(0.3,\y*\yscale) -- 
		(-0.3,\y*\yscale)
		node[font=\tiny, left=-3]{\y};
		
	\def\xscale{ 1/0.25 }

	\foreach \x in {1,2,...,2}
		\draw[thin]
		(\x*\xscale, 0.3) -- 
		(\x*\xscale,-0.3) 
		node[font=\tiny, below=-3]{\x};
	
	\def\endline{12.621}
	
	\draw[dashed, thin]
	(0,\endline*\yscale)
	node[font=\tiny, above=2, left=-2]{$12.6\cong$}
	%
	-- node[font=\tiny, sloped, above=-2]{fim}
	%
	(\maxX,\endline*\yscale) node(endline){};
	
	\def\vT{ 10*60/100 }
	\def\vL{ 20*\vT }

	% Lebre
	\path[draw, thin, Blue]
	(0,0) 
	-- node[font=\tiny, sloped, near end, above=-2]{Lebre}
	(2*\xscale,0) --
	(2*\xscale + 1*\xscale, 1*\vL*\yscale);
	
	% Tartaruga
	\path[draw, thin, Red] 
	(0,0) 
	-- node[font=\tiny, sloped, near start, above=-2]{Tartaruga}
	++(4*\xscale,4*\vT*\yscale);

\end{tikzpicture}

% P1 - b)
\subsection{}
\begin{flalign*}
&
	\Delta t*\frac{10.0\,c\,m}{s}\frac{60}{100}
	-(\Delta t-2)20
	* &\\& *
	\frac{10.0\,c\,m}{s}\frac{60}{100}
	\implies \Delta t\cong 2.1\,min
&
\end{flalign*}

% P1 - c)
\subsection{}
\begin{flalign*}
&
	\Delta S = \Delta t*v_{t}=2.1*6\cong 12.6\,m
&
\end{flalign*}

\end{multicols}

% P2
\section{}

\bm{x_{(t)} = 2.0+3.0\,t-1.0\,t^2}

\begin{multicols}{2}

% P2 - a)
\subsection{}
\begin{flalign*}
&
	x_{(3)}=2.0+3.0*3-1.0(3)^2= 2.0\,m
&
\end{flalign*}

% P2 - b)
\subsection{}
\begin{flalign*}
&
	v_{(3)}=x'_{(3)} = 3.0+2*3=9\,m/s
&
\end{flalign*}

% P2 - c)
\subsection{}
\begin{flalign*}
&
	a_{(3)}=x''_{(3)}=-1.0\,m/s^2
&
\end{flalign*}

\end{multicols}

% P3
\section{}

\begin{multicols}{2}

% P3 - a)
\subsection{}
\begin{flalign*}
&
	\Delta t_{\text{min}}\,a_{\text{max}}
	=|\Delta \vec v| = v_i
	\implies &\\& \implies
	\Delta t_{\text{min}}=100/5=20\,s
&
\end{flalign*}

\subsection{}
\begin{flalign*}
&
	\iff \frac{a_{\text{max}}\,(\Delta t_{\text{min}})^2}{2}
	= \Delta S_{\text{max}} 
	< &\\& <
		0.80\,K\,m
	\implies
		\frac{5*(20)^2}{2}=1000\,m
	= &\\& =
		1\,K\,m \nleq 0.8\,K\,m
&
\end{flalign*}

\end{multicols}

% P4
\section{}

% P4 - a)
\subsection{}
%\begin{itemize}
%	
%	\item%
%	\begin{flalign*}
% 	&
%		a_j = 
%		\left\{
%		\begin{array}{ll}
%		%
%			a_j\,m/s^2\quad&\forall\,t\in[0\,s,20.0\,s)
%			\\
%			-5.00\,c\,m/s^2\quad&\forall\,t\geq20.0\,s	
%		%
%		\end{array}
%		\right.
%	&
%	\end{flalign*}
%	
%\end{itemize}

\begin{flalign*}
&
	\Delta S 
	= 
		a_{0}(t_{0})^2/2
		+ v_{1}\,t_{1} 
		+ a_{1}(t_{1})^2/2
	;\
		t_{Total} = t_{0} + t_{1}
	; &\\&
		v_{1} = t_{0}\,a_{0}
	\implies
		\Delta S 
	=
		\frac{v_{1}}			 {t_{Total}-t_{1}}
		\frac{(t_{Total}-t_{1})^2}{2}
	+	v_1\,t_1 
	+ 	a_1(t_1)^2/2
	= &\\& =
		\frac{10.0}	  {160-20}
		\frac{(160-20)^2}{2}
	+	10.0*20.0
	-	5.00*10^{-2}(20.0)^2/2
	\cong
		890\,m
&
\end{flalign*}

% P4 - b)
\subsection{}
\begin{flalign*}
&
	\text{Maria ganha} \iff
		t_M < t_J
	;\
		t_M = t_0+t_1
	;\
		a_0\,(t_0)^2/2 = l_1
	\implies &\\& \implies
		\sqrt{2\,l_1/a_0} + t_1
	=
		\sqrt{2*200/0.100} + 128
	\cong 191\,s \not< 160\,s
	&\\& \therefore	\text{João ganha}
&
\end{flalign*}

% P5
\section{}

% P5 - a)
\subsection{}
\begin{flalign*}
&
	\vec a = \Delta \vec v/\Delta t
	;\
		\Delta \vec S = \vec v_i\,\Delta t 
			    + \vec a\,(\Delta t)^2/2
	\implies
		\vec v_i\,(\Delta v/a)
		+ \vec a\,(\Delta v/a)^2/2
		= \Delta \vec S
	\implies &\\& \implies
		\frac{\vec a}{a^2}
		\,\frac{(\Delta v)^2}{2}
	=
		\Delta \vec S-\vec v_i\,\Delta v/a
	\implies
		a^2\,\hat\imath
		- a\,v_i\,\hat\imath\,\Delta v/\Delta S
		+a\hat\imath\,\frac{(\Delta v)^2}{2\,\Delta S}
	=
		\vec 0
	\implies &\\& \implies
		a\,\hat\imath\,
		\left(
			a
			+ \frac{v_i\,\Delta v}{\Delta S}
			- \frac{(\Delta v)^2}{2\,\Delta S}
		\right)
		= \vec 0
	\implies
		\vec a
	= 
		\frac{(\Delta v)(\Delta v-2\,v_i)}{2\,\Delta S}
		\,\hat\imath
	= &\\& =
		\frac{(420-280)(420-280-2*420)}{2*0.10}
	=
		-490\,\hat\imath\,K\,m/s^2
&
\end{flalign*}

% P5 - b)
\subsection{}
\begin{flalign*}
&
	\Delta t = \Delta v/ a = \frac{420-280}{490\,K}
	\cong 286*10^{-6}s
&
\end{flalign*}

% P5 - c)
\subsection{}
\begin{flalign*}
&
	\Delta \vec S = \vec v_i\,\Delta t 
			    + \vec a\,(\Delta t)^2/2
	;\ 
		\vec a = \Delta \vec v/\Delta t
	;\
		\Delta \vec v = -\vec v_i
	\implies &\\& \implies
		\Delta S\,\hat\imath
	=
		\frac{v_i^2}{a}\,\hat\imath
		-\frac{v_i^2}{2\,a}\,\hat\imath
	=
		\frac{v_i^2}{2\,a}\,\hat\imath
	=
	\left(
		\frac{420^2}{2*490\,K}
	\right)\,\hat\imath
	\cong
		18\,\hat\imath\,c\,m
&
\end{flalign*}

% P6
\section{}

\subsection{}
\begin{flalign*}
&
	\Delta t_{Total} = \Delta t_1 + \Delta t_2
	;\
		a_1\,\Delta t_1 = a_2\,\Delta t_2
	;\
		\Delta S 
	=
		a_1\,(\Delta t_1)^2/2 
		+ v_2\,\Delta t_2 
		- a_2\,(\Delta t_2)^2/2
	; &\\&
		v_2 = a_1\,\Delta t_1
	\implies
		\Delta t_{Total} 
	= 
		\sqrt{ \frac{2\,a_2\,\Delta S}
				  {a_2\,a_1+a_1^2} }
		+ \frac{a_1}{a_2}\,
		\sqrt{ \frac{2\,a_2\,\Delta S}
				  {a_2\,a_1+a_1^2} }
	= &\\& =
		\left( 1 + \frac{0.100}{0.500} \right)
		\sqrt{ \frac{2*0.500*1.00\,K}
				  {0.500*0.100+0.100^2} }
	\cong
		155\,s;
&\\\\&
	\Delta t_1
	=
		\sqrt{ \frac{2\,a_2\,\Delta S}
				  {a_2\,a_1+a_1^2} }
	=
		\sqrt{ \frac{2*0.500*1.00\,K}
				  {0.500*0.100+0.100^2} }
	\cong 129\,s
&
\end{flalign*}






\end{document}











% !TEX root = ./F1-Aulas.5.tex
\providecommand\mainfilename{"./F1-Aulas.tex"}
\providecommand \subfilename{}
\renewcommand   \subfilename{"./F1-Aulas.5.tex"}
\documentclass[\mainfilename]{subfiles}

% \tikzset{external/force remake=true} % - remake all

\begin{document}

% \graphicspath{{\subfix{./.build/figures/F1-Aulas.5}}}
% \tikzsetexternalprefix{./.build/figures/F1-Aulas.5/}

\mymakesubfile{5}
[Física 1]
{Aula 04/15: Molas} % Subfile Title
{Aula 04/15: Molas} % Part Title

\begin{sectionBox}1{Bloco caindo em uma mola} % S
    
    \subsection{Trabalho pela graviade}
    \begin{flalign*}
        &
            w_{F_g} = \int{\vv{F_g}\cdot\odif{\vv{d}}}
            = m\,g\,d
            \cong
            \cdots
        &
    \end{flalign*}

    \subsection{Trabalho pela mola}
    \begin{flalign*}
        &
            w_{F_k} = \int{\vv{F_{k}}\cdot\odif{\vv{d}}}
            = -k\,\int{d\odif{d}}
            = -k\,d^2/2
        &
    \end{flalign*}

    \subsection{Velocidade do bloco no momento que encosta na mola}
    \begin{flalign*}
    &
        m\,v^2/2 
        = k\,d^2/2
        \implies
        v = \sqrt{k\,d^2/m}
    &
    \end{flalign*}

    \subsection{Altura que partil}
    \begin{flalign*}
    &
        m\,g\,h = m\,v^2/2
        \implies
        h = v^2/(2\,g)
    &
    \end{flalign*}
    
\end{sectionBox}



\end{document}
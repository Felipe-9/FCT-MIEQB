% !TEX root = ./F1-Aulas.1.tex
\providecommand\mainfilename{"./F1-Aulas.tex"}
\providecommand \subfilename{}
\renewcommand   \subfilename{"./F1-Aulas.1.tex"}
\documentclass[\mainfilename]{subfiles}

% \tikzset{external/force remake=true} % - remake all

\begin{document}

% \graphicspath{{\subfix{./.build/figures/F1-Aulas.1}}}
% \tikzsetexternalprefix{./.build/figures/F1-Aulas.1/}

\mymakesubfile{1}
[Fisica 1]
{Aula Teórica 03/18: Vetores} % Subfile Title
{Aula Teórica 03/18: Vetores} % Part Title

\begin{sectionBox}1{Introdução} % S1
    
    \begin{BM}
        \Big\{\vv{A};\vv{B}\Big\}\in R^n
    \end{BM}

    \begin{multicols}{2}
        \paragraph{Produto Escalar (Interno):}
        \begin{BM}
            \vv{A} \cdot \vv{B}
            = \\
            = \myvert{\vv{A}}\myvert{\vv{B}}\cos{(\theta)} 
            = \\
            = \sum_{k=1}^{n}{a_k\,b_k}
        \end{BM}

        \paragraph{Produto Vetorial (Externo):}
        \begin{BM}
            \vv{A}*\vv{B}
            = \\
            = \myvert{\vv{A}}\myvert{\vv{B}}\sin{(\theta)}(\hat{A}*\hat{B})
            = \\
            = (\hat{A}*\hat{B})
            \,\sum_{k=1}^{n-1}{
                \left(\sum_{j=k+1}^{n}{(a_k\, b_j)}\right)
            }
        \end{BM}
    \end{multicols}
    
    \subsection{Regra da mão direita}
    Ao se fazer produto vetorial, com a mão apontada para o vetor inicial, direcione a palma da mão em direção e sentido ao segundo vetor, e assim o polegar estará apontado para a direção e sentido do vetor resultante
    
\end{sectionBox}

\end{document}
% !TEX root = ./F1-Aulas.3.tex
\providecommand\mainfilename{"./F1-Aulas.tex"}
\providecommand \subfilename{}
\renewcommand   \subfilename{"./F1-Aulas.3.tex"}
\documentclass[\mainfilename]{subfiles}

% \tikzset{external/force remake=true} % - remake all

\begin{document}

% \graphicspath{{\subfix{./.build/figures/F1-Aulas.3}}}
% \tikzsetexternalprefix{./.build/figures/F1-Aulas.3/}

\mymakesubfile{3}
[Física 1]
{Aula 03/26: Força e Movimento: Leis de Newton} % Subfile Title
{Aula 03/26: Força e Movimento: Leis de Newton} % Part Title

\begin{sectionBox}1{Primeira lei de Newton: Lei da Inércia} % S1
    
    \begin{BM}
        \sum{\vv{F}}=0 \implies
    \end{BM}
    \(\implies\) Corpo continua a perpetuar seu estado de movimento, comportamento inercial
    
\end{sectionBox}

\end{document}
\documentclass[12pt]{article}

% Maths
\usepackage{amssymb} 		
\usepackage{amsmath} 
\usepackage[utf8]{inputenc} %useful to type directly diacritic characters

% Vectors
%\usepackage{esvect} 	% Vector over-arrow
%\usepackage{tikz}		% Vector diagrams

%\renewcommand{\vec}{\vv} % Vecto over-arrow

% Chem
%\usepackage{chemformula} 	% formulas quimicas
%\usepackage{chemfig} 		% Estruturas quimicas

% Colors
\usepackage{xcolor}

\definecolor{DarkBlue}	{HTML}{252A36}
\definecolor{LightGreen}	{HTML}{7CCC6C}
\definecolor{DarkGreen}	{HTML}{008675}

\pagecolor{DarkBlue!110!}
\color{DarkGreen!20!}

\begin{document}

\title{Fisica 1 - P4\\Trabalho Experimental 5}
\date{29/03}

\maketitle
\tableofcontents
\break

\section{Introdução}
Medir o modulo da aceleração da gravidade usando um pêndulo simples oscilando em um grande raio e de pequeno angulo de oscilação

\section{Calculos}
\begin{flalign*}
T &= \frac{2\,\pi\,\sqrt L}{\sqrt g} &
\end{flalign*}

\section{10?}
Calcular o valor experimental de $\sqrt L$; L: Raio de oscilação do angulo
\begin{flalign*}
\text{incerteza de g}u(g) &= \sqrt{\frac{dg}{dm}u\,m} \text{ incompleto} &
\end{flalign*}

\section{4}


\section{Notas de aula:}
\begin{intemize}
\item Propagação de Incertezas
\item 
\end{intemize}

\end{document}

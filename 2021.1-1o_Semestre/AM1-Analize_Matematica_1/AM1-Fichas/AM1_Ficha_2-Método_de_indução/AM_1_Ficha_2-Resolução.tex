\documentclass[12pt]{article}

% Clickable Table of contents
\usepackage[hidelinks]{hyperref}

% Multicols
\usepackage{multicol}

% Maths
\usepackage{amssymb} 		
\usepackage{amsmath} 
\usepackage[utf8]{inputenc} %useful to type directly diacritic characters

\newcommand{\bm}[1]{{\boldmath{\large{\begin{align*} #1 \end{align*}}}}}

\renewcommand{\cos}[2][ ]{\text{cos}^{#1}\left( {#2} \right)}
\renewcommand{\sin}[2][ ]{\text{sin}^{#1}\left( {#2} \right)}
\renewcommand{\tan}[2][ ]{\text{tan}^{#1}\left( {#2} \right)}

% Vectors
%\usepackage{esvect} 	% Vector over-arrow
%\usepackage{tikz}		% Vector diagrams

%\renewcommand{\vec}{\vv} % Vecto over-arrow

% Chem
%\usepackage{chemformula} 	% formulas quimicas
%\usepackage{chemfig} 		% Estruturas quimicas

% Colors
\usepackage{xcolor}

\definecolor{DarkBlue}{HTML}{252A36}
\definecolor{LightGreen}{HTML}{7CCC6C}
\definecolor{DarkGreen}{HTML}{008675}

\pagecolor{DarkBlue!110!}
\color{DarkGreen!20!}

\renewcommand{\thesection}{Exercício \arabic{section} }
\renewcommand{\thesubsection}{\arabic{section} - \alph{subsection})}


\begin{document}

\title{Ficha 2\\Método de indução}
\author{}

\maketitle
\tableofcontents
\break

% Q1
\section{}

% Q1 - a)
\subsection{$
	A = \left\{
		\sum_{k=1}^n 1/2^k=1-1/2^n\  
		\forall\ n\in\mathbb{N} 
	\right\}
$}
\begin{flalign*}
& 1\in A  \iff 	 \sum_{k=1}^1 1/2^k=1-1/2^1
		 	 \implies 1/2=1/2  &\\
& m+1\in A\ \forall\ m\in\mathbb{N} 
			 \iff 	 \sum_{k=1}^{m+1} 1/2^k
			 			 = 1-1/2^{m+1} \implies &\\
&		 	 \implies \sum_{k=1}^{m} 1/2^k+1/2^{m+1}
						 = 1-1/2^{m+1} &
\end{flalign*}

% Q1 - b)
\subsection{$ 
	B = \left\{
		\sum_{k=1}^n\frac{1}{k(k+1)}=\frac{n}{n+1}\ 
		\forall\ n\in\mathbb{N}
	\right\}
$}
\begin{flalign*}
&1\in B
	\iff 	 	\sum_{k=1}^1\frac{1}{k(k+1)}=\frac{1}{1+1}
	\implies 1/2=1/2; &\\
&m+1\in B\ \forall\ m\in\mathbb{N} 
	\iff		\sum_{k=1}^{m+1}\frac{1}{k(k+1)}
	=			\frac{m+1}{m+1+1}\implies&\\
&	\implies \sum_{k=1}^{m}\frac{1}{k(k+1)}
	+ 			\frac{1}{(m+1)(m+2)}
	= 			\frac{m+1}{m+2} \implies &\\
&	\implies \sum_{k=1}^{m}\frac{1}{k(k+1)}
	=			\frac{(m+1)^2-1}{(m+1)(m+2)}
	=			\frac{m^2+2\,m+1-1}{(m+1)(m+2)} =&\\
&	=			\frac{m(m+2)}{(m+1)(m+2)}
	=			\frac{m}{m+1}&
\end{flalign*}

% Q1 - c)
\subsection{$
	C = \left\{
		\sum_{k=1}^n k\ k!=(n+1)!-1\ 
		\forall\ n\in\mathbb{N}
	\right\}
$}
\begin{flalign*}
&	1\in C
	\iff 		\sum_{k=1}^1 k\ k!=(1+1)!-1
	\implies 1\ 1!=2-1
	\implies 1=1 &\\
&	m+1\in C\ \forall\ m\in\mathbb{N}
	\iff		\sum_{k=1}^{m+1} k\ k!=(m+1+1)!-1 \implies &\\
&	\implies \sum_{k=1}^m k\ k!
	+			(m+1)((m+1)!)
	=			(m+2)((m+1)!)-1 \implies &\\
&	\implies \sum_{k=1}^m k\ k!
	= 			((m+1)!)(m+2-m-1)-1
	=			(m+1)!-1 &
\end{flalign*}

% Q1 - d)
\subsection{} 
\bm{ 
	D = \left\{
		(\cos{x}+i\,\sin{x})^n=\cos{n\,x}+i\,\sin{n\,x} \right. \\ \left.
		\forall\ n\in\mathbb{N},\ 
		\forall\ x\in\mathbb{R},\
		i = \sqrt{-1}
	\right\} 
}
\begin{flalign*}
&	1\in D
		\iff (\cos{x}+i\,\sin{x})^1=\cos{1\,x}+i\,\sin{1\,x} &\\
&	m+1\in D\ \forall\ m\in\mathbb{N}
		\iff (\cos{x}+i\,\sin{x})^{m+1} = &\\ 
&		= \cos{(m+1)x}+i\,\sin{(m+1)x}
		\implies (\cos{x}+i\,\sin{x})^m = &\\
&		= \frac{\cos{m\,x}\cos{x}-\sin{m\,x}\sin{x}
				  +i(\sin{m\,x}\cos{x}+\sin{x}\cos{m\,x})}
				 {\cos{x}+i\,\sin{x}} = &\\
&		= \frac{  \cos{m\,x}\cos{x}
				  + i\,\cos{x}\sin{m\,x}
				  + i\,\sin{x}\cos{m\,x}
				  - \sin{x}\sin{m\,x}}
				 {\cos{x}+i\,\sin{x}}= &\\
&		= \frac{ \cos{m\,x}\cos{x}
				  +\cos{x}\,i\,\sin{m\,x}
				  +i\,\sin{x}\,\cos{m\,x}
				  +i^2\,\sin{x}\sin{m\,x}}
				 {\cos{x}+i\,\sin{x}} = &\\
&		= \frac{(\cos{x}+i\,\sin{x})(\cos{m\,x}+i\,\sin{m\,x})}
				 {\cos{x}+i\,\sin{x}}
		= \cos{m\,x}+i\,\sin{m\,x} &
\end{flalign*}

% Q2
\section{}

% Q2 - a)
\subsection{$ 9^n-3 $ é multiplo de $6$}
\begin{flalign*}
& 
	9^n-3 \text{ é multiplo de } 6 
	\iff 
		\exists\ k\in\mathbb{N}
		: 9^n-3=6\,k
		\ \forall\ n\in\mathbb{N}\backslash\{0\} 
	\iff &\\& \iff 
		9^n-3 = 6\,k
	\iff 
		(3^{2})^n/3
		= 3^{2\,n-1} 
		= 2\,k + 1 
& \\\\ &
	9^n-3 \text{ é multiplo de } 6 
	\iff 
		\exists\ k\in\mathbb{N}
		: 9^n-3=6\,k
		\ \forall\ n\in\mathbb{N}\backslash\{0\} 
	\iff &\\& \iff 
		\left\{ \begin{array}{ll}
			n = 1 \implies 9^1-3 = 6
			\\
			n = m + 1
			\implies
				9^{m+1}-3
				= 9\,9^m-3
				= 9\,(6\,k+3)-3
				= \\ =
				6\,(9\,k+4);
				\ (9\,k+4)\in\mathbb{N}\ \forall\ k\in\mathbb{N}
		\end{array} \right.
&
\end{flalign*}

% Q2 - b)
\subsection{$ 6^n-1 $ é múltiplo de 5}
\begin{flalign*}
%&	
%	6^n-1 \text{ é múltiplo de 5 } 
%	\iff 
%		\exists\ k\in\mathbb{N}
%		: 6^n-1=5\,k\ 
%		\forall\ n\in\mathbb{N} 
%	\iff &\\& \iff  
%	(5+1)^n-1
%	= 
%		\sum_{i=0}^{n  } 
%		\left( 
%			\frac{n!\,5^{n-i}1^{i} }{i!(n-i)!}
%		\right) 
%		- 1
%	= 
%		n!
%		\sum_{i=0}^{n-1} \left( 
%			\frac{5^{n-i}}{i!(n-i)!}
%		\right) 
%	= &\\&
%	= 5 \left( \sum_{i=0}^{n-1} 5^{n-i-1} \right)
%	= 5 \left( \sum_{i=1}^{n  } 5^{n-i} \right) 
%	= 5 \left( \sum_{i=0}^{n-1} 5^{i} \right);
%	\ 
%		\sum_{i=0}^{n-1} 5^{i}
%		\in\mathbb{N}
%		\ \forall\ n\in\mathbb{N}
%&
%\\\\
&
	6^n-1 \text{ é múltiplo de 5 } 
	\iff 
		\exists\ k\in\mathbb{N}
		: 6^n-1=5\,k\ 
		\forall\ n\in\mathbb{N} 
	\iff &\\& \iff  
	\left\{ \begin{array}{ll}
		n = 1
		\implies
			6^1-1
			= 5
		\\
		n = m + 1
		\implies
			6^{m+1}-1
			= 6\,6^m-1
			= 6\,(5\,k+1)-1
			= \\
			= 5\,(6\,k+1);
			\ (6\,k+1)\in\mathbb{N}
			\ \forall\, k\in\mathbb{N}
	\end{array} \right.
&
\end{flalign*}

% Q2 - c)
\subsection{$ 3\,n^2 + 3\,n $ é múltiplo de 6}
\begin{flalign*}
%&	3\,n^2+3\,n \text{ é múltiplo de 6}
%	\iff 
%		\exists\ k\in\mathbb{N}
%		: 3\,n^2+3\,n=6k
%		\ \forall\ n\in\mathbb{N} 
%	\iff 
%	&\\&
%	\iff 
%		3\,n^2+3\,n 
%		= 3\,n\,(n+1)
%		= 6\,k
%	\iff
%		n\,(n+1)=2\,k
%	\iff
%	&\\&
%	\iff
%		\left\{
%		n\,(n+1)\in\mathbb{N}:n\,(n+1)=2\,k\ \forall\ n\in\mathbb{N}:
%		n=2\, m\vee n=2\,m+1\ \forall\ m\in\mathbb{N}
%		\right\}
%	=
%	&\\&
%	=
%		\left\{
%		n\,(n+1)\in\mathbb{N}:n\,(n+1)=2\,k\ \forall\ n\in\mathbb{N}
%		\right\} 
%&
%\\\\
&
	3\,n^2+3\,n \text{ é múltiplo de 6}
	\iff 
		\exists\ k\in\mathbb{N}
		: 3\,n^2+3\,n=6\,k
		\ \forall\ n\in\mathbb{N} 
	\iff 
	&\\&
	\iff
	\left\{ \begin{array}{ll}
		n = 1
		\implies
			3*1^2+3*1
			= 6
		\\
		n = m + 1
		\implies
			3\,(m+1)^2+3\,(m+1)
			= 3\,(m+1)\,(m+1+1)
			= \\
			= 3\,m^2+3\,m+2\,(3\,m^2+3\,m)/m
			= 6\,k+2\,(6\,k)/m
			= 6\,(k+2k/m)
			= \\
			= 6\,(k+m+1);
			\ (k+m+1)\in\mathbb{N}
				\ \forall\ \{m,k\}\subset\mathbb{N}
	\end{array} \right.
&
\end{flalign*}

% Q2 - d) Extra
\subsection{Extra: $5^n-1$ é multiplo de 4}
\begin{flalign*}
%&
%	5^n-1\text{ é multiplo de 4}
%	\iff
%		\exists\ k\in\mathbb{N}: 5^n-1=4\,k\ \forall\ n\in\mathbb{N}
%	\iff &\\& \iff
%		5^n-1
%		= (4+1)^n-1
%		= \sum_{i=0}^{n  } 4^{n-i}\,1^{i}-1
%		= \sum_{i=0}^{n-1} 4^{n-i}
%		= 
%			4 \left(
%				\sum_{i=0}^{n-1} 4^{n-i-1}
%			\right)
%		= &\\& =
%			4 \left(
%				\sum_{i=0}^{n-1} 4^{n-i}
%			\right)
%		=
%			4 \left(
%				\sum_{i=0}^{n-1} 4^{i}
%			\right)
%		= 4\,k
%		\implies
%			k\in\mathbb{N}: k=\sum_{i=0}^{n-1}4^i\ \forall\ n\in\mathbb{N}
%& 
%\\\\ 
&
	5^n-1\text{ é multiplo de 4}
	\iff
		\exists\ k\in\mathbb{N}: 5^n-1=4\,k\ \forall\ n\in\mathbb{N}
	\iff &\\& \iff
	\left\{ \begin{array}{ll}
  		n=1
			\implies 5^1-1=4\,k\implies k=0;\\
		n=m+1
		\implies
			5^{m+1}-1
			= 5\,5^{m}-1
			= 5\,(5^m-1) + 4
		=\\=
			5\,(4\,k) + 4
			= 4\,(5\,k+1)
	\end{array} \right.
&
\end{flalign*}

% Q3
\section{} 
\bm{
	I_i \text{ é um intervalo aberto}&\ \forall\,i\in\mathbb{N}\cap[1,n]; \\
	\bigcap_{i=1}^{n}I_i&\neq\emptyset
}
\begin{flalign*}
&
	A = \bigcup_{i=1}^n I_i \text{ é um intervalo aberto}
	\iff
		A = \text{Int}(A)
	\iff
		V_{\epsilon}a\subset A\ \forall\,a\in A
	\iff &\\& \iff
	\left\{ \begin{array}{ll}
		\exists B\subset A 
			: B\subset I_i
			\ \forall\,i\in\mathbb{N}\cap[1,n]
		\impliedby \bigcap_{i=1}^n I_i\neq\emptyset
		\\
		V_{\epsilon}b\subset I_i
			\ \forall\,b\in I_i
			\ \forall\,i\in\mathbb{N}\cap[1,n]
	\iff 
		I_i = \text{Int}(I_i)\ \forall\,i\in\mathbb{N}\cap[1,n]
	\iff \\ \iff 
		I_i \text{ é um intervalo aberto}\ \forall\,i\in\mathbb{N}\cap[1,n]
		\end{array} \right.
&
\end{flalign*}

% Q4
\section{}


% Q4 - a)
\subsection{$
	(1+k)^n\geq 1+n\,k
	\ \forall\,n\in\mathbb{N}
	\quad \text{para } k > -1 \text{ fixado}
$}
\begin{flalign*}
%&
%	(1+k)^n
%	= \sum_{i=0}^{n  } 1^{n-i}\,k^i
%	= \sum_{i=0}^{n  } k^i
%	= \sum_{i=1}^{n  } k^i + 1
%	=k\sum_{i=0}^{n-1} k^{i} + 1
%	\geq 1 + n\,k
%	= &\\&
%	= 1 +  \sum_{i=1}^{n  } k
%	= 1 + k\sum_{i=0}^{n-1} 1
%	\implies &\\& \implies
%		\sum_{i=1}^{n-1} k^i
%	\geq 
%		\sum_{i=1}^{n-1} 1
%&
%\\\\
&
	(1+k)^n\geq 1+n\,k
	\ \forall\,n\in\mathbb{N},
	\ \forall\,k\in\mathbb{R}: k>-1
	\iff &\\& \iff
	\left\{ \begin{array}{ll}
		n = 0
		\implies 
			(1+k)^0 = 1
			\geq 	1 + 0\,k = 1;
		\\
		n = m + 1
		\implies 
			(1+k)^{m+1}
			= (1+k)\,(1+k)^m
			;\ (1+k) > 0
			\ \forall\, k>-1
			\implies \\ \implies
			(1+k)\,(1+k)^m
			\geq (1+k)\,(1+m\,k)
			= \\
			= 1+m\,k+k + m\,k^2
			\geq 1+(m+1)\,k
			=	  k+1+m\,k
			\implies \\ \implies
				m\,k^2\geq0
	\end{array} \right.
&
\end{flalign*}

% Q4 - b)
\subsection{$
	\sum_{k=1}^{n}k^2<(n+1)^3\ \forall\,n\in\mathbb{N}
$}
\begin{flalign*}
&
	\sum_{k=1}^{n}k^2<(n+1)^3\ \forall\,n\in\mathbb{N}
	\iff &\\& \iff
	\left\{ \begin{array}{ll}
		n = 0 
		\implies \sum_{k=1}^{0}k^2=0<(0+1)^3=1 
		\\
		n = m + 1 
		\implies
			\sum_{k=1}^{m+1}k^2
			= \sum_{k=1}^{m}k^2+(m+1)^2;
			\\ 
			\sum_{k=1}^{m}k^2>0 : k^2>0\ \forall\,k\in\mathbb{N}
		\implies
			\sum_{k=1}^{m}k^2+(m+1)^2
			< \\
			< (m+1)^3+(m+1)^2
			= (m+1+1)(m+1)^2
			< (m+1+1)^3
			\implies \\ \implies
				(m+1)^2<(m+2)^2
			\implies
				|m+1|<|m+2|
			; m\geq 0\ \forall\, m\in\mathbb{N}
			\implies\\\implies
				0 < 1
	\end{array} \right.
&
\end{flalign*}

% 4 - c)
\subsection{$
	\sum_{k=1}^{n}1/(2^k+1)<1-1/2^n
	\ \forall\,n\in\mathbb{N}\backslash \{0\}
$}
\begin{flalign*}
&
	\sum_{k=1}^{n}1/(2^k+1)<1-1/2^n\ \forall\,n\in\mathbb{N}\backslash\{0\}
	\implies &\\& \implies
	\left\{ \begin{array}{ll}
		n = 1 \implies
			\sum_{k=1}^{1}1/(2^k+1)
			= 1/3
			< 1-1/2^1
			= 1/2 
		\\
		n = m + 1 \implies
			\sum_{k=1}^{m+1}1/(2^k+1) 
			= \sum_{k=1}^{m}1/(2^k+1) + 1/(2*2^m+1)
			< \\
			< 1-1/2^{m} + 1/(2*2^m+1)
			= 1 - 1/(2*2^{m}) - 1/(2*2^{m}) 
			+ \\
			+ 1/(2*2^m+1)
			< 1-1/2^{m+1}
			= 1-1/(2*2^m)
			\implies \\ \implies
			1/(2*2^m+1)
			< 1/(2*2^{m})
			\implies 1>0
	\end{array} \right.
&
\end{flalign*}

% Q4 - d)
\subsection{$
	\sum_{k=1}^{n}1/k^2\leq 2-1/n
	\ \forall\,n\in\mathbb{N}
$}
\begin{flalign*}
&
	\sum_{k=1}^{n}1/k^2\leq 2-1/n
	\ \forall\,n\in\mathbb{N}\backslash\{0\}
	\implies &\\& \implies
	\left\{ \begin{array}{ll}
		n = 1 \implies
			\sum_{k=1}^{1}1/k^2
			= 1
			\leq 2-1/1
			= 1
		\\
		n = m + 1 \implies
			\sum_{k=1}^{m+1}1/k^2
			= \sum_{k=1}^{m}1/k^2 + 1/(m+1)^2
			\leq \\
			\leq 2-1/m+1/(m+1)^2
			\leq 2-1/(m+1)
			= 2-(m+1)/(m+1)^2
			\implies \\ \implies
				(m+1+1)/(m+1)^2
				\leq 1/m
			\implies 
				m^2+2\,m
				\leq m^2+2\,m+1
			\implies \\ \implies
				0 \leq 1
	\end{array} \right.
&
\end{flalign*}

% Q6
\section{}\bm{
	u_1 = -1, 
	&& u_{n+1}=\frac{u_n}{1-2\,u_n}, 
	&& \forall\,n\in\mathbb{N}
}
\begin{flalign*}
&
	u_n=1/(1-2\,n)\ \forall\,n\in\mathbb{N}
	\iff &\\& \iff
	\left\{ \begin{array}{ll}
		n = 1 \implies
			1/(1-2) = -1/1 = u_1
		\\
		n = m + 1 \implies
			\frac{1}{1-2\,(m+1)}
			= \frac{1}{-1-2\,m}
			= \frac{1}{1-2\,m}
			* \frac{1-2\,m}{1-2\,m-2}
			= \\
			= \frac{1/(1-2\,m)}{(1-2\,(1/(1-2\,m))}
			= u_m/(1-2\,u_m)
			= u_{m+1}
	\end{array} \right.
&
\end{flalign*}

\break

% Q6
\section{} \bm{
	e_n:=\left( 1+ 1/n \right)^n && (n\in\mathbb{N})
}

% Q6 - a)
\subsection{$
	{n\choose k}\frac{1}{n^k}\leq \frac{1}{k!}
	\quad
		\forall\,n\in\mathbb{N},
	\ \forall\,k\in\mathbb{N}\cap[0,n]
$}

\begin{flalign*}
&
	{n\choose k}\frac{1}{n^k}\leq \frac{1}{k!}
	\quad
		\forall\,n\in\mathbb{N},
		\ \forall\,k\in\mathbb{N}\cap[0,n]
	\iff &\\& \iff
		{n\choose k}\frac{1}{n^k}
	= 
		\frac{1}{k!}\,\frac{n!}{(n-k)!n^k}
	\leq  
		\frac{1}{k!}
	\implies
		n!
	\leq 
		(n-k)!n^k
	\implies &\\& \implies
		\prod_{i=1}^{k-1}(n-i)
	\leq 
		n^k
	\implies
		\log_n\left( \prod_{i=1}^{k-1}(n-i) \right)
	\leq 
		k
	\implies
		\sum_{i=1}^{k-1} \log_n(n-i)
	\leq &\\& \leq
		\sum_{i=1}^{k-1} \log_n(n)
	=
		k-1
	\leq k
&	
\end{flalign*}

% Q6 - b)
\subsection{$
	\sum_{k=0}^{n}\frac{1}{k!}\leq 3-1/n
	\quad\forall\,n\in\mathbb{N}\backslash\{0\}
$}
\begin{flalign*}
&
	\sum_{k=0}^{n}\frac{1}{k!}\leq 3-1/n
	\quad\forall\,n\in\mathbb{N}\backslash\{0\}
	\implies &\\& \implies
		\sum_{k=2}^{n}\frac{1}{k!}
		\leq 
			1-1/n
		= (n-1)!(n-1)/n!
	\implies &\\& \implies
		\sum_{k=2}^{n}\frac{n!}{k!}
		= \sum_{k=0}^{n-2}\frac{n!}{(n-k)!}
		= \sum_{k=0}^{n-2}\frac{n!}{(n-k)!}
		\leq
			n!-(n-1)!\cdots
&
\\\\
&
	\sum_{k=0}^{n}\frac{1}{k!}\leq 3-1/n
	\quad\forall\,n\in\mathbb{N}\backslash\{0\}
	\iff &\\& \iff
	\left\{ \begin{array}{ll}
		n = 1 \implies 
			\sum_{k=0}^{1}1/k!
			= 1
			\leq 3-1/1
			= 2
		\\
		n = m + 1 \implies
			\sum_{k=0}^{m+1}1/k!
		=
			1/(m+1)!+\sum_{k=0}^{m}1/k!
		\leq \\
		\leq 
			m^{-1}/(m+1) + 3 - 1/m
		=
			(1+m^{-1})/(m+1)- 1/(m+1) 
		+ \\
		+ 
			3 - 1/m
		\leq 
			3-1/(m+1)
		\implies
			\frac{1}{m+1} + \frac{1}{m(m+1)} - \frac{1}{m}
		\leq
			0
		\implies \\ \implies
			1+1/m
		\leq
			1+1/m
	\end{array} \right.
&
\end{flalign*}

% Q6 - c)
\subsection{$
	e_m\leq 3\quad\forall\,n\in\mathbb{N}
$ \color{red!60!}{Duvida}}
\begin{flalign*}
&
	e_m\leq 3\quad\forall\,n\in\mathbb{N}
	\iff 
		(1+1/n)^n
	=	\sum_{i=0}^{n}{n \choose i}\frac{1}{n^i} 1^{n-i}
	= 	\sum_{i=0}^{n}\frac{n!}{i!(n-i)!}n^{-i}
	=	&\\& =
		1 
	+
		\sum_{i=1}^{n}\frac{n!\,n^{-i}}{i!(n-i)!}
	\leq
		3
	\implies 
		\sum_{i=1}^{n}\frac{n!\,n^{-i}}{i!(n-i)!} \leq 2
	\iff &\\& \iff
	\left\{ \begin{array}{ll}
		n = 0 \implies
			\sum_{i=1}^{0}\frac{0!\,0^{-i}}{i!(0-i)!} = 0 \leq 2
		\\
		n = m + 1 \implies
			\sum_{i=1}^{m+1}\frac{(m+1)!\,(m+1)^{-i}}{i!(m+1-i)!}
		=
			\sum_{i=1}^{m+1}\frac{(m+1)\,(m!)\,(m+1)^{-i}}
										{i!\,(m+1-i)\,(m-i)!}
		= \\ =
			(m+1)
			\sum_{i=1}^{m+1}\frac{(m!)\,(m+1)^{-i}}
										{(m+1-i)\,i!\,(m-i)!}
		\leq 
			(m+1)
			\sum_{i=1}^{m+1}\frac{(m!)\,(m)^{-i}}
										{i!\,(m-i)!}
		\leq
			\frac{(m+1)}{(m+1-i)}
			2
		\leq \\ \leq
			2\cdots
	\end{array} \right.
&
\end{flalign*} 




%	\left\{ \begin{array}{ll}
%		n = 0 \implies
%		\\
%		n = m + 1 \implies
%	\end{array} \right.


\end{document}
















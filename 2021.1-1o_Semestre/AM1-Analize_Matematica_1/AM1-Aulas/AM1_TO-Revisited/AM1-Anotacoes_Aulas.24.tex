\part{11/05 - Primitivação}\relax

\begin{sectionBox}b[0mm]{}

\begin{BM}[align*][\Large]
	f:\mathrm{I}\to\mathbb{R}
\quad
	F \text{ é primitiva de } f
\iff	F'=f
\end{BM}\relax

\vspace{-12mm}

\begin{BM}[align*][\Huge]
	\int f(x)\,\mathrm{d}x= F(x)+c
\end{BM}\relax

\end{sectionBox}



\begin{multicols}{2}

\begin{sectionBox}1{Exercício}

\begin{BM}
	F:
	\begin{cases}
		F(0)=0
	\\	f=\sin(x)
	\end{cases}
\end{BM}\relax

\begin{flalign*}
&
	F=1-\cos(x)
&
\end{flalign*}

\end{sectionBox}

\begin{sectionBox}1{Propriedades de Primitivas}

\begin{BM}[flalign*]
&
	\int (k\,f(x)+g(x))\,\mathrm{d}x\,
=	&\\&
=	\int k\,f(x)\,\mathrm{d}x
+	\int g(x)\,\mathrm{d}x\,
=	&\\&
=	k\,F(x) + C_f + G(x)+ C_g
&
\end{BM}\relax

\end{sectionBox}

\begin{sectionBox}2{Exercicio}

\begin{BM}
	f(x)=1+x+x^2/2!+x^3/3!
\end{BM}\relax

\begin{flalign*}
&
	F(x)
=	x+x^2/2+x^3/6!+x^4/4!+C\,
=	&\\&
=	x+x^2/2+x^3/6!+x^4/4!+1
&
\end{flalign*}

\end{sectionBox}

\begin{sectionBox}1{Primitivas Imediatas}

\begin{BM}
	f(x)=e^{2\,x}
\end{BM}\relax

\begin{flalign*}
&
	F(x)=\int e^{2\,x}\,\mathrm{d}x
=	e^{2\,x}{\color{Red}/2}+c
\impliedby &\\&
\impliedby
	F'(x)={\color{Red}2}\,e^{2\,x}{\color{Red}/2}=e^{2\,x}
&
\end{flalign*}

\end{sectionBox}

\begin{sectionBox}2{Exercicio}
\begin{BM}
	f(x)=\cos(2\,x)
\end{BM}\relax

\begin{flalign*}
&
	F(x)
=	\int \cos(2\,x)\,\mathrm{d}x
=	\sin(2\,x)/2+C
\impliedby &\\&
\impliedby
	F'(x)=2\,\cos(2\,x)/2=f(x)
&
\end{flalign*}
\end{sectionBox}

\begin{sectionBox}2{Exercicio}
\begin{BM}
	f(x)=2\,x/(1+x^2)
\end{BM}\relax

\begin{flalign*}
&
	F(x)
=	\int 2\,x\,\mathrm{d}x/(1+x^2)
=	\ln(1+x^2)+c
\impliedby &\\&
\impliedby
	F'(x)=\frac{2\,x}{1+x^2}
&
\end{flalign*}
\end{sectionBox}

\end{multicols}



























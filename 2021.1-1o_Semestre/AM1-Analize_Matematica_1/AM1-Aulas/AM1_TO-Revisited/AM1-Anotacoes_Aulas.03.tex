% !TEX root = ./AM1-Anotacoes_Aulas.03.tex
\providecommand\mainfilename{"./AM1-Anotacoes_Aulas.tex"}
\providecommand \subfilename{}
\renewcommand   \subfilename{"./AM1-Anotacoes_Aulas.03.tex"}
\documentclass[\mainfilename]{subfiles}

% \tikzset{external/force remake=true} % - remake all

\begin{document}

% \graphicspath{{\subfix{./.build/figures/AM1-Anotacoes_Aulas.03}}}
% \tikzsetexternalprefix{./.build/figures/AM1-Anotacoes_Aulas.03/graphics/}

\mymakesubfile{3}
[AM1]
{19/03 -- Indução por igualdade} % Subfile Title
{19/03 -- Indução por igualdade} % Part Title

\begin{questionBox}1{ % Q1
} % Q1
	\begin{multicols}{2}
        \begin{questionBox}2{ % Q1.1
            Prove que algum numero pertence ao conjunto
        } % Q1.1
            \answer{}
            \begin{flalign*}
                &
                    \text{Seja } V
                    =\left\{
                        P_{(n)}\quad\forall\,n\in\dominio
                    \right\}
                &
            \end{flalign*}
        \end{questionBox}
        \begin{questionBox}2{ % Q1.2
            Prove que o proximo pertence ao conjunto
        } % Q1.2
            \answer{}
            \begin{flalign*}
                &
                    P_{(x)} \in V;
                    \,P_{(x+1)}\in V
                &
            \end{flalign*}
        \end{questionBox}
    \end{multicols}
\end{questionBox}

\begin{exampleBox}1{ % E
} % E
    \begin{BM}
        \sum\limits_{i=0}^{n} \frac{1}{2^i} 
        = 2-\frac{1}{2^n}
        \quad
        \forall\,n\in\mathbb{N}_0
    \end{BM}
\end{exampleBox}

\begin{sectionBox}b{} % S
    \begin{multicols}{2}
        \begin{sectionBox}2{\(n=0\)} % S
            \begin{flalign*}
                &
                    \sum\limits_{i=0}^{0} \frac{1}{2^i}
                    = 2-\frac{1}{2^0}
                    = 2-1=1
                &
            \end{flalign*}
        \end{sectionBox}
        \begin{sectionBox}2{\(n=m+1\)} % S
            \begin{flalign*}
                &
                    \sum\limits_{i=0}^{m+1}{\frac{1}{2^i}}
                    = \sum\limits_{i=0}^{m}{
                        \frac{1}{2^i} + \frac{1}{2^{m+1}}
                    }
                    = &\\&
                    = 2-\frac{1}{2^{m}}+\frac{1}{2^{m+1}}
                    = 2-\frac{1}{2^{m+1}}
                &
                \end{flalign*}
        \end{sectionBox}
    \end{multicols}
\end{sectionBox}

\end{document}
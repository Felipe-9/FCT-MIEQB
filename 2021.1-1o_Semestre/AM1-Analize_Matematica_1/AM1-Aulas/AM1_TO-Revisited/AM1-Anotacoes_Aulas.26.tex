\part{17/05 - Introdução a Trigonometria Hiperbólica}

\begin{multicols}{2}

\begin{sectionBox}1b{Funções Trigonométricas Hiperbólicas}
\begin{BM}
	\cosh(x)=\frac{e^x+e^{-x}}{2}
\\	\sinh(x)=\frac{e^x-e^{-x}}{2}
\end{BM}\relax

\vspace{-10mm}

\begin{BM}
	\cosh'(x)=\sinh(x)
\\	\sinh'(x)=\cosh(x)
\end{BM}

{
%\pgfplotsset{height=7cm, width= .6\textwidth}

\begin{center}
\begin{tikzpicture}
\begin{axis}
[	xmajorgrids= true,
	legend pos= south east
]
	
	% Legends
	\addlegendimage{empty legend}
	\addlegendentry[Blue]{$\cosh(x)$}
	\addlegendimage{empty legend}
	\addlegendentry[Red]{$\sinh(x)$}
	
	\addplot
	[	smooth,
		thick,
		Blue,
		domain = -2:2,
		samples = 0.2*\mysampledensity,
	] {cosh(\x)};
	
	\addplot
	[	smooth,
		thick,
		Red,
		domain = -2:2,
		samples = 0.2*\mysampledensity,
	] {sinh(\x)};
	
\end{axis}
\end{tikzpicture}
\end{center}
}

\end{sectionBox}

\begin{sectionBox}2{Relação fundamental da trigonometria hiperbólica clássica}
\begin{BM}
	\cosh^2(x)-\sinh^2(x)=1
\end{BM}

\begin{flalign*}
&
	\cosh^2(x)-\sinh^2(x)\,
=	&\\&
=	\left(\frac{e^x+e^{-x}}{2}\right)^2
-	\left(\frac{e^x-e^{-x}}{2}\right)^2
=	&\\&
=	(1/4)
	\left(
		e^{2\,x}+2+e^{-2\,x}-e^{2\,x}+2-e^{-2\,x}
	\right)
=	1
&
\end{flalign*}
\end{sectionBox}

\begin{sectionBox}1{Exemplo: $f(x)=\sqrt{1+x^2}$}
\begin{flalign*}
&
	F(x)
=	\int\sqrt{1+x^2}\,\mathrm{d}x\,
=	&\\&
=	\int\sqrt{1+\sinh^2(t)}\,\odv{\sinh(t)}{t}\,\mathrm{d}t
=	\int\cosh^2(t)\,\mathrm{d}t\,
=	&\\&
=	\int\left(\frac{e^t+e^{-t}}{2}\right)^2\,\mathrm{d}t
=	\int\frac{e^{2\,t}+2+e^{-2\,t}}{4}\,\mathrm{d}t\,
=	&\\&
=	(1/4)
	\left(
		e^{2\,t}/2+2\,t-e^{-2\,t}/2
	\right)+c\,
=	&\\&
=	\left(\frac{e^{t}-e^{-t}}{2}\right)^2(1/4)+t/4+c\,
=	&\\&
=	\frac{\sinh^2{t}+t}{4}+c
=	\frac{x^2+\arcsinh(x)}{4}+c
&
\end{flalign*}
\end{sectionBox}

\begin{sectionBox}1{Exemplo: $f(x)=\sqrt{1-x^2}$}
\begin{flalign*}
&
	F(x)
=	\int\sqrt{1-x^2}\,\mathrm{d}x\,
=	&\\&
=	\int\sqrt{1-\sin^2(t)}\,\odv{\sin(t)}{t}\,\mathrm{d}t
=	\int\cos^2(t)\,\mathrm{d}t\,
=	&\\&
=	\int(\cos(2\,t)/2+1/2)\,\mathrm{d}t
=	\sin(2\,t)/2+t/2+c\,
=	&\\&
=	\sin(2\,\arcsin(x))/4+\arcsin(x)/2+c\,
=	&\\&
=	(2/4)(\sin(\arcsin(x)\,\cos(\arcsin(x))))
+	&\\&
+	\arcsin(x)/2+c
=	(x\,\sqrt{1-x^2}+\arcsin(x))/2+c
&
\end{flalign*}
\end{sectionBox}

\begin{sectionBox}1{Desafio: $f(x)=1/\sqrt{1+x^2}$}
\begin{flalign*}
&
	F(x)
=	\int\sqrt{1+x^2}^{-1}\,\mathrm{d}x\,
=	&\\&
=	\int\sqrt{1+\sinh^2(t)}^{-1}\,\odv{\sinh(x)}{t}\,\mathrm{d}t
=	\int\frac{\cosh(t)}{\cosh(t)}\,\mathrm{d}t\,
=	&\\&
=	t+c
=	\arcsinh(x)+c
&
\end{flalign*}
\end{sectionBox}

\end{multicols}



































\part{23/04 - Continuidade: Teoremas}

\begin{multicols}{2}

\vspace{5mm}

\noindent%
\begin{minipage}{\linewidth}

\BMold{
	f \text{ é continua em } [a,b]
}\relax

\begin{flalign*}
&
\impliedby
	f \text{ é continua em } x\in (a,b)\,
\land &\\&
\land
	\lim_{x\to a^-} f(x) = f(a)
\land
	\lim_{x\to b^+} f(x) = f(b)
&
\end{flalign*}


\end{minipage}

\vspace{5mm}

\noindent%
\begin{minipage}{\linewidth}

\section{Teorema de Bolzano}
\label{bolzano}

\BMold{
	f: [a,b] \to \mathbb{R}
\land
	f \text{ é continua em } [a,b]
\\	: f(a) < f(b)\quad\forall\,k: f(a) < k < f(b)
\\	\implies\exists\,c\in(a,b): f(c) = k
}\relax


\end{minipage}

\vspace{5mm}

\noindent%
\begin{minipage}{\linewidth}

\section{Teorema de Weierstrass}
\label{weirestrass}

\BMold{
	f: [a,b]\to\mathbb{R}
\land
	f \text{ é continua em } [a,b]
\\
	\therefore\,\exists\,\{x_{\max}, x_{\min}\}\subset [a,b]
:	\\
:	f(x_{\min}) \leq f(x) \leq f(x_{\max})
}\relax

\end{minipage}

\vspace{5mm}

\noindent%
\begin{minipage}{\linewidth}

\section{Corolário de Bolzano e Weirestrass}

\BMold{
	f: [a,b]\to\mathbb{R}\land f\text{ é continua em } [a,b]
\\	\therefore
	f([a,b]) = [x_{\max}, x_{\min}]
}\relax

\end{minipage}

\vspace{5mm}

\noindent%
\begin{minipage}{\linewidth}

\subsection{Exemplo Aplicação}

\BMold{e^{-x}=x}\relax

\begin{flalign*}
&
	f:[0,1]\to\mathbb{R};\quad f(x) = e^{-x}-x
\implies &\\&
\implies
	f(0) = 1
;\	f(1) = 1/e-1 < 0
&\\&
	\therefore\exists\,c\in[0,1]:e^{-c} = c
&
\end{flalign*}

\end{minipage}

\vspace{5mm}

\noindent%
\begin{minipage}{\linewidth}

\subsection{Exemplo Aplicação}

\BMold{g:[0,\pi]\to\mathbb{R};\quad g(x)=x\,\sin(x)}\relax
\begin{flalign*}
&
	g(0)=g(\pi)=0\therefore\exists\,x_{\max}\geq0
&
\end{flalign*}

\end{minipage}

\end{multicols}
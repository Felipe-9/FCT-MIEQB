\part{04/05 - Formula de Taylor}

\begin{multicols}{2}

\vspace{5mm}

\noindent%
\begin{minipage}{\linewidth}

\section{Aproximação com funções polinomiais}

\begin{BM}
	f(x)=e^x
	\begin{cases}
		f  (0)=1
	\\	f' (0)=1
	\\	f''(0)=1
	\end{cases}
\end{BM}

\subsection{Polinômio de grau 1}
\begin{flalign*}
&
	f_1(x)=f'(0)(x-0)+f(0)=x+1
&
\end{flalign*}

\subsection{Polinômio de grau 2}
\begin{flalign*}
&
	f_2(x)=f''(0)\frac{(x-0)^2}{2}+f'(x-0)+f(0)=x^2/2+x+1
&
\end{flalign*}

\subsection{Polinômio de grau 3}
\begin{flalign*}
&
	f_2(x)
=	f'''(0)\frac{(x-0)^3}{6}
+	f''(0)\frac{(x-0)^2}{2}\,
+	&\\&
+	f'(x-0)
+	f(0)
=	x^3/6+x^2/2+x+1
&
\end{flalign*}

{
%\pgfplotsset{height=7cm, width= .6\textwidth}


\begin{center}
\begin{tikzpicture}
\begin{axis}
[	xmajorgrids= true,
	legend pos= north west
]
	% Legends
	\addlegendimage{empty legend}
	\addlegendentry[Red]{$e^x$}
	\addlegendimage{empty legend}
	\addlegendentry[Green]{$x+1$}
	\addlegendimage{empty legend}
	\addlegendentry[Blue]{$x^2/2+x+1$}
	\addlegendimage{empty legend}
	\addlegendentry[Purple]{$x^3/6+x^2/2+x+1$}
	
	\addplot
	[	smooth,
		thick,
		Red,
		domain = -2:2,
		samples = 0.4*\mysampledensity,
	] {exp(x)};
	
	\addplot
	[	smooth,
		thick,
		Green,
		domain = -2:2,
		samples = 0.4*\mysampledensity,
	] {x+1};
	
	\addplot
	[	smooth,
		thick,
		Blue,
		domain = -2:2,
		samples = 0.4*\mysampledensity,
	] {x^2/2+x+1};
	
	\addplot
	[	smooth,
		thick,
		Purple,
		domain = -2:2,
		samples = 0.4*\mysampledensity,
	] {x^3/6+x^2/2+x+1};
	
\end{axis}
\end{tikzpicture}
\end{center}
}

\end{minipage}

\vspace{5mm}

\noindent%
\begin{minipage}{\linewidth}

\section{Formula de Taylor com grau $n$ e resto de lagrange}
\label{Taylor}

\begin{BM}[flalign*]
&	f:\ \exists\,\odv[ord=n+1]{f}{x}
	\quad\forall\,x\in\mathrm{I}:a\in\mathrm{I}
\implies &\\&
\implies
	f_{(\text{ord}=n,\ \text{em}=a)}\,
=	&\\&
=	\sum_{k=0}^{n}\odv[ord=k]{f}{x}\frac{(x-a)^k}{k!}
+	r_n(x-a)
&\\&
	\forall\,x\in\mathrm{I}\backslash\{a\}
&
\end{BM}

\begin{flalign*}
&
	r_n(x-a)=\odv[ord=n+1]{f(c)}{x}\frac{(x-a)^{n+1}}{(n+1)!}
\qquad c\in [x,a]
&
\end{flalign*}


\end{minipage}

\vspace{5mm}

\noindent%
\begin{minipage}{\linewidth}

\subsection{Exemplo: Ordem 3 em 0}
\begin{questionBox}{}
\begin{BM}
	f(x)=\sin(x)
\end{BM}

\begin{flalign*}
&
	f_3(x)
=	\sin(0)
+	\cos(0)\,x
-	\sin(0)\,x^2/2\,
+	&\\&
-	\cos(0)\,x^3/6
+	\sin(c)\,x^4/24\,
=	&\\&
=	x-x^3/6+\sin(c)\,x^4/24
&
\end{flalign*}
\end{questionBox}


\end{minipage}


\end{multicols}















\part{28/05 - Integrais Indefinidas}

\begin{multicols}{2}

\begin{sectionBox}1{Propriedade de Integração}
\paragraph{Por convenção}

\begin{BM}
	\int_{a}^{b}f(x)\,\mathrm{d}x
=	\int_{b}^{a}f(x)\,\mathrm{d}x
\end{BM}

\end{sectionBox}

\begin{sectionBox}1{Propriedade de Integração}
\begin{BM}[align*][\normalsize]
	f
	\begin{cases}
		f:\mathrm{I}\to\mathbb{R}\,\land
	\\	\text{integrável em }\mathrm{I}
	\end{cases}
\\	\{a,b,c\}\in\mathrm{I}
\end{BM}\relax

\begin{BM}[flalign*]
&
	\int_{a}^{b}f(x)\,\mathrm{d}x
=	&\\&
=	\int_{a}^{c}f(x)\,\mathrm{d}x
+	\int_{c}^{b}f(x)\,\mathrm{d}x
&
\end{BM}
\end{sectionBox}

\begin{sectionBox}1{Integral Indefinida}
\begin{BM}[align*][\normalsize]
	f
	\begin{cases}
		f:\mathrm{I}\to\mathbb{R}
	\\	f 
		\text{ integrável em todo subintervalo}
	\\	\text{compacto de }
		\mathrm{I}
	\end{cases}
\end{BM}

\paragraph{Definição:}
\begin{BM}[align*][\normalsize]
	F_a:x\to\int_{a}^{x}f(t)\,\mathrm{d}t
\end{BM}
\end{sectionBox}\relax

\begin{sectionBox}1{Teorema fundamental do cálculo}

\begin{BM}[align*][\normalsize]
	f:\mathrm{I}\to\mathbb{R}
\end{BM}\relax

\begin{BM}
	F_a(x)=\int_{a}^{x}f(t)\,\mathrm{d}t
\end{BM}

$F_a$ é Primitiva de $f$ que se anula em $x=a$
\end{sectionBox}

\begin{sectionBox}1{Regra de Barrow}
\label{Barrow}

\begin{BM}[align*][\normalsize]
	f
	\begin{cases}
		\text{continua em } [a,b]
	\\	F' = f
	\end{cases}
\end{BM}

\begin{BM}
	\int_{a}^{b}f(x)\,\mathrm{d}x
=	F(b)-F(a)
=	\Delta F(x)\big|_{a}^{b}
\end{BM}

\end{sectionBox}\relax

\newpage

\begin{sectionBox}1{Exercicio}

\begin{BM}[align*][\normalsize]
	\int_{0}^{1}x^2\,\mathrm{d}x
\end{BM}

{
%\pgfplotsset{height=7cm, width= .6\textwidth}


\begin{center}
\begin{tikzpicture}
\begin{axis}
[	xmajorgrids= true,
	legend pos= north west
]
	% Legends
	\addlegendimage{empty legend}
	\addlegendentry{$x^2$}
	
	\addplot
	[	fill=White,
		fill opacity=0.3,
		smooth,
		thick,
%		Red,
		domain = 0:1,
		samples = 0.1*\mysampledensity,
	] {x^2} \closedcycle;
	
\end{axis}
\end{tikzpicture}
\end{center}
}

\begin{flalign*}
&
	\int_{0}^{1}x^2\,\mathrm{d}x
=	\Delta F(x)\big|_{0}^{1}
=	(1/3)(1^3-0^3)=1/3
&
\end{flalign*}

\end{sectionBox}\relax

\begin{sectionBox}1{Exercicio}

\begin{BM}[align*][\normalsize]
	\int_{0}^{\pi}\sin(x)\,\mathrm{d}x
\end{BM}

{
%\pgfplotsset{height=7cm, width= .6\textwidth}


\begin{center}
\begin{tikzpicture}
\begin{axis}
[	xmajorgrids= true,
	legend pos= north west
]
	% Legends
	\addlegendimage{empty legend}
	\addlegendentry{$\sin(x)$}
	
	\addplot
	[	fill=White,
		fill opacity=0.3,
		smooth,
		thick,
%		Red,
		domain = 0:pi,
		samples = 0.1*\mysampledensity,
	] {sin(deg(\x))} \closedcycle;
	
\end{axis}
\end{tikzpicture}
\end{center}
}\relax

\begin{flalign*}
&
	\int_{0}^{\pi}\sin(x)\,\mathrm{d}x
=	\Delta(-\cos(x))\big|_{0}^{\pi}
=	2
&
\end{flalign*}

\end{sectionBox}\relax

\begin{sectionBox}1{Exercicio}
\begin{BM}[align*][\normalsize]
	\int_{0}^{\ln(3)}e^x\,\mathrm{d}x
\end{BM}



{
%\pgfplotsset{height=7cm, width= .6\textwidth}


\begin{center}
\begin{tikzpicture}
\begin{axis}
[	xmajorgrids= true,
	legend pos= north west
]
	% Legends
	\addlegendimage{empty legend}
	\addlegendentry{$e^x$}
	
	\addplot
	[	fill=White,
		fill opacity=0.3,
		smooth,
		thick,
%		Red,
		domain = 0:1.09861228866811,
		samples = 0.3*\mysampledensity,
	] {e^\x} \closedcycle;
	
\end{axis}
\end{tikzpicture}
\end{center}
}\relax

\begin{flalign*}
&
	\int_{0}^{\ln(3)}e^x\,\mathrm{d}x
=	\Delta(e^x)\big|_{0}^{\ln(3)}
=	2
&
\end{flalign*}

\end{sectionBox}\relax


\begin{sectionBox}1{Exercicio}
\begin{BM}[align*][\normalsize]
	\int_{-1}^{1}\mathrm{d}x/(1+x^2)
\end{BM}

{
%\pgfplotsset{height=7cm, width= .6\textwidth}


\begin{center}
\begin{tikzpicture}
\begin{axis}
[	xmajorgrids= true,
	legend pos= north west
]
	% Legends
	\addlegendimage{empty legend}
	\addlegendentry{$e^x$}
	
	\addplot
	[	fill=White,
		fill opacity=0.3,
		smooth,
		thick,
%		Red,
		domain = -1:1,
		samples = 0.2*\mysampledensity,
	] {1/(1+\x^2)} \closedcycle;
	
\end{axis}
\end{tikzpicture}
\end{center}
}\relax

\begin{flalign*}
&
	\int_{-1}^{1}\mathrm{d}x/(1+x^2)
=	\Delta
	\left(
		\arctan(x)
	\right)\big|_{-1}^{1}
=	\pi/2
&
\end{flalign*}

\end{sectionBox}\relax

\end{multicols}





























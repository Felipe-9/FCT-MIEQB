\part{32/05 - Teorema Fundamental do Cálculo}

\section*{Area dentre funções}\relax

\begin{multicols}{2}

\begin{sectionBox}1[0mm]{Exercicio}
\begin{BM}[align*][\normalsize]
	f(x)=x+6;\ g(x)=x^2
\end{BM}

\begin{sectionBox}3{$\{a,b\}$}
\begin{flalign*}
&
	x
=	\frac{-1\pm\sqrt{1-4*6*(-1)}}{-1*2}
=	\frac{1\mp5}{2}
&\\&
	\therefore
	a=-2; b=3
&
\end{flalign*}
\end{sectionBox}\relax

{
%\pgfplotsset{height=7cm, width= .6\textwidth}


\begin{center}
\begin{tikzpicture}
\begin{axis}
[	xmajorgrids= true,
	legend pos= north west
]
	% Legends
	\addlegendimage{empty legend}
	\addlegendentry[Red]{$x+6$}
	\addlegendimage{empty legend}
	\addlegendentry[Blue]{$x^2$}
	
	\addplot
	[
		name path= A,
		smooth,
		thick,
		Red,
		domain = -3:4,
		samples = 2,
	] {x+6};
	
	\addplot
	[
		name path= B,
		smooth,
		thick,
		Blue,
		domain = -3:4,
		samples = 10,
	] {x^2};
	
	\addplot
	[
		White,
		opacity=.3,
	]
	fill between
	[
		of= A and B,
		soft clip={domain=-2:3}
	];
	
\end{axis}
\end{tikzpicture}
\end{center}
}\relax

\begin{flalign*}
&
	\int_{-2}^{3}(x+6-x^2)\,\mathrm{d}x
	\Delta
	\left(
		x^2/2+6\,x-x^3/3
	\right)\big|_{-2}^{3}
=	&\\&
=	5+30-11/3
=	116/3
&
\end{flalign*}

\end{sectionBox}\relax


\vfill


\begin{sectionBox}1[0mm]{Exercicio}

\begin{BM}[align*][\normalsize]
	f(x)=|x|;\ g(x)=x^3
\end{BM}

\begin{sectionBox}3{$\{a,b\}$}
\begin{flalign*}
&
	x:f(x)-g(x)=|x|-x^3
=	&\\&
=	x(1-x^2)=0
	\quad\forall\,x\geq0
&\\&
	\therefore
	a=0;\ b=1
&
\end{flalign*}
\end{sectionBox}

{
%\pgfplotsset{height=7cm, width= .6\textwidth}


\begin{center}
\begin{tikzpicture}
\begin{axis}
[	ymax=1.25, ymin=-0.25,
	          xmin=-0.5,
	xmajorgrids= true,
	legend pos= north west,
]
	% Legends
	\addlegendimage{empty legend}
	\addlegendentry[Red]{$|x|$}
	\addlegendimage{empty legend}
	\addlegendentry[Blue]{$x^3$}
	
	\addplot
	[
		name path= A,
		thick,
		Red,
		domain = -1:2,
		samples = 10,
	] {abs(\x)};
	
	\addplot
	[
		name path= B,
		smooth,
		thick,
		Blue,
		domain = -1:2,
		samples = 0.2\mysampledensity,
	] {\x^3};
	
	\addplot
	[
		White,
		opacity=.3,
	]
	fill between
	[
		of= A and B,
		soft clip={domain=0:1}
	];
	
\end{axis}
\end{tikzpicture}
\end{center}
}\relax

\begin{flalign*}
&
	\int_{0}^{1}(x-x^3)\,\mathrm{d}x
=	\Delta
	\left(
		x^2/2-x^4/4
	\right)\big|_{0}^{1}
=	1/4
&
\end{flalign*}

\end{sectionBox}\relax



\newpage

\section*{Integrais Indefinidas Compostas}

\begin{sectionBox}1{Exemplo}

\begin{BM}[align*][\normalsize]
	H(x)=\int_{0}^{\ln(x)}\sin(e^t)\,\mathrm{d}t
\end{BM}

\begin{flalign*}
&
	H'(x)
=	\sin(e^{\ln(x)})/x
=	\sin(x)/x
&
\end{flalign*}

\end{sectionBox}



\begin{sectionBox}1{Exemplo}

\begin{BM}[align*][\normalsize]
	F(x)=\int_{x}^{1}e^t\,\mathrm{d}t
\end{BM}

\begin{flalign*}
&
	F'(x)=-e^x
&
\end{flalign*}

\end{sectionBox}



\begin{sectionBox}1{Exemplo}

\begin{BM}[align*][\normalsize]
	H(x)=\int_{x}^{x^2}f(t)\,\mathrm{d}t
\end{BM}

\begin{flalign*}
&
	H'(x)=f(x^2)\,2\,x-f(x)
&
\end{flalign*}

\end{sectionBox}



\begin{sectionBox}1{Exemplo}
\begin{BM}
	H(x)=\int_{x}^{\tan(x)}f(t)\,\mathrm{d}t
\end{BM}

\begin{flalign*}
&
	H'(x)=f(\tan(x))/(1+\tan^2(x))-f(x)
&
\end{flalign*}
\end{sectionBox}



\end{multicols}


































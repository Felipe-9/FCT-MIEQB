\part{12/04 - Sucessão de Cauchy}

\BMold{
&
	u_{(n)}: | u_{(m)} - u_{(n)} | < \varepsilon
\quad\forall\,\{m,n,p\}\in\mathbb{N}: m>p\land n>p
&
}

\label{cauchy}

\begin{multicols}{2}


\section{Criterio suficiente para uma sucessão ser Cauchy}

\begin{flalign*}
&
	|u_{(n+2)} - u_{(n+1)}|\leq \alpha |u_{(n+1)} - u_{(n)}|
	&\\&
	\forall\,n\in\mathbb{N}: n>p\land \alpha\in(0,1)
\implies	&\\&
\implies
	u_{(n)} \text{ é Cauchy}
&
\end{flalign*}

\vfill

\subsection{Exemplo}

\begin{align*}
&
	u_{(n)} = 
	\begin{cases}
 	%
		0\quad& n=1
	\\	(2/3)\,u_{(n-1)}+1\quad& n > 1
	%
	\end{cases}
&
\end{align*}

\begin{flalign*}
&
	|u_{(n+2)} - u_{(n+1)}|
=	&\\&
=	|(2/3)\,u_{(n+1)}+1-(2/3)\,u_{(n)}-1|
=	&\\&
=	(2/3)\,|u_{n+1}-u_{(n)}|
&\\&	\therefore
	u_{(n)} \text{ é Cauchy}
&
\end{flalign*}


\section{Convergencia}

\begin{align*}
	u_n
	\begin{cases}
		0					&\quad n=1
	\\	u_{(n-1)}\,2/3 + 1	&\quad n>1
	\end{cases}
\end{align*}

\begin{flalign*}
&
	u_{n} \to l
;\,	u_{n}\,2/3 + 1\to l\,2/3 + 1
\quad \therefore l=l\,2/3 + 1
\implies &\\&
\implies
	l = 3
&
\end{flalign*}

\end{multicols}
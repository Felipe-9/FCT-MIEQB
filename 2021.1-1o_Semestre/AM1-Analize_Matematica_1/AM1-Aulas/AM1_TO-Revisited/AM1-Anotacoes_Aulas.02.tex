% !TEX root = ./AM1-Anotacoes_Aulas.02.tex
\providecommand\mainfilename{"./AM1-Anotacoes_Aulas.tex"}
\providecommand \subfilename{}
\renewcommand   \subfilename{"./AM1-Anotacoes_Aulas.02.tex"}
\documentclass[\mainfilename]{subfiles}

% \tikzset{external/force remake=true} % - remake all

\begin{document}

% \graphicspath{{\subfix{./.build/figures/AM1-Anotacoes_Aulas.02}}}
% \tikzsetexternalprefix{./.build/figures/AM1-Anotacoes_Aulas.02/graphics/}

\mymakesubfile{2}
[AM1]
{16/03 -- Conceitos Básicos II} % Subfile Title
{16/03 -- Conceitos Básicos II} % Part Title

% Conjunto aberto e fechado
\begin{sectionBox}b{} % S
    \begin{multicols}{2}

        \begin{sectionBox}1{Conjunto Aberto} % S
            \label{conjunto aberto}
            
            \begin{flalign*}
                &
                    X \text{ é um conjunto aberto}
                    \iff &\\&
                    \iff
                    X=\interior(X)
                    % \hyperref[interior]{\text{Int}}(X)
                &
            \end{flalign*}
    
        \end{sectionBox}
    
        % Conjunto Fechado
        \begin{sectionBox}1{Conjunto Fechado} % S
            \label{conjunto fechado}
            
            \begin{flalign*}
                &
                    X \text{ é um conjunto fechado}
                    \iff &\\&
                    \iff
                    X
                    = \interior(X)
                    % \hyperref[interior]{\text{Int}(X)}
                    \cup\fronteira(X)
                    % \hyperref[fronteira]{\text{Fr}(X)}
                &
            \end{flalign*}
        \end{sectionBox}

    \end{multicols}
\end{sectionBox}

\begin{sectionBox}b{} % S
    \begin{multicols}{2}

        \begin{sectionBox}1{Feixe} % S
            % Feixe de X
            \label{feixe}
            \begin{flalign*}
                &
                    \bar X
                    =\interior(X)
                    % \hyperref[interior]{\text{Int}(X)}
                    \cup\fronteira(X)
                    % \hyperref[fronteira]{\text{Fr}(X)}
                &
            \end{flalign*}
        \end{sectionBox}
        \begin{sectionBox}1{Ponto de Acumulação} % S
            \label{ponto de acumulacao}
            \begin{flalign*}
                &
                    x \text{ é um ponto de acumulação de } X
                    \iff &\\&
                    \iff
                    \vizinhanca{x}\cap\left( X\backslash\{x\} \right)
                    \neq \emptyset
                &
            \end{flalign*}
        \end{sectionBox}
    
    \end{multicols}
\end{sectionBox}

\end{document}
\part{03/05 - Teoremas da diferenciabilidade}

\begin{multicols}{2}

\vspace{5mm}

\noindent%
\begin{minipage}{\linewidth}

\section{Teorema de Rolle}
\label{Rolle}

\begin{BM}
	f
	\begin{cases}
		f:[a,b]\to\mathbb{R}\,\land
	\\	f \text{ é continua em } [a,b]\,\land
	\\	\exists\,f'(x)\quad\forall\,x\in(a,b)\,\land
	\\	f(a) = f(b)
	\end{cases}
\\\\
	\therefore
	\begin{array}{ll}
		\exists\,c\in[a,b]: f'(c) = 0\,\land
	\\	\max(f),\min(f) \subset f(c)
	\end{array}
\end{BM}\relax

\begin{BM}
	\min(f) = x_0 \in (a,b):
\\	f(x_0) \leq f(x)\quad\forall\,x\in(a,b)
\end{BM}

\begin{flalign*}
&
	\lim_{x\to x_0^+}
	\frac
		{f(x)- f(x_0)}
		{x-x_0}
\geq 0;
	\lim_{x\to x_0^-}
	\frac
		{f(x)- f(x_0)}
		{x-x_0}
\leq 0
&\\&
	\therefore
	f'(x_0) = 0
&
\end{flalign*}

\end{minipage}

\vspace{5mm}

\noindent%
\begin{minipage}{\linewidth}
\subsection{Demonstração}

\begin{BM}
	h
	\begin{cases}
		\mathbb{R}	\to \mathbb{R}
	\\	x			\to x\,\sin(x)
	\end{cases}
\end{BM}

{
%\pgfplotsset{height=7cm, width= .6\textwidth}

\begin{center}
\begin{tikzpicture}
\begin{axis}
	
	% Legends
	\addlegendimage{empty legend}
	\addlegendentry[
		font=\boldmath\footnotesize,
		Red,
	]{$x\,\sin(x)$}
	
	\addplot
	[	smooth,
		Red,
		domain=-30:30,
		samples= 3*\mysampledensity,
	]	{x*sin(deg(x))};
	
	\addplot
	[	smooth,
		Red,
		mark=o,
		domain=-30:30,
		draw= none,
	] coordinates {
		(-29.84513020910300, -29.84513020910300)
		(-26.70353755551320,  26.70353755551320)
		(-23.56194490192340, -23.56194490192340)
		(-20.42035224833360,  20.42035224833360)
		(-17.27875959474380, -17.27875959474380)
		(-14.13716694115400,  14.13716694115400)
		(-10.99557428756420, -10.99557428756420)
		( -7.85398163397441,   7.85398163397441)
		( -4.71238898038462,  -4.71238898038462)
		( -1.57079632679483,   1.57079632679483)
		(  1.57079632679496,   1.57079632679496)
		(  4.71238898038475,  -4.71238898038475)
		(  7.85398163397454,   7.85398163397454)
		( 10.99557428756430, -10.99557428756430)
		( 14.13716694115410,  14.13716694115410)
		( 17.27875959474390, -17.27875959474390)
		( 20.42035224833370,  20.42035224833370)
		( 23.56194490192350, -23.56194490192350)
		( 26.70353755551330,  26.70353755551330)
		( 29.84513020910310, -29.84513020910310)
	};
	
	\addplot
	[	smooth,
		Green,
		mark=o,
		domain=-30:30,
		draw= none,
	] coordinates {
		(-28.27433388230810, 0)
		(-25.13274122871830, 0)
		(-21.99114857512850, 0)
		(-18.84955592153870, 0)
		(-15.70796326794890, 0)
		(-12.56637061435910, 0)
		( -9.42477796076931, 0)
		( -6.28318530717952, 0)
		( -3.14159265358979, 0)
		(0, 0)
		(  3.14159265358979, 0)
		(  6.28318530717958, 0)
		(  9.42477796076937, 0)
		( 12.56637061435920, 0)
		( 15.70796326794900, 0)
		( 18.84955592153880, 0)
		( 21.99114857512860, 0)
		( 25.13274122871840, 0)
		( 28.27433388230820, 0)
	};
	
\end{axis}
\end{tikzpicture}
\end{center}
}

\end{minipage}

\vspace{5mm}

\noindent%
\begin{minipage}{\linewidth}

\section{Teorema de Lagrange}
\label{Lagrange}

\begin{BM}
	f
	\begin{cases}
		f:[a,b] \to \mathbb{R}
	\\	f \text{ é continua em } [a,b]
	\\	\exists\, f'(x)\ \forall\,x\in(a,b)
	\end{cases}
\\\\
	\therefore
	\exists\, c\in(a,b):f'(c)
=	\frac{f(b)-f(a)}{b-a},
	\\
	f(b) = f(a) + f'(c)\,(b-a)
\end{BM}

\end{minipage}

\vspace{5mm}

\noindent%
\begin{minipage}{\linewidth}

\subsection{Corolário}

\begin{BM}
	f
	\begin{cases}
		f:[a,b]\to\mathbb{R}\,\land
	\\	f \text{ é continua em } [a,b]\,\land
	\\	\exists\,f'(x)\quad\forall\,x\in(a,b)\,\land
	\\	f'(x)>0\quad\forall\,x\in(a,b)
	\end{cases}
\\	\therefore
	f \text{ é estritamente crescente em } (a,b)
\end{BM}\relax

\begin{flalign*}
&
	\{x,y\} \in (a,b): f(x) < f(y)
\implies &\\&
\implies
	f(y) = f(x) + f'(c)\,(y-x)
&
\end{flalign*}

\end{minipage}

\vspace{5mm}

\noindent%
\begin{minipage}{\linewidth}

\subsection{Ideia de Demonstração do teorema de Lagrange}
\paragraph{Nota:} REVER

\begin{BM}
	h(x) = (b-a)\,f(x) - (f(b)-f(a))(x-a)
\end{BM}\relax

\begin{flalign*}
&
	\begin{cases}
		h(a) = (b-a)\,f(a)
	\\	h(b) = (b-a)\,f(b)-(f(b)-f(a))(b-a) = (b-a)\,f(a)
	\end{cases}
&\\&
\implies
	h'(x)= (b-a)f'(x)-(f(b)-f(a))=0
\implies &\\&
\implies
	f(b)=f(a)+f'(x)(b-a)
&
\end{flalign*}

\end{minipage}

\vspace{5mm}

\noindent%
\begin{minipage}{\linewidth}

\section{Regra de L'Hospital-Cauchy}
\begin{BM}
	a\in\mathbb{R}\cup\{-\infty\};\quad
	\varepsilon>0;
\\	\{f,g\}
	\begin{cases}
		\text{continua em } (a,\infty)\,\land
	\\	\left(
		\left(
			\text{diferenciaveis em } (a,a+\varepsilon)\,\land
			a\in\mathbb{R}
		\right)\,\lor
		\right.
	\\
		\left.
			\text{diferenciaveis em } (-\infty,-\varepsilon)\,\land
			a=-\infty
		\right)
	\end{cases}
\end{BM}

\begin{BM}
	\lim_{x\to a^+}f(x)
=	\lim_{x\to a^+}g(x)
=	\begin{cases}
		0
	\\	+\infty
	\end{cases}
\end{BM}

\begin{BM}
	\exists L\in\mathbb{R}\cup\{-\infty,+\infty\}:
	L=\lim_{x\to a^+}f'(x)/g'(x)
\implies \\
\implies
	\lim_{x\to a^+}f(x)/g(x)=\lim_{x\to a^+}f'(x)/g'(x)=L
\end{BM}


\end{minipage}

\end{multicols}





























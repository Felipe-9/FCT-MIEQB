\part{16/04}

\begin{multicols}{2}

\noindent%
\begin{minipage}{\linewidth}

\section{}

\BMold{
	H(x)
=	\begin{cases}
		\phantom{-}1	& \quad x\geq0
	\\	-1			& \quad x < 0
	\end{cases}
}

\begin{tikzpicture}
\begin{axis}
	[]
	
	\addplot
	[
		Red,
	] coordinates {
		(-5, -1)
		(-4, -1)
		(-3, -1)
		(-2, -1)
		(-1, -1)
		( 0, -1)
	};
	
	\node [draw= Red, circle, minimum size= 1mm] at (axis cs:0,-1){};
	
	\addplot
	[
		Red,
	] coordinates {
		( 0,  1)
		( 1,  1)
		( 2,  1)
		( 3,  1)
		( 4,  1)
		( 5,  1)
	};
	
	\node [fill= Red, circle, minimum size= 1mm] at (axis cs:0,1){};
	
	
\end{axis}
\end{tikzpicture}


\end{minipage}

\vspace{5mm}

\noindent%
\begin{minipage}{\linewidth}

\section{Definição derivada}
\BMold{
	\lim_{x\to a}
	\frac
		{f(x) - f(a)}
		{x-a}
}

\begin{enumerate}[label=(\roman{enumi})]
\item $f$ não é obrigada a estar definida em $x=a$
\item $(x_n)$ é uma sucessão que aproxima de a por valores
	diferentes de a
\end{enumerate}


\end{minipage}

\vspace{5mm}

\noindent%
\begin{minipage}{\linewidth}

\section{Existencia de limite}
\BMold{
	\exists \lim_{x\to a}f(x)
\quad \iff
}\relax

\begin{flalign*}
&
\iff
\begin{cases}
	\exists \lim_{x\to a^-}f(x)
\\	\exists \lim_{x\to a^+}f(x)
\\	\lim\limits_{x\to a^+}f(x) = \lim\limits_{x\to a^-}f(x)
\end{cases}
&
\end{flalign*}

\end{minipage}

\vspace{5mm}
\noindent
\begin{minipage}{\linewidth}

\section{Exemplo}

\BMold{
	g(a)
=	\begin{cases}
		x^2	& \quad x<0
	\\	x	& \quad x\geq0
	\end{cases}
}\relax

\begin{flalign*}
&
	\exists\lim_{x\to0} g(x)
\iff
	\lim_{x\to 0^-}g(x) = 0
;\	\lim_{x\to 0^+}g(x) = 0
&
\end{flalign*}


\end{minipage}

\vspace{5mm}

\noindent%
\begin{minipage}{\linewidth}

\section{Exemplo}

\begin{tikzpicture}
\begin{axis}
	
	\addlegendimage{empty legend}
	\addlegendentry%
	[
		font=\boldmath\footnotesize,
		Red,
	] {$f(n)=\sin(1/n)$}
	
	\addplot
	[	smooth,
		Red,
		domain= -0.2:0.2,
		samples= 20*\mysampledensity,
	] {
		sin(deg(1/x))
	};
	
\end{axis}
\end{tikzpicture}

\subsection{Usando sucessões para encontrar limites}

\BMold{
	x(n) = 1/(n\,\pi)\quad n\in\mathbb{N}\backslash\{0\}
}\relax

\begin{flalign*}
&
	f(x(n))
=	\sin\left( \frac{1}{1/(n\,\pi)} \right)
=	\sin(n\,\pi)\quad n\in\mathbb{N}\backslash\{0\}
&
\end{flalign*}

\begin{tikzpicture}
\begin{axis}
	
	\addlegendimage{empty legend}
	\addlegendentry%
	[
		font=\boldmath\footnotesize,
		Red,
	] {$f(n)=\sin(1/x(n))$}
	
	\addplot
	[	smooth,
		DRed,
		opacity= 0.6,
		domain= 0:0.1,
		samples= 5*\mysampledensity,
	] {
		sin(deg(1/x))
	};
	
	\addplot
	[
		Red,
		mark= o,
		domain= 0:0.1,
	] coordinates {
		(0.0795774715459477,0)
		(0.0636619772367581,0)
		(0.0530516476972984,0)
		(0.0454728408833987,0)
		(0.0397887357729738,0)
		(0.0353677651315323,0)
		(0.0318309886183791,0)
		(0.0289372623803446,0)
		(0.0265258238486492,0)
		(0.0244853758602916,0)
		(0.0227364204416993,0)
		(0.0212206590789194,0)
		(0.0198943678864869,0)
		(0.0187241109519877,0)
		(0.0176838825657661,0)
		(0.01675315190441,0)
		(0.0159154943091895,0)
		(0.0151576136277996,0)
	};
	
\end{axis}
\end{tikzpicture}


\end{minipage}

\vspace{5mm}

\noindent%
\begin{minipage}{\linewidth}

\section{}

\BMold{
	h(x) = x\,\sin(1/x)\quad x\in\mathbb{R}\backslash\{0\}
}\relax

\begin{tikzpicture}
\begin{axis}
	
	\addlegendimage{empty legend}
	\addlegendentry%
	[
		font=\boldmath\footnotesize,
		Red,
	] {$h(n)=x\,\sin(1/x)$}
	
	\addplot
	[	smooth,
		Red,
		domain= -0.1:0.1,
		samples= 10*\mysampledensity,
	] {
		x*sin(deg(1/x))
	};
	
\end{axis}
\end{tikzpicture}

\begin{flalign*}
&
	h(x)
=	x\,\sin(1/x)
=	\frac{\sin(1/x)}{1/x}
&
\end{flalign*}

\end{minipage}

\end{multicols}
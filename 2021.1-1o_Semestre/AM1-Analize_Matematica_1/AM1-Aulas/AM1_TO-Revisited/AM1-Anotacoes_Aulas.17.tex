\part{26/04 - Funções Inversas}

%\BMold{
%	f:\mathbf{I}\to\mathbf{J}
%\land
%	\mathbf{I}, \mathbf{J} \subset \mathbb{R}
%\land
%	f \text{ é injetiva e subjetiva (bijetiva) }
%}\relax

\begin{BM}[align*][\Large]
&
	f:\mathbf{I}\to\mathbf{J}
\land
	\mathbf{I}, \mathbf{J} \subset \mathbb{R}
\land
	f \text{ é injetiva e subjetiva (bijetiva) }
&
\end{BM}\relax

\begin{BM}[align*][\Large]
&
	f^{-1}: \mathbf{J} \to \mathbf{I}
	\quad 
	\begin{cases}
		f^{-1}\circ f(x) = x&\quad\forall\,x\in\mathbf{I}
	\\	f\circ f^{-1}(y) = y&\quad\forall\,y\in\mathbf{J}
	\end{cases}
&
\end{BM}
\label{função inversa}

\vspace{5mm}

\begin{multicols}{2}

\noindent%
\begin{minipage}{\linewidth}

\section{Exemplo}

\begin{BM}
	f\phantom{^{-1}}&:[0,+\infty] \to [0,+\infty]\quad f(x)\phantom{^{-1}} = x^2
\\
	f^{-1}&: [0,+\infty] \to [0,+\infty]\quad f^{-1}(x) = \sqrt{x}
\end{BM}

\end{minipage}

\vspace{5mm}

\noindent%
\begin{minipage}{\linewidth}

\section{Exemplo}

\begin{BM}
	f:[0,1] \to \mathbb{R}\quad f(x)= 2x+1
\end{BM}\relax

\subsubsection{$f$ é injetiva?}
\begin{flalign*}
&
	\impliedby f \text{ é estritamente \hyperref[sucessao monotona]{monotona}}
&
\end{flalign*}\relax

\subsubsection{Contradomínio de $f$}
\begin{flalign*}
&
=	[1,3]
&
\end{flalign*}

\subsubsection{$f^{-1}$}
\begin{flalign*}
&
	f^{-1}: [1,3] \to [0,1]\quad f^{-1}(y) = \frac{y-1}{2}
&
\end{flalign*}

\end{minipage}

\vspace{5mm}

\noindent%
\begin{minipage}{\linewidth}

\section{Exemplo}

\begin{BM}
	\sin&: [-\pi/2,\pi/2]\to[-1,1]
\\	\arcsin&: [-1,1]\to[-\pi/2,\pi/2]
\end{BM}


\end{minipage}

\vspace{5mm}

\noindent%
\begin{minipage}{\linewidth}

\section{Exemplo}

\begin{BM}
	\tan&: [-\pi/2,\pi/2]\to[-\infty, \infty]
\\	\arctan&: [-\infty, \infty]\to[-\pi/2,\pi/2]
\end{BM}


\end{minipage}

\end{multicols}
\part{09/04 - Limites}

\BMold{
&
	\limsup u_{(n)} 
= 	\hyperref[supremo]{\text{Sup}}
	\{
	l: l\text{ é sublimite de } u
	\}
&\\&
	\liminf u_{(n)} 
= 	\hyperref[infimo]{\text{Inf}}
	\{
	l: l\text{ é sublimite de } u
	\}
&
}

\begin{flalign*}
&
	u_{(n)}\to l
\iff
	\limsup u_{(n)} = \liminf u_{(n)} = l
&
\end{flalign*}


\section{Lema geral das funções enquadradas}

\begin{align*}
&
	\{ u_{(n)}, v_{(n)}, w_{(n)}\}: \mathbb{N}\to\mathbb{R}
:	w_{(n)}\leq u_{(n)}\leq v_{(n)}
\implies	&\\&
\implies
	\liminf w_{(n)}
\leq 
	\liminf u_{(n)}
\leq	\limsup u_{(n)}
\leq 
	\limsup v_{(n)}
&
\end{align*}


\begin{multicols}{2}


\subsection{Exemplo}
\begin{align*}
&
	u_{(n)}= \left( 1+\frac{1+(-1)^n}{n} \right)^n
&
\end{align*}

\begin{flalign*}
&
	1^n=1
\leq	u_{(n)}
\leq	\left( 1+\frac{2}{n} \right)^n
\to e^2
&
\end{flalign*}

\vfill

\section{Lema}
``Se $u_{(n)}\to l$  então a sucessão \\$s_{(n)}=\frac{\sum_{i=0}^{n}u_{(n)}}{n}\to l$''

\end{multicols}
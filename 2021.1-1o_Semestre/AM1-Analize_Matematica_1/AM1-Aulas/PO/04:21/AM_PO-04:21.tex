\documentclass[12pt]{article}

% Linguagem
\usepackage[portuguese]{babel}

% Clickable Table of contents
\usepackage{hyperref}
\hypersetup{
	hidelinks=true,
	colorlinks=true,
	linkcolor=DarkGreen!20!LightGreen!25!
}

% Table of contents
\usepackage{tocloft}
\setlength{\cftsecnumwidth}{5em}
\setlength{\cftsubsecnumwidth}{4em} % Fix subsection width
% Fix space between subsection items on toc
%\renewcommand\cftsecafterpnum{\vskip0.05em}

% Multicols
\usepackage{multicol}

% Customize Chapter
%\usepackage{titlesec}
%\titleformat{\chapter}[hang]{\Huge\bfseries\color{DarkGreen!75!}}{\thechapter\hspace{20pt}{$|$}\hspace{20pt}}{0pt}{\Huge\bfseries}

% Appendix
%\usepackage{appendix}

% Maths
\usepackage{amssymb} 		
\usepackage{amsmath} 
\usepackage[utf8]{inputenc} %useful to type directly diacritic characters

\newcommand{\bm}[1]{{\boldmath{\large{\begin{align*} #1 \end{align*}}}}}

% Vectors
%\usepackage{esvect} 	% Vector over-arrow
%\renewcommand{\vec}{\vv} % Vecto over-arrow

% Tikz
%\usepackage{tikz}		
%\usepackage{pgfmath}  	% calculations
%\usepackage{varwidth}  % List inside TikzPicture

% Chem
%\usepackage{chemformula} 	% formulas quimicas
%\usepackage{chemfig} 		% Estruturas quimicas

%\newcommand{\mol}[1]{ \text{mol}_{\ch{ #1 }} } % mol

% Tabular
%\usepackage{multirow}
%\usepackage{siunitx} % Column S: align on decimal


% Colors
\usepackage{xcolor}

\definecolor{DarkBlue}	{HTML}{252A36}
\definecolor{LightGreen}{HTML}{7CCC6C}
\definecolor{DarkGreen}	{HTML}{008675}

\colorlet{White}{DarkGreen!20!}
\colorlet{Black}{DarkBlue!110!}

\pagecolor{Black}
\color{White}

%\definecolor{Red}  {HTML}{FF7E79}
%\definecolor{Blue} {HTML}{6666FF}
%\definecolor{Green}{HTML}{66FF66}

% Counters
\counterwithin*{section}{part} % Reset section on part

% Section and Subsection Customization
\renewcommand\thesection{Questão \arabic{section} }
\renewcommand\thesubsection{\arabic{section} - \alph{subsection}) }

\begin{document}

\title{\bfseries\color{DarkGreen!75!}%
	AM 1 - Ficha 5 Resolução\\
	Limites e Continuidade de funções%
}
\author{Felipe Pinto - 61387}
\date{21/04 - 2021.1}

\maketitle
\tableofcontents
\break

% Questões
{\color{DarkGreen!75}\part{Questões}}

\renewcommand\thesection{Questão \arabic{section}}
\renewcommand\thesubsection{%
	Q\arabic{section} - \alph{subsection})%
}
\renewcommand\thesubsubsection{(\,\roman{subsubsection}\,)}

% 
\setcounter{section}{3}

% Q4
\section{}
\bm{
	H_{(x)} =
	\left\{
	\begin{array}{ll}
	%
		0 \quad & x <    0 \\
		1 \quad & x \geq 0
	%
	\end{array}
	\right.
}

% Q4 - a)
\subsection{}
\begin{flalign*}
&
\iff
	\lim_{x\to0^-} H_{(x)} = 0 
	\neq 
	\lim_{x\to0^+} H_{(x)} = 1
&
\end{flalign*}

% Q4 - b)
\subsection{Incompleta}

\begin{multicols}{2}

% Q4 - b) (i)
\subsubsection{$ H_{(x-1)} $}
\begin{flalign*}
&
	y\in\mathbb{R}: 
	\lim_{x\to y^-} H_{(x-1)}
	\neq &\\& \neq
	\lim_{x\to y^+} H_{(x-1)}
\iff
	y-1 = 0
\iff &\\& \iff
	y=1
&
\end{flalign*}

% Q4 - b) (ii)
\subsubsection{$ (H_{(x)} - H_{(x-1)})\,x $}
\begin{flalign*}
&
	y\in\mathbb{R}: 
	\lim_{x\to y^+} (H_{(x)} - H_{(x-1)})\,x
	\neq &\\& \neq
	\lim_{x\to y^-} (H_{(x)} - H_{(x-1)})\,x
\implies &\\& \implies
	y - 1 = 0
\implies
	y = 1
&
\end{flalign*}

\end{multicols}

\setcounter{section}{6}

% Q7
\section{}

% Q7 - a)
\subsection{$ 
	f_{(x)} = \sin(x^2)/x,
	\quad x\in\mathbb{R}\backslash\{0\}
$}
\begin{flalign*}
&
	\lim_{x\to 0^-} f_{(x)} = 0 
	= \lim_{x\to 0^+} f_{(x)}= 0
\quad \therefore
	\bar f_{(x)} =
	\left\{
	\begin{array}{ll}
	%
		\sin(x^2)/x \quad & x\neq 0
		\\
		0 \quad & x=0
	%
	\end{array}
	\right.
&
\end{flalign*}

% Q7 - b)
\subsection{$
	g_{(x)} = e^{-1/(1-x^2)},
	\quad x\in\,(-1,1)
$}
\begin{flalign*}
&
	\lim_{x\to -1^+} g_{(x)}
=
	e^{\lim_{x\to-1^+}\left(\frac{-1}{1-x^2}\right)}
=
	e^{-\infty} = 0
; &\\&
	\lim_{x\to 1^-} g_{(x)}
=
	e^{\lim_{x\to1^-}\left(\frac{-1}{1-x^2}\right)}
=
	e^{-\infty} = 0
&\\& \therefore
	\bar g_{(x)} =
	\left\{
	\begin{array}{ll}
	%
		e^{-1/(1-x^2)} \quad& x\in\,(-1,1)
		\\
		0	\quad& x=\{-1,1\}
	%
	\end{array}
	\right.
&
\end{flalign*}


% Q7 - c)
\subsection{$ 
	h_{(x)} = e^{\tan(x)},
	\quad x\in\,(-\pi/2,\pi/2)
$}
\begin{flalign*}
&
	\lim_{x\to(-\pi/2)^+} h_{(x)}
=
	e^{\lim_{x\to(-\pi/2)^+}(\tan(x))}
=
	e^{-\infty} = 0
; &\\&
	\lim_{x\to(\pi/2)^-} h_{(x)}
=
	e^{\lim_{x\to(\pi/2)^-}(\tan(x))}
=
	e^{\infty} = \infty
&\\& \therefore
	\nexists\,\bar h_{(x)} \text{ pois função não é prolongavel por continuidade em }\pi/2
&
\end{flalign*}

% Questões Extras
{\color{DarkGreen!75}\part{Extras}}

\renewcommand\thesection{Extra \arabic{section}}
\renewcommand\thesubsection{%
	E\arabic{section} - \alph{subsection})%
}

\section{$ \lim_{x\to0}x/\tan(x) $}
\begin{flalign*}
&
=	\lim_{x\to0} \frac{x}{\sin(x)} \,\lim_{x\to0}\cos(x) = 1
&
\end{flalign*}

\section{$ \lim_{x\to0}\sin(x)/\sqrt{x^2} $}
\begin{flalign*}
&
=
	\lim_{x\to0} = \sin(x)/|x|
;\
	\lim_{x\to0^+}\sin(x)/x = 1
\neq	
	\lim_{x\to0^+}\sin(x)/(-x) = -1
&\\& \therefore
	\nexists\,\lim_{x\to0}\sin(x)/\sqrt{x^2}
&
\end{flalign*}

\section{$\lim_{x\to0} \sin^2(x)/\sqrt{x^4}$}
\begin{flalign*}
&
=	\lim_{x\to0} \left( \frac{\sin(x)}{x} \right)^2 = 1
&
\end{flalign*}

\end{document}











\documentclass[12pt]{article}

% Linguagem
\usepackage[portuguese]{babel}

% Clickable Table of contents
\usepackage{hyperref}
\hypersetup{
	hidelinks=true,
	colorlinks=true,
	linkcolor=DarkGreen!20!LightGreen!25!
}

% Table of contents
%\usepackage{tocloft}
%\setlength{\cftsubsecnumwidth}{3em} % Fix subsection width
% Fix space between subsection items on toc
%\renewcommand\cftsubsecafterpnum{\vskip5pt}

% Multicols
\usepackage{multicol}

% Customize Chapter
\usepackage{titlesec}

\titleformat{\chapter}[hang]{\Huge\bfseries\color{DarkGreen!75!}}{\thechapter\hspace{20pt}{$|$}\hspace{20pt}}{0pt}{\Huge\bfseries}

% Appendix
%\usepackage{appendix}

% Maths
\usepackage{amssymb} 		
\usepackage{amsmath} 
\usepackage[utf8]{inputenc} %useful to type directly diacritic characters

\newcommand{\bm}[1]{{\boldmath{\large{\begin{align*} #1 \end{align*}}}}}

% Vectors
%\usepackage{esvect} 	% Vector over-arrow
%\usepackage{tikz}		% Vector diagrams
%\usetikzlibrary{calc}  % Vector calculations
%\usepackage{varwidth}  % List inside TikzPicture

%\renewcommand{\vec}{\vv} % Vecto over-arrow

% Chem
%\usepackage{chemformula} 	% formulas quimicas
%\usepackage{chemfig} 		% Estruturas quimicas

% Colors
\usepackage{xcolor}

\definecolor{DarkBlue}	{HTML}{252A36}
\definecolor{LightGreen}{HTML}{7CCC6C}
\definecolor{DarkGreen}	{HTML}{008675}

\pagecolor{DarkBlue!110!}
\color{DarkGreen!20!}


%\definecolor{Red}  {hsb}{0  ,.6,1}
%\definecolor{Blue} {hsb}{0.6,.6,1}
%\definecolor{Green}{hsb}{0.3,.6,1}

% Counters
%\counterwithin*{section}{part} % Reset section on part

% Resolução de listas
\renewcommand\thesection{Questão \arabic{section} }
\renewcommand\thesubsection{\arabic{section} - \alph{subsection}) }

\begin{document}

\title{\color{DarkGreen!75!}\bfseries{AM 1 - PO\\Resolução Lista 4}}
\author{Felipe Pinto - 61387}
\date{12/04 2021.1}

\maketitle
\tableofcontents
\break

\setcounter{section}{3}

% Q4
\section{$
	u_1 = 1;\quad u_{n+1}=\sqrt{2\,u_n}
$}

% Q4 - a)
\subsection{$ u_n\in[1,3]\quad\forall\,n\in\mathbb{N} $}
\begin{flalign*}
&
	\iff 
		1 \leq u_n \leq 3
	\iff
		2 \leq 2\,u_n \leq 6
	\iff
		1 \leq \sqrt{2} 
	\leq 
		\sqrt{2\,u_n} 
	\leq 
		\sqrt{6} \leq 3
&
\end{flalign*}

% Q4 - b)
\subsection{$
		| u_{n+2}-u_{n+1} | 
	\leq 
		\frac{\sqrt{2}}{2}|u_{n+1}-u_n|
$}
\begin{flalign*}
&
	\iff 
		\left| \sqrt{2\,u_{n+1}}-\sqrt{2\,u_n} \right|
	=
		\left|
			\frac{2\,u_{n+1}-2\,u_n        }
				  {\sqrt{2\,u_{n+1}}+\sqrt{2\,u_n}}
		\right|
	;
		u_n \geq 1\quad\forall\,n\in\mathbb{N}
	\implies &\\& \implies
		\frac{2}{\sqrt{2\,u_{n+1}}+\sqrt{2\,u_n}}
		\left| u_{n+1}-u_n \right|
	\leq 
		\frac{2}{2\sqrt{2}}
		\left| u_{n+1}-u_n \right|
	=
		\frac{\sqrt{2}}{2}
		\left| u_{n+1}-u_n \right|
&
\end{flalign*}

% Extra 1
\section{$u_1 = 0;\quad u_{n+1}=1.5\,u_n+1$}

\subsection{Prove que $u_n$ é convergente}
\begin{flalign*}
&
	\iff 
		1
	\leq
		\frac{\left| u_{n+2}-u_{n+1} \right|}
			  {\left| u_{n+1}-u_n     \right|}
	=
		\frac{\left| 1.5\,u_{n+1}+1-1.5\,u_{n}+1 \right|}
			  {\left| u_{n+1}-u_n     \right|}
	=
		1.5
		\frac{\left| \,u_{n+1}-\,u_{n}+4/3 \right|}
			  {\left| u_{n+1}-u_n     \right|}
&
\end{flalign*}

\end{document}










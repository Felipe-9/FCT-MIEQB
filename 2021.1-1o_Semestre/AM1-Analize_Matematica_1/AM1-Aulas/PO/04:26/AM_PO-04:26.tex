\documentclass[12pt]{article}

% Linguagem
\usepackage[portuguese]{babel}

% Clickable Table of contents
\usepackage{hyperref}
\hypersetup{
	hidelinks=true,
	colorlinks=true,
	linkcolor=DarkGreen!20!LightGreen!25!
}

% Table of contents
\usepackage{tocloft}
\setlength{\cftsecnumwidth}{5em}
\setlength{\cftsubsecnumwidth}{4em} % Fix subsection width
% Fix space between subsection items on toc
%\renewcommand\cftsubsecafterpnum{\vskip5pt}

% Multicols
\usepackage{multicol}

% Customize Chapter
%\usepackage{titlesec}
%\titleformat{\chapter}[hang]{\Huge\bfseries\color{DarkGreen!75!}}{\thechapter\hspace{20pt}{$|$}\hspace{20pt}}{0pt}{\Huge\bfseries}

% Appendix
%\usepackage{appendix}

% Maths
\usepackage{amssymb} 		
\usepackage{amsmath} 
\usepackage[utf8]{inputenc} % useful to type directly diacritic characters

\newcommand{\bm}[1]{%
	{\boldmath{\large{\begin{align*} #1 \end{align*}}}}%
}

% Vectors
%\usepackage{esvect} 	% Vector over-arrow
%\renewcommand{\vec}{\vv} % Vecto over-arrow

% Tikz
%\usepackage{tikz}		
%\usepackage{pgfmath}  	% calculations
%\usepackage{varwidth}  % List inside TikzPicture

% Chem
%\usepackage{chemformula} 	% formulas quimicas
%\usepackage{chemfig} 		% Estruturas quimicas

%\newcommand{\mol}[1]{ \text{mol}_{\ch{ #1 }} } % mol

% Tabular
%\usepackage{multirow}
%\usepackage{siunitx} % Column S: align on decimal


% Colors
\usepackage{xcolor}

\definecolor{DarkBlue}	{HTML}{252A36}
\definecolor{LightGreen}{HTML}{7CCC6C}
\definecolor{DarkGreen}	{HTML}{008675}

\colorlet{White}{DarkGreen!20!}
\colorlet{Black}{DarkBlue!110!}

\pagecolor{Black}
\color{White}

%\definecolor{Red}  {HTML}{FF7E79}
%\definecolor{Blue} {HTML}{6666FF}
%\definecolor{Green}{HTML}{66FF66}

% Counters
\counterwithin*{section}{part} % Reset section on part

\begin{document}

\title{\bfseries\color{DarkGreen!75!}%
	AM 1 - Ficha 6 Resolução\\
	Teoremas da Continuidade e Função Inversa%
}
\author{Felipe Pinto - 61387}
\date{26/04 - 2021.1}

\maketitle
\tableofcontents
\break

% Questões
{\color{DarkGreen!75}\part{Questões}}

\renewcommand\thesection{Questão \arabic{section}}
\renewcommand\thesubsection{%
	Q\arabic{section} - \alph{subsection})%
}
\renewcommand\thesubsubsection{(\,\roman{subsubsection}\,)}

% Q1
\section{}

\setcounter{subsection}{2}

% Q1 - c)
\subsection{$ 
	\cos(1/x)\,x/(x+1)
	= \sin(x)/x
	\quad x\in(0,\infty) 
$ Refazer}
\begin{flalign*}
&
\iff
	f_{(x)} = \frac{x}{x+1}\,\cos(1/x)-\frac{\sin(x)}{x}
\implies &\\& \implies
	\lim_{x\to0^+} f_{(x)} = 0-1 = -1<0
; &\\&
	\lim_{x\to\infty} f_{(x)}
	= 1-0 = 1>0
&\\& \cdots
&
\end{flalign*}

\setcounter{section}{2}

% Q3
\section{}

% Q3 - a)
\subsection{}
Não, como o ontradomionio é $\mathbb{R}\backslash\{0\}$ para quaisquer $a<0<b$ existem $\{a', b\}\in\mathbb{R}:$
\begin{align*}
	f_{(a')} = a 
	\quad
	\text{e}
	\quad 
	f_{(b')} = b
\end{align*}
Portanto $f_{(a')} < 0 < f_{(b')}$
\\
Não existe pois ela possui termos maiores e menores que zero, e como é continua zero deve ser incluso.

% Q3 - b)
\subsection{}%{$\exists f_{(x)}\in\mathbb{R}: f_{(x)}\in[0,1]$}

% Q3 - c)
\subsection{$ \exists\,f_{(x)}\in\mathbb{R}: f_{(x)} \in\mathbb{I}\quad\forall\,x\in\mathbb{Q} $}
sim:
\begin{flalign*}
&
	c_{(x)} = x + \sqrt{2}
&\\& \cdots
&
\end{flalign*}

% Q3 - d)
\subsection{$ f_{(x)}\in\mathbb{R}: \cdots $}
sim,
\begin{flalign*}
&
	d_{(x)} = 
	\left\{
	\begin{array}{l l}
 	%
		1 \quad& x\in\mathbb{Q}\\
		0 \quad& x\in\mathbb{R}\backslash\mathbb{Q}
	%
	\end{array}
	\right.
&
\end{flalign*}

\setcounter{section}{6}

% Q7
\section{}

% Q7 - a)
\subsection{$ 
	h_{(x)} = \ln(\sqrt{x-1} + 1)\quad
	: [1,\infty)\mapsto\mathbb{R}
$}

\subsubsection{$ h_{(x)} \text{ é injetiva} $}
\begin{flalign*}
&
\iff
	x_1 = x_2
	\quad\forall\,\{x_1,x_2\}\in[1,\infty)
	: h_{(x_1)} = h_{(x_2)}
\implies &\\& \implies
	\ln(\sqrt{x_1-1}+1)
=
	\ln(\sqrt{x_2-1}+1)
\implies
	|x_1-1| = |x_2-1|
; &\\&
	\{ x_1,x_2 \} \in[1,\infty)
\implies
	x_1 = x_2
&
\end{flalign*}

\subsubsection{$I = \text{CD}_{h_{(x)}}$}

\subsubsection{$ h^{-1} $}
\begin{flalign*}
&
\iff
	x = \ln\left(\sqrt{h^{-1}_{(x)} - 1} + 1\right)
\implies
	h^{-1}_{(x)} = (e^{x}-1)^2-1
&\\& \therefore
	h^{-1}_{(x)} = (e^{x}-1)^2-1
	: [0,\infty) \mapsto [1,\infty)
&
\end{flalign*}


\end{document}











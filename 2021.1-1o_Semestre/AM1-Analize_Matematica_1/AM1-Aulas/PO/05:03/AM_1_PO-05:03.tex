\documentclass[12pt]{article}

% Geometry
\usepackage{geometry}
\geometry{
	a4paper, % 210mm por 297mm
	top=15mm,
	left=15mm,
	right=15mm
}

% Linguagem
\usepackage[portuguese]{babel}	% Babel
%\usepackage{polyglossia}		% Polyglossia
%\setdefaultlanguage[variant=brazilian]{portuguese}

% Graphics
%\usepackage{graphics}

% calc
\usepackage{calc}

% Clickable Table of contents
\usepackage{hyperref}
\hypersetup{
	hidelinks=true,
	colorlinks=true,
	linkcolor=DarkGreen!20!LightGreen!25!
}

% Table of contents
\usepackage{tocloft}
\setlength{\cftsecnumwidth}{25mm}		% Fix section width
%\setlength{\cftsubsecnumwidth}{15mm}
%% Fix space between subsection items on toc
%\renewcommand\cftsubsecafterpnum{\vskip5pt}

% Multicols
\usepackage{multicol}

% Customize Chapter
%\usepackage{titlesec}
%\titleformat{\chapter}[hang]{\Huge\bfseries\color{DarkGreen!75!}}{\thechapter\hspace{20pt}{$|$}\hspace{20pt}}{0pt}{\Huge\bfseries}

% Appendix
%\usepackage{appendix}

% siunix: SI units
%\usepackage{siunitx}
%\sisetup{
%	scientific-notation = engineering,
%	exponent-to-prefix = true,
%	exponent-product = *,
%	round-mode = figures,
%%	round-precision = 3,
%%	round-minimum = 0.01
%}

% Maths
\usepackage{amssymb} 		
\usepackage{amsmath} 

\newcommand{\bm}[1]{{\boldmath{\large{\begin{align*} #1 \end{align*}}}}}

% Vectors
%\usepackage{esvect} 	% Vector over-arrow
%\renewcommand{\vec}{\vv} % Vecto over-arrow

% Tikz
%\usepackage{tikz}		
%\usepackage{pgfmath}  	% calculations
%\usepackage{varwidth}  % List inside TikzPicture

% Chem
%\usepackage{chemformula} 	% formulas quimicas
%\usepackage{chemfig} 		% Estruturas quimicas

%\newcommand{\mol}[1]{ \text{mol}_{\ch{ #1 }} } % mol

% Tabular
%\usepackage{multirow}

% Colors
\usepackage{xcolor}

\definecolor{DarkBlue}	{HTML}{252A36}
\definecolor{LightGreen}{HTML}{7CCC6C}
\definecolor{DarkGreen}	{HTML}{008675}

\colorlet{White}{DarkGreen!20!}
\colorlet{Black}{DarkBlue!110!}

\pagecolor{Black}
\color{White}

%\definecolor{Red}  {HTML}{FF7E79}
%\definecolor{Blue} {HTML}{6666FF}
%\definecolor{Green}{HTML}{66FF66}

% Counters
%\counterwithin*{section}{part} % Reset section on part

% Section and Subsection Customization
\renewcommand\thesection{Questão \arabic{section}}
\renewcommand\thesubsection{%
	\arabic{section} - \alph{subsection})%
}
\renewcommand\thesubsubsection{(\roman{subsubsection})}

\begin{document}

\title{\bfseries\color{DarkGreen!75!}%
	AM 1 - Ficha 7\\Diferenciabilidade%
}
\author{Felipe Pinto - 61387}
\date{}

\newgeometry{left=25mm, right=25mm}

\maketitle

% Remove contents title inside of \tableofcontents 
\section*{Conteúdo}
\renewcommand{\contentsname}{}

\begin{multicols}{2} \tableofcontents \end{multicols}

\restoregeometry



% Q1
\section{}


% Q1 - a)
\subsection{}
\bm{
	f_{1\,(x)} = 
	\left\{
	\begin{array}{ll}
 	%
		0\quad&\text{se } x<0\\
		x^2\quad&\text{se } x\geq0
	%
	\end{array}
	\right.
}

\begin{flalign*}
&
	f'_{1\,(x)}
=	\lim_{x\to0^-} \frac{f'_{1\,(x)} - f_{1\,(0)}}{x-0}
=	\lim_{x\to0^-}\frac{0-0}{x}
=	0
=	\lim_{x\to0^+} \frac{f'_{1\,(x)} - f_{1\,(0)}}{x-0}
=	\lim_{x\to0^+} \frac{x^2-0}{x}
=	0
&\\&
\therefore f'_{1\,(x)} \text{ é diferenciavel em 0}
&
\end{flalign*}


% Q1 - b)
\subsection{}
\bm{
	f_{2\,(x)} = 
	\left\{
	\begin{array}{ll}
 	%
		\sin(2\,x)\quad&\text{se } x<0\\
		e^{2\,x}-1\quad&\text{se } x\geq0
	%
	\end{array}
	\right.
}

\begin{flalign*}
&
	f'_{2\,(x)}
=	\lim_{x\to0^-}\frac{f'_{2\,(x)} - f_{2\,(0)}}{x-0}
=	\lim_{x\to0^-}\frac{\sin(2\,x)-0}{x}
=	2
=	\lim_{x\to0^+} \frac{f'_{2\,(x)} - f_{2\,(0)}}{x-0}
=	\lim_{x\to0^+} \frac{e^{2\,x}-1-0}{x}
=	2
&\\&
\therefore f'_{2\,(x)} \text{ é diferenciavel em 0}
&
\end{flalign*}


% Q1 - c)
\subsection{}
\bm{
	f_{3\,(x)} = 
	\left\{
	\begin{array}{ll}
 	%
		|x|^{1.5}\sin(1/x)\quad&\text{se } x\neq0\\
		0\quad&\text{se } x=0
	%
	\end{array}
	\right.
}

\begin{flalign*}
&
	f'_{3\,(x)}
=	\lim_{x\to0^-}\frac{f'_{3\,(x)} - f_{3\,(0)}}{x-0}
=	\lim_{x\to0^-}\frac{|x|^{1.5}\,\sin(1/x)-0}{x}
=	\lim_{x\to0^-}-|x|^{0.5}\,\sin(1/x)
=	0
=	&\\&
=	\lim_{x\to0^+} \frac{f'_{3\,(x)} - f_{3\,(0)}}{x-0}
=	\lim_{x\to0^+} \frac{|x|^{1.5}\,\sin(1/x)-0}{x}
=	\lim_{x\to0^+} |x|^{0.5}\,\sin(1/x)
=	0
&\\&
\therefore f'_{3\,(x)} \text{ é diferenciavel em 0}
&
\end{flalign*}


% Q1 - d)
\subsection{}
\bm{
	f_{4\,(x)} = 
	\left\{
	\begin{array}{ll}
 	%
		e^{-1/x^2}\quad&\text{se } x\neq0\\
		0\quad&\text{se } x=0
	%
	\end{array}
	\right.
}

\begin{flalign*}
&
	f'_{4\,(x)}
=	\lim_{x\to0^-}\frac{f'_{4\,(x)} - f_{4\,(0)}}{x-0}
=	\lim_{x\to0^-}\frac{e^{-1/x^2}-0}{x}
=	\lim_{x\to0^-}\frac{1/x}{e^{-(1/x)^2}}
=	0
=	&\\&
=	\lim_{x\to0^+} \frac{f'_{4\,(x)} - f_{4\,(0)}}{x-0}
=	\lim_{x\to0^+} \frac{e^{-1/x^2}-0}{x}
=	\lim_{x\to0^-}\frac{1/x}{e^{-(1/x)^2}}
=	0
&\\&
\therefore f'_{4\,(x)} \text{ é diferenciavel em 0}
&
\end{flalign*}



% Q2
\section{}


% Q2 - a)
\subsection{}
\bm{
	g_{1\,(x)} = 
	\left\{
	\begin{array}{ll}
 	%
		0\quad&\text{se } x<0\\
		x\quad&\text{se } x\geq0
	%
	\end{array}
	\right.
}

\begin{flalign*}
&
	g'_{1\,(x)}
=	\lim_{x\to0^-}\frac{g'_{1\,(x)} - g_{1\,(0)}}{x-0}
=	\lim_{x\to0^-}\frac{0-0}{x}
=	0
=	\lim_{x\to0^+} \frac{g'_{1\,(x)} - g_{1\,(0)}}{x-0}
=	\lim_{x\to0^+} \frac{x-0}{x}
=	1
&\\&
\therefore g'_{1\,(x)} \text{ não é diferenciavel em 0}
&
\end{flalign*}


% Q2 - b)
\subsection{}
\bm{
	g_{2\,(x)} = 
	\left\{
	\begin{array}{ll}
 	%
		\sin(x)\quad&\text{se } x<0\\
		1-\cos(x)\quad&\text{se } x\geq0
	%
	\end{array}
	\right.
}

\begin{flalign*}
&
	g'_{2\,(x)}
=	\lim_{x\to0^-}\frac{g'_{2\,(x)} - g_{2\,(0)}}{x-0}
=	\lim_{x\to0^-}\frac{\sin(x)-1-\cos(0)}{x}
=	\lim_{x\to0^-}\frac{\sin(x)}{x}
=	1
=	&\\&
=	\lim_{x\to0^+} \frac{g'_{2\,(x)} - g_{2\,(0)}}{x-0}
=	\lim_{x\to0^+} \frac{1-\cos(x)-1-\cos(0)}{x}
=	\lim_{x\to0^+} \frac{1-\cos^2(x)}{x(1+\cos(x))}
=	&\\&
=	\lim_{x\to0^+} \frac{\sin^2(x)}{x^2}
*	\lim_{x\to0^+} \frac{x}{1+\cos(x)}
=	0
&\\&
\therefore g'_{2\,(x)} \text{ não é diferenciavel em 0}
&
\end{flalign*}


% Q2 - c)
\subsection{}
\bm{
	g_{3\,(x)} = \sqrt{\sin(x)\,x}\quad x\in[-\pi,\pi]
}

\begin{flalign*}
&
	g'_{3\,(x)}
=	\lim_{x\to0^-} \frac{g'_{3\,(x)} - g_{3\,(0)}}{x-0}
=	\lim_{x\to0^-}
	\frac{\sqrt{\sin(x)\,x}-\sqrt{\sin(0)\,0}}{x}
=	\lim_{x\to0^-} -\sqrt{\sin(x)\,x/x^2}
=	-1
=	&\\&
=	\lim_{x\to0^+} \frac{g'_{3\,(x)} - g_{3\,(0)}}{x-0}
=	\lim_{x\to0^+} 
	\frac{\sqrt{\sin(x)\,x}-\sqrt{\sin(0)\,0}}{x}
=	
	\lim_{x\to0^+} \sqrt{\sin(x)\,x/x^2}
=	1
&\\&
\therefore g'_{3\,(x)} \text{ não é diferenciavel em 0}
&
\end{flalign*}



\setcounter{section}{8}



% Q8
\section{}


\begin{multicols}{2}


% Q8 - a)
\subsection{}
\bm{
	f_{(a)} = g_{(a)} = 0 \quad e \quad g'_{(a)}\neq0
}

\begin{flalign*}
&
	\lim_{x\to a} \frac{f_{(x)}}{g_{(x)}}
=	\lim_{x\to a} 
	\frac{\cfrac{f_{(x)} - f_{(a)}}{x-a}}
		{\cfrac{g_{(x)} - g_{(a)}}{x-a}}
= 	\frac{f'_{(a)}}{g'_{(a)}}
&
\end{flalign*}

\subsection{}
\begin{flalign*}
&
	\lim_{x\to0} \frac{x^2+x}{x\,e^x+\sin(x)}
;\	&\\&
	f_{(0)} = 0^2+0 = 0
;\	&\\&
	g_{(0)} = 0*e^0+\sin(0) = 0
;\	&\\&
	g'_{(0)} = 0*e^0 + \cos(0) = 1 \neq 0
&\\& \therefore
	\lim_{x\to0} \frac{x^2+x}{x\,e^x+\sin(x)}
=	&\\&
=	\lim_{x\to0} \frac{2*x+1}{e^x+x^2\,e^x+\cos(x)}
=	1/2
&
\end{flalign*}



\end{multicols}

\end{document}











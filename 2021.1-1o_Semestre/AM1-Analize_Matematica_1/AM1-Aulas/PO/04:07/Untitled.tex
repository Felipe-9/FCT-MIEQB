\documentclass[12pt]{article}

% Linguagem
%\usepackage[portuguese]{babel}

% Clickable Table of contents
\usepackage[hidelinks]{hyperref}
\hypersetup{
	hidelinks=true,
	colorlinks=true,
	linkcolor=DarkGreen!20!LightGreen!25!
}

% Table of contents
\usepackage{tocloft}
\setlength{\cftsecnumwidth}{5.5em} 
\setlength{\cftsubsecnumwidth}{3em} % Fix subsection width
%Fix space between subsection items on toc
\renewcommand\cftsubsecafterpnum{\vskip5pt}

% Multicols
%\usepackage{multicol}

% Customize Chapter
\usepackage{titlesec}

\titleformat{\chapter}[hang]{\Huge\bfseries\color{DarkGreen!75!}}{\thechapter\hspace{20pt}{$|$}\hspace{20pt}}{0pt}{\Huge\bfseries}

% Appendix
%\usepackage{appendix}

% Maths
\usepackage{amssymb} 		
\usepackage{amsmath} 
\usepackage[utf8]{inputenc} %useful to type directly diacritic characters

\newcommand{\bm}[1]{{\boldmath{\large{\begin{align*} #1 \end{align*}}}}}

% Vectors
%\usepackage{esvect} 	% Vector over-arrow
%\usepackage{tikz}		% Vector diagrams
%\usetikzlibrary{calc}  % Vector calculations
%\usepackage{varwidth}  % List inside TikzPicture

%\renewcommand{\vec}{\vv} % Vecto over-arrow

% Chem
%\usepackage{chemformula} 	% formulas quimicas
%\usepackage{chemfig} 		% Estruturas quimicas

% Colors
\usepackage{xcolor}

\definecolor{DarkBlue}	{HTML}{252A36}
\definecolor{LightGreen}{HTML}{7CCC6C}
\definecolor{DarkGreen}	{HTML}{008675}

\pagecolor{DarkBlue!110!}
\color{DarkGreen!20!}


%\definecolor{Red}  {hsb}{0  ,.6,1}
%\definecolor{Blue} {hsb}{0.6,.6,1}
%\definecolor{Green}{hsb}{0.3,.6,1}

% Counters
\counterwithin*{section}{part} % Reset section on part

% Resolução de listas
\renewcommand\thesection{Questão \arabic{section} }
\renewcommand\thesubsection{%
	\arabic{section} - \alph{subsection})%
}

\begin{document}

\title{AM 1 - PO\\Resolução Ficha 3 e 4}
\author{Felipe Pinto - 61387}
\date{07/04 - 2021.1}

\maketitle
\tableofcontents
\break

\part{Ficha 3}

\setcounter{section}{7}
\setcounter{subsection}{3}
% d)
\subsection{$
	\lim_{n\to\infty}
	\sum_{k=1}^{n}\frac{n^2+k}{n^3+1}
	$ incompleto}
\begin{flalign*}
&
	\implies 
		\lim_{n\to\infty}
		n\frac{n^2+1}{n^3+1}
	\leq
		\lim_{n\to\infty}
		\sum_{k=1}^{n}\frac{n^2+k}{n^3+1}
	\leq
		\lim_{n\to\infty}
		n\frac{n^2+n}{n^3+1}
	\implies &\\& \implies
		\lim_{n\to\infty}
		n\frac{1/n+1/n^2}{1+1/n^2}
	\leq
		\lim_{n\to\infty}
		\sum_{k=1}^{n}\frac{n^2+k}{n^3+1}
	\leq
		\lim_{n\to\infty}
		n\frac{1/n^2+1/n^3}{1+1/n^4}
&
\end{flalign*}

% Q8
\section{}

% Q8 - a)
\subsection{$
	\lim_{n\to\infty} 
	\left( 1 + \frac{1}{(2\,n+(-1)^n)} \right)^n
$}
\begin{flalign*}
&
	\implies
		\lim_{n\to\infty} 
		\left( 1 + \frac{1}{(2\,n+1)} \right)^n
	\leq
		\lim_{n\to\infty} 
		\left( 1 + \frac{1}{(2\,n+(-1)^n)} \right)^n
	\leq &\\& \leq
		\lim_{n\to\infty} 
		\left( 1 + \frac{1}{(2\,n-1)} \right)^n
	\implies
		\lim_{n\to\infty}
		\left(
			\left( 1 + \frac{1}{(2\,n+1)} \right)^{2\,n+1}
		\right)^{\frac{n}{2\,n+1}}
	\leq &\\& \leq
		\lim_{n\to\infty} 
		\left( 1 + \frac{1}{(2\,n+(-1)^n)} \right)^n
	\leq
		\lim_{n\to\infty} 
		\left(
			\left( 1 + \frac{1}{(2\,n-1)} \right)^{2\,n-1}
		\right)^{\frac{n}{2\,n-1}}
	\implies &\\& \implies
		\lim_{n\to\infty} 
		\left( 1 + \frac{1}{(2\,n+(-1)^n)} \right)^n
	=
		\sqrt{e}
&
\end{flalign*}

% Q8 - b)
\subsection{$
	\lim_{n\to\infty}
	\left( 1+\frac{2+(-1)^n}{n} \right)^{\sqrt{n}}
$}
\begin{flalign*}
&
	\implies
		\lim_{n\to\infty}
		\left( 1+\frac{1}{n} \right)^{\sqrt{n}}
	\leq
		\lim_{n\to\infty}
		\left( 1+\frac{2+(-1)^n}{n} \right)^{\sqrt{n}}
	\leq
		\lim_{n\to\infty}
		\left( 1+\frac{3}{n} \right)^{\sqrt{n}}
	\implies &\\& \implies
		\lim_{n\to\infty}
		\left(
			\left( 1+\frac{1}{n} \right)^{n}
		\right)^{\sqrt{n}/n}
	\leq
		\lim_{n\to\infty}
		\left( 1+\frac{2+(-1)^n}{n} \right)^{\sqrt{n}}
	\leq &\\& \leq
		\lim_{n\to\infty}
		\left(
			\left( 1+\frac{3}{n} \right)^{n}
		\right)^{\sqrt{n}/n}
	\implies
		(e^3)^0
	\leq
		\lim_{n\to\infty}
		\left( 1+\frac{2+(-1)^n}{n} \right)^{\sqrt{n}}
	\leq
		(e^1)^0
	\implies &\\& \implies
		\lim_{n\to\infty}
		\left( 1+\frac{2+(-1)^n}{n} \right)^{\sqrt{n}}
	=	1
&
\end{flalign*}

\part{Ficha 4}

% Q1
\section{Determine os sublimites das seguintes sucessões limitadas indicando os seus limites superior e inferior}

% Q1 - a)
\subsection{$
	\cos(n\,\pi/4)
$}
\begin{flalign*}
&
	\cos(n\,\pi/4)\in\{ -1,-\sqrt{2}/2,0,\sqrt{2}/2,1 \}
	\ \forall\,n\in\mathbb{N};
&\\&
	\overline\lim\, a_n=1;\ \underline\lim\, a_n=-1
&
\end{flalign*}

\subsection{$
	\left( 1+(-1)^n/n \right)^n
$}
\begin{flalign*}
&
	\text{sublimites} = \{ e, e^{-1} \};
&\\&
	\overline\lim\, b_n=e;\ \underline\lim\, b_n=e^{-1}
&
\end{flalign*}

% Q1 - c)
\subsection{$
	n\,\sin\left( \frac{1+(-1)^n}{n} \right)
$ Incompleto}
\begin{flalign*}
&
	n\,\sin\left( \frac{1+(-1)^n}{n} \right)
	=
	\left\{\begin{array}{ll}
		n\,\sin\left( 2/n \right)
		\quad\forall\,n\text{ par}
		\\
		n\,\sin\left( 0/n \right)
		\quad\forall\,n\text{ impar}	
	\end{array}\right.;
	&\\&
		\lim_{n\to\infty} n\,\sin(0)=0;
	&\\&
		\lim_{n\to\infty} n\,\sin{2}
	=
		\lim_{n\to\infty}
		\frac{2/n}{2/n}2
	= 	2
	\implies
	sublim=\{ 0,2 \}
&\\&
	\cdots
&
\end{flalign*}

\subsection{$
	\arctan((-1)^n\,n)
$}
\begin{flalign*}
&
	\arctan((-1)^n\,n)
	=
		\left\{ \begin{array}{ll}
			\arctan(n)
			\quad\forall\,n\text{ par}
			\\
			\arctan(-n)
			\quad\forall\,n\text{ impar}
		\end{array} \right.
	\implies &\\& \implies
		\left\{ \begin{array}{ll}
			\lim_{n\to\infty}
			\arctan(n)=\pi/2
			\\
			\lim_{n\to\infty}
			\arctan(-n)=-\pi/2
		\end{array} \right.
	\implies &\\& \implies
		\text{sublim}(e_n)=\{ -\pi/2,\pi/2 \};
	\
		\overline\lim\, e_n=\pi/2;
	\ 
		\underline\lim\, e_n=-\pi/2
& 
\end{flalign*}





\end{document}














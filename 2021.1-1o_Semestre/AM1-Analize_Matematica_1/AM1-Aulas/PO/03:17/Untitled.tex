\documentclass[12pt]{report}

\usepackage{xcolor}
\pagecolor{black}
\color{white}

\usepackage{amssymb} %maths
\usepackage{amsmath} %maths
\usepackage[utf8]{inputenc} %useful to type directly diacritic characters

\begin{document}

\title{Ficha 2 - Método de indução.}
\author{}
\date{}

\maketitle

\paragraph{Exercício Extra: Prove: }
$$ 1+r+\dots + r^{n-1}=\frac{1-r^n}{1-r} $$
\begin{itemize}
\item $ n=1\implies 1=\frac{1-r}{1-r} \implies 1=1 $
\item $ n=m+1 \implies 1+r+\dots +r^{m-1}+r^m =\frac{1-r^{m+1}}{1-r}=\frac{1-r^{m-1}*r^2}{1-r}=\frac{}{} $ 
\end{itemize}

\paragraph{Questão 1}
\subparagraph{(a)} $ \sum_{k=1}^{n}{\frac{1}{2^k}} = 1 - \frac{1}{2^n}  $
\begin{itemize}

	\item $ n=1 \implies $	
	$$ \sum_{k=1}^{1}{\frac{1}{2^k}} = 1- \frac{1}{2^n} \implies \frac{1}{2}= 1-\frac{1}{2} $$

	\item $ n=m+1\implies $
	$$ \sum_{k=1}^{m+1}{\frac{1}{2^k}}= \sum_{k=1}^{m}{\frac{1}{2^k}}+\frac{1}{2^{m+1}} = 1-\frac{1}{2^{m}}+\frac{1}{2^m+1}=\cdots $$

\end{itemize}

\subparagraph{(b)} $ \sum_{k=1}^{n}\frac{1}{k(k+1)}=\frac{n}{n+1} $
\begin{itemize}
\item $ n=1 \implies $
$$ \sum_{k=1}^{n}\frac{1}{k(k+1)}=\frac{n}{n+1} $$ 
\end{itemize}




\end{document}
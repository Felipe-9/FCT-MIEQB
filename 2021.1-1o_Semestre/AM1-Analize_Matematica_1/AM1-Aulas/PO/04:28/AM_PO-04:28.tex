\documentclass[12pt]{article}

% Linguagem
\usepackage[portuguese]{babel}

% Clickable Table of contents
\usepackage{hyperref}
\hypersetup{
	hidelinks=true,
	colorlinks=true,
	linkcolor=DarkGreen!20!LightGreen!25!
}

% Table of contents
\usepackage{tocloft}
\setlength{\cftsecnumwidth}{5em}
\setlength{\cftsubsecnumwidth}{4em} % Fix subsection width
% Fix space between subsection items on toc
\renewcommand\cftsubsecafterpnum{\vskip5pt}

% Multicols
\usepackage{multicol}

% Customize Chapter
%\usepackage{titlesec}
%\titleformat{\chapter}[hang]{\Huge\bfseries\color{DarkGreen!75!}}{\thechapter\hspace{20pt}{$|$}\hspace{20pt}}{0pt}{\Huge\bfseries}

% Appendix
%\usepackage{appendix}

% Maths
\usepackage{amssymb} 		
\usepackage{amsmath} 
\usepackage[utf8]{inputenc} % useful to type directly diacritic characters

\newcommand{\bm}[1]{%
	{\boldmath{\large{\begin{align*} #1 \end{align*}}}}%
}

% Vectors
%\usepackage{esvect} 	% Vector over-arrow
%\renewcommand{\vec}{\vv} % Vecto over-arrow

% Tikz
%\usepackage{tikz}		
%\usepackage{pgfmath}  	% calculations
%\usepackage{varwidth}  % List inside TikzPicture

% Chem
%\usepackage{chemformula} 	% formulas quimicas
%\usepackage{chemfig} 		% Estruturas quimicas

%\newcommand{\mol}[1]{ \text{mol}_{\ch{ #1 }} } % mol

% Tabular
%\usepackage{multirow}
%\usepackage{siunitx} % Column S: align on decimal


% Colors
\usepackage{xcolor}

\definecolor{DarkBlue}	{HTML}{252A36}
\definecolor{LightGreen}{HTML}{7CCC6C}
\definecolor{DarkGreen}	{HTML}{008675}

\colorlet{White}{DarkGreen!20!}
\colorlet{Black}{DarkBlue!110!}

\pagecolor{Black}
\color{White}

%\definecolor{Red}  {HTML}{FF7E79}
%\definecolor{Blue} {HTML}{6666FF}
%\definecolor{Green}{HTML}{66FF66}

% Counters
\counterwithin*{section}{part} % Reset section on part

\begin{document}

\title{\bfseries\color{DarkGreen!75!}%
	AM 1 - Ficha 6 Resolução\\%
	Teoremas da Continuidade e Função Inversa%%
}
\author{Felipe Pinto - 61387}
\date{28/04 - 2021.1}

\maketitle
\tableofcontents
\break

% Questões
{\color{DarkGreen!75}\part{Questões}}

\renewcommand\thesection{Questão \arabic{section}}
\renewcommand\thesubsection{%
	Q\arabic{section} - \alph{subsection})%
}
\renewcommand\thesubsubsection{(\,\roman{subsubsection}\,)}

\setcounter{section}{5}

% Q6
\section{$ 
	f:\mathbb{R}\mapsto\mathbb{R}\\
	\quad \text{diferenciável em } x=1;\\
	\quad f_{(1)} = 2\text{ e } {f'}_{(1)} = 3
$}

\setcounter{subsection}{3}

% Q6 - d)
\subsection{$ {f'}_{(e^{2\,x})};\quad x=0 $}

% Q6 - e)
\subsection{$ {f^{-1}}_{(x)};\quad x=2 $}


% Q7
\section{}

% Q7 - a)
\subsection{$ a_{(x)} = \ln(x) + \cos(2\,x) + e^{3\,x} + \arctan(5\,x) $}
\begin{flalign*}
&
	a'_{(x)}
=
	\frac{1}{x} 
	- 2\,\sin(2\,x) 
	+ 3\,x\,e^{3\,x} 
	+ \frac{5}{1+25\,x^2}
&
\end{flalign*}

% Q7 - b)
\subsection{$ b_{(x)} = (x^5+x)\,\sin(x) $}
\begin{flalign*}
&
	b'_{(x)} = (5\,x^4+x)\,\sin(x) + (x^5+x)\,\cos(x)
&
\end{flalign*}

% Q7 - c)
\subsection{$ c_{(x)} = \ln(\cos(x)) $}
\begin{flalign*}
&
	c'_{(x)} = \cdots
&
\end{flalign*}

% Q7 - d)
\subsection{$ d_{(x)} = e^{-x}/(x^2+1) $}
\begin{flalign*}
&
	d'_{(x)} = \frac{-e^{-x}(x^2+1) - e^{-x}(2\,x)}
			    {(x^2+1)^2}
&
\end{flalign*}

% Q7 - e)
\subsection{$ e_{(x)} = \tan(x) + \cot(x) $}
\begin{flalign*}
&
	e'_{(x)} = \frac{1}{\cos^2(x)} - \frac{1}{\sin^2(x)}
&
\end{flalign*}

% Q7 - f)
\subsection{$ f_{(x)} = (x^2+1)\,\arctan(x) $}
\begin{flalign*}
&
	f'_{(x)} = (2\,x)\,\arctan(x) + (x^2+1)\,\frac{1}{1+x^2}
	= 2\,x\,\arctan(x) + 1
&
\end{flalign*}

\setcounter{subsection}{7}

% Q7 - h)
\subsection{$ h_{(x)} = e^{\ln^2(x)} $}
\begin{flalign*}
&
	h'_{(x)} = 2\,\ln(x)\,\frac{1}{x}\,e^{\ln^2(x)}
&
\end{flalign*}

% Q7 - i)
\subsection{$ i_{(x)} = \cos(\arcsin(x)) $}
\begin{flalign*}
&
	i'_{(x)} = -\sin(\arcsin(x))\,\frac{1}{\sqrt{1-x^2}}
	= \frac{-x}{\sqrt{1-x^2}}
&
\end{flalign*}

% Q7 - j)
\subsection{$ j_{(x)} = \sinh(x)/\cosh(x) $}
\begin{flalign*}
&
	j'_{(x)} = \frac{\cosh^2(x) - \sinh^2(x)}{\cosh^2(x)}
	= \cosh^{-2}(x)
&
\end{flalign*}

% Nota
\subsubsection{Nota:}
\begin{flalign*}
&
	\sinh(x) = \frac{e^{x} - e^{-x}}{2}
\qquad 
	(\sinh(x))' = x'\,\cosh(x)
&\\&
	\cosh(x) = \frac{e^{x} + e^{-x}}{2}
\qquad
	(\cosh(x))' = x'\,\sinh(x)
&
\end{flalign*}

% Q7 - k)
\subsection{$ k_{(x)} = \arctan(\ln(1+x^2)) $}
\begin{flalign*}
&
	k'_{(x)} = \frac{(2\,x)/(1+x^2)}
			    {1+\ln^2(1+x^2)}
&
\end{flalign*}

% Q7 - l)
\subsection{$ l_{(x)} = x/(1+e^{1/x}) $}
\begin{flalign*}
&
	l'_{(x)} 
= 
	\frac{1+e^{1/x} + x\,e^{1/x}/x^2}
	     {(1+e^{1/x})^2}
=
	\frac{1+e^{1/x} + e^{1/x}/x}
	     {(1+e^{1/x})^2}
&
\end{flalign*}

% Q7 - m)
\subsection{$ m_{(x)} = \ln(\ln(x)) $}
\begin{flalign*}
&
	m'_{(x)} 
= 
	\frac{1}{\ln(x)}\,\frac{1}{x}
&
\end{flalign*}


% Extras
{\color{DarkGreen!75}\part{Questões}}

\renewcommand\thesection{E\arabic{section}}
\renewcommand\thesubsection{%
	E\arabic{section} - \alph{subsection})%
}
\renewcommand\thesubsubsection{(\,\roman{subsubsection}\,)}

% E1
\section{$ f_{(x)} = e^{3\,x}/(e^x-2) $}
\begin{flalign*}
&
	f'_{(x)} 
= 
	\frac{ 3\,e^{3\,x}(e^x-2) - e^{3\,x}\,e^x }
	     { (e^x-2)^2 }
&
\end{flalign*}







\end{document}











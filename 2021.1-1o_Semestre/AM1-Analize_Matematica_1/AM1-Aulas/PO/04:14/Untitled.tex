\documentclass[12pt]{article}

% Linguagem
\usepackage[portuguese]{babel}

% Clickable Table of contents
\usepackage{hyperref}
\hypersetup{
	hidelinks=true,
	colorlinks=true,
	linkcolor=DarkGreen!20!LightGreen!25!
}

% Table of contents
%\usepackage{tocloft}
%\setlength{\cftsubsecnumwidth}{3em} % Fix subsection width
% Fix space between subsection items on toc
%\renewcommand\cftsubsecafterpnum{\vskip5pt}

% Multicols
%\usepackage{multicol}

% Customize Chapter
%\usepackage{titlesec}
%\titleformat{\chapter}[hang]{\Huge\bfseries\color{DarkGreen!75!}}{\thechapter\hspace{20pt}{$|$}\hspace{20pt}}{0pt}{\Huge\bfseries}

% Appendix
%\usepackage{appendix}

% Maths
\usepackage{amssymb} 		
\usepackage{amsmath} 
\usepackage[utf8]{inputenc} %useful to type directly diacritic characters

\newcommand{\bm}[1]{{\boldmath{\large{\begin{align*} #1 \end{align*}}}}}

% Vectors
%\usepackage{esvect} 	% Vector over-arrow
%\renewcommand{\vec}{\vv} % Vecto over-arrow

% Tikz
%\usepackage{tikz}		% Vector diagrams
%\usetikzlibrary{calc}  % Vector calculations
%\usepackage{varwidth}  % List inside TikzPicture

% Chem
%\usepackage{chemformula} 	% formulas quimicas
%\usepackage{chemfig} 		% Estruturas quimicas

% Colors
\usepackage{xcolor}

\definecolor{DarkBlue}	{HTML}{252A36}
\definecolor{LightGreen}{HTML}{7CCC6C}
\definecolor{DarkGreen}	{HTML}{008675}

\pagecolor{DarkBlue!110!}
\color{DarkGreen!20!}

%\definecolor{Red}  {hsb}{0  ,.6,1}
%\definecolor{Blue} {hsb}{0.6,.6,1}
%\definecolor{Green}{hsb}{0.3,.6,1}

% Counters
%\counterwithin*{section}{part} % Reset section on part

% Section and Subsection Customization
\renewcommand\thesection{Questão \arabic{section} }
\renewcommand\thesubsection{\arabic{section} - \alph{subsection}) }

\begin{document}

\title{\bfseries\color{DarkGreen!75!}%
	AM 1 - Resolução Ficha 5\\Limites e Continuidade de funções%
}
\author{Felipe Pinto - 61387}
\date{14/04 - 2021.1}

\maketitle
\tableofcontents
\break

% Q1
\section{}

\setcounter{subsection}{5}

% Q1 - f)
\subsection{$ \lim_{x\to\infty} e^{\cos(x)}/x $}
\begin{flalign*}
&
	=
		\lim_{x\to\infty} e^{\cos(x)}
		*\lim_{x\to\infty} 1/x
	=
		\lim_{x\to\infty} e^{\cos(x)}*0
	;\ 
		e^{\cos(x)} \text{ é uma função limitada}
	\implies &\\& \implies
		\lim_{x\to\infty} e^{\cos(x)}/x = 0
&
\end{flalign*}

% Q1 - g)
\subsection{$ \lim_{x\to 0^+} e^{1/x} $}
\begin{flalign*}
&
	= e^{\lim_{x\to 0^+} 1/x}
	= e^{+\infty}
	= +\infty
&
\end{flalign*}

% Q1 - h)
\subsection{$ \lim_{x\to 0^-} e^{1/x} $}
\begin{flalign*}
&
	= e^{\lim_{x\to 0^-} 1/x}
	= e^{-\infty}
	= 0
&
\end{flalign*}

% Q2
\section{Prove $\nexists\,f(x)$}

% Q2 - a)
\subsection{$ \nexists\,a(x) = x/|x| : x = 0 $}
\begin{flalign*}
&
	\iff
		\lim_{x\to0^+} a(x) = 1
	\neq
		\lim_{x\to0^-} a(x) = -1
&
\end{flalign*}

% Q2 - b)
\subsection{$ \nexists\,b(x) = \frac{x^2-1}{|x-1|} : x = 1 $}
\begin{flalign*}
&
	\iff
		\lim_{x\to1^+} b(x)
	=
		\lim_{x\to1^+} \frac{(x+1)(x-1)}{x-1}
	=
		2
	\neq
		\lim_{x\to1^-} b(x)
	= &\\& =
		\lim_{x\to1^-} \frac{(x+1)(x-1)}{1-x}
	=
		\lim_{x\to1^-} \frac{-(x+1)}{1}
	=
		-2
&
\end{flalign*}

% Q2 - c)
\subsection{$ \nexists\,c(x) = \arctan(e^{1/x}) : x = 0 $}
\begin{flalign*}
&
	\iff
		\lim_{x\to0^+} c(x)
	=
		\arctan(e^{\lim_{x\to0^+}1/x})
	=
		\arctan(e^{\infty})
	=
		\arctan(\infty)
	=
		\pi/2
	\neq &\\& \neq
		\lim_{x\to0^-} c(x)
	=
		\arctan(e^{\lim_{x\to0^-}1/x})
	=
		\arctan(e^{-\infty})
	=
		\arctan(0)
	=
		0
&
\end{flalign*}

% Q2 - d)
\subsection{$
	\nexists\,d(x) = e^{\cos(x)} : x = +\infty
$ Incompleta}
\begin{flalign*}
&
	\iff
		\lim_{x\to\infty} d(x)
	;\ x_n = 2\,n\,\pi\quad\forall\,n\in\mathbb{N}
	&\\&
	\cdots
&
\end{flalign*}

% Q3
\section{}

% Q3 - a)
\subsection{$ \lim_{x\to0} \tan(x)/x $ Incompleto}
\begin{flalign*}
&
	=
		\lim_{x\to0} \cos^-1(x)\frac{\sin(x)}{x}
	=
		\lim_{x\to0} \cdots
&
\end{flalign*}

% Q3 - b)
\subsection{$ \lim_{x\to0} (1-e^{2\,x})/x $ Incompleto}
\begin{flalign*}
&
	= \cdots
&
\end{flalign*}

% Q3 - c)
\subsection{$ \lim_{x\to0} (1-e^{2x})/\sin(3\,x) $ Incompleto}
\begin{flalign*}
&
	=
		\lim_{x\to0} 
		\frac{1-e^{2x}}{2\,x}
		\frac{3\,x}{\sin(3\,x)}
		\frac{2\,x}{3\,x}
	=
		\cdots
&
\end{flalign*}

% Q3 - d)
\subsection{$ \lim_{x\to0} \ln(x+1)/x $ Incompleto}
\begin{flalign*}
&
	x+1 = e^y
	\implies
		\lim_{x\to0} \ln(x+1)/x
	=
		\lim_{x\to0} \frac{y}{e^y-1}
	=
		\cdots
&
\end{flalign*}

% Q3 - e)
\subsection{$ \lim_{x\to0} (1-\cos(3\,x))/x^2 $}
\begin{flalign*}
&
	=
		\lim_{x\to0} \frac{1-\cos^2(3\,x)}{x^2(1+\cos(3\,x))}
	=
		\lim_{x\to0} \frac{\sin^2(3\,x)}{x^2(1+\cos(3\,x))}
	=
		\lim_{x\to0} 
		\frac{\sin^2(3\,x)}{(3\,x)^2}
		\frac{9\,x^2}{x^2(1+\cos(3\,x))}
	= &\\& =
		\lim_{x\to0} 
		\left(
		\frac{\sin(3\,x)}{3\,x}
		\right)^2
		\lim_{x\to0}
		\frac{9}{1+\cos(3\,x)}
	=
		9/2
&
\end{flalign*}







\end{document}













\documentclass[12pt]{article}

% Maths
\usepackage{amssymb} 		
\usepackage{amsmath} 
\usepackage[utf8]{inputenc} %useful to type directly diacritic characters

\newcommand{\bm}[1]{{\boldmath{\large{\begin{align*} #1 \end{align*}}}}}

% Vectors
%\usepackage{esvect} 	% Vector over-arrow
%\usepackage{tikz}		% Vector diagrams

%\renewcommand{\vec}{\vv} % Vecto over-arrow

% Chem
%\usepackage{chemformula} 	% formulas quimicas
%\usepackage{chemfig} 		% Estruturas quimicas

% Colors
\usepackage{xcolor}

\definecolor{DarkBlue}	{HTML}{252A36}
\definecolor{LightGreen}	{HTML}{7CCC6C}
\definecolor{DarkGreen}	{HTML}{008675}

\pagecolor{DarkBlue!110!}
\color{DarkGreen!20!}

\renewcommand{\thesection}{Exercício \arabic{section}}
\renewcommand{\thesubsection}{\arabic{section} - \alph{subsection})}

\begin{document}

\title{AM 1 - Resolução ficha 3}
\date{03/29}

\maketitle
\tableofcontents
\break

\section{}

\section{}

\section{}

\subsection{}

\subsection{}

\subsection{}

% Q3 - d)
\subsection{}
\begin{flalign*}
& 	 \lim_{n\to\infty}\frac{n\,\cos(n^2)}{n^2+1}
	=\lim_{n\to\infty}\frac{n^{-1}\cos(n^2)}{1+1/n^2}=0 &
\end{flalign*}

% Q3 - e)
\subsection{}
\begin{flalign*}
&	  \lim_{n\to\infty}\cfrac{2\,n\,e^{1/n}}{\sqrt{n^2+5  }}
	= \lim_{n\to\infty}\cfrac{2\,   e^{1/n}}{\sqrt{1+5/n^2}}
	= 2 &
\end{flalign*}

% Q3 - f)
\subsection{}
\begin{flalign*}
&   \lim_{n\to\infty}\frac{5^n+4^n}{2\,5^{n+1}+1}
	=\lim_{n\to\infty}\frac{1/5+(4/5)^n\,1/5}{2+1/5^{n+1}}=\frac{1}{10} &
\end{flalign*}

% Q3 - g)
\subsection{}
\begin{flalign*}
&	  \lim_{n\to\infty} \frac{3^n+e^n}{3^n+\pi^n}
	= \lim_{n\to\infty} \cfrac{\cfrac{3^n}{\pi^n}
										+\left(\cfrac{e}{\pi}\right)^n}
									  {\cfrac{3^n}{\pi^n}+1}
	= 0 &
\end{flalign*}

% Q3 - h
\subsection{}
\begin{flalign*}
&	  \lim_{n\to\infty}\frac{n^2\,9^n+n^3}{(n+2)^2\,3^{2n+1}}
	= \lim_{n\to\infty}\frac{n^2\,9^n+n^3}{(n+2)^2\,9^n\,3}
	= \lim_{n\to\infty}\frac{1+n/9^n}{\left( 1+2/n \right)^2\,3}
	= 1/3 &
\end{flalign*}

\section{}

% Q5
\section{}

% Q5 - a)
\subsection{}
\begin{flalign*}
&	  \lim_{n\to\infty} \sqrt{n+5}-\sqrt{2n-1} 
	= \lim_{n\to\infty} \frac{n+5-2n+1}{\sqrt{n+5}+\sqrt{2n-1}} =&\\
&	= \lim_{n\to\infty} \frac{-1+6/n}{\sqrt{1/n+5/n^2}+\sqrt{2/n-1/n^2}}
	= -\infty &
\end{flalign*}

% Q5 - b)
\subsection{}
\begin{flalign*}
&	  \lim_{n\to\infty} \sqrt{n^2+3\,n}-\sqrt{n^2-n+1}
	= \lim_{n\to\infty} \cfrac{n^2+3\,n-n^2+n-1}
									  {\sqrt{n^2+3\,n}+\sqrt{n^2-n+1}} =&\\
&	= \lim_{n\to\infty} \cfrac{4\,n-1}
									  {\sqrt{n^2+3\,n}+\sqrt{n^2-n+1}} 
	= \lim_{n\to\infty} \cfrac{4-1/n}
									  {\sqrt{1+3/n}+\sqrt{1-1/n+1/n^2}}
	= 2 &
\end{flalign*}

\subsection{}

% Q5 - d)
\subsection{}
\begin{flalign*}
&	  \lim_{n\to\infty} \sqrt{\ln(e^4\,n+1)}
							 -\sqrt{\ln(n+2)} 
	= \lim_{n\to\infty}\frac{\ln(e^4\,n+1)-\ln(n+1)}
								   {\sqrt{\ln(e^4\,n+1)}+\sqrt{\ln(n+2)}} = &\\
&	= \lim_{n\to\infty}\cfrac{\ln\left( \cfrac{e^4\,n+1}{n+1} \right)}
									 {\sqrt{\ln(e^4\,n+1)}+\sqrt{\ln(n+2)}}
	= \lim_{n\to\infty}\cfrac{\ln\left( \cfrac{e^4+1/n}{1+1/n} \right)}
									 {\sqrt{\ln(e^4\,n+1)}+\sqrt{\ln(n+2)}}
	= 0 &
\end{flalign*}

% Q5 - e)

\subsection{}
\begin{flalign*}
&	  \lim_{n\to\infty} n\ln(n+1)-n\ln(n)
	= \lim_{n\to\infty} n\ln\left( \frac{n+1}{n} \right)
	= \lim_{n\to\infty} \ln(1+1/n)^n
	= \ln(e^1)
	= 1 &
\end{flalign*}


\subsection{}
\subsection{}
\subsection{}
\subsection{}

% Q5 - j)
\subsection{}
\begin{flalign*}
&	  \lim_{n\to\infty} ( 1+1/n )^{n^2}\,(1-1/n)^{n^2}
	= \lim_{n\to\infty} \left( 1-1/n^2 \right)^{n^2}
	= e^{-1} &
\end{flalign*}

\section{Extra}
\begin{flalign*}
&	  \lim_{n\to\infty}\frac{n\,3^n+e^n}{(n+\sqrt{n})\,3^{n+1}+n^{100}}
	= \lim_{n\to\infty}\frac{1+\frac{e^n}{n\,3^n}}{3+\frac{3}{\sqrt n}+\frac{n^{99}}{3^n}}
	= \lim_{n\to\infty}\frac{1+n^{-1}\left(\frac{e}{3}\right)^n}{3+\frac{3}{\sqrt n}+\frac{n^{99}}{3^n}}
	= 1/3 &
\end{flalign*}

\paragraph{Nota:} funções exponenciais crescem sempre mais rápido que qualquer outra

\end{document}










\documentclass[12pt]{article}

% Linguagem
\usepackage[portuguese]{babel}

% Clickable Table of contents
\usepackage{hyperref}
\hypersetup{
	hidelinks=true,
	colorlinks=true,
	linkcolor=DarkGreen!20!LightGreen!25!
}

% Table of contents
\usepackage{tocloft}
\setlength{\cftsecnumwidth}{5em}
\setlength{\cftsubsecnumwidth}{4em} % Fix subsection width
% Fix space between subsection items on toc
%\renewcommand\cftsecafterpnum{\vskip0.05em}

% Multicols
%\usepackage{multicol}

% Customize Chapter
%\usepackage{titlesec}
%\titleformat{\chapter}[hang]{\Huge\bfseries\color{DarkGreen!75!}}{\thechapter\hspace{20pt}{$|$}\hspace{20pt}}{0pt}{\Huge\bfseries}

% Appendix
%\usepackage{appendix}

% Maths
\usepackage{amssymb} 		
\usepackage{amsmath} 
\usepackage[utf8]{inputenc} %useful to type directly diacritic characters

\newcommand{\bm}[1]{{\boldmath{\large{\begin{align*} #1 \end{align*}}}}}

% Vectors
%\usepackage{esvect} 	% Vector over-arrow
%\renewcommand{\vec}{\vv} % Vecto over-arrow

% Tikz
%\usepackage{tikz}		
%\usepackage{pgfmath}  	% calculations
%\usepackage{varwidth}  % List inside TikzPicture

% Chem
%\usepackage{chemformula} 	% formulas quimicas
%\usepackage{chemfig} 		% Estruturas quimicas

%\newcommand{\mol}[1]{ \text{mol}_{\ch{ #1 }} } % mol

% Tabular
%\usepackage{multirow}
%\usepackage{siunitx} % Column S: align on decimal

% Colors
\usepackage{xcolor}

\definecolor{DarkBlue}	{HTML}{252A36}
\definecolor{LightGreen}{HTML}{7CCC6C}
\definecolor{DarkGreen}	{HTML}{008675}

\colorlet{White}{DarkGreen!20!}
\colorlet{Black}{DarkBlue!110!}

\pagecolor{Black}
\color{White}

%\definecolor{Red}  {HTML}{FF7E79}
%\definecolor{Blue} {HTML}{6666FF}
%\definecolor{Green}{HTML}{66FF66}

% Counters
\counterwithin*{section}{part} % Reset section on part

\begin{document}

\title{\bfseries\color{DarkGreen!75!}%
	AM 1 - Resolução Ficha 5\\
	Limites e Continuidade de funções%
}
\author{Felipe Pinto - 61387}
\date{19/04 - 2021.1}

\maketitle
\tableofcontents
\break

% Questões
{\color{DarkGreen!75}\part{Questões}}

\renewcommand\thesection{Questão \arabic{section}}
\renewcommand\thesubsection{%
	Q\arabic{section} - \alph{subsection})%
}

\setcounter{section}{2}

% Q3
\section{}

\setcounter{subsection}{6}

% Q3 - g)
\subsection{$\lim_{x\to0}\arcsin(3\,x)/x$}
\begin{flalign*}
&
y = \arcsin(3\,x)
\implies
	\lim_{x\to0}\arcsin(3\,x)/x
=	\lim_{x\to0}\frac{y}{\sin(y)/3}
=	3\,\lim_{x\to0}\frac{y}{\sin(y)}
=	3
&
\end{flalign*}

% Questões Extras
{\color{DarkGreen!75}\part{Extras}}

\renewcommand\thesection{Extra \arabic{section}}
\renewcommand\thesubsection{%
	E\arabic{section} - \alph{subsection})%
}

% E1
\section{Incompleto}
\bm{
f:\mathbb{R}\to\mathbb{R};\quad
f(x) =
\left\{
\begin{array}{ll}
	(1-\cos(x))/x\quad & x<0
	\\
	0\quad & x=0
	\\
	x\,\cos(1/x)\quad&x>0
\end{array}
\right.
}
\begin{flalign*}
&
f\text{ é continua em x=0}
\iff	\lim_{x\to 0^-}f(x) = \lim_{x\to 0^+}f(x)
;\	\lim_{x\to 0^-}f(x)
=	\lim_{x\to 0^-}\frac{1-\cos(x)}{x}
&\\&
=	\lim_{x\to 0^-}\frac{\sin^2(x)}{x(1+\cos(x))}
=	\lim_{x\to 0^-}\left(\frac{\sin(x)}{x}\right)^2
	\,\lim_{x\to 0^-}\frac{x}{1+\cos(x)}
=	0
&\\&
	\lim_{x\to 0^+}f(x)
=	\cdots
&
\end{flalign*}

% E2
\section{$\lim_{x\to1} \sin(x-1)/|x-1| $}
\begin{flalign*}
&
=	\lim_{x\to1}
	\left\{
	\begin{array}{ll}
	%	
		\sin(x-1)/x-1\quad & x>0
		\\
		\sin(x-1)/1-x\quad & x<0
	%
	\end{array}
	\right.
;\ &\\&	\lim_{x\to1^+} \sin(x-1)/|x-1| = 1
;\		\lim_{x\to1^{-1}} \sin(x-1)/|x-1| = -1
&\\&	\therefore \nexists \lim_{x\to1} \sin(x-1)/|x-1|
&
\end{flalign*}

% E3
\section{$ \lim_{x\to0} x\,\cos(1/x)/(x-\sqrt{x}) $ Incompleto}

\begin{flalign*}
&
=	\lim_{x\to0} \frac{\cos(1/x)}{1-\sqrt{1/x}}
=	\cdots
=	0
&
\end{flalign*}

% E4
\section{$ \lim_{x\to0} \tan(x)/x\,\cos(x) $}
\begin{flalign*}
&
=	\lim_{x\to0} \frac{\sin(x)}{x\,\cos(x)\,\cos(x)}
=	\lim_{x\to0} (\sin(x)/x)
	\, \lim_{x\to0} 1/\cos^2(x)
=	1
&
\end{flalign*}

% E5
\section{$ 
	\lim_{x\to+\infty} 
	\frac {5^{x}  +2{x+1}}
		{6^{x-1}+e^x} 
$}
\begin{flalign*}
&
=	\lim_{x\to+\infty}
	\frac{(5/6)^x + 2(2/6)^x}
	     {(1)^x/6 + (e/6)^x }
=	\frac{0   + 0}
	     {1/6 + 0}
=	0
&
\end{flalign*}

% E6
\section{}
\bm{
	f(x) = 
	\left\{
	\begin{array}{lll}
	%
		\log(x^2+1) \quad & x<0
		\\
		a\,\arctan(x\,\pi/4) \quad & 0\leq x\leq 1
		\\
		(x^2-2\,x+1)/(x-1) \quad & x>1
	%
	\end{array}
	\right.
}
\begin{flalign*}
&
	\lim_{x\to1^-} a\,\arctan(x\,\pi/4)
=	\lim_{x\to1^+} (x^2-2\,x+1)/(x-1)
=	\lim_{x\to1^+} (x-1)^2/(x-1)
=	0
&\\&
\therefore a = 0
&
\end{flalign*}

% E7
\section{}
\bm{
	f(x) = 
	\left\{
	\begin{array}{ll}
 	%
		-e^{1/x} \quad & x<0 
		\\
		\log(1/(1+x^2)) \quad & x>0
	%
	\end{array}
	\right.
}

% E7 - a)
\subsection{$
	\lim_{x\to-\infty} f(x)
$ e $
	\lim_{x\to+\infty} f(x)
$}

\begin{flalign*}
&
	= -e^{\lim_{x\to+\infty}1/x} = -1
&\\\\&
	= \log\left( \lim_{x\to-\infty}\frac{1}{1+x^2} \right)
	= -\infty
&
\end{flalign*}

% E7 - b)
\subsection{$f(x)$ é continua em 0}
\begin{flalign*}
&
\iff
	\exists\,x\in\mathbb{R} 
	:\lim_{x\to0^-}f(x) = \lim_{x\to0^+}f(x)
\iff
	\lim_{x\to0^-} -e^{1/x}
=	0
= &\\& =
	\lim_{x\to0^+} \log(1/(1+x^2))
=	0
&
\end{flalign*}



\end{document}











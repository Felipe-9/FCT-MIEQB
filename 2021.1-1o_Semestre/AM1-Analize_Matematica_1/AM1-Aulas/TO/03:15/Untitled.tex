\documentclass[12pt]{article}

\usepackage{color}
\pagecolor{black}
\color{white}

\usepackage{amssymb} %maths
\usepackage{amsmath} %maths
\usepackage[utf8]{inputenc} %useful to type directly diacritic characters

\newcommand{\Fr}{ {\text{Fr}} }
\newcommand{\Interior}{ {\text{Int}} }
\newcommand{\Sup}{ {\text{Sup}} }
\newcommand{\Ext}{ {\text{Ext}} }


\title{AM 1}
\author{}

\begin{document}

\maketitle
\break
%\tableofcontents
\break

\section{Visinhança}
\begin{flalign*}
&	V_\epsilon(x)=X\subset \mathbb{R} \iff \{ x\in X;\ X= [ x-\epsilon, x+\epsilon ] \} &
\end{flalign*}

\section{Feixe}
\begin{flalign*}
& 	\overline X=\Fr (X) \cup X &
\end{flalign*}

\section{Ponto Interior}

\begin{flalign*}
&	x \in \Interior(X) \iff \left\{ x\in X;\ \exists\ \epsilon\in \mathbb{R}: V_{\epsilon}(x) \subset X \right\} &
\end{flalign*}

\section{Ponto Fronteira}
\begin{flalign*}
&	x\in \Fr(X) \iff \left\{ \forall\ \epsilon \in \mathbb{R}: V_{\epsilon}(x) \not\subset X \wedge V_{\epsilon}(x) \cap X \neq \emptyset \right\} &
\end{flalign*}

\section{Supremo}
\begin{flalign*}
&	\Sup(X)= x \iff \left\{ \nexists\ y: \{x,y\}\in \overline X;\ y>x;\ y\neq x \right\} &
\end{flalign*}

\section{Ponto Exterior}
\begin{flalign*}
&	x\in \Ext(X) \iff \left\{ x\in\mathbb{R};\ \exists\ \epsilon \in\mathbb{R} : V_{\epsilon}(x) \cap X =  \emptyset  \right\} &
\end{flalign*}

\section{Ponto de Acumulação}

\begin{flalign*}
&	x\text{ é ponto de acumulação de }X
	\iff \{ 
		V_{\epsilon}(x)\cap X- \{ x \} \neq \emptyset 
		: \forall\ \epsilon > 0
	\} &
\end{flalign*}

\section{Derivada do conjunto}
Conjunto dos pontos de acumulação de um conjunto
\begin{flalign*}
X' &= \{ x\in\mathbb{R}:V_{\epsilon}(x)\cap X- \{ x \} \neq \emptyset 
		: \forall\ \epsilon > 0 \} &
\end{flalign*}

\section{Conjunto \underline{Aberto}}
\begin{flalign*}
A\text{ é um conjunto aberto} &\iff A=\text{int}(A) &
\end{flalign*}

\section{Conjunto \underline{Fechado}}
\begin{flalign*}
A\text{ é um conjunto fechado} &\iff \overline A= A &
\end{flalign*}


\end{document}







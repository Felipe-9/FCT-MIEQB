\documentclass{article}



% lua
\usepackage{luacode}



% Geometry
\usepackage{geometry}
\geometry{
	papersize=	{240mm, 320mm},	% ( 4:3 )	SVGA x 0.4
	top=	.12\paperheight,
	bottom=	.12\paperheight,
	left=	.06\paperwidth,
	right=	.06\paperwidth,
	portrait= true,
}
%	papersize=	{300mm, 400mm},	% ( 4:3 )	SVGA x 0.5
%	papersize=	{240mm, 320mm},	% ( 4:3 )	SVGA x 0.4
%	papersize=	{180mm, 240mm},	% ( 4:3 )	SVGA x 0.3
%	papersize=	{229mm, 305mm},	% ( 4:3 )	ArchA/Arch1
%	papersize=	{320mm, 512mm},	% (16:10)
%	papersize=	{280mm, 448mm},	% (16:10)
%	papersize=	{240mm, 384mm},	% (16:10)
%	a4paper, %	{210mm, 297mm},	% (√2:1)	A4



% Font size
\usepackage[fontsize=12pt]{fontsize}



% Linguagem
\usepackage[portuguese]{babel}	% Babel
%\usepackage{polyglossia}		% Polyglossia
%\setdefaultlanguage[variant=brazilian]{portuguese}



%% Graphics
%\usepackage{graphics}
%\graphicspath{ {./resources/images} }



% calc
\usepackage{calc}



% Table of contents
\usepackage{tocloft}
\setcounter{tocdepth}{2}	% remove subsubsection from toc

%% part
%\renewcommand\cftpartfont{\bfseries}
%%\renewcommand\cftpartafterpnum{\vspace{0mm}}
%\setlength\cftbeforepartskip{6mm}
%
%% sec
%\renewcommand\cftsecfont{\bfseries} % Font
%\renewcommand\cftsecpagefont{}	 % page number font
%\renewcommand\cftsecleader{\cftdotfill{\cftdotsep}} % Dots
%\setlength\cftbeforesecskip{3mm}
%\setlength\cftsecindent{0mm}
%%\setlength{\cftsecnumwidth}{25mm}		% Fix section width
%
%% subsec
%\setlength\cftsubsecindent{0mm}
%%\setlength{\cftsubsecnumwidth}{15mm}
%
%% tab (table)
%\setlength\cfttabindent{0mm}

% part
\renewcommand\cftpartfont{\bfseries}
\renewcommand\cftpartafterpnum{\vspace{1mm}}
\setlength\cftbeforepartskip{4mm}

% sec
\renewcommand\cftsecfont{} % Font
\renewcommand\cftsecpagefont{}	 % page number font
\renewcommand\cftsecleader{\cftdotfill{\cftdotsep}} % Dots
\setlength\cftbeforesecskip{0mm}
\setlength\cftsecindent{0mm}
%\setlength{\cftsecnumwidth}{25mm}		% Fix section width

% subsec
\setlength\cftsubsecindent{0mm}
%\setlength{\cftsubsecnumwidth}{15mm}

% tab (table)
\setlength\cfttabindent{0mm}



% filecontents
%\usepackage{filecontents}	% Create files



% Multicols
\usepackage{multicol}
\setlength{\columnsep}{.05\textwidth}



% enumitem
\usepackage{enumitem} % modify enumerate index
%\setlist[itemize]{left = -9mm}
%\setlist[enumerate]{left= -5.4mm}



% titlesec
\usepackage{titlesec}

% Part customization
\titleclass{\part}{straight}
\titleformat{\part}
	[block]							% shape
	{\huge\bfseries\color{Emph}}			% format
	{\thepart\hspace{5mm}{$|$}}			% label
	{5mm}							% sep
	{\huge\bfseries}					% before-code
	[\vspace{0.5mm}]					% after-code
\counterwithin*{section}{part} % Reset section on part
%\@addtoreset{section}{part}
%
%% Chapter customization
%\titleclass{\chapter}{straight}
%\titleformat{\chapter}
%	[block]
%	{\Huge\bfseries\color{Emph}}
%	{\thechapter\hspace{5mm}{$|$}}
%	{5mm}
%	{\Huge\bfseries}
%	[\vspace{0.5mm}]



%% Appendix
%\usepackage{appendix}



% siunix: SI units
%\usepackage{siunitx}
%\sisetup{
%	scientific-notation = false,	% scientific / engineering / false
%	exponent-to-prefix = true,
%	exponent-product = *,
%	round-mode = places,		% figures/places
%	round-precision = 2,
%	round-minimum = 0.01
%}



% Maths
\usepackage{amsmath, amssymb}
\usepackage{bm} % Boldmath

\newcommand\BM[2][align*]{{\large\boldmath\bfseries%
	\begin{#1}
		#2
	\end{#1}%
}}

\newcommand\vizinhanca[2][\delta]{%
	\hyperref[vizinhanca]{V_{#1}(#2)}%
}

\newcommand\converge{{\,\xrightarrow{\text{converge}}\,}}



%% Vectors
%\usepackage{esvect} 	% Vector over-arrow
%\renewcommand{\vec}{\vv} % Vecto over-arrow



%% Tikz
%\usepackage{tikz}		
%\usepackage{pgfmath}  	% calculations
%\usepackage{varwidth}  % List inside TikzPicture


%%% pgf
%% pgfmath
%\usepackage{pgfmath}  	% calculations
%
%% pgfplots
%\usepackage{pgfplots}
%	\pgfplotsset{
%		compat=newest,
%		width=.90\textwidth,	% width
%		height= .22\textheigth,	% height
%		major grid style= {
%			very thin, 
%			color= White!60!Black
%		},
%		ticklabel style={
%			/pgf/number format/.cd,
%				set thousands separator={\,},
%		tick style= {color= White!60!Black},
%		% Extra ticks
%		},
%		every extra x tick/.style={
%			tick style= {draw=none},
%			major grid style= 
%				{draw, thin, color= White!90!Black},
%			ticklabel pos= top,
%		},
%		every extra y tick/.style={
%			tick style= {draw=none},
%			major grid style= 
%				{draw, thin, color= White!90!Black},
%			ticklabel pos= right,
%		},
%	}
%
% pgfplotstable
%\usepackage{pgfplotstable}



%% Tabular
%\usepackage{multirow}
%\usepackage{float}	% table position H(ere)
%\restylefloat{table}
%\usepackage{longtable}
% 
%\setlength\tabcolsep{6mm}		% Width
%\renewcommand\arraystretch{1.25}	% Height
%
%% booktabs
%\usepackage{booktabs}
%\setlength\heavyrulewidth{.75pt} 	% Top and bottom rule
%\setlength\lightrulewidth{.50pt} 	% Middle rule
%\usepackage{colortbl} 			% Color Cells



%% Chem
%\usepackage{chemformula} 	% formulas quimicas
%\usepackage{chemfig} 		% Estruturas quimicas
%\usepackage{modiagram}		% Molecular orbital diagram
%\setmodiagram{
%	names, 
%	labels, 
%	labels-fs=\tiny, 
%	AO-width=6mm,
%}
%
%\newcommand{\mol}[1]{ \unit{\mole\of{\ch{ #1 }}} } % mol



%% Code
%% Run on terminal: lualatex --shell-escape [file[.tex]]
%\usepackage{shellesc}
%\usepackage{minted}
%%\usemintedstyle{fruity}
%%\usemintedstyle{paraiso}
%%\usemintedstyle{rainbow_dash}
%%\usemintedstyle{solarized-dark}
%\usemintedstyle{stata-dark}
%%\usemintedstyle{native}



% Colors
\usepackage{xcolor}

\definecolor{DarkBlue}  {HTML}{252A36}
\definecolor{LightGreen}{HTML}{7CCC6C}
\definecolor{DarkGreen} {HTML}{008675}

\colorlet{White}{DarkGreen!20}
\colorlet{Black}{DarkBlue!110}
%\colorlet{Black}{DarkBlue!70!black}
\colorlet{Emph}{ DarkGreen!70!White}
\colorlet{Background}{White!5!Black}

\pagecolor{Black}
\color{White}

% mycolors
% Pallete
\definecolor{red}   {HTML}{FF0000}
\definecolor{orange}{HTML}{FF7300}
\definecolor{yellow}{HTML}{FFEA00}
\definecolor{green} {HTML}{2AFF00}
\definecolor{cyan}  {HTML}{00FFBF}
\definecolor{blue}  {HTML}{0055FF}
\definecolor{purple}{HTML}{9500FF}
\definecolor{pink}  {HTML}{FF0080}
% Light
\newcommand\Light{!63!White}
\colorlet{Red}   {red\Light}
\colorlet{Orange}{orange\Light}
\colorlet{Yellow}{yellow\Light}
\colorlet{Green} {green\Light}
\colorlet{Cyan}  {cyan\Light}
\colorlet{Blue}  {blue\Light}
\colorlet{Purple}{purple\Light}
\colorlet{Pink}  {pink\Light}
% Dark
\newcommand\Dark{!45!Black}
\colorlet{DRed}   {Red\Dark}
\colorlet{DOrange}{Orange\Dark}
\colorlet{DYellow}{Yellow\Dark}
\colorlet{DGreen} {Green\Dark}
\colorlet{DCyan}  {Cyan\Dark}
\colorlet{DBlue}  {Blue\Dark}
\colorlet{DPurple}{Purple\Dark}
\colorlet{DPink}  {Pink\Dark}



% tcolorbox
\usepackage{tcolorbox}
\tcbset{
	coltext=	White,		% text  color
	coltitle=	White,		% title color
	fonttitle=	\bfseries,	% title font
	colback=	Background,	% background color
	colframe=	Background,	% border     color
	arc=		3mm,		% Curvature
	}
%% Mybox - Incomplete
%\newtcbox{\mybox}{on line,
%  colframe=mycolor,colback=mycolor!10!white,
%  boxrule=0.5pt,arc=4pt,boxsep=0pt,left=6pt,right=6pt,top=6pt,bottom=6pt}



% mytitle and myauthor
\newcommand\mytitle{{Análise Matemática I - 2021.1}}
\newcommand\myauthor{{Felipe B. Pinto 61387 - MIEQB}}



% title, author and date
\title{\huge\bfseries\color{Emph}\mytitle}
\author{\Large\myauthor}
\date{\Large\today}



% hyperref
\usepackage{hyperref}
\hypersetup{
	% Links customization
	hidelinks=true,
	colorlinks=true,
	linkcolor=LightGreen!25!White,
	% PDF customization
	pdfpagelayout=OneColumn,
	pdftitle=\mytitle,
	pdfauthor=\myauthor,
}
% Fix links when reseting section on part
\renewcommand\theHsection{\thepart.\thesection}


%% questions, subquestions and subsubquestions
%
%% question
%\newcounter{question}[section]
%\renewcommand{\thequestion}{Questão \arabic{question}}
%\newcommand{\question}[1]{
%     % reset inner counters
%     \setcounter{subquestion}{0}
%     \setcounter{subsubquestion}{0}
%     % add section*
%     \refstepcounter{question}
%     \section*{\thequestion\quad#1}
%     % add to toc
%     \addcontentsline{toc}{section}{%
%     	\thequestion\quad#1%
%	}
%}
%
%% subquestion
%\newcounter{subquestion}[subsection]
%\renewcommand\thesubquestion{%
%	Q\arabic{question} - \alph{subquestion})%
%}
%\newcommand{\subquestion}[1]{
%	% reset inner counters
%	\setcounter{subsubquestion}{0}
%	% add subsection*
%     \refstepcounter{subquestion}
%     \subsection*{\thesubquestion\quad#1}
%     % add to toc
%     \addcontentsline{toc}{subsection}{%
%     	\thesubquestion\quad#1%
%	}
%}
%
%% subsubquestion
%\newcounter{subsubquestion}[subsubsection]
%\renewcommand{\thesubsubquestion}{(\roman{subsubquestion})}
%\newcommand{\subsubquestion}[1]{
%	% add subsubsection*
%     \refstepcounter{subsubquestion}
%     \subsubsection*{\thesubsubquestion\quad#1}
%     % add to toc
%     \addcontentsline{toc}{subsubsection}{%
%     	\thesubsubquestion\quad#1%
%	}
%}



%% Incompleto
%
%% Listoftemas: tema e subtema
%\newlistof{temas}{prog}{}
%%% Add to document
%%\section*{Programa}
%%\begin{multicols}{2} \listoftemas \end{multicols}
%% tema
%\newcounter{tema}[section]
%\newcommand\tema[1]{%
%	\refstepcounter{tema}%
%	\addcontentsline{prog}{temas}{%
%		\thetema\textsuperscript{o} Tema: #1%
%	}%
%}
%% subtema
%\newcounter{subtema}[subsection]
%\newcommand\subtema[1]{%
%	\refstepcounter{subtema}%
%	\addcontentsline{prog}{temas}{%
%		\textbullet\quad#1%
%	}
%}



% subsubsection customization
\renewcommand\thesubsubsection{(\roman{subsubsection})}



\begin{document}


\newgeometry{
	top=	.12\paperheight,
	bottom=	.12\paperheight,
	left=	.15\paperwidth, 
	right=	.15\paperwidth,
}


\maketitle
\label{title}


% Table of Contents

\renewcommand{\contentsname}{} % remove title

\section*{Conteúdo}
\begin{multicols}{2} \tableofcontents \end{multicols}


%% Temas
%\newlistof{temas}{prog}{}
%
%\section*{Programa}
%\begin{multicols}{2} \listoftemas \end{multicols}
%
%\newcounter{tema}[section]
%
%\newcommand\tema[1]{%
%	\refstepcounter{tema}%
%	\addcontentsline{prog}{temas}{%
%		\thetema.\quad#1%
%	}%
%}
%
%\newcounter{subtema}[subsection]
%\newcommand\subtema[1]{%
%	\refstepcounter{subtema}%
%	\addcontentsline{prog}{temas}{%
%		\textbullet\quad#1%
%	}
%}
%
%\tema{test}
%\subtema{test}

\newpage

\section*{Programa}

\begin{multicols}{2}
\begin{enumerate}[label=\arabic*.]
	
	\item {\bfseries 
		\hyperref[topologia elementar na reta r]
		{Topologia elementar da recta real}
	}
	\begin{enumerate}
	[label=\theenumi\arabic*., left = -5.4mm]
		
		\item \hyperref[pontos]{Pontos}:
		\begin{itemize}[left = -9mm]
		
			\item \hyperref[vizinhanca]
				{Vizinhança de um ponto}
				
			\item \hyperref[interior]
				{Ponto Interior e Exterior}
				
			\item Ponto Isolado
			\item Ponto Aderente
			\item \hyperref[ponto de acumulacao]
				{Ponto de Acumulação}
			
		\end{itemize}
		
		\item \hyperref[conjuntos]{Conjuntos}:
		\begin{itemize}[left = -9mm]
		
			\item \hyperref[conjunto aberto]
				 {Aberto e Fechado}
				 
			\item \hyperref[conjunto limitado]
				 {Limitado}
				 
			\item Compacto
		
		\end{itemize}
		
	\end{enumerate}
	
	\vspace{3mm}
	
	\item {\bfseries 
		\hyperref[inducao matematica]{Indução Matemática}
		e \hyperref[sucessoes]{Sucessões}
	}
	\begin{enumerate}
	[label=\theenumi\arabic*., left = -5.4mm]
	
		\item \hyperref[inducao matematica]
			 {Princípio de Indução Matemática}
		
		\item \
		\begin{itemize}[left = -9mm]
			
			\item Noção de Convergencia
			
			\item \hyperref[sucessao limitada]
				 {Limite de uma Sucessão}
				 
			\item \hyperref[algebra de limites]
				 {Algebra de Limites}
				 
			\item \hyperref[subsucessoes]{Subsuscessões}
			\item Sublimites
			\item Teoremas Fundamentais
			\item \hyperref[sucessao de cauchy]
				 {Sucessões de Cauchy}
			
		\end{itemize}
	
	\end{enumerate}
	
	\vspace{3mm}
	
	\item {\bfseries
		Limites e Continuidade em R
	}
	\begin{enumerate}
	[label=\theenumi\arabic*., left = -5.4mm]
		
		\item Limite de uma Função:
		\begin{itemize}[left = -9mm]
		
			\item Segundo Cauchy
			\item Segundo Heine
			\item Álgebra de Limites
		
		\end{itemize}
		
		\item \
		\begin{itemize}[left = -9mm]
		
			\item Continuidade de uma função num ponto
			\item Prolongamento por Continuidade
			\item Teorema de Bolzano
			\item Teorema de Weierstrass
			\item Continuidade da função composta
			\item Continuidade da função inversa
				 para a composição de funções
			\item Funções Inversas Clássicas
		\end{itemize}
		
	\end{enumerate}
	
	\vspace{3mm}
	
	\item {\bfseries
		Calculo Diferencial em R
	}
	\begin{enumerate}
	[label=\theenumi\arabic*., left = -5.4mm]
		
		\item \
		\begin{itemize}[left = -9mm]
		
			\item Definição de diferenciabilidade
				 num ponto
			\item Interpretação geométrica
			\item Derivada de uma função
			\item Composta da Função inversa
			\item Derivada da Função inversa
			\item Teorema de Rolle
			\item Teorema de Lagrange
			\item Derivada e Monotonia
			\item Teorema de Darboux
			\item Teorema de Cauchy
			\item Regra de L'Hospital-Cauchy
		
		\end{itemize}
		
		\item Teorema de Taylor e Aplicações
		
	\end{enumerate}
	
	\vspace{3mm}
	
	\item {\bfseries
		Cálculo Integral em R
	}
	\begin{enumerate}
	[label=\theenumi\arabic*., left = -5.4mm]
	
		\item Primitivas:
		\begin{itemize}[left = -9mm]
			
			\item por Partes
			\item por Substituição
			\item de Funções Racionais
			\item de Funções Irracionais
			\item de Funções Transcendentes
			
		\end{itemize}
		
		\item \
		\begin{itemize}[left = -9mm]
		
			\item Integral de Riemann
			\item Teorema do valor médio
			\item Teorema Fundamental do Cálculo Integral
			\item Regra de Barrow
			\item Integração por partes
			\item Integração por substituição
			\item Aplicação ao cálculo de áreas
			
		\end{itemize}
		
		\item Integrais Impróprios
		
	\end{enumerate}	
	
\end{enumerate}
\end{multicols}

%1. Topologia elementar da recta real.
%
%1.1 Vizinhança de um ponto. Ponto interior, exterior, fronteiro, isolado, aderente e de acumulação.
%
%1.2 Conjunto aberto, fechado, limitado e compacto.
%
% 
%
%2. Indução Matemática e Sucessões
%
%2.1 Princípio de Indução Matemática.
%
%2.2 Noção de convergência e limite de uma sucessão. Álgebra de limites. Subsucessões. Sublimites. Teoremas Fundamentais. Sucessões de Cauchy. 
%
%3. Limites e Continuidade em R
%
%3.1 Limite de uma função segundo Cauchy e segundo Heine. Álgebra de limites.
%
%3.2 Continuidade de uma função num ponto. Prolongamento por continuidade. Teorema de Bolzano e Teorema de Weierstrass. Continuidade da função composta. Continuidade da função inversa para a composição de funções. Funções inversas clássicas.
%
%4. Cálculo Diferencial em R
%
%4.1 Definição de diferenciabilidade num ponto. Interpretação geométrica. Derivada de uma função. Derivada da função composta e derivada da função inversa. Teorema de Rolle, Teorema de Lagrange. Derivada e monotonia. Teorema de Darboux e Teorema de Cauchy. Regra de L''Hospital-Cauchy.
%
%4.2 Teorema de Taylor e aplicações.
%
%5. Cálculo Integral em R
%
%5.1 Primitivas. Primitivação por partes. Primitivação por substituição. Primitivação de funções racionais. Primitivação de funções irracionais e de funções transcendentes.
%
%5.2 Integral de Riemann. Teorema do valor médio. Teorema Fundamental do Cálculo Integral. Regra de Barrow. Integração por partes e integração por substituição. Aplicação ao cálculo de áreas. 
%
%5.3 Integrais impróprios.

\restoregeometry


\newpage



\part{Background}



\newpage



\part{Topologia Elementar na Reta $\mathbb{R}$}
\label{topologia elementar na reta r}



\section{Vizinhança $V_{\delta}$}
\label{vizinhanca}

\begin{flalign*}
&
	\vizinhanca{x} = (x-\delta, x+\delta)
	\quad \delta\in\mathbb{R}
&
\end{flalign*}



\section*{Pontos:}
\label{pontos}



% Majorante e Minorante
\begin{multicols}{2}
	
	
\section{Minorante}
\label{minorante}

\begin{flalign*}
&
	m \in \text{Minorante}(X)
\iff &\\&
\iff
	m\in\mathbb{R}
\land
	x \geq m
	\ \forall\,x\in X
&
\end{flalign*}


%\vfill


\section{Majorante}
\label{majorante}

\begin{flalign*}
&
	m \in \text{Majorante}(X)
\iff &\\&
\iff
	m\in\mathbb{R}
\land
	x \leq m\ \forall\,x\in X
&
\end{flalign*}


\end{multicols}



% Infimo e Supremo
\begin{multicols}{2}

\section{Infimo Inf}
\label{Infimo}

\subsubsection{Standalone}

\begin{flalign*}
&
	\text{Inf}(X) = i
\iff &\\&
\iff	
	i\in\mathbb{R}
\land
	x \geq m
	\ \forall\,x\in X	
\land &\\&
\land
	\nexists\,y\in\mathbb{R}\backslash X
	: i<y<x
	\ \forall\,x\in X
&
\end{flalign*}


\subsubsection{Usando \hyperref[vizinhanca]{Vizinhança}}

\begin{flalign*}
&
	\text{Inf}(X) = i
\iff &\\&
\iff	
	i\in\mathbb{R}
\land
	x\geq i\ \forall\,x\in X
\land
	\vizinhanca{i}\cap X\neq\emptyset
&
\end{flalign*}


% \vfill


\section{Supremo Sup}
\label{Supremo}

\subsubsection{Standalone}

\begin{flalign*}
&
	\text{Sup}(X) = s
\iff &\\&
\iff
	s\in\mathbb{R}
\land
	x\leq s\ \forall\,x\in X
\land &\\&
\land
	\nexists\,y\in\mathbb{R}\backslash X
	: x<y<s
&
\end{flalign*}


\subsubsection{Usando \hyperref[vizinhanca]{Vizinhança}}

\begin{flalign*}
&
	\text{Sup}(X) = s
\iff &\\&
\iff
	s\in\mathbb{R}
\land
	x\leq s\ \forall\,x\in X
\land
	\vizinhanca{s}\cap X\neq\emptyset
&
\end{flalign*}


\end{multicols}



% Minimo e Maximo
\begin{multicols}{2}
	
	
\section{Minimo Min}
\label{minimo}

\subsubsection{Standalone}
\begin{flalign*}
&
	\text{Min}(X) = m
\iff &\\&
\iff
	m\in X
\land
	m \leq y\ \forall\,y\in m
&
\end{flalign*}


\subsubsection{Usando \hyperref[majorante]{Minorante}}
\begin{flalign*}
&
	\text{Min}(X) = m
\iff &\\&
\iff
	m\in X
\land
	m\in \hyperref[majorante]{\text{Minorante}(X)}
&
\end{flalign*}


% \vfill


\section{Maximo Max}
\label{maximo}

\subsubsection{Standalone}
\begin{flalign*}
&
	\text{Max}(X) = m
\iff &\\&
\iff
	m\in X
\land
	m \geq y\ \forall\,y\in m
&
\end{flalign*}


\subsubsection{Usando \hyperref[majorante]{Majorante}}
\begin{flalign*}
&
=	\text{Max}(X) = m
\iff &\\&
\iff
	m\in X
\land
	m\in \hyperref[majorante]{\text{Majorante}(X)}
&
\end{flalign*}


\end{multicols}



% Interior e Exterior
\begin{multicols}{2}


\section{Interior Int}
\label{interior}

\subsubsection{Standalone}

\begin{flalign*}
&
	x\in\text{Int}(X)
\iff &\\&
\iff
	\exists\,\delta>0:(x-\delta,x+\delta)\subseteq X
&
\end{flalign*}


\subsubsection{Usando \hyperref[vizinhanca]{Vizinhança}}

\begin{flalign*}
&
	x\in\text{Int}(X)
\iff &\\&
\iff
	\vizinhanca{x}\subseteq X
&
\end{flalign*}


% vfill


\section{Exterior Ext}
\label{exterior}

\subsubsection{Standalone}

\begin{flalign*}
&
	x\in\text{Ext}(X)
\iff &\\&
\iff
	\exists\,\delta\in\mathbb{R}\backslash\{0\}
	:(x-\delta,x+\delta)\cap X =\emptyset
&
\end{flalign*}


\subsubsection{Usando \hyperref[vizinhanca]{Vizinhança}}

\begin{flalign*}
&
	x\in\text{Ext}(X)
\iff &\\&
\iff
	\vizinhanca{x}\cap X =\emptyset
&
\end{flalign*}


\end{multicols}



\section{Ponto de Acumulação}
\label{ponto de acumulacao}

\begin{flalign*}
&
	x \text{ é um ponto de acumulação de } X
\iff	&\\&
\iff
	\vizinhanca{x}\cap\left( X\backslash\{x\} \right)
\neq \emptyset
&
\end{flalign*}



\section{Fronteira Fr}
\label{fronteira}

\begin{flalign*}
&
	f\in\text{Fr}(X)
\iff &\\&
\iff
	\vizinhanca{f}\cap X\neq\emptyset
\land
	\vizinhanca{f}\cap\mathbb{R}\backslash X\neq\emptyset
&
\end{flalign*}


	
\section*{Conjuntos:}
\label{conjuntos}



\section{Limitado}
\label{conjunto limitado}

\begin{flalign*}
&
	X \text{ é um conjunto limitado}
\iff &\\&
\iff
	\{
	m_1 \leq x \leq m_2
	\quad\forall\,x\in X 
	: m_1 \in \hyperref[majorante]{\text{Minorante}(X)}
,\	  m_2 \in \hyperref[majorante]{\text{Majorante}(X)}
	\}
&
\end{flalign*}


\begin{multicols}{2}

\section{Aberto}
\label{conjunto aberto}

\begin{flalign*}
&
	X \text{ é um conjunto aberto}
\iff &\\&
\iff
	X=\hyperref[interior]{\text{Int}}(X)
&
\end{flalign*}


% vfill


\section{Fechado}
\label{conjunto fechado}

\begin{flalign*}
&
	X \text{ é um conjunto fechado}
\iff &\\&
\iff
	X
=	\hyperref[interior]{\text{Int}(X)}
\cup	\hyperref[fronteira]{\text{Fr}(X)}
&
\end{flalign*}


\end{multicols}



% Feixe e Ponto de Acumulação
\begin{multicols}{2}

\section{Feixe $\overline X$}
\label{feixe}

\begin{flalign*}
&
	\bar X
= 	\hyperref[interior]{\text{Int}(X)}
\cup	\hyperref[fronteira]{\text{Fr}(X)}
&
\end{flalign*}

\end{multicols}



\newpage



\part{Indução Matemática}
\label{inducao matematica}


\section{Indução por Igualdade}
\label{inducao por igualdade}


\begin{center}
\begin{itemize}

	\item Prove que é valido para algum $n$
	\item Prove que é valido para $n+1\quad\forall\,n:$ seja valido para n

\end{itemize}
\end{center}

\begin{flalign*}
&
	\text{Seja } V
=	\left\{
	P_{(n)}\quad\forall\,n \in \text{Dominio}
	\right\}
&\\&
	P_{(x)} \in V;
\	P_{(x+1)}\in V
&
\end{flalign*}



\subsection{Exemplo}
\BM{
	\sum\limits_{i=0}^{n} \frac{1}{2^i} 
= 	2-\frac{1}{2^n}
\quad
	\forall\,n\in\mathbb{N}_0
}


\begin{multicols}{2}


\subsubsection{$n=0$}
\begin{flalign*}
&
	\sum\limits_{i=0}^{0} \frac{1}{2^i}
=	1
= 	2-\frac{1}{2^0}
=	2-1=1
&
\end{flalign*}


\subsubsection{$n=m+1$}
\begin{flalign*}
&
	\sum\limits_{i=0}^{m+1} \frac{1}{2^i}
=	\sum\limits_{i=0}^{m} \frac{1}{2^i} + \frac{1}{2^{m+1}}
=	&\\&
=	2-\frac{1}{2^{m}}+\frac{1}{2^{m+1}}
=	2-\frac{1}{2^{m+1}}
&
\end{flalign*}


\end{multicols}


\newpage


\section{Indução por Desigualdade}
\label{inducao por desigualdade}

\begin{center}
\begin{itemize}
	
	\item Prove que é valido para algum $n$
	\item Prove que é valido para $n+1\quad\forall\,n:$ seja valido para n
	
\end{itemize}
\end{center}


\begin{multicols}{2}

\subsection{Exemplo 1}
\BM{ n\leq 2^{n-1}\quad\forall\,n\in\mathbb{N} }


\subsubsection{$ n=1 $}
\begin{flalign*}
&
	1\leq 2^{1-1}= 1
&
\end{flalign*}

\subsubsection{$ n=m+1 $}
\begin{flalign*}
&
	m+1
\leq 2^{m-1}+1
= 	\frac{2^{m}+2}{2}
\leq 2^{m+1-1} = 2^{m}
\implies &\\&
\implies
	2 
\leq 2^{m+1}-2^m
=	2^m(2-1)
=	2^m
&
\end{flalign*}


\vfill


\subsection{Exemplo 2}
\BM{ n^2\leq 2^n\quad\forall\,n\in\mathbb{N}\land n\geq4 }

\subsubsection{$n=4$}

\begin{flalign*}
&
	4^2\leq 2^4 = 4^2
&
\end{flalign*}

\subsubsection{$n=m+1$}

\begin{flalign*}
&
	(m+1)^2
=	m^2+2\,m+1
\leq	&\\&
\leq
	2^m+2\,m+1
\leq 
	2 * 2^{m}
=	2^{m+1}
\implies &\\&
\implies
	2\,m+1 \leq 2^m
;\	&\\&
	\begin{cases}
	%
		m=4
	\implies 
		2*4+1 = 9 
	\leq 2^4 = 16
	\\	
		m=n+1
	\implies 
		2*(n+1)+1=2\,n+1+2
	\leq \\
	\leq 2^n+2
	\leq 2*2^{n} = 2^{n+1}
	\implies
		2\leq 2^n
	%
	\end{cases}
&
\end{flalign*}


\end{multicols}



\newpage



\part{Sucessões}
\label{sucessoes}


\BM{ u_{(n)}: \mathbb{N}\to\mathbb{R} }


\section{Sucessão Monótona}
\label{sucessao monotona}


\begin{multicols}{2}


\subsection{Decresente}
\label{sucessao monotona decrescente}

\begin{flalign*}
&
	u_{(n)}\geq u_{(n+1)}
	\quad\forall\,n\in\mathbb{N}
&
\end{flalign*}


% \vfill


\subsection{Crescente}
\label{sucessao monotona crescente}

\begin{flalign*}
&
	u_{(n)}\leq u_{(n+1)}
	\quad\forall\,n\in \mathbb{N}
&
\end{flalign*}


\end{multicols}



\section{Sucessão Limitada}
\label{sucessao limitada}

\begin{flalign*}
&
	u_{(n)} \text{ é limitada}
\iff &\\&
\iff
	m_1 \leq u_{(n)} \leq m_2
\quad\forall\,n\in\mathbb{N}
&
\end{flalign*}


\begin{multicols}{2}


\section{Sucessão Convergente}
\label{sucessao convergente}


\begin{flalign*}
&
	u_{(n)} \text{ converge para } l\in\mathbb{R}
\iff &\\&
\iff
	\exists\,p\in\mathbb{N}: u_{(n)}\in\vizinhanca{l}
	\quad\forall\,n>p
&
\end{flalign*}


\vfill


\paragraph{Nota:}

\begin{flalign*}
&
	u_{(n)}\in\vizinhanca{l}
\iff	|u_{(n)}-l| < \delta
\iff &\\&
\iff	l-\delta < u_{(n)} < l+\delta
&
\end{flalign*}


\end{multicols}



\subsection{Exemplo}
\BM[flalign*]{
&
\quad
	u_{(n)} = 1/\sqrt{2\,n-1}
;\ &\\& \quad
	u_{(n)} > 0
;\ 	\delta = 1/10
&
}


\begin{flalign*}
&
\iff	0 < \frac{1}{\sqrt{2\,n-1}} < \frac{1}{10}
\iff	&\\&
\iff	0 < \frac{1}{2\,n-1} < \frac{1}{100}
;\
	2\,n-1>100
\implies	&\\&
\implies
	\lfloor101/2\rfloor<n
&
\end{flalign*}



\newpage



\section{Algebra de Limites}
\label{algebra de limites}


\begin{multicols}{2}

\subsection{}

\begin{flalign*}
&
	\bm{u_{(n)}} \textbf{ é \hyperref[sucessao convergente]{convergente}}
\iff	&\\&
\iff
	u_{(n)} \in \vizinhanca[\epsilon]{l}
	\quad\forall\,\epsilon>0
\iff	&\\&
\iff
	l-\epsilon<u_{(n)}<l+\epsilon
\iff	&\\&
\iff
	\exists\,\{m_1, m_2\}\subset\mathbb{R}:
	m_1<u_{(n)}<m_2
	&\\&
	\quad\forall\,n\in\mathbb{N}
\iff	&\\&
\iff
	\bm{u_{(n)}}\textbf{ é \hyperref[sucessao limitada]{limitada}}
&
\end{flalign*}



\subsection{}


\begin{flalign*}
&
	\bm{u_{(n)}} \textbf{ é \hyperref[sucessao monotona]{monotona} e \hyperref[sucessao limitada]{limitada}}
\iff	&\\&
\iff
	(u_{(n)} < u_{(n+1)} \lor u_{(n)} > u_{(n+1)})
\land &\\&
\land
	\exists\,\{m_1,m_2\}\subset\mathbb{R}
	: m_1<u_{(n)}<m_2\quad\forall\,n\in\mathbb{N}
\implies	&\\&
\implies
	\exists\,\hyperref[supremo]{\text{Sup}(u_{(n)})}
	\in\mathbb{R}\backslash u_{(n)}
\iff	&\\&
\iff
	\bm{u_{(n)}}\textbf{ é \hyperref[sucessao convergente]{convergente}}
&
\end{flalign*}



\subsection{}

\boldmath\begin{flalign*}
&	
	u_{(n)} \textbf{ e } v_{(n)} \textbf{ são \hyperref[sucessao convergente]{convergentes}}\,
\land	&\\&
\land
	u_{(n)} \leq v_{(n)}\quad\forall\,n\in\mathbb{N}
\implies	&\\&
\implies
	\lim\limits_{n\to\infty}u_{(n)}
\leq	\lim\limits_{n\to\infty}v_{(n)}
&
\end{flalign*}



\subsection{}

\begin{flalign*}
&
	\bm{u_{(n)}} \textbf{ é \hyperref[sucessao limitada]{limitada}}
\bm{\land}
	\bm{v_{(n)} \to 0}
\implies	&\\&
\implies
	0
\leq	|u_{(n)}*v_{(n)}|=|u_{(n)}|*|v_{(n)}|
\leq	&\\&
\leq	M
	*0=0
	\quad\forall\,M\in\hyperref[majorante]{\text{Majorante}(u_{n})}
\implies	&\\&
\implies
	\bm{u_{(n)}*v_{(n)}\to 0}
&
\end{flalign*}



\end{multicols}



\begin{multicols}{2}



\section{Lema geral das sucessões enquadradas}

\begin{flalign*}
&
	\{u_{(n)}, v_{(n)}, w_{(n)}\}:\mathbb{N}\to\mathbb{R}
\,\land	&\\&
\land
	v_{(n)}\leq u_{(n)}\leq w_{(n)}
	\quad\forall\,n\in\mathbb{N}
\,\land	&\\&
\land
	\exists\,l\in\mathbb{R}
	:\{v_{(n)},w_{(n)}\} \converge l
\implies	&\\&
\implies
	u_{(n)} \converge l
&
\end{flalign*}



\subsection{Exemplo: $ u_{(n)}=\sin(n)/n $}

\begin{flalign*}
&
	\frac{-1}{n}
\leq	\frac{\sin(n)}{n}
\leq	\frac{1}{n}
;\
	\left\{\frac{-1}{n}, \frac{1}{n}\right\}\to 0
\implies	&\\&
\implies
	\frac{\sin(n)}{n}\to0
&
\end{flalign*}



\end{multicols}



\section{Subsucessões}
\label{subsucessoes}



\subsection{Subsucessão Convergente}
\label{subsucessao convergente}



\section{Sucessão de Cauchy}
\label{sucessao de cauchy}




\newpage



\part{Limites}

\section{Lema geral das funções enquadradas}


\subsection{Exemplo: $ u_{(n)}=\sin(n)/n $}

\begin{flalign*}
&
	\frac{-1}{n}
\leq	\frac{\sin(n)}{n}
\leq	\frac{1}{n}
;\
	\left\{\frac{-1}{n}, \frac{1}{n}\right\}\to 0
\implies	&\\&
\implies
	\frac{\sin(n)}{n}\to0
&
\end{flalign*}






\newpage



\part{Continuidade em $\mathbb{R}$}




\end{document}



\documentclass[12pt]{article}

% Geometry
\usepackage{geometry}
\geometry{
	a4paper, % 210mm por 297mm
	top=15mm,
	left=15mm,
	right=15mm
}

% Linguagem
\usepackage[portuguese]{babel}	% Babel
%\usepackage{polyglossia}		% Polyglossia
%\setdefaultlanguage[variant=brazilian]{portuguese}

% Graphics
%\usepackage{graphics}

% calc
\usepackage{calc}

% Clickable Table of contents
\usepackage{hyperref}
\hypersetup{
	hidelinks=true,
	colorlinks=true,
	linkcolor=DarkGreen!20!LightGreen!25!
}

% Table of contents
\usepackage{tocloft}
\setlength{\cftsecnumwidth}{25mm}		% Fix section width
\setlength{\cftsubsecnumwidth}{15mm}
% Fix space between subsection items on toc
%\renewcommand\cftsubsecafterpnum{\vskip5pt}

% Multicols
\usepackage{multicol}

% Customize Chapter
%\usepackage{titlesec}
%\titleformat{\chapter}[hang]{\Huge\bfseries\color{DarkGreen!75!}}{\thechapter\hspace{20pt}{$|$}\hspace{20pt}}{0pt}{\Huge\bfseries}

% Appendix
%\usepackage{appendix}

% siunix: SI units
\usepackage{siunitx}
\sisetup{
	scientific-notation = engineering,
	exponent-product = *,
	round-mode = figures,
%	round-precision = 3,
%	round-minimum = 0.01
}

% Maths
\usepackage{amssymb} 		
\usepackage{amsmath} 

%\newcommand{\bm}[1]{{\boldmath{\large{\begin{align*} #1 \end{align*}}}}}

% Vectors
%\usepackage{esvect} 	% Vector over-arrow
%\renewcommand{\vec}{\vv} % Vecto over-arrow

% Tikz
%\usepackage{tikz}		
%\usepackage{pgfmath}  	% calculations
%\usepackage{varwidth}  % List inside TikzPicture

% Chem
%\usepackage{chemformula} 	% formulas quimicas
%\usepackage{chemfig} 		% Estruturas quimicas

%\newcommand{\mol}[1]{ \text{mol}_{\ch{ #1 }} } % mol

% Tabular
%\usepackage{multirow}

% Colors
\usepackage{xcolor}

\definecolor{DarkBlue}	{HTML}{252A36}
\definecolor{LightGreen}{HTML}{7CCC6C}
\definecolor{DarkGreen}	{HTML}{008675}

\colorlet{White}{DarkGreen!20!}
\colorlet{Black}{DarkBlue!110!}

\pagecolor{Black}
\color{White}

%\definecolor{Red}  {HTML}{FF7E79}
%\definecolor{Blue} {HTML}{6666FF}
%\definecolor{Green}{HTML}{66FF66}

% Counters
%\counterwithin*{section}{part} % Reset section on part

% Section and Subsection Customization
\renewcommand\thesection{Questão \arabic{section}}
\renewcommand\thesubsection{%
	\arabic{section} - \alph{subsection})%
}
\renewcommand\thesubsubsection{(\roman{subsubsection})}


\begin{document}

\title{\bfseries\color{DarkGreen!75!}%
	IEQB Ficha 1 - Resolução\\
	Conversão de unidades e cálculos em engenharia%
}
\author{Felipe Pinto - 61387}

\newgeometry{left=25mm, right=25mm}

\maketitle

{\noindent\Large\bfseries Conteúdo}
\renewcommand{\contentsname}{}
\begin{multicols}{2}
	\tableofcontents
\end{multicols}

\restoregeometry

\begin{multicols}{2}

% Q1
\section{}
\begin{flalign*}
&
	1\,\unit{P}
	\,\frac{\unit{\g.\cm^{-1}.\s^{-1}}}{\unit{P}}
	\,\frac{1\,\unit{lbm}}{453.6\,\unit{\g}}
	\,\frac{30.48\,\unit{\cm}}{1\,\unit{ft}}
= &\\& \cong
	\qty[round-precision=4]
	{0.06719576}{\g.lbm.ft^{-1}.s^{-1}}
&
\end{flalign*}

% Q2
\section{}
\begin{flalign*}
&
	R 
	= 8.314\,\unit{\kg.\m^2.s^{-2}.\mol^{-1}.\kelvin^{-1}}
	\,\frac{\unit{\Pa}}{\unit{\kg.\m^{-1}.\s^{-2}}}
	* &\\& *
	\,\frac{9.869*10^{-6}\,\unit{atm}}{\unit{\Pa}}
	\,\frac{c^3}{(10^{-2})^3}
\cong
	\qty[round-precision=4, per-mode=fraction]
	{82.050866}{\frac{\cm^3\,atm}{\mole\,\kelvin}}
&
\end{flalign*}

\end{multicols}

% Q3
\section{$ \frac{g(\rho_L-\rho_G)D_b^3}{\sigma\,D_0} = 6 $}
\begin{flalign*}
&
\implies
	D_b 
= 
	\sqrt[3]{\frac{6\,\sigma\,D_0}{g(\rho_L-\rho_{G})}}
=
	\sqrt[3]{
		\frac{6*70.8\,\unit{dyn.\cm^{-1}}*1\,\unit{\mm}}
			{32.174\,\unit{ft.\s^{-2}}
			(
				1\,\unit{\g.\cm^{-3}}
				-0.081\,\unit{lbm.ft^{-3}}
			)}}
= &\\& = 
	\sqrt[3]{
		\frac{
			6*70.8\,\unit{\g.\cm.\s^{-2}.\cm^{-1}}
			*0.1\,\unit{\cm}
			}
			{32.174*30.48\,\unit{\cm.\s^{-2}}
			(
				1\,\unit{\g.\cm^{-3}}
				-\frac{0.081*453.59237}{28316.846592}
				\,\unit{\g.\cm^{-3}}
			)}}
= &\\& =
	\sqrt[3]{
		\frac{6*70.8*0.1}
			{
				32.174*30.48
				(1-\frac{0.081*453.59237}
					   {28316.846592}
				)
			}
		}\unit{\cm}
\cong
	\qty[round-precision=3]{0.351352285561243}{\cm}
&
\end{flalign*}

\begin{multicols}{2}

% Q4
\section{}
\begin{flalign*}
&
	300\,\unit{\joule\per\min}
	\,\frac{\unit{hp}}
		  {745.69987158227022\,\unit{\watt}}
	\,\frac{\unit{\watt}}{\unit{\joule\per\s}}
	\,\frac{\unit{\min}}{60\,\unit{s}}
\cong &\\& \cong
	\qty[round-precision=3]{0.006705110447975}{hp}
&
\end{flalign*}

\ \vspace{-1pt}

% Q5
\section{}
\begin{flalign*}
&
	1\unit{\newton}
	\,\frac{1\,\unit{lbf}}{4.448222\,\unit{\newton}}
\cong
	\qty[round-precision=1, scientific-notation = false]
	{0.224808923655339}{lbf}
&
\end{flalign*}

\ \vspace{-1pt}

% Q6
\section{}

% Q6 - a)
\subsection{%
	\num[scientific-notation=true, round-precision=3]
	{12200}%
}

% Q6 - b)
\subsection{%
	\num[scientific-notation=true, round-precision=6]
	{12200.0}%
}

% Q6 - c)
\subsection{%
	\num[scientific-notation=true, round-precision=4]
	{0.003040}%
}

\ \vspace{-1pt}

% Q7
\section{}

% Q7 - a)
\subsection{%
	\num[scientific-notation=fixed, 
		fixed-exponent=0, 
		round-mode=off]
	{1.34d5}, 3%
}

% Q7 - b)
\subsection{%
	\num[scientific-notation=fixed, 
		fixed-exponent=0, 
		round-mode=off]
	{1.340d-2}, 4%
}

\ \vspace{-1pt}

% Q8
\section{}

% Q8 - a)
\subsection{3}

% Q8 - b)
\subsection{2}

\ \vspace{-1pt}

% Q9
\section{$ D_{(m)} = 3\,t_{(s)}+4 $}

% Q9 - a)
\subsection{velocidade e espaço}

% Q9 - b)
\subsection{$ 
	3 = \unit{\metre\per\second};\quad
	4 = \unit{\metre}
$}

\end{multicols}

% Q10
\section{}
\begin{flalign*}
&
	\unit{N_{pr}}
=	\unit{\frac{C_p.\mu}{K}}
=	\frac{
			583\,\unit{\joule.\kg^{-1}.\celsius^{-1}}
		\,	0.802\,\unit{\kg.\m^{-1}.s^{-1}}
		}
		{
			0.286\,\unit{\watt.\m^{-1}.\celsius^{-1}}
		}
= 	\frac{ 583*0.802 }{ 0.286 }
\cong
	1650_{\text{sem calc}}
\cong 
	\num[scientific-notation = false, round-precision=3]
	{1634.846153846153846}_{\text{com calc}}
&
\end{flalign*}

% Q11
\section{$ 
	K\,(\unit{\mole.\cm^{-3}.\s^{-1}}) 
=	1.2*10^{5}\,\exp(-2000/1.987\,T)
$}

\begin{flalign*}
&
\iff	1.2*10^{5}
	\,\exp 
		\left(
		\frac{-2000\,\unit{cal.\mole^{-1}}}
			{1.987\,T\,\unit{\kelvin}}
		\right)
	\,\unit{\mole.\cm^{-3}.\s^{-1}}
= &\\& =
	1.2*10^{5}\,\unit{\mole.\cm^{-3}.\s^{-1}}
	\,\exp
		\left(
		\frac{-2000\,\unit{cal.\mole^{-1}}}
			{
				1.987\,\unit{cal.\mole^{-1}.\kelvin^{-1}}
			\,	T\,\unit{\kelvin}
			}
		\right)
&
\end{flalign*}

% Q12
\section{$ \rho = 80.5\,\exp(8.27*10^{-7}\,P) $}

% Q12 - a)
\subsection{}
\begin{flalign*}
&
\iff	80.5
	\,\exp(
		8.27*10^{-7}
		\,P\,\unit{lbf\per in\squared}
		)
	\,\unit{lbm.ft^{-3}}
=	80.5\,\unit{lbm.ft^{-3}}
	\,\exp(
		8.27*10^{-7}\unit{in\squared lbf^{-1}}
		\,P\,\unit{lbf.in^{-2}}
		)	
&
\end{flalign*}

% Q12 - b)
\subsection{}
\begin{flalign*}
&
	80.5\,\unit{lbm.ft^{-3}}
	\,\exp(
		8.27*10^{-7}\unit{in\squared lbf^{-1}}
		\,9.00*10^{6}\,\unit{\newton.\m^{-2}}
		)
= &\\& =
	80.5\,\unit{lbm.ft^{-3}}
	\,\frac{\unit{\g}}{0.002204622622\,\unit{lbm}}
	\,\left(
		\frac{0.03280839895\,\unit{ft}}{\unit{\cm}}
	  \right)^3
	* &\\& *
	\exp
		\left(
		8.27*10^{-7}\unit{in\squared lbf^{-1}}
		\,9.00*10^{6}\,\unit{\newton.\m^{-2}}
		\,\frac{0.224809\,\unit{lbf}}{\unit{\newton}}
		\,\left(
			\frac{\unit{\m}}{39.3700787402\,\unit{in}}
		  \right)^2
		\right)
\cong
	\qty[round-precision=3]
	{1.290879074419363}{\gram\per\cubic\cm}
&
\end{flalign*}

% Q12 - b)
\subsection{}
\begin{flalign*}
&
	\rho\,\unit{\gram\per\cubic\cm}
=	80.5\,\unit{lbm.ft^{-3}}
	\,\frac{\unit{\g}}{0.002204622622\,\unit{lbm}}
	\,\left(
		\frac{0.03280839895\,\unit{ft}}{\unit{\cm}}
	  \right)^3
	* &\\& *
	\exp
		\left(
		8.27*10^{-7}\unit{in\squared lbf^{-1}}
		\,P\,\unit{\newton.\m^{-2}}
		\,\frac{0.224809\,\unit{lbf}}{\unit{\newton}}
		\,\left(
			\frac{\unit{\m}}{39.3700787402\,\unit{in}}
		  \right)^2
		\right)
\cong &\\& \cong
	\frac{80.5*0.03280839895^3}{0.002204622622}
	\,\unit{\gram\per\cubic\cm}
	* \exp
		\left(
		\frac{8.27*10^{-7}*0.224809}{39.3700787402^2}
		\unit{\metre\squared \newton^{-1}}
		\,P\,\unit{\newton.\m^{-2}}
		\right)
\cong &\\& \cong
	\qty[round-precision=3]{1.289486301499866}{\gram\per\cubic\cm}
	* \exp
		\left(
		\qty[round-precision=3]
		{0.000000000119946}{\metre\squared \newton^{-1}}
		\,P\,\unit{\newton.\m^{-2}}
		\right)
&
\end{flalign*}

% Q12 - d)
\subsection{}
Liquido, pois sua densidade varia pouco com a pressão

\end{document}











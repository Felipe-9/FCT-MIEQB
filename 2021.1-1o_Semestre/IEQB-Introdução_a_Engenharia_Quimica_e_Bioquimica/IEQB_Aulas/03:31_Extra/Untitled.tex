\documentclass[12pt]{article}

% Linguagem
%\usepackage[portuguese]{babel}

% Multicols
%\usepackage{multicol}

% Maths
%\usepackage{amssymb} 		
%\usepackage{amsmath} 
%\usepackage[utf8]{inputenc} %useful to type directly diacritic characters

%\newcommand{\bm}[1]{{\boldmath{\large{\begin{align*} #1 \end{align*}}}}}

% Vectors
%\usepackage{esvect} 	% Vector over-arrow
%\usepackage{tikz}		% Vector diagrams

%\renewcommand{\vec}{\vv} % Vecto over-arrow

% Chem
%\usepackage{chemformula} 	% formulas quimicas
%\usepackage{chemfig} 		% Estruturas quimicas

% Colors
\usepackage{xcolor}

\definecolor{DarkBlue}	{HTML}{252A36}
\definecolor{LightGreen}{HTML}{7CCC6C}
\definecolor{DarkGreen}	{HTML}{008675}

\pagecolor{DarkBlue!110!}
\color{DarkGreen!20!}

% Resolução de listas
%\renewcommand\thesection{Questão \arabic{section} }
%\renewcommand\thesubsection{\arabic{section}-\alph{subsection}) }

\begin{document}

\title{IEQB - Introdução}
\date{30/03}

\maketitle
\tableofcontents
\break

\section{aula 1 e 2}

\subsection{Provas}
2 testes

\subsubsection{datas}
1o - 8 maio
2o - 19 junho (pode variar)

\subsection{resumo disciplina}
ver um processo químico e fazer balanço material(massa ou mol) e fazer balanços energéticos (energia necessária para processo químico)

\section{Aula 3}
algarismos significativos, balanceamento e conversão de unidades

\section{Aula 4}
balanceamento de fluxos, sem reações químicas

\subsection{Caudal}
fluxo de mistura
\begin{itemize}
\item $\dot n$ Caudal molar 
\item $\dot v$ Caudal volumetrico
\item $\dot m$ Caudal massico
\end{itemize}

\paragraph{Caudal volumétrico:} é importante dizer também a temperatura, pois essa varia bastante o volume do fluido quando gasoso

\subsection{Fração de componentes em um caudal}
\begin{itemize}
\item Fracção mássica $x_A$
\item Fracção molar $y_A$
\item Percentagem mássica
\item Percentagem molar
\end{itemize}

fracção de um componente em uma mistura

\paragraph{fracção molar média}
somatório do produto da fracção molar pela massa molar de cada componente

\section{aula 5}


\end{document}













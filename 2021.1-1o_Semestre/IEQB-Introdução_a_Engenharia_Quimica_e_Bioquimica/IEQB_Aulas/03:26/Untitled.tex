\documentclass[12pt]{article}

% Maths
%\usepackage{amssymb} 		
%\usepackage{amsmath} 
%\usepackage[utf8]{inputenc} %useful to type directly diacritic characters

% Vectors
%\usepackage{esvect} 	% Vector over-arrow
%\usepackage{tikz}		% Vector diagrams

%\renewcommand{\vec}{\vv} % Vecto over-arrow

% Chem
\usepackage{chemformula} 	% formulas quimicas
%\usepackage{chemfig} 		% Estruturas quimicas

% Colors
\usepackage{xcolor}

\definecolor{DarkBlue}	{HTML}{252A36}
\definecolor{LightGreen}	{HTML}{7CCC6C}
\definecolor{DarkGreen}	{HTML}{008675}

\pagecolor{DarkBlue!110!}
\color{DarkGreen!15!}

\begin{document}

\title{IEQB}
\date{03/26}

\maketitle
\tableofcontents
\break

\section{Cálculos com balanço materiais}

\subsection{Flowsheet (Diagrama esquemático)}
Esquematização de dados para rápida leitura
\begin{itemize}
\item \textbf{Completar} o diagrama com toda a informação existence
\item \textbf{Identificar} as variáveis desconhecidas
\end{itemize}

\subsection{Graus de liberdade}
É a diferença entre equações conhecidas e numero de variáveis desconhecidas

\subsection{Escalamento de processos}
Pre-resolver na escala básica de calculo para depois mudar-la

\subsection{Ex: (Completar)}
\newcommand{\exI}{\ch{MeOH}}
%\begin{flalign*}
%&	m_3=m_2+m_1 = 350 Kg_{m_3}/h &\\
%&	\frac{x\, Kg_{\ch{MeOH}}}{Kg_{m_3}} 
%	\, \frac{m_3\,Kg_{m_3}}{h};\ 
%	m_3\, Kg_{m_3}= m_1\, Kg_{m_1}+m_2\, Kg_{m_2}; &\\
%&	\frac{150\,Kg_{m_2}}{h}
%	\,\frac{70\,Kg_{\ch{MeOH}}}{100\,Kg_{m_2}}
%	+ \frac{200\,Kg_{m_1}}{h}
%	\, \frac{40\,Kg_{\ch{MeOH}}}{100\,Kg_{m_1}}
%	\implies &\\
%&	X_{\ch{MeOH}} &
%\end{flalign*}

\subsection{ex:}
\begin{flalign*}
&	 C_1 = C_4-C_5 = 60\,Kg/h;\ C_2=C_1+C_6=90\,Kg/h;\ C_3=C_2-C_7=60\,Kg/h &\\
&	\frac{A_1\,Kg_A}{Kg}
	\, \frac{60\, Kg}{h}
	= \frac{100\,Kg}{h}
	\, \frac{0.5\, Kg_{A_4}}{Kg}
	- \frac{40\, Kg}{h}
	\, \frac{0.9\, Kg_{A_5}}{Kg}
	= \frac{14\,Kg_{A_1}}{60\,Kg} \cong 23.33\%\, Kg_{A_1}/Kg &
\end{flalign*}

\section{Random}



\end{document}

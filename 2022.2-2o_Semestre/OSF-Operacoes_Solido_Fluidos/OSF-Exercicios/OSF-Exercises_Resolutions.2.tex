% !TEX root = ./OSF-Exercises_Resolutions.2.tex
\providecommand\mainfilename{"./OSF-Exercises_Resolutions.tex"}
\providecommand \subfilename{}
\renewcommand   \subfilename{"./OSF-Exercises_Resolutions.2.tex"}
\documentclass[\mainfilename]{subfiles}

% \tikzset{external/force remake=true} % - remake all

\begin{document}

% \graphicspath{{\subfix{./.build/figures/OSF-Exercises_Resolutions.2}}}
% \tikzsetexternalprefix{./.build/figures/OSF-Exercises_Resolutions.2/graphics/}

\mymakesubfile{2}
[OSF]
{Redução da granulometria de sólidos} % Subfile Title
{Redução da granulometria de sólidos} % Part Title

\begin{questionBox}1{ % Q4
    Se se regularem uns rolos de moagem de 1\,\unit{\metre} de diâmetro de tal modo que as superfícies de moagem fiquem à distância de 12.5\,\unit{\milli\metre} e o ângulo de presa for \ang[round-precision=0]{31}
} % Q4
    \begin{questionBox}2{ % Q4.1
        qual é o tamanho máximo de partículas que se deveria introduzir nos rolos?
    } % Q4.1
        \answer{}
        \begin{flalign*}
            &
                \cos\alpha
                = \cos(31/2)
                =\frac{r_1+b}{r_1+r_2}
                =\frac{
                    (1.0/2)+(12.5/2)
                }{
                    (1.0/2)+r_2
                }
                \implies &\\&
                \implies
                r_2
                =\frac{
                    0.5+6.25
                }{
                    \cos(31/2)
                }-0.5
                =\frac{
                    0.5+6.25
                }{
                    \cos(31/2)
                }-0.5
                \cong
                \qty{6.504759944566266}{\metre}
            &
        \end{flalign*}
    \end{questionBox}
    \begin{questionBox}2{ % Q4.2
        Se a capacidade real da máquina é 12\% da teórica, calcular o ritmo de produção em \unit{\kilo\gram\per\second}, quando a funcionar a 2.0\,\unit{\hertz}, se a superfície de trabalho dos rolos tiver 0.4\,\unit{\metre} de comprimento e se a alimentação pesar 2500\,\unit{\kilo\gram/\metre^3}.
    } % Q4.2
        \answer{}
        \begin{flalign*}
            &
                \dot{m}
                = z\,A\,\mu\,\rho
            &
        \end{flalign*}
    \end{questionBox}
\end{questionBox}

\end{document}
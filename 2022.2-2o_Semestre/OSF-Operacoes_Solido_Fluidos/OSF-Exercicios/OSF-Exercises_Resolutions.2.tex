% !TEX root = ./OSF-Exercises_Resolutions.2.tex
\documentclass["./OSF-Exercises_Resolutions.tex"]{subfiles}

% \tikzset{external/force remake=true} % - remake all

\begin{document}

% \graphicspath{{\subfix{./.build/figures/OSF-Exercises_Resolutions.2}}}
% \tikzsetexternalprefix{./.build/figures/OSF-Exercises_Resolutions.2/graphics/}

\mymakesubfile{2}[OSF]
{Redução da granulometria de sólidos} % Subfile Title
{Redução da granulometria de sólidos} % Part Title

\begin{questionBox}1m{} % Q1
  A material is crushed in a Blake jaw crusher and the average size of particle reduced from \qty*{50}{\mm} to \qty*{10}{\mm} with the consumption of energy at the rate of \qty*{13.0}{\kilo\W/\kg.\s}. What will be the consumption of energy needed to crush the same material of average size \qty*{75}{\mm} to an average size of \qty*{25}{\mm}:

  \begin{questionBox}2{Assuming Rittinger's law applies?} % Q1.1
    \answer{\eqref{eq:q2.1 answer}}
    
    Energy comsumption
    \begin{tcolorbox}
      \begin{gather*}
        E(75,25)/\unit{\kilo\W/\kg.\s}
        = \mathText{Rittinger's Law}
        = K_R\,f_c
        \,\left( (25)^{-1}-(75)^{-1} \right)
        \cong \num{5.128205128205128e-3}
        \,\left( (25)^{-1}-(75)^{-1} \right)
        \cong \num{1.367521367521368e-4}
        % 
        \yesnumber\label{eq:q2.1 answer}
        % 
        ; \\
        E(30,10)/\unit{\kilo\W/\kg.\s}
        = 13.0
        = \mathText{Rittinger's Law}
        = K_R\,f_c
        \,\left( (10)^{-1}-(30)^{-1} \right)
        \implies 
        K_3\,f_c
        \cong \num{5.128205128205128e-3}
      \end{gather*}
    \end{tcolorbox}

  \end{questionBox}
  \begin{questionBox}2{Assuming Kick's law applies?} % Q1.2
    \answer{\eqref{eq:q2.2 answer}}

    Calculating energy comsumption
    \begin{tcolorbox}
      \begin{gather*}
        E(75,25)/\unit{\kilo\W/\kg.\s}
        = \mathText{Kick's law}
        = K_k\,f_c\,\ln{
          \frac{L_1}{L_2}
        }
        \cong \num{8.077354149274954}
        \,\ln{
          \frac{75}{25}
        }
        \cong \num{8.873880528317809}
        % 
        \yesnumber\label{eq:q2.2 answer}
        % 
        ; \\
        E(50,10)/\unit{\kilo\W/\kg.\s}
        = 13.0
        = \mathText{Kick's law}
        = K_k\,f_c\,\ln{
          \frac{L_1}{L_2}
        }
        = K_k\,f_c\,\ln{
          \frac{50}{10}
        }
        \implies
        K_k\,f_c
        = \frac{13}{\ln(50/10)}
        \cong \num{8.077354149274954}
      \end{gather*}
    \end{tcolorbox}
  \end{questionBox}

  \begin{questionBox}*2{Which of these results would be regarded as being more reliable and why?} % Qindex
    \answer{}
    A reductiong of \((75\to25)\unit{\mm}\) can be considered coarse for which Kick's law is more accurate
  \end{questionBox}
\end{questionBox}

\begin{questionBox}2m{} % Qindex
  A crusher was used to crush a material whose compressive strength was \qty*{22.5}{\mega\N/\m^2}. The size of the feed was \textit{minus} \qty*{50}{\mm}, plus \qty*{40}{mm}, and the power required was \qty*{13.0}{\kilo\W/\kg/\s}. The screen analysis of the product was as follows:
  \begin{center}
    \vspace{1ex}
    \begin{tabular}{l S S[round-precision=0]}
      \toprule

      \multicolumn{2}{c}{Minimum size/\unit{\mm}}
      & \multicolumn{1}{c}{Quantity Product/\%}

      \\\midrule

      Through & 6     & 100 \\
      On      & 4     & 26  \\
      On      & 2     & 18  \\
      On      & 0.75  & 23  \\
      On      & 0.50  & 8   \\
      On      & 0.25  & 17  \\
      On      & 0.125 & 3   \\
      Through & 0     & 5

      \\\bottomrule
    \end{tabular}
  \end{center}
  What would be the power required to crush \qty*{1}{\kg/s} of a material of compressive strength \qty*{45}{\MN/\m^2} from a feed \textit{minus} \qty*{45}{\mm}
  
  \answer{}

  Calculating energy required
  \begin{tcolorbox}
    Reduction from \textit{minus} \qty{45}{\mm} can be considered
  \end{tcolorbox}

\end{questionBox}

\setcounter{question}{3}
\begin{questionBox}1m{} % Q4
  Se se regularem uns rolos de moagem de 1\,\unit{\metre} de diâmetro de tal modo que as superfícies de moagem fiquem à distância de 12.5\,\unit{\milli\metre} e o ângulo de presa for \ang[round-precision=0]{31}

  \begin{questionBox}2{} % Q4.1
    qual é o tamanho máximo de partículas que se deveria introduzir nos rolos?

    \answer{}

    \begin{flalign*}
      &
      \cos\alpha
      = \cos(31/2)
      =\frac{r_1+b}{r_1+r_2}
      =\frac{
        (1.0/2)+(12.5/2)
      }{
        (1.0/2)+r_2
      }
      \implies &\\&
      \implies
      r_2
      =\frac{
        0.5+6.25
      }{
        \cos(31/2)
      }-0.5
      =\frac{
        0.5+6.25
      }{
        \cos(31/2)
      }-0.5
      \cong
      \qty{6.504759944566266}{\metre}
      &
    \end{flalign*}

  \end{questionBox}

  \begin{questionBox}2{ } % Q4.2
    Se a capacidade real da máquina é 12\% da teórica, calcular o ritmo de produção em \unit{\kilo\gram\per\second}, quando a funcionar a 2.0\,\unit{\hertz}, se a superfície de trabalho dos rolos tiver 0.4\,\unit{\metre} de comprimento e se a alimentação pesar 2500\,\unit{\kilo\gram/\metre^3}.

    \answer{}

    \begin{flalign*}
      &
      \dot{m}
      = z\,A\,\mu\,\rho
      &
    \end{flalign*}

  \end{questionBox}

\end{questionBox}


\end{document}

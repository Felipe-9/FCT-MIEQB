% !TEX root = ./OSF-Exercises_Resolutions.3.tex
\providecommand\mainfilename{"./OSF-Exercises_Resolutions.tex"}
\providecommand \subfilename{}
\renewcommand   \subfilename{"./OSF-Exercises_Resolutions.3.tex"}
\documentclass[\mainfilename]{subfiles}

% \tikzset{external/force remake=true} % - remake all

\begin{document}

% \graphicspath{{\subfix{./.build/figures/OSF-Exercises_Resolutions.3}}}
% \tikzsetexternalprefix{./.build/figures/OSF-Exercises_Resolutions.3/graphics/}

\mymakesubfile{3}
[OSF]
{Movimento de Particulas num fluido} % Subfile Title
{Movimento de Particulas num fluido} % Part Title

\begin{questionBox}1{ % Q1
    Sujeita-se a elutriação uma mistura finamente moída de galena e calcário na proporção de 1 para 4 em peso, mediante uma corrente ascendente de água, que flui a 0.5\,\unit{\centi\metre/\second}. Supondo que a distribuição de tamanhos é a mesma para ambos os materiais e corresponde à que se indica no quadro seguinte, faça a estimativa da percentagem de galena no material arrastado e no material que fica para trás. Considere a viscosidade absoluta da água igual a 1\,\unit{\milli\newton.\second.\metre^{-2}} e use a equação de Stokes.
} % Q1
    \begin{center}
        \setlength\tabcolsep{3mm}        % width
        % \renewcommand\arraystretch{1.25} % height
        \vspace{1ex}
        \begin{tabular}{l *{8}{C}}
            \toprule
            
            Diâmetro (mícrons) 
            & 20 & 30 & 40 & 50 & 60 & 70 & 80 & 100
            \\ 
            \% em peso de finos 
            & 15 & 28 & 48 & 54 & 64 & 72 & 78 & 88
            
            \\\bottomrule
        \end{tabular}
        \vspace{2ex}
    \end{center}

    \paragraph{Dados:} 
    \begin{itemize}
        \item densidade da galena \(=7.5\)
        \item densidade do calcáreo \(=2.7\)
    \end{itemize}

    \answer{}

    \begin{flalign*}
        &
            \frac{F}{
                (\pi\,(d/2)^2)
                \,\rho\,u^2
            }
            = \frac{
                3\,\pi\,\mu\,u\,d
            }{
                (\pi\,(d/2)^2)
                \,\rho\,u^2
            }
        &
    \end{flalign*}

\end{questionBox}

\end{document}
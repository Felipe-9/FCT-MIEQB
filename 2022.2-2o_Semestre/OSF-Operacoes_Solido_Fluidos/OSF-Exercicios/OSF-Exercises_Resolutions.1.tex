% !TEX root = ./OSF-Exercises_Resolutions.1.tex
\providecommand\mainfilename{"./OSF-Exercises_Resolutions.tex"}
\providecommand \subfilename{}
\renewcommand   \subfilename{"./OSF-Exercises_Resolutions.1.tex"}
\documentclass[\mainfilename]{subfiles}

% \tikzset{external/force remake=true} % - remake all

\begin{document}

% \graphicspath{{\subfix{./.build/figures/OSF-Exercises_Resolutions.1}}}
% \tikzsetexternalprefix{./.build/figures/OSF-Exercises_Resolutions.1/graphics/}

\mymakesubfile{1}
[OSF]
{Subfile Title} % Subfile Title
{Subfile Title} % Part Title

\setcounter{question}{2}

\begin{questionBox}1{ % Q3
    \sisetup{round-precision=0}
    A análise granulométrica de um material em pó numa base de peso é representada por uma linha recta que vai de \qty{0}{\percent} em peso na dimensão de partícula de \qty{1}{\micro\metre} até \qty{100}{\percent} em peso na dimensão de partícula de \qty{101}{\micro\metre}.
} % Q3
    \begin{questionBox}2{ % Q3.1
        Calcular o diâmetro médio em volume das partículas que constituem o sistema.
    } % Q3.1
        \answer{}
        \begin{flalign*}
            &
                \bar{d_V}
                = \frac
                {\int{d\,\odif{V}}}
                {\int{   \odif{V}}}
                = \frac
                {\rho_s\int{d\,\odif{V}}}
                {\rho_s\int{   \odif{V}}}
                = \frac
                {\int{d\,\odif{x}}}
                {\int{   \odif{x}}}
                = \frac
                {\int_0^1{(100\,x+1)\,\odif{x}}}
                {\int_0^1{            \odif{x}}}
                = &\\&
                = \frac{
                    100
                    \,\int_0^1{
                        (x+1/100)
                        \,\odif[var]{x+1/100}
                    }
                }{1-0}
                = 100/2
                \,\adif{\left(
                    (x+1/100)^2
                \right)}\big\vert_0^1
                = &\\&
                = 100/2
                \,\left(
                    (1+1/100)^2
                    - (1/100)^2
                \right)
                = 100/2
                \,\left(
                    1+2/100
                \right)
                = &\\&
                = \qty{51}{\micro\metre}
            &
        \end{flalign*}
    \end{questionBox}
    \begin{questionBox}2{ % Q3.2
        Calcular o diâmetro médio superficial das partículas que constituem o sistema.
    } % Q3.2
        \answer{}
        \begin{flalign*}
            &
                \bar{d}_s
                = \frac{
                    \int{d\,\odif{s}}
                }{
                    \int{\odif{s}}
                }
                = \frac{
                    \int{d\,\odif{\left(
                        n\,\dot{k}\,d^2
                    \right)}}
                }{
                    \int{\odif{\left(
                        n\,\dot{k}\,d^2
                    \right)}}
                }
                % = &\\&
                = \frac{
                    \int{d\,\odif{\left(
                        \left(
                            \frac{x}{\ddot{k}\,d^3\,\rho_s}
                        \right)\,\dot{k}\,d^2
                    \right)}}
                }{
                    \int{\odif{\left(
                        \left(
                            \frac{x}{\ddot{k}\,d^3\,\rho_s}
                        \right)\,\dot{k}\,d^2
                    \right)}}
                }
                = &\\&
                = \frac{
                    \int{
                        \frac{
                            \dot{k}
                        }{
                            \ddot{k}\,\rho_s
                        }\,\odif{x}
                    }
                }{
                    \int{
                        \frac{\dot{k}}{
                            \ddot{k}\,d\,\rho_s
                        }
                        \odif{x}
                    }
                }
                % = &\\&
                = \frac{
                    \frac{
                        \dot{k}
                    }{
                        \ddot{k}\,\rho_s
                    }
                    \int{
                        \odif{x}
                    }
                }{
                    \frac{\dot{k}}{
                        \ddot{k}\,\rho_s
                    }
                    \int{
                        d^{-1}\,\odif{x}
                    }
                }
                &\\[6ex]&
                d_s
                = 1\Biggr{/}\sum{x_i/d_i}
                = 1\Biggr{/}\int_0^1{
                    \frac{\odif{x}}{d}
                }
                = 1\Biggr{/}\int_0^1{
                    \frac{\odif{x}}{
                        100\,x+1
                    }
                }
                = &\\&
                = 100\Biggr{/}\int_0^1{
                    \frac{\odif{x}}{
                        x+1/100
                    }
                }
                = 100\Biggr{/}\int_0^1{
                    \frac{\odif[var]{x+1/100}}{
                        x+1/100
                    }
                }
                = &\\&
                = 100\Biggr{/}\left(
                    \adif{\ln(x+1/100)}\big\vert_0^1
                \right)
                = &\\&
                = 100\Big{/}\left(
                    \ln(1+1/100)
                    -\ln(1/100)
                \right)
                = &\\&
                = 100/\ln{101}
                % \cong &\\&
                \cong
                \SI{21.667906533553168}{\micro\metre}
            &
        \end{flalign*}
    \end{questionBox}
\end{questionBox}

\end{document}
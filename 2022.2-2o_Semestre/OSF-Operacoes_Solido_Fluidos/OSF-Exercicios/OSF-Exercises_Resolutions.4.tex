% !TEX root = ./OSF-Exercises_Resolutions.4.tex
\providecommand\mainfilename{"./OSF-Exercises_Resolutions.tex"}
\providecommand \subfilename{}
\renewcommand   \subfilename{"./OSF-Exercises_Resolutions.4.tex"}
\documentclass[\mainfilename]{subfiles}

% \tikzset{external/force remake=true} % - remake all

\begin{document}

% \graphicspath{{\subfix{./.build/figures/OSF-Exercises_Resolutions.4}}}
% \tikzsetexternalprefix{./.build/figures/OSF-Exercises_Resolutions.4/graphics/}

\mymakesubfile{4}
[OSF]
{Sendimentação e Espressamento} % Subfile Title
{Sendimentação e Espressamento} % Part Title

\begin{questionBox}1{ % Q1
    Um ensaio de decantação em tubo de ensaio foi realizado com uma suspensão de carbonato de cálcio (massa específica: 2710\,\unit{\kilo\gram/\metre^3}) em água, cuja concentração inicial é de 236\,\unit{\gram/\litre}. Os resultados do ensaio vêm expressos na seguinte tabela.
} % Q1
    \begin{center}
        \vspace{1ex}
        \begin{tabular}{C C}
            \toprule
            
                \multicolumn{1}{c}{Tempo (\unit{\hour})}
                & \multicolumn{1}{c}{Altura de interface (\unit{\centi\metre})}
            
            \\\midrule
            
               0.0  & 36.0
            \\ 0.25 & 32.4
            \\ 0.5  & 28.6
            \\ 1.0  & 21.0
            \\ 1.75 & 14.7
            \\ 3.0  & 12.3
            \\ 4.75 & 11.6
            \\ 12.0 & 9.8
            \\ 20.0 & 8.0
            
            \\\bottomrule
        \end{tabular}
        \vspace{2ex}
    \end{center}
    \answer{}
    \begin{center}
        \vspace{1ex}
        \begin{tabular}{*{4}{C}}
            \toprule
            
                \multicolumn{1}{c}{t (\unit{\second})}
                & \multicolumn{1}{c}{C (\unit{\kilo\gram/\metre^3})}
                & \multicolumn{1}{c}{e (\unit{\frac{\metre^3\of{sólido}}{\metre^3\of{solução}}})}
                & \multicolumn{1}{c}{u (\unit{\gram/\metre^3})}
            
            \\\midrule
            
               0.0   & \num{236}                & \num{0.9129151291512915} & 
            \\ 0.25  & \num{262.22222222222223} & \num{0.9032390323903239} & 
            \\ 0.5   & \num{297.06293706293707} & \num{0.8903826800505767} & 
            \\ 1.0   & \num{404.57142857142856} & \num{0.8507116499736426} & 
            \\ 1.75  & \num{577.9591836734694}  & \num{0.7867309285337751} & 
            \\ 3.0   & \num{690.7317073170732}  & \num{0.7451174511745118} & 
            \\ 4.75  & \num{732.4137931034483}  & \num{0.7297366077109047} & 
            \\ 12.0  & \num{866.938775510204}   & \num{0.6800963928006627} & 
            \\ 20.0  & \num{1062}               & \num{0.6081180811808118} & 
            
            \\\bottomrule
        \end{tabular}
        \vspace{2ex}
    \end{center}
    \begin{questionBox}2{ % Q1.1
        Determine a concentração de sólidos na zona de espessado em função do tempo
    } % Q1.1
        \answer{}
        \begin{flalign*}
            &
                C\,h\,A
                = C_0\,h_0\,A 
                % \implies
                \implies
                C = C_0\,\frac{h_0}{h}
            &
        \end{flalign*}
    \end{questionBox}
    \begin{questionBox}2{ % Q1.2
        Determine a porosidade na zona de espessado em função do tempo.
    } % Q1.2
        \answer{}
        \begin{flalign*}
            &
                e: 
                C 
                = (1-e)\,\rho_s
                \implies
                e = 1-\frac{C}{2710}
            &
        \end{flalign*}
    \end{questionBox}
    \begin{questionBox}2{ % Q1.3
        question
    } % Q1.3
        body
    \end{questionBox}
\end{questionBox}

\end{document}
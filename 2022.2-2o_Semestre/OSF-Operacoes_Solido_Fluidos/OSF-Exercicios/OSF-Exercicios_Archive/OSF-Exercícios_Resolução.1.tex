% !TEX root = ./OSF-Exercícios_Resolução.1.tex
\providecommand\mainfilename{"./OSF-Exercícios_Resolução.tex"}
\providecommand \subfilename{}
\renewcommand   \subfilename{"./OSF-Exercícios_Resolução.1.tex"}
\documentclass[\mainfilename]{subfiles}

% \graphicspath{{\subfix{../images/}}}
% \tikzset{external/force remake=true} % - remake all

\begin{document}

\mymakesubfile{1}
[OSF]
{Exercícios}
{Exercícios}

\begin{questionBox}1{}
    
    A análise por peneiração duma amostra de um sólido finamente moído produziu o seguinte resultado:

    \begin{table}[H]\centering
        \setlength\tabcolsep{3mm}        % width
        \begin{tabular}{*{4}{c}}
            
            \\\toprule
            
                \multicolumn{1}{c}{Peneiro}
                &
                \multicolumn{2}{c}{
                        \begin{tabular}{c}
                            Dimensão da abertura
                            \\
                            (mm)
                        \end{tabular}
                }
                &
                \multicolumn{1}{c}{
                    \begin{tabular}{c}
                        Percentagem do produto
                        \\
                        (\% em número)
                    \end{tabular}
                }
            
            \\\midrule
            
                1 & Passando por    & 6.00  & 100
            \\  2 & Retido em       & 4.00  & 26
            \\  3 & Retido em       & 2.00  & 18
            \\  4 & Retido em       & 0.75  & 23
            \\  5 & Retido em       & 0.50  & 8
            \\  6 & Retido em       & 0.25  & 17
            \\  7 & Retido em       & 0.125 & 3
            \\  8 & Passando por    & 0.125 & 5
            
            \\\bottomrule
            
        \end{tabular}
    \end{table}
    
\end{questionBox}

\begin{questionBox}2{}
    
    Calcule a fração de superfície do sólido retido no peneiro 3

    \paragraph*{RS} 4.3
    
\end{questionBox}

\end{document}
% !TEX root = ./OSF-Exercícios_Resolução.2.tex
\providecommand\mainfilename{"./OSF-Exercícios_Resolução.tex"}
\providecommand \subfilename{}
\renewcommand   \subfilename{"./OSF-Exercícios_Resolução.2.tex"}
\documentclass[\mainfilename]{subfiles}

% \graphicspath{{\subfix{../images/}}}
% \tikzset{external/force remake=true} % - remake all

\begin{document}

\mymakesubfile{2}
[OSF]
{Redução da Granulometria de Sólidos}
{Redução da Granulometria de Sólidos}

% Q1
\begin{questionBox}1{}
    
    Tritura-se um material num triturador de maxilas Blake e reduz-se o tamanho médio das partículas de 50 mm para 10 mm, com um consumo de energia de 13.0 kw s kg-1. Qual será o consumo de energia necessário para triturar o mesmo material do tamanho médio 75 mm até à dimensão média de 25 mm
    
\end{questionBox}

% Q1.1
\begin{questionBox}2{}

    supondo aplicável a lei de Rittinger, e

    \begin{flalign*}
        &
            L_1
            = 50\,\unit{\milli\metre}
            \to
            L_2 = 10\,\unit{\milli\metre}
            &\\&
            E=13.0\,\unit{\kilo\joule/\kilo\gram}
            &\\&
            E = C(L_2^{-1}-L_1^{-1})
            % \implies &\\&
            \implies
            C = 162.5\,\unit{\kilo\joule/\kilo\gram}
        &
    \end{flalign*}
    
    % \paragraph*{RS} 4.3
    
\end{questionBox}

% Q1.2
\begin{questionBox}2{}
    
    supondo aplicável a lei de Kick?

    \begin{flalign*}
        &
            L_1 = 50\,\unit{\milli\metre}
            \to
            L_2 = 10\,\unit{\milli\metre}
            &\\&
            E=13.0\,\unit{\kilo\joule/\kilo\gram}
            &\\&
            E = C\,ln(L_1/L_2)
            % \implies &\\&
            \implies
            C = 8.08\,\unit{\kilo\joule/\kilo\gram}
            &\\[1.5ex]&
            L_{(75\to25)\unit{\milli\metre}}
            \implies &\\&
            \implies
            E 
            = C\,\ln(L_1/L_2)
            = 8.08\,\ln(75/25)
            \cong 
            \num{8.88}\,\unit{\kilo\joule\per\kilo\gram}
        &
    \end{flalign*}
    
\end{questionBox}

% Q2
\begin{questionBox}1m{}
    
    Usou-se um triturador para triturar um material cuja resistência à compressão era de 22.5\,\unit{\mega\newton/\metre^2}. O tamanho da alimentação era menor que 50\,\unit{\milli\metre}, maior que 40\,\unit{\milli\metre} e a potência necessária era 13.0\,\unit{\kilo\watt\,\second\,\kilo\gram^{-1}}. A análise por peneiração do produto produziu o seguinte resultado:

    \begin{table}[H]\centering
        \setlength\tabcolsep{3mm}        % width
        \begin{tabular}{*{4}{c}}
            
            \\\toprule
            
                \multicolumn{1}{c}{Peneiro}
                &
                \multicolumn{2}{c}{
                        \begin{tabular}{c}
                            Dimensão da abertura
                            \\
                            (mm)
                        \end{tabular}
                }
                &
                \multicolumn{1}{c}{
                    \begin{tabular}{c}
                        Percentagem do produto
                        \\
                        (\% em número)
                    \end{tabular}
                }
            
            \\\midrule
            
                1 & Passando por    & 6.00  & 100
            \\  2 & Retido em       & 4.00  & 26
            \\  3 & Retido em       & 2.00  & 18
            \\  4 & Retido em       & 0.75  & 23
            \\  5 & Retido em       & 0.50  & 8
            \\  6 & Retido em       & 0.25  & 17
            \\  7 & Retido em       & 0.125 & 3
            \\  8 & Passando por    & 0.125 & 5
            
            \\\bottomrule
            
        \end{tabular}
    \end{table}

    Qual seria a potência necessária para triturar um \unit{\kilo\gram} por segundo de um material com resistência à compressão de 45\,\unit{\mega\newton\per\metre^2} a partir de uma alimentação de menor que 45\,\unit{\milli\metre}, maior que 40\,\unit{\milli\metre} para dar um produto de tamanho médio de 0.50\,\unit{\milli\metre}?

    \begin{flalign*}
        &
            E
            = C'\,f_C\,(
                L_2^{-0.5}
                - L_1^{-0.5}
            )
            &\\&
            C'
            = E\,\left(
                f_C (
                    L_2^{-0.5}
                    - L_1^{-0.5}
                )
            \right)
            \cong
            \num{1.903}
            \implies &\\&
            \implies
            E_{(4.25\to0.5)\unit{\milli\metre}}
            = \num{1.903}*22.5\,(
                0.5^{-0.5}
                - 4.25^{-0.5}
            )
            \cong
            108.0
            \implies &\\&
            \implies
            P 
            = 1\,\unit{\frac{\kilo\gram}{\sec}}
            * 108.0\unit{\frac{\kilo\joule}{\kilo\gram}}
            = 108.0
        &
    \end{flalign*}

    \begin{flalign*}
        &
            L_1 
            = \frac{40+50}{2}
            = 45
            L_{(45)\unit{\milli\metre}}
            \implies &\\&
            \implies
            E 
            = 2\,C'\,f_C (
                L_2^{-0.5}
                - L_1^{-0.5}
            )
            \implies
            C'
            = E \left(
                2\,f_C (
                    L_2^{-0.5}
                    - L_1^{-0.5}
                )
            \right)^{-1}
            = &\\&
            = E \left(
                2*22.5\,\unit{\mega\newton\per\metre^2} (
                    0.5^{-0.5}
                    - 45^{-0.5}
                )
            \right)^{-1}
            \cong
            E\,\num{0.017564997313181}
            \num{465.65}
        &
    \end{flalign*}

    

    \paragraph*{RS} 4.68
    
\end{questionBox}

\setcounter{question}{5}

% Q6
\begin{questionBox}1{}
    
    Um moinho de bolas com 1.2\,\unit{\metre} de diâmetro está a trabalhar a 0.80\,\unit{\hertz} verificando-se que o moinho não está a trabalhar satisfatoriamente. Sugere alguma modificação nas condições de funcionamento?

    \paragraph*{RS}
    \begin{flalign*}
        &
            w_o
            = [
                w_c/2,
                w_c\,2/3
            ]   
            = [
                \sqrt{g/r}/2,
                \sqrt{g/r}\,2/3
            ]
            = &\\&
            = [
                \sqrt{\num{9.780327}/(1.2/2)}/2,
                \sqrt{\num{9.780327}/(1.2/2)}\,2/3
            ]
            = [
                \num{2.018696671122237},
                \num{2.691595561496316}
            ]
            &\\&
            \therefore
            w >> w_o
        &
    \end{flalign*}
    
\end{questionBox}

\end{document}
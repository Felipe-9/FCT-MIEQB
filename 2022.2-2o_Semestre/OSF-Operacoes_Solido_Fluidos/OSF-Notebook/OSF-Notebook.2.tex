% !TEX root = ./OSF-Notebook.2.tex
\documentclass[OSF-Notebook.tex]{subfiles}

% \tikzset{external/force remake=true} % - remake all

\graphicspath{{\subfix{./.build/figures/OSF-Notebook.2}}}
% \tikzsetexternalprefix{./.build/figures/OSF-Notebook.2/graphics/}

\begin{document}

\mymakesubfile{2}[OSF]
{Particulate Solids} % Subfile Title
{Particulate Solids} % Part Title

\begin{sectionBox}1{Introduction} % S1
  The key characteristics of an individual particle include composition, size, shape, density, and hardness.
  \paragraph*{Composition} determines properties such as \emph{density and conductivity} if the particle is \emph{uniform}. However, in many cases the particle is porous or it may consist of a continuous matrix in which small particles of a second material are distributed.
  \paragraph*{Particle size} is important because it affects properties such as the \emph{surface per unit volume and the rate at which a particle will settle in a fluid}
  \paragraph*{A Particle Shape} A particle shape may \emph{be regular}, such as spherical or cubic, or it may be \emph{irregular} like a piece of broken glass.
  \begin{description}[
    leftmargin=!,
    labelwidth=\widthof{Irregular} % Longest item
  ]
    \item[Regular] can be defined by mathematical equations
      \begin{itemize}
        \begin{multicols}{4}
          \item Sphere
          \item Cylinder
          \item Cube\dots
          \end{multicols}
      \end{itemize}

    \item[Irregular] and properties of irregular particles are usually found by \emph{comparison to} specific characteristics of \emph{a regularly shaped particle}.
  \end{description}

  \subsection*{Single particles:}
  \begin{itemize}
    \begin{multicols}{2}
      \item Shape
      \item hardness
      \item compressive resistence
      \item electrical charge
      \item (intraparticle) porosity
      \end{multicols}
  \end{itemize}

  \subsection*{Bulk solids:}
  \begin{itemize}
    \begin{multicols}{2}
      \item Particle size distribution
      \item (interparticle) porosity
      \item humidity
      \item agglomeration
      \item flowability
      \item \dots
      \end{multicols}
  \end{itemize}

  \subsection*{Solids suspensions (heterogenius mixture in a fluid, gás or
  liquid)}
  \begin{itemize}
    \begin{multicols}{2}
      \item Particle size distribution
      \item Concentration of solids
      \item viscosity of suspension
      \item flocculation
\item settleability
      \item \dots
      \end{multicols}
  \end{itemize}
\end{sectionBox}

\section{Particle Characterization} % S2

\begin{sectionBox}2{Single Particles} % S2.1
The \emph{simplest shape of a particle is the sphere}. Because of its \emph{symmetry}, the particle looks exactly the same from whatever direction it is viewed, and behaves in the same manner in fluid, regardless of its orientation. No other particle shape has this characteristic. However, \emph{perfect spheres are rarely found} and, generally, a typical particle may not be of any regular known shape. In fact, \emph{irregular-shaped particles are common}. They are typically observed in nature and regularly handled and processed in many industries. Frequently, \emph{the size of a particle of an irregular shape is defined in terms of the size of an equivalent sphere}, although the particle is represented by a sphere of a different size according to the property that is selected. Some of the important sizes of equivalent spheres are that the spheres have the same:
  \begin{itemize}
    \item Volume as the particle
    \item surface area as the particle.
    \item surface area per unit volume (i.e. specific surface area) as the particle.
    \item area as the particle when projected onto a plane perpendicular to its direction of motion.
    \item projected area as the particle, as viewed from above, when lying in its position of
    \item maximum stability such as on a microscope slide, for example.
    \item ability to pass through the same size of square aperture as the particle, such as on a screen for example.
    \item same settling velocity as the particle in a specified fluid.
  \end{itemize}
  Depending on the process of interest, the relevant particle size is typically chosen as the method to define the particle size. A measure of particle shape that is frequently used is the sphericity, \(\psi\), defined as:
  \begin{BM}
    \psi=\frac
    {surface area of sphere of same volume as particle}
    {surface area of particle}
  \end{BM}
\end{sectionBox}

\begin{sectionBox}*2{Regular shapes} % S2.1.1
  \paragraph*{Sphere}

  \begin{BM}[align*]
    \text{Volume:}\quad&
    \pi\,d^3/6
    = 4\,\pi\,r^3/3
    \\
    \text{Surface Area:}\quad&
    \pi\,d^2
    = 4\,\pi\,r^2
    \\
    \text{Projected area in a plane:}\quad&
    \pi\,d^2/2
    = \pi\,r^2
  \end{BM}
  \paragraph*{Note:} Spheres are special as its compleetely symmetrical whilist the others depend on the orientation
\end{sectionBox}

\begin{sectionBox}*2{Irregular shapes} % S2.1.2

  \begin{itemize}
    \item Cannot be identified by math equations
    \item Characteristic dimension \textit{d}
      \begin{itemize}
        \item Sphere Diameter with same volume 
          \\ \(V_{\text{particle}}=V_{\text{sphere}}\)
        \item Sphere Diameter with same Surface area
          \\ \(S_{\text{particle}}=S_{\text{sphere}}\)
        \item Sphere Diameter with same Surface per volume
          \\\(
            \frac
            {S_{\text{particle}}}
            {V_{\text{particle}}}
            = \frac
            {S_{\text{Sphere}}}
            {V_{\text{Sphere}}}
          \)
      \end{itemize}
  \end{itemize}
  \subsection*{Derivando propriedades:}
  \begin{description}[
      leftmargin=!,
      labelwidth=\widthof{Surface area} % Longest item
    ]
    \begin{multicols}{2}
      \item[Length]       \( L=d\)
      \item[Surface area] \( S=\dot{k}\,d^2\)
      \item[Volume]       \( V=\ddot{k}\,d^3\)
      \item[Mass]         \( m=\rho_S\,V=\rho_S\,\ddot{k}\,d^3\)
      \end{multicols}
  \end{description}
  \begin{itemize}
    \begin{multicols}{2}
      \item Surface factor: \(\dot{k}_{\text{sphere}}=\pi\)
      \item Volume factor: \(\ddot{k}_{\text{sphere}}=\pi/6\)
      \end{multicols}
  \end{itemize}
\end{sectionBox}

\subsection{Measurement of Particle Size} % S2.2
\begin{sectionBox}*2{Sieving \(>\qty*{50}{\micro\m}\)} % S2.3.0.1
  Sieve analysis may be carried out using a \emph{nest of sieves, each lower sieve being of smaller aperture size}. Generally, sieve series is arranged so that the ratio of aperture sizes on consecutive sieves is \(2\), \(2^{1/2}\), or \(2^{1/4}\) according to the closeness of sizing that is required. \emph{The sieves may either be hand-shaken or mounted on a vibrator}, which should be designed to give a degree of vertical movement and horizontal vibration.
  \begin{center}
    \includegraphics[width=0.6\textwidth]{sieves.png}
  \end{center}
\end{sectionBox}

\begin{sectionBox}2{Particle Size Distribution (PSD)} % S2.3
  \vspace{-3ex}
  \begin{BM}
    \text{Cumulative} \iff \text{Frequency}
  \end{BM}
  \begin{center}
    \includegraphics[width=.6\textwidth]{psdcurves.png}
  \end{center}
  Most particulate systems of practical interest consist of particles of a wide range of sizes and it is necessary to give a \emph{quantitative indication of the mean size and of the spread of sizes}. The results of a size analysis can most conveniently be represented by means of a \emph{cumulative mass fraction curve, in which the proportion of particles (\textit{x}) smaller than a certain size (\textit{d}) is plotted against that size (\textit{d})}.\\

  When the particle sizes are determined by image analysis, the results are represented by a cumulative number fraction curve, from which the mass fraction curve can be obtained

  The distribution of particle sizes can be seen more readily by \emph{plotting a size frequency curve}, in which the slope (\(\odv{x}{d}\)) of the cumulative curve is plotted against particle size (\textit{d}). The most frequently occurring size is then shown by the maximum of the curve.
\end{sectionBox}

\begin{sectionBox}*2{PSD Distribution base} % S2.3.0.1
  \begin{center}
    \vspace{1ex}
    % \setlength\tabcolsep{6mm}        % width
    \renewcommand\arraystretch{2.00} % height
    \begin{tabular}{l@{\quad} LL}
      \toprule

      \multicolumn{1}{c}{Property}
      & \multicolumn{1}{c}{Whole}
      & \multicolumn{1}{c}{Fraction}

      \\\midrule
      Number
      & n_i,d_i
      & n_i
      % 
      \\Length (\unit{\metre})
      & l=n_i\,d_i
      & l_i=\frac{n_i\,d_i}{\sum{n_j\,d_j}}
      % 
      \\Surface (\unit{\metre^2})
      & s = n_i\,\dot{k}\,d_i^2
      & s_i = \frac{n_i\,d_i^2}{\sum{n_j\,d_j^2}}
      % 
      \\Volume (\unit{\metre^3})
      & v = n_i\,\ddot{k}\,d_i^3
      & v_i = \frac{n_i\,d_i^3}{\sum{n_j\,d_j^3}}
      % 
      \\Mass (\unit{\kilo\gram})
      & x   = n_i\,\rho\,\ddot{k}\,d_i^3
      & x_i = \frac{n_i\,d_i^3}{\sum{n_j\,d_j^3}}

      \\\bottomrule
    \end{tabular}
    \vspace{2ex}
  \end{center}
  
  \subsection*{Mean diameter:}
  Can be based on different properties of solid like weight, number or volume.
  \begin{BM}
    \bar{d_{\alpha}}
    = \frac{\int{d\,\odif{\alpha}}}{\int{\odif{\alpha}}}
    = \frac{\sum{d_i\,\alpha_i}}{\sum{\alpha_i}}
    : \quad\alpha
    \begin{cases}
      x: & \text{Weight}
      \\ n: & \text{Number}
      \\ v: & \text{Volume}
      \\ s: & \text{Surface}
      \\ l: & \text{Lenght}
    \end{cases}
  \end{BM}

  \begin{BM}[flalign*]
    &
    % ------------------------------------------------ %
    %                 Weight and Number                %
    % ------------------------------------------------ %
    \text{Weight and Number}
    &\\&
    \begin{cases}
      \text{Measured in number }n\\
      \bar{d_{x}}
      = \bar{d_{v}}
      = \cfrac
      {\int{d\,\odif{x}}}
      {\int{\odif{x}}}
      = \cfrac
      {\int{d\,\odif{(n\,\rho\,\ddot{k}\,d^3)}}}
      {\int{   \odif{(n\,\rho\,\ddot{k}\,d^3)}}}
      = {\color{Emph}
        \cfrac
        {\int{n\,d^3\odif{d}}}
        {\int{n\,d^2\odif{d}}}
      }
      = \\
      = \cfrac
      {\sum{d_i\,(n_i\,\rho\,\ddot{k}\,d_i^3)}}
      {\sum{     (n_i\,\rho\,\ddot{k}\,d_i^3)}}
      = {\color{Emph}
        \cfrac
        {\sum{n_i\,d_i^4}}
        {\sum{n_i\,d_i^3}}
      }
      = \\
      \text{Measured in Weight }x\\
      = \cfrac
      {\sum{x_i\,d_i}}
      {\sum{x_i}}
      = {\color{Emph}
       \sum{x_i\,d_i}
      }
    \end{cases}
    % ------------------------------------------------ %
    %                      Surface                     %
    % ------------------------------------------------ %
    &\\[3ex]&
    \text{Surface}
    &\\&
    \begin{cases}
      \text{Measured in number }n\\
      \bar{d}_S
      = \cfrac
      {\int{d\,\odif{s}}}
      {\int{\odif{s}}}
      = \cfrac
      {\int{d\,\odif{(n\,\dot{k}\,d^2)}}}
      {\int{   \odif{(n\,\dot{k}\,d^2)}}}
      = {\color{Emph}
        \cfrac
        {\int{d^2\,n_i\odif{d}}}
        {\int{d  \,n_i\odif{d}}}
      }
      = \\
      = \cfrac
      {\sum{d_i\,(n_i\,\dot{k}\,d_i^2)}}
      {\sum{     (n_i\,\dot{k}\,d_i^2)}}
      = {\color{Emph}
        \cfrac
        {\sum{n_i\,d_i^3}}
        {\sum{n_i\,d_i^2}}
      }
      = \\
      \text{Measured in Weight }x\\
      = \cfrac
      {\sum{
          \left(
            \frac{x_i}{\rho\,\ddot{k}\,d_i^3}
      \right)\,d_i^3}}
      {\sum{\left(
            \frac{x_i}{\rho\,\ddot{k}\,d_i^3}
      \right)\,d_i^2}}
      = \cfrac
      {\sum{x_i}}
      {\sum{x_i/d_i}}
      = {\color{Emph}
        \cfrac
        {1}
        {\sum{x_i/d_i}}
      }
    \end{cases}
    % ------------------------------------------------ %
    %                      Lenght                      %
    % ------------------------------------------------ %
    &\\[3ex]&
    \text{Lenght}&\\&
    \begin{cases}
      \text{Measured in number }n\\
      \bar{d}_L
      = \cfrac
      {\int{d\,\odif{(n\,d)}}}
      {\int{   \odif{(n\,d)}}}
      = {\color{Emph}
        \cfrac
        {\int{d\,n\,\odif{d}}}
        {\int{   n\,\odif{d}}}
      }
      = \\
      = \cfrac
      {\sum{d_i\,(n_i\,d_i)}}
      {\sum{     (n_i\,d_i)}}
      = {\color{Emph}
        \cfrac
        {\sum{n_i\,d_i^2}}
        {\sum{n_i\,d_i  }}
      }
      =\\
      \text{Measured in Weight }x\\
      = \cfrac
      {\sum{\left(
            \frac{x_i}{\rho\,\ddot{k}\,d_i^3}
      \right)\,d_i^2}}
      {\sum{\left(
            \frac{x_i}{\rho\,\ddot{k}\,d_i^3}
      \right)\,d_i  }}
      = {\color{Emph}
        \cfrac
        {\sum{x_i/d_i  }}
        {\sum{x_i/d_i^2}}
      }
    \end{cases}
    &\\[3ex]&
    \text{Volume}&\\&
    \begin{cases}
      \bar{d}_V
      = \cfrac
      {\int{d\,\odif{n\,\ddot{k}\,d^3}}}
      {\int{   \odif{n\,\ddot{k}\,d^3}}}
      = {\color{Emph}
        \cfrac
        {\int{n\,d^3\odif{d}}}
        {\int{n\,d^2\odif{d}}}
      }
      = \\
      = \cfrac
      {\sum{d_i\,(n_i\,\ddot{k}\,d_i^3)}}
      {\sum{     (n_i\,\ddot{k}\,d_i^3)}}
      = \cfrac
      {\sum{n_i\,d_i^4}}
      {\sum{n_i\,d_i^3}}
      \\\text{Assuming all particles have the same size}\\
      \sum{n_i\,\ddot{k}\,\bar{d}_V^3}
      =\ddot{k}\,\bar{d}_V^3\,\sum{n_i}
      = \sum{n_i\,\ddot{k}\,d_i^3}
      \implies\\
      \implies
      \bar{d}_V
      = {\color{Emph}
        \sqrt{
          \frac
          {\sum{n_i\,d_i^3}}
          {\sum{n_i}}
        }
      }
    \end{cases}
  \end{BM}

\end{sectionBox}

\begin{exampleBox}1{} % E1
  The size analysis of a powdered material on a mass basis is represented by a straight line from 0\% mass at \qty*{1}{\micro\m} particle size to 100\% mass at \qty*{101}{\micro\m} particle size. Calculate the surface mean diameter of the particles constituting the system.
  \begin{center}
    \includegraphics[width=.4\textwidth]{example.1.png}
    % Straight ascending line going from (1,0) to (101,1), 
    % y title: Mass fraction (x)
    % x title: Size, d(\micro\metre)
  \end{center}
  \answer{}
  \begin{flalign*}
    &
    \bar{d}_s
    = \frac
    {\sum{s\,d}}
    {\sum{s}}
    = &\\[3ex]&
    = \frac
    {\sum{
        x
        \,\frac
        {\dot{k}}
        {\rho\,\ddot{k}\,d}
        \,d
    }}
    {\sum{
        x
        \,\frac
        {\dot{k}}
        {\rho\,\ddot{k}\,d}
    }}
    = \frac
    {\sum{
        x
    }}
    {\sum{
        \frac{x}{d}
    }}
    = \frac
    {1}
    {\sum{ \frac{x}{d} }}
    = \frac
    {1}
    {\int{ \frac{\odif{x}}{100\,x+1} }}
    = \frac
    {100}
    {\int{ \frac{\odif{100\,x+1}}{100\,x+1} }}
    = \frac
    {100}
    {\ln{101/1}}
    \cong \qty
    {21.667906533553168}
    {\micro\metre}
    %
    %
    %
    ; &\\[3ex]&
    s
    = n\,\dot{k}\,d^2
    = \left(
      \frac{x}{\rho\,\ddot{k}\,d^3}
    \right)
    \,\dot{k}\,d^2
    = x
    \,\frac
    {\dot{k}}
    {\rho\,\ddot{k}\,d}
    &
  \end{flalign*}
\end{exampleBox}

\end{document}

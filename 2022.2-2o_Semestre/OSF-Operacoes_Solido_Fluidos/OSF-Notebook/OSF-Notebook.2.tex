% !TEX root = ./OSF-Notebook.2.tex
\documentclass["OSF-Notebook.tex"]{subfiles}

% \tikzset{external/force remake=true} % - remake all

\graphicspath{{\subfix{./.build/figures/OSF-Notebook.2}}}
% \tikzsetexternalprefix{./.build/figures/OSF-Notebook.2/graphics/}

\begin{document}

\mymakesubfile{5}[OSF]
{Particle Size Reduction and Enlargement} % Subfile Title
{Particle Size Reduction and Enlargement} % Part Title

\begin{sectionBox}1{Introduction} % S1
  Our objective is to take a \emph{feed} spend energy applying \emph{unit operations} to it so that we get a \emph{product} with smaller particles
\end{sectionBox}

\begin{sectionBox}*2{Unit operations} % S1.0.1
  Normally a crusher or a mill is an unitary operation that receives a feed and spends energy to reduce to a product

  The whole process takes several stages of size reduction each specifying on reducing its specific feed particle sizes:
  \begin{enumerate}
    \item Coarse reduction
    \item intermediate reduction
    \item Fine reduction
  \end{enumerate}
\end{sectionBox}

\begin{sectionBox}*2{Particle properties} % S1.0.2

  \begin{itemize}
    \item \emph{Size of the feed:} energy needed is very different, caracterizes reduction in:
      \vspace{-2ex}
      \begin{itemize}
        \begin{multicols}{3}
          \item coarse 
          \item interediate 
          \item fine
        \end{multicols}
      \end{itemize}
    \item \emph{Compressive strength:} minimum compressive strength that causes solid fracture,
    \item \emph{Brittleness (fragilidade):} Poor capacity to resistst impact and vibration of load
    \\Most of inorganic non-metalic materials are brittle materials (ex: glass)
    \vspace{-2ex}
    \begin{itemize}
      \begin{multicols}{2}
        \item High compressive strength
        \item low tensile strength
      \end{multicols}
    \end{itemize}
    \item \emph{Stickiness:} adherence to equipment
      \\ causes considerable difficulty in size reduction
    \item \emph{Soapiness:} how low is the friction coefficient of the material surface
    \item \emph{Humidity:} how wet (water) is the solid
    \\between \(\qtyrange*{5}{50}{\percent}\) solids tend to cake
  \item \emph{Friability:} Measures the tendency for a solid to break when handling
    \begin{itemize}
      \item A \emph{crystaline} material will break along well defined planes
      \item Energy to break \(\propto\) Size\(^{-1}\)
    \end{itemize}
  \end{itemize}
\end{sectionBox}

\begin{sectionBox}*2{Types of forces} % S1.0.3
  Prasher suggest that four \emph{basic patterns} may be identified. 
  \begin{description}[
    leftmargin=!,
    labelwidth=\widthof{} % Longest item
    ]
    \item[Impact:] Particle concussion by a single rigid force.
    \item[Compression:] Particle disintegration by two rigid forces.
    \item[Shear:]  Produced by a fluid or by particle-particle interaction.
    \item[Attrition:] Arising from particles scrapping against one another or against a rigid surface.
  \end{description}
\end{sectionBox}

\section{Size reduction of Solids} % S2
\setcounter{subsection}{2}

\begin{sectionBox}2{Energy Requirement for Size Reduction} % S2.1
  Although its impossible to estimate the accurate amount of energy required in order to effect size reduction of a given material, a number of empirical laws have been proposed:
  \begin{center}
    \vspace{1ex}
    \setlength\tabcolsep{3mm}        % width
    % \renewcommand\arraystretch{1.25} % height
    \begin{tabular}{l @{\ } l @{} C C C}
      \toprule

        \multicolumn{1}{c}{Law}
        & \multicolumn{1}{c}{\begin{tabular}{c}
          Accurate for
          \\type of grinding
        \end{tabular}}
        & p & L_0 & L

      \\\midrule

        Kick's & Coarse
        & -1.0 
        & \qtyrange*{50}{1500}{\mm}
        & \qtyrange*{5}{50}{\mm}

        \\ Bonds & Intermediate
        & -1.5
        & \qtyrange*{2}{50}{\mm}
        & \qtyrange*{0.1}{5}{\mm}

        \\ Rittinger's & Fine
        & -2.0
        & \qtyrange*{2}{5}{\mm}
        & < \qty*{0.1}{\mm}

      \\\bottomrule
    \end{tabular}
    \vspace{2ex}
  \end{center}
  These laws may all derive from the basic differential equation:
  \begin{BM}
    \odv{E}{L} = -C\,L^p
  \end{BM}
  Which states that the \emph{energy \(\odif{E}\)} required to effect a small change in \emph{size \(\odif{L}\)} of unit mass of material is a simple \emph{power function of the size (\(-C\,L^p\))}
\end{sectionBox}

\begin{sectionBox}*2{Rittinger's Law} % S2.1.0.1
  \begin{BM}
    E
    = C\,\adif{\left( L^{-1} \right)}
    = k_r\,f_c\,\adif{\left( L^{-1} \right)}
  \end{BM}

  Since \emph{surface} of unit mass of material \emph{\(\propto L^{-1}\)}, the interpretation of this law is that \emph{\(E\propto\) increase of surface area}
  
  Rittinger's law is applicable mainly to that part of the process where a new surface is being created and holds most accurately for \emph{fine grinding (\(L_0=\qtyrange*{2}{5}{\mm}\land L<\qty*{0.1}{\mm}\))} where increase in surface per unit mass of material is large.

  \begin{flalign*}
    &
      \odv{E}{L} = -C\,L^p \land p = -2.0
      \implies &\\&
      \implies
      \int{\odif{E}}
      = E
      = \int{-C\,L^{-2}\,\odif{L}}
      = C\,\adif{( L^{-1} )}
      = K_R\,f_c\,\adif{( L^{-1} )}
    &
  \end{flalign*}
\end{sectionBox}

\begin{sectionBox}*2{Kick's Law} % S2.1.0.2
  \begin{BM}
    E 
    = -C\,\adif{\ln{ L }}
    = -K_K\,f_C\,\adif{\ln{ L }}
  \end{BM}

  This supposes that \emph{energy required is directly related} to the \emph{reduction ratio \(L_0/L\)}

  Kick's law more closely relates to the energy required to effect elatic deformation before fracure occurs, and is \emph{more accurate} than Rittinger's law for \emph{coarse crushing} where the amount of surface produced is considerably less.

  \begin{flalign*}
    &
      \odv{E}{L} = -C\,L^p \land p = -1.0
      \implies &\\&
      \implies
      \int{\odif{E}} = E
      = \int{-C\,L^{-1}\,\odif{L}}
      = -C\,\adif{\ln{L}}
      = -K_K\,f_c\,\adif{\ln{L}}
    &
  \end{flalign*}
\end{sectionBox}

\begin{sectionBox}*2{Bond's Law} % S2.1.0.3
  \begin{BM}
    E
    = 2\,C\,\adif{(1/\sqrt{L})}
    = E_i
    \,\sqrt{ \frac{100}{L} }
    \,\left(1-\frac{1}{\sqrt{q}}\right)
    ;\quad q = \frac{L_0}{L}
  \end{BM}

  Bond calls \emph{\(E_i\) the work index}, and express it as the amoun t of \emph{energy required to reduce} the unit mass of material from size \emph{\(\infty\to\qty*{100}{\micro\metre}\)}, that is \(q=\infty\). the size of the material is take as the size of the square hole through 80\% if the material will pass.

  \begin{flalign*}
    &
      \odv{E}{L} = -C\,L^p \land p=-1.5
      \implies &\\&
      \implies
      \int{\odif{E}} = E
      = \int{-C\,L^{-1.5}\,\odif{L}}
      = 2\,C
      \,\adif{\left(
        \frac{1}{\sqrt{L}}
      \right)}
      ; \quad 
      C=5\,E_i
      \land
      q  = \frac{L_0}{L}
      \implies &\\&
      \implies
      E
      = 2*5\,E_i\,\left(
        \frac{1}{\sqrt{L}}
        -\frac{1}{\sqrt{q\,L}}
      \right)
      = E_i
      \,\sqrt{\frac{100}{L}}
      \left(1-\frac{1}{\sqrt{q}}\right)
    &
  \end{flalign*}
\end{sectionBox}

\begin{exampleBox}1{} % E1
  A material is crushed in a blake jaw cruser such that the average size of particle is reduced from \qtyrange{50}{10}{\mm} with the consumption of energy of \qty*{13.0}{\kW/(\kg/\s)}. What would be the consumption of energy needed to crush the same material of an average size of \qty*{75}{\mm} to an average size of \qty{25}{\mm}

  which of these results would be regarded as more reliable and why?
  \answer{}
  Kick's law for the average sizes being grinded are common for fine reduction which's kick's law is more accurate
\end{exampleBox}
\begin{exampleBox}2{} % E1.1
  Assuming Rittinger's law applies?
  \answer{}
  \begin{flalign*}
    &
      E
      = K_R\,f_c\,\adif{L^{-1}}
      = &\\&
      = 162.5*(
        25^{-1}-75^{-1}
      )
      \,\unit{\kW/(\kg/\s)}
      = \qty
      {4.333333333333333}
      {{\kW/(\kg/\s)}}
      %
      %
      %
      ; &\\[3ex]&
      K_R\,f_c
      = \frac{E}{L^{-1}-L_0^{-1}}
      = \frac{13.0}{10^{-1}-50^{-1}}
      \,\unit{\frac{\kW/(\kg/\s)}{\mm}}
      = 162.5
      \,\unit{\frac{\kW/(\kg/\s)}{\mm}}
    &
  \end{flalign*}
\end{exampleBox}
\begin{exampleBox}2{} % E1.2
  Assuming Kick's law applies?
  \answer{}
  \begin{flalign*}
    &
      E
      =-K_K\,f_c\,\adif{\ln{L}}
      \cong &\\&
      \cong \num{-8.077354149274954}
      \,\ln{25/75}
      \,\unit{\kW/(\kg/\s)}
      \cong\qty
      {8.873880528317809}
      {\kW/(\kg/\s)}
      %
      %
      %
      ; &\\[3ex]&
      -K_K\,f_c
      = \frac{E}{\ln{L/L_0}}
      = \frac{13.0}{\ln{10/50}}
      \,\unit{\kW/(\kg/\s)}
      \cong \qty
      {-8.077354149274954}
      {\kW/(\kg/\s)}
    &
  \end{flalign*}
\end{exampleBox}

\begin{sectionBox}1{Types of Crushing Equipment} % S3
  \begin{center}
    \vspace{1ex}
    \setlength\tabcolsep{2mm}        % width
    % \renewcommand\arraystretch{1.25} % height
    \begin{tabular}{*3{l}}
      \toprule

      Coarse Crushers
      & Intermediate Crushers
      & Fine Crushers

      \\\midrule
      
      % Coarse Crushers
      \begin{tabular}{l}
        \emph{Stag Jaw crusher}  
        \\ Dodge jaw crusher 
        \\ Gyratory crusher  
        \\ \dots             
      \end{tabular}
      & 

      % Intermediate Crushers
      \begin{tabular}{l}
          \emph{Crushing rolls} 
        \\ Disc crusher        
        \\ Edge runner mill    
        \\ Hammer mill         
        \\ Single roll crusher 
        \\ Pin mill            
        \\ symons disc crusher 
        \\ \dots               
      \end{tabular}
      & 

      % Fine Crushers
      \begin{tabular}{l}
          Buhrstone mill
        \\ Roller mill
        \\ NEI pendulum mill
        \\ Griffin mill
        \\ Ring roller mill (lopulco)
        \\ \emph{Ball mill}
        \\ Tube mill
        \\ Hardinge mill
        \\ Babcock mill 
        \\ \dots
      \end{tabular}


      \\\bottomrule
    \end{tabular}
    \vspace{2ex}
  \end{center}
\end{sectionBox}


\subsection{Coarse Crushers} % S3.1

\begin{sectionBox}*2{Jaw Crushers} % S3.1.0.1
  The Stag jaw crusher has a fixed jaw and a moving jaw pivoted at the
top with the crushing faces formed of manganese steel.
  \begin{center}
    \includegraphics[width=0.6\textwidth]{jawcrusher.png}
  \end{center}
\end{sectionBox}

\subsection{Intermediate Crushers} % S3.2

\begin{sectionBox}*2{Crushing rolls} % S3.2.0.1
  Two rolls, one in adjustable bearings, rotate in opposite directions, and the clearance between them can be adjusted according to the size of feed and the required size of product. \emph{Main forces: compressive/attrition}
  \begin{center}
    \includegraphics[width=.6\textwidth]{crushingrolls.png}
  \end{center}
  
  \subsection*{Nip angle}
  \begin{BM}
    \cos{\alpha/2}
    = \frac
    {r_{roll}+b/2}
    {r_{roll}+r_0}
    \leq31^\circ
  \end{BM}
  \begin{description}
    \item[\(r_{roll}\):] Roll radius
    \item[\(r_0\):] Feed particle size
    \item[\(b\):] Distance between rolls
    \item[\(\alpha\leq31^\circ\):] Maximum value for nip angle
    \item \(\tan{\alpha}=\mu\) The coefficient of friction
  \end{description}
  \begin{center}
    \includegraphics[width=0.4\textwidth]{nipangle.png}
  \end{center}
\end{sectionBox}

\subsection{Fine Crushers} % S3.3

\begin{sectionBox}*2{Ball Mill} % S3.3.0.1
  In its simplest form, \emph{the ball mill consists of a rotating hollow cylinder, partially filled with balls}, with its axis either horizontal or at a small angle to the horizontal. The material to be ground may be fed in throughja hollow trunnion at one end, and the product leaves through a similar trunnion at the other end. The gutlet is normally covered with a coarse screen to prevent the escape of the balls.
  \begin{center}
    \includegraphics[width=.6\textwidth]{ballmill.png}
  \end{center}
During grinding, the \emph{balls wear and are constantly replaced by new ones} so that the mill contains balls of various ages, and, hence, of various sizes. This is advantageous because the large balls deal effectively with the feed, and the small ones are responsible for giving a fine product.

  \subsection*{Rotation speed}
  A ball mill has a \emph{critical rotation} speed (\([w_c]=\unit{\radian/\s}\)) that must be avoided. At the critical point, the ball (with mass \textit{m}) is subject to a centrifugal force (\(m\,u/r^2\)) equal to the gravitational force (\(m\,g\))
  \begin{BM}
    w_c=g/r
  \end{BM}
  The \emph{optimal rotation} speed (\([w_o]=\unit{\radian/\s}\)) is choosen below \(w_c\) for optimal efficiency
  \begin{BM}
    w_o\sim\alpha\,w_c;\quad 
    \alpha\in\myrange{0.50,0.75}
  \end{BM}
\end{sectionBox}

\begin{exampleBox}1{} % E3
  A ball mill, \qty*{1.2}{\m} in diameter, is operated at \qty*{0.80}{\hertz}, and it is found that the mill is not working properly. Should any modification in the conditions of operation be suggested
  \answer{}
  \begin{flalign*}
    &
      w_o \in\myrange{0.50\,w_c,0.75\,w_c}
      \cong &\\&
      \cong\myrange{
        \num{0.321285553812258},
        \num{0.481928330718387}
      }\,\unit{\hertz}
      %
      %
      %
      ; &\\[3ex]&
      w_c
      = \sqrt{\frac{g}{r}}
      \cong \sqrt{\frac{\num{ 9.780327 }}{0.6}}
      \cong \qty
      {4.037393342244473}
      {\radian/\s}
      \cong \qty
      {0.642571107624516}
      {\hertz}
    &
  \end{flalign*}
  rotation speed should be reduced to optimal \(w_o\) values
\end{exampleBox}

\end{document}

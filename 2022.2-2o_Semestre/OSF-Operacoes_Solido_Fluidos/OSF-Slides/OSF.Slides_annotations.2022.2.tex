% !TEX root = ./OSF-Slides_annotations.2022.2.tex
\documentclass["./OSF-Slides_annotations.tex"]{subfiles}

\graphicspath{{\subfix{./.build/figures/OSF-Slides_annotations.2022.2/}}}
% \tikzset{external/force remake=true} % - remake all

\begin{document}

\mymakesubfile{2}
[OSF]
{Anotações Slide 2} % Subfile Title
{Slide} % Part Title

\begin{enumerate}[label ={\arabic{part}.\arabic{enumi}}]
    \item Properties of single particles: size, shape
    \item Particle size distribution: Mean diameter
    \item Methos to measure particle size: sieving, elutriation
\end{enumerate}

\begin{sectionBox}1{Própriedades Fisicoquímicas de particulas sólidas} % S1
    
    \setlength\tabcolsep{2mm}        % width
    \renewcommand\arraystretch{1.25} % height

    \begin{table}[H]\centering
        \begin{tabular}{c | c | c}
            
            \toprule
            
                \multicolumn{1}{c |}{
                    \begin{tabular}{c}
                        Particulas\\[-.5em]Individuais
                    \end{tabular}
                }
            &   \multicolumn{1}{c |}{
                    \begin{tabular}{c}
                        Aglomerados\\[-.5em]Sólidos
                    \end{tabular}
                }
            &   \multicolumn{1}{c}{
                    \begin{tabular}{c}
                        Suspensões\\[-.5em]Sólidas
                    \end{tabular}
                }
            
            \\[1.5em]
                
                \begin{tabular}{c}
                    Tamanho
                    \\Forma
                    \\Dureza
                    \\Compressividade
                \end{tabular}
            &   \begin{tabular}{c}
                    Distribuição\\[-.5em]do tamanho\\[-.5em]de particulas
                    \\ Porosidade\\[-.5em](dentre particulas)
                    \\ Humidade
                    \\ Aglomeração
                    \\ Flutuabilidade
                \end{tabular}
            &   \begin{tabular}{c}
                    Distribuição\\[-.5em]do tamanho\\[-.5em]de particulas
                    \\ Concentração\\[-.5em]dos sólidos
                    \\ Viscosidade\\[-.5em]da suspenção
                    \\ Floculação
                    \\ Liquidabilidade
                \end{tabular}
            
            \\\bottomrule
            
        \end{tabular}
    \end{table}
    
\end{sectionBox}

\section*{Porpriedades de particulas individuais}

\begin{sectionBox}2m{Forma} % S1.2
    
    Divididas em Forma
    \begin{itemize}
        \begin{multicols}{3}
            \item Regular
            \item Complexa
            \item Irregular
        \end{multicols}
    \end{itemize}

    \begin{sectionBox}3{Forma Regular} % S1.2 (i)
        
        Possuem geometrias bem definidas por equações matemáticas

        \paragraph*{Exemplos}
        \begin{itemize}
            \item Esferas (simetria independe da orientação)
            \begin{multicols}{2}
                \item Cubos
                \item Cilindros
            \end{multicols}
        \end{itemize}
        
    \end{sectionBox}

    \begin{sectionBox}3{Complexa} % S1.2 (ii)
        
        Bastate usadas em colunas compactas em engenharia química e biológica,
        Geralmente desenhadas para maximizar area de superficie por volume.

        \paragraph*{Exemplos}\ 
        \begin{multicols}{4}\centering
        %    \begin{center}
                \includegraphics[width=.2\textwidth]{Screenshot 2022-10-21 at 21.02.53.png}
                \includegraphics[width=.2\textwidth]{Screenshot 2022-10-21 at 21.03.05.png}
                \includegraphics[width=.2\textwidth]{Screenshot 2022-10-21 at 21.03.14.png}
                \includegraphics[width=.2\textwidth]{Screenshot 2022-10-21 at 21.03.22.png}
                \includegraphics[width=.2\textwidth]{Screenshot 2022-10-21 at 21.03.32.png}
        %    \end{center}
        \end{multicols}
        
    \end{sectionBox}
    
    \begin{sectionBox}3{} % S1.2 (iii)
        
        Não podem ser definidas com equações matemáticas,
        tamanho indefinido que varia com orientação.
        
        Recebe aproximação com uma esfera com diametro baseado em algum dos seguintes critérios, esse diametro é o valor característico da particula
        \begin{itemize}
            \item Mesmo volume
            \item Mesma area de superfície
            \item Mesma proporção entre superfície por volume
        \end{itemize}

        \paragraph*{Dimensões derivadas} podem ser adiquiridas a partir do valor característico
        \begin{itemize}
            \begin{multicols}{2}
                \item Largura: \(L=d\)
                \item Area de Superfície \(S = k'\,d^2\)
                \item Volume \(vol = k''\,d^3\)
                \item Massa \(m = \rho_s\,vol\)
            \end{multicols}
        \end{itemize}

        \paragraph*{onde}
        \begin{itemize}
            \begin{multicols}{2}
                \item \(k'\) Fator se uperfície\\(\(\pi/2\) para esferas)
                \item \(k''\) Fator de volume\\(\(\pi/6\) para esferas)
            \end{multicols}
        \end{itemize}
        
    \end{sectionBox}

\end{sectionBox}

\begin{sectionBox}1{Distribuição do Tamanho de partículas} % S2
    
    \begin{multicols}{2}\centering
        \includegraphics[width=.4\textwidth]{Screenshot 2022-10-22 at 00.39.50.png}
        \includegraphics[width=.4\textwidth]{Screenshot 2022-10-22 at 00.42.35.png}
    \end{multicols}

    As curvas de distribuição podem ser montadas levando em conta diferentes tipos de medições
    \begin{itemize}
        \begin{multicols}{3}
           \item[n --] numérica
           \item[L --] largura
           \item[S --] Superfície
           \item[vol --] volume
           \item[x --] peso
        \end{multicols}
    \end{itemize}

    \begin{center}
        \includegraphics[width=.4\textwidth]{
            Screenshot 2022-10-21 at 22.33.53.png
        }
    \end{center}

\end{sectionBox}

\begin{sectionBox}2{Tamanhos médios baseados em volume} % S2.1
    
    \begin{BM}
        x_1 = n_1\,k'\,d_1^3\,\rho_{s}
        \implies \\
        % \implies
        \begin{aligned}
            \int_0^1\odif{x} &= k'\,\rho_{s}\,\int d^3\,\odif{n}
            \\
            \odif{x} &= k'\,\rho_{s}\,d^3\,\odif{n}
        \end{aligned}
        \qquad
        \sum x_1 = k'\,\rho_{s}\,\sum n_1\,d_1^3 = 1
    \end{BM}

    \begin{itemize}
        \begin{multicols}{2}
            \item[\(x_1\):] Fração Peso das partículas
            \item[\(n_1\):] Quantidade de partículas
            \item[\(k'\):] Constante sensivel ao formato
            \item[\(d_1\):] Diametro das partículas
            \item[\(\rho_S\):] Densidade do material
        \end{multicols}
    \end{itemize}

    \begin{sectionBox}3{Diametro médio (Volume)} % S2.3
        
        \begin{BM}
            d_v 
            = \frac{\int_0^1{d\,\odif{x}}}{\int_0^1{\odif{x}}} 
            = \int_0^1{d\,\odif{x}}
            \qquad
            = \frac{\sum(d_1\,x_1)}{\sum{x_1}}
            = \sum(d_1\,x_1)
        \end{BM}

        \begin{flalign*}
            &
                d_v
                = \frac{\sum(d_1\,x_1)}{\sum{x_1}}
                = \frac{
                    \rho_s\,k'\,\sum{(n_1\,d_1^4)}
                }{
                    \rho_s\,k'\,\sum{(n_1\,d_1^3)}
                }
                = \frac{
                    \sum{(n_1\,d_1^4)}
                }{
                    \sum{(n_1\,d_1^3)}
                }
            &
        \end{flalign*}

        \paragraph*{Diametro volumétrico médio}
        \begin{BM}
            d'_v = \sqrt[-3]{\sum{(x_1/d_1^3)}}
        \end{BM}

        \begin{flalign*}
            &
                k'\,d'_v\,\sum n_1 
                = \sum(k'\,n_1\,d_1^3)
                % \implies &\\&
                \implies
                d'_v 
                = \sqrt[3]{
                    \frac{\sum{x_1}}{
                        \sum{(x_1/d_1^3)}
                    }
                }
                = \sqrt[-3]{
                    \sum{(x_1/d_1^3)}
                }
            &
        \end{flalign*}
        
    \end{sectionBox}
    
\end{sectionBox}

\begin{sectionBox}2{Tamanhos baseados em superfícies} % S1.2

    \begin{BM}
        d'_s
        = \sqrt{
            \frac{
                \sum\left(n_1\,d_1^2\right)
            }{
                \sum n_1
            }
        }
        = \sqrt{
            \frac{
                \sum\left(x_1/d_1\right)
            }{
                \sum\left(x_1/d_1^3\right)
            }
        }
    \end{BM}

    \begin{flalign*}
        &
            d'_s
            = \sqrt{
                \frac{d_s}{
                    k_2\,\sum{n_1}
                }
            }
            = \sqrt{
                \frac{d_s}{
                    k_2\,\sum{n_1}
                }
            }
            = \sqrt{
                \cfrac{
                    \left(
                        \cfrac{
                            \sum{(n_1\,d_1\,s_1)}
                        }{
                            \sum{n_1\,s_1}
                        }
                    \right)
                }{
                    k_2\,\sum{n_1}
                }
            }
            = &\\&
            = \sqrt{
                \cfrac{
                    \sum{(n_1\,d_1\,(k_2\,d_1^2))}
                }{
                    k_2\,\sum{n_1}
                    \,\sum{(n_1\,(k_2\,d_1^2))}
                }
            }
            = \sqrt{
                \cfrac{
                    \sum{(n_1\,d_1^3)}
                }{
                    k_2\,\sum{n_1}
                    \,\sum{n_1\,d_1^2}
                }
            }
            = \sqrt{
                \cfrac{
                    \sum{x_1}
                }{
                    k_2
                    \,\sum{n_1}
                    \,\sum{(x_1/d_1)}
                }
            }
            % = \sqrt{
            %     \frac{
            %         \sum{n_1\,s_1}
            %     }{
            %         k_2\,\sum{n_1}
            %     }
            % }
            % = \sqrt{
            %     \frac{
            %         \sum{k_2\,n_1\,d_1^2}
            %     }{
            %         k_2\,\sum{n_1}
            %     }
            % }
        &
    \end{flalign*}

    \begin{BM}
        d_s 
        = \frac{\sum{n_1\,d_1\,s_1}}{\sum{n_1\,s_1}}
        = \frac{\sum{n_1\,d_1\,(k''\,d_1^2)}}{\sum{n_1\,d_1\,(k''\,d_1^2)}}
        = \frac{\sum{n_1\,d_1^3}}{\sum{n_1\,d_1^2}}
    \end{BM}
    
\end{sectionBox}

\end{document}

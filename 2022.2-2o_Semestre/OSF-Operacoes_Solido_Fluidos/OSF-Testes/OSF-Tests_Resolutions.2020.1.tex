% !TEX root = ./OSF-Tests_Resolutions.2020.1.tex
\providecommand\mainfilename{"./OSF-Tests_Resolutions.tex"}
\providecommand \subfilename{}
\renewcommand   \subfilename{"./OSF-Tests_Resolutions.2020.1.tex"}
\documentclass[\mainfilename]{subfiles}

% \tikzset{external/force remake=true} % - remake all

\begin{document}

% \graphicspath{{\subfix{./.build/figures/OSF-Tests_Resolutions.2020.1}}}
% \tikzsetexternalprefix{./.build/figures/OSF-Tests_Resolutions.2020.1/graphics/}

\mymakesubfile{1}
[OSF]
{Teste 2021.1} % Subfile Title
{Teste 2021.1} % Part Title

\begin{questionBox}1{} % Q1
    
    Uma amostra de sólidos divididos com massa específica de 2.89\,\unit{\gram/\centi\metre^3} e tamanho de partícula inferior ou igual a 73\,\unit{\milli\metre} foi classificada por elutriação multi-estágio com 4 colunas em série. Uma suspensão com 1370\,\unit{\gram} de sólido e água (viscosidade = 1\,\unit{\centi P} e massa específica de 1\,\unit{\gram/\centi\metre}) atravessou o sistema com os resultados indicados na tabela:

    \begin{table}[H]\centering
        \begin{tabular}{c c r r}
            

            \setlength\tabcolsep{1mm}        % width
            \renewcommand\arraystretch{1.25} % height

            \\\toprule
            
                \multicolumn{1}{c}{Coluna}
            &   \multicolumn{1}{c}{
                \begin{tabular}{c}
                    Tamanho de\\[-1ex]sendimentados (\unit{\micro\metre})
                \end{tabular}
            }
            &   \multicolumn{1}{c}{
                \begin{tabular}{c}
                    Massa de\\[-1ex]sendimentados (\unit{\gram})
                \end{tabular}
            }
            &   \multicolumn{1}{c}{
                \begin{tabular}{c}
                    Velocidade da\\[-1ex]Suspenção (\unit{\gram})
                \end{tabular}
            }
            
            \\\midrule
            
                
            
            \\\bottomrule
            
        \end{tabular}
    \end{table}
    
\end{questionBox}

\begin{questionBox}2{} % Q1.2
    
    Complete a tabela com as velocidades de escoamento (\unit{\metre/\second}) em cada coluna. [4 val]
    
\end{questionBox}

\end{document}
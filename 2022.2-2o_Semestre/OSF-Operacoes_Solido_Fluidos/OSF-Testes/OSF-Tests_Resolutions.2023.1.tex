% !TEX root = ./OSF-Tests_Resolutions.2023.1.tex
\providecommand\mainfilename{"./OSF-Tests_Resolutions.tex"}
\providecommand \subfilename{}
\renewcommand   \subfilename{"./OSF-Tests_Resolutions.2023.1.tex"}
\documentclass[\mainfilename]{subfiles}

% \tikzset{external/force remake=true} % - remake all

\begin{document}

% \graphicspath{{\subfix{./.build/figures/OSF-Tests_Resolutions.2023.1}}}
% \tikzsetexternalprefix{./.build/figures/OSF-Tests_Resolutions.2023.1/graphics/}

\mymakesubfile{1}
[OSF]
{Teste 2023 Resolução} % Subfile Title
{Teste 2023 Resolução} % Part Title

\begin{questionBox}1{ % Q1
    Um moinho de rolos com 1\,\unit{\metre} de diâmetro e 0.4\,\unit{\metre} de comprimento gasta 16.3\,\unit{\kilo\joule/\kilo\gram} para reduzir o tamanho dum sólido com tamanho entre 15 e 35\,\unit{\milli\metre} até um tamanho final médio de 4.7\,\unit{\milli\metre}. O sólido tem massa específica de 1870\,\unit{\kilo\gram/\metre^3} e resistência ao esmagamento de 31\,\unit{\mega\newton/\metre^2}. O moinho opera a uma velocidade angular de 18\,\unit{\rpm} e sabe-se que o caudal mássico real é 15\% do teórico.
} % Q1
    \begin{questionBox}2{ % Q1.1
        Que distância entre rolos escolheria por forma a que o moinho opere adequadamente?
    } % Q1.1
        \answer{}
        \begin{flalign*}
            &
                % r_1: raio do moinho
                % r_2: raio maximo do sólido
                b:&\\&
                \cos(\beta/2)
                = \frac{r_1+b/2}{r_1+r_2}
                \implies &\\&
                \implies
                b
                = 
                2\,\cos(\beta/2)\,(
                    r_1+r_2
                )
                -2\,r_1
                = &\\&
                = 
                2\,\cos(31/2)\,(
                    (1/2)
                    +(35\E-3/2)
                )
                -2\,(1/2)
                \cong &\\&
                \cong
                \qty{-2.642480929075}{\milli\metre}
            &
        \end{flalign*}
    \end{questionBox}
    \begin{questionBox}2{ % Q1.2
        Calcule a potência do moinho nas condições descritas.
    } % Q1.2
    \end{questionBox}
    \begin{questionBox}2{ % Q1.3
        Qual seria a potência de operação se a distribuição de tamanhos em base mássica fosse: 
        \begin{itemize}
            \begin{multicols}{2}
                \item 10\% - 3\,\unit{\milli\metre}
                \item 20\% - 4\,\unit{\milli\metre}
                \item 60\% - 5\,\unit{\milli\metre}
                \item 10\% - 6\,\unit{\milli\metre}
            \end{multicols}
        \end{itemize}
    } % Q1.3
    \end{questionBox}
\end{questionBox}

% \begin{questionBox}1{} % Q1

    
%     Redução de tamanhos
%     Tamanho médio: 5\,\unit{\micro\metre}
%     Reduzido a tamanho inferior a 100 mícron.
%     Análize geométrica segue uma reta que vai de 0\% em numero de dimenção de 0 mícrom a 100\% em número dimensão de particula a 100 mícron.
    
% \end{questionBox}

% \begin{questionBox}2{} % Q1.1
    
%     A distribuição de tamanhos do produto referida é de referencia ou cumulativa? Justifique
%     % [1 val]

%     \paragraph*{RS:}Pela caracterização de uma reta de 0\% a 100\% no eixo horizontal e 0 a 100 mícron no eixo vertical o gráfico aponta uma distribuição culmulativa.
%     \\

%     \begin{itemize}
%         \begin{multicols}{2}
%            \item Referencia: Diferenciada, comportamento de histograma
%            \item Culmulativa: Integrada, injetiva, frequencia de 0 a 1
%         \end{multicols}
%     \end{itemize}
    
% \end{questionBox}

\begin{questionBox}2{} % Q1.2
    
    Calcule o diametro médio em base mássica das partículas
    % 1cq 10\E-6 pa

    \begin{BM}
        x_1 = n_1\,k'\,d_1^3\,\rho_{s}
    \end{BM}

    \begin{flalign*}
        &
            \odv{d_1}{n_1}
            = \frac{1-0}{100-0}
            = 1\E-2
            &\\&
            \int_0^1\odif{x_1}
            = \adif{n_1\,k'\,d_1^3\,\rho_{s}}
            &\\&
            \bar{d}_m
            = \frac{
                \int_0^1{d^3\,\odif{n}}
            }{
                \int_0^1{d^2\,\odif{n}}
            }
            = \frac{
                \int_0^1{
                    \left(
                        \frac{x}{n\,k'\,\rho_s}
                    \right)\,\odif{n}
                }
            }{
                \int_0^1{
                    \left(
                        \frac{x}{n\,k'\,\rho_s}
                    \right)^{2/3}
                    \,\odif{n}
                }
            }
            = \cfrac{
                \int_0^1{
                    \left(
                        \frac{x}{n\,k'\,\rho_s}
                    \right)\,\odif{n}
                }
            }{
                \int_0^1{
                    \left(
                        \frac{x}{n\,k'\,\rho_s}
                    \right)^{2/3}
                    \,\odif{n}
                }
            }
            &\\&
            d = 
            % 
            % d_m
            % = \frac{
            %     \int_0^1{d_m\,\odif{x}}
            % }{
            %     \int_0^1{\odif{x}}
            % }
        &
    \end{flalign*}
    
\end{questionBox}

\begin{questionBox}2{} % Q1.3
    
    A energia específica desta operação é \(E=78.0\,\unit{\kilo\gram}\). A resistência ao esmagamento é 33.0\,\unit{\mega\pascal}. Calcule a constante que caracteriza o equipamento de redução.

    \paragraph*{RS}
    
\end{questionBox}

% Dica: Força de atrito a area q intereça é projetada num plano a area das contas é da particula representada num plano

\begin{questionBox}1{} % Q2
    
    \begin{BM}
        V_p = \frac{\pi}{4}\left(
            d^2-d_i^2
        \right)
        h
    \end{BM}
    
\end{questionBox}

\begin{questionBox}2{} % Q2.1
    
    Apesar do anel de Raschig ter uma forma regular, ela não é Simétrica como a esfera. Que parâmetro propõe para avaliar a assimetria? Calcule-o e interprete.

    \paragraph*{RS:} Proporção dentre area da projeçã de superfície quando vertical e quando horizontal

    \begin{flalign*}
        &
            \frac{
                \pi\,(d^2-d_i^2)
            }{
                d*h
            }
            = \frac{
                \pi\,((6\E-3)^2-(4.8\E-3)^2)
            }{
                6\E-3*6\E-3
            } 
            = \frac{
                \pi\,((6)^2-(4.8)^2)
            }{
                6*6
            } 
            \cong
            \num
            [round-precision=4]
            {1.130973355292326}
        &
    \end{flalign*}
    
\end{questionBox}

\end{document}
% !TEX root = ./OSF-Assignments_Resolução.1.tex
\providecommand\mainfilename{"./OSF-Assignments_Resolução.tex"}
\providecommand \subfilename{}
\renewcommand   \subfilename{"./OSF-Assignments_Resolução.1.tex"}
\documentclass[\mainfilename]{subfiles}

% \graphicspath{{\subfix{../images/}}}
% \tikzset{external/force remake=true} % - remake all

\begin{document}

\mymakesubfile{1}
[OSF]
{Assignment A: Distribuição de tamanhos de sólidos -- Resolução} % Subfile Title
{Assignment A: Distribuição de tamanhos de sólidos} % Part Title

\begin{questionBox}1{} % Q1
    
    Uma amostra dum sólido finamente moído foi classificado num sistema de peneiração usando a série padrão I.M.M. (Institution of Mining and Metallurgy, UK) tendo produzido o seguinte resultado:
    
    \begin{table}[H]\centering
        \begin{tabular}{c l r r}
            
            \\\toprule
            
                \multicolumn{1}{c}{Peneiro}
            &   \multicolumn{2}{c}{
                \begin{tabular}{l}
                    Abertura
                    \\
                    nominal (\unit{\micro\metre})
                \end{tabular}
                }
            &   \multicolumn{1}{c}{
                    \begin{tabular}{c}
                        Número de
                        \\
                        Partículas
                    \end{tabular}
                }
            
            \\\midrule
            
                 1 & Passando por & 2540 & 13715
            \\   2 & Retido em    & 1574 & 18
            \\   3 & Retido em    & 1270 & 74
            \\   4 & Retido em    & 1056 & 629
            \\   5 & Retido em    &  792 & 2597
            \\   6 & Retido em    &  635 & 3324
            \\   7 & Retido em    &  422 & 2131
            \\   8 & Retido em    &  347 & 1443
            \\   9 & Retido em    &  254 & 822
            \\  10 & Retido em    &  211 & 580
            \\  11 & Retido em    &  180 & 417
            \\  12 & Retido em    &  157 & 368
            \\  13 & Retido em    &  139 & 329
            \\  14 & Retido em    &  127 & 310
            \\  15 & Retido em    &  107 & 238
            \\  16 & Retido em    &   84 & 172
            \\  17 & Passando por &   63 & 122
            
            \\\bottomrule
            
        \end{tabular}
    \end{table}

\end{questionBox}

\section*{Determine e represente graficamente:}

\begin{questionBox}2{} % Q1.1
    
    Curva de distribuição de frequência e cumulativa em base número
    
\end{questionBox}

\begin{questionBox}2{} % Q1.2
    
    Curva de distribuição de frequência e cumulativa em base comprimento 
    
\end{questionBox}

\begin{questionBox}2{} % Q1.3
    
    Curva de distribuição de frequência e cumulativa em base superficial.
    
\end{questionBox}

\begin{questionBox}2{} % Q1.4
    
    Curva de distribuição de frequência e cumulativa em base volúmica.
    
\end{questionBox}

\begin{questionBox}2{} % Q1.5
    
    Curva de distribuição de frequência e cumulativa em base mássica.
    
\end{questionBox}

\section*{Calcule o tamanho médio e desvio padrão:}

\begin{questionBox}2{} % Q1.6
    
    Determine os diâmetros médios em base número, comprimento, superfície, volume e peso
    
\end{questionBox}

\begin{questionBox}2{} % Q1.7
    
    Determine o desvio padrão em base número, comprimento, superfície, volume e peso
    
\end{questionBox}

\end{document}
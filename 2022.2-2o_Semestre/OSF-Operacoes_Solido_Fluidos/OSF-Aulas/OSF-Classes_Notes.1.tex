% !TEX root = ./OSF-Aulas_Anotações.1.tex
\providecommand\mainfilename{"./OSF-Aulas_Anotações.tex"}
\providecommand \subfilename{}
\renewcommand   \subfilename{"./OSF-Aulas_Anotações.1.tex"}
\documentclass[\mainfilename]{subfiles}

% \graphicspath{{\subfix{../images/}}}
% \tikzset{external/force remake=true} % - remake all

\begin{document}

\mymakesubfile{1}
[OSF]
{Aula}
{Aula}  

\begin{sectionBox}1{Energia com base na redução de tamanho}
    
    % \paragraph{Classificação de tamanho}
    % \begin{table}[H]\centering
    %     \begin{tabular}{c c c}
            
    %         \\\toprule
            
    %             \multicolumn{1}{c}{Fine}
    %         &   \multicolumn{1}{c}{Intermediate}
    %         &   \multicolumn{1}{c}{Coarse}
            
    %         \\\midrule
            
                
            
    %         \\\bottomrule
            
    %     \end{tabular}
    % \end{table}

    \begin{BM}
        \odv{E}{L} = -C\,L^p
    \end{BM}
    \begin{itemize}
        \item[\textit{E}:] Energia Gasta para redução \([E]= \unit{\kilo\joule\per\kilo\gram}\)
        \item[\textit{L}:] Tamanho dos sólidos \([L]=\unit{\metre}\)
        \item[\textit{C}:] Constante empírica relacionada com as propríedades empíricas do equipamento
        \item[\textit{P}:] constante empírica relacionada com a grandeza dos sólidos
    \end{itemize}
    
\end{sectionBox}

\begin{sectionBox}2{Redução Fina}
    
    
    \begin{itemize}
        \item Energia ultilizada de forma mais eficiente
        \item Energia necessária almenta com o tamanho do recebimento decresce
    \end{itemize}

    \begin{BM}
        E = C\,(L_2^{-1}-L_1^{-1})
        \\
        E = K_r\,f_C\,(L_2^{-1}-L_1^{-1})
    \end{BM}
    
\end{sectionBox}

\begin{sectionBox}2{Redução Intermediata}
    
    \begin{BM}
        E = 2\,C\,(\sqrt{L_2}^{-1}-\sqrt{L_1}^{-1})
        \\
        E = E_i\,\sqrt{\frac{100}{L_2}}(1-\sqrt{q}^{-1})
    \end{BM}
    
\end{sectionBox}

\end{document}
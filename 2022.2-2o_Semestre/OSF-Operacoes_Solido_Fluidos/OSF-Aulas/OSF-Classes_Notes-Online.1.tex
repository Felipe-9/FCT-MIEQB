% !TEX root = ./OSF-Classes_Notes-Online.1.tex
\providecommand\mainfilename{"./OSF-Classes_Notes-Online.tex"}
\providecommand \subfilename{}
\renewcommand   \subfilename{"./OSF-Classes_Notes-Online.1.tex"}
\documentclass[\mainfilename]{subfiles}

% \tikzset{external/force remake=true} % - remake all

\begin{document}

% \graphicspath{{\subfix{./.build/figures/OSF-Classes_Notes-Online.1}}}
% \tikzsetexternalprefix{./.build/figures/OSF-Classes_Notes-Online.1/graphics/}

\mymakesubfile{1}
[OSF]
{Introdução} % Subfile Title
{Introdução} % Part Title

\begin{sectionBox}1{Physichomemical Properties of solids} % S
    
    \subsection{Single particles:}
    \begin{itemize}
        \begin{multicols}{2}
            \item Shape
            \item hardness
            \item compressive resistence
            \item electrical charge
            \item (intraparticle) porosity
        \end{multicols}
    \end{itemize}

    \subsection{Bulk solids:}
    \begin{itemize}
        \begin{multicols}{2}
            \item Particle size distribution
            \item (interparticle) porosity
            \item humidity
            \item agglomeration
            \item flowability
            \item \dots
        \end{multicols}
    \end{itemize}

    \subsection{Solids suspensions (heterogenius mixture in a fluid, gás or liquid)}
    \begin{itemize}
        \begin{multicols}{2}
            \item Particle size distribution
            \item Concentration of solids
            \item viscosity of suspension
            \item flocculation
            \item settleability
            \item \dots
        \end{multicols}
    \end{itemize}
    
\end{sectionBox}

\section*{Properties of single particles}

\begin{sectionBox}2{Shape} % S
    
    \subsubsection{Regular shape:}
    particles which its geometry is well defined by matematical equations
    \begin{itemize}
        \begin{multicols}{4}
            \item Sphere
            \item Cylinder
            \item Cube\dots
        \end{multicols}
    \end{itemize}
    \paragraph*{Sphere:}
    \begin{BM}[align*]
        \text{Volume:}\quad&
        \pi\,d^3/6
        = 4\,\pi\,r^3/3
        \\
        \text{Surface Area:}\quad&
        \pi\,d^2
        = 4\,\pi\,r^2
        \\
        \text{Projected area in a plane:}\quad&
        \pi\,d^2/2
        = \pi\,r^2
    \end{BM}
    \paragraph*{Note:} Spheres are special as its compleetely symmetrical whilist the others depend on the orientation
    
    \subsubsection{Complex regular shapes}
    \begin{itemize}
        \item Maximize surface area per Volume
        \item Used in packed columns in chem and biochem enginneering
    \end{itemize}


\end{sectionBox}

\begin{sectionBox}3{Irregular shapes} % S
    \begin{itemize}
        \item Cannot be identified by math equations
        \item Characteristic dimension \textit{d}
        \begin{itemize}
            \item Sphere Diameter with same volume 
            \\\(V_{\text{particle}}=V_{\text{sphere}}\)
            \item Sphere Diameter with same Surface area
            \\\(S_{\text{particle}}=S_{\text{sphere}}\)
            \item Sphere Diameter with same Surface per volume
            \\\(
                \frac
                {S_{\text{particle}}}
                {V_{\text{particle}}}
                = \frac
                {S_{\text{Sphere}}}
                {V_{\text{Sphere}}}
            \)
        \end{itemize}
    \end{itemize}
    \subsection*{Derivando propriedades:}
    \begin{description}[
        leftmargin=!,
        labelwidth=\widthof{Surface area} % Longest item
    ]
        \begin{multicols}{2}
            \item[Length] \(L=d\)
            \item[Surface area] \(S=\dot{k}\,d^2\)
            \item[Volume] \(V=\ddot{k}\,d^3\)
            \item[Mass] \(m=\rho_S\,V=\rho_S\,\ddot{k}\,d^3\)
        \end{multicols}
    \end{description}
    \begin{itemize}
        \begin{multicols}{2}
            \item Surface factor: \(\dot{k}_{\text{sphere}}=\pi\)
            \item Volume factor: \(\ddot{k}_{\text{sphere}}=\pi/6\)
        \end{multicols}
    \end{itemize}

\end{sectionBox}

\begin{sectionBox}2{Particle Distribution Curves} % S

    \begin{BM}
        \text{Cumulative}
        \iff
        \text{Frequency}
    \end{BM}

    \paragraph*{Diff ways to measure quantity}
    \begin{description}[
        leftmargin=!,
        labelwidth=\widthof{} % Longest item
    ]
        \begin{multicols}{3}
            \item[n:] Number fraction
            \item[l:] lenght fraction
            \item[s:] surface fraction
            \item[v:] volume fraction
            \item[x:] weight fraction
        \end{multicols}
    \end{description}

    \begin{center}
        \vspace{1ex}
        \begin{tabular}{l@{:\quad} LL}
            \toprule

                \multicolumn{1}{c}{Property}
                & \multicolumn{1}{c}{Whole}
                & \multicolumn{1}{c}{Fraction}

            \\\midrule
                Number
                & n_i,d_i
                & n_i
                % 
                \\Length (\unit{\metre})
                & l=n_i\,d_i
                & l_i=\frac{n_i\,d_i}{\sum{n_j\,d_j}}
                % 
                \\Surface (\unit{\metre^2})
                & s = n_i\,\dot{k}\,d_i^2
                & s_i = \frac{n_i\,d_i^2}{\sum{n_j\,d_j^2}}
                % 
                \\Volume (\unit{\metre^3})
                & v = n_i\,\ddot{k}\,d_i^3
                & v_i = \frac{n_i\,d_i^3}{\sum{n_j\,d_j^3}}
                % 
                \\Mass (\unit{\kilo\gram})
                & x   = n_i\,\rho\,\ddot{k}\,d_i^3
                & x_i = \frac{n_i\,d_i^3}{\sum{n_j\,d_j^3}}
            
            \\\bottomrule
        \end{tabular}
        \vspace{2ex}
    \end{center}

    \subsection*{Mean diameter:}
    Can be based on different properties of solid like weight, number or volume.
    \begin{BM}
        \bar{d_{\alpha}}
        = \frac{\int{d\,\odif{\alpha}}}{\int{\odif{\alpha}}}
        = \frac{\sum{d_i\,\alpha_i}}{\sum{\alpha_i}}
        : \alpha
        \begin{cases}
               x: & \text{Weight}
            \\ n: & \text{Number}
            \\ v: & \text{Volume}
            \\ s: & \text{Surface}
            \\ l: & \text{Lenght}
        \end{cases}
    \end{BM}
    \begin{BM}[flalign*]
        &
        % ------------------------------------------------ %
        %                 Weight and Number                %
        % ------------------------------------------------ %
        \text{Weight and Number}&\\&
        \begin{cases}
            \text{Measured in number }n\\
            \bar{d_{x}}
            = \bar{d_{v}}
            = \cfrac
            {\int{d\,\odif{(n\,\rho\,\ddot{k}\,d^3)}}}
            {\int{   \odif{(n\,\rho\,\ddot{k}\,d^3)}}}
            = {\color{Emph}
                \cfrac
                {\int{n\,d^3\odif{d}}}
                {\int{n\,d^2\odif{d}}}
            }
            = \\
            = \cfrac
            {\sum{d_i\,(n_i\,\rho\,\ddot{k}\,d_i^3)}}
            {\sum{     (n_i\,\rho\,\ddot{k}\,d_i^3)}}
            = {\color{Emph}
                \cfrac
                {\sum{n_i\,d_i^4}}
                {\sum{n_i\,d_i^3}}
            }
            = \\
            \text{Measured in Weight }x\\
            = {\color{Emph}
                \cfrac
                {\sum{x_i\,d_i}}
                {\sum{x_i}}
            }
        \end{cases}
        % ------------------------------------------------ %
        %                      Surface                     %
        % ------------------------------------------------ %
        &\\[3ex]&
        \text{Surface}&\\&
        \begin{cases}
            \text{Measured in number }n\\
            \bar{d}_S
            = \cfrac
            {\int{d\,\odif{(n\,\dot{k}\,d^2)}}}
            {\int{   \odif{(n\,\dot{k}\,d^2)}}}
            = {\color{Emph}
                \cfrac
                {\int{d^2\,n_i\odif{d}}}
                {\int{d  \,n_i\odif{d}}}
            }
            = \\
            = \cfrac
            {\sum{d_i\,(n_i\,\dot{k}\,d_i^2)}}
            {\sum{     (n_i\,\dot{k}\,d_i^2)}}
            = {\color{Emph}
                \cfrac
                {\sum{n_i\,d_i^3}}
                {\sum{n_i\,d_i^2}}
            }
            = \\
            \text{Measured in Weight }x\\
            = \cfrac
            {\sum{
                \left(
                    \frac{x_i}{\rho\,\ddot{k}\,d_i^3}
                \right)\,d_i^3}}
            {\sum{\left(
                \frac{x_i}{\rho\,\ddot{k}\,d_i^3}
            \right)\,d_i^2}}
            = {\color{Emph}
                \cfrac
                {\sum{x_i}}
                {\sum{x_i/d_i}}
            }
        \end{cases}
        % ------------------------------------------------ %
        %                      Lenght                      %
        % ------------------------------------------------ %
        &\\[3ex]&
        \text{Lenght}&\\&
        \begin{cases}
            \text{Measured in number }n\\
            \bar{d}_L
            = \cfrac
            {\int{d\,\odif{(n\,d)}}}
            {\int{   \odif{(n\,d)}}}
            = {\color{Emph}
                \cfrac
                {\int{d\,n\,\odif{d}}}
                {\int{   n\,\odif{d}}}
            }
            = \\
            = \cfrac
            {\sum{d_i\,(n_i\,d_i)}}
            {\sum{     (n_i\,d_i)}}
            = {\color{Emph}
                \cfrac
                {\sum{n_i\,d_i^2}}
                {\sum{n_i\,d_i  }}
            }
            =\\
            \text{Measured in Weight }x\\
            = \cfrac
            {\sum{\left(
                \frac{x_i}{\rho\,\ddot{k}\,d_i^3}
            \right)\,d_i^2}}
            {\sum{\left(
                \frac{x_i}{\rho\,\ddot{k}\,d_i^3}
            \right)\,d_i  }}
            = {\color{Emph}
                \cfrac
                {\sum{x_i/d_i  }}
                {\sum{x_i/d_i^2}}
            }
        \end{cases}
        &\\[3ex]&
        \text{Volume}&\\&
        \begin{cases}
            \bar{d}_V
            = \cfrac
            {\int{d\,\odif{n\,\ddot{k}\,d^3}}}
            {\int{   \odif{n\,\ddot{k}\,d^3}}}
            = {\color{Emph}
                \cfrac
                {\int{n\,d^3\odif{d}}}
                {\int{n\,d^2\odif{d}}}
            }
            = \\
            = \cfrac
            {\sum{d_i\,(n_i\,\ddot{k}\,d_i^3)}}
            {\sum{     (n_i\,\ddot{k}\,d_i^3)}}
            = \cfrac
            {\sum{n_i\,d_i^4}}
            {\sum{n_i\,d_i^3}}
            \\\text{Assuming all particles have the same size}\\
            \sum{n_i\,\ddot{k}\,\bar{d}_V^3}
            =\ddot{k}\,\bar{d}_V^3\,\sum{n_i}
            = \sum{n_i\,\ddot{k}\,d_i^3}
            \implies\\
            \implies
            \bar{d}_V
            = {\color{Emph}
                \sqrt{
                    \frac
                    {\sum{n_i\,d_i^3}}
                    {\sum{n_i}}
                }
            }
        \end{cases}
    \end{BM}
\end{sectionBox}

\end{document}
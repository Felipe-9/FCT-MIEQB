% !TEX root = ./OSF-Aulas_Anotações.2.tex
\providecommand\mainfilename{"./OSF-Aulas_Anotações.tex"}
\providecommand \subfilename{}
\renewcommand   \subfilename{"./OSF-Aulas_Anotações.2.tex"}
\documentclass[\mainfilename]{subfiles}

% \graphicspath{{\subfix{../images/}}}
% \tikzset{external/force remake=true} % - remake all

\begin{document}

\mymakesubfile{2}
[OSF]
{Aula}
{Aula}

\begin{sectionBox}1{Equipment Size for Reduction}
    
    
    
\end{sectionBox}

\begin{sectionBox}2{Jaw Crusher}
    
    Duas mandibulas com dentes compatíveis
    
\end{sectionBox}

\part*{Capítulo 3}

\begin{definitionBox}1{Reynold Number}
    
    \begin{BM}
        Re' = F(\pi\,d\,\rho/\mu)^{-1}
    \end{BM}
    
\end{definitionBox}

\begin{definitionBox}1{Navier-Stokes}
    
    \begin{BM}
        F = 3\,\pi\,\mu\,d\,u
    \end{BM}

    Aplicado ao movimento de uma esféra em velocidades baixas proporcionalmente.
    Apenas valido para valores pequenos de \(Re\) pois a linha em valores maiores devia.
    
\end{definitionBox}

\begin{sectionBox}1{Queda livre de uma esfera em um líquido}
    
    \begin{multicols}{2}
       
        \begin{minipage}{\textwidth}
            
            \paragraph*{Forças relacionadas}
            \begin{itemize}
                \item Atrito
                \item Gravidade
                \item Buoyance
            \end{itemize}

        \end{minipage}
    
        \begin{BM}
            \vec{F_g}
            = \frac{\pi\,r^3}{6}\rho_{s}\,\vec{g}
            \\
            \vec{F_I}
            = \frac{\pi\,r^3}{6}\rho\,\vec{g}
            \\
            \vec{F_g} - \vec{F_I}
            = \frac{\pi\,r^3}{6}(\rho_s-\rho)\,\vec{g}
        \end{BM}
    \end{multicols}

    \paragraph*{Fases}
    O fenomeno pode ser definido em 4 etapas definidas pelo numero do Reynold
    \begin{BM}[align*]
           1\E-4       & <    Re < 0.2          % & \quad \text{}
        \\ 0.2         & \leq Re < 500\lor1000  % & \quad \text{}
        \\ 500\lor1000 & \leq Re < Ca\,2\E5     % & \quad \text{}
        \\ Ca\,2\E5    & < Re                   % & \quad \text{}
    \end{BM}
    % \begin{enumerate}
    %     \item \(1\E-4       <    Re < 0.2\)
    %     \item \(0.2         \leq Re < 500\lor1000\)
    %     \item \(500\lor1000 \leq Re < Ca\,2\E5\)
    %     \item \(Ca\,2\E5 < Re\)
    % \end{enumerate}
    
\end{sectionBox}

\begin{sectionBox}*2{Atrito}
    
    Ha duas formas de atrito agindo sobre um corpo dentro do flúido
    \begin{itemize}
        \item Superfície: Relacionado com a viscosidade do flúido
        \item Forma: Relacionado com a diferença de pressão do flúido entre os lados do corpo
    \end{itemize}

    \begin{sectionBox}*3{Atrito de Superfície}
        
        Relacionado com a viscosidade do flúido

        \begin{BM}
            F = 3\,\pi\,\mu\,u\,d
        \end{BM}

        \paragraph*{Stronkes Law}
        \begin{BM}
            \frac{R'}{\rho\,u^2}
            = 12\,\frac{\mu}{u\,d\,\rho}
            = 12\,Re'^{-1}
        \end{BM}
        Limite da lei de Stronks: \(Re' = 2\)
        
    \end{sectionBox}
    
\end{sectionBox}

\end{document}
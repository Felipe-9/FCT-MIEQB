% !TEX root = ./OSF-Classes_Notes-Online.2.tex
\providecommand\mainfilename{"./OSF-Classes_Notes-Online.tex"}
\providecommand \subfilename{}
\renewcommand   \subfilename{"./OSF-Classes_Notes-Online.2.tex"}
\documentclass[\mainfilename]{subfiles}

% \tikzset{external/force remake=true} % - remake all

\begin{document}

% \graphicspath{{\subfix{./.build/figures/OSF-Classes_Notes-Online.2}}}
% \tikzsetexternalprefix{./.build/figures/OSF-Classes_Notes-Online.2/graphics/}

\mymakesubfile{2}
[OSF]
{Equipment Size reduction} % Subfile Title
{Equipment Size reduction} % Part Title

\begin{sectionBox}1{Crushing rolls} % S
    
    % Triturador de rolos
    Compress particles between two rolls

    % Nip angle
    \begin{BM}
        \cos{\beta/2}
        = \frac
        {r_1+b/2}
        {r_1+r_2}
    \end{BM}
    \begin{description}[
        leftmargin=!,
        labelwidth=\widthof{} % Longest item
    ]
        \item[\(r_1\):] Roll radius
        \item[\(r_2\):] Feed particle size
        \item[\(b\):] Distance between rolls
        \item[\(\beta=31^\circ\):] Parametrizado em 31º
    \end{description}
    
\end{sectionBox}

\begin{sectionBox}1{Jaw Crushers} % S
    
    % Triturador de Maxilas
    Crush the particles between two jaw like contraptions
    \begin{itemize}
        \item Two jaws one fixed one mobile
        \item Nip angle \(\sim30^\circ\)
        \item Coarse reduction
    \end{itemize}
    
\end{sectionBox}

\begin{sectionBox}1{Energy for size reduction} % S
    
    Very low efficiency \(\numrange{0.1}{2.0}\) but effetive size reduction
    \begin{itemize}
        \item In producing elastic deformation of the particles before fracture occurs
        \item In causing elastic distortion of the equipment.
        \item In friction between particles, and between particles and the machine.
        \item In noise, heat and vibration in the plant, and In friction losses in the plant itself.
    \end{itemize}

    \subsection*{Empirical Law of Energy for size reduction}
    \begin{BM}
        \odv{E}{L}
        =-C\,L^p
    \end{BM}
    \begin{description}[
        leftmargin=!,
        labelwidth=\widthof{\(E\)} % Longest item
    ]
        \item[\(E\)] Energy spent for size reduction 
        {[\unit{\kilo\joule/\kilo\gram}]}
        \item[\(L\)] size of solids calculated as the mean diameter based on voume (\(d_V/\unit{\metre}\)) \\Alternativelly Bond's diameter (\(d_{bond}/\unit{\metre}\)) can be used, its estimated as the mesh size trough which 80\% of the material passes in a sieving characterization experiment.
        \item[\(C\)] empirical constant related to the solid properties and equipment properties
        \item[\(p\)] empirical constant related to the size of solids
        \begin{itemize}
            \item \(=-1.0\) Coarse reduction
            \item \(=-1.5\) Intermediate reduction
            \item \(=-2.0\) Fine reduction
        \end{itemize}
    \end{description}
    Specifying \(p\) for each reduction and integrating we derive the following three laws
    \subsection*{Rittinger's law (Fine reduction,\(p=-2.0\))}
    \begin{BM}
        E 
        = C\,\adif{L^{-1}}
        = K_R\,f_c\,\adif{L^{-1}}
        \\
        C=K_R\,f_c
        \begin{cases}
              K_R: &\text{Depende do triturador}
            \\f_c: &\text{Depende das particulas}
        \end{cases}
    \end{BM}
    \paragraph*{Note:} Greater efficiency, \(E\propto d^{-1}\)
    \subsection*{Bond's Law (Intermediate reduction, \(p=-1.5\))}
    \begin{BM}
        E
        =2\,C\,\adif{L^{-1/2}}
        =E_i\,\sqrt{100/L_1}
        \left(
            1 - q^{-1/2}
        \right)
        \\
        \begin{cases}
            C=E_i\,\sqrt{100}
            \\
            q=L_0/L_1
        \end{cases}
    \end{BM}
    \paragraph*{Notes:} Intermediate Efficiency, \(E\propto L_0^{-1}\)
    \subsection*{Kick's Law (Coarse reduction, \(p=-1.0\))}
    \begin{BM}
        E
        =C\,\adif{\ln{L}}
        =K_K\,f_c\,\adif{\ln{L}}
        \\
        C=K_K\,f_c
        \begin{cases}
              K_K: &\text{Depende do triturador}
            \\f_c: &\text{Depende das particulas}
        \end{cases}
    \end{BM}
    \paragraph*{Notes:} Less energy efficiency, 
    \(E\propto L_0/L_1\)
    
\end{sectionBox}

\end{document}
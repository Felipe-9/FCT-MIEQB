% !TEX root = ./QOG-Slides_Anotações.2.tex
\providecommand\mainfilename{"./QOG-Slides_Anotações.tex"}
\providecommand \subfilename{}
\renewcommand   \subfilename{"./QOG-Slides_Anotações.2.tex"}
\documentclass[\mainfilename]{subfiles}

% \tikzset{external/force remake=true} % - remake all

\begin{document}

\graphicspath{{\subfix{./.build/figures/QOG-Slides_Anotações.2}}}
% \tikzsetexternalprefix{./.build/figures/QOG-Slides_Anotações.2/graphics/}

\mymakesubfile{2}
[QOG]
{Anotações Slide 2} % Subfile Title
{Slide} % Part Title

\part*{Nomenclatura: Alcanos}

\begin{sectionBox}1{Prefixo} % S1.1
    
    Relacionado ao numero de carbonos
    
    \begin{enumerate}
        \begin{multicols}{4}
            \item met
            \item et
            \item prop
            \item but
            \item pen
            \item hex
            \item hep
            \item oct
            \item non
            \item dec
            \item undec
            \item dodec
            \item tridec
            \item tetradec
            \item pentadec
            \item hexadec
            \item heptadec
            \item octadec
            \item nonadec
            \item icos
        \end{multicols}
        \setcounter{enumi}{29}
        \item tria
    \end{enumerate}

\end{sectionBox}

\begin{sectionBox}1{Intermediários} % S1.2
    
    referencia as ligações
    (suprime o ``an'', ex: etano\to etino)

    \begin{description}
        \begin{multicols}{2}
            \item[an] Somente ligações simples
            \item[en] 1 ligação dupla
            \item[dien] 2 ligações duplas
            \item[in] 1 ligação tripla
            \item[diin] 2 ligações triplas
            \item[enin] 1 dupla e 1 tripla
        \end{multicols}
    \end{description}

\end{sectionBox}

\begin{sectionBox}1{Sufixos 1} % S1.3
    
    \setchemfig{
        angle increment={15},
        atom sep={2em}
    }

    \begin{description}
        \begin{multicols}{2}
            \item[o:] Hidrocarboneto \chemfig{CH2(-[12])}
            \item[ol:] Álcool e enol \chemfig{OH(-[12])}
            \item[al:] Aldeído \chemfig{C(-[12])(=[1]O)(-[-1]H)}
            \item[ona:] Cetona \chemfig{O=[12]}
            \item[óico:] Ácido Carboxilo \chemfig{C(-[12])(=[1]O)(-[-1]OH)}
            \item[amina:] \chemfig{NH2(-[12])}
            \item[amida:] \chemfig{
                R
                -[ 1]C(-[ 2]H)
                -[-1]N(-[-2]R')
                -[ 1]R
            }
            \item[sulfônico:] Ácido Sulfônico 
            \chemfig{
                S
                (=[  6]O)
                (=[- 3]O)
                (-[- 1]OH)
                (-[-10]R)
            }
            \item[óxi:] Éter
            \chemfig{
                R_1
                -[ 1]O
                -[-1]R_2
            }
            \item[ato\ \dots\ ila:] Éster
            \chemfig{
                C
                (-[-11]R_1)
                (=[  2]O)
                (-[ -1]O-[1]R_2)
            }
            \item[ato de \dots:] Sal Orgânico
            \chemfig{
                C
                (-[-11]R_1)
                (=[  2]O)
                (-[ -1]O-[1,.1em]Metal)
            }
        \end{multicols}
    \end{description}
    
\end{sectionBox}

\begin{exampleBox}1b{} % E
    \subsection*{Hidrocarbonetos}
    \vspace{-4ex}
    \begin{multicols}{3}
        \begin{exampleBox}2m{}
            \centering
            \ch{CH4}
            \\\iupac{met|ano}
        \end{exampleBox}
        
        \begin{exampleBox}2m{}\centering
            \ch{CH3-CH3}
            \\\iupac{et|ano}
        \end{exampleBox}
    
        \begin{exampleBox}2m{}\centering
            \ch{CH3-(CH2)10-CH3}
            \\\iupac{dodec|ano}
        \end{exampleBox}
    \end{multicols}
    
    \subsection*{Álcools e Enols}\\
    \vspace{-4ex}
    \begin{multicols}{2}
        
        \begin{exampleBox}2{}\centering
            \ch{CH3-OH}
            \\\iupac{met|an|ol}
        \end{exampleBox}
        
        \begin{exampleBox}2{}\centering
            \ch{CH3-CH2-OH}
            \\\iupac{et|an|ol}
        \end{exampleBox}
        
        \begin{exampleBox}2{}\centering
            \ch{CH3-CH2-CH=CH-OH}
            \\\iupac{but|en|ol}
        \end{exampleBox}
        
        \begin{exampleBox}2{}\centering
            \ch{CH3-(CH2)2-CH=CH-OH}
            \\\iupac{hex|en|ol}
        \end{exampleBox}
        
    \end{multicols}
        
    \subsection*{Aldeídos}\\
    \vspace{-4ex}
    \begin{multicols}{2}
    
        \begin{exampleBox}2{}\centering
            \ch{HCHO}
            \\\iupac{met|an|al}
        \end{exampleBox}
    
        \begin{exampleBox}2{}\centering
            \ch{CH3-CHO}
            \\\iupac{et|an|al}
        \end{exampleBox}
    
    \end{multicols}
        
    \subsection*{Cetona}\\
    \vspace{-4ex}
    \begin{multicols}{2}
    
        \begin{exampleBox}2{}\centering
            \ch{CH3-CO-CH3}
            \\\iupac{prop|an|ona}
        \end{exampleBox}
    
        \begin{exampleBox}2{}\centering
            \ch{CH3-CH2-CO-CH3}
            \\\iupac{but|an|ona}
        \end{exampleBox}
    
    \end{multicols}
    
    \subsection*{Eter}\\
    \vspace{-4ex}
    \begin{multicols}{2}
    
        \begin{exampleBox}2{}\centering
            \ch{CH3-O-CH3}
            \\\iupac{met|oxi|met|an|o}
        \end{exampleBox}
    
        \begin{exampleBox}2{}\centering
            \ch{CH3-CH2-O-CH2-CH3}
            \\\iupac{et|oxi|et|an|o}
        \end{exampleBox}
    
    \end{multicols}
    
    \subsection*{Amina}\\
    \vspace{-4ex}
    \begin{multicols}{2}
    
        \begin{exampleBox}2{}\centering
            \ch{CH3-NH-CH3}
            \\\iupac{di|met|il|amina}
        \end{exampleBox}
    
        \begin{exampleBox}2{}\centering
            \ch{CH3-N(-CH3)-CH3}
            \\\iupac{tri|met|il|amina}
        \end{exampleBox}
    
    \end{multicols}
    
    \subsection*{Amida}\\
    \vspace{-4ex}
    \begin{multicols}{2}
    
        \begin{exampleBox}2{}\centering
            \ch{HCO-NH2}
            \\\iupac{met|an|amida}
        \end{exampleBox}
    
        \begin{exampleBox}2{}\centering
            \ch{CH3-CO-NH2}
            \\\iupac{et|an|amida}
        \end{exampleBox}
    
    \end{multicols}

    \subsection*{Ácido Carboxilico}\\
    \vspace{-4ex}
    \begin{multicols}{2}
    
        \begin{exampleBox}2{}\centering
            \ch{HCOOH}
            \\\iupac{Ácido met|an|oico}
        \end{exampleBox}
    
        \begin{exampleBox}2{}\centering
            \ch{CH3-COOH}
            \\\iupac{Ácido et|an|oico}
        \end{exampleBox}
    
    \end{multicols}
        
    \subsection*{Éster}\\
    \vspace{-4ex}
    \begin{multicols}{2}
    
        \begin{exampleBox}2{}\centering
            \ch{HCOO-CH3}
            \\\iupac{met|an|oato de met|ila}
        \end{exampleBox}
    
        \begin{exampleBox}2{}\centering
            \ch{CH3-HCOOCH-CH3}
            \\\iupac{et|an|oato de et|ila}
        \end{exampleBox}
    
    \end{multicols}

    \subsection*{Sal Orgânico}\\
    \vspace{-4ex}
    \begin{multicols}{2}
    
        \begin{exampleBox}2{}\centering
            \ch{CH3-COO-Na^{+}}
            \\\iupac{et|an|oato de sódio}
        \end{exampleBox}
    
        \begin{exampleBox}2{}\centering
            \ch{HCOO-K^{+}}
            \\\iupac{met|an|oato de potássio}
        \end{exampleBox}
    
    \end{multicols}

    \subsection*{Anidrido de Ácido}
    \vspace{-4ex}
    \begin{multicols}{2}
    
        \begin{exampleBox}2{}\centering
            \ch{CH3-COO-CO-CH3}
            \\\iupac{et|an|oato de sódio}
        \end{exampleBox}
    
        \begin{exampleBox}2{}\centering
            \ch{HCOO-K^{+}}
            \vspace{1ex}
            \iupac{met|an|oato de potássio}
        \end{exampleBox}
    
    \end{multicols}
\end{exampleBox}


\begin{sectionBox}1{Radicais Substituintes} % S4
    
    Quando há ramificações, para poder nomear precisa seguir o passo a passo
    
    \begin{sectionBox}2b{Identificar a cadeia mais longa} % S
        
        \begin{center}
            \includegraphics[width=.8\textwidth]{Screenshot 2022-11-06 at 21.36.38}
        \end{center}
    
        em empate escolhe a que tem mais substituintes
        
    \end{sectionBox}
    
    \begin{sectionBox}2{Nomear os substituintes} % S
        
        \begin{description}
            % \begin{multicols}{2}
                \item[halogenio] recebe o nome do halogenio \ch{Br, F, Cl, I}
                \item[Cadeia org simples] recebe prefixo ``il''
                \item[Canonicos:] Estruturas canonicas que devemos conhecer
            % \end{multicols}
        \end{description}
    
        \paragraph*{Multiplos substituintes identicos} se juntam recebendo um prefixo numérico
    
        \begin{exampleBox}*3{Exemplo}\centering
            
            \chemfig[
                angle increment=30, 
                atom sep={2em}
            ]{
                -[ 1]
                -[-1](-[-3])
                -[ 1](-[ 3])
                -[-1]
                -[ 1]
            }
            \qquad
            \iupac{\textcolor{Emph}{di}|metil|hex|ano}
            
        \end{exampleBox}
    
        \paragraph*{Complexidade da cadeia} simples usa o prefixo numérico comum, complexa (multiplas bifurcações) usa os prefixos:
        \begin{enumerate}
            \setcounter{enumi}{1}
            \begin{multicols}{3}
                \item bis
                \item tris
                \item tetrakis
            \end{multicols}
        \end{enumerate}
    
        \paragraph*{Estruturas Canonicas}
        \begin{description}
            \begin{multicols}{3}
                
                \setchemfig{
                    angle increment=15,
                    atom sep={1.5em}
                }
    
                \item[\iupac{iso|prop|il}]
                \chemfig{
                    {}
                    -[1]
                    (-[-2]R')
                    (-[ 4]CH3)
                }
                \item[\iupac{iso|but|ano}]
                \chemfig{
                    {}
                    -[1]
                    (-[-2]-[2]R')
                    (-[ 4]CH3)
                }
                \item[\iupac{sec-but|il}]
                \chemfig{
                    {}
                    -[1]
                    (
                        -[-2](-[-7]CH3)
                        -[2]R'
                    )
                }
                \item[\iupac{tert-but|il}]
                \chemfig{
                    {}
                    -[2]
                    (-[  4]CH3)
                    (-[ 10]CH3)
                    -[-2]R'
                }
                \item[\iupac{neopent|il}]
                \chemfig{
                    {}
                    -[2]
                    (-[  4]CH3)
                    (-[ 10]CH3)
                    -[-2]
                    -[ 2]R'
                }
    
            \end{multicols}
             
        \end{description}
        
    \end{sectionBox}
    
    \begin{sectionBox}2{Numerar a cadeia} % S
        
        Se nomeia da extremidade mais próxima aos substituintes
    
        \begin{exampleBox}*2{Exemplo}\centering
            
            \setchemfig{
                angle increment=30,
                atom sep={2em}
            }
    
            \chemfig{
                {\textcolor{Emph}{1}}
                -[ 1]{\textcolor{Emph}{2}}(-[ 3])
                -[-1]{\textcolor{Emph}{3}}
                -[ 1]{\textcolor{Emph}{4}}
            }
            \qquad
            \chemfig{
                {\textcolor{Emph}{1}}
                -[ 1]{\textcolor{Emph}{2}}
                -[-1]{\textcolor{Emph}{3}}
                -[ 1]{\textcolor{Emph}{4}}(-[ 3])
                -[-1]{\textcolor{Emph}{5}}(-[-3])
                -[ 1]{\textcolor{Emph}{6}}
                -[-1]{\textcolor{Emph}{7}}
                -[ 1]{\textcolor{Emph}{8}}
                -[-1]{\textcolor{Emph}{9}}
            }
            
        \end{exampleBox}
        
    \end{sectionBox}
    
    \begin{sectionBox}2{Nomear em ordem \textcolor{Emph}{alfabética}} % S4.4
        
        Não leva em conta o prefixo numérico de multiplas ramificações exceto quado as ramificações são complexas
        
    \end{sectionBox}
\end{sectionBox}


\begin{exampleBox}1b{ % E2
} % E2
    
    \setchemfig{
        angle increment=30,
        atom sep={2em}
    }

    \begin{multicols}{2}

        \begin{exampleBox}2{}\centering
            \chemfig{
                -[ 1](-[3])
                -[-1]
                -[ 1]
            }\\
            \iupac{2-met|il|but|ano}
        \end{exampleBox}

        \begin{exampleBox}2{}\centering
            \chemfig{
                {\textcolor{Emph}{8}}
                -[ 1]{\textcolor{Emph}{7}}
                -[-1]{\textcolor{Emph}{6}}
                -[ 1]{\textcolor{Emph}{5}}(-[3]-[1])
                -[-1]{\textcolor{Emph}{4}}
                -[ 1]{\textcolor{Emph}{3}}
                -[-1]{\textcolor{Emph}{2}}(-[-3])
                -[ 1]{\textcolor{Emph}{1}}
            }\\
            \iupac{5-et|il-2-met|il-oct|ano}
        \end{exampleBox}

        \begin{exampleBox}2{}\centering
            \chemfig{
                {\textcolor{Emph}{8}}
                -[ 1]{\textcolor{Emph}{7}}
                -[-1]{\textcolor{Emph}{6}}
                -[ 1]{\textcolor{Emph}{5}}(-[3]-[1])
                -[-1]{\textcolor{Emph}{4}}
                -[ 1]{\textcolor{Emph}{3}}
                -[-1]{\textcolor{Emph}{2}}(-[-1])(-[-5])
                -[ 1]{\textcolor{Emph}{1}}
            }\\
            \iupac{5-et|il-2-di|met|il-oct|ano}
        \end{exampleBox}

        \begin{exampleBox}2{}\centering
            \chemfig{
                {\textcolor{Emph}{8}}
                -[ 1]{\textcolor{Emph}{7}}
                -[-1]{\textcolor{Emph}{6}}
                -[ 1]{\textcolor{Emph}{5}}(
                    -[3]
                    (-[1])
                    (-[5])
                    (-[4])
                )
                -[-1]{\textcolor{Emph}{4}}
                -[ 1]{\textcolor{Emph}{3}}(-[3]-[1])
                -[-1]{\textcolor{Emph}{2}}
                -[ 1]{\textcolor{Emph}{1}}
            }
            \\
            \iupac{5-(1,1-di|met|il|et|il)-3-et|il|oc|an|o}
        \end{exampleBox}

    \end{multicols}

    \begin{exampleBox}2{}\centering
        
        \chemfig[angle increment=30]{
                    {\textcolor{Emph}{8}}
                -[ 1]{\textcolor{Emph}{7}}(-[3]I)
                -[-1]{\textcolor{Emph}{6}}
                -[ 1]{\textcolor{Emph}{5}}(
                    -[3]
                    (-[1]Cl)
                    (-[5])
                )
                -[-1]{\textcolor{Emph}{4}}
                -[ 1]{\textcolor{Emph}{3}}
                -[-1]{\textcolor{Emph}{2}}(-[-1])(-[7])
                -[ 1]{\textcolor{Emph}{1}}
                -[-1]Br
        }\\
        \iupac{
            1-bromo
            -5-(1-cloro|et|il)
            -7-Iodo
            -2-di|met|il
            |oct|an|o
        }
        
    \end{exampleBox}
\end{exampleBox}


\end{document}
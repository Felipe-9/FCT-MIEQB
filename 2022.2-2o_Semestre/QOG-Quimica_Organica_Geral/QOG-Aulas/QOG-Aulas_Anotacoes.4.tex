% !TEX root = ./QOG-Aulas_Anotações.4.tex
\providecommand\mainfilename{"./QOG-Aulas_Anotações.tex"}
\providecommand \subfilename{}
\renewcommand   \subfilename{"./QOG-Aulas_Anotações.4.tex"}
\documentclass[\mainfilename]{subfiles}

% \graphicspath{{\subfix{../images/}}}
% \tikzset{external/force remake=true} % - remake all

\begin{document}

\mymakesubfile{4}
[QOG]
{Aula 25/10/2022} % Subfile Title
{Aula 25/10/2022} % Part Title

\begin{sectionBox}1{??} % S1
    
    \begin{center}
        \ch{CH3CH3 + 7/2 O2 -> 2 CO2 + 3 H2O}
    \end{center}
    
\end{sectionBox}

\begin{sectionBox}1{Haletos Organicos} % S2
    
    Um haleto é um composto químico binário, do qual uma parte é um átomo de halogênio e a outra parte é um elemento ou radical menos eletronegativo do que o halogênio, para fazer um composto fluoreto, cloreto, brometo, iodeto, astatídeo ou teoricamente tennesside.

    \paragraph*{Exemplo: Clorofluorocarboneto (CFSs)} 
    \begin{itemize}
        \item Gases que tem halogenios, toxicos por causa da reatividade dos halogenios
        \item Compostos que átomos de cloro e fluor ligados a cadeias carbonicas
        \item Em geral pequenas
    \end{itemize}
    
\end{sectionBox}


\begin{sectionBox}1m{Clivagem} % S3
    
    Em química, a clivagem de ligações, ou fissão de ligações, é a divisão de ligações químicas. Isso geralmente pode ser chamado de dissociação quando uma molécula é dividida em dois ou mais fragmentos.

    \begin{sectionBox}2{Clivagem homolítica} % S3.1
        
        \begin{itemize}
            \item Divisão igual de eletrons entre cada cada átomo
            \item Os dois elétrons em uma ligação covalente clivizada são divididos igualmente entre os produtos. 
            \item A \textit{energia de dissocação de ligação} de uma ligação é a quantidade de energia necessária para cissar a ligação homolisticamente.
            \item Essa mudança de entalpia é uma medida da força da ligação.
        \end{itemize}
    \end{sectionBox}

    \begin{sectionBox}2{Clivagem heterolítica} % S3.2
        
        \begin{itemize}
            \item a ligação se rompe de tal forma que o par de elétrons originalmente compartilhado permanece com um dos fragmentos.
            \item Este processo também é conhecido como fissão iônica.
        \end{itemize}

    \end{sectionBox}
    
\end{sectionBox}

\begin{sectionBox}1{Hiperconjugação} % S4
    
    A hiperconjugação é a interação estabilizadora que resulta da interação dos elétrons em uma ligação \chemsigma{} (geralmente \ch{C-H} ou \ch{C-C}) com um p-orbital vazio ou parcialmente preenchido adjacente ou um \chempi-orbital para dar um orbital molecular estendido que aumenta a estabilidade do sistema.
    
\end{sectionBox}

\end{document}
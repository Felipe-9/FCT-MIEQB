% !TEX root = ./QOG-Exercises_Resolutions.2020.2.4.tex
\providecommand\mainfilename{"./QOG-Exercises_Resolutions.tex"}
\providecommand \subfilename{}
\renewcommand   \subfilename{"./QOG-Exercises_Resolutions.2020.2.4.tex"}
\documentclass[\mainfilename]{subfiles}

% \tikzset{external/force remake=true} % - remake all

\begin{document}

% \graphicspath{{\subfix{./.build/figures/}}}

\mymakesubfile{4}
[QOG]
{Execicios 2020.2} % Subfile Title
{Execicios 2020.2} % Part Title


\begin{questionBox}1{ % Q1
    Escreva a fórmula estrutural para cada um dos principais produtos, obtidos na reacção do brometo de metilo com:
} % Q1
    Brometo de metilo: \ch{CH3Br}

    \vspace{2ex}
    \begin{enumerate}[label=\alph{enumi}.]
        \begin{multicols}{3}
            \item \ch{NaOH}
            \item \ch{KOCH2CH3}
            \item Benzoato de sódio
            \item Azida de lítio
            \item Cianeto de potássio
            \item \ch{NaSH}
        \end{multicols}
    \end{enumerate}
\end{questionBox}

\begin{questionBox}2{ % Q1.1
    \ch{NaOH}
} % Q1.1
    O nucleófilo no hidróxido de sódio é o ião hidroxilo carregado negativamente. A reacção que ocorre é uma substituição nucleofílica do brometo pelo hidroxilo. O produto da reacção é o álcool metílico.
    \vspace{2ex}
    \begin{center}
        \ch{
            NaOH + H2O -> Na^{+} + OH^{-} + CH3Br -> CH3OH + NaBr\aq
        }
    \end{center}
    \vspace{-5ex}
    \begin{table}[H]\centering
        \begin{tabular}{lll}
            
            \\\toprule
            
                \multicolumn{1}{c}{Composto}
            &   \multicolumn{1}{c}{Nome}
            &   \multicolumn{1}{c}{Papel}
            
            \\\midrule
            
                \ch{OH^{-}} & Ião Hidroxilo & Nucleofilo
            \\  \ch{CH3Br} & Brometo de Metilo & Substrato
            \\  \ch{CH3OH} & Metanol & Produto
            \\  \ch{Br^-} & Ião Brometo & Grupo abandonante
            
            \\\bottomrule
            
        \end{tabular}
    \end{table}
\end{questionBox}

\begin{questionBox}2{ % Q1.2
    \ch{KOCH2CH3}
} % Q1.2
    \begin{center}
        \ch{
            KOCH2CH3 + H2O -> K^{+} + OCH2CH3^{-} + CH3Br -> CH3OCH2CH3 + KBr\aq
        }
    \end{center}
    \vspace{-5ex}
    \begin{table}[H]\centering
        \begin{tabular}{lll}
            
            \\\toprule
            
                \multicolumn{1}{c}{Composto}
            &   \multicolumn{1}{c}{Nome}
            &   \multicolumn{1}{c}{Papel}
            
            \\\midrule
            
                \ch{OCH2CH3^{-}} & Ião \dots & Nucleofilo
            \\  \ch{CH3Br} & Brometo de Metilo & Substrato
            \\  \ch{CH3OCH2CH3} & Metoxietano & Produto
            \\  \ch{Br^-} & Ião Brometo & Grupo abandonante
            
            \\\bottomrule
            
        \end{tabular}
    \end{table}
\end{questionBox}

\setcounter{subquestion}{3}

\begin{questionBox}2{ % Q1.4
    Azida de lítio (\ch{N3Li2})
} % Q1.4
    \begin{center}
        \ch{
            N3Li2 + H2O -> 2 Li^{+} + N3^{-} + CH3Br -> CH3N3^{-} + 2 Li^{+} + Br^{-}
        }
    \end{center}
    \vspace{-5ex}
    \begin{table}[H]\centering
        \begin{tabular}{lll}
            
            \\\toprule
            
                \multicolumn{1}{c}{Composto}
            &   \multicolumn{1}{c}{Nome}
            &   \multicolumn{1}{c}{Papel}
            
            \\\midrule
            
                \ch{CN^{-}} & Ião Azida & Nucleofilo
            \\  \ch{CH3Br} & Brometo de Metilo & Substrato
            \\  \ch{CH3CN} & Azida de Metilo & Produto
            \\  \ch{Br^-} & Ião Brometo & Grupo abandonante
            
            \\\bottomrule
            
        \end{tabular}
    \end{table}
\end{questionBox}

\begin{questionBox}2{ % Q1.5
    Cianeto de Potássio \ch{KCN}
} % Q1.5
    \begin{center}
        \ch{
            KCN + H2O -> K^{+} + CN^{-} + CH3Br -> CH3CN + KBr\aq
        }
    \end{center}
    \vspace{-5ex}
    \begin{table}[H]\centering
        \begin{tabular}{lll}
            
            \\\toprule
            
                \multicolumn{1}{c}{Composto}
            &   \multicolumn{1}{c}{Nome}
            &   \multicolumn{1}{c}{Papel}
            
            \\\midrule
            
                \ch{CN^{-}} & Ião Cianeto & Nucleofilo
            \\  \ch{CH3Br} & Brometo de Metilo & Substrato
            \\  \ch{CH3CN} & Cianeto de Metilo & Produto
            \\  \ch{Br^-} & Ião Brometo & Grupo abandonante
            
            \\\bottomrule
            
        \end{tabular}
    \end{table}
\end{questionBox}

\begin{questionBox}1{ % Q2
    Qual o produto orgânico obtido quando o \iupac{1-bromo-3-cloropropano} é colocado a reagir com 1 mole equivalente de cianeto de sódio em metanol aquoso?
} % Q2
    \begin{center}
        \ch{
            ClCH2CH2CH2Br + NaCN -> ClCH2CH2CH2CN + Br^{-}
        }
    \end{center}

    \paragraph*{Nota:} Brometo maior e menos eletronegativo que o cloreto
\end{questionBox}

\begin{questionBox}1{ % Q3
    A projecção de Fischer do (+)-2-bromooctano está abaixo representada. Escreva a projecção de Fisher do (-)-2-octanol, obtido por substituição nucleofílica do haleto, com inversão de configuração.
} % Q3
    \begin{center}
        \chemfig[angle increment=30]{
            (-[:-90]{C}(-[:0,1.3]{H2(CH2)4CH3}))
            (-[:  0]Br)
            (-[: 90]CH3)
            (-[:180]H)
        }
    \end{center}
\end{questionBox}

\end{document}
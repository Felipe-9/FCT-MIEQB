% !TEX root = ./QOG-Exercises_Resolutions.3.tex
\providecommand\mainfilename{"./QOG-Exercises_Resolutions.tex"}
\providecommand \subfilename{}
\renewcommand   \subfilename{"./QOG-Exercises_Resolutions.3.tex"}
\documentclass[\mainfilename]{subfiles}

% \graphicspath{{\subfix{./.build/figures/}}}
% \tikzset{external/force remake=true} % - remake all

\begin{document}

\mymakesubfile{2}
[QOG]
{Lista 3 Resolução: Esterioisomeria} % Subfile Title
{Lista: Esterioisomeria} % Part Title

\begin{questionBox}1{ % Q1
    Indique nos compostos seguintes, quais os centros estereogénicos:
} % Q1
    
\end{questionBox}

\setchemfig{
    angle increment={30}
}

\begin{questionBox}2{ % Q1.1
    \iupac{3-Bromo|pent|an|o}
} % Q1.1

    \begin{center}
        \chemfig{
            C
            (-[ 3]H)
            (-[-3]Br)
            (-[ 0]CH2-CH3)
            (-[ 6]CH2-[6]CH3)
        }
    \end{center}

    Não há centro esterio
\end{questionBox}

\begin{questionBox}2{ % Q1.2
    \iupac{1-Bromo-2-met|il|but|an|o}
} % Q1.2

    \begin{center}
        \chemfig{
            C^{*}
            (-[ 3]H)
            (-[-3]CH3)
            (-[ 0]CH2-Br)
            (-[ 6]CH2-[6]CH3)
        }
    \end{center}

\end{questionBox}

\begin{questionBox}2{ % Q1.2
    \iupac{2-Bromo-2-met|il|but|an|o}
} % Q1.2

    \begin{center}
        \chemfig{
            C
            (-[ 3]CH3)
            (-[-3]Br)
            (-[ 0]CH3)
            (-[ 6]CH2-[6]CH3)
        }
    \end{center}

    Não ha centro de esterio

\end{questionBox}

\begin{questionBox}1{ % Q2
    Localize planos e / ou centros de simetria nos compostos:
} % Q2
\end{questionBox}

\begin{questionBox}2{ % Q2.1
    \iupac{(Z)-1,2-Dicloro-eteno}
} % Q2.1
    
    \begin{center}
        \chemfig{
            C
            (-[( 1+6)]Cl)
            (-[(-1+6)]H)
            =C
            (-[( 1)]H)
            (-[(-1)]Cl)
        }
    \end{center}

\end{questionBox}

\begin{questionBox}2{ % Q2.2
    \iupac{cis-1,2-Di|cloro|ciclo|prop|an|o}
} % Q2.2
    
    \begin{center}
        \chemfig{
            C
            *3(
                -C
                (<:[( 1+0)]H)
                (<[(-1+0)]Cl)
                -C
                (<:[( 1+4)]H)
                (<[(-1+4)]Cl)
                -
            )
            (<[( 1+8)]H)
            (<:[(-1+8)]H3C)
        }
    \end{center}

\end{questionBox}

\setcounter{question}{3}

\begin{questionBox}1{ % Q4
    Assinale a configuração absoluta R ou S dos seguintes compostos:
} % Q4
    

\end{questionBox}

\begin{questionBox}2{ % Q4.1
    \iupac{(+)-2-Metil-1-butanol}
} % Q4.1

    \begin{center}
        \chemfig{
            C
            (<:[3]H)
            ( -[0]CH2OH)
            ( <[(-1+6)]H3C)
            ( -[(+1+6)]H3CH2C)
        }
    \end{center}

\end{questionBox}

\begin{questionBox}1{ % Q5
    O ácido tartárico, \ch{CO2HCH(OH)CH(OH)CO2H}, possui 3 estereoisómeros, apesar de possuir 2 centros estereogénicos. Desenhe-os utilizando a notação traço-cunha e projecções de Fischer e, atribua as configurações absolutas aos centros estereogénicos.
} % Q5
    
    \iupac{2,3-di-hidroxi-but|an|ó|ico}

    \begin{center}
        \chemfig{
            C
            (-[( 3+6)]H)
            (-[(-3+6)]OH)
            (-[( 0+6)]C(=[(1+6)]O)(-[(-1+6)]HO))
            -C
            (-[ 3]H)
            (-[-3]OH)
            (-[ 0]C(=[1]O)(-[-1]OH))
        }
    \end{center}

\end{questionBox}

\end{document}
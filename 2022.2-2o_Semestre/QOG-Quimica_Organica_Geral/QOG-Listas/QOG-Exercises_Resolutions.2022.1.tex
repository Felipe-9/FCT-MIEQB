% !TEX root = ./QOG-Exercises_Resolutions.2022.1.tex
\providecommand\mainfilename{"./QOG-Exercises_Resolutions.tex"}
\providecommand \subfilename{}
\renewcommand   \subfilename{"./QOG-Exercises_Resolutions.2022.1.tex"}
\documentclass[\mainfilename]{subfiles}

% \tikzset{external/force remake=true} % - remake all

\begin{document}

% \graphicspath{{\subfix{./.build/figures/QOG-Exercises_Resolutions.2022.1}}}
% \tikzsetexternalprefix{./.build/figures/QOG-Exercises_Resolutions.2022.1/graphics/}

\mymakesubfile{1}
[QOG]
{Exercicios 2022 Resoluções: Ligações Químicas} % Subfile Title
{Exercicios 2022 Resoluções: Ligações Químicas} % Part Title

\begin{questionBox}1{ % Q1
    Calcule a carga formal de cada um dos átomos nas seguintes estruturas de Lewis:
}
    \setchemfig{
        angle increment = 90
    }
    
    \begin{questionBox}2{ % Q1.1
        Tribrometo de fósforo
    }
        \begin{center}
            \chemfig[angle increment=120]{
                P^{+3}
                (-[0.75]Br^{-1})
                (-[1.75]Br^{-1})
                (-[2.75]Br^{-1})
            }
        \end{center}
    \end{questionBox}
    
    \begin{questionBox}2{
        Ácido Sufúrico
    } % Q1.2
        \begin{center}
            \schemestart
                \chemfig{
                    S
                    (-[0]O-[0]H)
                    (-[2]O-[2]H)
                    (-[1]O)
                    (-[3]O)
                }
                \arrow{<->}
                \chemfig{
                    S
                    (-[0]O-[0]H)
                    (-[2]O-[2]H)
                    (=[1]O)
                    (=[3]O)
                }
            \schemestop
        \end{center}
        \begin{multicols}{2}
           \begin{minipage}{.4\textwidth}
                \phantom{
                    Estrutura errada,
                    não respeita a regra do octeto
                }
            \end{minipage}
    
           \begin{minipage}{.4\textwidth}
                Estrutura errada,
                não respeita a regra do octeto
           \end{minipage}
        \end{multicols}
        
    \end{questionBox}
\end{questionBox}

\setcounter{question}{2}

\begin{questionBox}1{
    Expanda as fórmulas condensadas, mostrando as ligações e electrões não partilhados:
} % Q3
    
    
    \begin{questionBox}2{} % Q3.1
        
        \ch{ClCH2CH2Cl}

        \begin{center}
            \chemfig{
                C
                (-[ 1]H)
                (-[-1]H)
                (-[ 2]Cl)
                (
                    - C
                    (-[ 1]H)
                    (-[-1]H)
                    (-[ 0]Cl)
                )
            }
        \end{center}
        
    \end{questionBox}

    \begin{questionBox}2{} % Q3.2
        
        \ch{(CH3)3CH}

        \begin{center}
            \chemfig{
                C
                (-H)
                (
                    -[1,1.5]C
                    (-[0]H)
                    (-[1]H)
                    (-[2]H)
                )
                (
                    -[2,1.5]C
                    (-[1]H)
                    (-[2]H)
                    (-[3]H)
                )
                (
                    -[3,1.5]C
                    (-[0]H)
                    (-[2]H)
                    (-[3]H)
                )
            }
        \end{center}
        
    \end{questionBox}

    \begin{questionBox}2{} % Q3.3
        
        \ch{(CH3)2CHCH=O}

        \begin{center}
            \chemfig{
                C
                (-[3]H)
                (-[1,1.5]C
                    (-[0]H)
                    (-[1]H)
                    (-[2]H)
                )
                (-[2,1]C
                    (-[1]H)
                    (-[2]H)
                    (-[3]H)
                )
                (-[0]C
                    (-[1]H)
                    (=[0]O)
                )
            }
        \end{center}
        
    \end{questionBox}
\end{questionBox}


\end{document}
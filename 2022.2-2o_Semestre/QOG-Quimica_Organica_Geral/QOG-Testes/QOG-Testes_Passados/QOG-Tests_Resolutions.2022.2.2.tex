% !TEX root = ./QOG-Tests_Resolutions.2022.2.2.tex
\providecommand\mainfilename{"./QOG-Tests_Resolutions.tex"}
\providecommand \subfilename{}
\renewcommand   \subfilename{"./QOG-Tests_Resolutions.2022.2.2.tex"}
\documentclass[\mainfilename]{subfiles}

% \tikzset{external/force remake=true} % - remake all

\begin{document}

% \graphicspath{{\subfix{./.build/figures/QOG-Tests_Resolutions.2022.2.2}}}
% \tikzsetexternalprefix{./.build/figures/QOG-Tests_Resolutions.2022.2.2/graphics/}

\mymakesubfile{2}
[QOG]
{Teste 2022.2 Resolução} % Subfile Title
{Teste 2022.2 Resolução} % Part Title

% \renewcommand\thesubquestion{Q\arabic{question} \alph{subquestion})}

\begin{questionBox}1{ % Q1
    Quais dos seguintes compostos formam ligações de hidrogénio entre as suas moléculas:
} % Q1
    \begin{enumerate}
        \begin{multicols}{3}
            \item \ch{CH3CH2OH}
            \item \ch{CH3CH2SH}
            \item \ch{CH3OCH2CH3}
        \end{multicols}
    \end{enumerate}
\end{questionBox}

\begin{questionBox}1{ % Q2
    Qual a conformação mais estável do cis-1,2-dibromociclohexano? Justifique desenhando a estrutura.
} % Q2
    body
\end{questionBox}

\begin{questionBox}1{ % Q3
    Que nucleófilos usaria para fazer reagir com o brometo de butilo e preparar os seguintes compostos?
} % Q3
    \begin{questionBox}2{ % Q3.1
        \ch{CH3CH2CH2CH2OH}
    } % Q3.1
        \begin{center}
            \ch{
                CH3CH2CH2CH2Br + OH^{-} -> CH3CH2CH2CH2OH
            }
        \end{center}
    \end{questionBox}

    \begin{questionBox}2{ % Q3.2
        \ch{CH3CH2CH2CH2OCH3}
    } % Q3.1
        \begin{center}
            \ch{
                CH3CH2CH2CH2Br + OCH3^{-} -> CH3CH2CH2CH2OCH3
            }
        \end{center}
    \end{questionBox}
\end{questionBox}

\end{document}
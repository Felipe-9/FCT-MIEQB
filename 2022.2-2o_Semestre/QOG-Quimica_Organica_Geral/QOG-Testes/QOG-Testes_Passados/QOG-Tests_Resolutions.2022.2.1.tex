% !TEX root = ./QOG-Tests_Resolutions.2022.2.1.tex
\providecommand\mainfilename{"./QOG-Tests_Resolutions.tex"}
\providecommand \subfilename{}
\renewcommand   \subfilename{"./QOG-Tests_Resolutions.2022.2.1.tex"}
\documentclass[\mainfilename]{subfiles}

% \tikzset{external/force remake=true} % - remake all

\begin{document}

% \graphicspath{{\subfix{./.build/figures/QOG-Tests_Resolutions.2022.2.1}}}
% \tikzsetexternalprefix{./.build/figures/QOG-Tests_Resolutions.2022.2.1/graphics/}

\mymakesubfile{1}
[QOG]
{Teste 2022.1 Resolução} % Subfile Title
{Teste 2022.1 Resolução} % Part Title

\begin{questionBox}1{ % Q1
    Desenhe espécies químicas com as seguintes características:
} % Q1
    \begin{multicols}{2}
        
        \begin{questionBox}2{ % Q1.1
            Uma estrutura contendo carbono com carga negativa
        } % Q1.1
            \begin{center}
                \chemfig[angle increment=30]{
                    C^{-}
                    (-[  7]H)
                    (-[- 1]H)
                    (-[- 2]H)
                }
            \end{center}
        \end{questionBox}

        \begin{questionBox}2{ % Q1.2
            Uma estrutura contendo azoto com carga positiva
        } % Q1.1
            \begin{center}
                \chemfig[angle increment=30]{
                    N^{+}
                    (-[  7]H)
                    (-[- 1]H)
                    (-[- 2]H)
                    (-[- 9]H)
                }
            \end{center}
        \end{questionBox}
        
    \end{multicols}

    % \begin{questionBox}2{ % Q1.3
    %     Qual a estrutura mais estável? Justfique
    % } % Q1.3
    %     \ch{NH^{3+}}
    % \end{questionBox}

\end{questionBox}

\begin{questionBox}1{ % Q2
    Desenhe as estruturas dos seguintes compostos:
} % Q2
    \begin{multicols}{2}
        \begin{questionBox}2{ % Q2.1
            \iupac{Iod|eto de iso|prop|il|o}
        } % Q2.1
            \begin{center}
                \chemfig[angle increment=30]{
                    C
                    (-[3]I)
                    (-[-1.2]CH3)
                    (-[(6+1.2)]CH3)
                }
            \end{center}
        \end{questionBox}

        \begin{questionBox}2{ % Q2.2
            \iupac{Sec-but|il|amina}
        } % Q2.2
            \begin{center}
                \chemfig[angle increment=30]{
                    -[-1]
                    -[+1]
                    (-[-1])
                    (-[+3]NH3)
                }
            \end{center}
        \end{questionBox}
    \end{multicols}
\end{questionBox}

\end{document}
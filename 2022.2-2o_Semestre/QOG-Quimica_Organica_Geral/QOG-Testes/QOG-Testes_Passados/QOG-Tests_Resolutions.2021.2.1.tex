% !TEX root = ./QOG-Tests_Resolutions.2021.2.1.tex
\providecommand\mainfilename{"./QOG-Tests_Resolutions.tex"}
\providecommand \subfilename{}
\renewcommand   \subfilename{"./QOG-Tests_Resolutions.2021.2.1.tex"}
\documentclass[\mainfilename]{subfiles}

% \tikzset{external/force remake=true} % - remake all

\begin{document}

% \graphicspath{{\subfix{./.build/figures/QOG-Tests_Resolutions.2021.2.1}}}
% \tikzsetexternalprefix{./.build/figures/QOG-Tests_Resolutions.2021.2.1/graphics/}

\mymakesubfile{1}
[QOG]
{Teste 2021.1 Resolução} % Subfile Title
{Teste 2021.1 Resolução} % Part Title

\renewcommand\thesubquestion{Q\arabic{question} \alph{subquestion})}

\begin{questionBox}1{ % Q1
    Use os símbolos \chemdelta+ e \chemdelta- para indicar a polaridade na ligação assinalada em cada um dos compostos e indique, justificando, qual das ligações é mais polar.
} % Q1
    \begin{multicols}{3}

        \begin{questionBox}2{ % Q1.1
            \ch{HO-H}
        } % Q1.1
            \paragraph*{RS:}
            \ch{HO^{\chemdelta-}-H^{\chemdelta+}}
        \end{questionBox}

        \begin{questionBox}2{ % Q1.2
            \ch{H3C-NH2}
        } % Q1.2
            \paragraph*{RS:}
            \ch{H3C^{\chemdelta^+}-N^{\chemdelta-}H}
        \end{questionBox}

        \setcounter{subquestion}{3}

        \begin{questionBox}2{ % Q1.4
            \ch{H2N-OH}
        } % Q1.4
            \paragraph*{RS:}
            \ch{H2N^{\chemdelta-}-O^{\chemdelta+}H}
        \end{questionBox}

    \end{multicols}

    \setcounter{subquestion}{2}

    \begin{questionBox}2{ % Q1.3
        \ch{F-Br}
    } % Q1.3

        \begin{answerBox}{
            \ch{F^{\chemdelta-}-Br^{\chemdelta+}}
        } % RS 
            É a ligação mais polar, não só pela diferença de EN mas tambem pelo tamanho atomico do Bromo, que terá muito menos influência nos eletrons da camada de valência
        \end{answerBox}
    \end{questionBox}
\end{questionBox}


\begin{questionBox}1{ % Q2
    Desenhe as estruturas dos seguintes compostos:
} % Q2
    \begin{multicols}{2}
        \setchemfig{angle increment=10}
        
        \begin{questionBox}2{ % Q2.1
            \iupac{Alcool-Iso|prop|ílico}
        } % Q2.1

            \begin{description}
                \item[Alcool] deve conter pelo menos um grupo \ch{OH}
                \item[Iso] O grupo está localizado no meio criando uma simetria
                \item[Prop] Cadeia de 3 Carbonos
            \end{description}
            \begin{center}
                \chemfig[angle increment=30]{
                    C
                    (-[-3]OH)
                    (
                        -[(+1+0)]C(-[(+3+0)]H)
                        -[(-1+0)]H
                    )
                    (
                        -[(-1-6)]C(-[(-3+6)]H)
                        -[(+1-6)]H
                    )
                }
            \end{center}
        \end{questionBox}

        \begin{questionBox}2{ % Q2.2
            \iupac{Iod|eto de \textit{sec}-but|ilo}
        } % Q2.2
            \begin{description}
            \item[Iod] Radical de Iodo
            \item[\textit{sec}] segundo carbono da cadeia
            \item[but] Cadeia de 4 carbonos
            \end{description}

            \chemfig[angle increment=30]{
                C(-[5]H)(-[-3]H)
                -[ 1]C(-[ 3]I)
                -[-1]C(-[-3]H)
                -[ 1]C(-[ 3]H)(-[-1]H)
            }

        \end{questionBox}

        \begin{questionBox}2{ % Q2.3
            \iupac{tert-but|il|amina}
        } % Q2.3
            \begin{description}
            \item[\textit{Tert}] Tetraedro
            \item[but] Cadeia de 3 Carbonos
            \item[Amina] Grupo amina (\ch{NH2})
            \end{description}

            \begin{center}
                \chemfig[angle increment=15]{
                    C
                    (-[  6]NH2)
                    (-[- 1]CH)
                    (-[- 3]CH)
                    (-[-10]CH)
                }
            \end{center}

        \end{questionBox}

    \end{multicols}
\end{questionBox}

\end{document}
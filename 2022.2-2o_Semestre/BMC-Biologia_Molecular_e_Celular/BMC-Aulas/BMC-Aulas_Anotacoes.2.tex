% !TEX root = ./BMC-Aulas_Anotações.2.tex
\providecommand\mainfilename{"./BMC-Aulas_Anotações.tex"}
\providecommand \subfilename{}
\renewcommand   \subfilename{"./BMC-Aulas_Anotações.2.tex"}
\documentclass[\mainfilename]{subfiles}

% \graphicspath{{\subfix{../images/}}}
% \tikzset{external/force remake=true} % - remake all

\begin{document}

\mymakesubfile{2}
[BMC]
{Aula}
{Aula}

\begin{sectionBox}1{Reticulo Endoplasmático}
    
    O retículo endoplasmático, ou ergastoplasma, é uma organela exclusivo de células eucariontes. Formado a partir da invaginação da membrana plasmática, constituído por uma rede de túbulos e vesículas achatados e interconectados, que se comunicam com o envoltório nuclear (carioteca).
    
\end{sectionBox}

\begin{sectionBox}2{Sintese de po}
    
    body
    
\end{sectionBox}

\end{document}
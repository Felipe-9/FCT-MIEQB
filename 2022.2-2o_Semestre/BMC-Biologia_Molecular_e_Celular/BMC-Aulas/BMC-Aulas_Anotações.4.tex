% !TEX root = ./BMC-Aulas_Anotações.4.tex
\providecommand\mainfilename{"./BMC-Aulas_Anotações.tex"}
\providecommand \subfilename{}
\renewcommand   \subfilename{"./BMC-Aulas_Anotações.4.tex"}
\documentclass[\mainfilename]{subfiles}

\graphicspath{{\subfix{./.build/figures/BMC-Aulas_Anotações.4/}}}
% \tikzset{external/force remake=true} % - remake all

\begin{document}

\mymakesubfile{4}
[BMC]
{Aula 02/11: Citoesqueleto} % Subfile Title
{Aula 02/11: Citoesqueleto} % Part Title

\begin{sectionBox}{} % S0
    
    \begin{itemize}
        \begin{multicols}{2}
            \item Microtúbulos
            \item microfilamentos
            \item filamentos intermédios
            \item Organização
            \item estrutura
            \item função
            \item Transporte intracelular
        \end{multicols}
    \end{itemize}
    
\end{sectionBox}

\begin{sectionBox}1{Filamentos Intermédios} % S1
    
    \begin{center}
        \includegraphics[width=.7\textwidth]{Screenshot 2022-11-02 at 12.06.30}
    \end{center}
    
\end{sectionBox}

\begin{sectionBox}1{Proteínas motoras dof filamentos de actina} % S2
    
    \begin{center}
        \includegraphics[width=.7\textwidth]{Screenshot 2022-11-02 at 12.10.03}
    \end{center}

    \begin{sectionBox}2{Complexo golgi} % S2.1
        
        Elas auxiliam no transporte de proteínas dentro de vesiculas, as vesiculas são elvoltas de indicadores que são reconhecidos pelos transportadoes e o tipo de indicadores são reconhecidos por transportadores seguindo diferentes caminhos
        
    \end{sectionBox}
    
\end{sectionBox}

\begin{sectionBox}1{Proteínas em volta de microtubolos} % S3
    
    Proteínas que tem afinidade pelos microtubulos almentando sua resistencia e eveitando a despolimerização, caso fosforiladas perdem afinidade
    
\end{sectionBox}




\end{document}
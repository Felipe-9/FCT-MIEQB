% !TEX root = ./BMC-Aulas_Anotações.3.tex
\providecommand\mainfilename{"./BMC-Aulas_Anotações.tex"}
\providecommand \subfilename{}
\renewcommand   \subfilename{"./BMC-Aulas_Anotações.3.tex"}
\documentclass[\mainfilename]{subfiles}

\graphicspath{{\subfix{./.build/figures/BMC-Aulas_Anotações.3}}}
% \tikzset{external/force remake=true} % - remake all

\begin{document}

\mymakesubfile{3}
[BMC]
{Aula 26/10/2022: Sinalização Celular} % Subfile Title
{Aula 26/10/2022: Sinalização Celular} % Part Title

\begin{sectionBox}1{Sinalização Endroquina} % S1
    
    Em biologia, a sinalização celular ou comunicação celular é a capacidade de uma célula de receber, processar e transmitir sinais com seu ambiente e consigo mesma. A sinalização celular é uma propriedade fundamental de toda a vida celular em procariontes e eucariontes.
    
\end{sectionBox}

\begin{sectionBox}1{Tipos de Cinalização} % S2
    
    \begin{multicols}{2}
        \paragraph*{Paraquina:} As celulas vizinhas
        \paragraph*{Autócrina:} A si própria
    \end{multicols}
    
\end{sectionBox}

\begin{sectionBox}2{Sinalização Paraquina} % S2.1
    
    A sinalização paracrina é uma forma de sinalização celular, um tipo de comunicação celular na qual uma célula produz um sinal para induzir mudanças nas células próximas, alterando o comportamento dessas células
    
\end{sectionBox}

\begin{sectionBox}2{Cinalização Autocrina} % S2.2
    
    A sinalização autocrina é uma forma de sinalização celular na qual uma célula secreta um hormônio ou mensageiro químico que se liga aos receptores autocrinos nessa mesma célula, levando a mudanças na célula. Isso pode ser contrastado com sinalização paracrina, sinalização intracrina ou sinalização endócrina clássica.
    
\end{sectionBox}

\begin{sectionBox}2{Sinalização de Contato} % S2.3
    
    Tipo de cinalização paraquína

    \paragraph*{Exemplo:} Quando uma célula sanguinea entra em contato com a parede celular que incita uma modificação do citoesqueleto permitindo a passagem pela membrana
    
\end{sectionBox}

\begin{sectionBox}1{Classes de Receptores de Superfície Celular} % S3
    
    \paragraph*{Receptores linkados a}
    \begin{itemize}
        \item canais ionicos
        \item G-Proteínas
        \item Enzimas
    \end{itemize}
    
\end{sectionBox}

\begin{sectionBox}2{G-Proteínas} % S3.1
    
    Proteína receptora ligada a G-Proteína faz 7 voltas na membrana celular 

    Ligada a neurotransmissores

    \begin{sectionBox}3{G-Proteínas} % S3.1 (i)
        
        Possuí 3 partes que se dividem em duas durante a operação

        \begin{itemize}
            \begin{multicols}{2}
                \item \(G_{\chemalpha}\)
                \item \(G_{\chembeta}\)
                \item \(G_{\chemgamma}\)
            \end{multicols}
        \end{itemize}
        
    \end{sectionBox}
    
\end{sectionBox}

\begin{sectionBox}1{Mensageiros secundarios} % S4
    
    \begin{itemize}
        \begin{multicols}{3}
            \item \ch{Ca+}
            \item Ciclico AMP
            \item Ciclico GMP
            \item \ch{IP3}
            \item Diacilglicerol
        \end{multicols}
    \end{itemize}
    
\end{sectionBox}

\begin{definitionBox}1{Variedade de Receptores} % DEF 1
    
    As celulas não invéstem muito na variedade de mensageiros e sim na variedade de receptores
    
\end{definitionBox}

\begin{sectionBox}1{Vias Canônicas} % S5
    
    \begin{center}
        \includegraphics[width=.9\textwidth]{Screenshot 2022-10-26 at 12.10.24}
    \end{center}

    \begin{sectionBox}2{Via Amarela} % S5.1
        
        \paragraph*{Mensageiros secundários}
        \begin{itemize}
            \begin{multicols}{2}
                \item Dioglicerol
                \item IP3
            \end{multicols}
        \end{itemize}
        % \paragraph*{Phospholipase C}
        % Usa como substrato fosfolipedos da membrana e converteos em algo q 
        % é soluvel e vai ao citosol
        
    \end{sectionBox}
    
\end{sectionBox}

\begin{sectionBox}2{Via do Diaglicerol (DAG)} % S
    
    \begin{center}
        \includegraphics[width=.9\textwidth]{Screenshot 2022-10-26 at 12.17.11}
    \end{center}
    
\end{sectionBox}

\begin{sectionBox}2{} % S
    
    \begin{center}
        \includegraphics[width=.9\textwidth]{Screenshot 2022-10-26 at 12.18.41}
    \end{center}

    \paragraph*{PKA}
    \begin{itemize}
        \item Inativa quando as bolas estão ligadas aos receptores
        \item Desbolqueada quando AMP ciclico se ligam
        \item Quando ativa fosforila seu substrato
    \end{itemize}
    
    \paragraph*{Substrato}
    \begin{itemize}
        \item Proteínas dentro da celula
    \end{itemize}

    \paragraph*{}
    
    
\end{sectionBox}

\begin{sectionBox}3m{Migração ao nucleo} % 
    
    \begin{center}
        \includegraphics[width=.6\textwidth]{Screenshot 2022-10-26 at 12.23.18}
    \end{center}

    Pode migrar para o nucleo quando ativa, fosforilando um fator de transcrissão de uma sequencia de DNA, a promovendo.
    
\end{sectionBox}

\begin{sectionBox}1{Visão macro} % S1
    
    \begin{center}
        \includegraphics[width=.32\textwidth]{Screenshot 2022-10-26 at 12.29.40}
        \includegraphics[width=.32\textwidth]{Screenshot 2022-10-26 at 12.37.30}
        \includegraphics[width=.32\textwidth]{Screenshot 2022-10-26 at 12.38.12}
    \end{center}

    \paragraph*{Vantagens de várias vias}
    \begin{itemize}
        \item Com poucos recursos (transmissores) podemos desencadear uma grande reação da celula (varias vias) assim poupando energia
        \item Existem processos regulatórios para controlar a intensidade de sinais evitando um "efeito em cadeia explosivo"
        \item O mesmo receptor pode rectrutar varias proteínas-G
    \end{itemize}

    
\end{sectionBox}

\begin{sectionBox}1{Modelo de ativação genética} % S1
    
    Modelo da activação génica por um receptor nuclear homodimérico na presença de uma hormona

    \begin{center}
        \includegraphics[width=.8\textwidth]{Screenshot 2022-10-26 at 12.39.58}
    \end{center}
    
\end{sectionBox}

\end{document}
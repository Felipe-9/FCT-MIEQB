% !TEX root = ./BMC-Exercicios_Resolução.2.tex
\providecommand\mainfilename{"./BMC-Exercicios_Resolução.tex"}
\providecommand \subfilename{}
\renewcommand   \subfilename{"./BMC-Exercicios_Resolução.2.tex"}
\documentclass[\mainfilename]{subfiles}

% \graphicspath{{\subfix{../images/}}}
% \tikzset{external/force remake=true} % - remake all

\begin{document}

\mymakesubfile{2}
[BMC]
{Exercicios}
{Exercicios}

% Q1
\begin{questionBox}1{}
    
    Os proteoglicanos são proteínas altamente glicosiladas (N- e O-glicosilações) que formam a matrix extracelular, tendo uma enorme capacidade de reter água. São, portanto, os principais componentes das cartilagens. Indique a ordem cronológica dos processos envolvidos desde a sua síntese até atingir a sua localização final na célula.

    \begin{enumerate}[label=\alph{enumi}. ]
        \item Sínteseproteicainiciadanocitoplasma
        \item transporteparaoGolgi
        \item Transcrição efectuada no núcleo
        \item translocação da porção da proteína com ribossoma para retículo endoplasmático
        \item reconhecimento da sequência sinal para translocar para retículo endoplasmático
        \item síntese completa e N-glicosilação
        \item reformulaçãodeN-glicosilaçãoeadiçãodeO-glicosilação
        \item exocitoseparameioextracelular
        \item vesículas secretórias
    \end{enumerate}

    \paragraph*{RS}
    c - a - e - d - f - b - g - i - h
    
\end{questionBox}

% Q2
\begin{questionBox}1{}
    
    Em geral, numa célula eucariota, a regulação da expressão e actividade de uma proteína podem ocorrer a vários níveis, desde a transcrição até modificações pós- tradução. Dê exemplos de modificações pós-tradução que possam alterar a função de uma proteína.

    \paragraph*{RS}
    Conjugação com outras proteínas ou peptídeos.
    Alteração da estrutuda da proteína, uma "Ativação" almentando seu efeito, exemplo disso é a conversão de pro-homonios em hormonios.
    
\end{questionBox}

% Q3
\begin{questionBox}1{}
    
    Que componente do sistema endomembranoso sintetisa lipídos e esteróides?

    \begin{enumerate}[label=\alph{enumi}) ]
        \item Golgi
        \item RE liso
        \item RE rugoso
        \item Ribossoma        
    \end{enumerate}

    \paragraph*{RS} b) - RE Liso
    
\end{questionBox}

% Q4
\begin{questionBox}1{}
    
    O transporte de vesiculas do RE para o Golgi é feito por difusão? Justifique

    \paragraph*{RS}
    Não, As vesiculas transportadoras possuem proteínas signaladas que regem seu transporte, aquelas com o sinal exit/transport são dirigidas pelas vias do citoesqueleto para o Golgi
    
\end{questionBox}

% Q5
\begin{questionBox}1{}
    
    Em relação ao complexo de Golgi, indique a opção incorreta

    \begin{enumerate}[label=\alph{enumi}) ]
        \item É composto por um conjunto de compartimentos membranares (cisternas).
        \item As proteínas que saem da face trans poderão ser transportadas para lisossomas, vesículas secretórias ou para a membrana plasmática.
        \item As proteínas que entram na face cis, vindas do retículo endoplasmático, são exclusivamente transportadas ao longo do complexo de Golgi, no sentido da face trans.
        \item É organizado de modo a que cada cisterna tenha um conjunto de enzimas específico e diferente de outra cisterna
    \end{enumerate}

    \paragraph*{RS} c) - As proteínas que entram na face cis, vindas do retículo endoplasmático, são exclusivamente transportadas ao longo do complexo de Golgi, no sentido da face trans.
    
\end{questionBox}

% Q6
\begin{questionBox}1{}
    
    As proteínas que não têm o "folding" (estrutura tri-dimensional) correcto:
    \begin{enumerate}[label=\alph{enumi})]
        \item são encapsuladas em vesículas e transportadas para o aparelho de Golgi para serem destruídas
        \item são glicosiladas no aparelho de Golgi
        \item sofrem acção de chaperones no lumen do retículo e poderão são degradadas no citosol
        \item são degradadas, numa reacção em cadeia, por diferentes proteases da membrana do retículo liso
    \end{enumerate}

    \paragraph*{RS} d) são degradadas, numa reacção em cadeia, por diferentes proteases da membrana do retículo liso
    
\end{questionBox}

\begin{questionBox}1{}
    
    Embora as vesículas possam parecer muito semelhantes ao microscópio, elas devem ser especializadas o suficiente para distinguir um organelo de outro, e diferentes cargas a tansportar. Quantas propriedades distintivas de vesículas consegue prever?

    \paragraph*{RS}
    As vesiculas são geradas com base na carga proteica que sai do ER, para formar a vesicula a carga interage com proteínas membranares do ER especificas para o transporte da proteína assim criando uma vesicula com proteínas membranares específicas para seu transporte, assim o diferencial da vesiculas está em suas proteínas, alguns dos sinais de transporte são os seguintes:
    \begin{itemize}
        \item Importar ao ER
        \item Importar a mitocondria
        \item Importar no Núcleo
        \item Importar ao peroxisomo
        \item transportar ao Golgi
    \end{itemize}
    
\end{questionBox}

\begin{questionBox}1{}
    
    A figura representa o resultado de uma experiência de pulso e caça para determinar o movimento de proteínas nas células. As células foram incubadas com um aminoácido radioactivo (3H-leucina) que será incorporado nas proteínas sintetizadas durante o período da experiência, permitindo a sua marcação e consequente detecção por autorradiografia. As células são lavadas em tampão para remover o pulso e são transferidas para um meio sem o precursor radioactivo, a caça. São recolhidas amostras periodicamente e analisadas por autorradiografia para determinar a localização celular de proteínas marcadas, tal como indicado na figura.


    \begin{center}
        \includegraphics[width=.9\textwidth]{BMC-Exericicios_Resolução.2.1}
    \end{center}
    
\end{questionBox}

\begin{questionBox}2{}
    
    Interprete os resultados da experiência de pulso e caça.

    \paragraph{RS} 
    Em \(T=0\) o citoplasma das celulas entram em contato com as proteínas marcadas pela primeira vez,
    em \(T=5'\) percebemos a presença das pronteínas no ER, indicando que estas foram transportadas para serem processadas, em diante vemos elas serem transportadas ao Golgi e eventualmente serem secretadas.
    A experiencia permite perceber o percurso das proteínas na celula, do contato a secressão.
    
\end{questionBox}

\begin{questionBox}2{}
    
    Se usasse a Brefeldina A, um inibidor do transporte vesicular RE-Golgi, qual o resultado esperado para este ensaio?

    \paragraph*{RS} Um acumulo das proteínas no RE, eventualmente este poderia acionar uma via biológica em resposta ao impedimento.

\end{questionBox}

\begin{questionBox}2{}
    
    Se ao \(T=45\,\unit{\min}\) deixasse de ver marcação, como interpretava esse resultado? Como comprovaria a sua hipótese?

    \paragraph*{RS}
    \begin{itemize}
        \item Proteínas teriam sido completamente secretadas
        \item Proteínas nunca saíram do Golgi devem ter sido modificadas a ponto de perder a característica de sinal, para essa hipótese ser válida teriamos de investigar a capacidade do complexo de modificar aminoácidos radioativos.
    \end{itemize}
    
\end{questionBox}

\end{document}
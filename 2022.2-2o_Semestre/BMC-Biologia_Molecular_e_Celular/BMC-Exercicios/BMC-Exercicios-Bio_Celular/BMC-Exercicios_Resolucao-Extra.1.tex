% !TEX root = ./BMC-Exercicios_Resolução-Extra.1.tex
\providecommand\mainfilename{"./BMC-Exercicios_Resolução.tex"}
\providecommand \subfilename{}
\renewcommand   \subfilename{"./BMC-Exercicios_Resolução-Extra.1.tex"}
\documentclass[\mainfilename]{subfiles}

% \graphicspath{{\subfix{../images/}}}
% \tikzset{external/force remake=true} % - remake all

\begin{document}

\mymakesubfile{4}
[BMC]
{Exercícios extra PubMed} % Subfile Title
{Extra: PubMed} % Part Title

\begin{questionBox}1m{Desafio} % Q1
    
    Precisamos de marcar núcleos em células em cultura, para identificar o no de células na preparação.\\

    Após uma pesquisa na PubMed, indique um método utilizado na investigação científica que permita marcar núcleos.\\

    Indique o nome da técnica utilizada e ilustre a sua resposta com imagens reais, com as respectivas legendas, retiradas de dois artigos científicos. Faça a lista referências bibliográficas utilizadas (formato ao seu critério).\\

    O seu trabalho deve conter os passos todos desde a pesquisa na PubMed até à escolha dos artigos (indique passo a passo o que fez).\\

    \paragraph*{RS:}

    Foram feitas pesquisas com os seguintes queries (sem contar variações que incluem "") sem sucesso:
    \begin{itemize}
        \begin{multicols}{2}
            \item Nucleus Marking
            \item Cell culture nucleus Marking
            \item Nucleus Marking count cell number
            \item Nucleus Singalling
            \item Nucleus Singalling Cell culture
            \item Nucleus Singalling cell counting
        \end{multicols}
    \end{itemize}

    Então voltei ao problema e tentei fazer uma pesquisa com base no objetivo em contexto:
    \begin{itemize}
        \begin{multicols}{2}
            \item Cell counting
            \item Cell counting nucleus
        \end{multicols}
    \end{itemize}

    Com a ultima pesquísa li brevemente as descrições dos artigos e encontrei o que queria encontrar no segundo.

    \begin{center}
        \includegraphics[width=1\textwidth]{
            ./.build/figures/Screenshot 2022-10-19 at 20.50.39.png
        }
        \includegraphics[width=1\textwidth]{
            ./.build/figures/Screenshot 2022-10-19 at 21.01.20.png
        }
    \end{center}

    A unica mensão sobre o nome do método é ``Nucleus Dyeing'' e ultiliza 6-diamidino-2-phenylindole (DAPI) \cite{r1}\\

    Para encontrar o segundo artigo usei duas keywords encontradas no primeiro com um modelo mais avançado de pesquisa que vim a descobrir.

    \begin{center}
        \includegraphics[width=\textwidth]{
            ./.build/figures/Screenshot 2022-10-19 at 22.32.21.png
        }
        \includegraphics[width=\textwidth]{
            ./.build/figures/Screenshot 2022-10-19 at 22.38.03.png
        }
    \end{center}

    Dentre os primeiros artigos ja encontrei um resultado. \cite{r2}


    % \bibliography{./.build/libraries/mylib.1.bib}
    % \bibliography{mylib.1}
        
\end{questionBox}

\begin{questionBox}1m{Pesquisa por autor} % Q2
    
    Uma outra opção que o sistema permite (entre muitas que podem explorar) é a pesquisa por autor, ie, conhecermos o autor e querermos saber o que esse autor publica.\\

    Exemplo: Christian de Duve

    Palavras de pesquisa: Duve C (a inicial do 1o nome vem depois do apelido)\\

    Sem recurso a mais nenhuma ferramenta de pesquisa, indiquem qual a área de investigação deste autor e, se conseguirem, qual a sua grande contribuição para a Biologia Celular?\\

    Indique 3 ou 4 referências que corroborem a sua resposta.\\


    \paragraph{RS} Inicio minha pesquisa com o query trivial
    \begin{center}
            \includegraphics[width=\textwidth]{
                ./.build/figures/Screenshot 2022-10-19 at 22.47.21.png
            }
            \includegraphics[width=\textwidth]{
                ./.build/figures/Screenshot 2022-10-19 at 23.01.40.png
            }
    \end{center}
    \cite{r4,r5,r6}

    Inicialmente é visto vários artigos dedicados a organelas específicas, pesquisando por intervalo de tempo podemos ver o foco de suas pesquisas mudando de organela para organela, até se encontrar o seguinte artigo: ``The peroxisome: a new cytoplasmic organelle'' \cite{r3} apontando sua descoberta
    
\end{questionBox}

\bibliographystyle{plain}
\bibliography{./.build/libraries/mylib.1}

\end{document}
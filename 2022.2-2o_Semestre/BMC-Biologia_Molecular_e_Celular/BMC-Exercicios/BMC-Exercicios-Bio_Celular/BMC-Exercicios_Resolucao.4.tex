% !TEX root = ./BMC-Exercicios_Resolução.4.tex
\providecommand\mainfilename{"./BMC-Exercicios_Resolução.tex"}
\providecommand \subfilename{}
\renewcommand   \subfilename{"./BMC-Exercicios_Resolução.4.tex"}
\documentclass[\mainfilename]{subfiles}

\graphicspath{{\subfix{./.build/figures/BMC-Exercicios_Resolução.4}}}
% \tikzset{external/force remake=true} % - remake all

\begin{document}

\mymakesubfile{5}
[BMC]
{Exercícios 5 Mitocondria}
{Mitocondria}

\begin{questionBox}1m{} % Q1
    
    Células de fígado de rato foram homogeneizadas e o extracto foi submetido a centrifugação em gradiente de equilíbrio de densidade (ver imagem abaixo), com gradientes de sucrose.

    \begin{center}
        \includegraphics[width=.8\textwidth]
        {./.build/figures/Screenshot 2022-10-19 at 14.57.22.png}
    \end{center}

    As fracções obtidas a partir desses gradientes foram testadas para moléculas marcadoras (isto é, moléculas características de organelos específicos). O resultado destes ensaios está representado na figura abaixo.
    
    \begin{center}
        \includegraphics[width=.8\textwidth]{
            ./.build/figures/Screenshot 2022-10-19 at 15.00.48.png
        }
    \end{center}

    As moléculas marcadoras têm as seguintes funções:
    \begin{itemize}
        \item \textbf{Citocromo oxidase} -- (ou complexo IV) última proteína da cadeia tranaportadora de electrões.
        \item \textbf{RNA ribossomal} -- componente primário dos ribossomas, envolvido na síntese proteica 
        \item \textbf{Catalase} -- enzima que catalisa a decomposição do peróxido de hidrogénio
        \item \textbf{Fosfatase acídica} -- enzima que remove um grupo fosfato do seu substrato, em pH ácido 
        \item \textbf{Citidilil transferase} -- enzima envolvida na síntese de fosfolípidos
        \item \textbf{Permease de Aminoácidos} -- proteína transportadora de aminoácidos através das membranas.
    \end{itemize}

    Com base nestas informações, indique a molécula marcadora e o número da fracção que é mais enriquecida em cada um dos seguintes organelos:

\end{questionBox}

\begin{multicols}{2}

    \begin{questionBox}2{} % Q1.1
        
        Retículo endoplasmático rugoso.
        
        \paragraph*{RS:} Citidilil Transferase, 11.25\%
        \paragraph*{RS:} Fração 15 - 
        
    \end{questionBox}
    
    \begin{questionBox}2{} % Q1.2
        
        Mitocôndrias.

        \paragraph{RS:} Citocromo oxidase, 5\%
        
    \end{questionBox}
    
    
    \begin{questionBox}2{} % Q1.3
        
        Lisossomas.

        \paragraph*{RS:} Fosfatase Acídica 22\%

        
    \end{questionBox}
    
    \begin{questionBox}2{} % Q1.4
        
        Peroxissoma.

        \paragraph{RS:} Catalase, 7.5\%
        
    \end{questionBox}
    
    \begin{questionBox}2{} % Q1.5
        
        Reticulo endoplasmático liso.

        \paragraph*{RS:} Permease de Aminoácidos
        
    \end{questionBox}

\end{multicols}


\begin{questionBox}2{} % Q1.6
    
    Os lisossomas serão mais ou menos densos que as mitocôndrias?
    


    Justifique com base na informação da figura.

\end{questionBox}

\begin{questionBox}1{} % Q2
    
    Explique de que forma a estrutura da mitocôndria suporta a sua função.
    
    \paragraph*{RS:} As mitocondrias possuem uma dupla camada de membramas plasmáticas criando um micro "Citoplasma" que varia de propríedades com o comum da célula, sua membrana interna possui uma extensão muito maior que a interna sendo acomodada no volume limitado com diversas dobras que criam em seu micro citoplasma areas de concentração, areas que resistem a livre difusão de ions, criando uma aparente concentração superior a real. Esse estado potencializa o seu funcionamento.

\end{questionBox}

\begin{questionBox}1{} % Q3
    
    Tendo em consideração os seus conhecimentos de bioenergética mitocondrial e mecanismos de morte celular, indique por que motivo o comprometimento da membrana mitocondrial externa (perda de integridade da membrana externa) conduz a uma diminuição drástica dos níveis de ATP na célula e indução da cascata apoptótica.

    
    
\end{questionBox}

\begin{questionBox}1{} % Q4
    
    Todos os processos seguintes ocorrem na mitocôndria de células de mamíferos, excepto:

    \begin{enumerate}[label=\alph{enumi})]
        \item glicólise
        \item síntese proteica 
        \item síntese de DNA 
        \item mutações no DNA
    \end{enumerate}

    \paragraph*{RS} a)
    
\end{questionBox}

\begin{questionBox}1{} % Q5
    
    b)
    
\end{questionBox}

\setcounter{question}{6}

\begin{questionBox}1{} % Q7
    
    a)
    
\end{questionBox}

\begin{questionBox}1{} % Q8
    
    Uma molécula capaz de funcionar como inibidor farmacológico de apoptose idealmente deveria actuar: (justifique a sua resposta)

    \paragraph*{RS} a)

    \paragraph*{Inibindo a atividade da apopitose-3}
    Não seria uma ma opção pois seria efetiva, mas como sua influencia está no final da via, a celula ja estaria muito comprometida para contiunar funcionando.
    
\end{questionBox}

\begin{questionBox}1m{} % Q9
    
    O gradiente de protões pode ser analisado com sondas fluorescentes cujos perfis de intensidade de emissão dependem do pH. Uma das sondas mais utilizadas para medir os gradientes de pH através das membranas mitocôndriais é o fluoróforo BCECF. O efeito do pH na intensidade de emissão do BCECF, excitado a 505 nm, está demonstrado na figura abaixo.

    \begin{center}
        \includegraphics[width=.6\textwidth]{Screenshot 2022-10-26 at 14.37.13}
    \end{center}

    Num determinado ensaio, vesículas seladas contendo este composto no interior sofreram um processo em que foram fundidas com vesículas formadas por membranas internas isoladas de mitocôndrias; desta forma, formaram-se vesículas com membrana dupla, com a sonda encapsulada no espaço intermembranar. Estas novas vesículas de dupla membrana foram incubadas num meio não fluorescente.\\

    Quando as vesículas foram incubadas num tampão fisiológico contendo NADH, ADP, Pi e O2, a fluorescência do BCECF no interior das vesículas diminuiu gradualmente de intensidade. O que é que esta diminuição da intensidade da fluorescência sugere acerca desta preparação de vesículas? Justifique.

    \paragraph*{RS}
    Produtos:
    \begin{itemize}
        \begin{multicols}{2}
            \item ATP
            \item NADH
        \end{multicols}
    \end{itemize}
    
\end{questionBox}

\end{document}
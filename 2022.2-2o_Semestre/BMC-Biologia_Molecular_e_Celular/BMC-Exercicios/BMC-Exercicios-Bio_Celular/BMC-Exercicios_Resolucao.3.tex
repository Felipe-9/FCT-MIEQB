% !TEX root = ./BMC-Exercicios_Resolução.3.tex
\providecommand\mainfilename{"./BMC-Exercicios_Resolução.tex"}
\providecommand \subfilename{}
\renewcommand   \subfilename{"./BMC-Exercicios_Resolução.3.tex"}
\documentclass[\mainfilename]{subfiles}

% \graphicspath{{\subfix{../images/}}}
% \tikzset{external/force remake=true} % - remake all

\begin{document}

\mymakesubfile{3}
[BMC]
{Exercicios}
{Exercicios}

\begin{questionBox}1{}
    
    Os lisossomas contém cerca de 40 tipos de enzimas hidrolíticas diferentes. Quais os mecanismos de protecção celular que evitam que estas enzimas degradem proteínas citoplasmáticas, no caso e haver uma ruptura da membrana lisossomal?

    \paragraph*{RS} As hodroláses só atuam em pH ácido, no evento da ruptura se deparam com o pH neutro do citozol se desativam
    
\end{questionBox}

\begin{questionBox}1{}
    
    Quais os mecanismos de protecção celular que evitam que proteases e fosfolipases degradem proteínas e fosfolípidos da membrana lisossomal?

    \paragraph*{RS}
    As proteínas da parede celular são muito glicosiladas com o intuito de proteger da ação das proteínas
    
\end{questionBox}

\begin{questionBox}1{}
    
    Os peroxissomas contêm enzimas oxidativas em grandes concentrações, como a catalase. Uma vez que, em termos evolutivos, as mitocôndrias passaram a desempenhar a maioria das reacções oxidativas numa célula, qual o papel dos peroxissomas nos eucariotas?

    \paragraph*{RS}
    Em celulas eucarióticas os peroxissomas se especializaram em realizar oxidações que não ocorrem nas mitocondrias.
    De forma geral as oxidações que ocorrem nos peroxissomos são a quebra de moléculas de ácido gordo principalmente as muito longas para serem ultilizadas pelas mitocôndrias.
    
\end{questionBox}

% Q4
\begin{questionBox}1{}
    
    Se tivesse à disposição todas os reagentes do laboratório, e pudesse fazer marcações específicas para componentes celulares para posterior observação ao microscópio, como distinguiria lisossomas de peroxissomas?

    \paragraph*{RS}
    Para determinação de um lisossoma marcava a membrada com proteínas pouco glicosilados, o desaparecimento do sinal confirmava a hipótese.
    
\end{questionBox}

% Q5
\begin{questionBox}1{}
    
    Alguns protocolos para detectar lisossomas e processos lisossomais in vitro pressupõem um período de jejum (“starvation”) das células. Explique o fundamento deste procedimento.

    \paragraph*{RS}
    Em periodo de jejum as celulas respondem acumulando aminoácidos essenciais dentro de lisossomas.
    
\end{questionBox}

% Q6
\begin{questionBox}1{}
    
    As proteínas desempenham as mais variadas funções biológicas se estiverem na sua conformação tridimensional adequada, isto é no seu estado nativo. A ocorrência de danos nas proteínas com consequente alteração conformacional, nomeadamente devido ao stress oxidativo, conduz à activação de mecanismos de controlo de qualidade proteico na célula. Descreva brevemente os principais mecanismos de que uma célula animal dispõe de forma a prevenir a acumulação de proteínas ``misfolded'', e indique a ordem cronológica pela qual estes são normalmente activados na resposta celular à acumulação de proteínas ``misfolded'' devido ao stress oxidativo.

    \paragraph*{RS}
    O stress de proteínas misfolded acumuladas desencadeam uma serie de reações chamada: "Unfolded Protein Response" (UPR) que tem três passos
    \begin{enumerate}
        \item Interromper a tradução de proteínas
        \item Degradação de proteínas desdobradas incorretamente
        \item Ativa as vias de sinalização que levam ao almoento da produç±ao de chaperões moleculares envolvidos no dobramento de proteínas
    \end{enumerate}

    
\end{questionBox}

\begin{questionBox}1{}
    
    Suponha que precisa de determinar se um determinado fármaco/molécula altera o fluxo autofágico numa célula eucariota, de modo a peceber se pode modular esta via num contexto patológico.
    Descubra e indique uma molécula que module positiva ou negativamente o fluxo autofágico/autofagia numa célula eucariota. Explique o seu mecanismo de acção, tendo em conta os seus alvos biológicos (proteínas/enzimas). Pode ilustrar a sua resposta com imagens/esquemas e respectivas legendas, tendo em conta o que é pedido nesta questão.

    \paragraph*{RS}
    Hidroxicloroquina que altera a altofagia dos lissosomas alcoolizandoas com uma aleração o pH.
    
\end{questionBox}

\end{document}
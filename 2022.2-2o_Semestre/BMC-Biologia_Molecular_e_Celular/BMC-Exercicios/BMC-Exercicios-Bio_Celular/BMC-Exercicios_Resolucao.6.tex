% !TEX root = ./BMC-Exercicios_Resolução.6.tex
\providecommand\mainfilename{"./BMC-Exercicios_Resolução.tex"}
\providecommand \subfilename{}
\renewcommand   \subfilename{"./BMC-Exercicios_Resolução.6.tex"}
\documentclass[\mainfilename]{subfiles}

\graphicspath{{\subfix{./.build/figures/BMC-Exercicios_Resolução.6/}}}
% \tikzset{external/force remake=true} % - remake all

\begin{document}

\mymakesubfile{6}
[BMC]
{Exercicios: Citoesqueleto} % Subfile Title
{Citoesqueleto} % Part Title

\begin{questionBox}1{} % Q1

    Imagine que é um farmacologista que está a testar uma droga que bloqueia o transporte celular, mas ainda não se conhece o seu mecanismo de acção. Ao microscópio, pode observar nitidamente vesículas intracelulares fluorescentes a deslocarem-se. Após a adição da droga, verifica que deixa de observar transporte no sentido do centríolo (MTOC), mas não no sentido oposto. Qual pensa que seja o alvo desta droga? Justifique.

    \begin{answerBox}{} % RS 
        % Afeta as Dineinas
        As dineinas são proteínas que seguem no sentido positívo ao negativo dos microtubulos em direção ao centriolo, elas que devem ter sido inativadas.
    \end{answerBox}

\end{questionBox}

\begin{questionBox}1{} % Q2
    
    Na doença de Alzheimer a proteína estabilizadora de microtúbulos \chemtau{} é hiper- fosforilada e perde afinidade para os microtúbulos, agregando. Preveja o que acontece à célula nestas circunstâncias

    \begin{answerBox}{} % RS 
        
        Perdendo a afinidade pelos microtubulos as porteínas \chemtau{} ligam-se entre si desestabilisando os microtubulos interrompendo o principal transporte celular, perdendo a estrutuda celular e desacoplando os axonios e interrompendo a sinápse.

    \end{answerBox}

\end{questionBox}

\begin{questionBox}1{} % Q3

    Qual das seguintes afirmações é falsa relativamente a proteínas motoras do citoesqueleto?

    \begin{enumerate}[label=\alph{enumi})]
        % \begin{multicols}{2}
            \item São todas ATPases
            \item Cada uma tem um sentido próprio de movimento ao longo dos microtúbulos
            \item Cada uma tem pelo menos um domínio globular que contém um local de ligação ao citoesqueleto
            \item Todas as proteínas motoras associadas aos microtúblos se movimentam em direção ao polo positivo dos microtúbulos        
        % \end{multicols}
    \end{enumerate}

    \begin{answerBox}{} % RS
        d)
    \end{answerBox}

\end{questionBox}

\begin{questionBox}1m{} % Q4
    
    Em relação aos constituintes do citosqueleto, indique a afirmação incorrecta:

    \begin{enumerate}[label=\alph{enumi})]
        \item Os filamentos de actina são formados por monómeros de actina ligados a ATP (ou ADP).
        \item Os microtúbulos são formados por heterodímeros de \chemalpha- e \chembeta-tubulina ligados a GTP (ou GDP).
        \item Os filamentos intermédios são formados nos pontos de nucleação chamados MTOC.
        \item Os filamentos de actina são importantes na manutenção da forma de uma célula e na sua alteração para adaptação a novas condições.
    \end{enumerate}

    \begin{questionBox}3{Polimerização invitro dos filamentos de actina} % Q
        começa com a autoassociação de três monômeros de G-actina para formar um trimer. A actina ligada ao ATP, em seguida, se liga à extremidade farpado, e o ATP é posteriormente hidrolisado.
    \end{questionBox}

    \begin{questionBox}3{Microtubulos} % Q
        Formados por heterodímeros de \chemalpha- e \chembeta-tubulina intimamente associados por ligação não covalentes.

        Cada monómero \chemalpha e \chembeta possui um local de ligação para uma molécula de GTP.
    \end{questionBox}

    \begin{questionBox}3{Polimerização dos Microtubulos} % 
        A Nucleação é o evento que inicia a formação de microtúbulos a partir do dímero da tubulina. Os microtúbulos são tipicamente nucleados e organizados por organelas chamados centros de organização de microtúbulos (MTOCs).
    \end{questionBox}

    \begin{answerBox}{} % RS 
        c)
    \end{answerBox}

\end{questionBox}

\begin{questionBox}1{} % Q5
    A despolimerização dos microtúbulos é inibida por:
    \begin{enumerate}[label=\alph{enumi})]
        \begin{multicols}{2}
            \item GTP
            \item GDP
            \item ADP
            \item proteína G
        \end{multicols}
    \end{enumerate}

    \begin{questionBox}3{Despolimerização} % Q
        Como a tubulina se adiciona à extremidade do microtúbulo no estado ligado ao GTP, propõe-se que exista uma tampa de tubulina ligada ao GTP na ponta do microtúbulo, protegendo-o da despolimerização.
    \end{questionBox}

    \begin{answerBox}{} % RS 
        a)
    \end{answerBox}
\end{questionBox}

\begin{questionBox}1{} % Q6

    Quando os microtúbulos crescem in vitro a partir de proteína tubulina purificada, e são colocados numa placa de vidro que foi previamente revestida com proteínas cinesinas ou dineínas, os microtúbulos parecer “surfar” em cima da placa, se estiverem presentes ATP e GTP. Como explica esta observação? O que aconteceria se adicionasse uma grande quantidade de ADP ou GDP em vez de ATP ou GDP?

    \begin{answerBox}{} % RS 
        As proteínas de transporte celular são ATPases, sem a presença de ATP elas param de funcionar, e sem o GTP os microtubulos se tornam instáveis.

        A principal diferença entre a experiencia e o interior de uma célula é que aqui as proteínas estão presas movendo os microtubulos ao invés de si
    \end{answerBox}

\end{questionBox}

\begin{questionBox}1{} % Q7

    Por que razão os filamentos intermediários não exibem instabilidade dinâmica?

    \begin{answerBox}{} % RS 
        Os filamentos intermediarios são formados por um aglomerado de dimeros que são duas moléculas ou monomeros alognados, por serem formados por moleculas longas não sofrem a instabilidade dinâmica
    \end{answerBox}

\end{questionBox}

\begin{questionBox}1{} % Q8

    Com base na imagem de fagocitose abaixo, indique um processo que:

    \begin{center}
        \includegraphics[width=.7\textwidth]{Screenshot 2022-11-02 at 14.24.30}
    \end{center}

\end{questionBox}

\begin{questionBox}2{} % Q8.1

    Envolva sinalização celular. De que tipo?

    \begin{answerBox}{} % RS 
        Sinalização de Contato entre bactéria e anticorpo/macrofago

        O contato do target as proteínas na membrana plasmática iniciando o processo da fagocitose
    \end{answerBox}

\end{questionBox}

\begin{questionBox}2{} % Q8.2

    Rearranjo do citoesqueleto. De que componente?
    

\end{questionBox}

\begin{questionBox}2{} % Q8.3
    
    Envolva activação lisossomal. Que processo?

\end{questionBox}

\end{document}
% !TEX root = ./BMC-Exercicios_Resolução.5.tex
\providecommand\mainfilename{"./BMC-Exercicios_Resolução.tex"}
\providecommand \subfilename{}
\renewcommand   \subfilename{"./BMC-Exercicios_Resolução.5.tex"}
\documentclass[\mainfilename]{subfiles}

% \graphicspath{{\subfix{../images/}}}
% \tikzset{external/force remake=true} % - remake all

\begin{document}

\mymakesubfile{5}
[BMC]
{Exercícios 5: Sinalização Celular} % Subfile Title
{Sinalização Celular} % Part Title


\begin{questionBox}1{} % Q1
    
    A ligação da acetilcolina a receptores do tipo M1 tem como consequência a activação da fosfolipase C. No entanto, a ligação da acetilcolina a receptores do tipo M2 provoca a inibição da adenil ciclase. Indique três motivos pelos quais a mesma molécula (acetilcolina) pode ter efeitos celulares tão diferentes.

    \paragraph*{RS}
    \begin{enumerate}
        \item O principal fator para a reação celular está nos tipos de receptores que interagem com o transmissor serem diferentes
        \item Tipos de vias metabólicas ativadas na celula serem diferentes
        \item Proteínas G-acopladas serem diferentes
        \item Proteína eletora ser diferente gerando mensageiro secundario diferente
    \end{enumerate}
    
\end{questionBox}

\begin{sectionBox}1m{Acetylcholine (ACh)} % S1
    
    Neutrotransmissor

    \paragraph*{Usado em}
    \begin{itemize}
        \item Junção neuromuscular (usado pelos neuronios motores para ativar musculos)
        \item Sistema nervoso autonomo, tanto como transmissor interno para o sistema nervoso simpático como produto final liberado pelo sistema nervoso para-simpático (principal neuro transmissor do sistema para-simpático).
    \end{itemize}

    \begin{sectionBox}2{Receptores} % S1.1
        
        Existem duas classes de receptores nomeados por químicos que podem seletivamente ativar cada tipo de receptor:

        \begin{sectionBox}3{Nicotinic} % S1.1 (i)
            
            São canais iônicos com ligandos que permitem a passagem de Ions de Sódio, Calcio e Potassio
            
        \end{sectionBox}
    
        \begin{sectionBox}3{Muscarinic} % S1.1 (ii)
            
            Possuem um mecanismo mais complexo com 5 subtipos nomeados de M1 a M5, todos funcionam como Receptores ligados a G-Proteínas.\\
    
            \paragraph*{M1, M3 e M5} 
            São G\(_q\)-Acopladas e alm
            \paragraph*{M2 e M4} São G\(_q\)-Acopladas
            
        \end{sectionBox}

    \end{sectionBox}

    
\end{sectionBox}

\begin{questionBox}1{} % Q2
    
    Preveja qual o efeito da adição de um antagonista dos receptores do tipo M2 na actividade da adenilciclase.
    
\end{questionBox}

\begin{questionBox}1{} % Q3
    
    Todas as frases seguintes, relativas a proteína G, são verdade, excepto:

    \begin{enumerate}[label=\alph{enumi})]
        \item as proteínas G transmitem o sinal da superfície da célula para o interior da célula 
        \item as proteínas G podem transmitir o sinal directamente ao núcleo
        \item as proteínas G actuam para amplificar o sinal e criar uma cascata de resposta na célula
        \item as proteínas G são intracelulares e catalizam a conversão de GTP em GDP
    \end{enumerate}

    \paragraph*{RS} b)
    
\end{questionBox}

\begin{questionBox}1{} % Q4
    
    As características dos receptores intracelulares que regulam a expressão génica incluem todas as seguintes, excepto:
    
    \begin{enumerate}[label=\alph{enumi})]
        \item domínio de ligação ao DNA
        \item domínio de ligação extracelular
        \item domínio de activação da transcrição
        \item são geralmente activados por moléculas lipofílicas
    \end{enumerate}

    \paragraph*{RS} b)
    
\end{questionBox}

\begin{questionBox}1{} % Q5
    
    Relativamente à sinalização parácrina:
    \begin{enumerate}[label=\alph{enumi})]
        \item envolve contacto directo entre 2 células
        \item é realizada a longas distâncias
        \item é realizada entre células adjacentes
        \item provoca despolarização na célula receptora
    \end{enumerate}

    \paragraph*{RS} c)
    
\end{questionBox}

\begin{questionBox}1{} % Q6
    
    No âmbito da comunicação entre células, discuta comparativamente a sinalização de contacto e a sinalização parácrina. Dê um exemplo representativo de cada um destes dois tipos de sinalização.
    
\end{questionBox}

\begin{questionBox}1{} % Q7
    
    Uma cinase de proteínas tipicamente:
    \begin{enumerate}[label=\alph{enumi})]
        \item hidrolisa proteínas
        \item adiciona grupos fosfato a proteínas c. estimula a adenilciclase
        \item polimerisa aminoácidos
        \item remove grupos fosfato de proteínas
    \end{enumerate}

    \paragraph*{RS} b)
    
\end{questionBox}

\begin{questionBox}1{} % Q8
    
    Os segundos mensageiros:
    \begin{enumerate}[label=\alph{enumi})]
        \item São neurotransmissores e/ou hormonas
        \item Podem regular a expressão génica, a actividade enzimática e alguns canais iónicos 
        \item activam as proteínas G
        \item fosforilam proteínas
        \item têm uma expressão constitutiva elevada
    \end{enumerate}

    \paragraph*{RS} b)

    
\end{questionBox}

\begin{questionBox}1{} % Q9
    
    Indique se as seguintes afirmações são verdadeiras (V) ou falsas (F):

    \begin{enumerate}[label=(\roman{enumi})]
        \item \line(1,0){2em}{} As hormonas esteróides ligam-se a receptores intracelulares e o complexo hormona-receptor controla directamente a síntese proteíca.
        \item \line(1,0){2em}{} Um bom 2o mensageiro está em elevadas concentrações intracelulares. 
        \item \line(1,0){2em}{} A PKA é tipicamente activada por hormonas esteróides.
        \item \line(1,0){2em}{} O cálcio não é um 2o mensageiro, porque é um ião positivo.
        \item \line(1,0){2em}{} Apesar de as respostas celulares poderem ser muito diversas, face a inúmeros estímulos, os mensageiros secundários são relativamente poucos.
    \end{enumerate}

    \paragraph*{RS}
    \begin{minipage}{\textwidth}
        \begin{enumerate}[label=(\roman{enumi})]
            \begin{multicols}{5}
                \item V
                \item F
                \item F
                \item F
                \item V
            \end{multicols}
        \end{enumerate}
    \end{minipage}
    
\end{questionBox}

\begin{questionBox}1{} % Q10
    
    Faça corresponder cada elemento da coluna da direita (A-D) a um só elemento da coluna da esquerda.

    \begin{multicols}{2}
       \begin{minipage}{.4\textwidth}
            \begin{enumerate}[label=(\roman{enumi})]
                \item \line(1,0){2em}{} AMPc
                \item \line(1,0){2em}{} cinase
                \item \line(1,0){2em}{} IP3
                \item \line(1,0){2em}{} Proteína G
            \end{enumerate}
       \end{minipage}
       \begin{minipage}{.5\textwidth}
            \begin{enumerate}[label=\Alph{enumi}:]
                \item Induz a liberação de \ch{Ca^{2+}} do RE
                \item Mensageiro intracelular
                \item Enzima que fosforila proteínas
                \item Proteína acoplada a receptores membranares
            \end{enumerate}
       \end{minipage}
    \end{multicols}

    \paragraph*{RS}
    \begin{minipage}{\textwidth}
        \begin{enumerate}[label=(\roman{enumi})]
            \begin{multicols}{4}
                \item B
                \item C
                \item A
                \item D
            \end{multicols}
        \end{enumerate}
    \end{minipage}
    
\end{questionBox}

\begin{questionBox}1{} % Q11
    
    A transcrição do gene X é controlada pelo factor de transcrição A. O gene X só é transcrito quando o factor A está fosforilado. Os dados relativos à distribuição do factor A e as actividades das proteínas cinase e fosfatase específicas do factor A estão identificadas na tabela abaixo.
    \begin{table}[H]\centering
        \setlength\tabcolsep{3mm}        % width
        \renewcommand\arraystretch{1.25} % height
        \begin{tabular}{c c c c}
            
            \\\toprule
            
                \multicolumn{1}{c}{Tecido}
            &   \multicolumn{1}{c}{Fator A}
            &   \multicolumn{1}{c}{
                    \begin{tabular}{c}
                        Atividade de\\
                        Proteína Cinase
                    \end{tabular}
                }
            &   \multicolumn{1}{c}{
                    \begin{tabular}{c}
                        Atividade de\\
                        Proteína Fosfatase
                    \end{tabular}
                }
            
            \\\midrule
            
                Musculo & + & - & -
            \\  Coração & + & + & -
            \\  Cérebro & + & - & +
            
            \\\bottomrule
            
        \end{tabular}
    \end{table}

    Destes 3 tecidos, o gene X será transcrito: 
    \begin{enumerate}[label=\alph{enumi})]
        \begin{multicols}{2}
            \item apenas no músculo
            \item apenas no coração
            \item apenas no cérebro
            \item apenas no cérebro e no coração
            \item no músculo, no coração e no cérebro
        \end{multicols}
    \end{enumerate}

    \paragraph*{RS} b)
    
\end{questionBox}

\begin{questionBox}1{} % Q12
    
    Se todos os sinais/moléculas para os quais a membrana é permeável, como a testosterona, entram em todas as células com as quais entram em contato, por que não afetam todos os tipos de células? Por exemplo, as mulheres têm baixos níveis de testosterona circulante, mas não crescem pelos faciais como os homens. Por que não?

    \paragraph*{RS} A existe uma falta da trascrição da expressão genética que gera receptores próprias para receber o transmissor relacionado.
    
\end{questionBox}

\begin{questionBox}1{} % Q13
    
    Resultados experimentais mostram que na presença do Composto X há um aumento da expressão de uma proteína Y em células em cultura. No entanto, o modo de ação do Composto X é ainda pouco conhecido e não se sabe ainda a via pela qual ele leva ao aumento de produção de proteína Y nas células.\\

    O que sabemos é que o tratamento com forskolin ou dibutyryl cAMP teve um efeito semelhante ao do composto X.\\

    Por outro lado, o tratamento com o composto X e o inhibitor 666-15 não mostrou aumento da expressão da proteína Y.\\

    Com base nestes resultados, proponha uma via canónica clássica de sinalização que explique os mecanismos de sinalização que medeiam o efeito do composto X na expressão da proteína Y.
    
\end{questionBox}

\end{document}
% !TEX root = ./BMC-Exercicios_Resolução.1.tex
\providecommand\mainfilename{"./BMC-Exercicios_Resolução.tex"}
\providecommand \subfilename{}
\renewcommand   \subfilename{"./BMC-Exercicios_Resolução.1.tex"}
\documentclass[\mainfilename]{subfiles}

\graphicspath{{\subfix{./.build/figures/BMC-Exercicios_Resolução.1/}}}
% \tikzset{external/force remake=true} % - remake all

\date{\Large 16 de Setembro de 2022}

\begin{document}

\mymakesubfile{1}
[BMC]
{Exercícios: Célula Eucariótica e Membrana Celular}
{Célula Eucariótica e Membrana Celular}

\begin{questionBox}1{} % Q1
    
    \subsection*{Qual das seguintes frases não é verdadeira?}

    \begin{enumerate}[label=\alph{enumi})]
        \item Todas as células replicam o seu DNA.
        \item As células são originadas a partir de uma célula pré-existente.
        \item Todas as células contém uma membrana plasmática, um núcleo delimitado por membrana nuclear, mitocôndrias, lisossomas, Golgi e outros organelos.
        \item A célula é a unidade estrutural da vida. 
        \item Todas as afirmações estão correctas.
    \end{enumerate}

    \begin{answerBox}{} % RS 
        c) e e)

        \paragraph*{c)} Células procarióticas não possuem núcleo, apenas uma região onde pode ser encontrada o DNA
    \end{answerBox}

    
\end{questionBox}

\begin{questionBox}1m{} % Q2
    
    \subsection*{No modelo de mosaico fluído da membrana celular:}

    \begin{enumerate}[label=\alph{enumi})]
        \item Os fosfolípidos são apenas de um tipo e apenas há heterogeneidade na composição proteica.
        \item A proteínas de membrana são sempre transmembranares.
        \item A fluidez é dada pela possibilidade de deslocamento das proteínas membranares na malha fixa constituída pela dupla camada fosfolipídica.
        \item A permeabilidade selectiva da membrana a macromoléculas e iões é maioritariamente definida pelas proteínas membranares.
        \item As caudas hidrofóbicas dos fosfolípidos estão direccionadas para o exterior.
    \end{enumerate}

    \begin{answerBox}{c) e d)} % RS 
        \paragraph*{a)} Há fosfolipídeos com 
    \end{answerBox}
    
\end{questionBox}

\begin{questionBox}1{} % Q3
    
    \subsection*{Um canal iónico:}
    \begin{enumerate}[label=\alph{enumi})]
        \item nunca altera a sua conformação tri-dimensional para a passagem dos iões 
        \item está localizado no lado extra-celular da membrana plasmática
        \item é responsável por formação de gradiente electro-químico
        \item normalmente não é selectivo relativamente ao ião que o atravessa
        \item Em situação de repouso da célula encontra-se fechado
    \end{enumerate}

    \paragraph*{RS:} c)
    
\end{questionBox}

\begin{questionBox}1{} % Q4

    \subsection*{Em que circunstância podemos suspeitar que um determinado transporte se realiza por transporte passivo?}

    \begin{enumerate}
        \item há uma proteína membranar com domínio ATPásico associada a esse transporte
        \item a taxa de transporte não satura
        \item diminuição da energia livre da célula não inibe o transporte 
        \item é efectuado a favor do gradiente
        \item hipóteses b) e c)
    \end{enumerate}
    
    \paragraph*{RS:} d)
    
\end{questionBox}

\stepcounter{question}
\begin{questionBox}1{} % Q6
    
    \subsection*{O processo de antiporte difere do de simporte porque:}
    \begin{enumerate}[label=\alph{enumi})]
        \item O antiporte envolve o transporte simultâneo de duas moléculas diferentes na mesma direcção e o simporte o transporte das duas moléculas é feito em direcções opostas.
        \item No simporte o transporte é feito para dentro da célula e o antiporte para fora.
        \item O simporte corresponde ao transporte simples de apenas um tipo de molécula e o antiporte corresponde ao transporte de diferentes moléculas em simultâneo.
        \item O simporte envolve o transporte simultâneo de duas moléculas diferentes na mesma direcção e o antiporte o transporte das duas moléculas é feito em direcções opostas.
        \item O antiporte é feito contra o gradiente e o simporte a favor do gradiente de concentração.
    \end{enumerate}

    \paragraph*{RS:} d)
    
\end{questionBox}

\begin{questionBox}1{} % Q7

    \subsection*{
        Os gráficos X e Y da figura representam o modo como varia a concentração de duas substâncias A (gráfico X) e B (gráfico Y) no interior e no exterior da célula.
    }

    \begin{center}
        \includegraphics[width=.8\textwidth]{Screenshot 2022-11-05 at 14.59.38}
    \end{center}

    \subsection*{Estas experiências permitem concluir que:}
    \begin{enumerate}[label=\alph{enumi})]
        \item À medida que o tempo passa, os meios intra e extracelular vão ficando isotónicos em relação à concentração da substância B
        \item A substância B passa do meio intracelular para o meio extracelular 
        \item O transporte referente aos gráfico Y é mediado por proteínas
        \item As proteínas responsáveis pelo transporte da substância B são consumidas no processo de transporte
        \item Hipóteses a) e c)
    \end{enumerate}
    
    \paragraph*{RS:} b) e c)
    
\end{questionBox}

\begin{questionBox}1{} % Q8
    
    \subsection*{Os gráficos X e Y da figura representam o modo como varia a concentração de duas substâncias A (gráfico X) e B (gráfico Y) no interior e no exterior da célula.}

    \begin{center}
        \includegraphics[width=.8\textwidth]{Screenshot 2022-11-05 at 14.59.38}
    \end{center}

    \subsection*{Qual destes transportes é passivo? E activo? Justifique.}
    
    \begin{answerBox}{} % RS 
        Primeiro transporte é passivo pois ao decorrer equivale as concentrações dentro e fora da celula, comportamento de difusão simples
    
        Segundo transporte é ativo pois age contra gradiente gerando uma desigualdade de concentrações.
    \end{answerBox}
    
\end{questionBox}


\begin{questionBox}1m{} % Q9
    
    \subsection*{Suponha que estão a estudar a composição de bicamadas lípidicas e de como são mantidas. Descobrem dois novos fosfolípidos, que não tinham sido caracterizados antes, e dão-lhes o nome de PLX e PLZ. Para caracterizar o comportamento de PLX e PLZ marcam a cabeça hidrofílica de cada fosfolípido com um composto químico. Este composto é estável quando o lípido se localiza na camada lipídica externa, mas instável quando se localiza na camada lipídica interna. Incorporaram as versões marcadas de PLX e PLZ no interior (na camada lipídica interna) ou no exterior da célula (na camada lipídica externa) e analisaram a intensidade do sinal emitido pelo composto químico ligado a cada um dos lípidos na membrana plasmática. Os resultados obtidos estão indicados nos gráficos abaixo.}

    \begin{center}
        \includegraphics[width=.8\textwidth]{Screenshot 2022-11-05 at 15.13.18}
    \end{center}

    
\end{questionBox}


\begin{questionBox}2{} % Q9.1
    
    \subsection*{Em que camada da membrana plasmática é que PLX e PLZ se localizam normalmente?}

    \begin{answerBox}{} % RS 
        \begin{description}
            \item[PLX] se mantem estável fora da celula enquanto instavel dentro por ter seu sinal decrescendo com o tempo, pode-se dizer que não se encontra na camada interna
            \item[PLY] se mantem estável tanto dentro quanto fora da celula por manterem o sinal inicial podendo ocupar ambos meios
        \end{description}
    \end{answerBox}
    
\end{questionBox}

\begin{questionBox}2{} % Q9.2
    
    \subsection*{Terá a célula flipases (enzimas que promovem o flip-flop dos fosfolipídos) que reconheçam algum destes fosfolípidos?}

    \begin{answerBox}{} % RS 
        \begin{description}
            \item[PLX] por ser instável dentro da celula não deve haver o flipflop
            \item[PLY] Por ser estável dentro e fora da celula é possivel haver o flipflop
        \end{description}
    \end{answerBox}
    
\end{questionBox}


\begin{questionBox}1m{} % Q10
    
    \subsection*{Para estudar a forma como as células acumulam stocks de energia e os usam para realizar as funções necessárias para permanecer vivas, tente desenhar um modelo de célula com as seguintes proteínas na sua superfície:}

    \begin{itemize}
        \item Um sistema de transporte activo de antiporte que bombeia a molécula A para o citosol e a molécula B para o espaço extracelular.
        \item Um sistema de simporte que transporta a molécula B e a molécula C para o citosol
        \item Um canal permeável à molécula A
    \end{itemize}

    \includegraphics[width=0.8\textwidth]{BMC-Exericicios.1-Q.10.png}
        
    \subsection*{Trace os caminhos que cada molécula segue para responder a esta pergunta:
    qual é o impacto líquido de todas as três proteínas membranares a trabalharem simultaneamente?}

    \begin{answerBox}{} % RS 
        \begin{enumerate}[label=\Alph{enumi}:]
            \item Está sendo bombeado para dentro da celula pela proteina antiporte porem se difusa pelo canal permeável para fora mantendo uma concentração levemente elevado dentro da celula
            \item Está sendo bombeado pela proteína antiporte apenas para fora da celula gerando um gradiente de maior concentração para fora
            \item Está sendo bombeada pela proteína simporte para dentro da celula gerando um gradiente de maior concentração no citoplasma
        \end{enumerate}
    \end{answerBox}
        
\end{questionBox}

\end{document}
% !TEX root = ./BMC-Exercícios_Resolução_Teste.1.tex
\providecommand\mainfilename{"./BMC-Exercícios_Resolução_Teste.tex"}
\providecommand \subfilename{}
\renewcommand   \subfilename{"./BMC-Exercícios_Resolução_Teste.1.tex"}
\documentclass[\mainfilename]{subfiles}

\graphicspath{{\subfix{./.build/figures/BMC-Exercícios_Resolução_Teste.1/}}}
% \tikzset{external/force remake=true} % - remake all

\begin{document}

\mymakesubfile{1}
[BMC]
{Exercicios para Testes 1} % Subfile Title
{Resolução} % Part Title

\begin{questionBox}1{ % Q1
    No modelo de mosaico fluído da membrana celular:
} % Q1

    \begin{enumerate}[label=\alph{enumi})]
        \item Os fosfolípidos são apenas de um tipo e apenas há heterogeneidade na composição proteica.
        \item A proteínas de membrana são sempre transmembranares.
        \item A fluidez é dada pela possibilidade de deslocamento das proteínas membranares na malha fixa constituída pela dupla camada fosofolipídica.
        \item A permeabilidade selectiva da membrana a macromoléculas e ioões é maioritariamente definida pelas proteínas membranares.
        \item As caudas hidrofóbicas dos fosfolípidos estão direccionadas para o exterior.
    \end{enumerate}
    
    \paragraph*{RS:} d)

\end{questionBox}

\begin{questionBox}1{ % Q1
    Um canal iónico:
} % Q2
    
    \begin{enumerate}[label=\alph{enumi})]
        \item nunca altera a sua conformação tri-dimensional para a passagem dos iões 
        \item está localizado no lado extra-celular da membrana plasmática
        \item é responsável por formação de gradiente electro-químico
        \item normalmente não é selectivo relativamente ao ião que o atravessa
        \item Em situação de repouso da célula encontra-se fechado
    \end{enumerate}

    \paragraph*{RS:} e)
    
\end{questionBox}

\begin{questionBox}1{ % Q3
    Qual das seguintes frases não é verdadeira?
} % Q3

    \begin{enumerate}[label=\alph{enumi})]
        \item Todas as células transcrevem o seu DNA.
        \item As células são originadas a partir de uma célula pré-existente.
        \item Todas as células contém uma membrana plasmática que delimita a fronteira entre o espaço intra e extracelular.
        \item A célula é a unidade funcional da vida.
        \item Todas as afirmações estão correctas.
    \end{enumerate}

    \begin{answerBox}{} % RS 
        Por eliminação, não encontro nenhuma errada, suspeito da a) por haver células que morrem antes de se dividir ou que estancam seu metabolismo como celulas neuronais.

        a) e e)
    \end{answerBox}

\end{questionBox}

\begin{questionBox}1{ % Q4
    Em que circunstância podemos suspeitar que um determinado transporte se realiza por transporte passivo?
} % Q4
    
    \begin{enumerate}[label=\alph{enumi})]
        \item há uma proteína membranar com domínio ATPásico associada a esse transporte
        \item a taxa de transporte não satura
        \item diminuição da energia livre da célula não inibe o transporte 
        \item é efectuado a favor do gradiente
        \item hipóteses b) e c)
    \end{enumerate}

    \paragraph*{RS:} c) e d)

\end{questionBox}

\begin{questionBox}1{} % Q5
    
    \subsection*{Qual das seguintes frases é verdadeira?}
    
    \begin{enumerate}[label=\alph{enumi})]
        \item No mesmo organismo cada grupo de células contém DNA diferente.
        \item As células são originadas a partir de uma célula pré-existente.
        \item Todas as células contém uma membrana plasmática, um núcleo delimitado por membrana nuclear, mitocôndrias, lisossomas, Golgi e outros organelos.
        \item Todas as células expressam as mesmas proteínas como agentes catalizadores.
        \item Um orgão é constituído por células todas semelhantes
    \end{enumerate}

    \begin{answerBox}{} % RS 
        Existem modificações feitas no DNA de células ao longo de sua vida, exemplo são pequenas sequencias de DNA que se inserem na sequencia dentro do núcleo de uma célula

        a)
    \end{answerBox}

\end{questionBox}


\begin{questionBox}1{ % Q6
    Qual das seguintes moléculas tem mais probabilidade de ter sido sintetizada no retículo endoplasmático liso?
} % Q6


    \begin{enumerate}[label=\alph{enumi})]
        \begin{multicols}{4}
            \item proteínas
            \item glucose
            \item fosfolípidos 
            \item aminoácidos
        \end{multicols}
    \end{enumerate}

    \begin{answerBox}{} % RS 
        RE listo tem como principais funções o metabolismo lípidico e a regulação dos níveis de calcio

        c)
    \end{answerBox}

\end{questionBox}

\begin{questionBox}1{} % Q7
    
    \subsection*{Em relação ao complexo de Golgi, indique a opção incorrecta.}

    \begin{enumerate}[label=\alph{enumi})]
        \item É composto por um conjunto de compartimentos membranares (cisternas).
        \item As proteínas que saem da face trans poderão ser transportadas para lisossomas, vesículas secretórias ou para a membrana plasmática.
        \item As proteínas que entram na face cis, vindas do retículo endoplasmático, são exclusivamente transportadas ao longo do complexo de Golgi, no sentido da face trans. 
        \item É organizado de modo a que cada cisterna tenha um conjunto de enzimas específico e diferente de outra cisterna.
    \end{enumerate}

    \paragraph*{RS:} c)

\end{questionBox}

\begin{questionBox}1{} % Q8
    
    \subsection*{Na via clássica de secreção de proteínas, a via intracelular seguida pelas proteínas é a seguinte:}
    \begin{enumerate}[label=\alph{enumi})]
        \item Retículo endoplasmático - Aparelho de Golgi - Grânulo de secreção
        \item Retículo end. rugoso - Retículo endoplasmático liso- Grânulo de secreção
        \item Aparelho de Golgi - Retículo endoplasmático rugoso - Grânulo de secreção 
        \item Núcleo - Retículo endoplasmático - Aparelho de Golgi - Grânulo de secreção
    \end{enumerate}

    \paragraph*{RS:} a)


\end{questionBox}

\begin{questionBox}1m{} % Q9
    
    \subsection*{Complete a legenda das estruturas 1-5 identificadas no esquema abaixo, que representa o processo de translocação co-traducional no retículo endoplasmático.}

    \begin{center}
        \includegraphics[width=.8\textwidth]{Screenshot 2022-11-05 at 17.33.01}
    \end{center}

    \begin{enumerate}
        \begin{multicols}{2}
            \item mRNA
            \item sequência de sinal
        \end{multicols}
        \begin{multicols}{3}
            \item \line(1,0){3em}
            \item \line(1,0){3em}
            \item \line(1,0){3em}
        \end{multicols}
    \end{enumerate}

    \begin{answerBox}{} % RS 
        \begin{enumerate}
            \begin{multicols}{3}
                \setcounter{enumi}{2}
                \item SRP
                \item SRP Receptor
                \item Translocum
            \end{multicols}
        \end{enumerate}

        \begin{center}
            \includegraphics[width=.8\textwidth]{Screenshot 2022-11-07 at 16.15.03}
        \end{center}

        \begin{description}
           \item[SRP] Signal Recognition Particle, Detecta a sequencia de sinal traduzida do mRNA e se prende ao ribossomo
           \item[SRP receptor] Se liga ao SRP espeçifico que está ligado a sequencia de cinal desejada, de uma forma a preparar para inserir a proteína produzida dentro do RE rugoso
           \item[Translocum] é a proteína na parede do RE rugoso que se abre para receber a proteína produzida em co-tradução
        \end{description}
    \end{answerBox}

\end{questionBox}

\begin{questionBox}1{} % Q10
    
    \section*{Uma proteína sintetizada por uma célula eucariota}
    \begin{enumerate}[label=\alph{enumi})]
        \item É entregue no lúmen do Golgi onde é clivado o ribossoma ao qual está ligada
        \item Pode conter uma sequência de sinalização que a encaminha para diferentes destinos sub-celulares, tais como a membrana plasmática, lisossomas, ou outra localizações celulares
        \item É traduzida no retículo endoplasmático liso se tem como destino final ser secretada 
        \item Normalmente passa pelo núcleo antes de ser exportada da célula
    \end{enumerate}

    \paragraph*{RS:} b)

\end{questionBox}

\begin{questionBox}1{} % Q11
    
    \subsection*{As vesículas revestidas de COP II}
    \begin{enumerate}[label=\alph{enumi})]
        \item Medeiam o transporte (retrógrado) do Golgi para o Retículo Endoplasmático
        \item Medeiam o transporte (retrógrado) entre o cis Golgi e o trans Golgi.
        \item Medeiam o transporte(anterógrado) entre o Retículo Endoplasmático e o Golgi. 
        \item São marcadas para serem secretadas.
    \end{enumerate}

    \paragraph*{RS:} b)
    
\end{questionBox}

\begin{questionBox}1{} % Q12
    
    \subsection*{Indique se as seguintes afirmações são verdadeiras ou falsas e corrija as falsas.}

\end{questionBox}

\begin{questionBox}2{} % Q12.1
    
    \subsection*{Os peroxissomas estão envolvidos na beta-oxidação de ácidos gordos.}

    \begin{answerBox}{} % RS 
        Os peroxissomas são vesiculas encarregadas de oxidar e quebrar acídos gordos muito grandes para serem quebrados nas mitocondrias\\
        
        Verdaderio.
    \end{answerBox}

\end{questionBox}

\begin{questionBox}2{} % Q12.2
    
    \subsection*{Os peroxissomas possuem uma bomba de protões na sua membrana, para garantir o pH ácido no lúmen do organelo.}

    \begin{answerBox}{} % RS 
        Falso, Os lisossomos possuem uma bomba de protões na sua membrana, para garantir o pH ácido no lúmen do organelo.
    \end{answerBox}

\end{questionBox}

\begin{questionBox}2{} % Q12.3
    
    \subsection*{A autofagia é sempre precedida de fagocitose.}



    \begin{answerBox}{} % RS 

        \paragraph*{Autofagia Possui 3 vias conhecidas}
        \begin{description}
           \item[Microfagia] Particulas pequenas que não precisam ser envolvidas por outra vesicula para serem transportadas
           \item[Macrofagia] Autofagia de uma vesicula envolvida por um macrofago
           \item[Autofagia mediada por chapron] Proteínas inuteis são marcadas por CMA(citosolic-chapron-complex) que são reconhecidas por proteínas da superfície dos lisossomas
        \end{description}

        Falsa
        \begin{itemize}
            \item autofagia envolvem apensas entidades previamente pertencentes a celula
            \item eterofagia é sempre precedida de fagocitose
        \end{itemize}
    \end{answerBox}

\end{questionBox}

\begin{questionBox}2{} % Q12.4
    
    \subsection*{Pensa-se que as vesículas precursoras dos lisossomas se destacam do complexo de Golgi.}

    \paragraph*{RS:} Verdadeira

\end{questionBox}

\begin{questionBox}2{} % Q12.5
    
    \subsection*{Os lisossomas permitem armazenar macromoléculas para situações de falta de nutrientes.}

    \begin{answerBox}{} % RS 
        Falsa
        \begin{itemize}
            \item Os lisossomas permitem a degradação de entidades que são inseridas em seu substrato
            \item Os vacuolos permitem armazenar macromoléculas para situações de falta de nutrientes.
        \end{itemize}
    \end{answerBox}

\end{questionBox}


\end{document}
% !TEX root = ./BMC-Exercicios_Resolução.3.tex
\providecommand\mainfilename{"./BMC-Exercicios_Resolução.tex"}
\providecommand \subfilename{}
\renewcommand   \subfilename{"./BMC-Exercicios_Resolução.3.tex"}
\documentclass[\mainfilename]{subfiles}

\graphicspath{{\subfix{./.build/figures/BMC-Exercicios_Resolução.3/}}}
% \tikzset{external/force remake=true} % - remake all

% \date{\Large 16 de Novembro de 2022}

\begin{document}

\renewcommand\theenumi{\alph{enumi}}

\mymakesubfile{3}
[BMC]
{Exercicios}
{Exercicios}

% \group{Lista}

% \begin{questionBox}1{ % Q1
%     Why do the fragments of DNA in gel electrophoresis travel away from the negative electrode?
% } % Q1
    
%     \begin{enumerate}
%         \item DNA is negatively charged so attracted to the positive end of the unit
%         \item DNA is positively charged to attracted to the negative end of the unit
%         \item the agarose gel in negatively charged 
%         \item the agarose gel is positively charged
%     \end{enumerate}

%     \paragraph*{RS:} a)

% \end{questionBox}

% \begin{questionBox}1{ % Q2
%     Which of the following features of DNA is primarily responsible for movement of DNA molecules in an electrical field?
% } % Q2
%     \begin{enumerate}
%         \item Nitrogenous base
%         \item Deoxyribose sugar
%         \item Phosphate
%         \item Complementary base pairing e. Antiparallel orientation
%     \end{enumerate}

%     \paragraph*{RS} c)
% \end{questionBox}

% \begin{questionBox}1{ % Q3
%     Gel electrophoresis enables scientists to
% } % Q3
%     \begin{enumerate}
%         \begin{multicols}{2}
%             \item separate DNA fragments. 
%             \item combine DNA fragments. 
%             \item count the genes in DNA. 
%             \item insert DNA in cells.
%         \end{multicols}
%     \end{enumerate}

%     \paragraph*{RS}
% \end{questionBox}

% \begin{questionBox}1{ % Q4
%     The rate at which DNA migrates through the gel is determined by:
% } % Q4
%     \begin{enumerate}
%         \item Molecular size of the DNA and the agarose gel concentration. 
%         \item Conformation of DNA and the applied voltage.
%         \item The length of the agarose gel and the negativity of the DNA. 
%         \item Both (a) and (b)
%         \item Both (a) and (c)
%     \end{enumerate}
% \end{questionBox}

% \begin{questionBox}1{ % Q5
%     Which is the primary purpose of using restriction enzymes in gel electrophoresis?
% } % Q5
%     \begin{enumerate}
%         \item  It allows the strands of DNA to be cut into various lengths for testing 
%         \item It restricts the number of base pairs that can be tested in a sample 
%         \item It makes the testing simpler by moving the strands into the gel faster 
%         \item It charges the DNA strands
%     \end{enumerate}
% \end{questionBox}

% \begin{questionBox}1{ % Q6
%     A student performed a gel electrophoresis experiment. The results are represented in the diagram below. Compared to the fragments at the top of the gel, the fragments at the lower end are
% } % Q6
    
%     \begin{center}
%         \includegraphics[width=.6\textwidth]{Screenshot 2022-11-23 at 15.42.24.png}
%     \end{center}

%     \begin{enumerate}
%         \begin{multicols}{2}
%             \item larger, and move slower 
%             \item larger, and move faster 
%             \item smaller, and move faster 
%             \item smaller, and move slower
%         \end{multicols}
%     \end{enumerate}

% \end{questionBox}

% \begin{questionBox}1{ % Q7
%     A scientist wishes to verify that a restriction digestion has successfully cut a linear DNA fragment. She decides to compare the cut and uncut DNA samples using agarose gel electrophoresis. Which of the following agarose gel results is a result that would indicate the linear piece of DNA was digested?
% } % Q7
%     \begin{enumerate}
%         \item One band in the uncut DNA lane and two smaller bands in the digested DNA lane
%         \item One band in the uncut DNA lane and a single smaller band in the digested DNA lane 
%         \item One band in the uncut DNA lane and three smaller bands in the digested DNA lane 
%         \item Two of the answers are correct
%         \item All of the answers are correct
%     \end{enumerate}
% \end{questionBox}

% \begin{questionBox}1{ % Q8
%     The following image represents a map of a piece of DNA, where each vertical line represents a recognition site for restriction enzyme BamHI. The numbers refer to the size of the pieces of DNA after digestion by this restriction enzyme.
% } % Q8

%     If this DNA was digested completely by BamHI, which of the following agarose gel results would you expect to see?\\

%     \begin{center}
%         \includegraphics[width=.8 \textwidth]{Screenshot 2022-11-23 at 15.44.51.png}\\
%         \includegraphics[width=.95\textwidth]{Screenshot 2022-11-23 at 15.45.41.png}
%     \end{center}

% \end{questionBox}

% \begin{questionBox}1{ % Q9
%     The following image represents the agarose gel results from the restriction digest of a 6 kb piece of DNA that possesses two restriction sites. Which of the following statements is true?
% } % Q9
    
%     \begin{center}
%         \includegraphics[width=.5\textwidth]{Screenshot 2022-11-23 at 15.46.26.png}
%     \end{center}

%     \begin{enumerate}
%         \item The restriction digest was incomplete because the smallest DNA band is missing. 
%         \item It is impossible to determine if the digest was successful.
%         \item The piece of DNA that was digested was not 6 kb long.
%         \item The 6 kb piece of DNA was completely digested by the restriction enzyme.
%         \item Two of the statements are true.
%     \end{enumerate}

% \end{questionBox}

% \begin{questionBox}1{ % Q10
% } % Q10

%     \begin{enumerate}
%         \item Plasmídeos são moléculas de DNA \line(1,0){2em}{},
%         \item unidades genéticas com \line(1,0){2em}{} independente do cromossoma. 
%         \item Nativamente, existem sobretudo em \line(1,0){2em}{} 
%         \item e podem existir em baixo ou elevado \line(1,0){2em}{} (2 palavras) dentro da célula, 
%         \item uma característica que depende da sua \line(1,0){2em}{} (2 palavras).
%         \item Os plasmídeos \line(1,0){2em}{} são derivados de plasmídeos naturais, aos quais foram acrescentadas ou removidas sequências específicas 
%         \item e podem ser agrupados em \line(1,0){2em}{} de plasmídeos. 
%         \item São ferramentas importantes para o processo de clonagem molecular, sendo dois elementos essenciais para este fim, 
%         \begin{enumerate}
%             \item a \line(1,0){2em}{} (3 palavras) 
%             \item e um gene que permita \line(1,0){2em}{} transformantes. 
%         \end{enumerate}
%         \item Podem incluir outros elementos, como \line(1,0){2em}{}.
%     \end{enumerate}

% \end{questionBox}

% \begin{questionBox}1{ % Q11
% } % Q11
    
%     \begin{enumerate}
%         \item As enzimas de restrição são \line(1,0){2em}{} do tipo II
%         \item que reconhecem sequências de estrutura \line(1,0){2em}{}, devido facto de serem homodímeros. 
%         \item As enzimas de restrição fazem parte do sistema de \line(1,0){2em}{} (2 palavras) das bactérias, 
%         \item que as protege da entrada de DNA exógeno, como no caso da infeção por \line(1,0){2em}{}.
%         \item Assim, para cada enzima de restrição existe uma enzima \line(1,0){2em}{} que reconhece a mesma sequência 
%         \item e adiciona um grupo \line(1,0){2em}{} a uma das bases.
%         \item Uma enzima de restrição pode gerar extremidades \line(1,0){2em}{} 
%         \item ou \line(1,0){2em}{} 
%         \item e a sua \line(1,0){2em}{} de corte 
%         \item \line(1,0){2em}{} com o aumento do número de nucleótidos da sequência de reconhecimento.
%     \end{enumerate}

% \end{questionBox}

\setcounter{question}{11}

\begin{questionBox}1m{ % Q12
    Establishment of phage lambda physical map:
} % Q12
    
    \begin{center}
        \includegraphics[width=1\textwidth]{Screenshot 2022-11-23 at 15.57.40.png}
    \end{center}
    
    % \begin{enumerate}
    %     \item how many restriction fragments did you theoretically expect for each enzyme (Bgll, BamHI and SalI)?
    %     \item How can you explain the observed discrepency?
    % \end{enumerate}

    \begin{questionBox}2{ % Q12.1
        how many restriction fragments did you theoretically expect for each enzyme (Bgll, BamHI and SalI)?
    } % Q12.1
        \begin{description}
           \item[BglII] 7\\
           Although for its smaller fragments dimensions in a real electrophoresis we probably wont be able to see them separated or still in the gel resulting in 5 bands.
           \item[BamHI] 6\\
           since we have only one big band with the rest close with good controll we might be able to separate those 5 similar bands whilist having the larger very close to the origin
           \item[SalI] 2 or 3\\
           Its hard to predict for this one as if we can separate the larger ones whilist having the smaller still in the gel.
        \end{description}
    \end{questionBox}

    \begin{questionBox}2{ % Q12.2
        How can you explain the observed discrepency?
    } % Q12.2
        We can see that the frequency is close between BglII and BamHI its significantly larger than SalI, the former most probably has a larger recognition sequence than the previous, whilist BglII and BamHI may either have similar but different sizes or a less frequent sequence if its the former we could guess BglII recognizes more cytosines or guanines for its higher frequency.
    \end{questionBox}

\end{questionBox}

% \begin{questionBox}1{ % Q13
%     Phage Lambda DNA was digested with 2 restriction enzymes (EcoRI and HindIII).
%     The sample of lane 2 contains Phage Lambda DNA not treated.
%     All the other samples (3 to 7) suffered digestion with 1 or 2 restriction enzymes. Two of these 5 samples (3 to 7) were previously treated with the metilase enzyme M.EcoRI (before suffering restriction).
%     Fill the table with the legend for the gel (for some lanes, more than 1 hypothesis exists). Consider that NO samples are repeated.
% } % Q13
    
%     \begin{center}
%         \includegraphics[width=.5\textwidth]{Screenshot 2022-11-23 at 15.59.18.png}
%     \end{center}

%     \vspace{-2ex}

%     \begin{table}[H]\centering
%         \begin{tabular}{c c c c}
            
%             \\\toprule
            
%                 \multicolumn{1}{c}{Lane}
%             &   \multicolumn{1}{c}{M.EcoRI}
%             &   \multicolumn{1}{c}{EcoI}
%             &   \multicolumn{1}{c}{HindII}
            
%             \\\midrule
            
%                 1
%                 & \multicolumn{3}{c}{DNA Ladder NZY III}
%                 \\ 2 & -- & -- & --
%                 \\ 3
%                 \\ 4
%                 \\ 5
%                 \\ 6
%                 \\ 7
            
%             \\\bottomrule
            
%         \end{tabular}
%     \end{table}

% \end{questionBox}

\end{document}
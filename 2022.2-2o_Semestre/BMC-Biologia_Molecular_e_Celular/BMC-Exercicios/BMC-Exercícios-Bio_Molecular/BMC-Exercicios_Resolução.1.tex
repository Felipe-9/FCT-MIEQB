% !TEX root = ./BMC-Exercicios_Resolução.1.tex
\providecommand\mainfilename{"./BMC-Exercicios_Resolução.tex"}
\providecommand \subfilename{}
\renewcommand   \subfilename{"./BMC-Exercicios_Resolução.1.tex"}
\documentclass[\mainfilename]{subfiles}

\graphicspath{{\subfix{./.build/figures/BMC-Exercicios_Resolução.1/}}}
% \tikzset{external/force remake=true} % - remake all

\date{\Large 16 de Setembro de 2022}

\begin{document}

\mymakesubfile{1}
[BMC]
{Exercicios}
{Exercicios}

\begin{questionBox}1{ % Q1
    A sequência nucleotídica de um DNA em dupla hélice é:
} % Q1

    Qual é a sequência da cadeia complementar (escrita de 5'-para-3')?

    \begin{answerBox}{} % RS 
        TGATTGTGGACAAAAATCC
    \end{answerBox}

\end{questionBox}

\begin{questionBox}1{ % Q2
    As cadeias de uma dupla hélice de DNA podem ser separadas por aquecimento. Se se aumentar a temperatura de uma solução contendo as três moléculas de DNA indicadas abaixo (apenas uma das cadeias está representada), por que ordem é que elas se separariam?
} % Q2

    \begin{enumerate}[label=\alph{enumi})]
        \item 5'-GCGGGCCAGCCCGAGTGGGTAGCCCAGG-3'
        \item 5'-ATTATAAAATATTTAGATACTATATTTACAA-3'
        \item 5'-AGAGCTAGATCGAT-3'
    \end{enumerate}

    \begin{answerBox}{} % RS
        Quanto maior o numero de guaninas e citosinas mais triplas pontes que implica no maior ponto de fusão

        b,c,a
    \end{answerBox}

\end{questionBox}

\begin{questionBox}1{ % Q3
    Faça corresponder o termo que melhor se adequa a cada uma das seguintes definições:
} % Q3
    \begin{itemize}
        \item A totalidade da informação genética transportada pelo DNA de uma célula ou de um organismo.
        \item A estrutura tri-dimensional do DNA, em que duas cadeias de DNA, que são mantidas juntas através de pontes de hidrogénio entre as bases azotadas, se enrolam à volta uma da outra.
        \item Descreve a orientação relativa das duas cadeias de DNA numa dupla hélice; a polaridade de uma das cadeias está orientada na direcção oposta à da outra.
        \item Dois nucleótidos numa molécula de DNA que são mantidos juntos através de pontes de hidrogénio.
    \end{itemize}

    \begin{answerBox}{} % RS 
        \begin{enumerate}
            \begin{multicols}{2}
                \item Genoma
                \item Dupla hélice
                \item 5'\to 3', antiparalela
                \item Bases Acetadas
            \end{multicols}
        \end{enumerate}
    \end{answerBox}
    
\end{questionBox}

\begin{questionBox}1{ % Q4
    Diga se as seguintes frases são verdadeiras ou falsas, e justifique a sua resposta no caso de as frases serem falsas?
} % Q4
    \begin{enumerate}
        \item Os ácidos nucleicos contêm acúcares.
        \item O DNA contém quatro diferentes bases azotadas, adenina, guanina, uracilo e citosina.
        \item Uma cadeia de DNA é polarizada porque uma das extremidades da cadeia é mais carregada do que a outra.
        \item Os pares de base G-C são mais estáveis que os pares de base A-T.
        \item As células humanas não contêm nenhuma molécula de DNA circular.
        \item As células eucariontes têm mitocondrias ou cloroplastos mas não ambos.
        \item No DNA bacteriano quase todas as sequências de DNA codificam para proteinas mas no genoma humano esta afirmação não é verdade.        
    \end{enumerate}
    
    \begin{answerBox}{} % RS 
        \begin{enumerate}
            \begin{multicols}{2}
                \item V
                \item F\\Apenas RNA contem uracilas
                \item F\\Molécula de DNA e RNA sempre é carregado de forma homogenea
                \item V
                \item F\\DNA mitocondrial é circular
                \item F\\Células de plantas contem mitocondrias
                \item V
            \end{multicols}
        \end{enumerate}
    \end{answerBox}
\end{questionBox}

\begin{questionBox}1{ % Q5
    Considere a representação dos componentes de um qualquer fragmento de DNA e faça a representação de um nucleotídeo de difosfato de adenina. Qual os números dos carbonos do açúcar envolvidos nas ligações?
} % Q5

    \begin{center}
        \includegraphics[width=.6\textwidth]{Screenshot 2022-11-10 at 17.17.40.png}
    \end{center}

    O grupo fosfato se liga ao carbono 5'

    \begin{center}
        \chemfig[angle increment=30]{
            N
            ([:35]*5(
                -*6(-N=-N=(-NH2)-=)
                -
                -N
                =
                -
            ))
            (
                -[-3,.6]O
                -[-3,]P
                (=[6,0.8]O)
                (-[0,.8]O^{-})
                (
                    -[-3,.6]O
                    -[-3]P
                    (=[6,0.8]O)
                    (-[0,.8]O^{-})
                    -[-3]O^{-}
                )
            )
        }
    \end{center}

\end{questionBox}

\begin{questionBox}1{ % Q6
    Considere a seguinte figura e complete as frases.
} % Q5

    \begin{enumerate}
        \item A figura representa um \line(1,0){3em}. 
        \item A base é uma \line(1,0){3em}. 
        \item A pentose é uma \line(1,0){3em}. 
        \item O fosfato \line(1,0){3em} é o que está ligado ao Carbono C5',
    \end{enumerate}
    
    \begin{center}
        \includegraphics[width=.8\textwidth]{Screenshot 2022-11-10 at 17.47.17.png}
    \end{center}

    \begin{enumerate}
        \begin{multicols}{2}
            \item Nucleotido trifosfatado
            \item Adeninda
            \item Ribose
            \item Fofato alpha
        \end{multicols}
    \end{enumerate}

\end{questionBox}

\begin{questionBox}1{ % Q7
    e acordo com a ``Regra de Chargaff'' qual a proporção de bases encontrada na molécula de DNA em cadeia dupla:
} % Q6
    \begin{enumerate}
        \begin{multicols}{2}
            \item C=G 
            \item C»T 
            \item C»G 
            \item C=T
        \end{multicols}
    \end{enumerate}

    a) e c)
\end{questionBox}

\begin{questionBox}1{ % Q8
    Um novo vírus foi isolado. A análise do seu genoma revelou ser constítuido de uma cadeia dupla de DNA contendo 14 \% de timinas. Baseando-se nesta informação qual a percentagem que prevê de citosinas?
} % Q8
    \begin{enumerate}
        \item 14\%
        \item 28\%
        \item 36\%
        \item 72\%
        \item Não pode ser determinado com base na informação disponível.
    \end{enumerate}

    c)

\end{questionBox}

\begin{questionBox}1{ % Q9
    Um novo bacteriófago, denominado PRR1, foi isolado e o seu material genético foi analizado. Verificou-se ser constituído por 25\% A, 33\% T, 22\% C e 20\% G. Como pode explicar estes valores?
} % Q9
    
\end{questionBox}

\begin{questionBox}1{ % Q10
    O DNA forma uma “right-handed helix”, ou seja, uma hélice dupla do DNA gira para o lado direito. Escolha destas três figuras qual a que melhor poderá representar o DNA.
} % Q10
    
    \begin{center}
        \includegraphics[width=.5\textwidth]{Screenshot 2022-11-10 at 17.55.12.png}
    \end{center}

    a)

\end{questionBox}

\begin{questionBox}1{ % Q11
    Tendo em atenção o número de bases necessárias para a dupla helice do DNA dar uma volta completa, identifique qual a opção indicada em baixo que melhor representa a molécula de DNA? Qual o nome geralmente dado a esta forma?
} % Q11
    
    \begin{center}
        \includegraphics[width=.9\textwidth]{Screenshot 2022-11-10 at 17.56.47.png}
    \end{center}

    \begin{itemize}
        \item 5
        \item 4 é a unica que gira ao contrário
    \end{itemize}

\end{questionBox}

\begin{questionBox}1{ % Q12
    Considere novamente a figura representada acima. Qual a estrutura que melhor representa a forma Z do DNA? Onde é que esta estrutura poderá ser encontrada?
} % Q12
\end{questionBox}

\begin{questionBox}1{ % Q13
    Considere que quer extrair ácidos nucleicos de um vírus de RNA, tipo coronavírus. Qual seria a escolha mais adequada de tampão de lise para lisar as partículas virais?
} % Q12
    
    \begin{center}
        \includegraphics[width=.8\textwidth]{Screenshot 2022-11-10 at 17.58.29.png}
    \end{center}

    \begin{itemize}[label=\alph{enumi}.]
        \item SDS; proteínase K; lisozima
        \item EDTA; SDS; lisozima
        \item SDS; proteinase K
        \item EDTA; SDS; proteínase K
        \item Nenhuma das hipóteses pode ser utilizadas
    \end{itemize}

    c)

\end{questionBox}

\begin{questionBox}1{ % Q14
    Considere que quer extrair DNA de uma bactéria Gram-negativa, como E. coli. Qual seria a escolha mais adequada de método para lisar as células?
} % Q14

    \begin{center}
        \includegraphics[width=.8\textwidth]{Screenshot 2022-11-10 at 18.01.41.png}
    \end{center}
    
    \begin{enumerate}[label=\alph{enumi}.]
        \item SDS; proteinase K; ultra-sons (método físico) 
        \item Nenhuma das hipóteses pode ser utilizadas 
        \item EDTA; SDS; centrifugação (método físico)
        \item SDS; proteínase K; lisozima
        \item EDTA; SDS; ciclos de congelação-descongelação (método físico)
    \end{enumerate}

    e)

\end{questionBox}

\begin{questionBox}1{ % Q15
    Considere que quer extrair DNA de células vegetais. Qual seria a escolha mais adequada de método para lisar as células?
} % Q15
    
    \begin{enumerate}[label=\alph{enumi}.]
        \item Nenhuma das hipóteses pode ser utilizadas
        \item EDTA; SDS; centrifugação (método físico)
        \item EDTA; SDS; ciclos de congelação-descongelação (método físico) 
        \item SDS; proteinase K; ultra-sons (método físico)
        \item EDTA; SDS; lisozima
    \end{enumerate}

    c)

\end{questionBox}

\begin{questionBox}1{ % Q16
    Pretende extrair apenas RNA pelo método de extração fenólica. Que condições utilizaria?
} % Q16
    \begin{enumerate}[label=\alph{enumi}.]
        \item Uma mistura de fenol-clorofórmio.
        \item Fenol a pH 7, seguido de um passo de digestão com DNase. 
        \item Fenol a pH 4.5 seguido de um passo de digestão com DNase. 
        \item Fenol a pH 4.5.
        \item Fenol a pH 7.
    \end{enumerate}

    \begin{answerBox}{} % RS 
        d). Com o pH ácido a 4.5 o DNA se encontra coagulado com proteínas deixando uma camada liquida de apenas RNA após a centrifugação;\\

        Isso funciona pois o pH ácido protona os grupos fosfátos do DNA tornando-o apolar.
    \end{answerBox}

\end{questionBox}

\end{document}
% !TEX root = ./BMC-Exercicios_Resolução.4.tex
\providecommand\mainfilename{"./BMC-Exercicios_Resolução.tex"}
\providecommand \subfilename{}
\renewcommand   \subfilename{"./BMC-Exercicios_Resolução.4.tex"}
\documentclass[\mainfilename]{subfiles}

\graphicspath{{\subfix{./.build/figures/BMC-Exercicios_Resolução.4/}}}
% \tikzset{external/force remake=true} % - remake all

% \date{\Large 16 de Novembro de 2022}

\begin{document}

\renewcommand\theenumi{\alph{enumi}}

\mymakesubfile{4}
[BMC]
{Exercicios}
{Exercicios}

\begin{questionBox}1{ % Q1
    What is the function of the enzyme DNA polymerase?
} % Q1
    \begin{enumerate}
        \item gluing together Okazaki fragments
        \item joining together nucleotides during replication 
        \item unzipping'' the two strands of DNA
    \end{enumerate}

    \paragraph*{RS:} b.
\end{questionBox}

\begin{questionBox}1{ % Q2
    Okazaki fragments occur with replicating:
} % Q2
    \begin{enumerate}
        \begin{multicols}{3}
            \item both strands
            \item the lagging strand 
            \item the leading strand
        \end{multicols}
    \end{enumerate}

    \paragraph*{RS:} b.
\end{questionBox}

\begin{questionBox}1{ % Q3
    Which of the following statements best explains the mechanism for DNA replication?
} % Q3
    \begin{enumerate}
        \item  DNA replication is reductive, because half the total DNA present is copied.
        \item DNA replication is semi-conservative, because each DNA strand serves as a template during replication.
        \item DNA replication is dispersive, because the two resulting DNA molecules are mixtures of parent and daughter DNA.
        \item DNA replication is conservative, because one resulting molecule is identical to the original and the other consists of two new strands.
    \end{enumerate}
    \paragraph*{RS:} b.
\end{questionBox}

\begin{questionBox}1{ % Q4
    In DNA replication, DNA "unwinds" to form two template strands: the leading strand and the lagging strand.
} % Q4
    \paragraph*{Which of the following statements about these strands is true?}
    \begin{enumerate}
        \item Okazaki fragments are used to synthesize the leading strand of DNA.
        \item The leading strand of DNA is synthesized continuously.
        \item DNA polymerase can only synthesize DNA on the leading strand.
        \item The lagging strand can only be synthesized once the leading strand has been completed.
    \end{enumerate}
    \paragraph*{RS} b.
\end{questionBox}

\begin{questionBox}1{ % Q5
    What enzyme breaks apart the hydrogen bonds between two strands of DNA?
} % Q5
    \begin{enumerate}
        \begin{multicols}{4}
            \item Histone
            \item Helicase
            \item Exonuclease 
            \item Endonuclease            
        \end{multicols}
    \end{enumerate}
    \paragraph*{RS} b.
\end{questionBox}

\begin{questionBox}1{ % Q6
    What enzyme replaces RNA primer on the lagging strand with DNA?
} % Q6
    \begin{enumerate}
        \begin{multicols}{4}
            \item Polymerase III 
            \item Ligase
            \item Polymerase I 
            \item Helicase
        \end{multicols}
    \end{enumerate}
    \paragraph*{RS} c.
\end{questionBox}

\begin{questionBox}1{ % Q7
    What enzyme will solve the problem of discontinuity in the lagging strand?
} % Q7
    \begin{enumerate}
        \begin{multicols}{2}
            \item Ligase
            \item Binding proteins 
            \item Helicase
            \item Polymerase I
        \end{multicols}
    \end{enumerate}
    \paragraph*{RS} a.
\end{questionBox}

\begin{questionBox}1{ % Q8
    what is the key element that kept the strands from binding back together once separated?
} % Q8
    \begin{enumerate}
        \begin{multicols}{2}
            \item Binding proteins
            \item Ligase
            \item Helicase 
            \item DNA wall
        \end{multicols}
    \end{enumerate}
    \paragraph*{RS} a.
\end{questionBox}

\begin{questionBox}1{ % Q9
    In Meselson and Stahl's experiment, \line(1,0){3em} generation(s) after cells were transferred from heavy-nitrogen medium to light nitrogen medium, all of the DNA was of hybrid density.
} % Q9
    \paragraph*{RS} 1
\end{questionBox}

\begin{questionBox}1{ % Q10
    In Meselson and Stahl's experiment, \line(1,0){3em} generation(s) after cells were transferred from heavy-nitrogen medium to light nitrogen medium, half of the DNA was of hybrid density.
} % Q10
    \paragraph*{RS} 2
\end{questionBox}

\begin{questionBox}1{ % Q11
    The main replication polymerase of E. coli is DNA polymerase \line(1,0){3em}. The enzyme that breaks the hydrogen bonds at the replication fork is called  \line(1,0){3em}. The protein that binds to single-stranded DNA to keep it from kinking up is abbreviated with the three letters  \line(1,0){3em}. The short RNA molecule made at the beginning of an Okazaki fragment is called an RNA  \line(1,0){3em}. Okazaki fragments are needed for replication on the  \line(1,0){3em} strand. A reverse transcriptase that is involved in replication of the tips of eukaryotic chromosomes is the enzyme  \line(1,0){3em}. The end of a eukaryotic chromosome is called the \line(1,0){3em}.
} % Q11
    \begin{enumerate}[label=\arabic{enumi}]
        \begin{multicols}{4}
            \item III
            \item helicase
            \item SSB
            \item primer
            \item lagging
            \item telomerase
            \item telomers
        \end{multicols}
    \end{enumerate}
\end{questionBox}

\begin{questionBox}1{ % Q12
    The replication of DNA is a complex process; all of the following statements are correct, EXCEPT
} % Q12
    \begin{enumerate}
        \item On the lagging strand, one RNA primer is required for the beginning of every Okazaki fragment.
        \item There is one replication fork in one replication bubble.
        \item DNA replication is considered to be a semi conservative process.
        \item In order to complete replication, the replication bubbles grow and merge together.
    \end{enumerate}
    \paragraph*{RS:} b.
\end{questionBox}

\begin{questionBox}1{ % Q13
    Using the given information, determine the correct order of the following events during the replication of the lagging strand.
} % Q13
    \begin{enumerate}
        \item The DNA double helix unwinds.
        \item The Okazaki fragments are joined.
        \item The RNA primase builds an RNA primer on the parent strand. 
        \item Nucleotides are added and matched to the parent strand.
    \end{enumerate}
    \paragraph*{RS:} a\to c\to d\to b
\end{questionBox}

\end{document}
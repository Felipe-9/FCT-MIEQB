% !TEX root = ./BMC-Slides_Anotações.1.tex
\providecommand\mainfilename{"./BMC-Slides_Anotações.tex"}
\providecommand \subfilename{}
\renewcommand   \subfilename{"./BMC-Slides_Anotações.1.tex"}
\documentclass[\mainfilename]{subfiles}

\graphicspath{{\subfix{./.build/figures/BMC-Slides_Anotações.1/}}}
% \tikzset{external/force remake=true} % - remake all

\begin{document}

\mymakesubfile{1}
[BMC]
{Slide 1: Célula Eucariótica} % Subfile Title
{Célula Eucariótica} % Part Title

\begin{sectionBox}1{Teoria Celular} % S1
    
    \paragraph*{Princípios}
    \begin{itemize}
        \item Todos os organismos são compostos por uma ou mais células
        \item As células são a unidade unidade básica estrutural de todos os seres vivos.
        \item Células surgem apenas por outras células pré existentes
    \end{itemize}

    Teoria ja não é universalmente aceita pois alguns cientistas consideram entidades não celulares como virus organismos vivos.
    
\end{sectionBox}

\begin{sectionBox}1m{Caracteristicas universais das células Eucarióticas} % S2
    
    \begin{itemize}
        \item \textbf{DNA}: Arquiva informações hereditárias
        \item \textbf{RNA}: transcrição de DNA
        \item \textbf{Tradução} do RNA é feita em ribossomas por polimerização de aminoácidos
        \item \textbf{Proteínas}: são usadas como agentes catalizadores
        \item \textbf{Membrana plasmática}: é o invóculo membranar e serve como interface intra e extra celular para compostos toxicos ou não
        \item \textbf{Citosol}: Englobado pela membrana plasmática, contem uma solução concentrada de compostos químicos.
        \item \textbf{Energia livre:} é requerida por todas as formas de vida
    \end{itemize}

    \begin{center}
        \includegraphics[width=1\textwidth]{hrsn1p8zwgn61.png}
    \end{center}
    
\end{sectionBox}

\begin{sectionBox}1m{Comparando Eucarióticas e Procarióticas} % S3
    
    \begin{center}
        \includegraphics[width=.6\textwidth]{Prokaryotesvseukaryotes.png}
    \end{center}


    \begin{sectionBox}*2{
        Procarióticas: Caracteristicas que se pode encontrar
    }

        \begin{description}
            \item[Nucleoide:] Uma região central da célula que contém seu DNA (não é um núcleo).

            \item[Ribossomo:] São responsáveis pela síntese de proteínas.
            
            \item[Parede celular:] Fornece estrutura e proteção do ambiente externo. A maioria das bactérias tem uma parede celular rígida feita de carboidratos e proteínas chamadas peptidoglicanos.
            
            \item[Membrana celular:] Cada procariota tem uma membrana celular, também conhecida como membrana plasmática, que separa a célula do ambiente externo.
            
            \item[Cápsula:] Algumas bactérias têm uma camada de carboidratos que envolve a parede celular chamada cápsula. A cápsula ajuda a bactéria a se fixar às superfícies.
            
            \item[Fímbrias:] São estruturas finas e semelhantes a cabelos que ajudam na fixação celular.
            
            \item[Pili:] São estruturas em forma de haste envolvidas em vários papéis, incluindo fixação e transferência de DNA.
            
            \item[Flagelada:] são estruturas finas e semelhantes a caudas que auxiliam no movimento.
        \end{description}

    \end{sectionBox}
    \begin{sectionBox}*2{
        Eucarióticas: Caracteristicas que podem encontrar
    }
            
        \begin{description}
            \item[Núcleo:] armazena as informações genéticas na forma de cromatina.

            \item[Nucleólus:] encontrado dentro do núcleo, o nucleolus é a parte das células eucarióticas onde o RNA ribossomal é produzido.

            \item[Membrana plasmática:] é uma bicamada fosfolipídica que envolve toda a célula e engloba as organelas internas.

            \item[Citoesqueleto ou parede celular:] fornece estrutura, permite o movimento celular e desempenha um papel na divisão celular.

            \item[Ribossomos:] são responsáveis pela síntese de proteínas.

            \item[Mitocôndrias:] são responsáveis pela produção de energia.

            \item[Citoplasma:] é a região da célula entre o envelope nuclear e a membrana plasmática.

            \item[Citosol:] é uma substância semelhante a um gel dentro da célula que contém os organelas.

            \item[Retículo endoplasmático:] é um organelo dedicado à maturação e transporte de proteínas.

            \item[Vesículas e vacúolos:] são sacos ligados à membrana envolvidos no transporte e armazenamento. 
        \end{description}

    \end{sectionBox}
    
\end{sectionBox}

\begin{sectionBox}1{Organismos pluricelulares} % S
    
    \begin{center}\bfseries
        Organismo
        \quad\leftarrow\quad Sistemas
        \quad\leftarrow\quad Órgãos
        \quad\leftarrow\quad Tecidos
        \quad\leftarrow\quad Células
    \end{center}

    \begin{description}
        \item[Organismo:] Sistemas funcionando em conjunto de forma ordenada
        \item[Sistemas:] Orgãos desempenhando funções especializadas
        \item[Orgãos:] Tecidos diferentes agrupados
        \item[Tecidos:] Células com a mesma morfologia/estrutura e função
    \end{description}
    
\end{sectionBox}

\end{document}
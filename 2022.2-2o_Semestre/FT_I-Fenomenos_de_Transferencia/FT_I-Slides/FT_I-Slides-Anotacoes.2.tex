% !TEX root = ./FT_I-Slides-Anotações.2.tex
\providecommand\mainfilename{"./FT_I-Slides-Anotações.tex"}
\providecommand \subfilename{}
\renewcommand   \subfilename{"./FT_I-Slides-Anotações.2.tex"}
\documentclass[\mainfilename]{subfiles}

% \graphicspath{{\subfix{../images/}}}
% \tikzset{external/force remake=true} % - remake all

\begin{document}

\mymakesubfile{2}
[FT\,I]
{Slide 2 -- Unidades e teorema \(\pi\) de Buckingham}
{Unidades e teorema \(\pi\) de Buckingham}

\begin{sectionBox}1{Teorema \(\pi\) de Buckingham} % S1
    
    \begin{BM}
        f(q_1,\dots,q_n) = g(\pi)
        \\
        \pi\subset\mathbb{R}^p
        : \pi_i = \prod_m q_j^{a_{i.j}}
        \\
        \{p,n,n\}\subset\mathbb{R} 
        : p = n-m
    \end{BM}

    \begin{itemize}
        \begin{multicols}{2}
            \item \(q\) : Parametros da função
            \item \(\pi\) : Gandezas adimensionais
            \item \(p\) : \# grupos adimensionais
            \item \(n\) : \# variáveis
            \item \(m\) : \# grandezas fundamentais presentes nas variáveis
        \end{multicols}
    \end{itemize}

    \paragraph{Note:} Se eu desejar uma relação funcional explícita para uma determinada variável, não devo incluí-la no conjunto de recurso (pois vai aparecer em mais do que um grupo adimensional e portante é mais difícil isolá-la num termo).
    
\end{sectionBox}

\begin{sectionBox}1{Numero de Reinald} % S2
    
    \begin{BM}
        Re = \frac{D\,V\,\rho}{\mu}
    \end{BM}

    \begin{itemize}
        \item \(v\) - velocidade média do fluido
        \item \({\displaystyle D}\) - longitude característica do fluxo, o diâmetro para o fluxo no tubo
        \item \(\mu\)  - viscosidade dinâmica do fluido
        \item \(\rho\)  - massa específica do fluido
    \end{itemize}

    A significância fundamental do número de Reynolds é que ele permite avaliar o tipo do escoamento (a estabilidade do fluxo) e pode indicar se flui de forma laminar ou turbulenta.

    \paragraph*{Exemplo:} Agua
    Limites \(2.000-2.400\)
    \begin{itemize}
        \item \(Re<2.000\) fluxo linear
        \item \(Re>2.400\) fluxo turbulento
        \item \(2.000<Re<2.400\) fluxo transitório
    \end{itemize}
    
\end{sectionBox}

\end{document}
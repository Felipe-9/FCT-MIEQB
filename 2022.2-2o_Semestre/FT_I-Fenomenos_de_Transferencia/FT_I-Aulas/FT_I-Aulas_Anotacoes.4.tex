% !TEX root = ./FT_I-Aulas_Anotações.4.tex
\providecommand\mainfilename{"./FT_I-Aulas_Anotações.tex"}
\providecommand \subfilename{}
\renewcommand   \subfilename{"./FT_I-Aulas_Anotações.4.tex"}
\documentclass[\mainfilename]{subfiles}

% \graphicspath{{\subfix{../images/}}}
% \tikzset{external/force remake=true} % - remake all

\begin{document}

\mymakesubfile{4}
[FT\,I]
{Aula}
{Aula}

\begin{sectionBox}1{} % S1
    
    Esperiencias de Reynald
    \\

    Em um fluxo linear pós uma agulha injetando fluido constantemente criando uma linha de corante
    Verificou que a espeçura da linha almenta e especulou se dispessar por todo o tubo
    \\

    Repetindo a experiencia com maior fluxo se percebeu distorções na linha
    \\

    Repetindo novamente com fluxo ainda maior se verificou uma rápida disperção

    \begin{BM}
        \tau_{total}
        = -(
            \mu+\rho\,E_{\tau}
        )\odv{v}{x}
        = \left(
            \mu
            +\rho\,\lambda^2\odv{v}{x}
        \right)
        \odv{v}{x}
    \end{BM}

    \paragraph{Camada limite:}
    Fronteira entre o fluido que sofre influência da parede e o inafetado
    
\end{sectionBox}

\end{document}
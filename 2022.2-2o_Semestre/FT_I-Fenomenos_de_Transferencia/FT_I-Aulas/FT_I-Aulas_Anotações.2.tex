% !TEX root = ./FT_I-Aulas_Anotações.2.tex
\providecommand\mainfilename{"./FT_I-Aulas_Anotações.tex"}
\providecommand \subfilename{}
\renewcommand   \subfilename{"./FT_I-Aulas_Anotações.2.tex"}
\documentclass[\mainfilename]{subfiles}

% \graphicspath{{\subfix{../images/}}}
% \tikzset{external/force remake=true} % - remake all

\begin{document}

\mymakesubfile{2}
[FT\,I]
{Aula 26/09/22 -- Transporte de propriedades: Calor, Massa, Quantidade de Movimento}
{Aula 26/09/22 -- Transporte de propriedades: Calor, Massa, Quantidade de Movimento}

\begin{sectionBox}1{Transporte Molecular}
    
    Ocorre em sólidos liq e gáses.

    \paragraph*{Da teoria Cinética dos gáses}
    \begin{BM}
        \Psi
        = -\frac{1}{6}l\,c\,\odv{\Gamma}{x}
        \cong 
        -\delta\,\odv{\Gamma}{x}
    \end{BM}
    \begin{itemize}
        \item[\(\delta\):] Difusividade de transporte
    \end{itemize}

    \paragraph*{Para gases reais e liquidos}
    \begin{BM}
        \frac{N_a}{A} = -D\,\odv{C_A}{x}
    \end{BM}
    \begin{itemize}
        \item[\(N_a\):] Taxa do transporte de massa (\si{\mole/\second})
        \item[\(A\):] Area do transporte (\si{\metre^2})
        \item[\(D\):] Massa da difusidade (\si{\metre^2/\second})
        \item[\(C_a\):] Concentração (\si{\mole/\metre^3}) 
    \end{itemize}
    
\end{sectionBox}

\begin{sectionBox}1{Transporte de Calor}
    
    \begin{BM}
        \Psi
        = -\delta\,\odv{\Gamma}{x}
        \\
        \frac{q}{A}
        = -\alpha
        \odv{(\rho\,C_p\,T)}{x}
        = -k\,\odv{T}{x}
    \end{BM}
    \begin{itemize}
        \item[\(\alpha\):] Difusividade térmica (\si{\metre^2/\second})
        \item[\(k\):] Condutividade térimica (\si{\frac{\joule}{\metre\,\celsius\,\second}})
    \end{itemize}
    
\end{sectionBox}

\begin{sectionBox}1{Transporte de Momento}
    
    \begin{BM}
        \Psi = -\delta\odv{\Gamma}{x}
        \\
        \tau_y
        = - \nu\odv{\rho\,v}{r}
        = - \nu\,\rho\,\odv{v}{r}
        = - \mu\odv{v}{r}
    \end{BM}
    
\end{sectionBox}

\begin{sectionBox}1{Forças que atuam em fluido}
    
    \paragraph{Tipos de Força}
    \begin{itemize}
        \item Forças físicas: atuam sem contáto físico
        \item Forças de Superfície: atuão com contáto físico e é necessária uma superfície de contato para a ação dessas forças 
    \end{itemize}
    
\end{sectionBox}

\begin{exampleBox}1{}
    
    Qual é a tensão tangencial que se deve aplicar a uma placa plana móvel que se encontra separada 1\,\si{\milli\metre} de outra placa plana fixa, para que ela se movimente a uma velocidade de 0.5\,\si{\metre/\second}, sabendo que entre as 2 existe água a 20\,\si{\celsius}? Se a placa tiver 1\,\si{\metre} de comprimento e 1.5\,\si{\metre} de largura, qual o valor da força aplicada?

    \begin{exampleBox}3{}
        
        \begin{flalign*}
            &
                \tau 
                = -\mu\odv{v}{x}
                \implies
                \int_{x_1}^{x_2} \tau\odif{x}
                = \tau\,\adif{x}\big\vert_{x_1}^{x_2}
                = -\int_{v_1}^{v_2} \mu\odif{v}
                = -\mu\adif{v}\big\vert_{v_1}^{v_2}
                \implies &\\&
                \implies
                \tau
                = -\mu\frac{v_2-v_1}{x_2-x_1}
                = -1.002\E-3\,\frac{0-0.5}{1\E-3}
                \cong
                \num{0.501}
            &
        \end{flalign*}
        
    \end{exampleBox}

    \begin{exampleBox}3{}
        
        \begin{flalign*}
            &
                F 
                = \tau\,A
                \cong \num{0.501}*1.5*1
                \cong \num{0.7515}
            &
        \end{flalign*}
        
    \end{exampleBox}
    
\end{exampleBox}

\begin{exampleBox}1{}
    
    Considere duas placas planas paralelas que estão separadas entre si de 5.1 cm. Uma delas movimenta-se a 5.1 cm s-1 e a outra, no sentido oposto a 17.8 cm s-1 A viscosidade (m) do fluído entre elas é constante e vale 363 lb ft-1 h-1
    
\end{exampleBox}

\begin{exampleBox}2{}
    
    Calcular a tensão de corte (t) em cada placa.
    
\end{exampleBox}

\begin{exampleBox}2{}
    
    Calcular a velocidade do fluído em intervalos de 1.3 cm duma placa à outra.
    
\end{exampleBox}

\begin{exampleBox}2{}
    
    Determinar a tensão de corte e os perfis de velocidade se o fluído não fôr
    
\end{exampleBox}

\end{document}
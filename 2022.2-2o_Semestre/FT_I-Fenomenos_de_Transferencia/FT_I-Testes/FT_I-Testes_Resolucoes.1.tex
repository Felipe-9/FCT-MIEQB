ra% !TEX root = ./FT_I-Testes_Resolucoes.1.tex
\providecommand\mainfilename{"./FT_I-Testes_Resolucoes.tex"}
\providecommand \subfilename{}
\renewcommand   \subfilename{"./FT_I-Testes_Resolucoes.1.tex"}
\documentclass[\mainfilename]{subfiles}

% \graphicspath{{\subfix{../images/}}}
% \tikzset{external/force remake=true} % - remake all

\begin{document}

\mymakesubfile{1}
[FT\,I]
{Teste I}
{Teste}


\setcounter{question}{1}
\begin{questionBox}1{} % Q2
    
    \begin{flalign*}
        &
            1\,\unit{\watt}
            = 1\unit{\kilo\gram\metre^2\second^{-3}}
        &
    \end{flalign*}
    
\end{questionBox}

\begin{questionBox}1{} % Q3
    
    \begin{flalign*}
        &
            3\,\unit{\pascal}
            = 3\,\unit{\kilo\gram\metre^{-1}\second^{-2}}
            = 3*10^{3-2}\,\unit{\gram\,\centi\metre^{-1}\second^{-2}}
            = 3*10\,\unit{\gram\,\centi\metre^{-1}\second^{-2}}
        &
    \end{flalign*}
    
\end{questionBox}

\begin{questionBox}1{} % Q4
    
    \begin{flalign*}
        &
            8\,\unit{\joule}
            = 8\,\unit{\newton\,\metre}
            = 8\,\unit{\kilo\gram\,\metre^2\,\second^{-2}}
            = 8\,\unit{\kilo\gram\,\metre^2\,\second^{-2}}
            \,\frac{\unit{lb}}{0.4536\,\unit{\kilo\gram}}
            \,\left(
                \frac{\unit{ft}}{0.3048\,\unit{\metre}}
            \right)^2
            = &\\&
            = 8\,\unit{lb\,ft^2\,\second^{-2}}
            \,\frac{1}{0.4536}
            \frac{1}{0.3048^2}
            \cong
            \qty{189.839689889060358}{lb\,ft^2\,\second^{-2}}
        &
    \end{flalign*}
    
\end{questionBox}

\begin{questionBox}1{} % Q5
    
    \begin{flalign*}
        &
            = \unit{M\,L^{-1}\,T^{-2}}
            = [A]
        &
    \end{flalign*}
    
\end{questionBox}

\begin{questionBox}1{} % Q6
    
    \begin{flalign*}
        &
            [V]^a
            = (\unit{L^{3}})^a
            = [k]
            \,[m]^b
            \,[\adif{T}]^{-6}
            \,[P]^n
            = (\unit{M})^b
            \,(\unit{T})^{-6}
            \,(\unit{M\,L^{-1}\,T^{-2}})^n
            = &\\&
            = \unit{
                M^{b+n}
                \,T^{-6-2\,n}
                \,L^{-n}
            }
            \implies
            \left\{
                \begin{aligned}
                    b+n = 0 \implies & b = 3
                    \\
                    -6-2\,n = 0 \implies & n = -3
                    \\
                    -n=3\,a \implies & a=1
                \end{aligned}
            \right\}
        &
    \end{flalign*}
    
\end{questionBox}

\begin{questionBox}1{} % Q7
    
    \begin{BM}
        \l[\adif{P}\r] = \unit{M\,L^{-1}\,T^{-2}}
        \\[2ex]
        [D] = \unit{L}
        \qquad
        [\omega] = \unit{M\,L^2\,T^{-2}}
        \\
        [\rho] = \unit{M\,L^{-3}}
        \qquad
        [G_v] = \unit{L^3\,T^{-1}}
    \end{BM}
    
\end{questionBox}

\begin{questionBox}1{} % Q8
    
    \begin{BM}
        \l[ V \r] = \unit{L\,T^{-1}}
        \\[2ex]
        [d] = \unit{L}
        \qquad
        [\mu] = \unit{M\,L^{-1}\,T^{-2}}
        \qquad
        [\gamma] = [\gamma_s] = \unit{M\,L^{-3}}
    \end{BM}
    
\end{questionBox}

\begin{questionBox}1{} % Q9
    
    \begin{BM}
        h
        = \frac{f\,L\,V^2}{2\,D\,g}
    \end{BM}

    \begin{flalign*}
        &
            [h]
            = \unit{L}
            = \frac{[f]\,[L]\,[V]^2}{[2]\,[D]\,[g]}
            = \frac{
                [f]
                \,\unit{L}
                \,(\unit{L\,T^{-1}})^2
            }{
                (\unit{L})\,(\unit{L\,T^{-2}})
            }
            = [f]
            \,\unit{L^1}
            \implies
            [f] = 1
        &
    \end{flalign*}
    
\end{questionBox}

\begin{questionBox}1{} % Q10
    
    \begin{flalign*}
        &
            G_s
            = v\,S
            = 23\,\pi\,(11\E-1)^2
            = 23\,\pi\,(11)^2\E-2
            \cong\num{8.7430523549e1}
        &
    \end{flalign*}
    
\end{questionBox}

\begin{questionBox}1{} % Q11
    
    \begin{BM}
        \bar{v}
        =\frac{D^2}{32\,\mu}
        \,\frac{-\adif{P}}{L}
    \end{BM}

    \begin{flalign*}
        &
        \bar{v}
        =\frac{D^2}{32\,\mu}
        \,\frac{-\adif{P}}{L}
        =\frac{(2*2.5\E-5)^2}{32*0.003}
        \,\frac{1.3\E3}{1.1\E-3}
        =\frac{(2*2.5)^2}{32*0.003}
        \,\frac{1.3}{1.1}
        \E-4
        \cong\num{3.07765151515151515e-2}
        &
    \end{flalign*}
    
\end{questionBox}

\begin{questionBox}1{} % Q12
    
    \begin{BM}
        \bar{v}
        =\frac{D^2}{32\,\mu}
        \,\frac{-\adif{P}}{L}
    \end{BM}

    \begin{flalign*}
        &
            \mu
            =\frac{D^2}{32\,\bar{v}}
            \,\frac{-\adif{P}}{L}
            =\frac{(8\E-2)^2}{32\,(G_s/S)}
            \,\frac{8\E6}{50}
            =\frac{(8\E-2)^2}{32\,((0.2/60)/(\pi*(8\E-2/2)^2))}
            \,\frac{8\E6}{50}
            = &\\&
            =\frac{
                8^3
            }{
                32*50
                \,\frac{
                    (0.2/60)
                }{
                    (\pi*(8/2)^2)
                }
            }
            \E-2
            = &\\&
            \cong\num{48.25486315913922105}
        &
    \end{flalign*}
    
\end{questionBox}

\setcounter{question}{16}
\begin{questionBox}1{} % Q17
    
    \begin{flalign*}
        &
            \int\tau\odif{x}
            =\int\mu\odif{v}
            \implies
            \mu
            = \tau\frac{\adif{x}}{\adif{v}}
            = 349\frac{7\E-2}{1}\,\unit{\pascal}
            \cong\qty{244.3}{poise}
        &
    \end{flalign*}
    
\end{questionBox}

\begin{questionBox}1{} % Q18

    \begin{BM}
        v_x(y)
        = \left(
            -\odv{P}{x}
        \right)
        \,\frac{H^2}{8\,\mu}
        \,\left(
            1-\left(
                \frac{2\,y}{H}
            \right)^2
        \right)
    \end{BM}

    \begin{flalign*}
        &
            \tau 
            = \frac{\mu}{\adif{y}}\adif{v}
            = \frac{\mu}{H}
            \,-\left(
                \left(
                    -\odv{P}{x}
                \right)
                \,\frac{H^2}{8\,\mu}
                \,\left(
                    1-\left(
                        \frac{2\,H}{H}
                    \right)^2
                \right)
            \right)
            = &\\&
            =-\left(
                11520
                \,\frac{3\E-1}{8}
                \,\left(
                    1-4
                \right)
            \right)
        &
    \end{flalign*}
    
\end{questionBox}

\end{document}
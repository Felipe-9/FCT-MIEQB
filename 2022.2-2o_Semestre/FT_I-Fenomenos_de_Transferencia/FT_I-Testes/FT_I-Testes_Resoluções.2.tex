% !TEX root = ./FT_I-Testes_Resoluções.2.tex
\providecommand\mainfilename{"./FT_I-Testes_Resoluções.tex"}
\providecommand \subfilename{}
\renewcommand   \subfilename{"./FT_I-Testes_Resoluções.2.tex"}
\documentclass[\mainfilename]{subfiles}

% \graphicspath{{\subfix{./.build/figures/}}}
% \tikzset{external/force remake=true} % - remake all

\begin{document}

\mymakesubfile{2}
[FT\,I]
{Teste 2} % Subfile Title
{Teste 2} % Part Title

\group{Prática}

\begin{questionBox}1{ % Q1
} % Q1
    

    \begin{itemize}
        \begin{multicols}{2}
            \item \(-\adif{P}_b = 520*10^{3}\)
            \item \(L = 190\)
            \item \(D =16*10^{-2}\)
            \item 2 Joelhos 90
            \item 4 v.Guilhotina
            \begin{itemize}
                \item 2: 3/4 Abertas
                \item 2: 1/4 Abertas
            \end{itemize}
            \item \(\varepsilon=1.44*10^{-3}\)
            \item \(G_v = 0.06\)
            \item \(P_t = P_{atm}\)
            \item \(\mu = 0.001\)
            \item \(\rho = 10^3\)
        \end{multicols}
    \end{itemize}

\end{questionBox}

\begin{questionBox}2m{ % Q1.2
    Altura máxima do tanque
} % Q1.2
    
    \begin{flalign*}
        &
            \max{Z_2}
            = h_{b}-h_{at}
            = \frac{-\adif{P}_{b}}{\rho\,g}
            -\frac{-\adif{P}_{at}}{\rho\,g}
            = &\\&
            = \left(
                -\adif{P}_{b}
                -4\,\phi\,\rho\,L_{eq}\,v^2/D
            \right)
            (\rho\,g)^{-1}
            = &\\&
            = \left(
                -\adif{P}_{b}
                -\frac{4\,\phi\,\rho\,(190+D(2*35+2*200+2*40))}{D}
                \left(
                    \frac{G_v}{\pi\,(D/2)^2}
                \right)^2
            \right)
            (\rho\,g)^{-1}
            = &\\&
            = 
            \frac{-\adif{P}_{b}}{\rho\,g}
            -\frac{
                4^3\,\phi\,(190+D(2*35+2*200+2*40))\,G_v^2
            }{
                \pi^2\,D^5\,g
            };
            &\\[1.5ex]&
            % 
            % 
            % 
            \phi\left(
                Re,\varepsilon/D
            \right)
            = \phi\left(
                \frac{\rho\,v\,D}{\mu},
                \frac{\varepsilon}{D}
            \right)
            = \phi\left(
                \frac{\rho\,D}{\mu}
                \frac{G_v}{\pi\,(D/2)^2}
                ,\frac{\varepsilon}{D}
            \right)
            = \phi\left(
                \frac{4\,\rho\,G_v}{\mu\,\pi\,D}
                ,\frac{\varepsilon}{D}
            \right)
            = &\\&
            = \phi\left(
                \frac{4*10^3*0.06}{0.001*\pi*16*10^{-2}}
                , \frac{1.44*10^{-3}}{16*10^{-2}}
            \right)
            = \phi\left(
                \frac{3}{\pi*2}*10^6
                , \frac{1.44*10^{-1}}{16}
            \right)
            \cong &\\&
            \cong \phi\left(
                \num{4.77464829275686e5}
                , 9*10^{-3}
            \right)
            \cong 0.00455
            &\\[1.5ex]&
            % 
            % 
            % 
            \therefore
            \max Z_2
            \cong &\\&
            \cong 
            \frac{520*10^{3}}{10^3*\num{9.780327}}
            -\frac{
                4^3*0.00455*(190+16*10^{-2}(2*35+2*200+2*40))\,(0.06)^2
            }{
                \pi^2\,(16*10^{-2})^5*\num{9.780327}
            }
            = &\\&
            = 
            \frac{520}{\num{9.780327}}
            -\frac{
                4^3*4.55*(190+16*10^{-2}(2*35+2*200+2*40))\,(0.06)^2
            }{
                \pi^2\,(16)^5*\num{9.780327}
            }*10^{7}
            \cong &\\&
            \cong
            % 28.792914871881125 + 53.167956449717888
            \SI{81.960871321599012}{\metre}
        &
    \end{flalign*}

\end{questionBox}

\begin{questionBox}2m{ % Q1.2
    Corrosão
} % Q1.2
    
    \begin{itemize}
        \item \(\varepsilon=4\,\varepsilon_0\)
    \end{itemize}

    \begin{flalign*}
        &
            \frac{G_{v.1}}{G_{v.0}}
            = G_{v.0}^{-1}
            \,v_1\,\pi\,(D/2)^2
            =\frac{\pi\,D^2}{G_{v.0}\,4}
            \left(
                \frac{Re_1\,\mu}{D\,\rho}
            \right)
            =\frac{\pi\,D}{G_{v.0}\,4}
            \left(
                \frac{Re_1\,\mu}{\rho}
            \right)
            =\frac{\pi\,D\,Re_1\,\mu}{G_{v.0}\,4\,\rho}
            % &\\[1.5ex]&
            % % 
            % % \varepsilon_0
            % % 
            % \varepsilon_0\left(
            %     Re_0,\phi_0
            % \right)
            % = \varepsilon_0\left(
            %     \frac{\rho\,v\,D}{\mu},
            %     \frac{-\adif{P}\,D}{4\,\rho\,L_{eq}\,v^2}
            % \right)
            % = &\\&
            % = \varepsilon_0\left(
            %     \frac{\rho\,D}{\mu}
            %     \frac{G_v}{\pi\,(D/2)^2}
            %     ,
            %     \frac{
            %         -\adif{P}\,D
            %     }{
            %         4\,\rho\,L_{eq}
            %     }
            %     \left(
            %         \frac{G_v}{\pi\,(D/2)^2}
            %     \right)^{-2}
            % \right)
            % % L_eq = 278
            % = &\\&
            % = \varepsilon_0\left(
            %     \frac{4\,\rho\,G_v}{\mu\,\pi\,D}
            %     ,
            %     \frac{
            %         h_{at}\,\rho\,g\,\pi^2\,D^5
            %     }{
            %         4^3\,\rho\,L_{eq}\,G_v^2
            %     }
            % \right)
            % = &\\&
            % = \varepsilon_0\left(
            %     \frac{4\,\rho\,G_v}{\mu\,\pi\,D}
            %     ,
            %     \frac{
            %         (h_b-Z_2)\,\rho\,g\,\pi^2\,D^5
            %     }{
            %         4^3\,\rho\,L_{eq}\,G_v^2
            %     }
            % \right)
            % = &\\&
            % = \varepsilon_0\left(
            %     \frac{4\,\rho\,G_v}{\mu\,\pi\,D}
            %     ,
            %     \frac{
            %         \left(
            %             \frac{-\adif{P}_b}{\rho\,g}
            %             -Z_2
            %         \right)
            %         \,\rho\,g\,\pi^2\,D^5
            %     }{
            %         4^3\,\rho\,L_{eq}\,G_v^2
            %     }
            % \right)
            % = &\\&
            % = \varepsilon_0\left(
            %     \frac{4\,\rho\,G_v}{\mu\,\pi\,D}
            %     ,
            %     \frac{
            %         \left(
            %             (-\adif{P}_b/\rho)
            %             -Z_2\,g
            %         \right)
            %         \,\pi^2\,D^5
            %     }{
            %         4^3\,L_{eq}\,G_v^2
            %     }
            % \right)
            % \cong &\\&
            % \cong \varepsilon_0\left(
            %     \frac{
            %         4*10^3*1.44*10^{-3}
            %     }{
            %         0.001*\pi*16*10^{-2}
            %     }
            %     ,
            %     \frac{
            %         \left(
            %             (520*10^{3}/10^3)
            %             -30*\num{9.780327}
            %         \right)
            %         \pi^2\,(16*10^{-2})^5
            %     }{
            %         4^3*278*(0.06)^2
            %     }
            % \right)
            % = &\\&
            % = \varepsilon_0\left(
            %     \frac{4*1.44}{\pi*16}
            %     \,10^{5}
            %     ,
            %     \frac{
            %         \left(
            %             520
            %             -30*\num{9.780327}
            %         \right)
            %         \pi^2\,16^5*10^{-6}
            %     }{
            %         4^3*278*(6)^2
            %     }
            % \right)
            % = &\\&
            % = \varepsilon_0\left(
            %     \frac{4*1.44}{\pi*16}
            %     \,10^{5}
            %     ,
            %     \frac{
            %         \left(
            %             520
            %             -30*\num{9.780327}
            %         \right)
            %         \pi^2\,16^5*10^{-6}
            %     }{
            %         4^3*278*(6)^2
            %     }
            % \right)
            % \cong &\\&
            % \cong
            % \varepsilon_0\left(
            %     \num{1.145915590261646e4}
            %     ,
            %     \num{3.661116018134129739e-3}
            % \right)
            % \cong 0
        &
    \end{flalign*}
% \end{questionBox}
% \begin{questionBox}{} % Q
    \begin{flalign*}
        &
            % 
            % Re_1
            % 
            Re_1\left(
                \phi\,Re^2,\varepsilon_1/D
            \right)
            = Re_1\left(
                \left(
                    \frac{-\adif{P}_{at}\,D}{
                        4\,L_{eq}\,v_1^2\,\rho
                    }
                \right)
                \left(
                    \frac{\bar{v}_1\,D\,\rho}{\mu}
                \right)^2
                ,\frac{4\,\varepsilon_0}{D}
            \right)
            = &\\&
            = Re_1\left(
                \left(
                    -\adfi{P}_b
                    -Z_2\,\rho\,g
                \right)
                \frac{
                    D^3\,\rho
                }{
                    4\,L_{eq}\,\mu^2
                }
                ,\frac{4\,\varepsilon_0}{D}
            \right)
            = &\\&
            = Re_1\left(
                \left(
                    520*10^3
                    -30*10^3*\num{9.780327}
                \right)
                \frac{
                    (16*10^{-2})^3*10^3
                }{
                    4*278*(10^{-3})^2
                }
                ,4*1.44*10^{-3}
            \right)
            = &\\&
            = Re_1\left(
                \left(
                    520
                    -30*\num{9.780327}
                \right)
                \frac{4^5}{278}*10^{6}
                ,4*1.44*10^{-3}
            \right)
            = &\\&
            = Re_1\left(
                \num{8.34634368920863309e8}
                ,0.00576
            \right)
            =4.5*10^5
            &\\[1.5ex]&
            % 
            % 
            % 
            \frac{G_{v.1}}{G_{v.0}}
            % = &\\&
            =
            \frac{\pi*16*10^{-2}}{0.06*4}
            \left(
                \frac{4.5*10^5*10^{-3}}{10^3}
            \right)
            =\frac{\pi*16*10^{-2}*4.5*10^5*10^{-3}}{0.06*4*10^3}
            = &\\&
            =\frac{\pi*2*4.5}{30}
            \cong\num{94.2477796076938}\%
            &\\[1.5]&
            \therefore
            \text{ Diminuiu } \num{5.7522203923062}\%
        &
    \end{flalign*}

\end{questionBox}

\group{Teórica}

\begin{questionBox}1{ % Q1
    3 mecanismos
} % Q1
    
    \begin{questionBox}3{ % Q
    } % Q
        Atrito entre fluido e placa que aplica uma resistencia ao movimento e caso haja grande fluxo comitar em turbulência
    \end{questionBox}
    
    \begin{questionBox}3{ % Q
    } % Q
        Aplicação de uma bomba para transporte de fluido de um tanque a outro onde devemos levar em consideração a variação de pressão e potencial gravitico
    \end{questionBox}

    \begin{questionBox}3{ % Q
        Diminuição da area de transporte de um fluido em multiplos tubos paralelos que permite maior fluxo linear evitando turbulencias
    } % Q
        body
    \end{questionBox}

\end{questionBox}

\begin{questionBox}1{ % Q
    Situações do dia a dia
} % Q
    
    Coléta de petróleo de um reservatório subterrâneo pode ser considerada como uma aplicação de bomba de diferentes potenciais gravíticos\\
    
    Transporte comercial de vinho-vias subterrâneas para transportadoras permitindo maior rendimento de uma fábrica e organização de distribuição do produto\\

    Balancemanto da distribuição de Caixas d'agua para residencias tendo que levar em conta a perda de energia e quantas residencias essa pode atender

\end{questionBox}

\end{document}
% !TEX root = ./FT_I-Exercicios_Resolução.3.tex
\providecommand\mainfilename{"./FT_I-Exercicios_Resolução.tex"}
\providecommand \subfilename{}
\renewcommand   \subfilename{"./FT_I-Exercicios_Resolução.3.tex"}
\documentclass[\mainfilename]{subfiles}

% \graphicspath{{\subfix{./.build/figures/}}}
% \tikzset{external/force remake=true} % - remake all

\begin{document}

\mymakesubfile{3}
[FT\,I]
{Exercicios} % Subfile Title
{Exercicios} % Part Title

\begin{questionBox}1m{ % Q1
    Calcular a queda de pressão devido ao atrito de um óleo que flui a uma velocidade média de 2.4\,\si{\metre\,\second^{-1}} através de um tubo liso com 30\,\si{\metre} de comprimento e 7.6\,\si{\centi\metre} de diâmetro (comparar com comprimento 20\,\si{\metre}, 30\,\si{\metre} e 50\,\si{\metre}).
} % Q1

    \begin{itemize}
        \begin{multicols}{2}
            \item \(\mu=5\,\si{\centi P}\)\\(comparar com 4\,\si{\centi P} e 8\,\si{\centi P})
            \item \(\rho = 960\,\si{\kilo\gram\,\metre^{-3}}\)
        \end{multicols}
    \end{itemize}

    \subsubsection*{
        E qual a queda de pressão devido ao atrito se rugosidade do tubo = 0.08\,\si{\milli\metre}? (comparar com 0.2\,\si{\milli\metre} e 0.8\,\si{\milli\metre}). E qual a queda de pressão devido ao atrito se tubo liso com 2 joelhos em ângulo recto?
    }

    \begin{answerBox}{(i)} % RS 
        \begin{flalign*}
            &
                -\adif{P}_{at}
                = 4\,\phi\,L\,\rho\,v^2/D;
                &\\&
                \phi\left(
                    Re,0
                \right)
                =
                \phi\left(
                    \frac{\rho\,v\,D}{\mu}
                    , 0
                \right)
                =
                \phi\left(
                    \frac{960*2.4*7.6\E-2}{
                        5\E-2*10^{-3}/10^{-2}
                    }
                    , 0
                \right)
                = &\\&
                =
                \phi\left(
                    \frac{9.6*2.4*7.6}{5}
                    \,10^{2}
                    , 0
                \right)
                \cong
                \phi\left(
                    \num{35020.8}
                    , 0
                \right)
                \cong 0.00275
                &\\&
                \therefore
                -\adif{P}_{at}
                \cong 4*0.00275*30*960*(2.4)^2/7.6\E-2
                = &\\&
                = \frac{4*2.75*3.0*9.60*(2.4)^2}{7.6}\E{2}
                \cong
                \num{2.40101052631578947e4}
            &
        \end{flalign*}

        \vspace{-4ex}

        
    \end{answerBox}

    \begin{answerBox}{(ii)} % RS 
        \begin{table}[H]\centering
            \begin{tabular}{*{4}{c}}
                
                \\\toprule
                
                    \multicolumn{1}{c}{L/\si{\metre}:}
                    & \multicolumn{1}{c}{20}
                    & \multicolumn{1}{c}{30}
                    & \multicolumn{1}{c}{50}
                
                \\\midrule
                
                    \( -\adif{P}_{at} \):
                    & \( \num{160.067368421052632e2}\)
                    & \( \num{2.40101052631578947e4}\)
                    & \( \num{400.168421052631579e2}\)
                
                \\\bottomrule
                
            \end{tabular}
        \end{table}
    \end{answerBox}

    \begin{answerBox}{(iii)} % RS 
        
        \begin{flalign*}
            &
                -\adif{P}_{at}
                = 4\,\phi\,L\,\rho\,v^2/D;
                &\\&
                \phi\left(
                    Re,\varepsilon/D
                \right)
                = \phi\left(
                    \num{35020.8},
                    0.08\E-3/7.6\E-2
                \right)
                = &\\&
                = \phi\left(
                    \num{35020.8},
                    8\E-3/7.6
                \right)
                \cong 
                \phi\left(
                    \num{35020.8},
                    \num{1.052631578947368e-3}
                \right)
                \cong
                0.0052
                &\\&
                \therefore
                -\adif{P}_{at}
                \cong 
                \num{2.40101052631578947e4}
                \,\frac{0.0052}{0.00275}
                \cong
                \num{4.5400926315789464727e4}
            &
        \end{flalign*}

    \end{answerBox}

    \begin{answerBox}{(iv)} % RS 
        \begin{flalign*}
            &
                \varepsilon>\varepsilon_0
                \implies
                \phi>\phi_0
                \implies
                -\adif{P}_{at}
                > 
                -\adif{P}_{at, 0}
            &
        \end{flalign*}
    \end{answerBox}

    \begin{answerBox}{(v)} % RS 
        \begin{flalign*}
            &
                -\adif{P}_{at}
                = 4\,\phi\,L_{eq}\,\rho\,v^2/D
                \cong 
                \num{2.40101052631578947e4}
                \frac{L_{eq}}{L}
                = &\\&
                = \num{2.40101052631578947e4}
                \frac{(L + 2*40*D)}{L}
                = \num{2.40101052631578947e4}
                (1+2*40*7.6\E-2/30)
                \cong &\\&
                \cong
                \num{2.887615326315789e4}
            &
        \end{flalign*}
    \end{answerBox}

\end{questionBox}

\begin{questionBox}1{ % Q2
    Corre água a 2.5\,\si{\deci\metre^3\,\second^{-1}} através dum alargamento súbito de um tubo de 3.6\,\si{\centi\metre} de diâmetro para um de 4.8\,\si{\centi\metre}. Qual é a perda de carga em \textit{m}?
} % Q2
    
    \begin{BM}
        -\adif{P}_{al\ arg}^{at}
        =\rho\,(v_1-v_2)^2/2
    \end{BM}


    \begin{answerBox}{} % RS 
        \begin{flalign*}
            &
                \frac{-\adif{P}_{at}}{\rho\,g}
                = \frac{
                    \rho\,(\adif{v})^2/2
                }{\rho\,g}
                = \frac{
                    \left(
                        \adif{\frac{G_v}{\pi\,r^2}}
                    \right)^2
                }{2\,g}
                = \left(
                    \frac{G_v}{\pi\,\adif{r^2}}
                \right)^2
                (2\,g)^{-1}
                \cong &\\&
                \cong 
                \left(
                    \frac{2.5\E-3}{
                        \pi\,(
                            (3.6\E-2/2)^2
                            -(4.8\E-2/2)^2
                        )
                    }
                \right)^2
                (2*\num{9.780327})
                = &\\&
                = 8\,\left(
                    \frac{2.5}{
                        \pi\,(
                            (3.6)^2
                            -(4.8)^2
                        )
                    }
                \right)^2
                \num{9.780327}^{-1}
                *10^{2}
                \cong
                \num{5.9028669606249e-2}
            &
        \end{flalign*}
    \end{answerBox}

\end{questionBox}

\begin{questionBox}1{ % Q3
    Qual é a queda de pressão, e a potência necessária para bombear 0.04\,\si{\metre^3\,\second^{-1}} de água, através dum condensador com 400 tubos de 4.5\,\si{\metre} de comprimento e diâmetro interno de 1\,\si{\centi\metre} sabendo que o coeficiente de contracção à entrada dos tubos (\(C_c\)) é 0.6 e rugosidade aço comercial = 0.046\,\si{\milli\metre} \(\mu = 10^{-3}\,\si{\kilo\gram\,\metre^{-1}\,\second^{-1}}\).
} % Q3

    \begin{BM}
        -\adif{P}_{cont}
        = \frac{\rho\,v^2}{2}
        (C_c^{-1}-1)^2
    \end{BM}

    \begin{answerBox}{(i)} % RS 
        \begin{flalign*}
            &
                -\adif{P}_{tot}
                = 
                -\adif{P}_{at\ 1}*n
                -\adif{P}_{at\ 2}
                = 
                n\,\frac{4\,\phi\,\rho\,v^2\,L_1}{D}
                + \frac{\rho\,v^2}{2}
                (C_c^{-1}-1)^2
                = &\\&
                = 
                \rho\,v^2
                \,\left(
                    \frac{4\,\phi\,L_1\,n}{D}
                    + \frac{(C_c^{-1}-1)^2}{2}
                \right)
                &\\[1.5em]&
                \phi\left(
                    Re,\varepsilon/D
                \right)
                = \phi\left(
                    \frac{D\,\rho\,\bar{v}}{\mu},
                    \varepsilon/D
                \right)
                = \phi\left(
                    \frac{10^{-2}*10^{3}*1.27}{10^{-3}},
                    \frac{0.046\E{-3}}{10^{-2}}
                \right)
                = &\\&
                = \phi\left(
                    1.27*10^{4},
                    0.046\E{-1}
                \right)
                \cong
                0.0052
                &\\[1.5em]&
                \therefore
                -\adif{P}_{tot}
                \cong 10^3\,(1.27)^2
                \,\left(
                    \frac{4*0.0052*4.5*400}{1\E-2}
                    + \frac{(0.6^{-1}-1)^2}{2}
                \right)
                \cong
                \num{6039.056022222222222e3}
            &
        \end{flalign*}
    \end{answerBox}

    \begin{answerBox}{(ii)} % RS 
        \begin{flalign*}
            &
                W_{bomba}
                =-\adif{P}_{at\ bomba}
                *\frac{G_v}{n}
                \cong
                \num{6039.056022222222222e3}
                *\frac{0.04}{400}
                \cong
                \num{6.03905602222222e2}
            &
        \end{flalign*}
    \end{answerBox}
    
\end{questionBox}

\begin{questionBox}1m{ % Q4
    Quer-se bombear água dum tanque para um depósito 12\,\si{\metre} acima do nível daquele, a um caudal de 1.25\,\si{\deci\metre^3\,\second^{-1}}, através dum tubo de ferro de 25\,\si{\milli\metre} de diâmetro e 30\,\si{\metre} de comprimento. O tanque e o reservatório encontram-se à pressão atmosférica.\\
    Qual é a potência da bomba necessária?
} % Q4
    
    \begin{itemize}
        \begin{multicols}{2}
            \item \( \mu  = 1.30*10^{-3}\,\si{\kilo\gram\,\metre^{-1}\,\second^{-1}} \)
            \item \( \rho = 1000\,\si{\kilo\gram\,\metre^{-3}}\)
            \item Rugosidade do ferro \(=0.046\,\si{\milli\metre}\)
        \end{multicols}
    \end{itemize}

    \begin{answerBox}{} % RS 
        \begin{flalign*}
            &
                W_{b}
                = -\adif{P}_{b}\,G_v
                = h_{b}\,\rho\,g\,G_v
                = &\\&
                = \left(
                    Z_2+\frac{v^2}{2\,g}
                    + h_{at}
                \right)
                \,\rho\,g\,G_v
                = &\\&
                = \left(
                    Z_2+\frac{v^2}{2\,g}
                    + \frac{-\adif{P}_{at}}{\rho\,g}
                \right)
                \,\rho\,g\,G_v
                = &\\&
                = 
                Z_2\,\rho\,g\,G_v
                + \left(
                    \frac{v^2}{2}
                    + \frac{4\,\phi\,\rho\,v^2\,L/D}{\rho}
                \right)
                \,\rho\,G_v
                = &\\&
                =
                Z_2\,\rho\,g\,G_v
                +\left(
                    \frac{1}{2}
                    + \frac{4\,\phi\,L}{D}
                \right)
                \,v^2
                \,\rho\,G_v
                = &\\&
                =
                Z_2\,\rho\,g\,G_v
                +\left(
                    \frac{1}{2}
                    + \frac{4\,\phi\,L}{D}
                \right)
                \,\left(
                    \frac{G_v}{\pi\,(D/2)^2}
                \right)^2
                \,\rho\,G_v
                = &\\&
                =
                Z_2\,\rho\,g\,G_v
                +\left(
                    2^3
                    + \frac{4^3\,\phi\,L}{D}
                \right)
                \,\frac{G_v^3\rho}{\pi^2\,D^4};
                % &\\[1.5em]&
                % W_{b}
                % = h_{at}\,G_v
                % = \frac{-\adif{P}_{at}}{\rho\,g}\,G_v
                % = \frac{4\,\phi\,\rho\,v^2/D}{\rho\,g}\,G_v
                % = \frac{4\,\phi}{D\,g}
                % \,\left(
                %     \frac{G_v}{\pi\,(D/2)^2}
                % \right)^2
                % \,G_v
                % = &\\&
                % = \frac{4^3\,\phi\,G_v^3}{D^5\,\pi^2\,g};
                &\\[1.5ex]&
                \phi\left(
                    Re,\varepsilon/D
                \right)
                = \phi\left(
                    \frac{\rho\,D\,\bar{v}}{\mu},
                    \varepsilon/D
                \right)
                = \phi\left(
                    \frac{10^3*25*10^{-3}2.55}{1.30*10^{-3}},
                    \frac{0.046\E{-3}}{25\E{-3}}
                \right)
                = &\\&
                = \phi\left(
                    \frac{25*2.55}{1.30}\,10^{3},
                    0.046/25
                \right)
                \cong \phi\left(
                    \num{49.038461538461538e3},
                    \num{1.84e-3}
                \right)
                \cong 0.00255
            &
        \end{flalign*}
    \end{answerBox}
    \begin{answerBox}*{} % RS 
        \begin{flalign*}
            &
                \therefore
                W_b
                % = &\\&
                = 12*10^{3}*\num{9.780327}*1.25*10^{-3}
                + &\\&
                +\left(
                    2^3
                    + \frac{4^3*0.00255*30}{25*10^{-3}}
                \right)
                \,\frac{
                    (1.25*10^{-3})^3*10^3
                }{
                    \pi^2\,(25*10^{-3})^4
                }
                = &\\&
                = 12*\num{9.780327}*1.25
                +\left(
                    2^3
                    + \frac{4^3*2.55*30}{25}
                \right)
                \,\frac{
                    (1.25)^3
                }{
                    \pi^2\,(25)^4
                }*10^{6}
                \cong
                \num{249.971455368270657}
            &
        \end{flalign*}
    \end{answerBox}

\end{questionBox}

\begin{questionBox}1m{ % Q5
    Pretende-se bombear 4\,\si{\deci\metre^3\,\second^{-1}} de uma solução de ácido sulfúrico através dum tubo de 2.5\,\si{\centi\metre} de diâmetro, em chumbo, e a uma altura de 25\,\si{\metre}. O tubo tem 30\,\si{\metre} de comprimento e contém dois joelhos em ângulo recto. Calcular a potência da bomba teoricamente necessária.
} % Q5
    
    \begin{itemize}
        \begin{multicols}{2}
            \item \( \rho_{sol\ ac}=1531\,\si{\kilo\gram\,\metre^{-3}}\)
            \item \( \mu_{sol\ ac} = 0.065\,\si{\kilo\gram\,\metre^{-1}\,\second^{-1}}\)
            \item rugosidade chumbo \(= 0.05\,\si{\milli\metre}\)
        \end{multicols}
    \end{itemize}

    \begin{answerBox}{} % RS
        \begin{flalign*}
            & 
                W_b
                = -\adif{P}_{b}\,G_v
                = h_b\,\rho\,g\,G_v
                = &\\&
                = \left(
                    Z_2
                    + h_{at}
                \right)
                \,\rho\,g\,G_v
                = &\\&
                = 
                Z_2\,\rho\,g\,G_v
                + \left(
                    \frac{-\adif{P}_{at}}{\rho\,g}
                \right)
                \,\rho\,g\,G_v
                = &\\&
                = 
                Z_2\,\rho\,g\,G_v
                + \left(
                    4\,\phi\,\rho\,v^2\,L_{eq}/D
                \right)
                \,G_v
                = &\\&
                = 
                Z_2\,\rho\,g\,G_v
                + \left(
                    \frac{G_v}{\pi\,r^2}
                \right)^2
                \,\left(
                    L + 2*40*D
                \right)
                \,\frac{4\,\phi\,\rho\,G_v}{D}
                = &\\&
                = 
                Z_2\,\rho\,g\,G_v
                + \frac{
                    4^3\,\phi\,\rho\,(L+2*40*D)\,G_v^2
                }{
                    \pi^2\,D^5
                }
                = &\\&
                = 
                Z_2\,\rho\,g\,G_v
                + \frac{
                    4^3\,\phi\,\rho\,(L+2*40*D)\,G_v^2
                }{
                    \pi^2\,D^5
                };
                &\\[1.5ex]&
                \phi\left(
                    Re,\varepsilon/D
                \right)
                = \phi\left(
                    \frac{\rho\,D\,\bar{v}}{\mu}
                    ,\varepsilon/D
                \right)
                = \phi\left(
                    \frac{\rho\,D}{\mu}
                    \frac{G_v}{\pi\,(D/2)^2}
                    ,\varepsilon/D
                \right)
                = &\\&
                = \phi\left(
                    \frac{\rho\,G_v\,4}{\mu\,\pi\,D}
                    ,\varepsilon/D
                \right)
                % = &\\&
                = \phi\left(
                    \frac{1531*4*10^{-3}*4}{0.065*\pi*2.5*10^{-2}}
                    ,\frac{0.05*10^{-3}}{2.5*10^{-3}}
                \right)
                = &\\&
                = \phi\left(
                    \frac{1.531*4^2}{6.5*\pi*2.5}*10^{4}
                    ,2*10^{-2}
                \right)
                % = &\\&
                \cong \phi\left(
                    \num{4.79835013658962e3}
                    ,2*10^{-2}
                \right)
                \cong 0.0069
    %         &
    %     \end{flalign*}
    % \end{answerBox}
    % \begin{answerBox}*{} % RS 
    %     \begin{flalign*}
    %         &
                &\\[1.5ex]&
                W_b
                \cong 25*1531*\num{9.780327}*4*10^{-3}
                + &\\&
                + \frac{
                    4^3*0.0069*1531*(30+2*40*2.5*10^{-2})*(4*10^{-3})^2
                }{
                    \pi^2*(2.5*10^{-2})^5
                }
                = &\\&
                =
                2.5*1.531*\num{9.780327}*40
                + \frac{
                    4^5*6.9*1.531*(3+2*4*2.5*10^{-2})
                }{
                    \pi^2*2.5^5
                }10^{5}
                \cong &\\&
                \cong
                \num{3.592985433843477507709e6}
            &
        \end{flalign*} 
    \end{answerBox}

\end{questionBox}

\end{document}
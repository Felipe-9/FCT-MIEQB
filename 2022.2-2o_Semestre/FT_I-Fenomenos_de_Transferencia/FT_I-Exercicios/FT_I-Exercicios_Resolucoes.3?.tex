% !TEX root = ./FT_I-Exercicios_Resolucoes.3.tex
\providecommand\mainfilename{"./FT_I-Exercicios_Resolucoes.tex"}
\providecommand \subfilename{}
\renewcommand   \subfilename{"./FT_I-Exercicios_Resolucoes.3.tex"}
\documentclass[\mainfilename]{subfiles}

% \graphicspath{{\subfix{../images/}}}
% \tikzset{external/force remake=true} % - remake all

\begin{document}

\mymakesubfile{3}
[FT\,I]
{Exercícios}
{Exercícios}

\setcounter{part}{2}


\begin{questionBox}1{
    Qual é a tensão tangencial que se deve aplicar a uma placa plana móvel que se encontra separada 1\,\unit{\milli\metre} de outra placa plana fixa, para que ela se movimente a uma velocidade de 0.5\,\unit{\metre\per\second}, sabendo que entre as 2 existe água a 20\,\unit{\celsius}?\\\\
    Se a placa tiver 1\,\unit{\metre} de comprimento e 1.5\,\unit{\metre} de largura, qual o valor da força aplicada?
} % Q1
    
\end{questionBox}

\begin{sectionBox}1{Reologia}
    
    Tipos de Fluidos
    
    \begin{sectionBox}2{Fluidos Newtonianos}
        
        \begin{BM}
            \odv{v}{x} = cte
        \end{BM}

        Viscosidade constante com a viscosidade de corte
        
    \end{sectionBox}

    \begin{sectionBox}2{}

        \begin{BM}
            \odv{v}{x} = f(x) \lor f(t)
            \\
            \mu_a = \frac{\tau_y}{\lvert\odv{v}{x}\rvert}
        \end{BM}
        \begin{itemize}
            \item[\(\mu_a\):] Viscosidade aparente (ponto a ponto)
        \end{itemize}
        
        Viscosidade variada com a v de corte
        Divididos em
        \begin{itemize}
            \item Viscosidade varia com o tempo
            \item Viscosidade não varia com o tempo
        \end{itemize}
        
    \end{sectionBox}
    
\end{sectionBox}

\begin{sectionBox}2{Plastico de Bingham}
    
    \begin{BM}
        \tau_y = \tau_0 - k\odv{v}{x}
    \end{BM}

    Até ser aplicada a tensão de corte o fluido se comporta como sólido 
    exemplo: 
    \begin{itemize}
        \item pasta de dentes
        \item Geleia
        \item Suspensões de argila em agua
    \end{itemize}
    
\end{sectionBox}

\begin{sectionBox}2{Fluidos pseudo plásticos}
    
    % \begin{BM}
    %     \tau_y = \tau_c-\mu\odv{v}{x}^n:n>1
    % \end{BM}

    Diminuem viscosidade aplicada tensão
    \paragraph*{Exemplo}
    \begin{itemize}
        \item Ketschup
    \end{itemize}
    
\end{sectionBox}

\begin{sectionBox}2{Fluidos Dilatantes}
    
    \begin{BM}
        \tau_y = -k\odv{v}{x}^n
    \end{BM}

    Almentam a resistencia quão maior a tensão aplicada em um curto periodo de tempo

    \paragraph*{Exemplo}
    \begin{itemize}
        \item Suspensões amido
        \item de silicato
        \item Areia da práia
        \item Areia movedissa
    \end{itemize}
    
\end{sectionBox}

\begin{questionBox}1{}
    
    Considere duas placas planas paralelas que estão separadas entre si de 5.1\,\unit{\centi\metre}. Uma delas movimenta-se a 5.1\,\unit{\centi\metre\per\second} e a outra, no sentido oposto a 17.8\,\unit{\centi\metre\per\second}. A viscosidade (\(\mu\)) do fluído entre elas é constante e vale 363\,\unit{lb\,ft^{-1}\,\hour^{-1}}.

    \begin{itemize}
        \item \(\mu = 363\,\unit{lb\,ft^{-1}\,\hour^{-1}}\)
        \item \(\tau_c = 0.4792\,\unit{\kilo\gram\,\metre^{-1}\,\second^{-2}}\)
    \end{itemize}
    
\end{questionBox}

\begin{questionBox}2{}
    
    Calcular a tensão de corte (\(\tau\)) em cada placa.
    
    
\end{questionBox}

\begin{questionBox}2{}
    
    Calcular a velocidade do fluído em intervalos de 1.3\,\unit{\centi\metre} duma placa à outra.

    \sisetup{
        scientific-notation = fixed,
        round-precision     = 3 
    }

    \begin{flalign*}
        &
            \tau\int_{x_1}^{x}\odif{x}
            =\mu\int_{v_1}^{v}\odif{v}
            \implies
            V 
            = V_1 - \frac{\tau}{\mu}(x-x_1)
            = 5.1 - \frac{6.74}{1.5}(x-0)
            = &\\&
            = \begin{cases}
                    5.1-\num{4.493333333333333}*1.3 
                    = -\num{0.741333333333333}
                \\  5.1-\num{4.493333333333333}*2.6
                    = -\num{6.582666666666667}
                \\  5.1-\num{4.493333333333333}*3.9
                    = -\num{12.424}
            \end{cases}
        &
    \end{flalign*}
    
\end{questionBox}

\begin{questionBox}2{}
    
    Determinar a tensão de corte e os perfis de velocidade se o fluído não fôr newtoniano, mas sim um plástico de Bingham com:

    \begin{flalign*}
        &
            \tau_y = \tau_c-\mu\odv{v}{x}
            \implies &\\&
            \implies
            (\tau_y-\tau_c)\int_{x_1}^{x_2}\odif{x}
            (\tau_y-\tau_c)\adif{x}\vert_{x_1}^{x_2}
            = -\mu\int_{V_2}^{V_1}\odif{V}
            = -\mu\adif{V}\vert_{v_1}^{v_2}
            \implies &\\&
            \implies
            \tau_y
            = -\mu\frac{v_2-v_1}{x_2-x_1}+\tau_c
            = \tau_a+\tau_c
            \cong (6.74+\num{4.7929})\unit{\gram\,\centi\metre^{-1}\,\second^{-2}}
            \cong &\\&
            \cong \num{11.5329}\,\unit{\gram\,\centi\metre^{-1}\,\second^{-2}}
            % \implies   &\\&
            \implies
            V 
            = V_1-\frac{\tau-\tau_c}{\mu}(x-x_1)
            \cong 
            (5.1-6.74\,x)\unit{\gram\,\centi\metre^{-1}\,\second^{-2}}
        &
    \end{flalign*}
    
\end{questionBox}

\begin{questionBox}1{}
    
    Um óleo flui laminarmente num tubo com diâmetro interno de 1.27\,\unit{\centi\metre} e um caudal de 4.55e-4\,\unit{\metre^3\per\second}. Sendo \(\mu=300\,\unit{\centi P}\) e a densidade de 959.8\,\unit{\kilo\gram/m^3}, calcular:

    \begin{flalign*}
        &
            \mu 
            = 300\,\unit{\centi P}
            % = 300\,\unit{\centi P}
        &
    \end{flalign*}
    
\end{questionBox}

\begin{questionBox}2{}
    
    A queda de pressão por metro de comprimento do tubo.
    
    \begin{flalign*}
        &
            \adv{P}{y}
            = \bar{V}
            \frac{32\,\mu}{D_1^2}
            &\\&
            \land
            \bar{V}
            = \frac{\rho_v}{S}
            = \frac{4.55\E-10}{\pi\,r^2}
            = \frac{4.55\E-10}{\pi\,(1.27/2)^2}
            \cong
            \num{3.593637760523929e-10}
            \implies &\\&
            \implies
            \bar{V}
            = \frac{32*300\E-2}{(1.27)^2}
            \frac{4.55\E-10}{\pi\,(1.27/2)^2}
            % = \frac{32*300}{(1.27)^2}
            % \frac{4.55}{\pi\,(1.27/2)^2}
            % \E-12
            \cong
            \num{21378.481235764562823e-12}
        &
    \end{flalign*}
    
\end{questionBox}


\begin{questionBox}2{}
    
    A tensão de corte nas paredes.

    \begin{flalign*}
        &
            \tau_{r=R}
            = \frac{\num{21378.481235764562823e-7}}{2}
            \frac{1.27}{2}
            \cong
            \qty{67945}{\gram\,\metre^{-1}\,\second^{-2}}
        &
    \end{flalign*}
    
    
\end{questionBox}

\begin{questionBox}2{}
    
    A velocidade no eixo do tubo.
    
    \begin{flalign*}
        &
            V_{r=0}
            = \frac{1}{4*0.3}
            *2.14\E5
            *\frac{1.27\E-2}{2}
            \cong
            7.18\,\unit{\metre\per\second}
        &
    \end{flalign*}
    
\end{questionBox}

\begin{questionBox}2{}
    
    A posição radial do ponto no qual a velocidade é igual à velocidade média.

    \begin{flalign*}
        &
            3.59
            = \frac{1}{4*0.3}
            2.14\E5
            \left(
                \frac{1.27\E-2}{2}
                - r^2
            \right)
            =0.0045
            \implies &\\&
            \implies
            r = 0.45\,\unit{\centi\metre}
        &
    \end{flalign*}
    
\end{questionBox}

\setcounter{question}{2}
\begin{questionBox}1{}
    
    Um óleo flui laminarmente num tubo com diâmetro interno de 1.27\,\unit{\centi\metre} e um caudal de \(4.55\E-4\)\,\unit{\metre^3\,\second^{-1}}. Sendo \(\mu = 300\,\unit{\centi P}\) e a densidade de 959.8\,\unit{\kilo\gram\,\metre^{-3}}, calcular:
    
\end{questionBox}

\begin{questionBox}2{}
    
    A queda de pressão por metro de comprimento do tubo.

    \begin{flalign*}
        &
            -\adv{P}{L}
            = \frac{32\,\mu}{D^2}\bar{v}
            = \frac{
                32*300\E-3
            }{
                (1.27\E-2)^2
            }\frac{4.55\E-4}{\pi(1.27\E-2/2)^2}
            = \frac{
                32*300\
            }{
                (1.27)^2
            }\frac{4.55}{\pi(1.27/2)^2}
            \E1
            % = \frac{
            %     32*300
            % }{
            %     (1.27)^2
            % }\frac{4.55}{\pi(1.27/2)^2}
            % \E-3
            \cong &\\&
            \cong
            \num{2.13785312279532624e5}
        &
    \end{flalign*}
    
\end{questionBox}

\begin{questionBox}2{}
    
    A tensão de corte nas paredes.

    \begin{flalign*}
        &
            \tau
            = -\adv{P}{L}\frac{r_1}{2}
            = \num{2.13785312279532624e5}
            \,\frac{(1.27\E-2/2)}{2}
            = \num{2.13785312279532624}
            \,\frac{(1.27/2)}{2}
            \E1
            \cong
            \num{6.787683664875161}
        &
    \end{flalign*}
    
\end{questionBox}

\begin{questionBox}2{}
    
    A velocidade no eixo do tubo.

    \begin{flalign*}
        &
            v
            = \frac{1}{4\,\mu}\left(-\adv{P}{L}\right)(r_1^2-r^2)
            \cong 
            \frac{1}{4*300\E-3}
            \,\num{2.13785312279532624e5}
            \,((1.27\E-2/2)^2-0)
            =\frac{1}{4*300}
            \,\num{2.13785312279532624}
            \,((1.27/2)^2-0)
            \E-2
            \cong
            \num{0.000718363187866}
        &
    \end{flalign*}
    
\end{questionBox}

\begin{questionBox}2{}
    
    A posição radial do ponto no qual a velocidade é igual à velocidade média.

    \begin{flalign*}
        &
            \frac{1}{4\,\mu}\left(-\adv{P}{L}\right)(r_1^2-R^2)
            = \bar{v}
            = \frac{G_v}{S}
            \implies &\\&
            \implies
            R 
            = -\sqrt{
                r_1^2
                -\frac{G_v}{S}
                4\,\mu
                \left(
                    -\adv{P}{L}
                \right)^{-1}
            }
            = &\\&
            = \sqrt{
                (1.27/2)^2
                -\frac{4.55\E-4}{\pi\,(1.27/2)^2}
                4*300\E-2
                * \num{213.785312279532624}^{-1}
            }
            = &\\&
            = \sqrt{
                (1.27/2)^2
                -\frac{4.55}{\pi\,(1.27/2)^2}
                4*300
                * \num{213.785312279532624}^{-1}
            }\E-4
            \cong
            \num{-4.445e-4}
        &
    \end{flalign*}
    
\end{questionBox}

\setcounter{part}{3}
\setcounter{question}{4}

\begin{questionBox}1{} % Q3-5
    
    Pretende-se bombear 4\,\unit{\deci\metre^3/\second} de uma solução de ácido sulfúrico através dum tubo de 2.5\,\unit{\centi\metre} de diâmetro, em chumbo, e a uma altura de 25\,\unit{\metre}. O tubo tem 30\,\unit{\metre} de comprimento e contém dois joelhos em ângulo recto. Calcular a potência da bomba teoricamente necessária. 
    \begin{itemize}
        \begin{multicols}{2}
            \item \(\rho_{\text{solução ácido}}=1531\,\unit{\kilo\gram/\metre^{3}}\);
            \item \(\mu_{\text{solução ácido}} = 0.065\,\unit{\kilo\gram\,\metre^{-1}\,\second^{-1}}\); 
        \end{multicols}
        \item \(\text{rugosidade chumbo} = 0.05\,\unit{\milli\metre}\).
    \end{itemize}

    \vspace{1ex}

    \begin{itemize}
        \begin{multicols}{3}
            \item \( G_v = 4\E-3\,\unit{\metre^3/s} \)
            \item \( D = 2.5\E-2\,\unit{\metre} \)
            \item \( \adif{z} = 25\,\unit{\metre} \)
            \item \( L = 30\,\unit{\metre} \)
            \item 2 Joelhos \angle{90}
        \end{multicols}
    \end{itemize}

    \begin{flalign*}
        &
            Pot_b
            = \adif{P}_b\,G_v
            = (
                h_{bomba}
                \,\rho
                \,g
            )\,G_v
            = \left(
                \begin{aligned}
                    &       \adif{P}
                    \\ + &  \frac{\adif{v^2}}{2\,g}
                    \\ + &  \frac{\adif{P}}{\rho\,g}
                    \\ + &  h_{at}
                \end{aligned}
            \right)
            \,\rho
            \,g
            G_v
            = \rho
            \,g
            \,G_v
            \,\left(
                \begin{aligned}
                    &       0
                    \\ + &  \frac{
                        \left(\frac{G}{\pi\,(D/2)^2}\right)^2
                    }{2\,g}
                    \\ + &  0
                    \\ + &  \frac{4\,\Phi\,v^2\,L}{g\,D}
                \end{aligned}
            \right)
            = &\\&
            = \rho
            \,g
            \,G_v
            \,\left(
                \begin{aligned}
                    &  \frac{
                        \left(\frac{G}{\pi\,(D/2)^2}\right)^2
                    }{2\,g}
                    \\ + &  \frac{
                        4
                        \,f\left(
                            Re,e/D
                        \right)
                        \,v^2\,L
                    }{g\,D}
                \end{aligned}
            \right)
            % = &\\&
            = \rho
            \,g
            \,G_v
            \,\left(
                \begin{aligned}
                    &  \frac{
                        \left(\frac{G}{\pi\,(D/2)^2}\right)^2
                    }{2\,g}
                    \\ + &  
                    \,f\left(
                        \frac{\rho\,v\,D}{\mu},\frac{\varepsilon}{D}
                    \right)
                    \,\frac{4\,v^2\,L}{g\,D}
                \end{aligned}
            \right)
            = &\\&
            = \rho
            \,G_v
            \,\left(\frac{G}{\pi\,(D/2)^2}\right)^2
            \,\left(
                \begin{aligned}
                    &  0.5
                    \\ + &
                    \,f\left(
                        \frac{
                            \rho
                            \,\left(\frac{G}{\pi\,(D/2)^2}\right)
                            \,D
                        }{\mu},
                        \frac{\varepsilon}{D}
                    \right)
                    \,\frac{
                        4
                        \,L
                    }{D}
                \end{aligned}
            \right)
            = &\\&
            = (1531)
            \,(4\E-2)
            \,\left(\frac{4\E-2}{\pi\,((2.5\E-2)/2)^2}\right)^2
            * &\\&
            *\left(
                \begin{aligned}
                    &  0.5
                    \\ + &
                    \,f\left(
                        \frac{
                            1531
                            \,\left(\frac{4\E-3}{\pi\,(2.5\E-2/2)^2}\right)
                            \,2.5\E-2
                        }{0.065},
                        \frac{0.05\E-3}{2.5\E-2}
                    \right)
                    \,\frac{
                        4
                        \,(30)
                    }{D}
                \end{aligned}
            \right)
            % v = 81.487330863050411
            % 406644.934984123351776
            % Re = 4.7983501365896222785\E4
            % \varepsilon/D = 2\E-3
            % f(Re,e/D) = 0.021
        &
    \end{flalign*}
    
\end{questionBox}

\end{document}
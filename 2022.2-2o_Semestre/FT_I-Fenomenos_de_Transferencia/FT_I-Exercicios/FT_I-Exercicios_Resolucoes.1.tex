% !TEX root = ./FT_I-Exercicios_Resolucoes.1.tex
\providecommand\mainfilename{"./FT_I-Exercicios_Resolucoes.tex"}
\providecommand \subfilename{}
\renewcommand   \subfilename{"./FT_I-Exercicios_Resolucoes.1.tex"}
\documentclass[\mainfilename]{subfiles}

% \graphicspath{{\subfix{../images/}}}
% \tikzset{external/force remake=true} % - remake all

\begin{document}

\mymakesubfile{1}
[FT I]
{Resolução}
{Resolução}

\begin{questionBox}1{} % Q1
    
    Indicar as dimensões em M, L, T, \(\theta\) das unidades de força, energia, pressão, potência e viscosidade (que são unidades derivadas) e definir estas unidades nos sistemas SI, CGS e Britânico.

    \begin{questionBox}3{Força} % Q1 (i)
        
        \begin{BM}
            \l[ F \r]
            = [m][a]
            = \unit{\gram\,\metre/\second^2}
            = \unit{M\,L\,T^{-2}}
        \end{BM}

        \begin{itemize}
            \begin{multicols}{3}
                \item[SI:]   \unit{\kilo\gram\,\metre\,\second^{-2}}
                \item[CGS:]  \unit{\gram\,\centi\metre\,\second^{-2}}
                \item[Brit:] \unit{lb\,ft\,\second^{-2}}
            \end{multicols}
        \end{itemize}
        
    \end{questionBox}

    \begin{questionBox}3{Energia} % Q1 (ii)
        
        \begin{BM}
            \l[ E \r]
            = [F][d]
            = \unit{M\,L\,T^{-2}}\,\unit{\metre}
            = \unit{M\,L\,T^{-2}\,L}
            = \unit{M\,L^2\,T^{-2}}
        \end{BM}

        \begin{itemize}
            \begin{multicols}{3}
                \item[SI:]   \unit{\kilo\gram\,\metre^2\,\second^{-2}}
                \item[CGS:]  \unit{\gram\,\centi\metre^2\,\second^{-2}}
                \item[Brit:] \unit{lb\,ft^2\,\second^{-2}}
            \end{multicols}
        \end{itemize}
        
    \end{questionBox}

    \begin{questionBox}3{Pressão}% Q1 (iii)
        
        \begin{BM}
            \l[ P \r]
            = [F]/[A]
            = \unit{M\,L\,T^{-2}}/\unit{\metre^2}
            = \unit{M\,L\,T^{-2}\,L^{-2}}
            = \unit{M\,T^{-2}\,L^{-1}}
        \end{BM}

        \begin{itemize}
            \begin{multicols}{3}
               \item[SI:]   \unit{\kilo\gram\,\second^{-2}\,\metre^{-1}}
               \item[CGS:]  \unit{\gram\,\second^{-2}\,\centi\metre^{-1}}
               \item[Brit:] \unit{lb\,\second^{-2}\,ft^{-1}}
            \end{multicols}
        \end{itemize}
        
    \end{questionBox}

    \begin{questionBox}3{Potência}% Q1 (iv)
        
        \begin{BM}
            \l[ P \r]
            = [E]/[s]
            = \unit{M\,L^2\,T^{-2}\,T^{-1}}
            = \unit{M\,L^2\,T^{-3}}
        \end{BM}

        \begin{itemize}
            \begin{multicols}{3}
               \item[SI:]   \unit{\kilo\gram\,\metre^2\,\second^{-3}}
               \item[CGS:]  \unit{\gram\,\centi\metre^2\,\second^{-3}}
               \item[Brit:] \unit{lb\,ft^2\,\second^{-3}}
            \end{multicols}
        \end{itemize}
        
    \end{questionBox}

    \begin{questionBox}3{Viscosidade} % Q1(v)
        
        \begin{BM}
            \left[
                -\mu
                \odv{v}{x}
            \right]
            = [\mu]\unit{(L/T)/L}
            = [F/A]
            = \unit{M\,L\,T^{-2}}
            / \unit{L^{2}}
            = \\
            = \unit{M\,L^{-1}\,T^{-2}}
            % \implies \\
            \implies
            [\mu]
            = \unit{
                \frac{
                    M\,L^{-1}\,T^{-2}
                }{
                    T^{-1}
                }
            } = \unit{
                M\,L^{-1}\,T^{-1}
            }
        \end{BM}

        \begin{itemize}
            \begin{multicols}{3}
               \item[SI:] \unit{\kilo\gram\,\metre^{-1}\,\second^{-1}}
               \item[CGS:] \unit{\gram\,\centi\metre^{-1}\,\second^{-1}}
               \item[Brit:] \unit{lb\,ft^{-1}\,\second^{-1}}
            \end{multicols}
        \end{itemize}
        
    \end{questionBox}

\end{questionBox}

\begin{questionBox}1{} % Q2
    
    Calcular, para cada grandeza derivada indicada em 1-1, os factores de conversão entre os três sistemas.

    \begin{questionBox}3{Força} % Q2 (i)
        
        \begin{flalign*}
            &
                \unit{M\,L\,T^{-2}}
                = 1\,\unit{\kilo\gram\,\metre\,\second^{-2}}
                = 10^{3+2}\,\unit{\gram\,\centi\metre\,\second^{-2}}
                = 10^5\,\unit{\gram\,\centi\metre\,\second^{-2}}
                = &\\&
                = 10^5\,\unit{\gram\,\centi\metre\,\second^{-2}}
                \frac{
                    \unit{lb}
                }{
                    453.59\,\unit{\gram}
                }
                \frac{
                    \unit{ft}
                }{
                    0.3048\,\unit{\metre}
                }
                % = &\\&
                = \frac{
                    10^{5-2}
                }{
                    453.59
                    * 0.3048
                }
                \,\unit{lb\,ft\,\second^{-2}}
                \cong \num{7.233051643583684}
                \,\unit{lb\,ft\,\second^{-2}}
            &
        \end{flalign*}
        
    \end{questionBox}

    \begin{questionBox}3{Energia} % Q2 (ii)
        
        \begin{flalign*}
            &
                \unit{M\,L^2\,T^{-2}}
                = \unit{\kilo\gram\,\metre^2\,\second^{-2}}
                = 10^{3+2^2}\unit{\gram\,\centi\metre^2\,\second^{-2}}
                = 10^{7}\unit{\gram\,\centi\metre^2\,\second^{-2}}
                = &\\&
                = 10^{7}\unit{\gram\,\centi\metre^2\,\second^{-2}}
                \frac{
                    \unit{lb}
                }{
                    453.59\,\unit{\gram}
                }
                \left(
                    \frac{
                        \unit{ft}
                    }{
                        0.3048\,\unit{\metre}
                    }
                \right)^2
                = \frac{
                    10^{7-2^2}
                }{
                    453.59*0.3048^2
                }
                \,\unit{lb\,ft^2\,\second^{-2}}
                \cong &\\&
                \cong 
                \num{23.730484394959594}
                \,\unit{lb\,ft^2\,\second^{-2}}
            &
        \end{flalign*}
        
    \end{questionBox}

    \begin{questionBox}3{Pressão} % Q2 (iii)
        
        \begin{flalign*}
            &
                \unit{M\,T^{-2}\,L^{-1}}
                = 1\,\unit{\kilo\gram\,\second^{-2}\,\metre^{-1}}
                = 10^{3-2}\,\unit{\gram\,\second^{-2}\,\centi\metre^{-1}}
                = 10\,\unit{\gram\,\second^{-2}\,\centi\metre^{-1}}
                = &\\&
                = 10\,\unit{\gram\,\second^{-2}\,\centi\metre^{-1}}
                \frac{
                    \unit{lb}
                }{
                    453.59\,\unit{\gram}
                }
                \frac{
                    0.3048\,\unit{\metre}
                }{
                    \unit{ft}
                }
                = \frac{
                    10^{1+2}*0.3048
                }{
                    453.59
                }
                \,\unit{lb\,\second^{-2}\,ft^{-1}}
                \cong &\\&
                \cong
                \num{0.6719724861659208}
                \,\unit{lb\,\second^{-2}\,ft^{-1}}
            &
        \end{flalign*}
        
    \end{questionBox}
    
\end{questionBox}

\begin{questionBox}1{} % Q3
    
    Agrupe as variáveis dos problemas que se seguem na forma de grupos adimensionais, aplicando o teorema \(\pi\) de Buckingham:
    
\end{questionBox}

\begin{questionBox}2{} % Q3.1
    
    Diferença de pressão entre as duas extremidades dum tubo pelo qual esteja a passar um fluído:

    \begin{BM}
        \Delta P = f(\rho,\mu,v,D,L)
    \end{BM}

    \begin{BM}
        \l[\adif{P}\r] = \unit{M\,L^{-1}\,T^{-2}}
        \\[1.5ex]
        [\rho] = \unit{M\,L^{-3}}
        \qquad
        [\mu] = \unit{M\,L^{-1}\,T^{-1}}
        \qquad
        [v] = \unit{L\,T^{-1}}
        \\
        [D] = \unit{L}
        \qquad
        [L] = \unit{L}
    \end{BM}

    \begin{questionBox}3{\(\pi_1\)} % (i)
        
        \begin{flalign*}
            &
                \pi_1
                = \frac{\adif{P}}{
                    D^a
                    \,v^b
                    \,\rho^c
                }
                \land
                \frac{
                    [\adif{P}]
                }{
                    [D]^a
                    \,[v]^b
                    \,[\rho]^c
                }
                = \frac{
                    \unit{
                        M
                        \,L^{-1}
                        \,T^{-2}
                    }
                }{
                    (\unit{L})^a
                    \,(\unit{L\,T^{-2}})^b
                    \,(\unit{M\,L^{-3}})^c
                }
                = &\\&
                = \unit{
                    M^{1-c}
                    \,L^{-1-a-b+3\,c}
                    \,T^{-2+2\,b}
                }
                = 1
                \implies &\\&
                \implies
                \left\{
                    \begin{aligned}
                        c = & 1
                        \\
                        b = & 2
                        \\
                        a = & -1-b+3\,c= 0
                    \end{aligned}
                \right\}
                \land
                \pi_1 = \frac{\adif{P}}{v^2\,\rho}
            &
        \end{flalign*}
        
    \end{questionBox}

    \begin{questionBox}3{\(\pi_2\)} % (ii)

        \begin{flalign*}
            &
                \pi_2
                = \frac{\mu}{
                    \,D^a
                    \,v^b
                    \,\rho^d
                }
                \land
                = \frac{[\mu]}{
                    \,[D]^a
                    \,[v]^b
                    \,[\rho]^c
                }
                = \frac{
                    \unit{
                        M
                        \,L^{-1}
                        \,T^{-1}
                    }
                }{
                    \,(\unit{L})^a
                    \,(\unit{L\,T^{-1}})^b
                    \,(\unit{M\,L^{-3}})^c
                }
                = &\\&
                = \unit{
                    M^{1-c}
                    \,L^{-1-a-b+3\,c}
                    \,T^{-1+b}
                }
                \implies
                \left\{
                    \begin{aligned}
                        c= & 1
                        \\
                        b= & 1
                        \\
                        a = & -1-b+3\,c =1
                    \end{aligned}
                \right\}
                \land
                \pi_2 = \frac{\mu}{
                    D\,v\,\rho
                }
            &
        \end{flalign*}

        \paragraph{Nota:} Para ter a formula em função de uma variável espeçifica não ha incluímos no grupo das variáveis de recurso
        
    \end{questionBox}

    \begin{questionBox}3{\(pi_3\)} % (ii)
        
        \begin{flalign*}
            &
                \pi_3
                = \frac{
                    L
                }{
                    D^a
                    \,v^b
                    \,\rho^d
                }
                = \frac{
                    L
                }{
                    D
                }
            &
        \end{flalign*}
        
    \end{questionBox}

\end{questionBox}

\begin{questionBox}2{} % Q3.2
    
    Força actuante sobre uma esfera no seio dum fluído em movimento
    relativamente a ela:

    \begin{BM}
        F = f(\rho,\mu,v_r,D)
    \end{BM}

    \begin{flalign*}
        &
            \left\{
                \begin{aligned}
                    \l[ F \r] & = \unit{M\,L\,T^{-2}}
                    \\[2ex]
                    \l[ \rho \r]  &= \unit{M/L^3}
                    &\l[ \mu \r]  &= \unit{M\,L^{-1}\,T^{-1}}
                    \\\l[ v_r \r] &= \unit{L/T}
                    &\l[ D \r]    &= \unit{L}
                \end{aligned}
            \right\}
            \implies &\\&
            \implies 
            \{D,v_r,\rho\}
        &
    \end{flalign*}

    \begin{itemize}
        \item 5 Numero de variáveis
        \item 3 Numero de grandezas fund presentes
        \item \(5-3=2\) grupos adimensionais
    \end{itemize}

    \begin{questionBox}{}
        
        \begin{flalign*}
            &
                \lvert F \rvert
                = {
                    F
                    \,[ \rho ]^a
                    \,[ v_r ]^b
                    \,[ D ]^c
                }
                = {
                    \lvert F \rvert
                    (\unit{
                        M
                        \, L
                        \, T^{-2}
                    })
                    \,(\unit{M/L^3})^a
                    \,(\unit{L/T})^b
                    \,(\unit{L})^c
                }
                = {
                    \lvert F \rvert
                    \unit{
                        M^{1+a}
                        \, L^{1-3\,a+b+c}
                        \, T^{-2-b}
                    }
                }
                \implies &\\&
                \implies
                \left\{
                    \begin{aligned}
                        1 + a = 0 \implies a = -1
                        \\ -2-b = 0 \implies b = -2
                        \\ {
                            1-3\,a + b + c 
                            = 1 -3\,(-1) + (-2) + c 
                            = 0
                            \implies
                            c = -2
                        }
                    \end{aligned}
                \right\}
                \implies &\\&
                \implies
                    \lvert F \rvert
                    = F / \rho\,v_r^2\,D^2
            &
        \end{flalign*}
        
    \end{questionBox}
    
\end{questionBox}

\begin{questionBox}2{} % Q3.3
    
    Potência necessária para accionar um ventilador:

    \begin{BM}
        P = f(\rho,\mu,N,D,Q)
    \end{BM}

    \begin{flalign*}
        &
            \left\{
                \begin{aligned}
                    \left[ P \right] &= {
                        \unit{\joule\per\second}
                        = \unit{\kilo\gram\metre^2/\second^3}
                        = \unit{M\,L^2/T^3}
                    }
                    \\[1.5ex]
                    \left[ \rho \right] &= \unit{M/L^3}
                    &
                    \left[ \mu \right] &= \unit{M/L\,T}
                    \\
                    \left[ N \right] &= \unit{T^{-1}}
                    &
                    \left[ D \right] &= \unit{L}
                    \\
                    \left[ Q \right] &= \unit{M\,L^2/T^2}
                \end{aligned}
            \right\}
        &
    \end{flalign*}

    \begin{itemize}
        \item 6 Variávies
        \item 3 Fundamentais
        \item \(6-3=3\) Adimensionais
    \end{itemize}

    \begin{questionBox}3{} % Q3.3 (i)
        
        \begin{flalign*}
            &
                \lvert P \rvert
                = {
                    P
                    \, [\rho]^a
                    \, [N]^b
                    \, [D]^c
                }
                = {
                    \lvert P \rvert
                    \, \unit{M\,L^2/T^3}
                    \, (\unit{M/L^3})^a
                    \, (\unit{T^{-1}})^b
                    \, (\unit{L})^c
                }
                = &\\&
                = {
                    \lvert P \rvert
                    \, (\unit{
                        M^{1+a}
                        \,L^{2-3\,a+c}
                        \,T^{-3-b}
                    })
                }
                \implies &\\&
                \implies
                \left\{
                    \begin{aligned}
                        1+a = 0 \implies a = -1
                        \\ -3-b=0\implies b = -3
                        \\ {
                            2-3\,a+c
                            = 2-3\,(-1)+c
                            = 0
                        }
                        \implies c = -5
                    \end{aligned}
                \right\}
                \implies &\\&
                \implies
                \lvert P \rvert
                = P/\rho\,N^{3}\,D^{5}
            &
        \end{flalign*}
        
    \end{questionBox}
    
\end{questionBox}

\begin{questionBox}2{} % Q3.4
    
    Força actuante sobre um corpo flutuante num líquido em movimento:

    \begin{BM}
        F = f(
            \rho,
            \mu,
            g,
            L,
            V_r
        )
    \end{BM}
    
    \begin{BM}
        \l[ F \r]
        = \unit{M\,L\,T^{-2}}
        \\[1.5ex]
        [\rho] = \unit{M\,L^{-3}}
        \qquad
        [\mu] = \unit{M\,L^{-1}\,T^{-2}}
        \qquad
        [g] = \unit{L\,T^{-2}}
        \\
        [L] = \unit{L}
        \qquad
        [v_r] = \unit{L\,T^{-1}}
        \\[1.5ex]
        \{
            \rho,L,v_r
        \}
    \end{BM}

    \begin{questionBox}3{\(\pi_1\)} % Q3.4 (i)
        
        \begin{flalign*}
            &
                \pi_1
                = \frac{
                    \mu
                }{
                    \rho^a\,L^b\,v_r^c
                }
                \land
                \frac{
                    [\mu]
                }{
                    [\rho]^a
                    \,[L]^b
                    \,[v_r]^c
                }
                = \frac{
                    \unit{
                        M^1
                        \,L^{-1}
                        \,T^{-2}
                    }
                }{
                    (\unit{M\,L^{-3}})^a
                    \,(\unit{L})^b
                    \,(\unit{L\,T^{-1}})^c
                }
                = &\\&
                = \unit{
                    M^{1-a}
                    \,L^{-1+3\,a-b-c}
                    \,T^{-2+c}
                }
                = 1
                % \implies &\\&
                \implies
                \left\{
                    \begin{aligned}
                        a = & 1
                        \\
                        c = & 2
                        \\
                        b = & -1+3\,a-c = 0
                    \end{aligned}
                \right\}    
                \land
                \pi_1 
                = \frac{
                    \mu
                }{
                    \rho\,v_r^2
                }
            &
        \end{flalign*}
        
    \end{questionBox}

    \begin{questionBox}3{\(\pi_2\)} % Q3.4 (ii)
        
        \begin{flalign*}
            &
                \pi_2
                = \frac{
                    g
                }{
                    \rho^a\,L^b\,v_r^c
                }
                \land
                \frac{
                    [g]
                }{
                    [\rho]^a
                    \,[L]^b
                    \,[v_r]^c
                }
                = \frac{
                    \unit{
                        L^{1}
                        \,T^{-2}
                    }
                }{
                    (\unit{M\,L^{-3}})^a
                    \,(\unit{L})^b
                    \,(\unit{L\,T^{-1}})^c
                }
                = &\\&
                = \unit{
                    m^{0-3\,a}
                    L^{1+3\,a-b-c}
                    \,T^{-2+c}
                }
                = 1
                \implies
                \left\{
                    \begin{aligned}
                        a = & 0
                        \\
                        c = & 2
                        \\
                        b = & 1+3\,a-c = -1
                    \end{aligned}
                \right\}
                \land
                \pi_2 = \frac{g}{L^{-1}\,v_r^{2}}
            &
        \end{flalign*}
        
    \end{questionBox}

\end{questionBox}



\end{document}
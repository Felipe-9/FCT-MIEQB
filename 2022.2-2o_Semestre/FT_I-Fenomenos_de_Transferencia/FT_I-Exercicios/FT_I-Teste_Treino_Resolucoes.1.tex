% !TEX root = ./FT_I-Teste_Treino_Resolucoes.1.tex
% !TEX root = ./FT_I-Exercicios_Resolucoes.tex
\providecommand\mainfilename{"./FT_I-Exercicios_Resolucoes.tex"}
\providecommand \subfilename{}
\renewcommand   \subfilename{"./FT_I-Teste_Treino_Resolucoes.1.tex.tex"}
\documentclass[\mainfilename]{subfiles}

% \graphicspath{{\subfix{../images/}}}
% \tikzset{external/force remake=true} % - remake all

\renewcommand\thequestion{Questão \arabic{question})}

\begin{document}

\mymakesubfile{1}
[FT\,I]
{Teste Treino: Resolução}
{Teste Treino}

\begin{questionBox}1{} % Q1
    
    Marque a alternativa em que são citadas apenas grandezas derivadas.

    \begin{enumerate}[label=\alph{enumi})]
        \item Força, velocidade, aceleração e distância; 
        \item Energia, aceleração e tempo;
        \item potência, velocidade, e trabalho;
        \item Energia, massa, potência e tempo.
        \item Energia, distância e força
    \end{enumerate}

    \paragraph*{RS:} c)
    
\end{questionBox}

\begin{questionBox}1{} % Q2
    
    Marque a alternativa em que são citadas apenas grandezas fundamentais.

    \begin{enumerate}[label=\alph{enumi})]
        \item Tempo, distância, força e energia;
        \item Temperatura, velocidade e comprimento; 
        \item Distância, massa e velocidade;
        \item Massa, força e tempo;
        \item Massa, distância e temperatura
    \end{enumerate}

    \paragraph*{RS:} e)
    
\end{questionBox}

\begin{questionBox}1{} % Q3
    
    Quais as unidades fundamentais de tensão?

    \begin{enumerate}[label=\alph{enumi})]
        \begin{multicols}{5}
            \item \unit{M\,L\,T}
            \item \unit{M\,L^2\,T}
            \item \unit{M\,L^2\,T^2 }
            \item \unit{M\,L^{-1}\,T^{-2} }
            \item \unit{M\,L^2\,T^{-3}}
        \end{multicols}
    \end{enumerate}

    \begin{flalign*}
        &
            [T] 
            = \unit{\newton/\metre^2}
            = \unit{\gram\,\metre\,\second^{-2}/\metre^2}
            = \unit{M\,L^{-1}\,T^{-2}}
        &
    \end{flalign*}

    \paragraph*{RS:} d)
    
\end{questionBox}

\setcounter{question}{4}

\begin{questionBox}1{} % Q5
    
    No Sistema Internacional, a pressão é dada em unidades de

    \begin{enumerate}[label=\alph{enumi})]
        \begin{multicols}{4}
            \item \unit{\kilo\gram\,\metre^{ 1}\,\second^{-2}}
            \item \unit{\kilo\gram\,\metre^{-2}\,\second^{ 3}}
            \item \unit{\kilo\gram\,\metre^{-1}\,\second^{-2}}
            \item \unit{\kilo\gram\,\metre^{-1}\,\second^{-3}}
        \end{multicols}
    \end{enumerate}

    \begin{flalign*}
        &
            [P]
            = \unit{\newton/\metre^2}
            = \unit{\kilo\gram\,\metre\,\second^{-2}/\metre^2}
            = \unit{\kilo\gram\,\metre^{-1}\,\second^{-2}}
        &
    \end{flalign*}

    \paragraph*{RS:} c)
    
\end{questionBox}

\begin{questionBox}1{} % Q6
    
    1 Newton representa:

    \begin{enumerate}[label=\alph{enumi})]
        \begin{multicols}{4}
            \item 1\,\unit{\kilo\gram\,\metre     \,\second^{-2}}
            \item 1\,\unit{\kilo\gram\,\metre^{-2}\,\second^{ 3}}
            \item 1\,\unit{\kilo\gram\,\metre^{-1}\,\second^{-2}}
            \item 1\,\unit{\kilo\gram\,\metre     \,\second^{-1}}
        \end{multicols}
    \end{enumerate}

    \paragraph*{RS:} a)
    
\end{questionBox}

\begin{questionBox}1{} % Q7
    
    A viscosidade de um líquido é de 1.3\,\unit{\centi P} (P = Poise, unidade de viscosidade no sistema c.g.s).Qual a viscosidade do líquido em unidades do sistema internacional.

    \begin{enumerate}[label=\alph{enumi})]
        \begin{multicols}{2}
            \item \(13 \E-3\,\unit{\kilo\gram\,\metre^{-1}\,\second^{-1}}\)
            \item \(1.3\E-5\,\unit{\kilo\gram\,\metre^{-1}\,\second^{-1}}\)
            \item \(1.3\E-3\,\unit{\kilo\gram\,\metre^{-1}\,\second^{-1}}\)
            \item \(1.3\E-2\,\unit{\kilo\gram\,\metre\,\second^{-1}}\)
        \end{multicols}
    \end{enumerate}

    \begin{flalign*}
        &
            1.3\,\unit{\centi P}
            = 1.3\E-2\,\unit{P}
            = 1.3\E-3\,\unit{\pascal}
            = 1.3\E-3\,\unit{\kilo\gram\,\metre^{-1}\,\second^{-2}}
        &
    \end{flalign*}

    \paragraph*{RS:} c)
    
\end{questionBox}

\begin{questionBox}1{} % Q8
    
    Na expressão \(A = F/B^2\), \textit{F} representa força e \textit{B} um comprimento. No sistema internacional de unidades (SI) a constante \textit{A} é expressa em:

    \begin{enumerate}[label=\alph{enumi})]
        \begin{multicols}{3}
            \item \unit{\kilo\gram\,\metre^{ 3}}
            \item adimensional
            \item \unit{\kilo\gram\,\metre^{-1}\second^{-2}}
            \item \unit{\kilo\gram\,\metre^{-1}\second^{-1}}
            \item \unit{\kilo\gram\,\metre^{-3}\second}
        \end{multicols}
    \end{enumerate}

    \begin{flalign*}
        &
            [A]
            = \frac{[F]}{[B]^2}
            = \unit{\frac{\newton}{\metre^2}}
            = \unit{\kilo\gram\,\metre^{-1}\,\second^{-2}}
        &
    \end{flalign*}

    \paragraph*{RS:} c)
    
\end{questionBox}

\begin{questionBox}1{} % Q9
    
    A queda de pressão devido ao alargamento súbito de um tubo pode ser calculada através da expressão abaixo.

    \begin{BM}
        (-\adif{P})^{\text{alargamento}}
        = \rho^b
        \frac{(v_1-v_2)^a}{2}
    \end{BM}

    onde \(\rho\) é a densidade do fluido, \(v_1\text{ e }v_2\) velocidadades do fluido antes e depois do alargamento do tubo.

    Calcule os valores de \textit{a} e \textit{b} para que a equação seja dimensionalmente correta.

    \begin{enumerate}[label=\alph{enumi})]
        \begin{multicols}{3}
            \item \(a=1,\ b=2\)
            \item \(a=1,\ b=1\)
            \item \(a=2,\ b=1\)
            \item \(a=3,\ b=1\)
            \item não sei
        \end{multicols}
    \end{enumerate}

    \begin{flalign*}
        &
            \left[
                (-\adif{P})^{\text{alargamento}}
            \right]
            = \unit{\kilo\gram\,\metre^{-1}\,\second^{-2}}
            = [\rho]^b
            \frac{[(v_1-v_2)]^a}{2}
            = \unit{
                (\kilo\gram/\metre^3)^b
                (\metre/\second)^a
            }
            = &\\&
            = \unit{
                \kilo\gram^b
                \,\metre^{-3\,b+a}
                \,\second^{-a}
            }
            &\\&
            \therefore
            \left\{
                \begin{aligned}
                    -a & = -2 \implies a=2
                    \\ b & = 1
                    \\ -3*b+a & = -3*1+2 = -1
                \end{aligned}
            \right\}
            = \left\{
                \begin{aligned}
                    {a=2} \\ {b = 1}
                \end{aligned}
            \right\}
        &
    \end{flalign*}

    \paragraph*{RS:} c)
    
\end{questionBox}

\begin{questionBox}1{} % Q10
    
    A velocidade, \textit{v}, de uma partícula esferica caindo lentamente num líquido muito viscoso pode ser expressa por \(v = f(d, \mu, \gamma, \gamma_s)\) onde \textit{d} é o diâmetro da partícula, \(\mu\) a viscosidade do líquido e \(\gamma\text{ e }\gamma_s\) sao as densidades do líquido e da partícula, respectivamente.


    Aplicando o teorema pi de Buckingham assinale qual o conjunto de recurso que deveria utilizar se pretender obter uma relação entre \textit{v} e as outras variáveis.

    \begin{enumerate}[label=\alph{enumi})]
        \begin{multicols}{4}
            \item \(d, \gamma, \gamma_s\)
            \item \(D, v, \mu\)
            \item \(d, \mu, \gamma\)
            \item \(d, \mu\)
        \end{multicols}
    \end{enumerate}

    \begin{flalign*}
        &
            [v] = \unit{L\,T^{-1}}
            &\\&
            {[d] = \unit{L}}
            \qquad {[\mu] = \unit{L^2\,T^{-1}}}
            \qquad {[\gamma] = [\gamma_s] = \unit{M\,L^{-3}}}
        &
    \end{flalign*}

    \paragraph*{RS:} c)
    
\end{questionBox}

\begin{questionBox}1{} % Q11
    
    Calcular o caudal de um fluido em \unit{\centi\metre^3\,\second^{-1}} se a velocidade média de passagem do fluido por um tubo com 1.27\,\unit{\centi\metre} de diâmetro for de 3.59\,\unit{\metre\,\second^{-1}}.

    \begin{enumerate}[label=\alph{enumi})]
        \begin{multicols}{3}
            \item 45.5\unit{\centi\metre^3\,\second^{-1}}
            \item 455 \unit{\centi\metre^3\,\second^{-1}}
            \item 8743\unit{\centi\metre^3\,\second^{-1}}
            \item 4.55\unit{\centi\metre^3}
            \item não sei
        \end{multicols}
    \end{enumerate}

    \begin{flalign*}
        &
            G
            = v\,S
            = v\,\pi\,r^2
            = 3.59\E2\,\pi\,(1.27/2)^2
            \,\unit{\centi\metre/\second}
            \cong 
            \qty{454.769962490004232}{\centi\metre/\second}
        &
    \end{flalign*}

    \paragraph*{RS} b)
    
\end{questionBox}

\begin{questionBox}1{} % Q12
    
    A velocidade média de um fluido através de uma tubo com 10\,\unit{\metre} de comprimento e 1.27\,\unit{\centi\metre} de diametro é 3.59\,\unit{\metre/\second}. A queda de pressão através do tubo é de \(21.36\E5\,\unit{\newton/\metre^2}\). Usando a equação de Hagen-Poiseiulle calcule a viscosidade do fluido assumindo um fluxo laminar?

    \begin{BM}
        \bar{v}
        = \frac{D^2}{32\,\mu}
        \frac{(-\adif{P})}{L}
    \end{BM}

    \vspace{2ex}

    \begin{enumerate}[label=\alph{enumi})]
        \begin{multicols}{2}
            \item 0.3  \,\unit{\kilo\gram\,\metre^{-1}\,\second^{-1}}
            \item 1.2  \,\unit{\kilo\gram\,\metre^{-1}\,\second^{-1}}
            \item 0.025\,\unit{\kilo\gram\,\metre^{-1}\,\second^{-1}}
            \item 0.3  \,\unit{\kilo\gram\,\metre^{-2}\,\second^{-2}}
        \end{multicols}
    \end{enumerate}

    \begin{flalign*}
        &
            \mu
            = \frac{
                D^2(-\adif{P})
            }{
                32\,\bar{v}\,L
            }
            = \frac{
                (1.27\E-2)^2*21.36\E5
            }{
                32*3.59*10
            }
            = \frac{
                (1.27)^2*21.36
            }{
                32*3.59*10
            }\E1
            % \cong &\\&
            \cong
            \num{2.99891573816156e-1}
        &
    \end{flalign*}

    \paragraph*{RS:} a)
    
\end{questionBox}

\begin{questionBox}1{} % Q13
    
    Considere duas placas planas paralelas (1 e 2), com um fluido entre elas, que estão separadas entre si de 1\,\unit{\milli\metre} (vêr figura). A placa inferior movimenta-se segundo \textit{y} à velocidade de \(A\,\unit{\metre\,\second^{-1}}\). A tensão de corte exercida sobre as placas é de 0.5\,\unit{\kilo\gram\,\metre^{-1}\,\second^{-2}} e a viscosidade do fluido entre as placas é \(1\E-3\,\unit{\kilo\gram\,\metre^{-1}\,\second^{-1}}\)

    \begin{center}
        \includegraphics[
            width = .8\textwidth
        ]{"Screenshot 2022-10-09 at 19.37.35.png"}
    \end{center}

    Calcule o valor de \textit{A}. Assinale a opção correcta:

    \begin{enumerate}[label=\alph{enumi})]
        \begin{multicols}{4}
            \item 0.5  \,\unit{\metre\,\second^{-1}}
            \item 5    \,\unit{\metre\,\second^{-1}}
            \item 50   \,\unit{\metre\,\second^{-1}}
            \item 0.05 \,\unit{\metre\,\second^{-1}}
        \end{multicols}
    \end{enumerate}

    \begin{flalign*}
        &
            A
            = \lvert V_1 \rvert
            &\\&
            -\int_{V_1}^{V_2}\mu\,\odif{V}
            = -\mu\,\adif{V}\big\vert_{V_1}^{V_2}
            = -\mu\,(V_2-V_1)
            = \int_{x_1}^{x_2}\tau\odif{x}
            = \tau\adif{x}\big\vert_{x_1}^{x_2}
            = \tau(x_2-x_1)
            \implies &\\&
            \implies
            A
            = \frac{\tau}{\mu}(x_2-x_1)-V_2
            = \frac{0.5\E3}{1}(1\E-3-0) - 0
            = 0.5
        &
    \end{flalign*}
    
\end{questionBox}

\begin{questionBox}1{} % Q14
    
    Considere o escoamento laminar de um fluido através de um tubo estacionário de raio 0.635\,\unit{\centi\metre} e comprimento 8\,\unit{\metre},representado na figura abaixo.

    \begin{center}
        \includegraphics[width=0.8\textwidth]{"Screenshot 2022-10-09 at 22.16.44.png"}
    \end{center}

    O perfil de velocidade para este escoamento é dado pela seguinte expressão:

    \begin{BM}
        v_r
        = \frac{1}{4\,\mu}
        \left(
            -\adv{P}{Y}
        \right)
        (R_1^2-r^2)
    \end{BM}

    Em que \textit{P} é a pressão e \(\mu=4\,\unit{\gram\,\centi\metre^{-1}\,\second^{-1}}\) a viscosidade do fluido.\\

    Neste escoamento, a velocidade média do fluido é igual a 1/2 da sua velocidade máxima. Se a velocidade média do fluido for 3.59\,\unit{\metre/\second}, qual a queda de pressão no tubo?

    \begin{enumerate}[label=\alph{enumi})]
        \begin{multicols}{3}
            \item \(2.28\E5\,\unit{\pascal}\)
            \item \(2.28\E6\,\unit{\pascal}\)
            \item \(4.56\E6\,\unit{\pascal}\)
            \item \(1.14\E5\,\unit{\pascal}\)
            \item Não sei
        \end{multicols}
    \end{enumerate}

    \vspace{-1ex}

    \begin{flalign*}
        &
            -\adif{P}
            = \frac{
                v_r
                \,4\,\mu
                \,\adif{Y}
            }
            {
                R_1^2-r^2
            }
            = \frac{
                \max{v}
                \,4\,\mu
                \,\adif{Y}
            }
            {
                R_1^2
            }
            = \frac{
                2\,\bar{v}
                \,4\,\mu
                \,\adif{Y}
            }
            {
                R_1^2
            }
            = &\\&
            = \frac{
                2*3.59
                *4*4\E2
                *8
            }
            {
                (0.635\E-2)^2
            }
            % = &\\&
            = \frac{
                2*3.59
                *4*4\E-1
                *8
            }
            {
                (0.635\E-2)^2
            }
            = &\\&
            = \frac{
                2*3.59
                *4*4
                *8
            }
            {
                (0.635)^2
            }
            \E3
            \cong
            \num{2.279223758447516895e6}
        &
    \end{flalign*}

    \paragraph*{RS:} b)
    
\end{questionBox}

\end{document}
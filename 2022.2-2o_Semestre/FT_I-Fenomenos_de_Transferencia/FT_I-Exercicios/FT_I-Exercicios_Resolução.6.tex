% !TEX root = ./FT_I-Exercicios_Resolução.6.tex
\providecommand\mainfilename{"./FT_I-Exercicios_Resolução.tex"}
\providecommand \subfilename{}
\renewcommand   \subfilename{"./FT_I-Exercicios_Resolução.6.tex"}
\documentclass[\mainfilename]{subfiles}

\graphicspath{{\subfix{./.build/figures/FT_I-Exercicios_Resolução.6}}}
% \tikzset{external/force remake=true} % - remake all

\begin{document}

\mymakesubfile{6}
[FT\,I]
{Exercicios}
{Exercicios}

\setcounter{question}{1}

\begin{questionBox}1{ % Q2
    Circula água a 2\,\si{\metre/\second} por um tubo de 2.5\,\si{\metre} de comprimento e 25\,\si{\milli\metre} de diâmetro. Sabendo que o tubo está a 320\,\si{\kelvin} e que a água entra a 293\,\si{\kelvin} e sai a 295\,\si{\kelvin}, qual é o valor do coeficiente de transferência de calor.
} % Q2
    \paragraph*{Agua:}
    \begin{itemize}
        \begin{multicols}{2}
            \item \(C_P=4181\,\si{\joule\,\gram^{-1}\,\kelvin^{-1}}\)
            \item \(\rho=1000\,\si{\kilo\gram\,\metre^{-3}}\)
        \end{multicols}
    \end{itemize}


    \begin{flalign*}
        &
            \bar{h}
            = h_{int}
            = \frac{Q}{A_{cont,int}\,\adif{T}_{\ln}}
            =
            \frac{
                G_{Agua}
                \,C_{P,agua}
                \,\adif{T}_{agua}
            } {
                A_{cont,int}
                \,\left(
                    \frac{\adif{\adif{T}}}{\adif{\ln{\adif{T}}}}
                \right)
            }
            = &\\&
            =
            \frac{
                \left(
                    \rho\,v\,A_{int}
                \right)
                \,C_{P,agua}
                \,\adif{T}_{agua}
            } {
                A_{cont,int}
                \,\left(
                    \adif{T_2}-\adif{T}_1
                \right)
            }
            \ln\frac{\adif{T}_2}{\adif{T}_1}
            = &\\&
            =
            \frac{
                \left(
                    1000*2*\pi*(25*10^{-3})^2
                \right)
                *4181
                *(295-293)
            } {
                \pi*(25*10^{-3}/2)\,2.5
                \,\left(
                    (320-295)-(320-293)
                \right)
            }
            \ln\frac{320-295}{320-293}
            = &\\&
            =
            -2*2*4181*10
            \ln\frac{320-295}{320-293}
            \cong
            \num{1287.096451960610103e1}
            % 1608.7
        &
    \end{flalign*}

\end{questionBox}


\end{document}
% !TEX root = ./TEQB-Exercicios_Resolvidos.tex
\providecommand\mainfilename{"./TEQB-Exercicios_Resolvidos.tex"}
\providecommand \subfilename{}
\renewcommand   \subfilename{"./TEQB-Exercicios_Resolvidos.2.tex"}
\documentclass[\mainfilename]{subfiles}

% \tikzset{external/force remake=true} % - remake all

% \renewcommand\thequestion{Questão \arabic{part}\,--\,\arabic{question}}
% \renewcommand\thesubquestion{\thequestion\ \alph{subquestion})}

\begin{document}

\graphicspath{{\subfix{./.build/figures/TEQB-Exercicios_Resolvidos.1}}}

\mymakesubfile{2}
[TEQB]
{Resolução dos Exercícios}
{Exercicios}

\sisetup{round-precision=3}
\setcounter{question}{9}

\begin{questionBox}1m{} % Q10
    
    Suponha que submete 1\,\unit{\mole} de uma data substância A ao seguinte processo a temperatura constante (200\,\unit{\kelvin}): compressão de (1\to19.7)\unit{\bar} até um volume de 24.2\,\unit{\centi\metre^3\per\mole}, compressão de (19.7\to50)\unit{\bar}, Tome para a entalpia de vaporização a 19.7\,\unit{\bar} o valor \(\adif{H}_{vap} = 27.8\,\unit{\kilo\joule\per\mole}\), assuma \(\alpha_{P,liq} = 3.25\E-4\,\unit{\kelvin^{-1}}\), e calcule as variações de entropia, entalpia e energia de Gibbs associadas ao processo.

    \begin{questionBox}3{Entalpia Compressão} % Q10 (i)
        \begin{flalign*}
            &
                \adif{H} 
                = \int n\,C_P\,\odif{T}
                = 0
            &
        \end{flalign*}

    \end{questionBox}

    \begin{questionBox}3{Entropia Compressão} % Q10 (ii)
        
        \begin{flalign*}
            &
                \adif{S}
                = \int \frac{n\,C_P\,\odif{T}}{T}
                + n\,R\,\ln\frac{P_1}{P_2}
                = 1*\num{8.314462618}*\ln\frac{1}{3.7}
                \cong -\num{10.87808432088394}
            &
        \end{flalign*}
        
    \end{questionBox}

    \begin{questionBox}3{Gibbs Compressão} % Q10 (iii)
        
        \begin{flalign*}
            &
                \adif{G}
                = \int\pdv{G}{P}_T
                \odif{P}
                = \int V\,\odif{P}
                = \int n\,R\,T\frac{\odif{P}}{P}
                = n\,R\,T\,\ln\frac{P_2}{P_1}
                \cong &\\&
                \cong
                1
                * \num{8.314462618}
                * 200
                \ln\frac{3.7}{1}
                \cong
                \num{2175.616864176789428}
            &
        \end{flalign*}
        
    \end{questionBox}

    \begin{questionBox}3{Entalpia condensação} % Q10 (iv)
        
        \begin{flalign*}
            &
                \adif{H}_{cond}
                = \int{
                    \pdv{H}{P}_T
                    \odif{P}
                }
                = \int{
                    V(1-\alpha_P\,T)
                    \odif{P}
                }
                = &\\&
                = 24.2\E-6*(1-3.25\E-4*200)(50-3.7)\E-5
                \cong 
                -27800\,\unit{\joule}
            &
        \end{flalign*}
        
    \end{questionBox}

    \begin{questionBox}3{Entropia condensação} % Q10 (v)
        
        \begin{flalign*}
            &
                \adif{S}_{cond}
                = \frac{\adif{H}_{cond}}{T}
                = \frac{-27800}{200}
            &
        \end{flalign*}
        
    \end{questionBox}

    \begin{questionBox}3{Gibbs Condensação} % Q10 (vi)
        
        \begin{flalign*}
            &
                \adif{G}_{cond}
                = 0
            &
        \end{flalign*}
        
    \end{questionBox}
    
\end{questionBox}

\begin{questionBox}1{} % Q11
    
    Calcule \(\adif{U}, \adif{H},\adif{S}\text{ e }\adif{G}\) para as seguintes transformações
    
\end{questionBox}

\begin{questionBox}2m{} % Q11.1
    
    Passagem de 20\,\unit{\gram} de água do estado (sólido, 0\,\unit{\celsius}, 1\,\unit{\atm}), ao estado (gás, 110\,\unit{\celsius}, 1\,\unit{\atm})

    \begin{questionBox}3{\(\adif{H}\)} % Q11.1 (i)
        
        \begin{flalign*}
            &
                \adif{H}
                = \left(
                    \begin{aligned}
                              & \adif{H}_{(s\to l), 0\unit{\celsius}}
                        \\  + & \adif{H}_{(l), (0\to100)\unit{\celsius}}
                        \\  + & \adif{H}_{(l\to g), 100\unit{\celsius}}
                        \\  + & \adif{H}_{(g), (100\to110)\unit{\celsius}}
                    \end{aligned}
                \right)
                % = &\\&
                = \left(
                    \begin{aligned}
                              & n*\adif{H}_{fus,0\unit{\celsius},1\unit{\atm}}
                        \\  + & n\int C_{p,(l)}\odif{T}
                        \\  + & n*\adif{H}_{vap,100\unit{\celsius},1\unit{\atm}}
                        \\  + & n\int C_{p,(g)}\odif{T}
                    \end{aligned}
                \right)
                = &\\&
                = \frac{m}{M}
                \left(
                    \begin{aligned}
                              & \adif{H}_{fus,0\unit{\celsius},1\unit{\atm}}
                        \\  + & C_{p,(l)}\adif{T}
                        \\  + & \adif{H}_{vap,100\unit{\celsius},1\unit{\atm}}
                        \\  + & C_{p,(g)}\adif{T}
                    \end{aligned}
                \right)
                % = &\\&
                = \frac{20}{18}*\left(
                    \begin{aligned}
                              & 6.01\E3
                        \\  + & 75(373.15-273.15)
                        \\  + & 40.7\E3
                        \\  + & 36(383.15-373.15)
                    \end{aligned}
                \right)
                \cong &\\&
                \cong 
                \num{60633.333333333333333}
            &
        \end{flalign*}
        
    \end{questionBox}

    \begin{questionBox}3{\(\adif{S}\)} % Q11.1 (ii)
        
        \begin{flalign*}
            &
                \adif{S}
                = \left(
                    \begin{aligned}
                              & \adif{S}_{(s\to l), 0\unit{\celsius}}
                        \\  + & \adif{S}_{(l), (0\to100)\unit{\celsius}}
                        \\  + & \adif{S}_{(l\to g), 100\unit{\celsius}}
                        \\  + & \adif{S}_{(g), (100\to110)\unit{\celsius}}
                    \end{aligned}
                \right)
                = \left(
                    \begin{aligned}
                            &
                            n*\adif{H}_{fus}/T_{fus}
                            \\ + & 
                            \int{
                                \pdv{S}{T}_P\odif{T}
                            }
                            \\ + & 
                            n*\adif{H}_{vap}/T_{vap}
                            \\ + & 
                            \int{
                                \pdv{S}{T}_P\odif{T}
                            }
                    \end{aligned}
                \right)
                = &\\&
                = \left(
                    \begin{aligned}
                            &
                            n*\adif{H}_{fus}/T_{fus}
                            \\ + & 
                            n\int{
                                C_{p,(l)}\frac{\odif{T}}{T}
                            }
                            \\ + & 
                            n*\adif{H}_{vap}/T_{vap}
                            \\ + & 
                            n\int{
                                C_{p,(g)}\frac{\odif{T}}{T}
                            }
                    \end{aligned}
                \right)
                % = &\\&
                = \frac{m}{M}
                \left(
                    \begin{aligned}
                        & 
                        \adif{H}_{fus}/T_{fus}
                        \\  + & 
                        C_{p,l}\ln\frac{100+273.15}{0+273.15}
                        \\  + & 
                        \adif{H}_{vap}/T_{vap}
                        \\  + & 
                        C_{p,(g)}\ln\frac{110+273.15}{100+273.15}
                    \end{aligned}
                \right)
                = &\\&
                = \frac{20}{18}\left(
                    \begin{aligned}
                        & 
                        6.01\E3/(0+273.15)
                        \\  + & 
                        75\ln\frac{100+273.15}{0+273.15}
                        \\  + & 
                        40.7\E3/(100+273.15)
                        \\  + & 
                        36\ln\frac{110+273.15}{100+273.15}
                    \end{aligned}
                \right)
                % \cong &\\&
                \cong
                \num{172.692216120302258}
            &
        \end{flalign*}
        
    \end{questionBox}

    \begin{questionBox}3{\(\adif{U}\)} % Q11.1 (iii)
        
        \begin{flalign*}
            &
                \adif{U}
                = \adif{H}
                - \adif{P\,V}
                = \adif{H}
                - P\adif{V}
                = \adif{H}
                - P\left(
                    \frac{n\,R\,T_f}{P}
                    - n*\frac{M}{\rho}
                \right)
                = &\\&
                = \adif{H}
                - n\left(
                    R\,T_f
                    - \frac{P\,M}{\rho}
                \right)
                = \adif{H}
                - \frac{m}{M}\left(
                    R\,T_f
                    - \frac{P\,M}{\rho}
                \right)
                = &\\&
                = (\num{60633.333333333333333})
                - \frac{20}{18}
                \left(
                    \num{8.314462618}
                    *(110+273.15)
                    -\frac{(1.01\E5)(18)}{0.92\E6}
                \right)
                \cong
                \num{57095.877483188690821}
            &
        \end{flalign*}
        
    \end{questionBox}

    \begin{questionBox}3{\(\adif{G}\)} % Q11.1 (iv)
        
        \begin{flalign*}
            &
                \adif{G}
                = \adif{H}
                - \adif{(T\,S)}
                = \adif{H}
                + T_i\,S_i
                - T_f\,S_f
                = &\\&
                = \adif{H}
                + T_i\,\left(
                    \begin{aligned}
                        & n\,S_{(l),25\unit{\celsius}}
                        \\ + &
                        \adif{S}_{(l),(25\to0)\unit{\celsius}}
                        \\ + &
                        \adif{S}_{(l\to s),0\unit{\celsius}}
                    \end{aligned}
                \right)
                - T_f\,\left(
                    \begin{aligned}
                        & n\,S_{(l),25\unit{\celsius}}
                        \\ + &
                        \adif{S}_{(l),(25\to100)\unit{\celsius}}
                        \\ + &
                        \adif{S}_{(l\to g),100\unit{\celsius}}
                        \\ + &
                        \adif{S}_{(g),(100\to110)\unit{\celsius}}
                    \end{aligned}
                \right)
                = &\\&
                = \adif{H}
                + T_i\,\left(
                    \begin{aligned}
                        & n\,S_{(l),25\unit{\celsius}}
                        \\ + &
                        \int{n\,C_{p\,(l)}\,\odif{T}/T}
                        \\ + &
                        n\,(-\adif{H}_{fus})/T_{fus}
                    \end{aligned}
                \right)
                % - &\\&
                - T_f\,\left(
                    \begin{aligned}
                        & n\,S_{(l),25\unit{\celsius}}
                        \\ + &
                        \int{n\,C_{p\,(l)}\,\odif{T}/T}
                        \\ + &
                        n\,\adif{H}_{vap}/T_{vap}
                        \\ + &
                        \int{n\,C_{p\,(g)}\,\odif{T}/T}
                    \end{aligned}
                \right)
                = &\\&
                = \adif{H}
                + \frac{T_i\,m}{M}\,\left(
                    \begin{aligned}
                        & S_{(l),25\unit{\celsius}}
                        \\ + &
                        C_{p\,(l)}\ln\frac{0+273.15}{25+273.15}
                        \\ + &
                        -\adif{H}_{fus}/T_{fus}
                    \end{aligned}
                \right)
                - &\\&
                - \frac{T_f\,m}{M}\,\left(
                    \begin{aligned}
                        & S_{(l),25\unit{\celsius}}
                        \\ + &
                        C_{p\,(l)}\ln\frac{100+273.15}{25+273.15}
                        \\ + &
                        \adif{H}_{vap}/T_{vap}
                        \\ + &
                        C_{p\,(g)}\ln\frac{110+273.15}{100+273.15}
                    \end{aligned}
                \right)
            &
        \end{flalign*}

    \end{questionBox}

    \begin{questionBox}{} % Q11.1 (iv)

        \begin{flalign*}
            &
                = (\num{60633.333333333333333})
                + \frac{(0+273.15)\,(20)}{(18)}\,\left(
                    \begin{aligned}
                        & 69.95
                        \\ + &
                        75\ln\frac{0+273.15}{25+273.15}
                        \\ + &
                        (-6.01\E3)/(0+273.15)
                    \end{aligned}
                \right)
                - &\\&
                - \frac{
                    (110+273.15)\,(20)
                }{(18)}
                \,\left(
                    \begin{aligned}
                        & 69.95
                        \\ + &
                        75\ln\frac{100+273.15}{25+273.15}
                        \\ + &
                        40.7\E3/(100+273.15)
                        \\ + &
                        36\ln\frac{110+273.15}{100+273.15}
                    \end{aligned}
                \right)
                % \cong
                % \num{60633.333333333333333}
                % +\num{12558.607155415047099}
                % +\num{-83783.095575914550613}
                \cong &\\&
                \cong
                \num{-10591.155087166170181}
            &
        \end{flalign*}
        
    \end{questionBox}

    
\end{questionBox}

\begin{questionBox}2m{} % Q11.2
    
    Passagem de 20\,\unit{\gram} de água do estado (gás, 100\,\unit{\celsius}, 0.5\,\unit{\bar}), ao estado (liquido, 100\,\unit{\celsius}, 50\,\unit{\bar}).

    \begin{questionBox}3{\(\adif{H}\)} % Q11.2 (i)
        
        \begin{flalign*}
            &
                \adif{H}_{
                    20\unit{\gram\of{\ch{H2O}}},
                    100\unit{\celsius},
                    (g\to l),
                    (0.5\to50)\unit{\bar}
                }
                = \left\{
                    \begin{aligned}
                        & \adif{H}_{(g),(0.5\to1)\unit{\bar}}
                        \\ +
                        & \adif{H}_{(g\to l),1\unit{\bar}}
                        \\ +
                        & \adif{H}_{(l),(1\to50)\unit{\bar}}
                    \end{aligned}
                \right\}
                = &\\&
                = \left\{
                    \begin{aligned}
                        & 0 \text{ (gás pft a T cnt.)}
                        \\ +
                        & n\,(-\adif{H}_{vap})
                        \\ +
                        & \int{\pdv{H}{P}_T\,\odif{P}}
                    \end{aligned}
                \right\}
                % = &\\&
                = \left\{
                    \begin{aligned}
                        - & n\,\adif{H}_{vap}
                        \\ 
                        + & \int{
                            \left(
                                vol
                                -T\,\pdv{vol}{T}_P
                            \right)
                            \,\odif{P}
                        }
                    \end{aligned}
                \right\}
                = &\\&
                = \left\{
                    \begin{aligned}
                        - & n\,\adif{H}_{vap}
                        \\
                        + & \int{
                            vol\,
                            \left(
                                1-\alpha_P\,T
                            \right)
                            \,\odif{P}
                        }
                        \\
                        & \text{(assumimos liq pouco sensível a pressão)}
                    \end{aligned}
                \right\}
                = &\\&
                = \left\{
                    \begin{aligned}
                        - & n\,\adif{H}_{vap}
                        \\ +
                        & \left(
                            \frac{n\,M}{\rho}
                        \right)
                        \,\left(
                            1-\alpha_P\,T
                        \right)
                        \,\adif{P}
                    \end{aligned}
                \right\}
                % = &\\&
                = n\left\{
                    \begin{aligned}
                        - & \adif{H}_{vap}
                        \\ +
                        & \left(
                            \frac{M}{\rho}
                        \right)
                        \,\left(
                            1-\alpha_P\,T
                        \right)
                        \,\adif{P}
                    \end{aligned}
                \right\}
                \dots
            &
        \end{flalign*}
        
    \end{questionBox}
    
\end{questionBox}

\setcounter{question}{14}

\begin{questionBox}1{ % Q15
    Calcula a pressão de vapor da água a 120\,\unit{\celsius} (\(\adif{H}_{vap}(100\unit{\celsius},1\,\unit{\atm}) = 40.7\,\unit{\kilo\joule\per\mole}\))
} % Q15
    
    \begin{flalign*}
        &
            \odv{P}{T}
            = T^{-1}\adv{H_m}{V_m}
            = \frac{\adif{H_m}}{T\,(V_{m,g}-V_{m\,l})}
            \cong \frac{\adif{H_m}}{T\,V_{m,g}}
            \cong \frac{\adif{H_m}}{T\,\left(
                R\,T/P
            \right)}
            =
            \frac{P\,\adif{H_m}}{R\,T^2}
            \implies &\\&
            \implies
            \int{\frac{\odif{P}}{P}}
            = \adif{\ln P}
            = \ln P_g/P_l
            = &\\&
            = \int{
                \frac{\adif{H_m}}{R\,T^2}
                \,\odif{T}
            }
            = \frac{\adif{H_m}}{R}
            \int{
                \frac{\odif{T}}{T^2}
            }
            % = &\\& 
            = \frac{P\,\adif{H_m}}{R}
            \adif{(-T^{-1})}
            \implies &\\&
            \implies
            P_g 
            \cong P_l\,\exp\left(
                \frac{\adif{H_m}}{R}
                \left(
                    T_{l}^{-1}
                    -T_{g}^{-1}
                \right)
            \right)
            = &\\&
            = 1.01*10^5\exp\left(
                \frac{40.7*10^3}{\num{8.314462618}}
                \left(
                    (100+273.15)^{-1}
                    -(120+273.15)^{-1}
                \right)
            \right)
            \cong &\\&
            \cong 
            \qty{1.968540460362481e5}{\pascal}
        &
    \end{flalign*}

\end{questionBox}

\begin{questionBox}1{ % Q16
    Calcule a temperatura de fusão da água a 500\,\unit{\bar}.
} % Q16

    \begin{itemize}
        \begin{multicols}{2}
            \item \(\rho_L = 1\,\unit{\gram/\centi\metre^3}\)
            \item \(\rho_S = 0.92\,\unit{\gram/\centi\metre^3}\)
            \item \(\adif{H}_{fus}(0\,\unit{\celsius},1\,\unit{\atm})= 6.01\,\unit{\kilo\joule\per\mole}\)
        \end{multicols}
    \end{itemize}

    \begin{flalign*}
        &
            \odv{P}{T}
            =T^{-1}\adv{H_{m}}{V_{m}}
            \implies
            \int\odif{P}
            =\adif{P}
            = \int\adv{H_m}{V_m}\frac{\odif{T}}{T}
            \cong &\\&
            \cong 
            \adv{H_m}{V_m}\int{\frac{\odif{T}}{T}}
            =\frac{\adif{H_m}}{\adif{18/\rho}}
            \adif{\ln(T)}
            =\frac{\adif{H_m}}{18(\rho_l^{-1}-\rho_s^{-1})}
            \ln\frac{T_s}{T_l}
            \implies &\\&
            \implies
            T_l 
            = T_g\,\exp\left(
                \frac{
                    \adif{P}
                    \,18(\rho_l^{-1}-\rho_s^{-1})
                }{
                    \adif{H}_m
                }
            \right)
            \cong &\\&
            \cong 273.15\,\exp\left(
                \frac{
                    (500-1.01)*10^{5}
                    \,18((10^{6})^{-1}-(0.92*10^{6})^{-1})
                }{
                    6.01*10^3
                }
            \right)
            = &\\&
            = 273.15\,\exp\left(
                \frac{
                    (500-1.01)
                    \,18(1-(0.92)^{-1})
                }{
                    6.01
                }*10^{-4}
            \right)
            \cong
            \qty{269.62325251655724}{\kelvin}
        &
    \end{flalign*}
    
\end{questionBox}

\begin{questionBox}1{ % Q17
    À temperatura de 0.01\,\unit{\celsius}, o gelo absorve 333.5\,\unit{\joule\per\gram} durante a fusão e a água liquida absorve 2490\,\unit{\joule\per\gram} na vaporização. A 0.01\,\unit{\celsius}, a pressão de vapor tanto do gelo como da água liquida é de 611\,\unit{\pascal}. Estime os valores da derivada \(\odv{P}{T}\) para a vaporização e para a sublimação da água a esta temperatura.
} % Q17

    \begin{questionBox}3{} % Q17 (i)
        
        \begin{flalign*}
            &
                \left(
                    \odv{P}{T}
                \right)_{vap}
                = \frac{\adif{H}_{vap,m}}{T\,\adif{V}_{vap}}
                = \frac{\adif{H}_{vap}\,M}{T\,\left(
                    V_{g} - V_{l}
                \right)}
                \cong \frac{\adif{H}_{vap}\,M}{T\,\left(
                    V_{g}
                \right)}
                = \frac{\adif{H}_{vap}\,M}{T\,\left(
                    \frac{R\,T}{P}
                \right)}
                = &\\&
                = \frac{\adif{H}_{vap}\,M\,P}{T^2\,R}
                = \frac{
                    2490*18*611
                }{
                    (0.01+273.15)^2
                    *\num{8.314462618}
                }
                \cong
                \num{44.141254254921421}
            &
        \end{flalign*}
        
    \end{questionBox}

    \begin{questionBox}3{} % Q17 (ii)
        
        \begin{flalign*}
            &
                \left(
                    \odv{P}{T}
                \right)_{sub}
                = \frac{
                    \adif{H}_{sub,m}
                }{
                    T\,\adif{V}_{sub}
                }
                = \frac{
                    \adif{H}_{fus,m}+\adif{H}_{vap,m}
                }{
                    T\,(V_g - V_s)
                }
                \cong \frac{
                    M(\adif{H}_{fus,g}+\adif{H}_{vap,g})
                }{
                    T\,V_g
                }
                = &\\&
                = \frac{
                    M(\adif{H}_{fus,g}+\adif{H}_{vap,g})
                }{
                    T\,\frac{R\,T}{P}
                }
                = \frac{
                    M(\adif{H}_{fus,g}+\adif{H}_{vap,g})\,P
                }{
                    R\,T^2
                }
                = &\\&
                = \frac{
                    18(2490+333.5)*611
                }{
                    (0.01+273.15)^2*\num{8.314462618}
                }
                \cong
                \num{50.053345939265314}
            &
        \end{flalign*}

    \end{questionBox}
    
\end{questionBox}

\begin{definitionBox}1{} % DEF 4
    
    Aproximações
    \begin{BM}
        \odv{P}{T}
        = \frac{\adif{H}}{T\,\adif{V}}
        \implies \\[1.5ex]
        \stackrel{\text{vap, sub, 2 approx}}{\implies}
        \odv{\ln P}{T}
        = \frac{\adif{H}}{R\,T^2}
        \stackrel{\text{3a Approx}}{\implies}
        \adif{\ln P}
        = -\frac{\adif{H}}{R}
        + \adif{T^{-1}}
        \\[1.5ex]
        \stackrel{\text{fus, 1 approx}}{\implies}
        \adif{P} = \adv{H}{V}\adif{\ln T}
    \end{BM}
    
\end{definitionBox}

\begin{questionBox}1{ % Q18
    As curvas de vaporização e sublimação de uma dada substãncia \textit{A} a temperaturas não muito afastadas do ponto triplo obedecem às equações I e II, respectivamente, com \(P/\unit{\pascal}\text{ e }T/\unit{\kelvin}\).
} % Q18
    
    \begin{BM}[align*]
         I:\quad & \ln P = 22.403-2045.5/T\\
        II:\quad & \ln P = 24.049 - 2308.2/T
    \end{BM}

    \begin{itemize}
        \begin{multicols}{3}
           \item \(C_{p,S} = 25\,\unit{\joule\,\kelvin^{-1}\,\mole^{-1}}\)
           \item \(V_{m,S} = 33.3\,\unit{\centi\metre^3/\mole}\)
           \item \(C_{p,L} = 56\,\unit{\joule\,\kelvin^{-1}\,\mole^{-1}}\)
           \item \(V_{m,L} = 35.2\,\unit{\centi\metre^3/\mole}\)
           \item \(C_{p,G} = 29\,\unit{\joule\,\kelvin^{-1}\,\mole^{-1}}\)
           \item \(\alpha_{p,L} = 4.25\E{-4}\,\unit{\kelvin}\)
        \end{multicols}
    \end{itemize}

    % \begin{questionBox}3{Temp do P triplo} % Q18 (i)
        
    %     \begin{flalign*}
    %         &
    %             \ln P
    %             = 22.403-2045.5/T_t
    %             = 24.049-2308.2/T_t
    %             \implies &\\&
    %             \implies
    %             T_t 
    %             = \frac{
    %                 2045.5 - 2308.2
    %             }{
    %                 22.403 - 24.049
    %             }
    %             \cong \num{159.59902794653706}
    %             \land &\\&
    %             \land
    %             P
    %             \cong \exp\left(
    %                 22.403-\frac{2045.5}{\num{159.59902794653706}}
    %             \right)
    %             \cong \num{14566.882369291699714}
    %         &
    %     \end{flalign*}
        
    % \end{questionBox}

    
\end{questionBox}

\begin{questionBox}2{ % Q18.1
    Calcule a pressão de fusão da substância \textit{A} a 166\,\unit{\kelvin}.
} % Q18.1
    
    \begin{flalign*}
        &
            % 
            % P fus
            % 
            \odv{P}{T}_{fus}
            = \frac{\adif{H}_{fus}}{T\,\adif{V}}
            = \frac{
                \adif{H}_{sub}-\adif{H}_{vap}
            }{
                T\,(\adif{V}_{l}-\adif{V}_{s})
            }
            = \frac{
                \adif{H}_{sub}-\adif{H}_{vap}
            }{
                T\,(\adif{V}_{l}-\adif{V}_{s})
            }
            \implies &\\&
            \implies 
            \int{\odif{P}}
            = \adif{P}
            = \int{
                \frac{
                    \adif{H}_{sub}-\adif{H}_{vap}
                }{
                    T\,(\adif{V}_{l}-\adif{V}_{s})
                }
                \odif{T}
            }
            = \frac{
                \adif{H}_{sub}-\adif{H}_{vap}
            }{
                \adif{V}_{l}-\adif{V}_{s}
            }
            \adif{\ln{T}}
            ; &\\[2ex]&
            % 
            % T p-triplo
            % 
            22.403-2045.5/T
            \cong 24.049-2308.2/T
            \implies
            T 
            \cong \frac{2308.2-2045.5}{24.049-22.403}
            % T = 159.59902794653706
            ; &\\[2ex]&
            % 
            % P p-triplo
            % 
            P
            \cong
            \exp{(22.403-2045.5/T)}
            \cong 
            \exp{\left(
                22.403
                -2045.5
                \frac{24.049-22.403}{2308.2-2045.5}
            \right)}
            % P = 14566.882369291699714
            ; &\\[2ex]&
            % 
            % H vap
            % 
            \odv{P}{T}_{vap}
            = \frac{\adif{H}_{vap}}{T\,\adif{V}_{vap}}
            = \frac{\adif{H_{vap}}}{T\,(V_{vap,g}-V_{vap\,l})}
            \cong \frac{\adif{H_{vap}}}{T\,V_{vap,g}}
            \cong \frac{\adif{H_{vap}}}{T\,\left(
                R\,T/P
            \right)}
            =
            \frac{P\,\adif{H_{vap}}}{R\,T^2}
            \implies &\\&
            \implies
            \frac{\odif{P}/P}{\odif{T}}
            = \frac{\odif{\ln{P}}}{\odif{T}}
            = \frac{\adif{H}_{vap}}{R\,T^2}
            = \odv{}{T}\left(
                22.403-2045.5/T
            \right)
            = 2045.5\,T^{-2}
            \implies &\\&
            \implies
            \adif{H}_{vap}
            \cong 
            2045.5\,R\,T^2/T^2
            =2045.5\,R
            ; &\\[2ex]&
            % 
            % H sub
            % 
            \odv{P}{T}_{sub}
            = \frac{\adif{H}_{sub}}{T\,\adif{V}_{sub}}
            = \frac{\adif{H_{sub}}}{T\,(V_{sub,g}-V_{sub\,s})}
            \cong \frac{\adif{H_{sub}}}{T\,V_{sub,g}}
            \cong \frac{\adif{H_{sub}}}{T\,\left(
                R\,T/P
            \right)}
            =
            \frac{P\,\adif{H_{sub}}}{R\,T^2}
            \implies &\\&
            \implies
            \frac{\odif{P}/P}{\odif{T}}
            = \frac{\odif{\ln{P}}}{\odif{T}}
            = \frac{\adif{H}_{sub}}{R\,T^2}
            = \odv{}{T}\left(
                24.049-2308.2/T
            \right)
            = 2308.2\,T^{-2}
            \implies &\\&
            \implies
            \adif{H}_{sub}
            \cong 
            2308.2\,R\,T^2/T^2
            =2308.2\,R
            ; &\\[3ex]&
            % 
            % Final
            % 
            \therefore
            P \cong
            \exp{\left(
                22.403
                -2045.5
                \frac{24.049-22.403}{2308.2-2045.5}
            \right)}
            % 14566.882369291699714
            + &\\&
            + \frac{
                2308.2\,R-2045.5\,R
            }{
                (35.2-33.3)*10^{-6}
            }
            \ln\cfrac{166}{
                \left(
                    \cfrac{2308.2-2045.5}{24.049-22.403}
                \right)
            }
            % 452.053089660095986
            % 45205308.966009598592236
            \cong
            \num{45219875.848378890292236}
            % R = \num{8.314462618}
        &
    \end{flalign*}
    
\end{questionBox}

\begin{questionBox}2m{ % Q18.2
    Calcule a variação de entropia associada à passagem de 1\,\unit{\mole} de \textit{A} de 0.1 a 250\,\unit{\bar}, a 165\,\unit{\kelvin}.
} % Q18.2
    
    \begin{flalign*}
        &
            \adif{S}=\sum_i \adif{S}_i
            ; &\\[2ex]&
            % 
            % P de vap a 165 K
            % 
            P_{vap}
            \cong \exp{\left(
                22.403
                -\frac{2045.5}{165}
            \right)}
            \cong
            \num{22159.693355788527735}
            \in[
                10*10^3,25000*10^3
            ]
            ; &\\[2ex]&
            % 
            % P de fus a 165 K
            % 
            P_{fus}
            = P_l
            + \frac{\adif{H}}{\adif{V}}
            \,\ln\frac{T}{T_0}
            \cong &\\&
            \cong 
            \num{14566.882369291699714}
            + \frac{
                (2308.2-2045.5)\,\num{8.314462618}
            }{
                (35.2-33.3)*10^{-6}
            }
            \,\ln{
                \cfrac{165}{
                    \left(
                        \cfrac{2308.2-2045.5}{24.049-22.403}
                    \right)
                }
            }
            \cong &\\&
            \cong
            \num{38273728.686279439933846}
            \notin[
                0.01*10^6,25*10^6
            ]
            ; &\\[3ex]&
        &
    \end{flalign*}

    % 
    % SI setup 1
    % 
    \sisetup{
        exponent-mode={fixed},
        fixed-exponent={0}
    }

    \begin{flalign*}
        &
            % 
            % Final
            % 
            \therefore
            \adif{S}_{(0.1\to250)\unit{\bar}}
            =\left(
                \begin{aligned}
                    &
                        \adif{S}_{(0.1\to\num{.22159693355788527735})\unit{\bar}}
                        & + \\ + &
                        \adif{S}_{\num{.22159693355788527735}\unit{\bar},(g\to l)}
                        & + \\ + &
                        \adif{S}_{(\num{.22159693355788527735}\to250)\unit{\bar}}
                    &
                \end{aligned}
            \right)
            = &\\&
            =\left(
                \begin{aligned}
                    &
                        \int_{165}^{165}{
                            n\,C_{p,g}\,\frac{\odif{T}}{T}
                        }
                        + n\,R\,\ln\frac{0.1}{\num{.22159693355788527735}}
                        & + \\ - &
                        {\adif{H}_{vap}}/{T_{vap}}
                        & + \\ + &
                        \int_{\num{.22159693355788527735}}^{250}{
                            \pdv{S}{P}_T\,\odif{P}
                        }
                    &
                \end{aligned}
            \right)
            = &\\&
            % 
            % SI setup 2
            % 
            \sisetup{
                exponent-mode={fixed},
                fixed-exponent={0}
            }
            =\left(
                \begin{aligned}
                    &
                        n\,R\,\ln\frac{0.1}{\num{.22159693355788527735}}
                        & + \\ - &
                        {2045.5\,R}/{T_{vap}}
                        & + \\ + &
                        \int_{\num{.22159693355788527735}}^{250}{
                            -\pdv{V}{T}_P\,\odif{P}
                        }
                    &
                \end{aligned}
            \right)
            = &\\&
            =\left(
                \begin{aligned}
                    &
                        n\,R\,\ln\frac{0.1}{\num{.22159693355788527735}}
                        & + \\ - &
                        {2045.5\,R}/{T_{vap}}
                        & + \\ - &
                        \left(
                            \alpha_P\,V
                        \right)
                        \,\int_{\num{.22159693355788527735}}^{250}{
                            \odif{P}
                        }
                    &
                \end{aligned}
            \right)
            = &\\&
            =\left(
                \begin{aligned}
                    &
                        n\,R\,\ln\frac{0.1}{\num{.22159693355788527735}}
                        & + \\ - &
                        {2045.5\,R}/{T_{vap}}
                        & + \\ - &
                        (\alpha_P\,V)
                        (250-\num{.22159693355788527735})
                        *10^5
                    &
                \end{aligned}
            \right)
            \cong &\\&
            \cong
            \left(
                \begin{aligned}
                    &
                        1*\num{8.314462618}\,\ln\frac{0.1}{\num{.22159693355788527735}}
                        % -6.615734187268481
                        & + \\ - &
                        {2045.5*\num{8.314462618}}/{165}
                        % -103.048945780666667
                        & + \\ - &
                        (4.25*10^{-4}*35.2*10^{-6})
                        (250-\num{.22159693355788527735})
                        *10^5
                        % -0.373668490987397
                    &
                \end{aligned}
            \right)
            % \cong &\\&
            \cong
            \num{-110.038348458922545}
        &
    \end{flalign*}
    
\end{questionBox}



\begin{questionBox}2{} % Q18.3
    
    Calcule a variação de entropia associada à passagem de 1\,\unit{\mole} de \textit{A} de 162 a 168\,\unit{\kelvin}, a 0.22\,\unit{\bar}.

    \begin{flalign*}
        &
            \adif{S}
            = \left(
                \begin{aligned}
                    &       
                        \adif{S}_{l,(162\to165)\unit{\kelvin}}
                    & + \\ + &  
                        \adif{S}_{(l\to g),165\unit{\kelvin}}
                    & + \\ + &  
                        \adif{S}_{g,(165\to168\unit{\kelvin})}
                    &
                \end{aligned}
            \right)
            = &\\&
            = \left(
                \begin{aligned}
                    &       
                        \int_{162}^{165}{
                            \pdv{S}{T}_P
                            \odif{T}
                        }
                    & + \\ + &  
                        \adif{H}_{vap}/165
                    & + \\ + &  
                        \int_{165}^{168}{
                            n\,C_{p,g}\,\frac{\odif{T}}{T}
                        }
                        + n\,R\,\ln\frac{P_1}{P_2}
                    &
                \end{aligned}
            \right)
            = &\\&
            = \left(
                \begin{aligned}
                    &       
                        \int_{162}^{165}{
                            (C_{p\,l}/T)
                            \odif{T}
                        }
                    & + \\ + &  
                        \adif{H}_{vap}/165
                    & + \\ + &  
                        \int_{165}^{168}{
                            n\,C_{p,g}\,\frac{\odif{T}}{T}
                        }
                    &
                \end{aligned}
            \right)
            = &\\&
            = \left(
                \begin{aligned}
                    &       
                        C_{p\,l}\,\ln\frac{165}{162}
                    & + \\ + &  
                        \adif{H}_{vap}/165
                    & + \\ + &  
                        n\,C_{p,g}\ln\frac{168}{165}
                    &
                \end{aligned}
            \right)
            \cong &\\&
            \cong 
            \left(
                \begin{aligned}
                    &       
                        56\ln\frac{165}{162}
                        % 1.027551765419006
                    & + \\ + &  
                        2045.5*\num{8.314462618}/165
                        % 103.074141121933333
                    & + \\ + &  
                        1*29\ln\frac{168}{165}
                        % 0.522536659577671
                    &
                \end{aligned}
            \right)
            % \cong &\\&
            \cong
            \num{104.62422954693001}
        &
    \end{flalign*}
    
\end{questionBox}

\begin{questionBox}1{ % Q19
    Com base na informação que segue para a substância X, calcule a entalpia de fusão de X a 81.64\,\unit{\kelvin}.
} % Q19

    \begin{BM}
        P_{fus}
        = (-142.94 + 0.019561\,T^{2.1075})\,10^6
    \end{BM}
    \begin{itemize}
        \begin{multicols}{2}
            \item \(V_{m,l} = 31.318\,\unit{\centi\metre^3\,\mole^{-1}}\)
            \item \(V_{m,s} = 29.413\,\unit{\centi\metre^3\,\mole^{-1}}\)
            \item \(T = 81.64\,\unit{\kelvin}\)
        \end{multicols}
    \end{itemize}

    \begin{flalign*}
        &
            \adif{H}_{fus}
            = T\,\adif{V}_{fus}
            \,\odv{P}{T}
            = T\,(V_{l}-V_{s})
            \,\odv{}{T}\left(
                (-142.94 + 0.019561\,T^{2.1075})\,10^6
            \right)
            = &\\&
            = T\,(V_{l}-V_{s})
            \left(
                0.019561*10^6*2.1075*T^{2.1075-1}
            \right)
            = &\\&
            = 81.64\,(31.318-29.413)*10^{-6}
            *0.019561*10^6*2.1075*(81.64)^{1.1075}
            \cong &\\&
            \cong
            \num{840.212581678643483}
        &
    \end{flalign*}

\end{questionBox}

\end{document}
% !TEX root = ./TEQB-Exercicios_Resolvidos.3.tex
\providecommand\mainfilename{"./TEQB-Exercicios_Resolvidos.tex"}
\providecommand \subfilename{}
\renewcommand   \subfilename{"./TEQB-Exercicios_Resolvidos.3.tex"}
\documentclass[\mainfilename]{subfiles}


% \tikzset{external/force remake=true} % - remake all

% \renewcommand\thequestion{Questão \arabic{part}\,--\,\arabic{question}}
% \renewcommand\thesubquestion{\thequestion\ \alph{subquestion})}

\begin{document}

\graphicspath{{./.build/figures/TEQB-Exercicios_Resolvidos.3}}

\mymakesubfile{3}
[TEQB]
{Exercicios}
{Exercicios}

\setcounter{question}{19}

\setcounter{question}{20}
\begin{questionBox}1{ % Q20
    Calcule a variação de energia de Gibbs associada à:
} % Q20

    \begin{itemize}
        \begin{multicols}{3}
            \item \(\rho_{l} = 1\,\unit{\gram\,\centi\metre^{-3}}\)
            \item \(\rho_{s} = 0.92\,\unit{\gram\,\centi\metre^{-3}}\)
            \item \(C_{p,s} = 35\,\unit{\joule\,\kelvin}\)
            \item \(C_{p,l} = 75\,\unit{\joule\,\kelvin}\)
            \item \(\adif{H}_{fus} = 6.01\,\unit{\kilo\joule\,\mole^{-1}}\)
        \end{multicols}
    \end{itemize}

\end{questionBox}

\begin{questionBox}2{ % Q20.1
    solidificação de água líquida sobrearrefecida a -5\,\unit{\celsius} e 1\,\unit{\atm}
} % Q20.1

    \begin{flalign*}
        &
            \adif{G}_{(-5\unit{\celsius},1\unit{\atm})}
            % &\\&
            = 
            \left(
                \begin{aligned}
                    &
                    \adif{G}_{0\unit{\celsius},1\unit{\atm},(l\to s)}
                    & + \\ + &
                    \adif{G}_{(0\to-5)\unit{\celsius},1\unit{\atm},s}
                    &
                \end{aligned}
            \right)
            &\\&
            = 
            \left(
                \begin{aligned}
                    &
                    0\ (\text{sobre a curva de fusão})
                    & + \\ + &
                    \int_{273.15}^{268.15}{
                        \pdv{\adif{G}}{T}_P
                        \odif{T}
                    }
                    &
                \end{aligned}
            \right)
            &\\&
            = 
            \int_{273.15}^{268.15}{
                \pdv{\adif{G}}{T}_P
                \odif{T}
            }
            ; &\\[2ex]&
            % 
            % \pdv{\adif{G}}{T}_P
            % 
            \odif{G}=-S\odif{T}+V\odif{P}
            \implies
            \left(
                \frac{\odif{G}}{\odif{T}}
            \right)_P
            =-S
            \implies
            \pdv{\adif{G}}{T}_P
            = -\adif{S}
            ; &\\[2ex]&
            \therefore
            \adif{G}_{(-5\unit{\celsius},1\unit{\atm})}
            = \int_{273.15}^{268.15}{
                -\adif{S}
                \odif{T}
            }
            \cong
            -\adif{S}
            \int_{273.15}^{268.15}{\odif{T}}
            = &\\&
            =
            -\frac{\adif{H}_{fus}}{T_{fus}}(273.15-268.15)
            =
            -\frac{6.01*10^3}{273.15}(273.15-268.15)
            \cong
            \num{110.012813472451041}
        &
    \end{flalign*}

\end{questionBox}

\begin{questionBox}2{ % Q20.2
    solidificação de água líquida a 0\,\unit{\celsius} e 10\,\unit{\atm}.
} % Q20.2

    \begin{flalign*}
        &
            \adif{G}_{(0\unit{\celsius},10\unit{\atm})}
            =\left(
                \begin{aligned}
                    &
                    \adif{G}_{(0\unit{\celsius},1\unit{\atm},l\to s)}
                    & + \\ + &
                    \adif{G}_{(0\unit{\celsius},(1\to10)\unit{\atm},s)}
                    &
                \end{aligned}
            \right)
            =\adif{G}_{(0\unit{\celsius},(1\to10)\unit{\atm},s)}
            = &\\&
            =\int_{1}^{10}{
                \pdv{\adif{G}}{P}_T
                \odif{P}
            }
            ; &\\[2ex]&
            \odif{G}
            = V\,\odif{P}
            -S\,\odif{T}
            \implies
            \left(
                \frac{\odif{G}}{\odif{P}}
            \right)_T
            = V
            \implies
            \left(
                \frac{\adif{G}}{\odif{P}}
            \right)_T
            = \adif{V}
            ; &\\[2ex]&
            \therefore
            \adif{G}_{(0\unit{\celsius},10\unit{\atm})}
            =\int_{1}^{10}{
                \adif{V}
                \odif{P}
            }
            \cong
            \adif{V}
            \int_{1}^{10}{
                \odif{P}
            }
            = &\\&
            =
            M_{\ch{H2O}}*\left(
                \rho_{s}^{-1}
                -\rho_{l}^{-1}
            \right)
            (10-1)*1.01*10^5
            = &\\&
            =
            18*\left(
                (0.92*10^{-6})^{-1}
                -(10^6)^{-1}
            \right)
            (10-1)*1.01*10^5
            = &\\&
            =
            18*\left(
                (0.92)^{-1}
                -1
            \right)
            (10-1)*1.01
            *10^{-1}
            \cong
            \num{1.422782608695652}
        &
    \end{flalign*}

\end{questionBox}

\setcounter{question}{21}
\begin{questionBox}1{ % Q22
    Representam-se na figura os volumes parciais molares da água e do etanol em função da composição das soluções que formam, a 20\,\unit{\celsius} e 1\,\unit{\bar}.
} % Q22

    \begin{center}
        \includegraphics[width=.6\textwidth]{Screenshot 2022-11-17 at 14.29.21-cutout.png}
    \end{center}
    
\end{questionBox}

\begin{questionBox}2{ % Q22.1
    É verdade que para preparar 2\,\unit{\deci\metre^3} de solução equimolar de água e etanol precisa de misturar 1\,\unit{\deci\metre^3} de cada espécie? E se não, que quantidades de cada componente são necessárias?
} % Q22.1
    
    \paragraph*{Equimolar:} \(x_{agua}=x_{etanol}=0.5\land n_{agua}=n_{etanol}\)
    
    \begin{flalign*}
        &
            V_{agua,*}
            = V_{agua,m,*}\,n_{agua}
            = V_{agua,m,*}\,(n_{t}/2)
            = \frac{V_{agua,m,*}}{2}
            \,\left(
                \frac{V_{sol}}{V_{sol,m}}
            \right)
            = &\\&
            = \frac{V_{agua,m,*}}{2}
            \,\left(
                \frac{V_{sol}}{0.5(V_{agua,m}+V_{etanol,m})}
            \right)
            % = &\\&
            = \frac{
                V_{agua,m,*}\,V_{sol}
            }{
                V_{agua,m}+V_{etanol,m}
            }
            \cong &\\&
            \cong 
            \frac{
                18.0*2*10^{3} 
            }{
                17.0+57.5
            }
            \cong
            \qty{483.221476510067114}{\centi\metre^3}
            ; &\\[2ex]&
            % 
            % V etanol
            % 
            V_{entaol,*}
            \cong
            \num{483.221476510067114}
            \frac{V_{entanol,m,*}}{V_{agua,m,*}}
            \cong \num{483.221476510067114}
            \frac{58.1}{18}
            \cong
            \qty[scientific-notation=fixed]{1559.731543624161074}{\centi\metre^3}
        &
    \end{flalign*}
    
\end{questionBox}

\begin{questionBox}2{ % Q22.2
    Calcule \(\adif{V}_{mist}\) para a solução equimolar.
} % Q22.2
    
    \begin{flalign*}
        &
            \adif{V}
            =
            V_{sol}
            - (
                V_{entaol,*}
                + V_{agua,*}
            )
            \cong 
            2*10^3
            - (
                \num{1559.731543624161074}
                + \num{483.221476510067114}
            )
            \cong
            \num{42.9530201342281}
        &
    \end{flalign*}

\end{questionBox}

\begin{questionBox}2{ % Q22.3
    Calcule o volume de água que teria que adicionar para, a partir da solução equimolar, preparar uma solução com composição \(x_{agua} = 0.90\), bem como o volume da solução obtida.
} % Q22.3
    
    \begin{flalign*}
        &
            \adif{V}_{agua,*}
            = V_{t,2} - V_{t,1}
            = V_{t,m,2}\,n_{t,2} - V_{t,1}
            = &\\&
            = \left(
                x_{agua}\,V_{agua,m}
                + (1-x_{agua})\,V_{etanol,m}
            \right)\left(
                n_{etanol,1}
                \frac{1}{1-x_{agua}}
            \right)
            - V_{t,1}
            = &\\&
            = 
            \frac{
                x_{agua}\,V_{agua,m}
                + (1-x_{agua})\,V_{etanol,m}
            }{1-x_{agua}}
            \left(
                \frac{n_{t,1}}{2}
            \right)
             - V_{t,1}
            = &\\&
            = 
            \frac{
                x_{agua}\,V_{agua,m}
                + (1-x_{agua})\,V_{etanol,m}
            }{2\,(1-x_{agua})}
            \,\left(
                \frac{V_{sol,1}}{
                    0.5(V_{agua,m,1}+V_{etanol,m,1})
                }
            \right)
            - V_{t,1}
            \cong &\\&
            \cong
            \frac{
                0.9*18.0
                + 0.1*53.0
            }{2*0.1}
            \,\left(
                \frac{2*10^3}{0.5(17.0+57.5)}
            \right)
            - 2*10^3
            \cong
            \qty{3771.812080536912752}{\centi\metre^3}
        &
    \end{flalign*}
    
\end{questionBox}

\begin{questionBox}1{
    Considere a seguinte experiência laboratorial. Precisaram-se 1600\,\unit{\gram} de um composto \textit{A} de massa molar 160\,\unit{\gram\,\mole^{-1}} que se adicionaram a 400\,\unit{\gram} de um composto \textit{B} de massa molar 40\,\unit{\gram\,\mole^{-1}}. Posteriormente, adicionaram-se 1.6\,\unit{\gram} de \textit{A} puro à totalidade da mistura obtida anteriormente, tendo-se detectado um aumento de volume de 1.2\,\unit{\centi\metre^{-3}} a 25\,\unit{\celsius}. Fez-se uma experiência semelhante, mas em que posteriormente se adicionou 4\,\unit{\gram} de \textit{B} à totalidade da mistura original, tendo-se detectado um aumento de volume de 13\,\unit{\centi\metre^{3}} a 20\,\unit{\celsius}.
} % Q23
    
    \begin{questionBox}3{ % Q23.1
        Calcule os volumes parciais molares de \textit{A} e de \textit{B} para a composição \(x= 0.5\).
    } % Q23.1
        \begin{flalign*}
            &
                V_{i,m}
                = \pdv{V}{n_i}_{T,P,n_j}
                % \land &\\&
                \land
                \frac{1600}{160}
                = \frac{400}{40}
                = 10
                \land
                \frac{1.6}{160}
                = 0.01 
                < \frac{4}{40}
                = 0.1
                \implies &\\[2ex]&
                \implies
                \begin{cases}
                    V_{A,m}
                    \cong \frac{1.2}{1.6/160}
                    &= 120\,\unit{\centi\metre^3/\mole}
                    \\
                    V_{B,m}
                    \cong \frac{13}{4/40}
                    &= 130\,\unit{\centi\metre^3/\mole}
                \end{cases}
            &
        \end{flalign*}
    \end{questionBox}

    \begin{questionBox}3{ % Q23.2
        Calcule a massa volúmica da solução de 1600\,\unit{\gram} de \textit{A} e 400\,\unit{\gram} de \textit{B} a 25\,\unit{\celsius}.
    } % Q23.2
        \begin{flalign*}
            &
                \rho_{sol,m}
                = \frac{m_{sol}}{V_{sol}}
                = \frac{m_{A}+m_{B}}{
                    n_A\,V_{A,m}
                    + n_B\,V_{B,m}
                }
                \cong \frac{1600+400}{
                    10*120 + 10*130
                }
                = 0.8\,\unit{\gram/\centi\metre^3}
            &
        \end{flalign*}
    \end{questionBox}
    
\end{questionBox}

\begin{questionBox}1m{ % Q24
    A figura representa o diagrama de equilíbrio líquido-vapor do sistema n-hexano (HEX) + etanol (ETA), a 55\,\unit{\celsius}. Esboce o diagrama que obteria se a solução de HEX e ETA fosse ideal.
} % Q24

    \begin{center}
        \includegraphics[width=.5\textwidth]{0.1.png}
    \end{center}

    \vspace{-5ex}

    \begin{flalign*}
        &
            P_{HEX,*} \cong 65 > P_{ETA,*} \cong 35
            \therefore \text{ HEX mais volátil}
            &\\[2ex]&
            P_{x=0.5}
            = 0.5(65+35) = 50
            \land y_{HEX} 
            = \frac{0.5*65}{50}
            = 0.65
            &\\[2ex]&
            x_{HEX,P=40}
            =\frac{P-P_{ETA,*}}{P_{HEX,*}-P_{ETA,*}}
            =\frac{40-35}{65-35}
            =1/6
            \land
            y_{HEX}
            = \frac{(1/6)*65}{40}
            \cong 
            \num[scientific-notation=fixed]
            {0.270833333333333}
        &
    \end{flalign*}

    \begin{center}
        \includegraphics[width=.6\textwidth]{0.2.png}
    \end{center}

\end{questionBox}

\begin{definitionBox}1{Log da prop molar de A} % DEF
    
    \begin{BM}
        \adif\ln{x_A} = \frac{\adif{H}}{R}\adif{(-T^{-1})}
        \\
        \ln{x_{A}}
        = \frac{-\adif{H}_{vap}}{R}
        \left(
            T_{vap,A}^{-1}
            - T_{vap,sol}^{-1}
        \right)
    \end{BM}
    
\end{definitionBox}

\begin{questionBox}1{ % Q25
    A pressão de vapor de uma dada substância A líquida obedece à equação \(\ln{P} = 20.221 - 5067.6/T\), com \([P]=\unit{\mmHg}\) e \([T]=\unit{\kelvin}\). Suponha que tem uma solução de um soluto não volátil no solvente A, e que a fracção molar do soluto é de 0.08. Calcule a temperatura de ebulição da solução à pressão de 1\,\unit{\atm}.
} % Q25
    
    \begin{itemize}
        \item \(1\,\unit{\atm} = 760\,\unit{\mmHg}\)
    \end{itemize}

    \begin{flalign*}
        &
            t_{vap,sol}
            = \left(
                t_{vap,A}^{-1}
                -\ln{x_{A}}
                \,\frac{R}{\adif{H}_{vap,A}}
            \right)^{-1}
            = &\\&
            = \left(
                \left(
                    \frac{5067.6}{20.221-\ln{P}}
                \right)^{-1}
                -(1-0.08)
                \,\frac{R}{\adif{H}_{vap,A}}
            \right)^{-1}
            = &\\&
            = \left(
                \frac{20.221-\ln{P}}{5067.6}
                -\frac{(1-0.08)\,R}{\adif{H}_{vap,A}}
            \right)^{-1}
            ; &\\[3ex]&
            % 
            % H vap A
            % 
            % \pdv{P}{T}_{vap,A}
            % = \frac{
            %     \adif{H}_{m}
            % }{
            %     T\,\adif{V}
            % }
            \odv{P}{T}_{vap}
            = \frac{\adif{H}_{vap}}{T\,\adif{V}_{vap}}
            = \frac{\adif{H_{vap}}}{T\,(V_{vap,g}-V_{vap\,l})}
            \cong \frac{\adif{H_{vap}}}{T\,V_{vap,g}}
            \cong \frac{\adif{H_{vap}}}{T\,\left(
                R\,T/P
            \right)}
            =
            \frac{P\,\adif{H_{vap}}}{R\,T^2}
            \implies &\\&
            \implies
            \frac{\odif{P}/P}{\odif{T}}
            = \frac{\odif{\ln{P}}}{\odif{T}}
            = \odv{}{T}\left(
                20.221-5067.6\,T^{-1}
            \right)
            = 5067.6\,T^{-2}
            = \frac{\adif{H_{vap}}}{R\,T^2}
            \implies &\\&
            \implies
            \adif{H}_{vap} = 5067.6\,R
            &\\[3ex]&
            % 
            % Final
            % 
            t_{vap,sol}
            = \left(
                \frac{20.221-\ln{P}}{5067.6}
                -\frac{\ln(1-0.08)\,R}{5067.6\,R}
            \right)^{-1}
            = &\\&
            = 
            \frac{5067.6}
            {20.221-\ln{(760*0.92)}}
            \cong
            \num{370.680753565886591}
        &
    \end{flalign*}

\end{questionBox}

\begin{questionBox}1{ % Q26
    Dissolvem-se 82.7\,\unit{\milli\gram} de um soluto pouco volátil em 100.0\,\unit{\centi\metre^3} de água. A 25\,\unit{\celsius}, a pressão osmótica desta solução é de 0.111\,\unit{\bar}. Calcule a massa molar do soluto. Calcule também \(\adif{T}_{fus}\) para esta solução, fazendo \(K_{fus,agua} = 1.86\,\unit{\kelvin\,\kilo\gram\,\mole^{-1}}\) e \(\rho_{agua} = 1\,\unit{\gram\,\centi\metre^{-3}}\). Comente o resultado.
} % Q26

    \begin{questionBox}3{ % Q26.1
    } % Q26.1
        \begin{flalign*}
            &
                M_i
                = \frac{m_{i}}{[i]}
                \cong m_{i}
                \left(
                    \frac{\Pi\,V_{agua}}{R\,T}
                \right)^{-1}
                =\frac{m_{i}\,R\,T}{\Pi\,V_{agua}}
                = &\\&
                =\frac{
                    82.7*10^{-6}
                    *\num{8.314462618}
                    *(273.15+25)
                }{0.111*10^5*100*10^{-6}}
                % = &\\&
                =\frac{
                    82.7
                    *\num{8.314462618}
                    *(273.15+25)
                }{0.111}*10^{-7}
                \cong &\\&
                \cong
                \num{1.846934651750802612613e-1}
            &
        \end{flalign*}
    \end{questionBox}

    \begin{questionBox}3{ % Q26.2
    } % Q26.2
        \begin{flalign*}
            &
                \adif{T}_{fus}
                = k_{fus}\,m_i
                = k_{fus}\,[i]/\rho_{agua}
                = \frac{k_{fus}}{\rho_{agua}}
                \,\frac{\Pi\,V_{agua}}{R\,T}
                = &\\&
                = \frac{
                    1.86*0.111*10^{5}*100*10^{-6}
                }{
                    \num{8.314462618}*(273.15+25)*1*10^{-3+6}
                }
                = \frac{
                    1.86*0.111
                }{
                    \num{8.314462618}*(273.15+25)
                }*10^{-2}
                \cong &\\&
                \cong
                \num{8.328502573395564e-7}
            &
        \end{flalign*}
    \end{questionBox}


\end{questionBox}

\setcounter{question}{28}
\begin{questionBox}1{ % Q29
    Com base na informação disponível para o sistema n-butanol + ciclohexano:
} % Q29
    
    \begin{center}
        % \includegraphics[width=.8\textwidth]{image.jpeg}
        \includegraphics[
            % width=.4\textwidth, 
            height=5cm,
        ]{Screenshot 2022-11-18 at 10.48.33.2-cutout2.png}
        \includegraphics[
            % width=.4\textwidth, 
            height=5cm,
        ]{Screenshot 2022-11-18 at 10.39.38.2-cutout.png}
    \end{center}

\end{questionBox}

\begin{questionBox}2{ % Q29 a)
    Calcule \(\gamma_{I,n-butanol}\) quando \(x_{n-butanol} = 0.9\), a 50\,\unit{\celsius}. Comente o resultado
} % Q29 a)



\end{questionBox}

\begin{questionBox}2{ % Q29 b)
    Calcule a pressão de vapor do ciclohexano a 50\,\unit{\celsius}.
} % Q29 b)

    \begin{flalign*}
        &
            \gamma_{He,II}
            = \frac{\gamma_{He,I}}{\gamma_{He,I,\infty}}
            ; &\\[2ex]&
        &
    \end{flalign*}

    \begin{table}[H]\centering
        \begin{tabular}{cc}
            
            \\\toprule
            
                \multicolumn{1}{c}{\(x_{He}\)}
                & \multicolumn{1}{c}{\(\gamma_{He,I}\)}
            
            \\\midrule
            
                0.10  & 5.3
            \\  0.05  & 6.1
            \\  0.025 & 7.7
            
            \\\bottomrule
            
        \end{tabular}
    \end{table}

\end{questionBox}

\begin{questionBox}2{ % Q29 c)
    Calcule \(\adif{G}_{mist,m}\) para a solução com \(x_{n-butano} = 0.90\), a 50\,\unit{\celsius}, e compare com o valor obtido para a solução ideal. Comente o resultado.
} % Q29 c)



\end{questionBox}

\begin{questionBox}2{ % Q29 d)
    Calcule a pressão de vapor da solução no ponto azeotrópico a 50\,\unit{\celsius}, atingido quando \(x_{ciclohexano}=0.90\).
} % Q29 d)
    

\end{questionBox}

\begin{questionBox}2{ % Q29 e)
    Calcule a entalpia de dissolução do ciclohexano líquido numa grande quantidade de \textit{n}-butanol.
} % Q29 e)

\end{questionBox}



\end{document}
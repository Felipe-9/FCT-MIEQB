% !TEX root = ./TEQB-Exercicios_Resolvidos.1.tex
\providecommand\mainfilename{"./TEQB-Exercicios_Resolvidos.tex"}
\providecommand \subfilename{}
\renewcommand   \subfilename{"./TEQB-Exercicios_Resolvidos.1.tex"}
\documentclass[\mainfilename]{subfiles}

% \tikzset{external/force remake=true} % - remake all

% \renewcommand\thequestion{Questão \arabic{part}\,--\,\arabic{question}}
\renewcommand\thesubquestion{\thequestion\ \alph{subquestion})}

\begin{document}

\graphicspath{{\subfix{./.build/figures/TEQB-Exercicios_Resolvidos.1}}}

\mymakesubfile{1}
[TEQB]
{Exercicios}
{Exercicios}

\sisetup{round-precision=3}


\begin{questionBox}1{} % Q1
    
    1\,\unit{\mole} de um gás perfeito, inicialmente a 25\,\unit{\celsius} e 1\,\unit{\bar}, sofre uma expansão. No estado final, \(T = 25\,\unit{\celsius}\) e \(P = 0.5\,\unit{\bar}\).

\end{questionBox}

\begin{questionBox}2m{} % Q1.1
    
    Calcule o trabalho de expansão posto em jogo quando o processo se dá seguindo dois percursos diferentes:
    

    \begin{questionBox}3{} % Q1.1 (i)

        Processo reversível a T constante

        \begin{flalign*}
            &
                w_{(1\to0.5)\unit{\bar}}
                = w_{(1\to0.5)\E5\unit{\pascal}}
                = -\int_{Vol_1}^{Vol_2}{
                    p_{ext}\,\odif{vol}
                }
                = -\int_{Vol_1}^{Vol_2}{
                    p\,\odif{vol}
                }
                = &\\&
                = -\int_{Vol_1}^{Vol_2}{
                    \left(
                        \frac{n\,R\,T}{vol}
                    \right)
                    \,\odif{vol}
                }
                % = &\\&
                = -n\,R\,T
                \int_{Vol_1}^{Vol_2}{
                    \frac{\odif{Vol}}{Vol}
                }
                = -n\,R\,T\,\ln\frac{Vol_2}{Vol_1}
                &\\&
                = -n\,R\,T\,\ln\frac{
                    n\,R\,T/p_2
                }{
                    n\,R\,T/p_1
                }
                = &\\&
                = -n\,R\,T\,\ln\frac{p_1}{p_2}
                \cong -(1)*(\num{8.314462618})*(25+273.15)*\ln\frac{1\E5}{0.5\E5}
                \cong -\num{1.718282075766483616e3}
            &
        \end{flalign*}

    \end{questionBox}

    \begin{questionBox}3{} % Q1.1 (ii)
        
        processo irreversível, mediante alívio súbito da pressão exterior para 0.5\,\unit{\bar}, seguida de expansão do gás contra essa pressão.

        \begin{flalign*}
            &
                w_{(1\to0.5)\unit{\bar}}
                = w_{(1\to0.5)\E5\unit{\pascal}}
                = - \int p_{ext}\odif{V}
                = - p_{ext}\int\odif{V}
                = - p_2\,\adif{V}
                = &\\&
                = - p_2\,(Vol_2-Vol_1)
                % = &\\&
                = - p_2
                \,\left(
                    \left(\frac{n\,R\,T}{p_2}\right)
                    -\left(\frac{n\,R\,T}{p_1}\right)
                \right)
                = &\\&
                = - n\,R\,T
                \,\left(
                    1
                    -\frac{p_2}{p_1}
                \right)
                % \cong &\\&
                \cong - (1)\,(\num{8.314462618})\,(25+273.15)
                \,\left(
                    1
                    -\frac{0.5\E5}{1\E5}
                \right)
                \cong &\\&
                \cong \num{-1.23947851477835e3}
            &
        \end{flalign*}
        
    \end{questionBox}
    
\end{questionBox}

\begin{questionBox}2{} % Q1.2
    
    Represente ambas as transformações num diagrama \(P\times V\).
    

    \begin{flalign*}
        &
            Vol_i
            = \frac{n\,R\,T}{p_i}
            = \frac{(1)\,(\num{8.314462618})\,(25+273.15)}{p_i}
            \cong 
            \frac{\num{2478.9570295567}}{p_i}
            \implies &\\&
            \implies
            \begin{cases}
                \begin{aligned}
                    Vol_1 &\cong \num{2478.9570295567}/1\E5 = \num{2.4789570295567e-2}
                    \\
                    Vol_2 &\cong \num{2478.9570295567}/0.5\E5 \cong \num{4.9579140591134e-2}
                \end{aligned}
            \end{cases}
        &
    \end{flalign*}

    \begin{center}
        \includegraphics[width=.8\textwidth]{./.build/figures/IMG_7940.png}
    \end{center}
    
\end{questionBox}

\begin{questionBox}2m{} % Q1.3

    Calcule \(\adif{U}\) e \textit{Q} para as alineas a.i e a.ii.
    
    \begin{questionBox}3{} % Q1.3 (i)
        
        \(\adif{U}\) de um gás perfeito depende apenas da temperatura 
        \begin{flalign*}
            &
                \adif{U}_{(i)}
                = \adif{U}_{(ii)}
                = 0
            &
        \end{flalign*}
        
    \end{questionBox}

    \begin{questionBox}3{} % Q1.3 (ii)
        
        \begin{flalign*}
            &
                Q = \adif{U}-W = -W
                \implies
                \begin{cases}
                    \begin{aligned}
                        Q_{(i)}  & = \num{1.718282075766483616e3}
                        \\
                        Q_{(ii)} & = \num{1.23947851477835e3}
                    \end{aligned}
                \end{cases}
            &
        \end{flalign*}
        
    \end{questionBox}
    
\end{questionBox}

\begin{definitionBox}1{} % D1
    
    \begin{BM}
        Q
        = \int_{T_1}^{T_2}{n\,C_{Vol}\,\adif{T}}
    \end{BM}

    \begin{itemize}
        \item Gás perfeito
        \item Independe da variação (ou não?) do volume
    \end{itemize}
    
\end{definitionBox}

\begin{questionBox}2m{} % Q1.4
    
    Deduza as expressões para \(\adif{U}\) e \(\adif{H}\) associados a cada um dos passos do percurso a.II

    \begin{questionBox}3{} % Q1.4 (i)
        
        \begin{flalign*}
            &
                \adif{U}_{
                    (1\to0.5)\unit{\bar},
                    Vol_1\to Vol_2
                } = 0 \text{ (Temperatura constante)}
                &\\[1.5ex]&
                \adif{U}_{(1\to0.5)\unit{\bar},Vol_1}
                = Q + W
                = Q_v - \int P_{ext}\,\odif{v}
                = Q_v - (0)
                = \int_{T_1}^{T_2}{
                    n\,C_v\,\odif{T}
                }
                = &\\&
                = n\,C_v\,\adif{T}\big\rvert_{T_1}^{T_2}
                = n\,C_v\,(T_2-T_1)
                = n\,C_v\,\left(
                    \frac{P_2\,V_1}{n\,R}
                    -T_1
                \right)
                = C_v\,\left(
                    \frac{P_2\,V_1}{R}
                    -n\,T_1
                \right)
                &\\[1.5ex]&
                \adif{U}_{0.5\unit{\bar},(Vol_1\to Vol_2)}
                = \adif{U}_{
                    (1\to0.5)\unit{\bar},
                    Vol_1\to Vol_2
                }
                - \adif{U}_{(1\to0.5)\unit{\bar},Vol_1}
                = &\\&
                = 0 
                - C_v\,\left(
                    \frac{P_2\,V_1}{R}
                    -n\,T_1
                \right)
            &
        \end{flalign*}
        
    \end{questionBox}

    \begin{questionBox}3{} % Q1.4 (ii)
        
        \begin{flalign*}
            &
                \adif{H}
                = \int_{T_1}^{T_2}{
                    n\,C_p\,\odif{T}
                }
                \implies 
                &\\[1.5ex]&
                \implies
                \adif{H}_{
                    (1\to0.5)\unit{\bar},
                    (Vol_1\to Vol_2)
                } = 0 \text{ (Temperatura constante)}
                &\\[1.5ex]&
                \implies
                \adif{H}_{(1\to 0.5)\unit{\bar}}
                = \int_{T_1}^{T_2}{
                    n\,C_p\,\odif{T}
                }
                = n\,C_p\,\adif{T}\big\rvert_{T_1}^{T_2}
                = n\,C_p\,\left(
                    \frac{P_2\,V_1}{n\,R} - T_1
                \right)
                = &\\&
                = C_p\,\left(
                    \frac{P_2\,V_1}{R} - n\,T_1
                \right)
                &\\[1.5ex]&
                \implies
                \adif{H}_{
                    0.5\,\unit{\bar},
                    (Vol_1\to Vol_2)
                }
                = \int_{T_1}^{T_2}{
                    n\,C_p\,\odif{T}
                }
                = C_p\,\left(
                    n\,T_1 - \frac{P_2\,V_1}{R}
                \right)
            &
        \end{flalign*}
        
    \end{questionBox}

    \paragraph{Nota:} Apenas para gases perfeitos (\(C_V \text{ e } C_p\) constantes)
    
\end{questionBox}


\begin{questionBox}1m{} % Q2
    
    Um \unit{\mole} de um gás perfeito, inicialmente à pressão de 8\,\unit{\bar} e à temperatura de 140\,\unit{\celsius}, é expandido adiabaticamente contra a atmosfera, até se estabelecer o equilíbrio de pressões. Tome \(C_v = 5/2\,R\) para o gás e calcule \(\Delta U\) e \(\Delta H\) para a tranformação.

    \paragraph*{Adiabático:} Sem troca de calor, Não ha variação de entropia, processo termodinâmico reversível

    \begin{questionBox}3{\(\adif{U}\)} % Q2 (i)
        
        \begin{flalign*}
            &
                \adif{U}_{
                    (8\to 1)\,\unit{\bar}
                }
                = &\\[1.5ex]&
                = \int_{T_1}^{T_2}{
                    n\,C_V\,\odif{T}
                }
                = n\,C_V\,\int_{T_1}^{T_2}{
                    \odif{T}
                }
                = n\,C_V\,\adif{T}\big\rvert_{T_1}^{T_2}
                = n\,C_V\,\left(
                    T_2-T_1
                \right)
                = &\\[1.5ex]&
                = Q+W
                = W
                = -\int_{Vol_1}^{Vol_2}{
                    P_{ext}\,\odif{Vol}
                }
                = -P_{ext}
                \,\int_{Vol_1}^{Vol_2}{
                    \odif{Vol}
                }
                = &\\&
                = -P_2\,\adif{Vol}\big\rvert_{Vol_1}^{Vol_2}
                = -P_2\,\left(
                    Vol_2-Vol_1
                \right)
                = P_2\,\left(
                    Vol_1-Vol_2
                \right)
                = &\\&
                = P_2\,\left(
                    \frac{n\,R\,T_1}{P_1}-\frac{n\,R\,T_2}{P_2}
                \right)
                % = &\\&
                = n\,R\,\left(
                    T_1\frac{P_2}{P_1}-T_2
                \right)
                \implies &\\[1.5ex]&
                \implies
                \adif{U}_{(8\to1)\unit{\bar}}
                = n\,R\,\left(
                    T_1\frac{P_2}{P_1}
                    -\left(
                        \frac{
                            \adif{U}_{(8\to1)\unit{\bar}}
                        }{n\,(R\,5/2)}
                        +T_1
                    \right)
                \right)
                \implies &\\&
                \implies
                \adif{U}_{(8\to1)\unit{\bar}}
                = \frac{5}{7}
                R\,n\,T_1\left(
                    \frac{P_2}{P_1}
                    -1
                \right)
                \cong &\\&
                \cong 
                \frac{5}{7}
                (\num{8.314462618})\,(1)\,(140+273.15)
                \,\left(
                    \frac{1.01\E5}{8\E5}
                    -1
                \right)
                \cong
                \num{-2143.883072507199375}
            &
        \end{flalign*}
        
    \end{questionBox}

    \begin{questionBox}3{\(\adif{H}\)} % Q2 (ii)
        
        \begin{flalign*}
            &
                \adif{H}_{(8\to 1)\,\unit{\bar}}
                = \int_{T_1}^{T_2}{
                    n\,C_p\,\odif{T}
                }
                = n\,C_p\,\int_{T_1}^{T_2}{
                    \odif{T}
                }
                = n\,(C_V+R)\,\adif{T}\big\rvert_{T_1}^{T_2}
                = &\\&
                = n\,(C_V+R)\,(T_2-T_1)
                = n\,((R\,5/2)+R)\,\left(
                    \left(
                        \frac{
                            \adif{U}_{(8\to1)\unit{\bar}}
                        }{
                            n\,(R\,5/2)
                        }
                        + T_1
                    \right)
                    -T_1
                \right)
                = &\\&
                = n\,R\,\frac{7}{2}\,\left(
                    \frac{
                        \adif{U}_{(8\to1)\unit{\bar}}
                    }{
                        n\,(R\,5/2)
                    }
                \right)
                = \adif{U}_{(8\to1)\unit{\bar}}
                \,\frac{7}{5}
                \cong &\\&
                \cong 
                (\num{-2143.883072507199375})
                \,\frac{7}{5}
                \num{8.314462618}
                \cong
                \num{-3.001436301510079125e3}
            &
        \end{flalign*}
        
    \end{questionBox}
    
\end{questionBox}


\begin{questionBox}1{} % Q3
    
    Um \unit{\mol} de um gás perfeito, inicialmente à pressão de 8\,\unit{\bar} e à temperatura de 140\,\unit{\celsius}, é expandido a
    diabaticamente até à pressão de 1\,\unit{\bar}. Tome \(C_V = 5/2\,R\) para o gás e calcule \(\Delta U\) e \(\Delta H\) para a transformação.
    
    \paragraph*{Nota:} Para verificarmos qual caminho será percorrido podemos usar a seguinte equação
    \begin{BM}
        \adif{(P\,Vol^\gamma)}=0
        \qquad
        \gamma = C_p/C_v
    \end{BM}

    \begin{questionBox}3{\(\adif{U}\)} % Q3 (i)
        
        \begin{flalign*}
            &
                \adif{U}
                = n\,R\,\frac{5}{2}\,(T_2-T_1)
                = n\,R\,\frac{5}{2}\,\left(
                    \frac{P_2\,V_2}{n\,R}
                    -T_1
                \right)
                = &\\&
                = n\,R\,\frac{5}{2}\,\left(
                    \frac{
                        P_2\,\left(
                            V_1\,\sqrt[\gamma]{P_1/P_2}
                        \right)
                    }{
                        n\,R
                    }
                    -T_1
                \right)
                = &\\&
                = n\,R\,\frac{5}{2}\,\left(
                    \frac
                    {
                        P_2\,(n\,R\,T_1/P_1)\,(P_1/P_2)^{5/7}
                    }
                    {
                        n\,R
                    }
                    -T_1
                \right)
                = n\,R\,\frac{5}{2}\,T_1\,\left(
                    (P_2/P_1)^{2/7}
                    -1
                \right)
                = &\\&
                = (1)\,(\num{8.314462618})\,\frac{5}{2}\,(140+273.15)
                \,\left(
                    (1\E5/8\E5)^{2/7}
                    -1
                \right)
                \cong
                \num{-3846.950295512115899}
            &
        \end{flalign*}
        
    \end{questionBox}

    \begin{questionBox}3{\(\adif{H}\)} % Q3 (ii)
        
        \begin{flalign*}
            &
                \adif{H}
                = n\,R\,\frac{7}{2}\,(T_2-T_1)
                = n\,R\,\frac{7}{2}\,\left(
                    T_1(P_2/P_1)^{2/7}
                    -T_1
                \right)
                = &\\&
                = n\,R\,\frac{7}{2}\,T_1\left(
                    (P_2/P_1)^{2/7}
                    -1
                \right)
                = &\\&
                = (1)\,(\num{8.314462618})\,\frac{7}{2}\,(140+273.15)
                \,\left(
                    (1\E5/8\E5)^{2/7}
                    -1
                \right)
                \cong
                \num{-5385.730413716962259}
            &
        \end{flalign*}
        
    \end{questionBox}
    
\end{questionBox}


\begin{questionBox}1{} % Q4
    
    Exalar ar durante o processo de respiração envolve empurrar o ar contra a pressão atmosférica. Um adulto médio exala cerca de 0.5\,\unit{\deci\meter^3} de ar quando expira. Imagine que o ar exalado é admitido num cilindro fechado com um êmbolo, e desloca o êmbolo de 0.5\,\unit{\deci\meter^3} contra a atmosfera.

\end{questionBox}

\begin{questionBox}2{} % Q4.1
    
    Calcule o trabalho associado ao processo.

    \begin{flalign*}
        &
            w
            = -\int P_{ext}\,\odif{V}
            = -P_{ext}\,\adif{V}
            \cong 
            -1.01\E{5}\,\unit{\pascal} 
            * 0.5\E{-3}\,\unit{\meter^3}
            = -50.5\,\unit{\joule}
        &
    \end{flalign*}
    
\end{questionBox}

\begin{questionBox}2{} % Q4.2
    
    Se um trabalho equivalente fosse utilizado para elevar um garrafão de água de 7\,\unit{\liter}, a que altura seria possível elevar o garrafão?

    \begin{flalign*}
        &
            w 
            = \int \vec F \cdot \vec{\odif{d}}
            = m\,g\,\adif{H}
            \implies
            \adif{H} = \frac{
                50.5
            }{
                7*\num{9.80665}
            }
            \cong
            \num{0.73565241079122}
        &
    \end{flalign*}
    
\end{questionBox}

\begin{sectionBox}1{} % S1
    
    Em processos espontaneos:
    \begin{BM}
        \Delta S_{tot} > 0
    \end{BM}
    Critério de espontanedade.

    \begin{BM}
        \odif{q_{rev}} > \odif{q_{irrev}}
    \end{BM}

    Desigualdade de Clausius
    \begin{BM}[align*]
        T\,\odif{S}&>\odif{q_{irrev}}
        \\ 
        T\,\odif{S}&\geq\odif{q}
    \end{BM}
    
\end{sectionBox}

\begin{definitionBox}1{Energia de Helmholtz} % D1
    
    \begin{BM}
        A\equiv U-T\,S
    \end{BM}
    
\end{definitionBox}

\begin{sectionBox}1{} % S2
    
    \begin{BM}
        \odif{A}\leq\odif{w}
    \end{BM}

    \begin{flalign*}
        &
            A\equiv U-T\,S
            \implies
            % &\\[1.5ex]&
            \odif{A} 
            = \odif{U}
            - T\,\odif{S}
            - S\,\odif{T}
            = \odif{q}
            + \odif{w}
            - T\,\odif{S}
            - S\,\odif{T}
            \implies &\\&
            \implies
            \odif{q}
            = \odif{A}
            - \odif{w}
            - T\,\odif{S}
            \leq
            T\,\odif{S}
            % \implies &\\&
            \implies
            \odif{A}\leq\odif{w}
        &
    \end{flalign*}
    
\end{sectionBox}

\begin{sectionBox}1{} % S3
    
    \begin{BM}
        \odif{G}\leq\odif{w_{adia}}
    \end{BM}

    \begin{flalign*}
        &
            G
            = H-T\,S
            \implies
            \odif{G}
            = \odif{H}
            - \odif{T\,S}
            = \odif{U}
            + \odif{P\,V}
            - T\,\odif{S}
            - S\,\odif{T}
            = &\\&
            = \odif{q}
            + \odif{w}
            + P\,\odif{V}
            + V\,\odif{P}
            - T\,\odif{S}
            - S\,\odif{T}
            = &\\&
            = \odif{q}
            + \odif{w}
            + P\,\odif{V}
            + V\,\odif{P}
            - T\,\odif{S}
            &\\&
            w
            = w_{adia}
            + w_{exp}
            = w_{adia}
            - \int P_{ext}\odif{V}
            \implies
            \odif{w}
            = \odif{w_{adia}}
            - p\,\odif{V}
            \implies &\\&
            \implies
            \odif{G}
            = \odif{q}
            + \odif{w_{adia}}
            - p\,\odif{V}
            + P\,\odif{V}
            + V\,\odif{P}
            - T\,\odif{S}
            = \odif{q}
            + \odif{w_{adia}}
            - T\,\odif{S}
            \implies &\\&
            \implies
            \odif{q} 
            = \odif{G} 
            - \odif{w_{adia}}
            + T\,\odif{S}
            \leq T\,\odif{S}
            \implies
            \odif{G}\leq \odif{w_{adia}}
        &
    \end{flalign*}

    \paragraph*{Conclusão}
    \(\adif{G}\equiv f()\) energia livre, do que?
    Do \textit{w} de expansão que muitas vezes não é relevante

    \(\adif{G}\) nos dá o \textit{w} maximo exceto o trabalho de expanão associado ao processo a \textit{T} e \textit{P} constantes do sistema fechado.

    \begin{sectionBox}2{}
        
        Se so for possível trabalho expansivo \(\odif{w_{adia}}=0\implies\odif{G}\leq 0\)
        Em sistema fechado a \textit{T} e \textit{P} constantes, com apenas trabalho expansivo possível o processo evoluem espontanea no sentido de minimisar \textit{G}. no equilíbrio: \textit{G} é constante
        
    \end{sectionBox}
    
\end{sectionBox}


\begin{questionBox}1{} % Q5
    
    Um recipiente de 0.5\,\unit{\metre^3} contém 2\,\unit{\mole} de um gás perfeito a 300\,\unit{\kelvin}. Imagine que o gás sofre uma expanção até ocupar um volume total de 5.0\,\unit{\metre^3}.
    
\end{questionBox}

\begin{questionBox}2{} % Q5.1
    
    Calcule \(\Delta U, Q, W, \Delta S\) e \(\Delta S_{tot}\) envolvidos ba expanção, na situação em que a mesma é realizada reversivelmente a T envolvidos na expanção, na situação em que a mesma é realizada reversivelmente a T constante.

    \begin{flalign*}
        &
            \Delta U = Q + W = 0
            \implies Q = -W
            = \int P_{Ext}\,\odif{v}
            = \int P\,\odif{v}
            = \int \frac{n\,R\,T}{V}\,\odif{v}
            = &\\&
            = n\,R\,T\int \frac{\odif{V}}{V}
            = n\,R\,T\,\ln\frac{V_2}{V_1}
            = 2*8.314*300*\ln\frac{5}{0.5}
            \,\unit{\joule}
            \cong -\qty{11486.21547789149749}{\joule}
        &
    \end{flalign*}

    \begin{questionBox}3{Entropia (\(\Delta S\))} % Q5.1 (i)
        
        \begin{flalign*}
            &
                \Delta S 
                = \int \odif{S}
                = \int \frac{\odif{q}_{rev}}{T}
                = T^{-1}\int \odif{q}
                = \frac{Q}{T}
                \cong 
                \frac{
                    \qty{11486.21547789149749}{\joule}
                }{
                    300\,\unit{\kelvin}
                }
                \cong
                \qty{38.287384926304992}{\joule\per\kelvin}
            &
        \end{flalign*}
        
    \end{questionBox}
    
\end{questionBox}

\begin{questionBox}2{} % Q5.2
    
    Calcule os valores de \(\Delta U, Q, W, \Delta S\) e \(\Delta S_{total}\) quando o gás vai do mesmo estado inicial ao mesmo estado final irreversivelmente, da forma lustrada na figura nesse caso o gás expande por remoção súbita de uma parede móvel que o separa de um recipiente de 4.5\,\unit{\metre^3} sob vácuo

    \begin{questionBox}3{Energia intérna} % Q5.2 (i)
        
        \begin{BM}
            \Delta U = 0
        \end{BM}

        \paragraph{Nota:} Etado incial = a final, por manter mesma temperatura não varia a energia interna
        
    \end{questionBox}

    \begin{questionBox}3{Trabalho e Calor} % Q5.2 (ii)
        
        \begin{BM}
            Q = -W = \int P_{ext}\,\odif{V} = 0
        \end{BM}
        
    \end{questionBox}

    \begin{questionBox}3{Entalpia \(\Delta S\)} % Q5.2 (iii)

        A variação de entalpia é igual ao valor anterior.
        
        sistema começa e acaba no mesmo estado \textit{a}

        \begin{flalign*}
            &
                \Delta S \cong \qty{38.287384926304992}{\joule\per\kelvin}\,\unit{\joule\per\kelvin}
                &\\&
                \Delta S_{viz} = 0
                &\\&
                \Delta S_{tot} 
                = \Delta S + \Delta S_{viz}
                \cong \qty{38.287384926304992}{\joule\per\kelvin}\,\unit{\joule\per\kelvin}
            &
        \end{flalign*}
        
    \end{questionBox}
    
\end{questionBox}

\begin{questionBox}1{} % Q6
    
    Considere que submete 1\,\unit{\mole} de um gás perfeito (\(C_v = 5/2\,R\)) ao processor reversível esquematizado.

    \begin{enumerate}
        \begin{multicols}{4}
            \item 57\,\unit{\celsius}
            \item 11.5\,\unit{\deci\meter^3}
            \item 17.0\,\unit{\celsius}
            \item 1.2\,\unit{\bar}, 22.1\unit{\celsius}
        \end{multicols}
    \end{enumerate}

    \begin{itemize}[left=3em]
        \begin{multicols}{3}
            \item[1 \to{} 2] P constante
            \item[2 \to{} 3] T constante
            \item[3 \to{} 4] Adiabático
        \end{multicols}
    \end{itemize}
    
\end{questionBox}

\begin{questionBox}2{} % Q6.1
    
    Calcule o trabalho e o calor postos em jogo em cada um dos percursos 1 \to{} 2 e 2 \to{} 3.

    \begin{questionBox}3{1 \to{} 2} % Q6.1 (i)
        
        \begin{flalign*}
            &
                Q_p
                = \int n\,C_p\,\odif{T}
                = n\,C_v\,R\,\Delta T
                = 1*\frac{7}{2}*8.314(17.0-57.0)
                \cong \num{-1163.96}
            &
        \end{flalign*}

        \begin{flalign*}
            &
                W
                = \adif{U} - Q
            &
        \end{flalign*}
        
    \end{questionBox}

    \begin{questionBox}3{2 \to{} 3} % Q6.1 (i)
        
        \begin{flalign*}
            &
                -Q = w 
                = -\int P_{ext}\,\odif{V}
                = -\int p\,\odif{V}
                = -\int \frac{n\,R\,T}{V}\,\odif{V}
                = -n\,R\,T\,\log\frac{V_3}{V_2}
            &
        \end{flalign*}
        
    \end{questionBox}
    
\end{questionBox}

\begin{questionBox}2{} % Q6.2
    
    Calcule \(\Delta H\) para o percurso 1 \to{} 4

    \begin{flalign*}
        &
            \adif{H}_{1\to{}4}
            = \int_1^4 n\,C_p\,\odif{t}
            = n\,C_p\,\adif{t}
            = 1*3.5
            * \left(
                \frac{12*22.1}{1*0.08314}
                - 330.15
            \right)
            \cong
            -321\,\unit{\joule}
        &
    \end{flalign*}
    
\end{questionBox}

\begin{questionBox}2{} % Q6.3
    
    Calcule \(\Delta S\) para o percurso 4 \to{} 1.

    \begin{flalign*}
        &
            \adif{S}_{4\to1}
            = \int_4^1 \frac{n\,C_v}{T}\,\odif{T}
            + n\,R\,\ln\frac{V_1}{V_4}
            = \int_4^1 \frac{n\,C_v}{T}\,\odif{T}
            + n\,R\,\ln\frac{P_4}{P_1}
        &
    \end{flalign*}
    
\end{questionBox}

\begin{questionBox}2{} % Q6.4
    
    Imagine uma forma irreversível de ir do estado 2 ao estado 3, e calcule o trabalho e \(\Delta U\) associados.

    Comente o resutado.
    
\end{questionBox}


\begin{questionBox}1{} % Q7
    
    
    
\end{questionBox}

\begin{questionBox}2{} % Q7.1
    
    Calcule a variação de energia interna associada à passagem de 5\,\unit{\mole} de metanol do estado (20\,\unit{\celsius}, liq, 1\,\unit{\bar}) ao estado (80\,\unit{\celsius}, gas, 1\,\unit{\bar})

    \begin{flalign*}
        &
            \Delta H_{(20\to80)\unit{\celsius}}
            = \sum\adif{H}
            = \int_{20}^{64.65} n\,C_p\,\odif{T}
            + n\,\adif{H_{vap}}
            + \int_{64.65}^{80} n\,C_p\,\odif{T}
            = &\\&
            = 5*80(64.65-20)
            + 5*37.6\E{3}
            + 5*48*(80-64.65)
            \cong
            \qty{209.544}{\kilo\joule}
            % 209.544\,\unit{\kilo\joule}
        &
    \end{flalign*}

    \begin{flalign*}
        &
            \adif{V_{l\to g}}
            = \frac{
                5*0.08314
                * (80+273.15) % = 146.804455
            }{
                1
            }
            - 5*\frac{1}{0.792}*32\E-3 % = 0.202020202020202
            \cong
            \num{146.602434797979798}
        &
    \end{flalign*}

    \begin{flalign*}
        &
            \adif{U}
            \cong 
            \num{209544}
            - 1\E5*\num{0.146602434797979798}
            \cong
            \num{194883.7565202020202}
        &
    \end{flalign*}
    
\end{questionBox}

\begin{questionBox}2{} % Q7.2
    
    Calcule a entropia molar do metanol sólido a -97.55\,\unit{\celsius} e 1\,\unit{\bar}.

    \begin{itemize}
        \begin{multicols}{2}
            \item \(C_{\rho,l} = 80\,\unit{\joule/\kelvin\mole}\)
            \item \(C_{\rho,g} = 48\,\unit{\joule/\kelvin\mole}\)
            \item \(\rho_{liq} = 0.792\,\unit{\gram\per\centi\meter^3}\)
            \item \(\Delta H_{l\to g,-97.55\,\unit{\celsius},1\,\unit{\bar}} = 3.2\,\unit{\kilo\joule\per\mole}\)
            \item \(\Delta H_{l\to g,64.65\,\unit{\celsius},1\,\unit{\bar}} = 37.6\,\unit{\kilo\joule\per\mole}\)
            \item \(\ch{CH3OH} = 32\,\unit{\gram\per\mole}\)
            \item \(S(g,80\,\unit{\celsius},1\,\unit{\bar}) = 247.9\,\unit{\joule\,\kelvin^{-1}\,\mole^{-1}}\)
        \end{multicols}
    \end{itemize}

    \begin{flalign*}
        &
            \adif{S}
            =\sum_i\adif{S_i}
        &
    \end{flalign*}

    \begin{flalign*}
        &
            \int \frac{n\,C_p\,\odif{T}}{T}
            =n\,C_p\int \frac{\odif{T}}{T}
            =n\,C_p\ln\frac{T_f}{T_i}
        &
    \end{flalign*}

    \begin{flalign*}
        &
            \int\frac{\odif{H}}{T} \dots
        &
    \end{flalign*}
    
\end{questionBox}

\begin{questionBox}1{} % Q8
    
    Calcule o trabalho máximo associado à condensação da água a 1.01\,\unit{\bar} e 100\,\unit{\celsius}
    \begin{itemize}
        \item \(\Delta H_{vap, 100\,\unit{\celsius}, 1\,\unit{\atm}} = 40.7\,\unit{\kilo\joule\per\mole}\)
    \end{itemize}

    \begin{flalign*}
        &
            \max{w}
            = \adif{A}_{gás\to liq}
            = \adif{U}
            - \adif{T\,S}
            = \adif{H}
            - P\,\adif{V}
            - T\,\adif{S}
            = - P\,\adif{V}
            = &\\&
            = - P\,(V_{liq} - V_{gas})
            \land
            V_{gás}/n
            = \frac{R\,T}{P}
            \cong 
            \frac{
                \num{8.314462618}\E-2*373.15
            }{
                1.01
            }
            \cong
            \qty{30.718234909967327}{\centi\metre^3\per\mole}
            \,
            \land &\\&
            \land
            V_{liq}
            \to 0
            % \implies &\\&
            \implies
            \max w
            \cong -1.01*(-\num{30.718234909967327}\E-3)
            \cong \num{3.102541725906e2}
        &
    \end{flalign*}
    
\end{questionBox}

\begin{questionBox}1{} % Q9
    
    Um gás perfeito(\(C_V=R\,5/2\)) realiza o ciclo reversível 1-2-3-4-1 representado na figura. Calcule:

    \paragraph*{Dados}
    \begin{multicols}{2}
        \begin{itemize}
            \item 1\to2, 3\to4: isotérmicos
            \item 2\to3, 4\to1: Adiabáticos
        \end{itemize}
    
        \begin{minipage}{\textwidth}
            \begin{enumerate}
                \item 0.5\,\unit{\deci\metre^3}
                \item 6.2\,\unit{\bar}, 5.0\,\unit{\deci\metre^3}, 100\,\unit{\celsius}
                \item 10.8\,\unit{\deci\metre^3}
                \item \(S = 175\)\,\unit{\joule\,\kelvin^{-1}\,\mole^{-1}}
            \end{enumerate}
        \end{minipage}
    \end{multicols}
    
\end{questionBox}

\begin{questionBox}2{} % Q9.1
    
    A pressão do gás em 3.
    
    \paragraph*{Nota}
    Se não for gás perfeito ex: liquido, sólido\dots
    Para dar conta de 1 percurso alvo, combinar passos a \textit{T} e \textit{P} constante

    \begin{BM}
        \adif{H}
        -\int n\,C_p\,\odif{t}
        :\adif{P}=0
    \end{BM}

    Mas nós \dots
    \begin{BM}
        \adif{P}=0\implies
        \adif{H}
        = \int\pdv{H}{T}_P\odif{T}
        = \int C_P\odif{T}
        \\
        \adif{P}=0\implies
        \adif{S}
        = \int\pdv{S}{T}_P\odif{T}
        = \int C_P\frac{\odif{T}}{T}
    \end{BM}
    
\end{questionBox}

\begin{definitionBox}1{} % D3
    
    Coeficiente de expanção a pressão constante

    \begin{BM}
        \pdv{V}{T}_P
        = \alpha_P\,V
    \end{BM}

    \begin{BM}
        \odif{G}
        = -S\,\odif{T}
        + V\,\odif{P}
        \\
        \pdv{G}{T}_P = -S
        \qquad
        \pdv{G}{P}_T = V
        \\
        \pdv{\pdv{G}{T}_P}{P}_T
        = \pdv{G}{T,P}_T
        = \pdv{G}{P,T}_P
        = n\pdv{\pdv{G}{P}_T}{T}_P
    \end{BM}

    \begin{flalign*}
        &
            \pdv{H}{P}_T
            = T\pdv{S}{P}_T
            + V\pdv{S}{P}_T
            = V
            - T\pdv{V}{T}_P
            = V
            - \alpha_P\,V\,T
            = V(
                1-\alpha_P\,T
            )
        &
    \end{flalign*}
    
\end{definitionBox}

\begin{questionBox}2{} % Q9.2
    
    O trabalho posto em jogo no percurso 2\to3
    
\end{questionBox}

\begin{questionBox}2{} % Q9.3
    
    \(\adif{G}\) no percurso 4\to1.
    
\end{questionBox}

\end{document}
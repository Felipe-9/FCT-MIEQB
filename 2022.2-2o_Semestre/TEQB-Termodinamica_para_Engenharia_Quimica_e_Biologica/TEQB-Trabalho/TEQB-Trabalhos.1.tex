% !TEX root = ./TEQB-Trabalhos.1.tex
\providecommand\mainfilename{"./TEQB-Trabalhos.tex"}
\providecommand \subfilename{}
\renewcommand   \subfilename{"./TEQB-Trabalhos.1.tex"}
\documentclass[\mainfilename]{subfiles}

% \graphicspath{{\subfix{../images/}}}
% \tikzset{external/force remake=true} % - remake all

\begin{document}

\mymakesubfile{1}
[TREQB]
{Trabalho 1} % Subfile Title
{Trabalho} % Part Title

\begin{definitionBox}1{Equação de Antoine} % DEF1
    
    \begin{BM}
        \log_10(p) = A-\frac{B}{C+T}
    \end{BM}

    \begin{itemize}[left=3em]
        \item[\textit{p}--] pressão de vapor absoluta
        \item[\textit{T}--] Temperatura em Celsius
        \item[\textit{A,B,C}--] Coeficientes de Antoine espeçificos a substancia
    \end{itemize}
    
\end{definitionBox}

\begin{definitionBox}1m{} % DEF2

    Curva de Vaporização
    
    \begin{BM}
        \ln\frac{P_2}{P_1}
        = -\frac{\adif{H}_{vap}}{R}
        \,\left(
            T_2^{-1}
            -T_1^{-1}
        \right)
        \\[1.5ex]
        (t_1,P_1) =\text{ponto triplo}
    \end{BM}
    % o ponto de ref da eq da curva de vap é o ponto triplo
    
    Curva de Sublimação

    \begin{BM}
        \ln\frac{P_2}{P_1}
        = -\frac{\adif{H}_{sub}}{R}
        \,\left(
            T_2^{-1}
            -T_1^{-1}
        \right)
    \end{BM}

    \begin{flalign*}
        &   
            \ln{P_2}
            = \left(
                -\frac{\adif{H}_{sub}}{R}
                \,\left(
                    T_2^{-1}
                    -T_1^{-1}
                \right)
                +\ln{P_1}
            \right)
        &
    \end{flalign*}

    \begin{BM}
        \adif{H}_{sub}
        = \adif{H}_{vap}
        + \adif{H}_{fus}
    \end{BM}

    % \begin{flalign*}
    %     &
    %         \odv{p}{T}
    %         = \adv{H}{V}\frac{1}{T}
    %         =\frac{
    %             \adif{H}
    %         }{
    %             \adif{\frac{n\,R\,T}{P}}\,T
    %         }
    %     &
    % \end{flalign*}
    
    Entalpia de vaporização

    \begin{enumerate}
        \item Abre o gráfico ``Heat of vaporization''
        \item Confere que é quase constante
        \item Fáz média de todos os valores acima de 20 graus do ponto triplo
    \end{enumerate}

    \begin{BM}
        \adif{H}_{vap}
        = 
    \end{BM}
    Curva de Fusão

    \begin{BM}
        P_2-P_1
        = \frac{\adif{H}_{fus}}{\adif{V}}
        \,\ln\frac{T_2}{T_1}
    \end{BM}

\end{definitionBox}


\end{document}
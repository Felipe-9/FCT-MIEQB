% !TEX root = ./TEQB_-_Aula_1_Anotações.tex
% !TEX root = ./Sub_File.tex
\providecommand\mainfilename{"./TEQB_-_Aulas_Anotações.tex"}
\providecommand \subfilename{}
\renewcommand   \subfilename{"./TEQB_-_Aula_1_Anotações.tex"}
\documentclass[\mainfilename]{subfiles}

% \graphicspath{{\subfix{../images/}}}
% \tikzset{external/force remake=true} % - remake all

\begin{document}

\mymakesubfile{1}
[TEQB]
{Aula}
{Aula}

\begin{definitionBox}1{Sistema}
    
    Objeto de estudo que pode interagir com a vizinhança

    \begin{itemize}
        \item Aberto: Tora massa e energia com a vizinhança
        \item Fexado: Troca energia com a vizinhança
        \item Isolado: Não interage com a vizinhança
    \end{itemize}
    
\end{definitionBox}

\begin{sectionBox}1{Processos}
    
    \begin{itemize}
        \item T constante: Isotérmicos
        \item P constante: Isobáricos
        \item V constante: Isocóricos
    \end{itemize}

    \paragraph{Adiabáticos:} Não ha troca de calor com a vizinhaça
    
\end{sectionBox}

\begin{sectionBox}1{Propríedades}
    
    \begin{itemize}
        \item Extensivas: dependem da extenção (volume, massa, tamanho)
        \item Intensívas: não dependem da extençao (preção ou propriedades Ext. dividida por massa ou N de mol)
    \end{itemize}
    
\end{sectionBox}

\begin{sectionBox}1{Leis da Termodinâmica}
    
    \begin{enumerate}
        \item Concervação de Energia
    \end{enumerate}
    
\end{sectionBox}

\begin{sectionBox}2{Concervação de Energia}
    
    \begin{BM}
        \Delta E + \Delta E_{viz} = 0
    \end{BM}
    
\end{sectionBox}

\begin{sectionBox}1{}
    
    \begin{itemize}
        \item \(q>0\) - Sistema recebe Q da viz
        \item \(q>0\) - Sistema cede Q da viz
        \item \(w<0\) - W realizado sobre vizinhança
        \item \(w>0\) - W realizado sobre Sistema
    \end{itemize}
    
\end{sectionBox}

\begin{questionBox}1{}
    
    trabalho expansivo

    \begin{flalign*}
        &
            w
            = -\int P_{ext}\,\delta v
            = -\int {\frac{n\,R\,T}{V}}\,\delta v
            = -n\,R\,T\int \frac{\delta v}{V}
            = -n\,R\,T\ln\frac{V_f}{V_i}
            = -n\,R\,T\ln\frac{2\,V_i}{V_i}
            = 1\,\unit{\mole}
            * 8.314\,\unit{\joule\kelvin^{-1}\mole^{-1}}
            * 298.15\,\unit{\kelvin}
        &
    \end{flalign*}
    
\end{questionBox}

\end{document}
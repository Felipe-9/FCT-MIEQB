% !TEX root = ./TEQB-Aulas_Anotações.4.tex
\providecommand\mainfilename{"./TEQB-Aulas_Anotações.tex"}
\providecommand \subfilename{}
\renewcommand   \subfilename{"./TEQB-Aulas_Anotações.4.tex"}
\documentclass[\mainfilename]{subfiles}

% \graphicspath{{\subfix{../images/}}}
% \tikzset{external/force remake=true} % - remake all

\begin{document}

\mymakesubfile{4}
[TEQB]
{Aula}
{Aula}

\begin{sectionBox}1{Equação de Gibbs} % S1
    
    \begin{BM}
        \odif{G}
        = -S\odif{T}
        + -V\odif{P}
        \\[1.5ex]
        \odif{G}
        = \pdv{G}{T}_P
        \odif{T}
        \qquad
        \odif{G}
        = \pdv{G}{P}_T
        \odif{P}
    \end{BM}

    \paragraph*{Condições}
    Sistema fechado com apenas trabalho expansivo.
    \\

    Para dar conta da variação do \# mols em um sistema fechado é necessário ocorrer uma reação química com troca de massa.
    
\end{sectionBox}


\begin{definitionBox}1{Poténcial Químico}
    
    \begin{BM}
        \mu_i = \pdv{G}{n_i}_{T,P,n_j}
    \end{BM}

    \begin{itemize}
        \item \(n\): Numero de mols de todas as espécies
        \item \(n_i\): Numero de mols de uma espécie
        \item \(n_j\): Numero de mols de todas as espécies exceto \(n_i\)
    \end{itemize}
    
\end{definitionBox}

\begin{sectionBox}2{Extenção das equações de Gibbs} % S1.1
    
    \begin{BM}
        \odif{G}
        = \pdv{G}{T}_{P,n}\,\odif{T}
        + \pdv{G}{P}_{T,n}\,\odif{P}
        + \sum_{i}{
            \pdv{G}{n}_{n_j,P,T}\,\odif{n_i}
        }
        = \\
        = \pdv{G}{T}_{P,n}\,\odif{T}
        + \pdv{G}{P}_{T,n}\,\odif{P}
        + \sum_{i}{
            \mu_i\,\odif{n_i}
        }
        = \\
        = \pdv{G}{T}_P\,\odif{T,n}
        + \pdv{G}{P}_T\,\odif{P,n}
        + \sum_{\alpha}{
            \sum_{i}{
                \mu_{i,\alpha}\,\odif{n_{i,\alpha}}
            }
        }
        \\[1.5ex]
        : n_j = n \backslash \{n_i\}
    \end{BM}

    \begin{itemize}
        \item \(n\): Numero de mols de todas as espécies
        \item \(n_i\): Numero de mols de uma espécie
        \item \(n_j\): Numero de mols de todas as espécies exceto \(n_i\)
        \item \(i\): Elemento
        \item \(\alpha\): Fase (sólida, liquida...)
    \end{itemize}

    \begin{itemize}[left = 10em]
        \item[\textbf{Temperatura}] Regula o estabelecimento do equilíbrio Térimico
        \item[\textbf{Pressão}] Regula o estabelecimento do equilíbrio Mecânico
        \item[\textbf{Portêncial Químico}] Regula o estabelecimento do equilíbrio Mássico
    \end{itemize}

    
\end{sectionBox}

\begin{exampleBox}1{}
    
    Fluido em dois estádos físicos com o primeiro mudando para o segundo

    \begin{flalign*}
        &
            0>\odif{G}
            =\mu_{\alpha}\,\odif{n_{\alpha}}
            +\mu_{\beta }\,\odif{n_{\beta }}
            =
            - \mu_{\alpha}\,\odif{n}
            + \mu_{\beta }\,\odif{n}
            = (\mu_{\beta }-\mu_{\alpha})\,\odif{n}
            \implies
            \mu_{\beta }<\mu_{\alpha}
        &
    \end{flalign*}

    no equilíbrio

    \begin{BM}
        \odif{G}=0
        \implies
        \mu_{\beta }<\mu_{\alpha}
    \end{BM}
    
\end{exampleBox}

\begin{sectionBox}1{Equação de Clapeyrom} % S2
    
    \begin{BM}
        \odv{P}{T}=\adv{S_m}{V_m}
    \end{BM}

    \begin{flalign*}
        &
            \mu_{1,\alpha}
            = \mu_{1,\beta}
            \land
            \mu_{2,\alpha}
            = \mu_{2,\beta}
            % \implies &\\&
            \implies
            \odif{\mu_{\alpha}}
            = \odif{\mu_{\beta}}
            \implies &\\&
            \implies
            \odif{G}_{m,\alpha}
            = \odif{G}_{m,\beta}
            \land
            G=m\,G_m
            \implies &\\&
            \implies
            m\,\odif{G}_{m}
            + G_m\,\odif{m}
            = -S\,odif{T}
            + V\,\odif{P}
            + \mu\,\odif{m}
            \implies &\\&
            \implies
            \odif{G}_m
            \land -S_m\,\odif{T}+V_m\,\odif{P}
            \implies &\\&
            \implies
            -S_{m,\alpha}\,\odif{T}
            +V_{m,\alpha}\,\odif{P}
            = -S_{m,\beta}\,\odif{T}
            +  V_{m,\beta}\,\odif{P}
            \implies &\\&
            \implies
            ( S_{m,\alpha}-S_{m,\beta})\,\odif{T}
            =(V_{m,\alpha}-V_{m,\beta})\,\odif{P}
            \implies &\\&
            \implies
            \odv{P}{T}=\adv{S_m}{V_m}
        &
    \end{flalign*}
    
\end{sectionBox}

\begin{sectionBox}1{} % S2
    
    \begin{BM}
        \ln P_2
        = \ln P_1
        + \frac{\adif{H_{vap}}}{R}
        \,\left(
            T_2^{-1}
            - T_1^{-1}
        \right)
    \end{BM}

    \begin{flalign*}
        &
            \adif{V_{vap}} 
            = V_{m,g} - V_{m,l}
            \approx V_{m,g}
            \approx \frac{R\,T}{P}
            \implies &\\&
            \implies
            \odv{P}{T}
            = \frac{\adif}{R\,T/P}
            \implies &\\&
            \implies
            \frac{\odif{P}}{P}
            = \frac{\adif{H_{vap}}}{R\,T}
            \,\odif{T}
            \implies &\\&
            \implies
            \int\odif{\ln{P}}
            = \ln P_2-\ln P_1
            = &\\&
            = \int\frac{\adif{H_{vap}}}{R}
            \,\frac{\odif{T}}{T^2}
            = \frac{\adif{H_{vap}}}{R}
            \,\int{
                \frac{\odif{T}}{T^2}
            }
            = \frac{\adif{H_{vap}}}{R}
            \,\left(
                T_2^{-1}
                - T_1^{-1}
            \right)
            \implies &\\&
            \implies
            \ln P_2
            = \ln P_1
            + \frac{\adif{H_{vap}}}{R}
            \,\left(
                T_2^{-1}
                - T_1^{-1}
            \right)
        &
    \end{flalign*}
    
    
\end{sectionBox}


\begin{sectionBox}1{} % S3
    
    Fusão

    \begin{flalign*}
        &
            \odv{P}{T}
            = \frac{\adif{H_{Vap}}}
            {T\,\adif{V_m}}
            \implies
            \int{\odif{P}}
            = \adif{P}
            = \int{
                \adv{H_{vap}}{V_m}
                \frac{\odif{T}}{T}
            }
            \cong 
            \adv{H_{vap}}{V_m}
            \ln\frac{T_2}{T_1}
        &
    \end{flalign*}
    
\end{sectionBox}

\end{document}
% !TEX root = ./TEQB-Aulas_Anotações.2.tex
\providecommand\mainfilename{"./TEQB-Aulas_Anotações.tex"}
\providecommand \subfilename{}
\renewcommand   \subfilename{"./TEQB-Aulas_Anotações.2.tex"}
\documentclass[\mainfilename]{subfiles}

% \graphicspath{{\subfix{../images/}}}
% \tikzset{external/force remake=true} % - remake all

\begin{document}

\mymakesubfile{2}
[TEQB]
{Anotações}
{Anotações}

\renewcommand\thequestion{Questão 1\,--\,\arabic{question}}
\renewcommand\thesubquestion{Q1\,--\,\arabic{question} \alph{subquestion})}

\begin{sectionBox}1{Gás Perfeito}
    
    \begin{BM}
        P\,V = n\,R\,T
        \\
        U = U_{(T)}
    \end{BM}
    
\end{sectionBox}

\begin{questionBox}1{}
    
    1\,\unit{\mole} de um gás perfeito, inicialmente a 25\,\unit{\celsius} e 1\,\unit{\bar}, sofre uma expansão. No estado final, \(T = 25\,\unit{\celsius}\) e \(P = 0.5\,\unit{\bar}\).

\end{questionBox}

\begin{questionBox}2m{}
    
    Calcule o trabalho de expansão posto em jogo quando o processo se dá seguindo dois percursos diferentes:

    \begin{questionBox}3{}

        Processo reversível a T constante
        
        \begin{flalign*}
            &
                \lvert w \rvert
                = 1718\,\unit{\joule}
            &\\\\&
                Q - W  
                = Q - 1718\,\unit{\joule}
                = \Delta U = 0
                \implies
                Q = 1718\,\unit{\joule}
            &
        \end{flalign*}

    \end{questionBox}

    \begin{questionBox}3{}
        
        processo irreversível, mediante alívio súbito da pressão exterior para 0.5\,\unit{\bar}, seguida de expansão do gás contra essa pressão.

        \begin{flalign*}
            &
                \lvert w \rvert
                = 1240\,\unit{\joule}
            &\\\\&
                Q - W  
                = Q - 1240\,\unit{\joule}
                = \Delta U = 0
                \implies
                Q = 1240\,\unit{\joule}
            &
        \end{flalign*}

        % \begin{flalign*}
        %     &
        %         Q - W  
        %         = Q - 1240\,\unit{\joule}
        %         = \Delta U = 0
        %         \implies
        %         Q = 1240\,\unit{\joule}
        %     &
        % \end{flalign*}
        
    \end{questionBox}
    
\end{questionBox}

\setcounter{subquestion}{2}
\begin{questionBox}2{}

    Calcule \(\Delta U\) e \(Q\) para as alineas a.I e a.II.
    
    \begin{flalign*}
        &
            \Delta U 
            = Q + W 
            = Q_v - \int P_{ext}\,\odif{v}
            = Q_v
        &
    \end{flalign*}
    
\end{questionBox}

\begin{questionBox}2{}
    
    Deduza as expressões para \(\Delta U\) e \(\Delta H\) associados a cada um dos passos do percurso a.II

    \begin{flalign*}
        &
            \Delta U
            = Q + W
            = Q_p - \int P_{ext} \odif{v}
            = Q_p - P \adif{v}
        &\\[1.5ex]&
            H 
            \equiv U + P\,V
            \implies &\\&
            \implies
            \Delta H
            = \Delta U + \Delta (P\,V)
            % = &\\&
            % = \Delta U + P\,\Delta V + \Delta P\,V
            % = \Delta U + P_2\,(V_2-V_1) + (P_2-P_1)\,V_2
            % = \Delta U + P_2\,V_2-P_2\,V_1 + P_2\,V_2-P_1\,V_2
            % = &\\&
            = \Delta U + (
                P_2\,V_2
                - P_1\,V_1
            )
            = \Delta U + P\,\Delta V
            = Q_p
        &
    \end{flalign*}

    \begin{flalign*}
        &
            \Delta U_{1\to 3} 
            = Q_V 
            = \int_{1}^{3} n\,C_V\,\odif{T}
            = n\,C_V\int_{1}^{3}\odif{T}
            = n\,C_V\,(T_3-T_1)
        &\\[1.5ex]&
            \Delta U_{3\to 2}
            = Q_p
            = \int_{3}^{2} n\,C_p\,\odif{T}
            = n\,C_p\,\int_{3}^{2} \odif{T}
            = n\,C_p\,(T_2-T_3)
        &\\[1.5ex]&
            \implies
            \Delta U_{1\to 2}
            = \Delta U_{1\to 3}
            + \Delta U_{3\to 2}
            = n\,C_V\,(T_3-T_1)
            + n\,C_p\,(T_2-T_3)
        &
    \end{flalign*}

    \paragraph{Nota:} Apenas para gases perfeitos (\(C_V \text{ e } C_p\) constantes)
    
\end{questionBox}

\begin{sectionBox}1{Entalpia nos gáses perfeitos}

    \begin{BM}
        H
        \equiv U + P\,V
        = U + n\,R\,T
        = H_{(T)}
    \end{BM}

    \begin{sectionBox}3{}
        
        \begin{BM}
            C_p = C_V + R
        \end{BM}
    
        \begin{flalign*}
            &
                \odif{H}
                = n\,C_p\,\odif{T}
                = \odif{U} + \odif{n\,R\,T}
                = n\,C_V\,\odif{T}
                + n\,R\,\odif{T}
                \implies &\\&
                \implies
                C_p = C_V + R
            &
        \end{flalign*}
        
    \end{sectionBox}

    \begin{sectionBox}3{}
        
        \begin{BM}
            p\,V^\gamma = cte
        \end{BM}
        
    \end{sectionBox}
    
\end{sectionBox}

\begin{sectionBox}1{}
    
    \begin{BM}
        Q_V = \Delta U = \int n\,C_V\,\odif{T}
        \\
        Q_p = \Delta H = \int n\,C_p\,\odif{T}
    \end{BM}
    
\end{sectionBox}

\begin{questionBox}1{}
    
    Um \unit{\mole} de um gás perfeito, inicialmente à pressão de 8\,\unit{\bar} e à temperatura de 140\,\unit{\celsius}, é expandido adiabaticamente contra a atmosfera, até se estabelecer o equilíbrio de pressões. Tome \(C_v = 5/2\,R\) para o gás e calcule \(\Delta U\) e \(\Delta H\) para a tranformação.

    \begin{questionBox}3{\(\Delta U\)}
        
        \begin{flalign*}
            &
                \Delta U_{
                    (8\to 1)\,\unit{\bar}
                }
                = Q+W
                = W
                = \int n\,C_V\,\odif{T}
                = \int n\,(C_p + R)\,\odif{T}
                = &\\&
                = n\,\frac{5\,R}{2}\,\adif{T}
                = 1*2.5*8.314\,(T_f - (140 + 274.15))
            &\\[1.5ex]&
                \Delta H_{
                    (8\to 1)\,\unit{\bar}
                }
                = \int n\,C_p\,\odif{T}
                = n\,\int (C_v-R)\,\odif{T}
                = &\\&
                = n\,(C_v+R)\,\adif{T}
                = 1*3.5*8.314\,(T_f - (140 + 274.15))
            &\\[1.5ex]&
                w 
                = -\int P_{ext}\,\odif{V}
                = -P_{ext}\int \odif{V}
                = -P_{ext}\adif{V}
                = -P_{ext}\left(
                    V_f-V_i
                \right)
                = &\\&
                = -P_{ext}\left(
                    \frac{n\,R\,T_f}{P_f}
                    - \frac{n\,R\,T_i}{P_i}
                \right)
                = &\\&
                = -1.01\E5
                \,\left(
                    (1*0.08314)\left(
                        \frac{T_f}{1.01}
                        - \frac{140 + 274.15}{8}
                    \right)
                \right)
                \,10^{-3}
                = &\\&
                = -1.01\E2 \left(
                    (1*0.08314)\left(
                        \frac{T_f}{1.01}
                        - \frac{140 + 274.15}{8}
                    \right)
                \right)
                = \Delta U
            &
        \end{flalign*}
        
    \end{questionBox}
    
\end{questionBox}

\end{document}
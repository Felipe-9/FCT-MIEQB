% !TEX root = ./TEQB-Aulas_Anotações.tex
% !TEX root = ./TEQB-Aulas_Anotações.3.tex
\providecommand\mainfilename{"./TEQB-Aulas_Anotações.tex"}
\providecommand \subfilename{}
\renewcommand   \subfilename{"./TEQB-Aulas_Anotações.3.tex"}
\documentclass[\mainfilename]{subfiles}

% \graphicspath{{\subfix{../images/}}}
% \tikzset{external/force remake=true} % - remake all

\begin{document}

\mymakesubfile{3}
[TEQB]
{}
{}

\begin{sectionBox}1{Máquina Térmica (MT)}
    
    
    
\end{sectionBox}

\begin{sectionBox}2{Gás perfeito em MT}
    
    \begin{BM}
        -w = Q_q + Q_f
    \end{BM}

    \begin{flalign*}
        &
            \Delta Q 
            = 0
            = Q_q + W + Q_f + w
            \cong
            Q_q + W + Q_f
            \implies
            -w = Q_q + Q_f
        &
    \end{flalign*}


    % \begin{BM}
        
    % \end{BM}
    
\end{sectionBox}

\begin{definitionBox}1{Ciclo de Carnot}
        
    \begin{itemize}
        \item 2 Isotérmicas
        \item 2 Adiabaticas
    \end{itemize}

    \begin{BM}
        0 = \oint \frac{\odif{q}}{T} + 0
    \end{BM}

    \begin{flalign*}
        &
            \Delta U = Q + W
            \implies
            \odif{U}
            = \odif{q} + \odif{W}
            = \odif{q} - P_{ext}\,\odif{V}
            = \odif{q} - p\,\odif{V}
            \land &\\&
            \land p = \frac{n\,R\,T}{V}
            % \implies &\\&
            \implies
            \frac{\odif{u}}{T}
            = \frac{\odif{q}}{T}
            - \frac{n\,R}{V}\,\odif{V}
            \implies
            \oint \frac{\odif{u}}{T}
            = \oint \frac{\odif{q}}{T}
            - \oint \frac{n\,R}{V}\,\odif{V}
        &
    \end{flalign*}
    
\end{definitionBox}


\begin{sectionBox}1{Eficiencia}
        
    \begin{BM}
        e 
        = 1 - \frac{T_f}{T_q}
    \end{BM}

    \begin{flalign*}
        &
            e 
            = -w / Q_q
            = \frac{Q_q + Q_t}{Q_q}
            = 1 + \frac{Q_t}{Q_q}
        &
    \end{flalign*}

    
    \begin{flalign*}
        &
            \Delta U = Q + W
            \implies
            \odif{U}
            = \odif{q} + \odif{W}
            = \odif{q} - P_{ext}\,\odif{V}
            = \odif{q} - p\,\odif{V}
            \land &\\&
            \land p = \frac{n\,R\,T}{V}
            % \implies &\\&
            \implies
            \frac{\odif{u}}{T}
            = \frac{\odif{q}}{T}
            - \frac{n\,R}{V}\,\odif{V}
            \implies
            0
            = \oint \frac{\odif{u}}{T}
            = &\\&
            = \oint \frac{\odif{q}}{T}
            - \oint \frac{n\,R}{V}\,\odif{V}
            = \oint \frac{\odif{q}}{T}
            = \int_1^2 \frac{\odif{q}}{T}
            + \int_2^3 \frac{\odif{q}}{T}
            + \int_3^4 \frac{\odif{q}}{T}
            + \int_4^1 \frac{\odif{q}}{T}
            = &\\&
            = \int_1^2 \frac{\odif{q}}{T}
            + \int_3^4 \frac{\odif{q}}{T}
            = \frac{Q_q}{T_q}
            + \frac{Q_f}{T_f}
            \implies
            \frac{Q_f}{Q_q}
            = - \frac{T_f}{T_q}
        &
    \end{flalign*}

    \begin{flalign*}
        &
            \therefore e 
            = 1 - \frac{T_f}{T_q}
        &
    \end{flalign*}
    
\end{sectionBox}

\begin{questionBox}1{}
    
    Um recipiente de 0.5\,\unit{\metre^3} contém 2\,\unit{\mole} de um gás perfeito a 300\,\unit{\kelvin}. Imagine que o gás sofre uma expanção até ocupar um volume total de 5.0\,\unit{\metre^3}.
    
\end{questionBox}

\begin{questionBox}2{}
    
    Calcule \(\Delta U, Q, W, \Delta S\) e \(\Delta S_{tot}\) envolvidos ba expanção, na situação em que a mesma é realizada reversivelmente a T envolvidos na expanção, na situação em que a mesma é realizada reversivelmente a T constante.

    \begin{flalign*}
        &
            \Delta U = Q + W = 0
            \implies Q = -W
            = \int P_{Ext}\,\odif{v}
            = \int P\,\odif{v}
            = \int \frac{n\,R\,T}{V}\,\odif{v}
            = &\\&
            = n\,R\,T\int \frac{\odif{V}}{V}
            = n\,R\,T\,\ln\frac{V_2}{V_1}
            = 2*8.314*300*\ln\frac{5}{0.5}
            \,\unit{\joule}
            \cong -\qty{11486.21547789149749}{\joule}
        &
    \end{flalign*}

    \begin{questionBox}3{Entropia (\(\Delta S\))}
        
        \begin{flalign*}
            &
                \Delta S 
                = \int \odif{S}
                = \int \frac{\odif{q}_{rev}}{T}
                = T^{-1}\int \odif{q}
                = \frac{Q}{T}
                \cong 
                \frac{
                    \qty{11486.21547789149749}{\joule}
                }{
                    300\,\unit{\kelvin}
                }
                \cong
                \qty{38.287384926304992}{\joule\per\kelvin}
            &
        \end{flalign*}
        
    \end{questionBox}
    
\end{questionBox}

\begin{questionBox}2{}
    
    Calcule os valores de \(\Delta U, Q, W, \Delta S\) e \(\Delta S_{total}\) quando o gás vai do mesmo estado inicial ao mesmo estado final irreversivelmente, da forma lustrada na figura nesse caso o gás expande por remoção súbita de uma parede móvel que o separa de um recipiente de 4.5\,\unit{\metre^3} sob vácuo

    \begin{questionBox}3{Energia intérna}
        
        \begin{BM}
            \Delta U = 0
        \end{BM}

        \paragraph{Nota:} Etado incial = a final, por manter mesma temperatura não varia a energia interna
        
    \end{questionBox}

    \begin{questionBox}3{Trabalho e Calor}
        
        \begin{BM}
            Q = -W = \int P_{ext}\,\odif{V} = 0
        \end{BM}
        
    \end{questionBox}

    \begin{questionBox}3{Entalpia \(\Delta S\)}

        A variação de entalpia é igual ao valor anterior.
        
        sistema começa e acaba no mesmo estado \textit{a}

        \begin{flalign*}
            &
                \Delta S \cong \qty{38.287384926304992}{\joule\per\kelvin}\,\unit{\joule\per\kelvin}
                &\\&
                \Delta S_{viz} = 0
                &\\&
                \Delta S_{tot} 
                = \Delta S + \Delta S_{viz}
                \cong \qty{38.287384926304992}{\joule\per\kelvin}\,\unit{\joule\per\kelvin}
            &
        \end{flalign*}
        
    \end{questionBox}
    
\end{questionBox}

\begin{sectionBox}1{Entropia de um gás perfeito}
    
    \begin{BM}
        \Delta S 
        = \int n\,\frac{C_v}{T}\,\odif{T}
        + n\,R\,\ln\frac{V_2}{V_1}
        = \\
        = \int n\,\frac{C_p}{T}\,\odif{T}
        + n\,R\,\ln\frac{P_1}{P_2}
    \end{BM}

    \begin{flalign*}
        &
            \odif{U}
            = n\,C_v\,\odif{T}
            = \odif{q}
            + \odif{w}
            = \odif{q}
            + P_{ext}\,\odif{V}
            = \odif{q}
            + p\,\odif{V}
            = \odif{q}
            + \frac{n\,R\,T}{V}\,\odif{V}
            \implies &\\&
            \implies
            \int \frac{\odif{q}}{T}
            = \int \frac{n\,C_V}{T}\,\odif{T}
            + \int \frac{n\,R}{V}\,\odif{V}
        &
    \end{flalign*}
    
\end{sectionBox}

\end{document}
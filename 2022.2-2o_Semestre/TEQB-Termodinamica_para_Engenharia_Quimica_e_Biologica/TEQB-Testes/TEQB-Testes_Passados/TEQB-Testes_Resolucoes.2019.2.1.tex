% !TEX root = ./TEQB-Testes_Resoluções.2019.2.1.tex
\providecommand\mainfilename{"./TEQB-Testes_Resoluções.tex"}
\providecommand \subfilename{}
\renewcommand   \subfilename{"./TEQB-Testes_Resoluções.2019.2.1.tex"}
\documentclass[\mainfilename]{subfiles}

% \graphicspath{{\subfix{../images/}}}
% \tikzset{external/force remake=true} % - remake all

\begin{document}

\mymakesubfile{1}
[TEQB]
{Teste 2019.2.1}
{Teste 2019.2.1}

\sisetup{round-precision=3}

\begin{questionBox}1{} % Q1
    
    \begin{center}
        \includegraphics[width = 0.8\textwidth]{./.build/figures/Screenshot 2022-10-17 at 12.17.32.png}
    \end{center}

    Considere que submete 1\,\unit{\mole} de um gás perfeito (\(C_V = R\,5/2\)) ao processo reversível representado.
    
\end{questionBox}

\begin{questionBox}2{ % Q1.1
    Calcule o trabalho posto em jogo no percurso 2\to3.
} % Q1.1

    \begin{flalign*}
        &
            W_{
                (30.2\to120.6)\unit{\celsius}
                P_{cnt}
            }
            = \int_{Vol_1}^{Vol_2} P_{ext}\odif{Vol}
            =  P_{ext}\int_{Vol_1}^{Vol_2}\odif{Vol}
            =  P_{ext}\adif{Vol}\big\rvert_{Vol_1}^{Vol_2}
            = &\\&
            =  P_{ext}\left(
                Vol_2 - Vol_1
            \right)
            % = &\\&
            =  P_{ext}\left(
                \frac{n\,R\,T_2}{P_2} 
                - \frac{n\,R\,T_1}{P_1}
            \right)
            =  n\,R\,\left(
                T_2 - T_1
            \right)
            = &\\&
            =  (1)\,(\num{8.314462618})\,\left(
                (120.6+273.15) - (30.2+273.15)
            \right)
            \cong
            \num{751.6274206672}
        &
    \end{flalign*}
    
\end{questionBox}

\begin{questionBox}2{} % Q1.2
    
    Calcule o trabalho e o calor postos em jogo no percurso 3\to4.

    \begin{questionBox}3{\textit{W}} % Q1.2 (i)
        \begin{flalign*}
            &
                W_{vol_{cnt}}
                = 0
            &
        \end{flalign*}
    \end{questionBox}

    \begin{questionBox}3{\textit{Q}} % Q1.2 (ii)
        \begin{flalign*}
            &
                Q_{
                    vol_{cnt},
                    (120.6\to-52.8)\unit{\celsius}
                }
                = \int_{T_3}^{T_4}{
                    n\,C_v\,\odif{T}
                }
                = n\,C_v\,\int_{T_3}^{T_4}{
                    \odif{T}
                }
                = n\,C_v\,\adif{T}\big\rvert_{T_3}^{T_4}
                = &\\&
                = (1)
                \,\left(\num{8.314462618}*5/2\right)
                \,((-52.8+273.15) - (120.6+273.15))
                \cong
                \num{-3604.319544903}
            &
        \end{flalign*}
    \end{questionBox}
    
\end{questionBox}

\begin{questionBox}2{} % Q1.3
    
    Calcule a pressão do gás no estado 1.

    \begin{flalign*}
        &
            P_1
            = \frac{n\,R\,T_1}{vol_1}
            = \frac{
                n\,R\,T_1
            }{
                vol_1
            }
            = \frac{
                n\,R\,T_1
            }{
                vol_2\sqrt[\gamma]{P_2/P_1}
            }
            = \cfrac{
                n\,R\,T_1
            }{
                \left(\cfrac{n\,R\,T_2}{P_2}\right)
                \,(P_3/P_1)^{C_v/C_P}
            }
            = &\\&
            = \cfrac{
                T_1
            }{
                T_2
                \,P_3^{-1}
                \,(P_3/P_1)^{5/7}
            }
            = \frac{
                T_1
                \,P_1^{5/7}
            }{
                T_2
                \,P_3^{5/7-1}
            }
            \implies
            P_1
            = \left(
                \frac{
                    T_1
                }{
                    T_2
                    \,\left(
                        \frac{n\,R\,T_3}{vol_3}
                    \right)^{5/7-1}
                }
            \right)^{1/(1-5/7)}
            = &\\&
            = \left(
                \frac{T_1}{T_2}
            \right)^{1/(1-5/7)}
            \,\frac{n\,R\,T_3}{vol_3}
            = \left(
                \frac{94.6+273.15}{30.2+273.15}
            \right)^{7/2}
            \,\frac{
                (1)\,(\num{8.314462618})\,(120.6+273.15)
            }{
                5.87\E-6
            }
            \cong &\\&
            \cong
            \num{1094.074858065844468e6}
        &
    \end{flalign*}
    
\end{questionBox}

\begin{questionBox}2{} % Q1.4
    
    Calcule \(\adif{S_{viz}}\) para o percurso 4\to5.

    \begin{flalign*}
        &
            \adif{S_{viz}}
            = - \adif{S}
            = - \left(
                \int{
                    n\,C_V\,\frac{\odif{T}}{T}
                }
                + n\,R\,\ln\frac{V_5}{V_4}
            \right)
            = - \int_{T_4}^{T_5}{
                n\,C_V\,\frac{\odif{T}}{T}
            }
            - n\,R\,\ln\frac{V_f}{V_i}
            = &\\&
            = 
            - n\,C_v\,\ln\frac{T_5}{T_4}
            - n\,R\,\ln\frac{
                (n\,R\,T_5/P_5)
            }{
                V_4
            }
            = &\\&
            = - n\,C_v\,\ln
            \left(
                \frac{
                    \frac{\adif{U}}{n\,C_v}
                    +T_4
                }{
                    T_4
                }
            \right)
            - n\,R\,\ln\frac{
                n\,R
                \,\left(
                    \frac{\adif{U}}{n\,C_v}
                    +T_4
                \right)
            }{
                V_4\,P_5
            }
            = &\\&
            = - n\,C_v\,\ln
            \left(
                \frac{\adif{U}}{n\,C_v\,T_4}
                +
                1
            \right)
            - n\,R\,\ln\frac{
                R
                \,\left(
                    \frac{\adif{U}}{C_v}
                    +T_4
                \right)
            }{
                V_4\,P_5
            }
            = &\\&
            = - (1)\,(\num{8.314462618}*5/2)\,\ln
            \left(
                \frac{208}{(1)\,(\num{8.314462618}*5/2)\,(-52.8+273.15)}
                +
                1
            \right)
            + &\\&
            - (1)\,(\num{8.314462618})\,\ln\frac{
                (\num{8.314462618})
                \,\left(
                    \frac{208}{\num{8.314462618}*5/2}
                    +(-52.8+273.15)
                \right)
            }{
                (5.87\E-3)\,(1.26\E5)
            }
            \cong
            -8.8
        &
    \end{flalign*}

    \paragraph*{Nota:} N aguento mais esses calculos, fazer com variáveis intermediárias é menos preciso porem mais fácil
    
\end{questionBox}

\setcounter{question}{2}
\setcounter{subquestion}{0}

\begin{questionBox}2{} % Q2.1
    
    Calcule a variação de energia interna associada à passagem da água gasosa, a 138.9\,\unit{\celsius} e 1.01\,\unit{\bar}, a água sólida, a 0\,\unit{\celsius} e 1.01\,\unit{\bar}.

    \begin{flalign*}
        &
            \adif{U}_{
                \ch{H2O\gas{}\to\sld{}},
                (138.9\to0)\unit{\celsius},1.01\unit{\bar}
            }
            \underset{P_{cnt}}{=}
            \adif{H} - P\adif{V}
            = &\\&
            = \left(
                \begin{aligned}
                    & 
                    \adif{H}_{
                        \ch{H2O\sld{}},
                        (138.9\to100)\unit{\celsius}
                    }
                    \\ + &
                    \adif{H}_{
                        \ch{H2O(\sld{}\to\lqd{})},
                        100\unit{\celsius}
                    }
                    \\ + &
                    \adif{H}_{
                        \ch{H2O\lqd{}},
                        (100\to0)\unit{\celsius}
                    }
                    \\ + &
                    \adif{H}_{
                        \ch{H2O(\lqd{}\to\sld{})},
                        0\unit{\celsius}
                    }
                \end{aligned}
            \right)
            - P(V_f-V_i)
            = &\\&
            = \left(
                \begin{aligned}
                    & 
                    n\,C_{p, (g)}((100+273.15)-(138.9+273.15))
                    \\ + &
                    n\,(-\adif{H}_{vap})
                    \\ + &
                    n\,C_{p,(l)}((0+273.15)-(100+273.15))
                    \\ + &
                    n\,(-\adif{H}_{fus})
                \end{aligned}
            \right)
            - P\left(
                \frac{n}{M}\,\rho_{(s)}
                -
                \frac{n\,R\,T_i}{P_i}
            \right)
            = &\\&
            = n\,\left(
                \begin{aligned}
                    & 
                    C_{p, (g)}(100-138.9)
                    \\ + &
                    (-\adif{H}_{vap})
                    \\ + &
                    C_{p,(l)}(-100)
                    \\ + &
                    (-\adif{H}_{fus})
                \end{aligned}
            \right)
            - n\left(
                \frac{P\,\rho_{(s)}}{M}
                -
                R\,T_i
            \right)
            \cong &\\&
            \cong n\,\left(
                \begin{aligned}
                    & 
                    (36)(100-138.9)
                    \\ + &
                    (-40.7\E3)
                    \\ + &
                    (75)(-100)
                    \\ + &
                    (-6.01\E3)
                \end{aligned}
            \right)
            - n\left(
                \frac{1.01\E5*18}{0.92\E6}
                -
                \num{8.314462618}\,(138.9+273.15)
            \right)
            \cong &\\&
            \cong
            (n)
            \num{-5.2186401765209621739e4}
            % \num{-5.56104e4}
            % \num{-3.423998234790378261e3}
        &
    \end{flalign*}
    
\end{questionBox}

\begin{questionBox}2m{} % Q2.2
    
    Calcule o trabalho máximo associado à transformação da alínea anterior.

    \begin{flalign*}
        &
            \max W
            = \adif{A}
            = \adif{U} - \adif{(T\,S)}
            = \adif{U} - (
                T_f\,S_f
                -T_i\,S_i
            )
            = \adif{U}
            -T_f\,S_f
            +T_i\,S_i
            = &\\&
            = \adif{U}
            -T_f\,\left(
                    S_{(l),25\unit{\celsius},1\unit{\bar}}
                    +\adif{S}_{
                        ((l),25\unit{\celsius},1\unit{\bar})
                        \to
                        f
                    }
                \right)
            + &\\&
            +T_i\,\left(
                S_{(l),25\unit{\celsius},1\unit{\bar}}
                +\adif{S}_{
                    ((l),25\unit{\celsius},1\unit{\bar})
                    \to
                    i
                }
            \right)
            = &\\&
            = \adif{U}
            - T_f\,\left(
                \begin{aligned}
                    & S_{(l),25\unit{\celsius},1\unit{\bar}}
                    \\ + &
                    \adif{S}_{
                        ((l),25\unit{\celsius},1\unit{\bar})
                        \to
                        f
                    }
                \end{aligned}
            \right)
            + T_i\,\left(
                \begin{aligned}
                    & S_{(l),25\unit{\celsius},1\unit{\bar}}
                    \\ + &
                    \adif{S}_{
                        ((l),25\unit{\celsius},1\unit{\bar})
                        \to
                        i
                    }
                \end{aligned}
            \right)
            = &\\&
            = \adif{U}
            - T_f\,\left(
                \begin{aligned}
                    & S_{(l),25\unit{\celsius},1\unit{\bar}}
                    \\ + &
                    \adif{S}_{(l),(25\to0)\unit{\celsius}}
                    \\ + &
                    \adif{S}_{(l\to s),0\unit{\celsius}}
                \end{aligned}
            \right)
            + T_i\,\left(
                \begin{aligned}
                    & S_{(l),25\unit{\celsius},1\unit{\bar}}
                    \\ + &
                    \adif{S}_{(l),(25\to100)\unit{\celsius}}
                    \\ + &
                    \adif{S}_{(l\to g),100\unit{\celsius}}
                    \\ + &
                    \adif{S}_{(g),(100\to138.9)\unit{\celsius}}
                \end{aligned}
            \right)
            = &\\&
            \underset{P_{cnt}}{=}
            \adif{U}
            - T_f\,\left(
                \begin{aligned}
                    & S_{(l),25\unit{\celsius},1\unit{\bar}}
                    \\ + &
                    \int{C_{p,(l)}\,\frac{\odif{T}}{T}}
                    \\ + &
                    (-\adif{H}_{fus})/T
                \end{aligned}
            \right)
            + T_i\,\left(
                \begin{aligned}
                    & S_{(l),25\unit{\celsius},1\unit{\bar}}
                    \\ + &
                    \int{C_{p,(l)}\,\frac{\odif{T}}{T}}
                    \\ + &
                    \adif{H}_{vap}/T
                    \\ + &
                    \int{C_{p,(g)}\,\frac{\odif{T}}{T}}
                \end{aligned}
            \right)
            = &\\&
            =
            \adif{U}
            - T_f\,\left(
                \begin{aligned}
                    & S_{(l),25\unit{\celsius},1\unit{\bar}}
                    \\ + &
                    C_{p,(l)}\ln\frac{0+273.15}{25+273.15}
                    \\ + &
                    (-\adif{H}_{fus})/T
                \end{aligned}
            \right)
            + T_i\,\left(
                \begin{aligned}
                    & S_{(l),25\unit{\celsius},1\unit{\bar}}
                    \\ + &
                    C_{p,(l)}\ln\frac{100+273.15}{25+273.15}
                    \\ + &
                    \adif{H}_{vap}/T
                    \\ + &
                    C_{p,(g)}\ln\frac{138.9+273.15}{100+273.15}
                \end{aligned}
            \right)
            \cong
        &
    \end{flalign*}

    \begin{flalign*}
        &
            \cong
            \num{-5.2186401765209621739e4}
            - (273.15)\,\left(
                \begin{aligned}
                    & 69.95
                    \\ + &
                    (75)\ln\frac{0+273.15}{25+273.15}
                    \\ + &
                    (-6.01\E3)/(273.15)
                \end{aligned}
            \right)
            + &\\&
            + (138.9+273.15)\,\left(
                \begin{aligned}
                    & 69.95
                    \\ + &
                    (75)\ln\frac{100+273.15}{25+273.15}
                    \\ + &
                    40.7\E3/(100+273.15)
                    \\ + &
                    (36)\ln\frac{138.9+273.15}{100+273.15}
                \end{aligned}
            \right)
            \cong &\\&
            % \cong
            % \num{-5.2186401765209621739e4}
            % - \num{11302.746439873542388}
            % + \num{82171.057741211071606}
            \cong
            \num{18681.909536127907479}
        &
    \end{flalign*}
    
\end{questionBox}

\begin{questionBox}2{} % Q2.3
    
    Em que medida a 3a lei da Termodinâmica complementa o sentido da 2a lei?

    
    
\end{questionBox}

\end{document}
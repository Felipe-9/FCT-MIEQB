% !TEX root = ./TEQB-Testes_Resoluções.2022.2.1.tex
\providecommand\mainfilename{"./TEQB-Testes_Resoluções.tex"}
\providecommand \subfilename{}
\renewcommand   \subfilename{"./TEQB-Testes_Resoluções.2022.2.1.tex"}
\documentclass[\mainfilename]{subfiles}


% \tikzset{external/force remake=true} % - remake all

\begin{document}

\graphicspath{{\subfix{./.build/figures/TEQB-Testes_Resoluções.2022.2.1/}}}

\mymakesubfile{1}
[TEQB]
{Teste 2022.2.1}
{Teste 2022.2.1}

\begin{questionBox}1{ % Q1
    Considere que tem 1\,\unit{\mole} de um gás perfeito (\(C_V = 5/2 R\)). Na figura estão representados estados deste gás (1, 2, 3 e 4) e transições reversíveis entre eles. \(P_1 = 4.0\,\unit{\bar}, T_1 = 293.15\,\unit{\kelvin}, T_4 = 197.27\,\unit{\kelvin}\), as transições 2\,\to\,3 e 4\,\to\,1 são adiabáticas, o calor envolvido na transição 1\,\to\,2 é de 5466\,\unit{\joule}, e o calor envolvido na transição 3\,\to\,4 é de -4100\,\unit{\joule}.
} % Q1
    % \begin{multicols}{2}
        \begin{itemize}
            \begin{multicols}{3}
                \item \(n = 1\,\unit{\mole}\)
                \item \(C_V = 5/2 R\)
                \item \(P_1 = 4.0\,\unit{\bar}\)
                \item \(T_1 = 293.15\,\unit{\kelvin}\)
                \item \(T_4 = 197.27\,\unit{\kelvin}\)
                \item 2\,\to\,3: Adiabática
                \item 4\,\to\,1: Adiabática
                \item \(\adif{H}_{1\to 2} = +5466\,\unit{\joule}\)
                \item \(\adif{H}_{3\to 4} = -4100\,\unit{\joule}\)
            \end{multicols}
        \end{itemize}

        \begin{center}
            \includegraphics[width=.4\textwidth]{Screenshot 2023-01-05 at 18.15.16.png}
        \end{center}
    % \end{multicols}
\end{questionBox}

\begin{questionBox}2{ % Q1.1
    Calcule \(T_2\)
} % Q1.1
    \begin{flalign*}
        &
            \int_1^2{n\,C_P\,\odif{T}}
            = n\,(C_V+R)\,\adif{T}
            = \adif{H}_{1\to2}
            = Q_{1\to2}
            \implies &\\&
            \implies
            T_2 
            = T_1 + \frac{Q_{1\to2}}{n\,(C_V+R)}
            \cong 293.15 + \frac{5466}{1*3.5*\num{8.314462618}}
            \cong
            \num{480.981055050187696}
        &
    \end{flalign*}
\end{questionBox}

\begin{questionBox}2{ % Q1.2
    Calcule \(T_3\) e \(V_3\)
} % Q1.2
    \subsubquestion{\(T_3\)}
    \begin{flalign*}
        &
            \int_{3\to4}{n\,C_V\,\odif{T}}
            = n\,(C_P+R)\,(T_4-T_3)
            = Q_{3\to4}
            \implies &\\&
            \implies
            T_3
            = T_4 - \frac{Q_{3\to4}}{n\,(C_V)}
            \cong 197.27 - \frac{-4100}{1*2.5*\num{8.314462618}}
            \cong
            \num{394.51666227370607}
        &
    \end{flalign*}

    \subsubquestion{\(V_3\)}
    \begin{flalign*}
        &
            P_3\,V_3^{\gamma}
            = \left(
                \frac{n\,R\,T_3}{V_3}
            \right)\,V_3^{C_P/C_V}
            = n\,R\,T_3\,V_3^{1.4-1}
            = n\,R\,T_3\,V_3^{0.4}
            \overset{2\to3}{\underset{\text{adiab. rev.}}{=}} &\\&
            = P_2\,V_2^{\gamma}
            = P_1\,\left(
                \frac{n\,R\,T_2}{P_2}
            \right)^{1.4}
            = \frac{n^{1.4}\,R^{1.4}\,T_2^{1.4}}{P_1^{1.4-1}}
            \implies &\\&
            \implies
            V_3
            = \left(
                \frac{
                    n^{0.4}\,R^{0.4}\,T_2^{1.4}
                }{
                    P_1^{0.4}\,T_3
                }
            \right)^{1/0.4}
            = \frac{n\,R\,T_2^{3.5}}{P_1\,T_3^{2.5}}
            \cong 
            \frac{
                1
                *\num{8.314462618}
                *(\num{480.981055050187696})^{3.5}
            }{
                4.0
                *10^{5}
                *(\num{394.51666227370607})^{2.5}
            }
            \cong
            \num{1.640811591757612695e-2}
        &
    \end{flalign*}
\end{questionBox}

\begin{questionBox}2{ % Q1.3
    Calcule \(W_{4\to1}\)
} % Q1.3
    \begin{flalign*}
        &
            W_{4\to1} + Q_{4\to1}
            = W_{4\to1}
            = \adif{U}_{4\to1}
            = \int_4^1{n\,C_V\,\odif{T}}
            = n\,C_V\,\adif{T}
            \cong &\\&
            \cong 
            1*2.5*\num{8.314462618}(293.15-197.27)
            \cong
            \num{1992.9766895346}
        &
    \end{flalign*}
\end{questionBox}

\begin{questionBox}2{ % Q1.4
    Calcule \(\adif{S}_{viz}\) no processo 1\to4\to3. (se não resolveu b, considere \(T_3 = 400\,\unit{\kelvin}\))
} % Q1.4
    \begin{flalign*}
        &
            \adif{S}_{viz,1\to4\to3}
            = - \adif{S}_{1\to3}
            = - \left(
                \int_1^3{n\,C_P\,\odif{T}/T}
                + n\,R\,\ln(P_1/P_3)
            \right)
            = &\\&
            = 
            - n\,3.5\,R\,\ln(T_3/T_1)
            - n\,R\,\ln\frac{P_1}{
                \left(
                    \frac{n\,R\,T_3}{V_3}
                \right)
            }
            = - n\,R\left(
                3.5\,\ln(T_3/T_1)
                + \ln\frac{P_1}{
                    \left(
                        \frac{n\,R\,T_3}{V_3}
                    \right)
                }
            \right)
            \cong &\\&
            \cong 
            \num{8.314462618}
            \left(
                - 3.5\,\ln\left(
                    \frac{
                        \num{394.51666227370607}
                    }{
                        283.15
                    }
                \right)
                % - 1.160896041443011
                - \ln\frac{4.0*10^5}{
                    \left(
                        \frac{
                            \num{8.314462618}
                            *\num{394.51666227370607}
                        }{
                            \num{1.640811591757612695e-2}
                        }
                    \right)
                }
                % - 0.69358277418779
            \right) % 1.854478815630801
            \cong
            \num{-15.418994788435206}
        &
    \end{flalign*}
\end{questionBox}

\begin{questionBox}2{ % Q1.5
    Imagine uma transição isotérmica reversível (realizada a \(T_4\)) entre o estado 4 e um estado 5, com \(W_{4\to5} = -3986\,\unit{\joule}\). Calcule \(V_5\). (se não resolveu b, considere \(T_3 = 400\,\unit{\kelvin} \text{ e } V_3 = 15.0\,\unit{\deci\metre^3}\))
} % Q1.5
    \begin{flalign*}
        &
            -n\,R\,T_4\ln(V_5/V_4)
            = W_{4\to5}
            \implies
            V_5 
            = V_4\,\exp{\left(
                -\frac{W_{4\to5}}{n\,R\,T_4}
            \right)}
            \cong &\\&
            \cong 
            \num{1.640811591757612695e-2}
            * \exp{\left(
                -\frac{-3986}{
                    1
                    *\num{8.314462618}
                    *197.27
                }
            \right)}
            \cong
            \num{18.641518344261281e-2}
        &
    \end{flalign*}
\end{questionBox}

\begin{questionBox}2{} % Q1.6

    \begin{questionBox}3{ % Q1.6.1
        Imagine uma forma de levar o gás de 1 a 3 de forma irreversível. Represente graficamente essa transição, bem como o trabalho associado.
    } % Q1.6.1
    \end{questionBox}

    \begin{questionBox}3{ % Q1.6.2
        O coeficiente de Joule-Thomson do \(H_2\) é negativo. Que consequências, em termos da \nth{1} Lei da Termodinâmica, poderão existir no desenho de um motor de combustão, quando o H2 passa através da válvula de saída do depósito a 200K, num processo a entalpia constante?
    } % Q1.6.2
    \end{questionBox}
\end{questionBox}

% \stepcounter{question}
% \setcounter{subquestion}{0}

\begin{questionBox}1{} % Q2
    \begin{itemize}
        \begin{multicols}{2}
            \item \(C_{p,L} = 255.7\,\unit{\joule\,\kelvin^{-1}\,\mole^{-1}}\)
            \item \(C_{p,G} = 239.0\,\unit{\joule\,\kelvin^{-1}\,\mole^{-1}}\)
            \item \(\adif{H}_{vap,(125.6\,\unit{\celsius}, 1.01\,\unit{\bar})} = 41.53\,\unit{\kilo\joule\,\mole^{-1}}\)
            \item \(\alpha_{p,liq} \approx 1.4*10^{-3}\,\unit{\kelvin^{-1}}\)
            \item \(\rho_{liq} = 0.703\,\unit{\gram\,\centi\metre^{-3}} \)
            \item \(M_{(n-octano)} = 114.23\,\unit{\gram\,\mole^{-1}}\)
        \end{multicols}
    \end{itemize}
\end{questionBox}

\begin{questionBox}2{ % Q2.1
    Calcule \(\adif{H} \text{ e } \adif{G}\) associados à passagem de 200\,\unit{\gram} de n-octano do estado (125.6\,\unit{\celsius}, gás, 0.5\,\unit{\bar}) ao estado (125.6\,\unit{\celsius}, líquido, 100\,\unit{\bar})
} % Q2.1

    % ---------------------------------- adif H ---------------------------------- %

    \begin{questionBox}3{} % Q2.1.1
        \begin{flalign*}
            &
                \adif{H}
                = &\\&
                = \left(
                    \begin{aligned}
                        &
                            \adif{H}_{gas,(0.5\to1.01)\,\unit{\bar}}
                        &+\\+&
                            \adif{H}_{(gas\to liq),1.01\,\unit{\bar}}
                        &+\\+&
                            \adif{H}_{liq,(1.01\to100)\,\unit{\bar}}
                        &
                    \end{aligned}
                \right)
                = &\\&
                = \left(
                    \begin{aligned}
                        &
                            0 \quad\text{(gas perfeito)}
                        &+\\+&
                            n\,\adif{H}_{vap}
                        &+\\+&
                            \int_{P_0}^{P_1}{
                                v\,(1-\alpha_{p}\,T)
                                \,\odif{P}
                            }
                        &
                    \end{aligned}
                \right)
                = &\\&
                = \left(
                    \begin{aligned}
                        &
                            (m/M)\,\adif{H}_{vap}
                        &+\\+&
                            (m/\rho)
                            \,(1-\alpha_{p}\,T)
                            \,(P_1-P_0)
                        &
                    \end{aligned}
                \right)
                = &\\&
                = \left(
                    \begin{aligned}
                        &
                            (200/114.23)*41.53*10^3
                            % 72712.947561936444016
                        &+\\+&
                            \frac{200*10^{-3}}{0.703*10^{3}}
                            *(1-(1.4*10^{-3})*(125.6+273.15))
                            *(100-1.01)
                            *10^{5}
                            % 1244.063513513513514
                        &
                    \end{aligned}
                \right)
                \cong &\\&
                \cong
                \num{73957.011075449953514}
            &
        \end{flalign*}
    \end{questionBox}

    % ---------------------------------- adif G ---------------------------------- %

    \begin{questionBox}3{} % Q2.1.2
        \begin{flalign*}
            &
                \adif{G}
                = &\\&
                = \left(
                    \begin{aligned}
                        &
                            \adif{G}_{gas,(0.5\to1.01)\,\unit{\bar}}
                        &+\\+&
                            \adif{G}_{(gas\to liq),1.01\,\unit{\bar}}
                        &+\\+&
                            \adif{G}_{liq,(1.01\to100)\,\unit{\bar}}
                        &
                    \end{aligned}
                \right)
                = &\\&
                = \left(
                    \begin{aligned}
                        &
                            \int_{P_0}^{P_1}{V\,\odif{P}}
                        &+\\+&
                            0
                        &+\\+&
                            \int_{P_1}^{P_2}{V\,\odif{P}}
                        &
                    \end{aligned}
                \right)
                = &\\&
                = \left(
                    \begin{aligned}
                        &
                            \int_{P_0}^{P_1}{
                                \frac{n\,R\,T}{P}
                                \,\odif{P}
                            }
                        &+\\+&
                            V\,\int_{P_1}^{P_2}{\odif{p}}
                            \quad(\text{vol liq constante em} \adif{P})
                        &
                    \end{aligned}
                \right)
                = &\\&
                = \left(
                    \begin{aligned}
                        &
                            (m/M)\,R\,T\,\ln(P_1/P_0)
                        &+\\+&
                            (m/\rho)\,(P_2-P_1)
                        &
                    \end{aligned}
                \right)
                = &\\&
                = \left(
                    \begin{aligned}
                        &
                            (200/114.23)
                            * \num{8.314462618}
                            * (125.6+273.15)
                            *\ln(1.01/0.5)
                            % 4081.316366474564322
                        &+\\+&
                            \frac{200*10^{-3}}{0.703*10^{3}}
                            *(100-1.01)
                            *10^5
                            % 2816.216216216216216
                        &
                    \end{aligned}
                \right)
                \cong &\\&
                \cong
                \num[
                    exponent-mode=fixed
                ]{6897.532582690780216}
            &
        \end{flalign*}
    \end{questionBox}
\end{questionBox}

\begin{questionBox}2{ % Q2.2
    Calcule \(\adif{S} \text{ e } \adif{U}\) associados à passagem de 200\,\unit{\gram} de n-octano do estado (50\,\unit{\celsius}, líquido, 1.01\,\unit{\bar}) ao estado (200\,\unit{\celsius}, gás, 0.5\,\unit{\bar})
} % Q2.2
    
    % ---------------------------------- adif S ---------------------------------- %

    \begin{questionBox}3{} % Q2.2.1
        \begin{flalign*}
            &
                \adif{S}
                = &\\&
                = \left(
                    \begin{aligned}
                        &
                            \adif{S}_{liq,1.01\unit{\bar},(50\to125.6)\unit{\celsius}}
                        &+\\+&
                            \adif{S}_{(liq\to gas),1.01\unit{\bar},125.6\unit{\celsius}}
                        &+\\+&
                            \adif{S}_{gas,1.01\unit{\bar},(125.6\to200)\unit{\celsius}}
                        &+\\+&
                            \adif{S}_{gas,(1.01\to0.5)\unit{\bar},200\unit{\celsius}}
                        &
                    \end{aligned}
                \right)
                = &\\&
                = \left(
                    \begin{aligned}
                        &
                            \int_{T_0}^{T_1}{n\,C_{p,liq}\,\odif{T}/T} + 0
                        &+\\+&
                            n\,\adif{H}_{vap}/T_1
                        &+\\+&
                            \int_{T_1}^{T_2}{n\,C_{p,gas}\,\odif{T}/T} + 0
                        &+\\+&
                            0 + n\,R\int_{P_2}^{P_3}{\odif{P}/P}
                        &
                    \end{aligned}
                \right)
                = &\\&
                = \left(
                    \begin{aligned}
                        &
                            n\,C_{p,liq}\ln(T_1/T_0)
                        &+\\+&
                            n\,\adif{H}_{vap}/T_1
                        &+\\+&
                            n\,C_{p,gas}\,\ln(T_2/T_1)
                        &+\\+&
                            n\,R\,\ln(P_3/P_2)
                        &
                    \end{aligned}
                \right)
                = &\\&
                = \left(
                    \begin{aligned}
                        &
                            255.7*\ln\left(
                                \frac{125.6+273.15}{50+273.15}
                            \right)
                            % 53.75275343916132
                        &+\\+&
                            41.53*10^3/(125.6+273.15)
                            % 104.150470219435737
                        &+\\+&
                            239.0*\ln\left(
                                \frac{200+273.15}{125.6+273.15}
                            \right)
                            % 40.887596316389782
                        &+\\+&
                            \num{8.314462618}
                            * \ln(0.5/1.01)
                            % -5.84587797545316
                        &
                    \end{aligned}
                \right)
                % 192.944941999533679
                * (200/114.23)
                \cong &\\&
                \cong
                \num{337.818334937465953}
            &
        \end{flalign*}
    \end{questionBox}

    % ---------------------------------- adif U ---------------------------------- %

    \begin{questionBox}3{} % Q2.2.2
        \begin{flalign*}
            &
                \adif{U}
                = \adif{H} - \adif{(P\,V)}
                = &\\&
                = \left(
                    \begin{aligned}
                        &
                            \adif{H}_{liq,1.01\unit{\bar},(50\to125.6)\unit{\celsius}}
                        &+\\+&
                            \adif{H}_{(liq\to gas),1.01\unit{\bar},125.6\unit{\celsius}}
                        &+\\+&
                            \adif{H}_{gas,1.01\unit{\bar},(125.6\to200)\unit{\celsius}}
                        &+\\+&
                            \adif{H}_{gas,(1.01\to0.5)\unit{\bar},200\unit{\celsius}}
                        &
                    \end{aligned}
                \right)
                - \adif{(P\,V)}
                = &\\&
                = \left(
                    \begin{aligned}
                        &
                            \int_{T_0}^{T_1}{n\,C_{P,l}\,\odif{T}}
                        &+\\+&
                            n\,\adif{H}_{vap}
                        &+\\+&
                            \int_{T_1}^{T_2}{n\,C_{P,g}\,\odif{T}}
                        &+\\+&
                            0
                        &
                    \end{aligned}
                \right)
                - (P_3\,V_3-P_0\,V_0)
                = &\\&
                = \left(
                    \begin{aligned}
                        &
                            n\,C_{P,l}\,(T_1-T_0)
                        &+\\+&
                            n\,\adif{H}_{vap}
                        &+\\+&
                            n\,C_{P,g}\,(T_2-T_1)
                        &
                    \end{aligned}
                \right)
                - P_3\,\left(
                    \frac{n\,R\,T_3}{P_3}
                \right)
                + P_0\,(m/\rho_{liq})
                = &\\&
                = \left(
                    \begin{aligned}
                        &
                            C_{P,l}\,(T_1-T_0)
                        &+\\+&
                            \adif{H}_{vap}
                        &+\\+&
                            C_{P,g}\,(T_2-T_1)
                        &+\\-&
                            R\,T_3
                        &+\\+&
                            P_0\,M/\rho_{liq}
                        &
                    \end{aligned}
                \right)
                (m/M)
                = &\\&
                = \left(
                    \begin{aligned}
                        &
                            255.7
                            *(125.6-50)
                            % 19330.92
                        &+\\+&
                            41.53*10^{3}
                            % 41530
                        &+\\+&
                            239.0
                            *(200-125.6)
                            % 17781.6
                        &+\\-&
                            \num{8.314462618}
                            *(200+273.15)
                            % 3933.9879877067
                        &+\\+&
                            1.01
                            *114.23
                            /(0.703*10^{6})
                            % 0.000164114224751
                        &
                    \end{aligned}
                \right)
                % 74708.532176407524751
                (200/114.23)
                \cong &\\&
                \cong
                \num[
                    exponent-mode=fixed
                ]{130803.698111542545305}
            &
        \end{flalign*}
    \end{questionBox}

\end{questionBox}

\end{document}
% !TEX root = ./TEQB-Testes_Resoluções.2.tex
\providecommand\mainfilename{"./TEQB-Testes_Resoluções.tex"}
\providecommand \subfilename{}
\renewcommand   \subfilename{"./TEQB-Testes_Resoluções.2.tex"}
\documentclass[\mainfilename]{subfiles}

% \tikzset{external/force remake=true} % - remake all

\begin{document}

% \graphicspath{{\subfix{../images/}}}

\mymakesubfile{1}
[TEQB]
{Teste 2 Resolução}
{Teste Resolução}

\begin{questionBox}1{ % Q1
    A temp afastadas do ponto crítico a curva de vap do but obedece à equação
} % Q1
    
    \vspace{-3ex}
    \begin{BM}
        \ln{(P*10^{-5})} = 13.440-3418.2/T
        \\
        \ln{P} = 5*\ln{10} + 13.440 - 3418.2/T
    \end{BM}

    \begin{itemize}
        \begin{multicols}{2}
            \item \(T_{(triplo,s-l-g)} = 134.86\,\unit{\kelvin}\)
            \item \(P_{sub,122.0\,\unit{\kelvin}} = 3.006*10^{-2}\)
            \item \(M_{butano}=58.12\,\unit{\gram/\mole}\)
            \item \(\rho_{(150.0\,\unit{\kelvin},P_{fus},s)} = 0.792\,\unit{\gram/\centi\metre^3}\)
            \item \(\rho_{(150.0\,\unit{\kelvin},P_{fus},l)} = 0.735\,\unit{\gram/\centi\metre^3}\)
            \item \(\alpha_{P,l} = 1.2*10^{-3}\,\unit{\kelvin^{-1}}\)
            \item \(\alpha_{P,s} = 4.3*10^{-3}\,\unit{\kelvin^{-1}}\)
            \item \(C_{p,s} = 84\,\unit{\joule\,\kelvin^{-1}\,\mole^{-1}}\)
            \item \(C_{p,l} = 132\,\unit{\joule\,\kelvin^{-1}\,\mole^{-1}}\)
            \item \(C_{p,g} = 95\,\unit{\joule\,\kelvin^{-1}\,\mole^{-1}}\)
        \end{multicols}
    \end{itemize}

\end{questionBox}

\begin{questionBox}2{ % Q1.1
    A pressão de fusão do butano a 150.0\,\unit{\kelvin}
} % Q1.1

    \begin{flalign*}
        &
            \odv{P}{T}_{fus}
            = \frac{\adif{H}_{fus}}{T_{fus}\,\adif{V}_{fus}}
            = \frac{
                \adif{H}_{fus}
            }{
                T_{fus}
                \,(
                    V_{l} - V_{s}
                )
            }
            = \frac{
                \adif{H}_{sub}-\adif{H}_{vap}
            }{
                T_{fus}
                \,(
                    M/\rho_{l} - M/\rho_{s}
                )
            }
            = &\\&
            = \frac{
                \adif{H}_{sub}-\adif{H}_{vap}
            }{
                T_{fus}\,M
                \,(\rho_{l}^{-1} - \rho_{s}^{-1})
            }
            ; &\\[2ex]&
            % 
            % H vap
            % 
            \odv{P}{T}_{vap}
            = \frac{\adif{H}_{vap}}{T\,\adif{V}_{vap}}
            = \frac{\adif{H_{vap}}}{T\,(V_{vap,g}-V_{vap\,l})}
            \cong \frac{\adif{H_{vap}}}{T\,V_{vap,g}}
            \cong \frac{\adif{H_{vap}}}{T\,\left(
                R\,T/P
            \right)}
            =
            \frac{P\,\adif{H}_{vap}}{R\,T^2}
            \implies &\\&
            \implies
            \frac{\odif{P}/P}{\odif{T}}
            = \frac{\odif{\ln{P}}}{\odif{T}}
            = \frac{\adif{H}_{vap}}{R\,T^2}
            % = &\\&
            = \odv{}{T}\left(
                (13.440-3418.2/T)*10^5
            \right)
            = &\\&
            = -3418.2*10^5\,T^{-2}
            % \implies &\\&
            \implies
            \adif{H}_{vap} \cong -3418.2*10^5\,R
            ; &\\[2ex]&
            % 
            % H sub
            % 
            \odv{P}{T}_{sub}
            = \frac{\adif{H}_{sub}}{T\,\adif{V}_{sub}}
            = \frac{\adif{H_{sub}}}{T\,(V_{sub,g}-V_{sub\,s})}
            \cong \frac{\adif{H_{sub}}}{T\,V_{sub,g}}
            \cong \frac{\adif{H_{sub}}}{T\,\left(
                R\,T/P
            \right)}
            \dots
            ; &\\[3ex]&
            \therefore
            P_{fus,150.0\,\unit{\kelvin}}
            \cong P_{sub,122.0\,\unit{\kelvin}}
            + \frac{
                \adif{H}_{sub}-\adif{H}_{vap}
            }{
                M(\rho_{l}^{-1} - \rho_{s}^{-1})
            }
            \ln\frac{T_l}{T_s}
        &
    \end{flalign*}

\end{questionBox}

\begin{questionBox}2{ % Q1.2
    \chemDelta{H} associado à passagem do butano no estado (161.0\,\unit{\kelvin}, 0.0002\,\unit{\bar})
    ao estado (161.0\,\unit{\kelvin}, 1300\,\unit{\bar})
} % Q1.2

    \begin{flalign*}
        &
            \adif{H}_{161.0\,\unit{\kelvin},(0.0002\to1300)\,\unit{\bar}}
            ; &\\[2ex]&
            % 
            % P vap a 161
            % 
            P_{vap,161.0\,\unit{\kelvin}}
            = P_l
            + \frac{\adif{H}}{\adif{V}}
            \,\ln\frac{T}{T_0}
        &
    \end{flalign*}

\end{questionBox}

\begin{questionBox}2{ % Q1.2
    \chemDelta{S} associado à passagem do butano no estado (170.0\,\unit{\kelvin}, 0.0004\,\unit{\bar})
    ao estado (150.0\,\unit{\kelvin}, 100\,\unit{\bar})
} % Q1.2
\end{questionBox}

\begin{questionBox}2{ % Q1.3
    \chemDelta{G} associado à fusão do butano a 150.0\,\unit{\kelvin} e 920\,\unit{\bar}.
} % Q1.3
    body
\end{questionBox}

\begin{questionBox}1{ % Q2
} % Q2
    
    \begin{itemize}
        \item \(V_{\ch{MeOH},m,*} =40.45\,\unit{\centi\metre^{3}\,\mole^{-1}}\)
        \item \(V_{\ch{H2O},m,*}  =18.01\,\unit{\centi\metre^{3}\,\mole^{-1}}\)
    \end{itemize}

\end{questionBox}

\begin{questionBox}2{ % Q2.1
    Calcule os volumes de metanol e água puros que são necessários para preparar 250\,\unit{\centi\metre^3} de uma solução com \(X_{\ch{MeOH}}=0.40\)
} % Q2.1

    

\end{questionBox}

\begin{questionBox}2{ % Q2.2
    Adicionou-se à solução da alínea anterior uma quantidade desconhecida de água, de modo a obter \(V_{\ch{MeOH,m}}=36.45\,\unit{\centi\metre^3\,\mole^{-1}}\). Calcule o volume da nova solução.
} % Q2.2

    \begin{flalign*}
        &
            % 
            % x MeOH 2
            % 
            x_{MeOH,2}(36.45-40.45)
            =x_{MeOH,2}(-4)
            \cong 0.2
            ; &\\[2ex]&
            % 
            % V sol 2
            % 
            V_{sol.2}
            = x_{agua,2}\,V_{agua,2}
            + x_{\ch{MeOH},2}\,V_{\ch{MeOH},2}
            = &\\&
            = (1-x_{MeOH,2})
            \,\left(
                V_{agua,m,2}\,n_{agua}
            \right)
            + x_{MeOH,2}
            \,V_{\ch{MeOH},2}
            = &\\&
            = (1-x_{MeOH,2})
            \,\left(
                V_{agua,m,2}
                \left(
                    x_{agua}\,n_t
                \right)
            \right)
            + x_{MeOH,2}
            \,V_{\ch{MeOH},2}
            = &\\&
            = (1-x_{MeOH,2})^2
            \,\left(
                V_{agua,m,2}
                \left(
                    \frac{n_{\ch{MeOH}}}{x_{\ch{MeOH}}}
                \right)
            \right)
            + x_{MeOH,2}
            \,V_{\ch{MeOH},2}
            = &\\&
            = \frac{(1-x_{MeOH,2})^2}{x_{\ch{MeOH}}}
            \,\left(
                V_{agua,m,2}
                \left(
                    x_{\ch{MeOH},1}*n_{t,1}
                \right)
            \right)
            + x_{MeOH,2}
            \,V_{\ch{MeOH},2}
            = &\\&
            = \frac{(1-x_{MeOH,2})^2\,V_{agua,m,2}\,x_{\ch{MeOH},1}}{x_{\ch{MeOH},2}}
            \,\left(
                \frac{V_{sol.1}}{
                    x_{\ch{MeOH},1}\,V_{\ch{MeOH},m,1}
                    +(1-x_{\ch{MeOH},1})\,V_{agua,m,1}
                }
            \right)
            + &\\&
            + x_{MeOH,2}
            \,V_{\ch{MeOH},2}
            \cong &\\&
            \cong \frac{(0.8)^2(18.01-1)*10^{-6}*0.4}{0.2}
            * &\\&
            *\left(
                \frac{250*10^{-6}}{
                     0.4*(40.45-1.8)*10^{-6}
                    +0.6*(18.01-1.8)*10^{-6}
                }
            \right)
            % 216.120066703724291
            + &\\&
            + 0.2
            \,(40.45-2)*10^{-6}
            \cong &\\&
            \cong
            \num{223.810066703724291e-6}
        &
    \end{flalign*}

    Incongruente: volume total não pode diminuir, algo está mal provavelmente nas relações iniciais de volume molar da água na solução 2
    ao invés de 
    \(x_{i,2}\,V_{i,2}\)
    devia ser
    \(x_{i,2}\,V_{i,m,2}\)
    e ao invés de 
    \(V_{sol,2}\)
    \(V_{sol,m,2}\),
    falta multiplicar pelo numero total de mols

    \begin{flalign*}
        &
            % 
            % x MeOH 2
            % 
            x_{MeOH,2}(36.45-40.45)
            =x_{MeOH,2}(-4)
            \cong 0.2
            ; &\\[2ex]&
            % 
            % V sol 2
            % 
            V_{sol.2}
            = \left(
                x_{agua,2}\,V_{agua,m,2}
                + x_{\ch{MeOH},2}\,V_{\ch{MeOH},m,2}
            \right)\,n_{t,2}
            = &\\&
            = \left(
                x_{agua,2}\,V_{agua,m,2}
                + x_{\ch{MeOH},2}\,V_{\ch{MeOH},m,2}
            \right)\,n_{t,2}
            = &\\&
            = \left(
                x_{agua,2}\,V_{agua,m,2}
                + x_{\ch{MeOH},2}\,V_{\ch{MeOH},m,2}
            \right)\left(
                \frac{n_{\ch{MeOH},2}}{x_{\ch{MeOH},2}}
            \right)
            = &\\&
            = \frac{
                \left(
                    x_{agua,2}\,V_{agua,m,2}
                    + x_{\ch{MeOH},2}\,V_{\ch{MeOH},m,2}
                \right)
            }{
                x_{\ch{MeOH},2}
            }
            \left(
                x_{\ch{MeOH},1}*n_{t,1}
            \right)
            = &\\&
            = \frac{
                \left(
                    x_{agua,2}\,V_{agua,m,2}
                    + x_{\ch{MeOH},2}\,V_{\ch{MeOH},m,2}
                \right)
                x_{\ch{MeOH},1}
            }{
                x_{\ch{MeOH},2}
            }
            * &\\&
            * 
            \left(
                \frac{V_{sol.1}}{
                    x_{\ch{MeOH},1}\,V_{\ch{MeOH},m,1}
                    +(1-x_{\ch{MeOH},1})\,V_{agua,m,1}
                }
            \right)
            = &\\&
            = \frac{
                \left(
                    x_{agua,2}\,V_{agua,m,2}
                    + x_{\ch{MeOH},2}\,V_{\ch{MeOH},m,2}
                \right)
                x_{\ch{MeOH},1}
                \,V_{sol.1}
            }{
                x_{\ch{MeOH},2}
                \left(
                    x_{\ch{MeOH},1}\,V_{\ch{MeOH},m,1}
                    +(1-x_{\ch{MeOH},1})\,V_{agua,m,1}
                \right)
            }
            \cong &\\&
            \cong \frac{
                \left(
                      0.8*(18.01-1) * 10^{-6}
                    + 0.2*(40.45-4) * 10^{-6}
                \right)
                0.4*250*10^{-6}
            }{
                0.2
                \left(
                    0.4*(40.45-1.8)  * 10^{-6}
                    +0.6*(18.01-1.8) * 10^{-6}
                \right)
            }
            \cong &\\&
            \cong
            \num{414.873342333042166e-6}
        &
    \end{flalign*}

\end{questionBox}

\begin{questionBox}2{ % Q2,3
    Defina o volume parcial mola de metanol numa mistura (metanol + água)
} % Q2,3
    
    \begin{flalign*}
        &
            V_{\ch{MeOH},m}
            =\frac{V_{sol}-(1-x_{\ch{MeOH}})\,V_{agua,m}}{x_{\ch{MeOH}}}
        &
    \end{flalign*}

\end{questionBox}

\end{document}
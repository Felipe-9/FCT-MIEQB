% !TEX root = ./TEQB-Testes_Resoluções.5.tex
\providecommand\mainfilename{"./TEQB-Testes_Resoluções.tex"}
\providecommand \subfilename{}
\renewcommand   \subfilename{"./TEQB-Testes_Resoluções.5.tex"}
\documentclass[\mainfilename]{subfiles}

% \tikzset{external/force remake=true} % - remake all

\begin{document}

% \graphicspath{{\subfix{../images/}}}

\mymakesubfile{1}
[TEQB]
{Reslução Exame}
{Reslução Exame}

\sisetup{
    % scientific / engineering / input / fixed
    exponent-mode           = fixed,
    exponent-to-prefix      = false,          % 1000 g -> 1 kg
    % exponent-product        = *,             % x * 10^y
    % fixed-exponent          = 0,
    round-mode              = places,        % figures/places/unsertanty/none
    round-precision         = 2,
    % round-minimum           = 0.01, % <x => 0
    % output-exponent-marker  = {\,\mathrm{E}},
}

\begin{questionBox}1{ % Q1
    1\,\unit{\mole} gas perfeito num processo reversivel
} % Q1
    \begin{enumerate}
        \begin{multicols}{4}
            \item 2.1\,\unit{\bar}, 19\,\unit{\celsius}
            \item 1.2\,\unit{\bar}
            \item 0.86\,\unit{\bar}
            \item .
        \end{multicols}
    \end{enumerate}

    \begin{itemize}
        \begin{multicols}{2}
            \item \(\adif{H}_{1\to2} = 844\)
            \item 2\to3: Adiabático \(P\,V^\gamma = Cte\)
            \item \(W_{3\to4} = -1671\)
            \item 3\to4: T constante
        \end{multicols}
    \end{itemize}

\end{questionBox}

\begin{questionBox}2{ % Q1.1
    T em 2
} % Q1.1
    \begin{flalign*}
        &
            T_2
            = T_1 + \frac{\adif{H}_{1\to2}}{n\,C_{P}}
            = T_1 + \frac{\adif{H}_{1\to2}}{n\,(C_{V}+R)}
            = (19+273.15) 
            + \frac{844}{1*3.5*\num{8.314462618}}
            \cong
            \num{321.152819330837617}
        &
    \end{flalign*}
\end{questionBox}

\begin{questionBox}2{ % Q1.2
    T em 3
} % Q1.2
    \begin{flalign*}
        &
            P_3\,V_3^{\gamma}
            = P_3
            \,\left(
                \frac{n\,R\,T_3}{P_3}
            \right)^{1.4}
            = \frac{n^{1.4}\,R^{1.4}\,T_3^{1.4}}{P_3^{0.4}}
            =P_2\,V_2^{\gamma}
            = \frac{n^{1.4}\,R^{1.4}\,T_2^{1.4}}{P_2^{0.4}}
            \implies &\\&
            \implies
            T_3
            = \left(
                \frac{
                    n^{1.4}\,R^{1.4}\,T_2^{1.4}
                    \,P_3^{0.4}
                }{
                    P_2^{0.4}
                    \,n^{1.4}\,R^{1.4}
                }
            \right)^{1/1.4}
            = T_2\left(
                \frac{P_3}{P_2}
            \right)^{0.4/1.4}
            \cong \num{321.152819330837617}
            \,\left(
                \frac{0.86}{1.2}
            \right)^{0.4/1.4}
            \cong
            \num{291.99391264520781}
        &
    \end{flalign*}
\end{questionBox}

\begin{questionBox}2{ % Q1.3
    P em 4
} % Q1.3
    \begin{flalign*}
        &
            -\int{n\,R\,T\,\odif{V}/V}
            = -n\,R\,T_3\,\ln(V_4/V_3)
            = -n\,R\,T_3\,\ln\left(
                \frac{n\,R\,T_4}{P_4}
                - \frac{n\,R\,T_3}{P_3}
            \right)
            = W_{3\to4}
            \implies &\\&
            \implies
            P_4
            = \left(
                \left(
                    \exp\left(
                        -\frac{W_{3\to4}}{n\,R\,T_3}
                    \right)
                    + \frac{n\,R\,T_3}{P_3}
                \right)
                \frac{1}{n\,R\,T_4}
            \right)^{-1}
            = &\\&
            = \left(
                \exp\left(
                    -\frac{
                        -1671
                    }{
                        1
                        *\num{8.314462618}
                        *\num{291.99391264520781}
                    }
                \right)
                % 1.990299753413374
                + \frac{
                    1
                    *\num{8.314462618}
                    *\num{291.99391264520781}
                }{
                    0.86*10^5
                }
                % 0.028229912457816
            \right)^{-1}
            % 0.495410108113721
            1
            *\num{8.314462618}
            *\num{291.99391264520781}
            \cong
            \num{1202.743022517987414}
            % 0.01202743022518
        &
    \end{flalign*}
\end{questionBox}

\begin{questionBox}1{ % Q2
    Sistema n-octano+entanol
} % Q2
\end{questionBox}

\begin{questionBox}2{ % Q2.1
    Calcule \(\gamma_{etanol,I}\) quando \(x_{n-octano} = 0.65\) a 75\,\unit{\celsius}
} % Q2.1
    \begin{flalign*}
        &
            \gamma_{etanol,I}
        &
    \end{flalign*}
\end{questionBox}

\begin{questionBox}1{ % Q3
    um pouco acima do ponto triplo (175.5\,\unit{\kelvin},\(1.86*10^{-6}\)\,\unit{\bar})
} % Q3

    \begin{BM}
        P_{fus} = -1077.3 + 6.1309\,T
        \\\gtrapprox (175.5\,\unit{\kelvin},1.86*10^{-6}\,\unit{\bar}) 
        \land P<500\,\unit{\bar}
    \end{BM}

    \begin{itemize}
        \begin{multicols}{2}
            \item \(V_{m,L} = 35.43\,\unit{\centi\metre^3\,\mole^{-1}}\)
            \item \(V_{m,S} = 32.47\,\unit{\centi\metre^3\,\mole^{-1}}\)
            \item \(P_{vap,L,215\,\unit{\kelvin}}=3.44*10^{-4}\,\unit{\bar}\)
            \item \(C_{P,s} = 52.0 \,\unit{\joule\,\kelvin^{-1}\,\mole^{-1}}\)
            \item \(C_{P,l} = 70.8 \,\unit{\joule\,\kelvin^{-1}\,\mole^{-1}}\)
            \item \(C_{P,g} = 39.0 \,\unit{\joule\,\kelvin^{-1}\,\mole^{-1}}\)
        \end{multicols}
    \end{itemize}
\end{questionBox}

\begin{questionBox}2{ % Q3.1
    Calcule P de sub a 173\,\unit{\kelvin}
} % Q3.1
    \begin{flalign*}
        &
            P_{sub}
            = (-1077.3+6.1309*T_t)
            \,\exp\left(
                \frac{-\adif{H}_{sub}}{R}
                (T_1^{-1}-T_0^{-1})
            \right)
            = &\\&
            = (-1077.3+6.1309*T_t)
            \exp\left(
                - \frac{T_1^{-1}-T_t^{-1}}{R}
                (\adif{H}_{fus}+\adif{H}_{vap})
            \right)
            &\\[3ex]&
            % -------------------------------- adif H fus -------------------------------- %
            \adif{H}_{fus}
            = T_t\,\adif{V}_{fus}\pdv{P}{T}_{fus}
            = T_t\,(V_1-V_0)\,\pdv{P}{T}_{fus}
            = &\\&
            = T_t\,(V_1-V_0)
            \,\pdv{}{T}\left(
                -1077.3 + 6.1309\,T_t
            \right)
            *10^6
            = &\\&
            = T_t\,(V_1-V_0)
            *6.1309
            *10^6
            = &\\&
            = 175.5
            *(35.43-32.47)*10^{-6}
            * 6.1309*10^6
            \cong
            \num{3184.879932}
            &\\[3ex]&
            % ----------------------------------- P vap ---------------------------------- %
            \ln{P_{vap,L}} = a + b/T
            \implies
            \ln{3.44*10} - b/215 = \ln{1.86*10^{-1}} - b/175.5
            \implies &\\&
            \implies
            b = \frac{
                \ln{(34.4/0.186)}
                % 5.220065169648288
            }{
                175.5^{-1}-215^{-1}
                % 0.001046842907308
            }
            \cong
            \num{4986.483772500214754}
            \implies &\\&
            \implies
            a \cong 
            3.44*10-\num{4986.483772500214754}/215
            \cong
            \num{11.207052220929234}
            \implies &\\&
            \implies
            \ln{P_{vap}} \cong \num{11.207052220929234} + \num{4986.483772500214754}/T
            &\\[3ex]&
            % -------------------------------- adif H vap -------------------------------- %
            \adif{H}_{vap}
            = R\,T^2\,\odv{\ln{P}}{T}
            \cong
            R\,T^2\,(\num{4986.483772500214754}/T_2)
            \cong \num{41459.932921716654014}
            &\\[5ex]&
            % ----------------------------------- final ---------------------------------- %
            P_{sub}
            \cong &\\&
            \cong
            (-1077.3+6.1309*175.5) % 2153.27295
            * &\\&
            * \exp\left(
                - \frac{173^{-1}-175.5^{-1}}{\num{8.314462618}} % 9.903360756627792e-6
                (\num{3184.879932}+\num{41459.932921716654014})
            \right)
            % 0.642663713541241
            \cong &\\&
            \cong
            \num{1383.830390314904}
        &
    \end{flalign*}

    \paragraph*{Nota}: Sobre a equação do P vap encontrada, esta não está correta, ao fazer a aproximação usando uma curva log em 1/T gerou valores em P muito superiores que o esperando, talves seja melhor parametrizar em alguma outra curva como uma reta.
\end{questionBox}

\begin{questionBox}2{ % Q3.2
    \(\adif{S} (180\,\unit{\kelvin},200\,\unit{\bar}) \to (300\,\unit{\kelvin},0.05\,\unit{\bar})\)
} % Q3.2
    \begin{flalign*}
        &
            \left\{
                \begin{aligned}
                    & P_{fus,180\,\unit{\kelvin}}
                    = (-1077.3+6.1309*180)*10^6
                    \cong
                    \num{262.62}*10^5>200*10^5
                    \\
                    & P_{vap,180\,\unit{\kelvin}}
                    = (
                        -\num{11.207052220929234}
                        +\num{4986.483772500214754}/180
                    )*10^5
                    \cong
                    \num{16.495635404072}*10^5>0.05*10^5
                    \\
                    & T_{fus,200\,\unit{\bar}}
                    = 200+1077.3/6.1309
                    \cong
                    \num{375.716452723091226}>300
                    \\
                    & T_{vap,200\,\unit{\bar}}
                    = \frac{4986.48}{\ln(200*10^5)-11.21}
                    \cong
                    \num{890.245281268830771}
                \end{aligned}
            \right\}
            &\\[3ex]&
            \adif{S}
            = &\\&
            = \left(
                \begin{aligned}
                    &
                        \adif{S}(l,180\,\unit{\kelvin},(200\to\num{16.495635404072})\,\unit{\bar})
                    &+\\+&
                        \adif{S}((l\to g),180\,\unit{\kelvin},\num{16.495635404072}\,\unit{\bar})
                    &+\\+&
                        \adif{S}(g,(180\to300)\,\unit{\kelvin},(\num{16.495635404072}\to0.05)\,\unit{\bar})
                    &
                \end{aligned}
            \right)
            = &\\&
            = \left(
                \begin{aligned}
                    &
                        0
                    &+\\+&
                        -n\,\adif{H}_{vap}/180
                    &+\\+&
                        \int{n\,C_{P,g}\,\odif{T}/T}
                        - n\,R\,\ln(P_1/P_0)
                    &
                \end{aligned}
            \right)
            = &\\&
            = \left(
                \begin{aligned}
                    &
                        -\num{41459.932921716654014}/180
                        % 230.332960676203611
                    &+\\+&
                        39*\ln{300/175.5}
                        - \num{8.314462618}
                        *\ln(0.05/16.5)
                        % 20.884398496994275
                    &
                \end{aligned}
            \right)
            \cong &\\&
            \cong
            \num{-209.448562179209325}
        &
    \end{flalign*}
\end{questionBox}

\begin{questionBox}2{ % Q3.3
    \(
        \adif{G} 
        (300\,\unit{\kelvin,10\,\unit{\bar}})
        \to
        (300\,\unit{\kelvin,0.05\,\unit{\bar}})
    \)
} % Q3.3
    \begin{flalign*}
        &
            \left\{
                \begin{aligned}
                    & P_{fus,300\,\unit{\kelvin}} 
                    = (1077.3 + 5.1309*300)*10^6
                    \cong
                    \num{2616.57}*10^6
                    >10*10^5
                \end{aligned}
            \right\}
            &\\[3ex]&
            % ---------------------------------- adif G ---------------------------------- %
            \adif{G}
            = &\\&
            = \left(
                \begin{aligned}
                    &
                        \adif{G}(s,(10\to P_fus)\,\unit{\bar})
                    &+\\+&
                        \adif{G}((s\to l),P_{fus}\,\unit{\bar})
                    &+\\+&
                        \adif{G}(l,(P_{fus}\to P_{vap})\,\unit{\bar})
                    &
                \end{aligned}
            \right)
            = &\\&
            = \left(
                \begin{aligned}
                    &
                        \int_{P_1}^{P_2}{V\,\odif{P}}
                    &+\\+&
                        0
                    &+\\+&
                        \int_{P_0}^{P_1}{V\,\odif{P}}
                    &
                \end{aligned}
            \right)
            = &\\&
            = \left(
                \begin{aligned}
                    &
                    V\,\int_{P_1}^{P_2}{\odif{p}}
                    \quad(\text{vol liq constante em} \adif{P})
                    &+\\+&
                        \int_{P_0}^{P_1}{
                            \frac{n\,R\,T}{P}
                            \,\odif{P}
                        }
                    &
                \end{aligned}
            \right)
            = &\\&
            = \left(
                \begin{aligned}
                    &
                        V_{l}\,(P_2-P_1)
                    &+\\+&
                        \frac{n\,R\,T}
                        \ln(P_3/P_2)
                    &
                \end{aligned}
            \right)
        &
    \end{flalign*}

    \paragraph*{Nota:} Continuo a sofrer com a ma parametrização da curva de P de vaporização
\end{questionBox}

\begin{questionBox}1{ % Q4
} % Q4
    \begin{itemize}
        \begin{multicols}{2}
            \item \(P_{vap,sol,30\,\unit{\celsius}} = 0.1026\,\unit{\bar}\)
            \item \(\ln{P_{vap,eta} = 14.502-5084.7/T}\)
            \item \(M_{eta} =  46.07\,\unit{\gram\,\mole^{-1}}\)
            \item \(M_{naf} = 128.17\,\unit{\gram\,\mole^{-1}}\)
            \item \(\rho_{sol} \approx \rho_{eta} = 0.789\,\unit{\gram\,\centi\metre^{-1}}\)
        \end{multicols}
    \end{itemize}

\end{questionBox}

\begin{questionBox}3{ % Q4.1
    Pressão osmótica da sol a 20 C
} % Q4.1
    \begin{flalign*}
        &
            \Pi
            = \frac{[i]\,P_{eta}}{n_{eta}}
        &
    \end{flalign*}
\end{questionBox}

\begin{questionBox}3{ % Q4.2
    T de vap sol a 0.951\,\unit{\bar}
} % Q4.2
    \begin{flalign*}
        &
            t_{vap,sol}
            = \left(
                t_{vap,eta}^{-1}
                -\ln{x_{eta}}
                \,\frac{R}{\adif{H}_{vap,eta}}
            \right)^{-1}
            = &\\&
            = \left(
                \left(
                    \frac{5084.7}{14.502-\ln{P}}
                \right)^{-1}
                -\ln(x_{eta})
                \,\frac{R}{\adif{H}_{vap,eta}}
            \right)^{-1}
            = &\\&
            = \left(
                \frac{14.502-\ln{P}}{5084.7}
                -\frac{\ln(x_{eta})\,R}{\adif{H}_{vap,eta}}
            \right)^{-1}
            ; &\\[3ex]&
            % -------------------------------- adif H vap -------------------------------- %
            \odv{P}{T}_{vap}
            = \frac{\adif{H}_{vap}}{T\,\adif{V}_{vap}}
            = \frac{\adif{H_{vap}}}{T\,(V_{vap,g}-V_{vap\,l})}
            \cong \frac{\adif{H_{vap}}}{T\,V_{vap,g}}
            \cong \frac{\adif{H_{vap}}}{T\,\left(
                R\,T/P
            \right)}
            =
            \frac{P\,\adif{H_{vap}}}{R\,T^2}
            \implies &\\&
            \implies
            \frac{\odif{P}/P}{\odif{T}}
            = \frac{\odif{\ln{P}}}{\odif{T}}
            = \odv{}{T}\left(
                14.502-5084.7\,T^{-1}
            \right)
            = 5084.7\,T^{-2}
            = \frac{\adif{H_{vap}}}{R\,T^2}
            \implies &\\&
            \implies
            \adif{H}_{vap} = 5084.7\,R
            % ----------------------------------- Final ---------------------------------- %
            t_{vap,sol}
            = \left(
                \frac{14.502-\ln{P}}{5084.7}
                -\frac{\ln(x_{eta})\,R}{5084.7\,R}
            \right)^{-1}
            = &\\&
            = 
            \frac{5084.7}
            {14.502-\ln{(0.951*x_{eta})}}
            \cong
            % \num{349.410095969054907}
        &
    \end{flalign*}
\end{questionBox}

\begin{questionBox}1{ % Q5
} % Q5
    \begin{itemize}
        \item \(T = 30\,\unit{\celsius}\)
        \item \(n_{oct,tot} = 3\,\unit{\mole}\)
        \item \(n_{agua,tot} = 2\,\unit{\mole}\)
        \item \(x_{oct,1} = 0.744\)
        \item \(x_{oct,2} = 0.002\)
    \end{itemize}
\end{questionBox}

\begin{questionBox}2{ % Q5.1
    Razão entre as quantidades (mol) da fase mais rica em n-octano e da fase mais rica em agua
} % Q5.1
    \begin{questionBox}3{ % Q5.1.1
        Fase mais rica em n-Octano
    } % Q5.1.1
        % \begin{flalign*}
        %     &
                
        %     &
        % \end{flalign*}
    \end{questionBox}
\end{questionBox}


\begin{minipage}{\textwidth}
    \phantom{1}
    \vspace{10cm}
\end{minipage}

\end{document}
% !TEX root = ./CM_A-Trabalho.1.tex
\providecommand\mainfilename{"./CM_A-Trabalho.tex"}
\providecommand \subfilename{}
\renewcommand   \subfilename{"./CM_A-Trabalho.1.tex"}
\documentclass[\mainfilename]{subfiles}

% \tikzset{external/force remake=true} % - remake all

\begin{document}

\graphicspath{{\subfix{./.build/figures/CM_A-Trabalho.1}}}
% \tikzsetexternalprefix{./.build/figures/CM_A-Trabalho.1/graphics/}

\mymakesubfile{1}
[CM\,A]
{Introdução} % Subfile Title
{Introdução} % Part Title

No nosso dia-a-dia contactamos várias vezes com polímeros ainda que, em caso geral, nem sequer estejamos completamente conscientes de que aquele objeto com o qual estamos a interagir seja polimérico, ou a que polímero se devem as suas características. De entre estes, um dos mais comuns é a Esponja. De entre o leque de componentes, um dos que é mais relevante para o mundo da Engenharia Química é o Poliuretano, material presente nas esponjas, frigoríficos, carros, ténis, entre outros.\par

O Poliuretano é um Polímero Sintético, criado pelo Professor Otto Bayer (Químico Alemão) em 1937. Como o nome indica, é constituído por várias ``Poli'' unidades do monómero ``Uretano'' ou Carbamato, ligados entre si por ligações covalentes. O Uretano tem a fórmula química: \ch{C3H7NO2}, de acordo com a configuração que podemos observar na figura abaixo:\par

\begin{wrapfigure}{l}{.37\textwidth}
    \includegraphics[width=.4\textwidth]{uretano.png}
\end{wrapfigure}
O Poliuretano é um Elastómero extremamente versátil, dado que pode assumir propriedades de Elastómeros Termoplásticos (\textit{TPEs}) ou Elastómeros Termoendurecíveis (\textit{TSEs}), dependendo de qual for a metodologia de processamento.\par

O facto de ser um Elastómero garante-lhe a tendência de voltar à sua forma original depois ser deformado (característica mais comummente associada à borracha). Quando a fonte de tensão é removida do sistema, o material tende a reverter a deformação sofrida, re-configurando as suas cadeias de monómeros para voltar ao estado inicial. Deste modo, são compostos que se comportam como mola e cujo comportamento pode ser previsto usando o Modelo do Oscilador Harmónico (VERIFICAR).\par

O Poliuretano Termoendurecível é usado para a criação de materiais que sacrificam essa propriedade para em troca obter uma grande durabilidade, dureza e rigidez. São também materiais que têm elevada resistência térmica, subindo o ponto de fusão consideravelmente, resultando na capacidade de aguentar temperaturas de ordem elevada (VALOR) e elevada resistência a fontes abrasivas, sendo por exemplo usado frequentemente na construção de edifícios mais eco-friendly (que será desenvolvido mais à frente na secção da \emph{Utilidade}).\par

Estas características advêm da capacidade dos polímeros termoendurecíveis formarem redes tridimensionais. A formação destas redes é promovida pelo processo de cura, mas mais notoriamente conhecido como cross-linking. A forma original destes polímeros são resinas líquidas, que após serem aquecidas são sujeitas a Processamento de Feixe de Elétrões ou Electron Beam Processing (EBP). Este método consiste no disparo de eletrões, que colidem com átomos de hidrogénio, libertando-os da cadeira, o que possibilita a formação de ligações covalentes entre, no caso do Poliuretano, os átomos de carbono de cadeias adjacentes. Deste modo, o Poliuretano adquire uma estrutura definida e resistente, com a tridimensionalidade referida acima. Porém, por muito que seja sequencialmente aquecido e arrefecido, nunca poderá voltar ao seu estado original, dado que se degrada e torna inutilizável. Ainda que materiais com estas características tenham um tempo de vida substancial, não são a melhor escolha de uma perspectiva ambiental, visto não serem recicláveis.\par

O Poliuretano Termoplástico não tem as propriedades do Termoendurecível, não significando porém que terá menos utilidades. Ao não sacrificar a capacidade de retornar à sua forma original, é extremamente útil, e muito relevante na nossa rotina! As esponjas para lavar a loiça são uma ferramenta indispensável, mas nem sempre nos perguntamos ``porque será que as esponjas conseguem sempre retornar à sua forma original?''. Primeiramente, termoplásticos são constituídos por stacking de cadeias lineares, sem fenómenos de cross-linking. As cadeias não se ligam por ligações covalentes, não obstante de poderem interagir por Forças de Van de Walls ou Pontes de Hidrogénio, tornando-se extremamente móveis e organizando-se de modo a formar estruturas cristalinas, ao contrário dos termoendurecíveis. Adicionalmente, estes polímeros podem ser sequencialmente aquecidos e arrefecidos, de modo a retornar ao seu formato original, sendo portanto recicláveis.




\end{document}
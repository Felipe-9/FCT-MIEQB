% !TEX root = ./CM_A-Slides_Annotations.2.0.tex
\providecommand\mainfilename{"./CM_A-Slides_Annotations.tex"}
\providecommand \subfilename{}
\renewcommand   \subfilename{"./CM_A-Slides_Annotations.2.0.tex"}
\documentclass[\mainfilename]{subfiles}

% \tikzset{external/force remake=true} % - remake all

\begin{document}

% \graphicspath{{\subfix{./.build/figures/CM_A-Slides_Annotations.2.0}}}
% \tikzsetexternalprefix{./.build/figures/CM_A-Slides_Annotations.2.0/graphics/}

\mymakesubfile{0}
[CM\,A]
{Anotações Aleatórias} % Subfile Title
{Anotações Aleatórias} % Part Title

\begin{sectionBox}1{Equação de Mark-Houwink-Sakurada} % S
    
    \begin{BM}
        \left[\eta\right]
        =K\,M_n^a
        \qquad
        \ln\eta=\ln K+a\,\ln M_n
    \end{BM}
    \begin{description}[
        leftmargin=!,
        labelwidth=\widthof{} % Longest item
    ]
        \item[\(\{K,a\}\)] Constantes de Mark-Houwink - Válidas para pares polímero/solvente a uma dada T
        \item[\(0<a<1\)] Relacionado com o tipo conformacional do polímero em solução
        \item[\(K\)] Relacionado com a geometria local da cadeia
        \item[\(a = 0.5\)] Solvente \chemtheta
        \item[\(a\cong0.5\)] comportamento em hélice
        \item[\(a\cong1.0\)] comportamento em bastonete
    \end{description}
    
\end{sectionBox}

\begin{sectionBox}1{Fatores que afetam a cristalinidade} % S
    
    \paragraph*{Simetria:} ++ cadeias simétricas permitem um empacotamento regular
    \paragraph*{Ligações intermoleculares} ++ cadeias com grupos que favoreçam atracções entre elas
    \paragraph*{Peso molecular} ++
    \paragraph*{Ramificações} -- diminuem a possibilidade de empacotamento devido a impedimento estereoquímico
    \paragraph*{Taticidade} -- polímeros isotáticos são mais cristalinos
    
\end{sectionBox}

\begin{sectionBox}1{Caracterização térmica dos polímeros} % S
    
    T baixa: polimeros sólidos
    T almenta provocando sussessivamente
    \begin{itemize}
        \item Rotação dos grupos substituintes \(T_{\beta}\)
        \item Rotação de grupos de átomos em torno do seu eixo à temperatura de transição vítrea Tg
        \item Movimentos das cadeias
        \item Comportamento de líquido viscoso a partir da temperatura de fusão Tm
    \end{itemize}
    
\end{sectionBox}

\end{document}
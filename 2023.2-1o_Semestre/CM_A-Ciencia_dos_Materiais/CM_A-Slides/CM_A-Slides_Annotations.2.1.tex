% !TEX root = ./CM_A-Slides_Annotations.2.1.tex
\providecommand\mainfilename{"./CM_A-Slides_Annotations.tex"}
\providecommand \subfilename{}
\renewcommand   \subfilename{"./CM_A-Slides_Annotations.2.1.tex"}
\documentclass[\mainfilename]{subfiles}

% \tikzset{external/force remake=true} % - remake all

\begin{document}

\graphicspath{{\subfix{./.build/figures/CM_A-Slides_Annotations.2.1}}}
% \tikzsetexternalprefix{./.build/figures/CM_A-Slides_Annotations.2.1/graphics/}

\mymakesubfile{1}
[CM\,A]
{Polímeros da vida quotidiana} % Subfile Title
{Polímeros da vida quotidiana} % Part Title

\begin{sectionBox}1{Lista de plásticos} % S
    
    \paragraph*{Plastico:} Tipo de polímero que tem a capacidade de ser coldado de diversas fórmas.

    \begin{center}
        \vspace{1ex}
        \begin{tabular}{l l}
            \toprule
            
               PE-LD & Polietileno de baixa densidade, 
            \\ PE-HD & Polietileno de alta densidade
            \\ PP    & Polipropileno 
            \\ PVC   & Policloreto de vinilo 
            \\ PS    & Poliestireno sólido
            \\ PS-E  & Poliestireno expandido 
            \\ PET   & Polietileno tereftalato 
            \\ PUR   & Poliuretano 
            
            \\\bottomrule
        \end{tabular}
        \vspace{2ex}
    \end{center}
    
\end{sectionBox}

\begin{sectionBox}1{Caracterização} % S
    
    \begin{description}[
        leftmargin=!,
        labelwidth=\widthof{Copolímero:} % Longest item
    ]
        \item[Polímero:] Conjunto de unidades moleculars (monómeros) ligadas covalentemente entre si
        \item[Copolímero:] Polímero formado por diferentes monomeros (A e B)
        \item[Monomero:] Unidade repetitiva da cadeia de poímeros
    \end{description}

    \subsection{Caracterização por natureza:}
    \begin{multicols}{2}
        \paragraph*{Naturais:} Não produzidos pelo homem, como: Seda, Lã, Ambar.
        \paragraph*{Sintéticos:} usados na produção de plástico, silicone, borracha.
    \end{multicols}

    \subsection{Caracterização por Cadeias:}
    \begin{sectionBox}*2m{Homopolímero} % S
        \begin{figure}\centering
            \includegraphics[width=.5\textwidth]{homo.png}
        \end{figure}
    \end{sectionBox}
    \begin{multicols}{2}
        \begin{sectionBox}*2m{Copolímero aleatório} % S
            \begin{figure}\centering
                \includegraphics[width=1\textwidth]{copoale.png}
            \end{figure}
        \end{sectionBox}
        \begin{sectionBox}*2m{Copolímero alternado} % S
            \begin{figure}\centering
                \includegraphics[width=1\textwidth]{copoalter.png}
            \end{figure}
        \end{sectionBox}
        \begin{sectionBox}*2m{Copolímero de blocos} % S
            \begin{figure}\centering
                \includegraphics[width=1\textwidth]{copoblock.png}
            \end{figure}
        \end{sectionBox}
        \begin{sectionBox}*2m{Copolímero ramificado} % S
            \begin{figure}\centering
                \includegraphics[width=1\textwidth]{coporamif.png}
            \end{figure}
        \end{sectionBox}
    \end{multicols}

    \paragraph*{Propriedades:}

    \begin{description}[
        leftmargin=!,
        % labelwidth=\widthof{} % Longest item
    ]
        \vspace{-3ex}
        \begin{multicols}{2}
            \item[Copolímero aleatório:] Propriedades \emph{intermédias} entre os homopolímeros A e B
            \item[Copolímero de blocos:] Depende do tamanho das sequencias A e B
            \item[Copolímero alternado:] Propriedades \emph{\pm{} intermédias} entre os homopolímeros A e B
            \item[Copolímero ramificado:] Modificação das propriedades do principal
        \end{multicols}
    \end{description}
\end{sectionBox}
    
    
\begin{sectionBox}1{Polímeros para lembrar} % S
    
    \paragraph*{Monomero:} Etoleno \chemfig{CH2=CH2}
    \vspace{1ex}
    \paragraph*{Polímero:}
    \chemfig{-[@{op}]
        C(-[90,.5])(-[-90,.5])
        -C(-[90,.5])(-[-90,.5])
    -[@{cl}]}
    \polymerdelim{op}{cl}
    \begin{center}
        \vspace{1ex}
        \begin{tabular}{l l}
            \toprule
            
                \multicolumn{1}{c}{X}
            
            \\\midrule
            
                \chemfig{-H} & Polietileno
                \\ \chemfig{-CH3} & Polipropuleno
                \\ \chemfig{-Cl} & Policloreto de vinilo
                \\ \chemfig[atom sep=1.5ex]{*6(=-=-=-)} 
                & Poliestireno
            
            \\\bottomrule
        \end{tabular}
        \vspace{2ex}
    \end{center}
    
\end{sectionBox}

\begin{sectionBox}1{Nomenclatura} % S
    
    \subsection{IUPAC}
    \begin{center}\large
        Poli/Poly\\+\\(<designação IUPAC do monomero com estrutura exata>)
    \end{center}
    \paragraph*{Ex:} \iupac{Poly (oxyspiro[3.5]nona-2,5-dien-7,1-ylene-4-cyclohexen-1,3-ylene)}
    \subsection{Nomenclatura geral}
    \begin{center}\large
        Poli/poly + (<monomero>)/<monomero>
    \end{center}
    \paragraph*{Ex.}
    \begin{itemize}
        \begin{multicols}{3 }
            \item Poli(propileno) (PP)
            \item Polipropileno (PP)
            \item Polietileno (PE)
        \end{multicols}
    \end{itemize}

    \subsection{Nomes comerciais}
    \begin{center}
        \vspace{1ex}
        \begin{tabular}{l l}
            \toprule
            
                \multicolumn{1}{c}{Comercial}
                & \multicolumn{1}{c}{IUPAC}
            
            \\\midrule
            
               NYLON    & poliamida
            \\ TEFLON   & politetrafluoretileno
            \\ TERYLENE & poliéster
            \\ PLEXIGLÁS& polimetacrilato de metilo
            \\ KEVLAR   & poliaramida
            
            \\\bottomrule
        \end{tabular}
        \vspace{2ex}
    \end{center}
    
\end{sectionBox}


\end{document}
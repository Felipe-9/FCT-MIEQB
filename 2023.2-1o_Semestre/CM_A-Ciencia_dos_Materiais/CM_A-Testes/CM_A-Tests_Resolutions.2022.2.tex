% !TEX root = ./CM_A-Tests_Resolutions.2022.2.tex
\providecommand\mainfilename{"./CM_A-Tests_Resolutions.tex"}
\providecommand \subfilename{}
\renewcommand   \subfilename{"./CM_A-Tests_Resolutions.2022.2.tex"}
\documentclass[\mainfilename]{subfiles}

% \tikzset{external/force remake=true} % - remake all

\begin{document}

% \graphicspath{{\subfix{./.build/figures/CM_A-Tests_Resolutions.2022.2}}}
% \tikzsetexternalprefix{./.build/figures/CM_A-Tests_Resolutions.2022.2/graphics/}

\mymakesubfile{2}
[CM\,A]
{Test 2022 Resolution} % Subfile Title
{Test 2022 Resolution} % Part Title

\part*{Teoric (T/F):}

\def\AT{\textcolor{green!40!foreground}{True}}
\def\AF{\textcolor{red!40!foreground}{False}}

\begin{questionBox}1{ % Q1
    Um polímero é um plástico, mas nem todos os plásticos são polímeros.
} % Q1
    \answer{}
    \AF{}, All plastics are polymers
\end{questionBox}
\begin{questionBox}1{ % Q2
    Um polímero é um conjunto de unidades moleculares ligadas covalentemente entre si.
} % Q2
    \answer{}
    \AT{}, Literalmente definição de polímeros
\end{questionBox}
\begin{questionBox}1{ % Q3
    A vulcanização da borracha corresponde à reticulação das suas cadeias com enxofre.
} % Q3
    \answer{}
    \AT{}, Descoberta por charles Goodyear 1830
\end{questionBox}
\begin{questionBox}1{ % Q4
    Um polímero do tipo AABAABAB é um copolímero aleatório.
} % Q4
    \answer{}
    \AT{}
\end{questionBox}
\begin{questionBox}1{ % Q5
    O grupo funcional característico de do polipropileno é um anel benzénico.
} % Q5
    \answer{}
    \AF{}, Propileno o grupo funcional é \ch{CH3}
\end{questionBox}
\begin{questionBox}1{ % Q6
    Um polímero sintético termoendurecível não pode ser reciclado. 
} % Q6
    \answer{}
    \AT{}, Polímeros fortemente reticulados, sem
    temperatura de fusão definida, porque a essa
    temperatura se degradam por quebra da
    reticulação.
\end{questionBox}
\begin{questionBox}1{ % Q7
    O nosso cabelo, unhas e músculos não são constituídos por polímeros de origem natural.
} % Q7
    \answer{}
    \AF{}, Constituidos por Queratina
\end{questionBox}
\begin{questionBox}1{ % Q8
    O polietileno, policloreto de vinilo e poliestireno são exemplos de materiais termoplásticos.
} % Q8
    \answer{}
    \AT{}
\end{questionBox}
\begin{questionBox}1{ % Q9
    O preço e a disponibilidade dos materiais termoplásticos está directamente relacionada com o preço do petróleo. 
} % Q9
    \answer{}
    \AT{}, A marioria dos plásticos são derivados do petróleo
\end{questionBox}
\begin{questionBox}1{ % Q10
    O teste de fio de cobre permite identificar materiais que contenham chumbo.
} % Q10
    \answer{}
    \AF{}, Identifica materias que contém cloro
\end{questionBox}
\begin{questionBox}1{ % Q11
    Os testes físico-químicos permitem identificar rigorosamente materiais poliméricos. 
} % Q11
    \answer{}
    \AF{}, Identificação rigorosa feita por:
    \begin{itemize}
        \item Espectrometria de infravermelho (\emph{FTIR})
        \item Ressonância magnética nuclear (\emph{RMN})
    \end{itemize}
\end{questionBox}
\begin{questionBox}1{ % Q12
    A massa molecular de um polímero pode ser definida pelo produto da massa molecular do monómero com o número de monómeros da cadeia.
} % Q12
    \answer{}
    \AT{}, Polimeros são sequencias de monomeros
\end{questionBox}
\begin{questionBox}1{ % Q13
    A reação de polimerização é um fenómeno aleatório sendo esta a razão pela qual não se definem massas moleculares médias.
} % Q13
    \answer{}
    \AF{},
\end{questionBox}
\begin{questionBox}1{ % Q14
    As propriedades dos polímeros dependem do seu peso molecular. 
} % Q14
    \answer{}
    \AT{}
\end{questionBox}
\begin{questionBox}1{ % Q15
    Quanto mais perto de 1 for o índice de polidispersividade de um polímero mais homogéneo será o material.
} % Q15
    \answer{}
    \AT{}, A razão entre \(M_w\text{ e }M_n\) é então uma medida da largura de distribuição, ou seja quanto mais
    afastado de 1 mais larga é a distribuição de peso molecular ou mais heterogéneo é o polímero.

    \begin{BM}
        \alpha=\frac{M_w}{M_n}
        ; \\[3ex]
        M_n
        = M_0\,x_n
        = \frac
            {\sum{M_0\,x_i\,N_i}}
            {\sum{N_i}}
        = \frac
            {\sum{M_i\,N_i}}
            {\sum{N_i}}
        ; \\[3ex]
        M_w
        = M_0\,x_w
        = \sum{M_0\,x_i\,\frac{w_i}{\sum{w_i}}}
        = \sum{M_i\,\frac{w_i}{\sum{w_i}}}
        = \\
        = \sum{M_i\,\frac{M_i\,N_i}{\sum{M_i\,N_i}}}
        = \sum{\frac{M_i^2\,N_i}{\sum{M_i\,N_i}}}
    \end{BM}
    \begin{description}[
        leftmargin=!,
        labelwidth=\widthof{\(M_W\)} % Longest item
    ]
        \item[\(\alpha\)] Indice de polidispersividade
        \item[\(M_0\)] massa molecular do monómero
        \item[\(N\)] número de cadeias moleculares
        \item[\(M_W\)] Massa molecular média ponderal 
    \end{description}
\end{questionBox}
\begin{questionBox}1{ % Q16
    A técnica de cromatografica líquida de exclusão molecular para determinação da massa molecular baseia-se no tamanho das cadeias poliméricas.
} % Q16
    \answer{}
    \AT{}, Durante o fluxo de um determinado
    solvente, cadeias de tamanhos diferentes
    percorrem caminhos diferetens ao longo
    da coluna de GPC\par
    Principio: separação fisíca das cadeias
    constituintes do polímero nos seus
    diferentes tamanhos
\end{questionBox}
\begin{questionBox}1{ % Q17
    A temperatura não altera a forma espacial das cadeias poliméricas. 
} % Q17
    \AF
\end{questionBox}
\begin{questionBox}1{ % Q18
    O escoamento dos materiais ocorre quando as forças intermoleculares enfraquecem pelo aumento da temperatura.
} % Q18
    \AT
\end{questionBox}
\begin{questionBox}1{ % Q19
    O movimento das cadeias de polímeros cristalinos no estado fundido é semelhante ao que ocorre num líquido de baixo peso molecular.
} % Q19
    \AF
\end{questionBox}
\begin{questionBox}1{ % Q20
    O tempo de Kuhn está associado à passagem pela temperatura de transição vítrea do material.
} % Q20
    \AT
\end{questionBox}
\begin{questionBox}1{ % Q21
    O tempo de Kuhn é o tempo que cada cadeia leva a percorrer uma distância comparável ao seu comprimento.
} % Q21
    \AT
\end{questionBox}
\begin{questionBox}1{ % Q22
    Um termoplástico amorfo é caracterizado por uma temperatura de transição vítrea.
} % Q22
    \AT
\end{questionBox}
\begin{questionBox}1{ % Q23
    Um termoplástico semi-cristalino apresenta um ponto de fusão definido.
} % Q23
    \AF
\end{questionBox}
\begin{questionBox}1{ % Q24
    Um termoplástico cristalino é caracterizado por uma temperatura de fusão e uma temperatura de transição vítrea.
} % Q24
    \AF
\end{questionBox}
\begin{questionBox}1{ % Q25
    A temperatura de fusão não depende da história térmica do polímero.
} % Q25
    \AF
\end{questionBox}
\begin{questionBox}1{ % Q26
    A temperatura de transição vítrea é uma transição de fase de 2ª ordem. 
} % Q26
    \AT
\end{questionBox}
\begin{questionBox}1{ % Q27
    Copolímeros apresentam duas Tg's.
} % Q27
    \AT
\end{questionBox}
\begin{questionBox}1{ % Q28
    Não podemos determinar a temperatura de amolecimento através de técnicas de calorimetria.
} % Q28
    \AT
\end{questionBox}
\begin{questionBox}1{ % Q29
    O processo de cristalização de um polímero é caracterizado por duas fases.
} % Q29
    \AT
\end{questionBox}
\begin{questionBox}1{ % Q30
    O modelo das micelas explica o comportamento mecânico dos materiais.
} % Q30
    \AT
\end{questionBox}
\begin{questionBox}1{ % Q31
    O modelo das esferulites não explica o padrão de simetria radial em cruz de malta.
} % Q31
    \AF
\end{questionBox}
\begin{questionBox}1{ % Q32
    O grau de cristalidade de um polímero semicristalino não pode ser aumentado através de um recozimento.
} % Q32
    \AF
\end{questionBox}
\begin{questionBox}1{ % Q33
    Maior simetria, maior peso molecular e mais ramificações dão origem a polímeros mais cristalinos.
} % Q33
    \AF
\end{questionBox}
\begin{questionBox}1{ % Q34
    Segundo a Lei de Newton a tensão depende de deformação.
} % Q34
    \AF
\end{questionBox}
\begin{questionBox}1{ % Q35
    Um fluido Newtoniano tem uma viscosidade que depende da velocidade de deformação.
} % Q35
    \AF
\end{questionBox}
\begin{questionBox}1{ % Q36
    Segundo a Lei de Hooke a tensão é independente da velocidade de deformação.
} % Q36
    \AT
\end{questionBox}
\begin{questionBox}1{ % Q37
    A Lei de Hooke descreve a proporcionalidade entre a tensão e a deformação do material.
} % Q37
    \AT
\end{questionBox}
\begin{questionBox}1{ % Q38
    Os polímeros têm normalmente um comportamento viscoelástico, caracterizado pelos modelos de Newton e Voight-Kelvin.
} % Q38
    \AF
\end{questionBox}
\begin{questionBox}1{ % Q39
    Num ensaio de relaxação de tensão aplica- se uma tensão constante medindo-se a deformação resultante em função do tempo.
} % Q39
    \AF
\end{questionBox}
\begin{questionBox}1{ % Q40
    Um elastómero é caracterizado por um módulo de Young baixo e dependente da temperatura.
} % Q40
    \AT
\end{questionBox}


\end{document}
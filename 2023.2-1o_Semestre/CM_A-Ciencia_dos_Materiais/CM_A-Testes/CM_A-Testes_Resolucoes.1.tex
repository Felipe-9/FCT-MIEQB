% !TEX root = ./CM_A-Testes_Resolucoes.1.tex.tex
\providecommand\mainfilename{"./CM_A-Testes_Resolucoes.tex"}
\providecommand \subfilename{}
\renewcommand   \subfilename{"./CM_A-Testes_Resolucoes.1.tex.tex"}
\documentclass[\mainfilename]{subfiles}

% \tikzset{external/force remake=true} % - remake all

\begin{document}

% \graphicspath{{\subfix{./.build/figures/CM_A-Testes_Resolucoes.1.tex}}}
% \tikzsetexternalprefix{./.build/figures/CM_A-Testes_Resolucoes.1.tex/graphics/}

\mymakesubfile{1}
[CM A]
{Teste 1 Resolução Pratica} % Subfile Title
{Teste 1 Resolução Pratica} % Part Title

\begin{questionBox}1{ % Q1
    \begin{itemize}
        \item \(r=1.357\,\si{\angstrom}\)
        \item CFC
    \end{itemize}
} % Q1
    \begin{questionBox}2{ % Q1.1
        a
    } % Q1.1
        \answer{}
        \begin{flalign*}
            &
                a 
                = 4\,r\,\cos(\pi/4)
                = 4\,r\,\sqrt{2}/2
                = r\,2^{3/2}
                = 1.357*2^{3/2}
                \,\si{\angstrom}
                \cong
                \SI{3.83817560828058}{\angstrom}
            &
        \end{flalign*}
    \end{questionBox}
    \begin{questionBox}2{ % Q1.2
        ind de miller plano do 3o pico
    } % Q1.2
        \answer{}
        \begin{center}
        \vspace{1ex}
        \begin{tabular}{*{6}{C}}
            \toprule
            
                h & k & l & N & CFC & 
            
            \\\midrule
            
                   1 & 0 & 0 &  1 & 0 &
                \\ 1 & 1 & 0 &  2 & 0 &
                \\ 1 & 1 & 1 &  3 & 1 & 1
                \\ 2 & 0 & 0 &  4 & 1 & 1
                \\ 2 & 1 & 0 &  5 & 0 &
                \\ 2 & 1 & 1 &  6 & 0 &
                \\ 2 & 2 & 0 &  6 & 1 & 1
            
            \\\bottomrule
        \end{tabular}
        \vspace{2ex}
        \end{center}
    \end{questionBox}
    \begin{questionBox}2{ % Q1.3
        Valor do ang de diff \(k_{\alpha\,Mo}(\lambda=0.71\,\si{\angstrom})\)
    } % Q1.3
        \answer{}
        \begin{flalign*}
            &
                2\,\theta
                = 2\,\arcsin\frac{
                    \lambda\,\sqrt{N}
                }{
                    2\,a
                }
                \cong 2\,\arcsin\frac{
                    0.71\,\sqrt{8}
                }{
                    2\,\num{3.83817560828058}
                }
                \cong
                \num{30.330813293608785}
                \cong 23.91
            &
        \end{flalign*}
    \end{questionBox}
    \begin{questionBox}2{ % Q1.4
        \begin{itemize}
            \item \(a=0.303\,\si{\nano\metre}\)
            \item \(k_{\alpha\,\ch{Cu}}=1.54\,\si{\angstrom}\)
            \item Primeiro pico \(2\,\theta=42.12^{\circ}\)
        \end{itemize}
        Tipo de rede e ang do 2 pico
    } % Q1.4
        \answer{}
        \begin{flalign*}
            &
                \lambda
                =2\,d\,\sin\theta
                =2\,\frac{a}{\sqrt{N}}\,\sin\theta
                \implies &\\&
                \implies
                N
                \cong \left(
                    \frac{2\,a\,\sin\theta}{\lambda}
                \right)^2
                = \left(
                    \frac{
                        2
                        \,(0.303\E1)
                        \,\sin(42.12/2)
                    }{1.54}
                \right)^2
                \cong &\\&
                \cong
                2
                % \implies &\\&
                \implies
                (110)
                \implies
                CCC
                &\\[3ex]&
                2\,\theta
                =2\arcsin\frac{
                    \lambda\,\sqrt{N}
                }{
                    2\,a
                }
                =2\arcsin\frac{
                    1.54\,\sqrt{4}
                }{
                    2*3.03
                }
                \cong
                \SI{61.094776326856902}{^\circ}
            &
        \end{flalign*}
    \end{questionBox}
\end{questionBox}

\begin{questionBox}1{ % Q2
    zinco em suf de zinc chapa de ferro merg sol aq de sulf de ferro
} % Q2
\end{questionBox}

\begin{questionBox}2{ % Q3.3
    ato por area (010)
} % Q3.3
    \answer{}
    \begin{flalign*}
        &
            \frac{2\,\pi\,r^2}
            =\pi\,r^2\,2
            =\pi*(1.371\E8)^2\,2
        &
    \end{flalign*}
\end{questionBox}

\end{document}
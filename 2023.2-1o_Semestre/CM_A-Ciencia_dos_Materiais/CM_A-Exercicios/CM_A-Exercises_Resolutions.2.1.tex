% !TEX root = ./CM_A-Exercises_Resolutions.2.1.tex
\providecommand\mainfilename{"./CM_A-Exercises_Resolutions.tex"}
\providecommand \subfilename{}
\renewcommand   \subfilename{"./CM_A-Exercises_Resolutions.2.1.tex"}
\documentclass[\mainfilename]{subfiles}

% \tikzset{external/force remake=true} % - remake all

\begin{document}

\graphicspath{{\subfix{./.build/figures/CM_A-Exercises_Resolutions.2.1}}}
\tikzsetexternalprefix{./.build/figures/CM_A-Exercises_Resolutions.2.1/graphics/}

\mymakesubfile{1}
[CM\,A]
{Polímers} % Subfile Title
{Polímers} % Part Title

\begin{questionBox}1{ % Q1
    O que são homopolímeros e copolímeros?
} % Q1
    \answer{}
    \begin{description}[
        leftmargin=!,
        labelwidth=\widthof{Homopolímeros} % Longest item
    ]
        \item[Homopolímeros] Cadeias da mesma unidade moleculares ligadas covalentemente entre si
        \item[Copolímeros] Homopolímeros porem composto por diferentes unidades moleculáres
    \end{description}
\end{questionBox}

\begin{questionBox}1{ % Q2
    Quais as diferenças entre copolímero aleatório e copolímero alternado.
} % Q2
    \answer{}
    \begin{description}[
        leftmargin=!,
        labelwidth=\widthof{Alternado} % Longest item
    ]
        \def\A{\color{Graph21}A}
        \def\B{\color{Graph22}B}
        \begin{multicols}{2}
            \item[Aleatório] {\A\A\B\A\A\B\B\B\A}
            \item[Alternado] {\A\B\A\B\A\B\A\B\A}
        \end{multicols}
    \end{description}
    \paragraph*{Nota:} Consideremos A e B diferentes monómeros
\end{questionBox}

\begin{questionBox}1{ % Q3
    Qual a motivação para a produção de copolímeros em substituição aos correspondentes homopolímeros?
} % Q3
    \answer{}
    Combinação e/ou modificação de propríedades do material formado
\end{questionBox}

\begin{questionBox}1{ % Q4
    Polímeros formados por mais de um tipo de unidade monomérica são chamados copolímeros. Um exemplo é o Nylon-66, no qual as unidades repetitivas são formadas por \iupac{1,6-diaminohexano} (\ch{H2N(CH2)6NH2}) e por ácido adípico (\ch{HOOC(CH2)4COOH}).
    \begin{enumerate}[label=\arabic{enumi}]
        \begin{multicols}{2}
            \def\A{\textcolor{Graph21}{A}}
            \def\B{\textcolor{Graph22}{B}}
            \item {\A\A\A\A\B\B\B\B}
            \item {\A\B\A\B\A\B\A\B}
            \item {\A\A\B\A\A\B\A\B}
            \item \chemfig[atom sep={1em}]{
                \A-\A-\A
                (-[90,2]\B)
                -\A-\A-\A
                (-[90,2]\B)
                -\A-\A
            }
        \end{multicols}
        Identifique nas figuras de 1 a 4 os tipos de copolímeros formados pelos monómeros A e B.
    \end{enumerate}
    \begin{enumerate}[label=\alph{enumi}.]
        \item \textcolor{Graph41}{1 — ramificado}, 
              \textcolor{Graph42}{2 — bloco}, 
              \textcolor{Graph43}{3 — alternado} e 
              \textcolor{Graph44}{4 — aleatório}.
        \item \textcolor{Graph42}{1 — bloco}, 
              \textcolor{Graph41}{2 — ramificado}, 
              \textcolor{Graph44}{3 — aleatório} e 
              \textcolor{Graph43}{4 — alternado}.
        \item \textcolor{Graph42}{1 — bloco}, 
              \textcolor{Graph43}{2 — alternado}, 
              \textcolor{Graph44}{3 — aleatório} e 
              \textcolor{Graph41}{4 — ramificado}.
        \item \textcolor{Graph44}{1 — aleatório}, 
              \textcolor{Graph42}{2 — bloco}, 
              \textcolor{Graph41}{3 — ramificado} e 
              \textcolor{Graph43}{4 — alternado}.
        \item \textcolor{Graph43}{1 — alternado}, 
              \textcolor{Graph41}{2 — ramificado}, 
              \textcolor{Graph42}{3 — bloco} e 
              \textcolor{Graph44}{4 — aleatório}.
    \end{enumerate}
    } % Q4
    \answer{C.}
    \begin{enumerate}
        \begin{multicols}{4}
            \item \textcolor{Graph42}{Bloco}
            \item \textcolor{Graph43}{Alternado}
            \item \textcolor{Graph44}{Aleatório}
            \item \textcolor{Graph41}{Ramificado}
        \end{multicols}
    \end{enumerate}
\end{questionBox}

\begin{questionBox}1{ % Q5
    Desenha uma secção de três unidades repetitivas dos seguintes polímeros:
} % Q5
    \begin{questionBox}3b{ % Q5.1
        Poliestireno
    } % Q5.1
        \begin{center}
            % \tikzset{external/remake next=true}
            \chemfig{
                -C(-[90])(-[-90])
                -C(-[90])(-[-90,.7]*6([-90,.4]=-=-=-))
                -C(-[90])(-[-90])
                -C(-[90])(-[-90,.7]*6([-90,.4]=-=-=-))
                -C(-[90])(-[-90])
                -C(-[90])(-[-90,.7]*6([-90,.4]=-=-=-))
                -
            }
        \end{center}
        \answer{}
        % \tikzset{external/remake next=true}
        \chemfig{
            -C(-[90])(-[-90])
            -C(-[90])(-[-90,.7]*6([-90,.4]=-=-=-))
            -
        }
    \end{questionBox}
    \begin{questionBox}3b{ % Q5.2
        Policloreto de vinilo
    } % Q5.2
        \begin{center}
            \chemfig{
                -C(-[90])(-[-90])
                -C(-[90])(-[-90]Cl)
                -C(-[90])(-[-90])
                -C(-[90])(-[-90]Cl)
                -C(-[90])(-[-90])
                -C(-[90])(-[-90]Cl)
                -
            }
        \end{center}
        \answer{}
        % \chemfig{-[@{op,.25}]
        %      C(-[90,.5])(-[-90,.5])
        %     -C(-[90,.5])(-[-90,.5]Cl)
        % -[@{cl,.75}]}
        % \polymerdelim{op}{cl}
        \chemfig{
            -C(-[90,.5])(-[-90,.5])
            -C(-[90,.5])(-[-90,.5]Cl)
            -
        }
    \end{questionBox}
\end{questionBox}

\begin{questionBox}1{ % Q6
    Considere as fórmulas destes dois polímeros.
    \begin{center}
        \chemfig{
            -C(-[2]H)(-[6]H)
            -C(-[2]H)(-[6]CH3)
            -
            \qquad
            -C(-[2]H)(-[6]H)
            -C(-[2]H)(-[6]Cl)
            -
        }
    \end{center}
    Os monómeros correspondentes aos polímeros I e II são, respectivamente,
} % Q6
    \begin{enumerate}[label=\alph{enumi}.]
        \def\pano{\textcolor{Graph21}{Propano}}
        \def\peno{\textcolor{Graph22}{Propeno}}
        \def\cano{\textcolor{Graph21}{cloroetano}}
        \def\ceno{\textcolor{Graph22}{cloroeteno}}
        \begin{multicols}{2}
            \item \pano{} e \cano
            \item \pano{} e \ceno
            \item \peno{} e \cano
            \item \peno{} e \ceno
        \end{multicols}
    \end{enumerate}
    \answer{a}
\end{questionBox}

\begin{questionBox}1{ % Q7
    Na tabela, são apresentadas algumas características de quatro importantes polímeros.
    } % Q7
    \def\X{\textcolor{Graph41}X}
    \def\Y{\textcolor{Graph42}Y}
    \def\Z{\textcolor{Graph43}Z}
    \def\W{\textcolor{Graph44}W}

    % \tikzset{external/force remake=true}
    \begin{center}
        \vspace{1ex}
        \setlength\tabcolsep{3mm} % width
        % \renewcommand\arraystretch{1.25} % height
        \begin{tabular}{c@{}c m{10em}}
            \toprule
            
                \multicolumn{1}{c}{Polímero}
                & \multicolumn{1}{c}{Estrutura química}
                & \multicolumn{1}{c}{Aplicações}
            
            \\\midrule
                    \X 
                    & \chemfig{-CH2-CH2-}
                    & Copos, sacos de plástico, embalágens de garrafas
                \\  \Y 
                    & \chemfig{-CH2-CH(-[6,.5]CH3)-}
                    & Fibras, cordas, assentos de cadeiras
                \\  \Z 
                    & \chemfig{-CH2-CH(-[6,.5]*6([6,.2]=-=-=-))-}
                    & Embalagens descartáveis de alimentos, pratos
                \\  \W 
                    & \chemfig{-CH2-CH(-[6,.5]Cl)-}
                    & Tubos, filmes para embalágens
            \\\bottomrule
        \end{tabular}
        \vspace{2ex}
    \end{center}
    \begin{questionBox}2{ % Q7.1
        Polipropileno, poliestireno e polietileno são, respectivamente, os polímeros:
    } % Q7.1
        \begin{enumerate}[label=\alph{enumi}.]
            \begin{multicols}{3}
                \item \X{}, \Y{} e \Z{}
                \item \X{}, \Z{} e \W{}
                \item \Y{}, \W{} e \Z{}
                \item \Y{}, \Z{} e \X{}
                \item \Z{}, \Y{} e \X{}
            \end{multicols}
        \end{enumerate}
        \answer{}
        \begin{description}[
            leftmargin=!,
            labelwidth=\widthof{} % Longest item
        ]
            \begin{multicols}{3}
                \item[\X] Polietileno
                \item[\Y] Polipropileno
                \item[\Z] Poliestireno
            \end{multicols}
        \end{description}
    \end{questionBox}
    \begin{questionBox}2{ % Q7.2
        Identifica o polímero \W
    } % Q7.2
        \answer{}
        Policloreto de vinilo
    \end{questionBox}
\end{questionBox}

\begin{questionBox}1{ % Q8
    Identifique os polímeros
} % Q8
    \answer{}
    \begin{center}
        \vspace{1ex}
        \begin{tabular}{*{3}{c}}
            \hline

              \cellF{Sal}
            & \cellT{Esferovite}
            & \cellT{DNA}
            \\\cellT{Cabelo}
            & \cellT{Unhas}
            & \cellF{Ferro}
            \\\cellF{Açúcar}
            & \cellT{Papel}
            & \cellF{Neon}
            \\\cellT{Nylon}
            & \cellF{Vinagre}
            & \cellF{Músculos}
            \\\cellF{Bicarbonato}
            & \cellT{Seda}
            & \cellT{Slime}
            \\\cellF{Vidro}
            & \cellT{Lã}
            & \cellT{Polietileno}
            \\\cellF{Madeira}
            & \cellF{Ouro}
            & \cellF{Prata}

            \\\hline
        \end{tabular}
        \vspace{2ex}
    \end{center}
\end{questionBox}

\end{document}
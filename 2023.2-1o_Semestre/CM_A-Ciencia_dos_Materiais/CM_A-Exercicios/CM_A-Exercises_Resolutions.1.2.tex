% !TEX root = ./CM_A-Exercicios_Resolucoes.2.tex
\providecommand\mainfilename{"./CM_A-Exercicios_Resolucoes.tex"}
\providecommand \subfilename{}
\renewcommand   \subfilename{"./CM_A-Exercicios_Resolucoes.2.tex"}
\documentclass[\mainfilename]{subfiles}

% \tikzset{external/force remake=true} % - remake all

\begin{document}

% \graphicspath{{\subfix{./.build/figures/CM_A-Exercicios_Resolucoes.2}}}
% \tikzsetexternalprefix{./.build/figures/CM_A-Exercicios_Resolucoes.2/graphics/}

\mymakesubfile{2}
[CM A]
{Difração de Raios X} % Subfile Title
{Difração de Raios X} % Part Title


\setcounter{question}{6}

\begin{questionBox}1{ % Q7
    Sabendo os critérios para determinar a existência de difração por parte de uma família de planos são na estrutura:
    \begin{description}[
        % leftmargin=!,
        % labelwidth=\widthof{} % Longest item
    ]
        \item [Cúbica simples (CS)]: todos os índices possíveis
        \item [Cúbica de Corpo Centrado (CCC)]: Soma dos índices é par
        \item [Cúbica de Faces Centradas (CFC)]: Indices todos pares ou todos ímpares
    \end{description}
    Assinale abaixo nas colunas correspondentes as reflexões possíveis para casa caso:
} % Q7
    \begin{center}
        \vspace{1ex}
        \begin{tabular}{*{7}{C}}
            \toprule
            
                h & k & l & N & CS & CFC & CCC
            
            \\\midrule
            
                   1 & 0 & 0 &  1 & \CellTRUE{X} & \CellFALSE{} & \CellFALSE{}
                \\ 1 & 1 & 0 &  2 & \CellTRUE{X} & \CellFALSE{} & \CellTRUE{X}
                \\ 1 & 1 & 1 &  3 & \CellTRUE{X} & \CellTRUE{X} & \CellFALSE{}
                \\ 2 & 0 & 0 &  4 & \CellTRUE{X} & \CellTRUE{X} & \CellTRUE{X}
                \\ 2 & 1 & 0 &  5 & \CellTRUE{X} & \CellFALSE{} & \CellFALSE{}
                \\ 2 & 1 & 1 &  6 & \CellTRUE{X} & \CellFALSE{} & \CellTRUE{X}
                \\ 2 & 2 & 0 &  8 & \CellTRUE{X} & \CellTRUE{X} & \CellTRUE{X}
                \\ 2 & 2 & 1 &  9 & \CellTRUE{X} & \CellFALSE{} & \CellFALSE{}
                \\ 3 & 0 & 0 &  9 & \CellTRUE{X} & \CellFALSE{} & \CellFALSE{}
                \\ 3 & 1 & 0 & 10 & \CellTRUE{X} & \CellFALSE{} & \CellTRUE{X}
                \\ 3 & 1 & 1 & 11 & \CellTRUE{X} & \CellTRUE{X} & \CellFALSE{}
                \\ 2 & 2 & 2 & 12 & \CellTRUE{X} & \CellTRUE{X} & \CellTRUE{X}
            
            \\\bottomrule
        \end{tabular}
        \vspace{2ex}
    \end{center}
\end{questionBox}

\begin{questionBox}1{ % Q8
    Os elementos do Grupo IV-A da tabela periódica apresentam uma estrutura cristalina designada de diamante em que as reflexões ocorrem nos planos nos quais os índices (\textit{h\,k\,l}) são: 
    \begin{enumerate}[label={\roman{enumi}:}]
        \item todos ímpares ou 
        \item todos pares e \(h+k+l = 4\,n\), i.e., a soma é um múltiplo de 4. 
    \end{enumerate}
    Determine as posições \(2\,\theta\) em que deverá obter os primeiros 12 picos de difração do Si (\(a_{\ch{Si}} = 5.4309\,\si{\angstrom}\)), utilizando o comprimento de onda da radiação
    \begin{itemize}
        \begin{multicols}{2}
            \item \(K\,\alpha_{\ch{Mo}} = 0.71073\,\si{\angstrom}\) 
            \item \(K\,\alpha_{\ch{Cu}} = 1.5406\,\si{\angstrom}\)
        \end{multicols}
    \end{itemize}
} % Q8
    \answer{}
    \begin{flalign*}
        &
            n\,\lambda
            =2\,d_{h\,k\,l}\,\sin(\theta)
            \implies &\\&
            \implies
            2\,\theta
            =2\,\arcsin{
                \frac{n\,\lambda}{2\,d_{h\,k\,l}}
            }
            =2\,\arcsin{
                \frac
                {1*\lambda}
                {
                    2\,\left(
                        a/\sqrt{h^2+k^2+l^2}
                    \right)
                }
            }
            = &\\&
            = 2\,\arcsin{
                \frac
                {\lambda}
                {2\,a/\sqrt{N}}
            }
            = 2\,\arcsin{
                \frac{\lambda\,\sqrt{N}}{2\,a}
            }
        &
    \end{flalign*}
    \begin{center}
    \vspace{1ex}
    \begin{tabular}{*{6}{C}}
        \toprule
        
            h & k & l & N 
            & 2\,\theta_{\ch{Mo}}
            & 2\,\theta_{\ch{Cu}}
        
        \\\midrule

               1 & 1 & 1 & 3  & \num{0.2271577799212618}  & \num{0.4964176526627561}
            \\ 2 & 2 & 0 & 8  & \num{0.37229643079182634} & \num{0.8255969055372879}
            \\ 3 & 1 & 1 & 11 & \num{0.43752076202560913} & \num{0.9795300408828223}
            \\ 4 & 0 & 0 & 16 & \num{0.5296401588834614}  & \num{1.2065589800025225}
            \\ 3 & 3 & 1 & 19 & \num{0.5784714580025393}  & \num{1.3330312444727548}
            \\ 4 & 2 & 2 & 24 & \num{0.6526399087814267}  & \num{1.5364343977148072}
            \\ 3 & 3 & 3 & 27 & \num{0.6938434696156168}  & \num{1.6572544356990346}
            \\ 5 & 1 & 1 & 27 & \num{0.6938434696156168}  & \num{1.6572544356990346}
            \\ 4 & 4 & 0 & 32 & \num{0.7583411174226298}  & \num{1.8624395464317756}
            \\ 5 & 3 & 1 & 35 & \num{0.7949950826472407}  & \num{1.9913128689472925}
            \\ 6 & 2 & 0 & 40 & \num{0.8533371822942181}  & \num{2.2261101048348797}
            \\ 5 & 3 & 3 & 43 & \num{0.8869453875337996}  & \num{2.38928459392476}
            
        \\\bottomrule
    \end{tabular}
    % 0.71073
    \vspace{2ex}
    \end{center}
\end{questionBox}

\begin{questionBox}1{ % Q9
    \begin{itemize}
        \item \(r_{\ch{Fe}} = 1.24\,\si{\angstrom}\)
        \item \(\lambda\,K\,\alpha_{\ch{Cu}} = 1.54\,\si{\angstrom}\)
        \item \(\lambda\,K\,\alpha_{\ch{Cr}} = 2.29\,\si{\angstrom}\)
    \end{itemize}
} % Q9
    \begin{questionBox}2{ % Q9.1
        Usando a lei de Bragg, calcule os ângulos de difração \(2\,\theta\) para os três primeiros
        picos do \ch{Fe-\chemalpha} (CCC) obtidos com uma ampola de cobre e com uma ampola de crómio.
    } % Q9.1
    \end{questionBox}
    \begin{questionBox}2{ % Q9.2
        Compare os dados obtidos a partir destes cálculos com os valores do espectro do aço ferramenta H13.
    } % Q9.2
    \end{questionBox}
\end{questionBox}

\begin{questionBox}1{ % Q10
    Considere uma estrutura cúbica simples. Liste por ordem crescente de densidade atómica os seguintes planos:
    \begin{itemize}
        \begin{multicols}{4}
            \item \{1\,0\,0\}
            \item \{1\,1\,0\}
            \item \{2\,1\,0\}
            \item \{1\,1\,1\}
            \item \{2\,1\,1\}
            \item \{3\,1\,1\}
            \item \{2\,2\,1\}
        \end{multicols}
    \end{itemize}
} % Q10
\end{questionBox}

\begin{questionBox}1{ % Q11
    Considere os seguintes ângulos de difração para os primeiros três picos do padrão de difração de raios X de um metal. Utilizou-se radiação monocromática que possui um
    comprimento de onda de 0.1542\,\si{\nano\metre}.
} % Q11
    \begin{center}
        \vspace{1ex}
        \begin{tabular}{c C}
            \toprule
            
                Ordem dos picos
                &
                \multicolumn{1}{c}{ângulo de difração}
            
            \\\midrule
            
                   1 & 38.6
                \\ 2 & 55.7
                \\ 3 & 70.0
            
            \\\bottomrule
        \end{tabular}
        \vspace{2ex}
    \end{center}
    \begin{questionBox}2{ % Q11.1
        Determinar se esta estrutura cristalina é CFC ou CCC, ou nenhuma delas, justificando a sua escolha.
    } % Q11.1
    \end{questionBox}
    \begin{questionBox}2{ % Q11.2
        Com base na seguinte tabela identifique qual dos metais possui esse padrão de difração
    } % Q11.2
        \begin{center}
            \vspace{1ex}
            \begin{tabular}{l c C}
                \toprule
                
                    \multicolumn{1}{c}{Metal}
                    &\multicolumn{1}{c}{
                        \begin{tabular}{c}
                            Estrutura\\Cristalina
                        \end{tabular}
                    }
                    &\multicolumn{1}{c}{
                        \begin{tabular}{c}
                            Raio Atómico\\(\si{\nano\metre})
                        \end{tabular}
                    }
                
                \\\midrule
                
                  Alumínio   & CFC  & 0.1431
                \\Cadmio     & HC   & 0.1490
                \\Crómio     & CCC  & 0.1249
                \\Cobalto    & HC   & 0.1253
                \\Cobre      & CFC  & 0.1278
                \\Ouro       & CFC  & 0.1442
                \\Ferro-a    & CCC  & 0.1241
                \\Chumbo     & CFC  & 0.1750
                \\Molibdénio & CCC  & 0.1363
                \\Níquel     & CFC  & 0.1246
                \\Platina    & CFC  & 0.1387
                \\Prata      & CFC  & 0.1445
                \\Tântalo    & CCC  & 0.1430
                \\Titânio-a  & HC   & 0.1445
                \\Tungsténio & CCC  & 0.1371
                \\Zinco      & HC   & 0.1332
                
                \\\bottomrule
            \end{tabular}
            \vspace{2ex}
        \end{center}
    \end{questionBox}
\end{questionBox}


\end{document}
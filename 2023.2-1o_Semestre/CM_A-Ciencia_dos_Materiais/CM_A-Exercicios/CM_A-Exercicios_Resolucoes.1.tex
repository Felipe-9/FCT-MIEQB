% !TEX root = ./CM_A-Exercicios_Resolucoes.1.tex
\providecommand\mainfilename{"./CM_A-Exercicios_Resolucoes.tex"}
\providecommand \subfilename{}
\renewcommand   \subfilename{"./CM_A-Exercicios_Resolucoes.1.tex"}
\documentclass[\mainfilename]{subfiles}

% \tikzset{external/force remake=true} % - remake all

\begin{document}

\graphicspath{{\subfix{./.build/figures/CM_A-Exercicios_Resolucoes.1}}}
\tikzsetexternalprefix{./.build/figures/CM_A-Exercicios_Resolucoes.1/graphics/}

\mymakesubfile{1}
[CM A]
{Exercicios} % Subfile Title
{Exercicios} % Part Title

\setcounter{question}{5}
\begin{questionBox}1{ % Q6
    O cobre tem uma estrutura CFC e um raio atómico de 1.278\,\si{\angstrom}. Quantas camadas de planos \{100\} existem ao longo da espessura de uma película de 1\,\si{\micro\metre} de espessura. Suponha que os planos (001) são paralelos às superfícies superior e inferior da película.
} % Q6
    \answer{}
    \begin{center}
        % \pgfplotsset{height=0.3\pageheight, width= .95\textwidth}
        
        % \tikzset{external/remake next=true}
        \begin{tikzpicture}
            \begin{axis}
            [
                z buffer = {sort}, % default|none|auto|sort|reverse x seq|reverse y seq|reverse xy seq
                % xlabel={\(x\)}, ylabel={\(y\)}, zlabel={\(z\)}
                % xmajorgrids = true,
                % legend pos  = north west,
                % 3d view,
                % perspective,
                % view = {10}{10}, % rot/elevation
                % hide axis,
                axis lines = {center}, % 3D center/box/left/right
                % axis on top,
                % ticks = {none} % minor/major/both/none,
            ]
                % Legends
                % \addlegendimage{empty legend}
                % \addlegendentry[Red]{\( x \)}
                
                % Conical surface
                \addplot3 [
                    % Aparence
                    surf,
                    opacity      = 0.5,
                    fill opacity = 1,
                    faceted color = foreground,
                    shader = faceted interp,
                    % Scope
                    data cs = cart, % cart/polar/polarrad
                    samples = \mysampledensitySimple,
                    % Variable
                    % variable   = z,
                    % variable y = z,
                    % domain   = -1:1,
                    % domain y = -1:1,
                ]{
                    sqrt(x^2+y^2)
                };
            
            \end{axis}
        \end{tikzpicture}
    \end{center}
\end{questionBox}

\part*{Estruturas Cristalinas -- Intersticios, Impurezas}

\setcounter{question}{11}

\begin{questionBox}1{ % Q12
    Quais são as posições intersticiais de maior volume nas redes CCC e CFC? Calcular o raio máximo dos átomos que podem entrar nessas posições.
} % Q12
    \answer{}
    \begin{center}
        \vspace{1ex}
        \begin{tabular}{l c c}
            \toprule
            
                & Oct. & Tetr.
            
            \\\midrule
            
                CFC & 0.414 & 0.255
                \\
                CCC & 0.155 & 0.291
            
            \\\bottomrule
        \end{tabular}
        \vspace{2ex}
    \end{center}
\end{questionBox}

\setcounter{question}{13}

\begin{questionBox}1{ % Q14
    Nos metais de estrutura CFC o escorregamento dá-se em planos do tipo \{111\} ao longo de direções <110> paralelas a esses planos. Escreva todas as combinações possíveis de plano e direção de escorregamento para estes metais.
} % Q14
    \answer{}
    \begin{flalign*}
        &
            \{1,1,1\}
            \begin{cases}
                (1,1,1)
                \\ 
                (\bar{1},1,1)
                , (\bar{1},\bar{1},1)
                , (\bar{1},1,\bar{1})
                , (\bar{1},\bar{1},\bar{1})
                \\ 
                (1,\bar{1},1)
                , (1,\bar{1},\bar{1})
                \\
                (1,1,\bar{1})
            \end{cases}
            &\\&
            [1,1,0]
            \begin{cases}
                [1,1,0]
                , [1,0,1]
                , [0,1,1]
                \\
                [\bar{1},1,0]
                , [\bar{1},\bar{1},0]
                , [1,\bar{1},0]
                \\
                [\bar{1},0,1]
                , [\bar{1},0,\bar{1}]
                , [1,0,\bar{1}]
                \\
                [0,\bar{1},1]
                , [0,\bar{1},\bar{1}]
                , [0,1,\bar{1}]
            \end{cases}
        &
    \end{flalign*}
\end{questionBox}

\begin{questionBox}1{ % Q15
    Usando os dados da tabela, compare o grau de solubilidade no estado sólido dos seguintes elementos no cobre: \ch{Zn, Pb, Si, Ni, Al} e \ch{Be}.
} % Q15
    \begin{center}
        \vspace{1ex}
        \setlength\tabcolsep{3mm}        % width
        % \renewcommand\arraystretch{1.25} % height
        \begin{tabular}{l *{4}{c}}
            \toprule
            
                Elemento
                & \begin{tabular}{c@{}c}
                    Raio
                    & 
                    \multirow{2}{*}{(\si{\nano\metre})}
                    \\atómico
                \end{tabular}
                & \begin{tabular}{c}
                    Estrutura\\Cristalina
                \end{tabular}
                & \begin{tabular}{c}
                    Eletro-\\negatividade
                \end{tabular}
                & Valência
            
            \\\midrule
            
                Cobre
                & 0.128 & CFC & 1.8 & +2
                \\
                Zinco
                & 0.133 & HC & 1.7 & +2
                \\
                Chumbo
                & 0.175 & CFC & 1.6 & +2,+4
                \\
                Silício
                & 0.117 
                & \begin{tabular}{c}
                    Cúbica\\Diamante
                \end{tabular}
                & 1.8 & +4
                \\
                Níquel
                & 0.125 & CFC & 1.8 & +2
                \\
                Alumínio
                & 0.143 & CFC & 1.5 & +3
                \\
                Berílio
                & 0.114 & HC & 1.5 & +2
            
            \\\bottomrule
        \end{tabular}
        \vspace{2ex}
    \end{center}
\end{questionBox}

\end{document}
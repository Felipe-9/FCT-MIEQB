% !TEX root = ./CM_A-Exercicios_Resolucoes.3.tex
\providecommand\mainfilename{"./CM_A-Exercicios_Resolucoes.tex"}
\providecommand \subfilename{}
\renewcommand   \subfilename{"./CM_A-Exercicios_Resolucoes.3.tex"}
\documentclass[\mainfilename]{subfiles}

% \tikzset{external/force remake=true} % - remake all

\begin{document}

% \graphicspath{{\subfix{./.build/figures/CM_A-Exercicios_Resolucoes.3}}}
% \tikzsetexternalprefix{./.build/figures/CM_A-Exercicios_Resolucoes.3/graphics/}

\mymakesubfile{3}
[CM A]
{Estruturas Cristalinas -- Interstícios, Impurezas} % Subfile Title
{Estruturas Cristalinas -- Interstícios, Impurezas} % Part Title

\setcounter{question}{11}

\begin{questionBox}1{ % Q12
    Quais são as posições intersticiais de maior volume nas redes CCC e CFC? Calcular o raio máximo dos átomos que podem entrar nessas posições.
} % Q12
    \answer{}
    \begin{center}
        \vspace{1ex}
        \begin{tabular}{l c c}
            \toprule
            
                & Oct. & Tetr.
            
            \\\midrule
            
                CFC & 0.414 & 0.255
                \\
                CCC & 0.155 & 0.291
            
            \\\bottomrule
        \end{tabular}
        \vspace{2ex}
    \end{center}
\end{questionBox}

\begin{questionBox}1{ % Q13
    Calcule o raio do maior interstício da rede do ferro-\chemgamma{} (CFC). O raio atómico do ferro na rede CFC é 0.129\,\si{\nano\metre} e os maiores intrestícios surgem em posições do tipo:
    \((1/2,0,0),(0,1/2,0),(0,0,1/2)\), etc.
} % Q13
\end{questionBox}

\begin{questionBox}1{ % Q14
    Nos metais de estrutura CFC o escorregamento dá-se em planos do tipo \{1\,1\,1\} ao longo de direções <110> paralelas a esses planos. Escreva todas as combinações possíveis de plano e direção de escorregamento para estes metais.
} % Q14
    \answer{}
    \begin{flalign*}
        &
            \{1,1,1\}
            \begin{cases}
                (1,1,1)
                \\ 
                (\bar{1},1,1)
                , (\bar{1},\bar{1},1)
                , (\bar{1},1,\bar{1})
                , (\bar{1},\bar{1},\bar{1})
                \\ 
                (1,\bar{1},1)
                , (1,\bar{1},\bar{1})
                \\
                (1,1,\bar{1})
            \end{cases}
            &\\&
            [1,1,0]
            \begin{cases}
                [1,1,0]
                , [1,0,1]
                , [0,1,1]
                \\
                [\bar{1},1,0]
                , [\bar{1},\bar{1},0]
                , [1,\bar{1},0]
                \\
                [\bar{1},0,1]
                , [\bar{1},0,\bar{1}]
                , [1,0,\bar{1}]
                \\
                [0,\bar{1},1]
                , [0,\bar{1},\bar{1}]
                , [0,1,\bar{1}]
            \end{cases}
        &
    \end{flalign*}
\end{questionBox}

\begin{questionBox}1{ % Q15
    Usando os dados da tabela, compare o grau de solubilidade no estado sólido dos seguintes elementos no cobre: \ch{Zn, Pb, Si, Ni, Al} e \ch{Be}.
} % Q15
    \begin{center}
        \vspace{1ex}
        \setlength\tabcolsep{1.5mm} % width
        % \renewcommand\arraystretch{1.25} % height
        \begin{tabular}{l *{4}{c}}
            \toprule
            
                Elemento
                & \begin{tabular}{c@{}c}
                    Raio atómico
                    \\(\si{\nano\metre})
                \end{tabular}
                & \begin{tabular}{c}
                    Estrutura\\Cristalina
                \end{tabular}
                & \begin{tabular}{c}
                    Eletro-\\negatividade
                \end{tabular}
                & Valência
            
            \\\midrule
            
                Cobre
                & 0.128 & CFC & 1.8 & +2
                \\
                Zinco
                & 0.133 & HC & 1.7 & +2
                \\
                Chumbo
                & 0.175 & CFC & 1.6 & +2,+4
                \\
                Silício
                & 0.117 
                & \begin{tabular}{c}
                    Cúbica\\Diamante
                \end{tabular}
                & 1.8 & +4
                \\
                Níquel
                & 0.125 & CFC & 1.8 & +2
                \\
                Alumínio
                & 0.143 & CFC & 1.5 & +3
                \\
                Berílio
                & 0.114 & HC & 1.5 & +2
            
            \\\bottomrule
        \end{tabular}
        \vspace{2ex}
    \end{center}
\end{questionBox}

\end{document}
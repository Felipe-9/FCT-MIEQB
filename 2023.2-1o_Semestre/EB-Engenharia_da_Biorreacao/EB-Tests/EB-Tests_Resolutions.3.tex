% !TEX root = ./EB-Tests_Resolutions.3.tex
\providecommand\mainfilename{"./EB-Tests_Resolutions.tex"}
\providecommand \subfilename{}
\renewcommand   \subfilename{"./EB-Tests_Resolutions.3.tex"}
\documentclass[\mainfilename]{subfiles}

% \tikzset{external/force remake=true} % - remake all

\begin{document}

% \graphicspath{{\subfix{./.build/figures/EB-Tests_Resolutions.3}}}
% \tikzsetexternalprefix{./.build/figures/EB-Tests_Resolutions.3/graphics/}

\mymakesubfile{3}
[EB]
{Exame 2023 Resolução} % Subfile Title
{Exame 2023 Resolução} % Part Title

\begin{questionBox}1{ % Q1
    Crescimento microbiano em 4 fases
} % Q1
    \begin{questionBox}2{ % Q1.1
        identificar fazes e propor balanço material
    } % Q1.1
        \answer{}
        \begin{enumerate}[label=(\roman{enumi})]
            \item lag\\
            Inicialmente as bactérias estão se adaptando ao meio sem haver producão significativa de produto ou crescimento da população\\
            \(r_{metabolismo}\,n_{micro}\)
            \item Fase esponencial\\
            Aqui o crescimento microbiano se dá em função do numero de celulas usando de todo os recursos de forma ambudante\\
            \(
                r_{metabolismo}\,n_{micro}
                + r_{crescimento}\,(
                    \adif{n_{micro}}+n_{micro}
                )
            \)
            \item Estacionária\\
            A cultura atingiu um equilíbrio entre multiplicação e morte celular e é uma questao de tempo até que os gastos energéticos do metabolismo e multiplicação depletem os recursos iniciando a fase da morte
            \(
                r_{metabolismo}\,n_{micro,\max}
            \)
            % 
            \item Morte\\
            Recursos se tornam escassos levando a um maior numero de morte que de multiplicações resultando declive de população
            \(
                r_{metabolismo}\,n_{micro}
            \)
        \end{enumerate}
    \end{questionBox}
    \begin{questionBox}2{ % Q1.2
        quando a taxa de cresc p conc em mss é diff para em numero
    } % Q1.2
        \answer{}
        Quando falta substrato
    \end{questionBox}
\end{questionBox}

\setcounter{question}{3}
\begin{questionBox}1{ % Q4
    impacto da viscosidade do meio sobre:
} % Q4
    \begin{questionBox}2{ % Q4.1
        Transf de massa global \(K_l\)
    } % Q4.1
        \answer{}
        um ambiente mais viscoso impede o livre transito de gases e substratos no meio, impedindo.
    \end{questionBox}
    \begin{questionBox}2{ % Q4.2
        area Interf gás liquido
    } % Q4.2
        \answer{}
        A area interfacial se dá pelas bolhas que são maiores em um meio mais viscoso diminuindo a area interf por volume
    \end{questionBox}
    \begin{questionBox}2{ % Q4.3
        V de transf de o2
    } % Q4.3
        Depende da area interfacial, será afetado negativamente
    \end{questionBox}
\end{questionBox}

\begin{questionBox}1{ % Q5
    \begin{itemize}
        \begin{multicols}{2}
            \item S:\ch{CH_{1.8}O_{0.5}N_{0.17}}
            \item \(M_{w\,S}=24.2\,\unit{\gram.\mole}\)
            \item X:\ch{C18H32O16}
            \item \(M_{w\,s=504.4\,\unit{\gram.\mole}}\)
            \item 2 reat em série CSTR, PFR
            \item \(F=100\,\unit{\litre.\hour^{-1}}\)
            \item \(S_0=60\,\unit{\gram.\litre}\)
            \item \(v_{s\,\max}=0.75\,\unit{\gram\of{S}/\gram\of{X}.\hour}\)
            \item \(S_2=0.15\,\unit{\gram.\litre}\)
            \item \(\mu_{\max}=0.21\,\unit{\hour^{-1}}\)
            \item \(K_s=10\,\unit{\gram/\litre}\)
            \item \(m_s=0\)
        \end{multicols}
    \end{itemize}
} % Q5
    \begin{center}\large
        \ch{
            a C3H8O3
            + b NH3
            ->
            x C18H32O16
            + p CH_{1.8}O_{0.5}N_{0.17}
            + c CO2
            + d H2O
        }
    \end{center}
    \begin{questionBox}2{ % Q5.1
        Coeff de rendimento \(Y_{P/S}/(\unit{\mole\of{P}/\mole\of{S}})\) no reator 1
        \begin{itemize}
            \begin{multicols}{2}
                \item \(Y_{X/S}=0.16\,\unit{\gram\of{X}/\gram\of{S}}\)
                \item \(Y_{X/P}=4.41\,\unit{\gram\of{X}/\gram\of{P}}\)
            \end{multicols}
        \end{itemize}
    } % Q5.1
        \answer{}
        \begin{flalign*}
            &
                Y_{P/S}
                = \frac{Y_{X/S}}{Y_{X/P}}
                = \frac{
                    0.16
                    \,\unit{
                        \frac
                        {\gram\of{X}}
                        {\gram\of{S}}
                    }
                    \,M_{w\,S}
                    \,\unit{\gram\of{S}/\mole\of{S}}
                }{
                    4.41
                    \,\unit{
                        \frac
                        {\gram\of{X}}
                        {\gram\of{P}}
                    }
                    \,M_{w\,P}
                    \,\unit{\gram\of{P}/\mole\of{P}}
                }
                = &\\&
                = \frac{
                    0.16\,M_{w\,S}
                }{
                    4.41\,M_{w\,P}
                }
                \,\unit{\frac{\mole\of{P}}{\mole\of{S}}}
                % = &\\&
                = \frac{
                    0.16*92.1
                }{
                    4.41*504.4
                }
                \,\unit{\frac{\mole\of{P}}{\mole\of{S}}}
                \cong &\\&
                \cong
                \qty{6.624695873591308e-3}{\frac{\mole\of{P}}{\mole\of{S}}}
            &
        \end{flalign*}
    \end{questionBox}
\end{questionBox}

\begin{questionBox}1{ % Q6
    \begin{itemize}
        \begin{multicols}{2}
            \item CSTR
            \item \(V=50\,\unit{\metre^3}\)
            \item \(z=2.5\,\unit{\metre}\)
            \item \(F=8\,\unit{\metre^3.\hour^{-1}}\)
            \item \(1.2\,\unit{vvm}\)
            \item \(d=0.08\,\unit{\milli\M}\)
            \item Estado estacionário
            \item \(Cp_{\ch{O2}}=1.03\,\unit{\milli\M}\)
            \item Tensão sup do ar: 65\,\unit{\gram/\second^2}
            \item \(\rho_{ar}=1.4\,\unit{\milli\gram/\centi\metre^3}\)
            \item Coeff de transf de massa global: \(0.037\,\unit{\centi\metre.\second^{-1}}\)
            \item Dif de \ch{O2}: \(2\E{-5}\,\unit{\centi\metre^2/\second}\)
            \item \(\rho_{cultura}=1\,\unit{\gram/\centi\metre^3}\)
        \end{multicols}
    \end{itemize}
    Encontre a v de trasnf de O2 (\unit{\kilo\gram\of{\ch{O2}}.\litre^{-1}.\hour^{-1}})
} % Q6
    body
\end{questionBox}

\begin{questionBox}1{ % Q7
    \begin{itemize}
        \begin{multicols}{2}
            \item \(V=50\,\unit{\metre^3}\)
            \item \(T=45\,\unit{\celsius}\)
            \item \(F=12\,\unit{\metre^3/\hour}\)
            \item \(1.5\,\unit{\gram/\litre.\hour}\)
            \item \(1\,\unit{\kilo\watt/\metre^3}\)
            \item \(Cp_{\ch{H2O}}=75.4\,\unit{\joule/\mole.\celsius}\)
            \item \(Q=460\,\unit{\kilo\joule/\mole\of{\ch{O2}}}\)
        \end{multicols}
    \end{itemize}
    Determine temp de saída \(T_1\)
} % Q7
    \answer{}
    \begin{flalign*}
        &
            T_1
            =T_0+\adif{T}
            ; &\\[3ex]&
            \adif{H}
            = \adif{H}_{rn}+W_s
            ; &\\[3ex]&
            \adif{H}_{rn}
            = Q*C_{\ch{O2}}*V
            = \left(
                \begin{aligned}
                    &
                        460\,\unit{\kilo\joule/\mole}
                    &*\\*&
                        1.5\,\unit{\gram/\litre.\hour}
                    &*\\*&
                        1000\,\unit{\litre/\metre^3}
                    &*\\*&
                        3600^{-1}\,\unit{\hour/\second}
                    &*\\*&
                        32^{-1}\,\unit{\mole/\gram}
                    &*\\*&
                        50\,\unit{\metre^3}
                    &
                \end{aligned}
            \right)
            \cong
            \qty{299.479166666667}{\kilo\joule/\second}
            ; &\\[3ex]&
            W_s
            = 1\,\unit{\kilo\watt/\metre^3}
            \,50\,\unit{\metre^3}
            = 50\,\unit{\kilo\watt}
            ; &\\[3ex]&
            \adif{H}
            \cong
            \qty{299.479166666667}{\kilo\joule/\second}
            +50\,\unit{\kilo\joule/\second}
            \cong
            \qty{349.479166666667}{\kilo\joule/\second}
            = &\\[3ex]&
            = M\,Cp\,\adif{T}= &\\&
            = (v*\rho_{\ch{H2O}})
            \,\left(
                75.4\,\unit{\joule/\mole.\celsius}
                \,\frac
                    {\unit{\mole}}
                    {M_{w\,\ch{H2O}}\,\unit{\gram}}
            \right)
            \,\adif{T}
            = &\\&
            = (12*1000\,\unit{\kilo\gram/\hour})
            \,\left(
                75.4\,\unit{\joule/\mole.\celsius}
                \,\frac
                    {\unit{\mole}}
                    {18\,\unit{\gram}}
            \right)
            \,\adif{T}
            \implies &\\[3ex]&
            \implies
            T_1/\unit{\celsius}
            = 10
            + \frac{
                \num{349.479166666667}
            }{
                \frac{12*1000}{3600}
                \frac
                    {75.4}
                    {18}
            }
            \cong\num{35.029011936339544}
        &
    \end{flalign*}
\end{questionBox}

\end{document}

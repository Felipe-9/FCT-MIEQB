% !TEX root = ./EB-Tests_Resolutions.1.tex
\documentclass["EB-Tests_Resolutions.tex"]{subfiles}

% \tikzset{external/force remake=true} % - remake all

\begin{document}

% \graphicspath{{\subfix{./.build/figures/EB-Tests_Resolutions.1}}}
% \tikzsetexternalprefix{./.build/figures/EB-Tests_Resolutions.1/graphics/}

\mymakesubfile{1}[EB]
{Teste Res 1 2024.1} % Subfile Title
{Teste Res 1 2024.1} % Part Title

\begin{questionBox}1{} % Q1
  Descrever as fases
  \answer{}
  \begin{enumerate}[label=\arabic{enumi}]
    \item Lag: Caracterizada por \(\mu\approx0\) é quand as celulas estão se adaptando ao meio, produzindo preparando novas enziamas
    \item Acceleration \(\mu\approx\mu_{\max}\) As as celulas iniciam o crescimento
    \item Growth \(s\gtrsim10\,K_B, \mu\approx\mu_{\max}\) Fase exponencial, substrato em ambudancia e nenhum limitador, as ceulas se multiplicam livremente
    \item Stationary \(s\ll K_B, \mu\approx0\) A grande quantidade de celulas consome o resto de substrato ou encontra algum outro limitante talvez de espaço, crescimento celular se iguala ao de morte
    \item Death \(s\approx 0\) Celulas exaustam o substrato e começam a morrer, declinando a população
  \end{enumerate}
\end{questionBox}

\begin{questionBox}1{} % Q2
  \begin{enumerate}[label=\alph{enumi}]
    \item B % a
    \item C % b
    \item A % c
    \item D % d
  \end{enumerate}
\end{questionBox}

\setcounter{question}{3}
\begin{questionBox}1{} % Q4
  \begin{itemize}
    \item B \ch{CH_{1.8}O_{0.5}N_{1.7}} \(MW_B=\qty*{24.2}{\g/\mole}\)
    \item P \ch{C_{18}H32O16} \(MW_{P}=\qty*{504.4}{\g/\mole}\)
    \item S \ch{C3H8O3} \(MW_{S} = \qty*{92.1}{\g/\mole}\)
    \item Fonte de N \ch{NH3}
    \item \(F=\qty*{100}{\L/\hour}\)
    \item \(S_0 = \qty*{60}{\g/\L}\)
    \item \(Y_{X/S}=\qty*{0.16}{\gram\of{X}/\gram\of{S}}\)
    \item \(K_S=\qty*{0.05}{\g/\L}\)
    \item \(\mu_{\max}=\qty*{0.12}{\litre^{-1}}\)
    \item \(m_s= 0\)
  \end{itemize}
\end{questionBox}

\begin{questionBox}2{} % Q4.1
  \(\mu=\qty*{0.1}{\hour^{-1}}\) det o vol do reator
  \answer{}
  \begin{center}\Large\bfseries
      \ch{
        C_{3}H_{8}O_{3}
        + {a} O2
        + {b} NH3
        -> {c} CH_{1.8}O_{0.5}N_{1.7}
        + {d} CO2
        + {e} H2O
        + {f} C18H32O16
      }
  \end{center}
  \begin{flalign*}
    &
      % \ch{
      %   C_{3}H_{8}O_{3}N_{0}
      %   + {a} O2
      %   + {b} NH3
      %   -> {c} CH_{1.8}O_{0.5}N_{1.7}
      %   + {d} CO2
      %   + {e} H2O
      % }
      \begin{cases}
          C: & 3 = c + d + f\,16
        \\ H: & 8+b = c\,1.8 + 2\,e + f\,32
        \\ O: & 3 + 2\,a + = c\,0.5 + 2\,d + e + f\,16
        \\ N: & b\,3 = c\,1.7
      \end{cases}
      % 
      % 
      % 
      &\\[3ex]&
      % \begin{cases}
      %    N: &-3 -- ammonia
      %   \\ C: & 4
      %   \\ H: & 1
      %   \\ O: & -2
      %   \\ P: & 5
      %   \\ S: & -6
      % \end{cases}
      \begin{cases}
        \gamma_{S} = 3*4 + 8 -2*3 = 14
        \\  \gamma_X = 4 + 1.8 -2*0.5 -3*1.7 = -0.3
      \end{cases}
      % 
      % 
      % 
      &\\[3ex]&
      Y_{XS}
      = \unit{\frac
        {\gram\of{\text{B}}}
        {\gram\of{\text{S}}}
      }
      = \frac
      {c*MW_{\text{B}}}
      {MW_S}
      = \frac
      {c*24.2}
      {92.1}
    &
  \end{flalign*}
\end{questionBox}
\end{document}

% !TEX root = ./EB-Testes_Resolucoes.1.tex
\providecommand\mainfilename{"./EB-Testes_Resolucoes.tex"}
\providecommand \subfilename{}
\renewcommand   \subfilename{"./EB-Testes_Resolucoes.1.tex"}
\documentclass[\mainfilename]{subfiles}

% \tikzset{external/force remake=true} % - remake all

\begin{document}

% \graphicspath{{\subfix{./.build/figures/EB-Testes_Resolucoes.1}}}
% \tikzsetexternalprefix{./.build/figures/EB-Testes_Resolucoes.1/graphics/}

\mymakesubfile{1}
[EB]
{Teste 1 Resolução} % Subfile Title
{Teste 1 Resolução} % Part Title

\begin{questionBox}1{ % Q1
    pross biotec em que x cresce aerobicamente, consumindo s e prod P
    \begin{center}\large\ch{
        S + O2 + NH3 -> X + P + CO2 + H2O
    }
    \end{center}
} % Q1
    \begin{questionBox}2{ % Q1.1
        Prop uma eq de bal mat ao substrato em cstr em estado estacio
    } % Q1.1
    \end{questionBox}
    \begin{questionBox}2{ % Q1.2
        Prop uma eq de bal mate ao subs em PFR
    } % Q1.2
    \end{questionBox}
\end{questionBox}

\begin{questionBox}1{ % Q2
    Considere as seguintes cine de sint de prod
    \begin{enumerate}[label={Caso \arabic{enumi}: },left=0px]
        \item \(r_p=\alpha\,\mu\,x\)
        \item \(r_p=\alpha\,\mu\,x+\beta\,x\)
        \item \(r_p=\beta\,x\)
    \end{enumerate}
} % Q2
    \begin{questionBox}2{ % Q2.1
        de aco com a class gaden, como class a cinet
    } % Q2.1
    \end{questionBox}

    \begin{questionBox}2{ % Q2.2
        qual o signif dos parametros \chemalpha, \chembeta
    } % Q2.2
        \answer{}
        \begin{description}
            \item[\chemalpha] fator relacionado com a quantidade de estrato consumida
            \item[\chembeta] Fator relacionado a quantidade de substrato estocado na biomassa
        \end{description}
    \end{questionBox}
\end{questionBox}

% Massa atomica do N=14

\begin{questionBox}1{ % Q3
    % um micro com form elementar \ch{CH_{1.75}O_{0.45}N_{0.16}} (\(MW=23.19\,\unit{\gram/\mole}\)) é usado para prod biom a partir de glicerol \ch{C3H8O3} (\(MW=92.09\,\unit{\gram/\mole}\)). A fonte de azo é \ch{NH3}. Consi o cres segue a lei de monod \(\mu_{\max}=0.25\,\unit{\hour^{-1}}\text{ e }k_s=0.05\,\unit{\gram/\litre}\), e que a manut é desp
    \begin{itemize}
        \item \ch{CH_{1.75}O_{0.45}N_{0.16}} (\(MW=23.19\,\unit{\gram/\mole}\))
        \item glicerol \ch{C3H8O3} (\(MW=92.09\,\unit{\gram/\mole}\))
        \item A fonte de azo é \ch{NH3}
        \item lei de monod \(\mu_{\max}=0.25\,\unit{\hour^{-1}}\text{ e }k_s=0.05\,\unit{\gram/\litre}\)
        \item e que a manut é desp
    \end{itemize}
} % Q3
    \begin{questionBox}2{ % Q3.1
        Det exp form 0.51\,\unit{\gram\of{P}/\gram\of{S}}. Avalie a ness de amon (rendimento amonia/biomassa) e de oxigenio (rend oxi/biomass) numa base mass
    } % Q3.1
        \answer{}
        \begin{center}\large\ch{
            a C3H8O3 
            + b O2 
            + c NH3
            ->
            d CH_{1.75}O_{0.45}N_{0.16}
            + e CO2
            + f H2O
        }
        \end{center}

        \begin{flalign*}
            &
                \mu 
                = \frac{\mu_{\max}\,S}{k_m+S}
                = \frac{0.25\,S}{0.05+S}
                = \frac{1}{0.2/S+4}
            &
        \end{flalign*}
        \subsubquestion{Amonina}
        \begin{flalign*}
            &
                Y_{P/S}
                = 0.51\,\frac{
                    \unit{\gram\of{P}}
                }{\unit{\gram\of{S}}}
                ; &\\[3ex]&
                Y_{\ch{NH3}/X}
                = \frac{
                    c\,\unit{\mole\of{\ch{NH3}}}
                }{
                    \unit{\mole\of{X}}
                }
                % \implies &\\&
                \implies
                c 
                = 0.16*d
                = 0.16
                * 0.51
                \cong
                \num{8.16e-2}
            &
        \end{flalign*}

        \subsubquestion{Oxigênio}
        \begin{flalign*}
            &
                Y_{\ch{O2}/X}
                = \frac{
                    0.45*d
                    + 2*e
                    + f
                }{
                    a*3
                    + b*2
                }
            &
        \end{flalign*}
    \end{questionBox}
\end{questionBox}

\begin{questionBox}2{ % Q3.2
    % Sabendo q o meio contem 4\%\,\unit{(w/v)} de glic e que pretende atingir uma conv de substrato de 98\% num reac 100 L
    \begin{itemize}
        \item glicerol 4\,\unit{\percent.w/v}
        \item Pretende atingit 98\,\unit{\percent}
        \item reator 100\,\unit{\litre}
        \item Modo continuo
        \item alimentado com meio estéril
        \item Estado estácionário
    \end{itemize}
    Det a conc de biomass e a prod vol em biomass.
} % Q3.2
\end{questionBox}

\begin{questionBox}2{ % Q3.3
    \begin{itemize}
        \item mesma cultura, condições, reator
        \item Fluxo pistão
        \item Crescimento é despresivel
        \item Concentração celular é de 20\,\unit{\gram.\litre}
        \item \(V_{s,\max}=0.75\,\unit{\gram\of{S}/\gram\of{X}.\hour}\)
        \item \(K_s=0.05\,\unit{\gram/\litre}\)
        \item manut desprezavel
    \end{itemize}
    det o caudal nesse reator
} % Q3.3
\end{questionBox}

\end{document}
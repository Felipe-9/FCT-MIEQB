% !TEX root = ./EB-Relatorio.4.tex
\providecommand\mainfilename{"./EB-Relatorio.tex"}
\providecommand \subfilename{}
\renewcommand   \subfilename{"./EB-Relatorio.6.tex"}
\documentclass[\mainfilename]{subfiles}

% \tikzset{external/force remake=true} % - remake all

\begin{document}

% \graphicspath{{\subfix{./.build/figures/EB-Relatorio.6}}}
% \tikzsetexternalprefix{./.build/figures/EB-Relatorio.6/graphics/}

\mymakesubfile{4}
{Conclusão} % Subfile Title
{Conclusão} % Part Title

Neste trabalho foi possível aplicar e verificar diretamente os conhecimentos sobre bioreatores com culturas aeróbicas e tudo o que isso implica, incluído principalmente o arejamento e agitação e homogeneização da , consolidando-os.\par
Com a análise das necessidades de oxigénio foram identificadas algumas das fases de crescimento, mas de forma mais pronunciada a fase exponencial e a fase estacionária.\par
Contudo alguns fatores podem ter interferido nos resultados obtidos, como por exemplo grande quantidade de espuma que se formou no reator, saindo até pelas entradas superiores do reator e que obrigou a ser retirada essa espuma. Isso pode ter retirado alguma parte da cultura ou do meio, o que pode causar uma variação com o valor inicial no reator.\par
Mesmo assim, considerando esse e qualquer outro erro que possa existir, quer seja experimental ou de cálculos,  consideramos que foi possível realizar a analise dos resultados e com eles retirar conclusões.


\end{document}
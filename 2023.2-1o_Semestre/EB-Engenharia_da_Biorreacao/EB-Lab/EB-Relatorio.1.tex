% !TEX root = ./EB-Relatorio.1.tex
\providecommand\mainfilename{"./EB-Relatorio.tex"}
\providecommand \subfilename{}
\renewcommand   \subfilename{"./EB-Relatorio.1.tex"}
\documentclass[\mainfilename]{subfiles}

% \tikzset{external/force remake=true} % - remake all

\begin{document}

% \graphicspath{{\subfix{./.build/figures/EB-Relatorio.1}}}
% \tikzsetexternalprefix{./.build/figures/EB-Relatorio.1/graphics/}

\mymakesubfile{1}
{Resumo} % Subfile Title
{Resumo} % Part Title

A realização desta atividade tem como objetivo a perceção do funcionamento da transferência de oxigénio em sistemas biológicos.\par
Neste trabalho foi usado um reator batch com agitação e arejamento através de um dispersor de oxigénio com uma cultura de microrganismos aeróbica, neste caso, bacteriana. Para se variar a concentração de oxigénio usou-se um respirómetro com um elétrodo de oxigénio ligado a um medidor de oxigénio e a concentração de oxigénio foi obtida pelo programa \textit{BioCTR}. Para permitir o fluxo do meio entre o reator e o respirómetro foi usado uma bomba peristáltica.\par
No início da experiência, antes da inoculação, foi realizado o desarejamento do reator com azoto até não existir oxigénio e depois rearejado com ar atmosférico (20.95\,\si{\percent\of{\ch{O2}}}). Através da medição de oxigénio nesse processo foi possível obter o valor de \(k'_{L\,a}\) antes da inoculação \SI{2.00513034376008}{\min^{-1}} e o valor de \(C^*_{\ch{O2}}\) de \(\SI{7.7766}{\milli\gram\of{O2}/\litre}\). Foi possível obter também a velocidade de transferência do oxigénio, o \(Q_{\ch{O2}}\) com o valor de \SI{2.735571878499499}{\milli\gram/\litre.\min}.\par
Após a realização da inoculação foram retiradas uma amostra do meio a cada 10 minutos e seguidamente era desligada a bomba peristáltica para que fosse medido o consumo de oxigénio ao longo do tempo. Com a amostra que era retirada era medida a densidade ótica a 600\,\si{\nano\metre}, o que permitiu quantificar o crescimento celular ao longo do tempo. Depois dessa medição a amostra retirada era devolvida ao meio para continuar o crescimento.\par
Através da análise de vários métodos, chegamos a conclusão de que o método que se adequava mais aos nossos resultados era o método dos 3 pontos. Através desse método obtivemos o valor de \SI{2.15405950243889E-02}{\min^{-1}} para \(\mu_{\max}\) e o valor de \SI{863.015205640127524}{\milli\gram/\litre} para o \(X_{\max}\). 
Foi possível observar claramente algumas fases de crescimento ao longo da atividade entre elas a fase exponencial e a fase estacionária. Obtivemos assim o valor de \(k'_{L\,a}\) na fase exponencial \SI{3.51769652354435E-01}{\min^{-1}} e determinamos os valores dos coeficientes de rendimento do consumo de oxigénio \(Y_{\ch{O2}/X}\) e de crescimento celular \(Y'_{X/S}\,\si{\milli\gram\of{X}/\milli\gram\of{S}}\) e de \(Y_{X/S}\,\si{\milli\gram\of{X}/\milli\gram\of{S}}\).

\end{document}
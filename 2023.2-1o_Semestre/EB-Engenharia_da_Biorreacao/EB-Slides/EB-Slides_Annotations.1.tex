% !TEX root = ./EB-Slides_Annotations.1.tex
\providecommand\mainfilename{"./EB-Slides_Annotations.tex"}
\providecommand \subfilename{}
\renewcommand   \subfilename{"./EB-Slides_Annotations.1.tex"}
\documentclass[\mainfilename]{subfiles}

% \tikzset{external/force remake=true} % - remake all

\begin{document}

% \graphicspath{{\subfix{./.build/figures/EB-Slides_Annotations.1}}}
% \tikzsetexternalprefix{./.build/figures/EB-Slides_Annotations.1/graphics/}

\mymakesubfile{2}
[EB]
{Cinética da Biorreação: Batch Reactors} % Subfile Title
{Cinética da Biorreação: Batch Reactors} % Part Title

\begin{sectionBox}1{Definitions} % S



\end{sectionBox}

\setcounter{section}{5}
\begin{sectionBox}1{Relationship between cell grownth and substrate consumption} % S

  \paragraph*{As rule of thumb}
  \begin{BM}
    S \approx  3\,K_S
  \end{BM}

\end{sectionBox}

\begin{sectionBox}1{Cell grownth phases} % S
    
  Elaborar cada fase de crescimento
  \begin{enumerate}
    \item Lag
    \item Exponential
    \item Stationary
    \item Death
  \end{enumerate}  
    
\end{sectionBox}

\begin{sectionBox}1{Biomassa} % S
    
    Elaboração da biomassa
    
\end{sectionBox}

\begin{sectionBox}1{Modelos para aproximar o crescimento celular} % S
    
  \begin{enumerate}
    \item Malthus model
    \item Verhulst Model
  \end{enumerate}
    
\end{sectionBox}

\begin{sectionBox}2{Verhulst Model} % S
    
    \begin{BM}
      \odv{X}{t} = k\,X(1-\beta\,X)
      \iff \\
      \iff
      X
      =\frac
      {X_0\,\exp{(k\,t)}}
      {1-\beta\,X_0\,(1-\exp{(k\,t)})}
      \iff
      t = \frac
      {\ln{\left(
        \frac
        {-x(x_{\max{}}-x_0)}
        {x_0(x-x_{\max{}})}
      \right)}}
      {\mu_{\max}}
      \\
      \begin{cases}
        k=\mu;\\\beta=X_{\max{}}^{-1}
      \end{cases}
    \end{BM}
    
\end{sectionBox}

\end{document}

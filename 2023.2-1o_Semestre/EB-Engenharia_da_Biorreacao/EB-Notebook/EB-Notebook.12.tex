% !TEX root = ./EB-Notebook.12.tex
\documentclass["EB-Notebook.tex"]{subfiles}

% \tikzset{external/force remake=true} % - remake all

\begin{document}

\graphicspath{{\subfix{./.build/figures/EB-Notebook.12}}}
% \tikzsetexternalprefix{./.build/figures/EB-Notebook.12/graphics/}

\mymakesubfile{12}[EB]
{12 Homogeneous Reactions} % Subfile Title
{Homogeneous Reactions} % Part Title

\setcounter{subpart}{7}
\subpart{Bioreactor Kinetics}


\begin{sectionBox}1{Batch Growth} % S1
      
  When cells are grown in batch culture, \emph{several phases} of cell growth are observed.
  The different phases of growth are more \emph{readily distinguished} when the \emph{logarithm of viable cell concentration is plotted against time}

  \begin{center}
    \includegraphics[height=20ex]{cellGrowthPhases.png}
    \includegraphics[height=20ex]{cellGrowthPhases.1.png}
  \end{center}
  \begin{center}
    \setlength\tabcolsep{2mm}        % width
    % \renewcommand\arraystretch{1.25} % height
    \vspace{1ex}
    \begin{tabular}{l @{\hspace{1em}} L p{ 20em } @{} }
      \toprule

       Phase
       & \begin{tabular}{@{}c}
         Specific\\growth rate
       \end{tabular}
       & Description

       \\\midrule

       Lag & \mu\approx0
       & Cells adapting to new environment

       \\ Acceleration & \mu<\mu_{\max{}}
       & Cell population starts growing

       \\ Growth & \mu\approx\mu_{\max{}}
       & Growth achieve maximum rate

       \\ Decline & \mu<\mu_{\max{}}
       & Culture reaches limitant in nutrients or build-up of products

       \\ Stationary & \mu=0
       & Cell death and birth equalize

       \\ Death & \mu<0
       & Cells dying faster than they can multiply

       \\\bottomrule
       \multicolumn{3}{r}{Growth rate: \(r_X = \odv{x}{t} = \mu\,x\)}
     \end{tabular}
     \vspace{2ex}
   \end{center}
\end{sectionBox}

\begin{sectionBox}*2{Growth rate} % S1.0.1
  The cell growth rate (\(r_X\)) measures the change in volumetric concentration of viable cells (\(\unitof{( x )}=\unit{\gram/\metre^3}\))
  \begin{BM}[equation]\label{eq:growRateInGrowthAndDecline}
    r_X = \odv{x}{t} = \mu\,x
    \implies 
    \adif{\ln{x}}
    = \mu\,t
  \end{BM}\vspace{-3ex}
  \begin{equation*}
          \dim{x} = \unit{M/L^3}
    ;\quad\dim{\mu} = \unit{T^{-1}}
  \end{equation*}

  \paragraph{Doubling time \(t_d\)}
  Cell growth rates are often expressed in terms of the time it takes to duplicate the population.
  \begin{BM}[equation]
    x=2\,x_0 : t=t_d = \frac{\log{2}}{\mu}
  \end{BM}
  \begin{flalign*}
    &
      t_d
      \implies
      \adif{\ln{x}}
      =\ln{\frac{x}{x_0}}
      =\ln{\frac{2\,x_0}{x_0}}
      =\ln{2}
      = \mu\,t_d
    &
  \end{flalign*}

\end{sectionBox}

\begin{sectionBox}1{Balanced Growth} % S2
  In an environment favourable for growth, cells regulate their metabolism and adjust the rates of various internal reactions so that a condition of balanced growth occurs. During balanced growth, \emph{the composition of the biomass remains constant}. For the biomass composition to remain constant during growth, the specific rate of production of each component in the culture must be equal to the cell specific growth rate \(\mu\).\\
  In most cultures, balanced growth occurs at the same time as exponential growth.
  \begin{BM}[equation]
    r_P = \mu\,p
  \end{BM}
  Another point is that the cell consumption is also constant
  \begin{BM}
    \odv{x}{t}=r_X
  \end{BM}
\end{sectionBox}

\begin{sectionBox}1{Effect of Substrate Concentration} % S3
  \emph{During the growth and decline phases of batch culture}, the specific growth rate of the cells depends on the concentration of nutrientes in the medium.\\[1ex]
  Often a \emph{single substrate} exerts a \emph{dominant influence on the rate of growth}; this component is know as the \emph{growth-rate-limiting substrate (\textit{S})}.\\[1ex]
  During balanced growth, the specific growth rate is related to the concentration of the growth limiting substrate by a homologue of the \textit{Michaelis--Menten} expresion, the \emph{Monod equation:}
  \begin{BM}[equation]
    \mu 
    = \frac
    {\mu_{\max{}}\,s}
    {K_S+s}
    = \frac
    {\mu_{\max{}}}
    {1+K_S/s}
  \end{BM}\vspace{-3ex}
  \begin{equation*}
    \dim{s} = \unit{\frac{M\of{S}}{L^3}}
  \end{equation*}
  \begin{center}
    \includegraphics[width=.5\textwidth]{monadEquationGraph.png}
    % Annotation: s=K_S => \mu = \mu_{max}/2
  \end{center}

  \subsection*{Rate behavious based on concentration of substrate}

  if \(\mu\) is dependent on the substrate concentration as indicated by the Monod equation, how can \(\mu\) remain constant during the growth phase? Typical values for the substrate constant (\(K_S\)) are very small in order of \unit{\mg/\L} for carbonhydrate substrates and \unit{\micro\g/\litre} for other compounds such as aminoacids.\\

  The behaviour of specific growth rate (\(\mu\)) with the concentration of substrate in relation to the substrate constant \(K_S\) follows:
  \begin{center}
    \vspace{1ex}
    \begin{tabular}{l L @{} L @{\(\implies\)} L}
      \toprule

          Growth        & s\gtrsim  &  K_S\,10             & \mu\approx\mu_{\max}
      \\  Decline Start & s\in      &  K_S *\myrange{10,1} & \mu = \mu_{\max}/(1+K_S/s)
      \\  Decline       & s\approx  &  K_S                 & \mu \approx \mu_{\max{}}/2
      \\  Decline End   & s <       &  K_S                 & \mu = \mu_{\max{}}/(1+K_S/s)
      \\  Statinoary    & s\ll      &  K_S                 & \mu \approx 0

      \\\bottomrule
    \end{tabular}
    \vspace{2ex}
  \end{center}

  Some examples for the substrate constant \(K_S\):

  \begin{center}
    \vspace{-1ex}
    \sisetup{
      round-mode={none}, % figures/places/unsertanty/none
    }
    \begin{tabular}{>{\itshape}l l S}
      \toprule

      \multicolumn{1}{l}{Microoragnism (genus)}
      & Limiting Substrate
      & \multicolumn{1}{C}{K_S / (\unit{\mg/\L})}

      \\\midrule

      \multirow[t]{1}{*}{Saccgaromyces} 
      & Glucose & 25

      \\[1ex] \multirow[t]{3}{*}{Escherichia} 
      & Glucose & 4.0
      \\& Lactose & 20
      \\& Phosphate & 1.6

      \\[1ex] \multirow[t]{1}{*}{Aspergillus} 
      & Glucose & 5.0

      \\[1ex] \multirow[t]{2}{*}{Candida} 
      & Glycerol & 4.5
      \\ & Oxygen & \numrange{0.042}{0.45}
      % \\ & Oxygen & \multicolumn{1}{C}{\numrange{0.042}{0.45}}

      \\[1ex] \multirow[t]{2}{*}{Pseudomonas} 
      & Methanol & 0.7
      \\& Methane & 0.4

      \\[1ex] \multirow[t]{4}{*}{Klebsiella} 
      & Carbon dioxide & 0.4
      \\& Magnesium & 0.56
      \\& Potassium & 0.39
      \\& Sulphate & 2.7

      \\[1ex] \multirow[t]{2}{*}{Hansenula} 
      & Methanol & 120.0
      \\& Ribose & 3.0
      
      \\[1ex] \multirow[t]{1}{*}{Cryptococcus} 
      & Thiamine & 1.4e-7

      \\\bottomrule
    \end{tabular}
    \vspace{2ex}
  \end{center}


  \subsection*{Limit of Monod equation}
  The Monod equation is by far the most frequently used expression relating to growth rate to substrate concentration. However, it is \emph{valid only for balanced growth} and should not be applied when growth conditions are changing rapidly. There are also other restrictions; for example, the Monod equations has been found to have limited applicability at \emph{extremely low substrate levels}. \emph{When growth is inhibited by high substrate or product concentrations, extra terms can be added to the Monod equation} to account for these effects.\\
  Several other kinetic expressions has been developed for cell growth these provide better correlations for experimental data in centain situations
\end{sectionBox}

\setcounter{subpart}{9}
\subpart{Production Kinetics in Cell Culture}

\begin{sectionBox}*1{Rate of Product Formation} % S1
  \begin{BM}[equation]\label{eq:rateOfProductFormation}
    r_P = q_P\,x
  \end{BM}\vspace{-3ex}
  \begin{align*}
          \dim{r_P} = \unit{\frac{M/L^{3}}{T}}
    ;\quad\dim{x}   = \unit{M\of{B}/L^{3}}
    ;\quad\dim{q_P} = \unit{\frac{1}{T}}
  \end{align*}
  \begin{description}[
    leftmargin=!,
    labelwidth=\widthof{\(q_P\)} % Longest item
  ]
    \item[\(r_P\)] is the volumetric rate of production formation
    \item[\(x\)]   is the biomass concentration
    \item[\(q_P\)] is the \textit{specific rate of product formation}
  \end{description}

  \subsection*{Production rate \(q_P\)}
  \begin{itemize}
    \item can be evaluated at any time during fermentation as the \emph{ratio of the production rate and biomass concentration}.
    \item Is \emph{not necersarly constant} during batch culture
    \item We can develop equations for \(q_P\) as a \emph{function of growth rate and other metabolic parameters} (Depending on wheter the product is linked to energy metabolism or not)
  \end{itemize}

\end{sectionBox}

\begin{sectionBox}1{Product Formation \emph{Directly} Coupled with Energy Metabolism} % S12.10.1
  For products formed using pathways that generate ATP, the \emph{rate of production is related to the cellular energy demand}. Growth is usually the major energy-requiring function of cells; therefore, if production is coupled to energy metabolism, product will be formed whenever there is growth. However, ATP is also required for other activites called \textit{maintenance}. \emph{Maintenance activities are carried out by living cells even in the absence of growth}. \emph{Products synthetised in energy pathways will be produced whenever maintenance functions are carried out} because ATP is required.
  \subsubsection*{Examples of maintenance:}
  \begin{itemize}
    \item Cell motility
    \item Turnover of cellular components
    \item Adjustment of menbrane potential and internal pH
  \end{itemize}
  Kinetic expressions for the rate of product formation must account for \textcolor{Emph32}{growth-associated} and \textcolor{Emph33}{maintenance-associated} production, as in the following equation
  \begin{BM}[equation]\label{eq:rateOfProductFormation2}
    r_P 
    = \textcolor{Emph32}{Y_{PX}\,r_X} 
    + \textcolor{Emph33}{m_P\,x}
  \end{BM}
  \begin{equation*}
    \dim{m_P} = \unit{\frac{1}{T\,M\of{B}}}
  \end{equation*}
  \begin{description}[
    leftmargin=!,
    labelwidth=\widthof{\(Y_{PX}\)} % Longest item
  ]
    \item[\(r_X\)] is the volumetric rate of biomass formation.
    \item[\(Y_{PX}\)] is the theoretical or true yield of product from biomass
    \item[\(m_P\)] is the \textit{specific rate of product formation due to maintenance}
    \item[\(x\)] is the biomass concentration
  \end{description}

  \begin{BM}[equation]
    q_P = Y_{PX}\,\mu + m_P
  \end{BM}
  Using \eqref{eq:rateOfProductFormation2}
  ,     \eqref{eq:rateOfProductFormation}
  and   \eqref{eq:growRateInGrowthAndDecline}
  \begin{flalign*}
    &
      q_P 
      = \frac{r_P}{x}
      = \frac{Y_{PX}\,r_X + m_P\,x}{x}
      = \frac{Y_{PX}\,(\mu\,x) + m_P\,x}{x}
      = Y_{PX}\,\mu + m_P
    &
  \end{flalign*}

\end{sectionBox}

\stepcounter{section}

\begin{sectionBox}1{Product Formation \emph{Not} Coupled with Energy Metabolism} % S4
  Production not involving energy metabolism is difficult to relate to growth because growth and product synthesis are somewhat dissociated. However, \emph{in some cases}, the rate of formation of nongrowth-associated products is directly proportional to biomass concentration, so that \emph{the production rate defined in \eqref{eq:rateOfProductFormation} can be applied with constant \(q_P\)}. Sometimes \(q_P\) is a complex function of the growth rate and must be expressed using empirical equations derived from experiment.
\end{sectionBox}

\end{document}

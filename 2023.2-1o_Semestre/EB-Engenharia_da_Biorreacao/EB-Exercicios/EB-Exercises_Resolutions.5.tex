% !TEX root = ./EB-Exercises_Resolutions.5.tex
\providecommand\mainfilename{"./EB-Exercises_Resolutions.tex"}
\providecommand \subfilename{}
\renewcommand   \subfilename{"./EB-Exercises_Resolutions.5.tex"}
\documentclass[\mainfilename]{subfiles}

% \tikzset{external/force remake=true} % - remake all

\begin{document}

% \graphicspath{{\subfix{./.build/figures/EB-Exercises_Resolutions.2}}}
% \tikzsetexternalprefix{./.build/figures/EB-Exercises_Resolutions.2/graphics/}

\mymakesubfile{5}
[EB]
{Resolução lista} % Subfile Title
{Resolução lista} % Part Title

\begin{questionBox}1{ % Q1
    Pretende-se operar um fermentador cilíndrico a uma temperatura de 40\,\unit{\celsius} e a uma taxa de arejamento de 0.02\,\unit{\centi\metre^3.\centi\metre^{-3}.\second^{-1}}. Considerando que o fermentador tem um diâmetro interno de 40\,\unit{\centi\metre}, uma altura de 2\,\unit{\metre} e um diâmetro de orifício de passagem de ar com 0.65\,\unit{\milli\metre}, calcule:\\
    A velocidade máxima de transferência de oxigénio para o meio de cultura com as seguintes características:
    \begin{itemize}
        \item Densidade do meio de cultura: 1\,\unit{\gram.\centi\metre^{-1}.\second^{-1}}
        \item Viscosidade do meio de cultura: \(\mu=1.5\E-2\,\unit{\gram.\centi\metre^{-1}.\second^{-1}}\)
        \item Tensão superfícial: 72\,\unit{\gram.\second^{-2}}
        \item Densidade do gás: \(\rho_g=1.4\E-3\,\unit{\gram,\centi\metre^{-3}}\)
        \item Difusividade do oxigénio: \(2\E-5\,\unit{\centi\metre^2.\second^{-1}}\)
        \item Concentração de equilíbrio do oxigénio no meio líquido a 40\unit{\celsius}: \(1.03\,\unit{\milli\molar}\)
    \end{itemize}
} % Q1
    \paragraph*{Nota:}Utilize a lei de Newton no cálculo da velocidade terminal:
    \begin{BM}
        v_t=\sqrt{
            \frac{3.33\,g\,\adif{\rho}}{\rho_L}
            \,D_P
        }
    \end{BM}
\end{questionBox}

\end{document}
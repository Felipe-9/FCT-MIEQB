% !TEX root = ./EB-Exercises_Resolutions.6.tex
\providecommand\mainfilename{"./EB-Exercises_Resolutions.tex"}
\providecommand \subfilename{}
\renewcommand   \subfilename{"./EB-Exercises_Resolutions.6.tex"}
\documentclass[\mainfilename]{subfiles}

% \tikzset{external/force remake=true} % - remake all

\begin{document}

% \graphicspath{{\subfix{./.build/figures/EB-Exercises_Resolutions.6}}}
% \tikzsetexternalprefix{./.build/figures/EB-Exercises_Resolutions.6/graphics/}

\mymakesubfile{6}
[EB]
{Exercises} % Subfile Title
{Exercises} % Part Title

\begin{questionBox}1{ % Q1
    A \emph{150\,\unit{\metre^3}} reactor is operated at \emph{35\,\unit{\celsius}} to produce biomass from glucose. The \emph{\ch{O2} consumption rate is 1.5\,\unit{\kilo\gram.\metre^{-3}.\hour^{-1}}}. The stirrer \emph{dissipates heat at the speed of 1\,\unit{\kilo\watt.\metre^{-3}}}. The \emph{cooling water flows at a temperature of 10\,\unit{\celsius}} and at a \emph{flow rate of 60\,\unit{\metre^3/\hour}}, it passes inside a coil placed inside the reactor. If the system operates in steady state, \emph{determine the temperature of the cooling water leaving the reactor}.
} % Q1
    \paragraph*{Data:}
    \begin{itemize}
        \begin{multicols}{2}
            \item \(Q=460\,\unit{\kilo\joule.\mole\of{O2}^{-1}}\)
            \item \(Cp_{\ch{H2O}}=75.4\,\unit{\joule.\mole^{-1}.\celsius^{-1}}\)
        \end{multicols}
    \end{itemize}
    \answer{}
    \begin{flalign*}
        &
            T_1
            = T_0+\adif{T}
            ; &\\[3ex]&
            \adif{H}
            =\adif{H}_{rxn}+W_s
            ; &\\[3ex]&
            \adif{H}_{rxn}
            = Q*C_{\ch{O2}}*V
            = \left(
                \begin{aligned}
                    &
                        460\,\unit{\kilo\joule.\mole^{-1}}
                    &*\\*&
                        1.5\,\unit{\kilo\gram.\metre^{-3}.\hour^{-1}}
                    &*\\*&
                        3600^{-1}\,\unit{\hour.\second^{-1}}
                    &*\\*&
                        32^{-1}\,\unit{\mole.\gram^{-1}}
                    &*\\*&
                        150\,\unit{\metre^3}
                    &
                \end{aligned}
            \right)
            \cong\qty{898.4375}{\kilo\joule.\second^{-1}}
            ; &\\[3ex]&
            W_s
            = 1\,\unit{\kilo\watt.\metre^{-3}}
            \,150\,\unit{\metre^3}
            = 150\,\unit{\kilo\watt}
            ; &\\[3ex]&
            \adif{H}
            \cong
            \qty{898.4375}{\kilo\joule.\second^{-1}}
            +150\,\unit{\kilo\joule.\second^{-1}}
            \cong
            \qty{1048.4375}{\kilo\joule.\second^{-1}}
            = &\\[3ex]&
            = M\,Cp\,\adif{T}
            = &\\&
            = (v*\rho_{\ch{H2O}})
            \,\left(
                75.4\,\unit{\joule.\mole^{-1}.\celsius^{-1}}
                \,\frac
                    {\unit{\mole}}
                    {M_{w\,\ch{H2O}}\,\unit{\gram}}
            \right)
            \,\adif{T}
            = &\\&
            = (60*1000\,\unit{\kilo\gram/\hour})
            \,\left(
                \frac{75.4}{18}
                \,\unit{\kilo\joule.\kilo\gram^{-1}.\celsius^{-1}}
            \right)
            \,\adif{T}
            \implies &\\[3ex]&
            \implies
            T_1/\unit{\celsius}
            = T_0+\adif{T}
            \cong 
            10+
            \left(
                \frac{
                    \num{1048.4375}
                    \,\unit{\kilo\joule.\second^{-1}}
                }{
                    \frac{75.4*60000}{18}
                    \,\unit{\kilo\joule.\celsius^{-1}.\hour^{-1}}
                    \,\frac
                        {\unit{\hour}}
                        {3600\,\unit{\second}}
                }
            \right)
            \cong
            \num{25.017407161803716}
        &
    \end{flalign*}
\end{questionBox}

\begin{questionBox}1{ % Q2
    A fermenter used to produce an antibiotic should have a temperature of \emph{27\,\unit{\celsius}}. After considering the oxygen requirements of the microorganisms and the heat dissipated by the stirrer, the maximum amount of \emph{heat to be transferred was estimated at 550\,\unit{\kilo\watt}}. The \emph{cooling water enters at a temperature of 10\,\unit{\celsius} and leaves at 25\,\unit{\celsius}}. The heat transfer coefficient in the fermentation fluid was estimated at 2150\,\unit{\watt.\metre^{-2}.\celsius{-1}} and the heat transfer coefficient of the cooling water has the value of 14000\,\unit{\watt.\metre^{-2}.\celsius^{-1}}. The steel cooling coil has an outer diameter of 8\,\unit{\centi\meter} and a thickness of 5\,\unit{\milli\metre}, the thermal conductivity of steel is 60\,\unit{\watt.\metre^{-1}.\celsius^{-1}}. Calculate the length of coil needed under these conditions.
} % Q2
    \answer{}
    \begin{flalign*}
        &
            \adif{T}
            = \frac{2\,T_{fluido}-(T_1+T_2)}{2}
            = \frac{2*28-(T_1+T_2)}{2}
            = 9.5\,\unit{\celsius}
            ; &\\[3ex]&
            Q=h\,A\,\adif{T}
            ; &\\[3ex]&
            h^{-1}
            = h_i^{-1}
            + h_e^{-1}
            + h_w^{-1}\,B
            = 2150^{-1}
            + 14000^{-1}
            + 60^{-1}
            * 5\E{-3}
            \cong
            1614\,\unit{\metre^{-2}.\celsius^{1}.\watt}
            &\\[3ex]&
            A
            =\frac{Q}{h\,\adif{T}}
            \cong\frac{550}{1614*9.5}
            \cong\num{35.870345007500165}
            \implies &\\[3ex]&
            \implies
            A=2\,\pi\,r\,l
            \implies
            l
            =\frac{A}{2\,\pi\,r}
            \cong
            \frac{\num{35.870345007500165}}{2\,\pi\,(8/2)}
            \cong
            142.8\,\unit{\metre}
        &
    \end{flalign*}
\end{questionBox}

\end{document}
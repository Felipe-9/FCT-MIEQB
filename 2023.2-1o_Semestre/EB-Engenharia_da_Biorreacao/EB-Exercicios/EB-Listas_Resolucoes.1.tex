% !TEX root = ./EB-Listas_Resolucoes.1.tex
\providecommand\mainfilename{"./EB-Listas_Resolucoes.tex"}
\providecommand \subfilename{}
\renewcommand   \subfilename{"./EB-Listas_Resolucoes.1.tex"}
\documentclass[\mainfilename]{subfiles}

% \tikzset{external/force remake=true} % - remake all

\begin{document}

% \graphicspath{{\subfix{./.build/figures/EB-Listas_Resolucoes.1}}}
% \tikzsetexternalprefix{./.build/figures/EB-Listas_Resolucoes.1/graphics/}

\mymakesubfile{1}
[EB]
{Exercicios} % Subfile Title
{Exercicios} % Part Title

\begin{questionBox}1{ % Q1
    Consider the culture of a bacterium with the following empirical formula:
    \begin{center}
        \ch{CH_{1.7}O_{0.46}N_{0.18}}
        (\(\text{M}_{w\,B} = 23.6\,\si{\gram/\mole}\))
    \end{center}
    This bacterium grows aerobically in a culture medium using glucose as a carbon source (\ch{C6H12O6}).\\
    Glucose and oxygen yield coefficients were experimentally determined:
    \begin{BM}[align*]
        Y_{X/S} &= 85\,\si{\gram\of{biomass}/\mole\of{glucose}}
        \\
        Y_{X/O2} &= 39\,\si{\gram\of{biomass}/\mole\of{O2}}
    \end{BM}
    This organism does not excrete appreciable amounts of metabolites under growing conditions.
} % Q1
    \begin{questionBox}2{ % Q1.1
        Show that the measured values of \ch{Y_{X/S}} and \ch{Y_{X/O2}} are consistent.
    } % Q1.1
        \answer{}
        \begin{center}\large
            \ch{
                a C6H12O6 % S:Glucose
                + b O2 % Aerobic:TRUE
                + c NH3 % N source
                -> 
                CH_{1.7}O_{0.46}N_{0.18} % X:Biomass
                + d CO2 + e H2O % Resp products
            }
        \end{center}
        \begin{flalign*}
            &
                Y_{X/S}
                =\frac
                {85\,\si{\gram\of{X}}}
                {\si{\mole\of{S}}}
                = \frac
                {1\,\si{\mole\of{X}}}
                {a\,\si{\mole\of{S}}}
                \,\frac
                {M_{w\,B}\,\si{\gram\of{X}}}
                {1\,\si{\mole\of{X}}}
                \implies &\\&
                \implies
                % 
                a 
                = M_{w\,B}/85 
                = 23.6/85 
                \cong\num{0.2776470588}
                &\\[3ex]&
                Y_{X/\ch{O2}}
                = \frac
                    {39\,\si{\gram\of{X}}}
                    {\si{\mole\of{\ch{O2}}}}
                = \frac
                    {1*M_{w\,B}\,\si{\gram\of{X}}}
                    {b\,\si{\mole\of{\ch{O2}}}}
                \implies &\\&
                \implies
                b
                = \frac{M_{w\,B}}{39}
                = \frac{23.6}{39}
                \cong \num{0.605128205128205}
            &
        \end{flalign*}
        \paragraph*{Balanço Mássico}
        \begin{flalign*}
            &
                \begin{cases}
                    \ch{C}:& 6\,a = 1+d
                    \\
                    \ch{O}:& 6\,a+2\,b=0.46+2\,d+e
                    \\
                    \ch{N}:& c=0.18
                    \\
                    \ch{H}:& 12\,a+3\,c=1.7+2\,e
                \end{cases}
                &\\[3ex]&
                \therefore
                % D
                d\cong6*\num{0.2776470588}-1\cong\num{0.6658823528}
                ; &\\[3ex]&
                % B
                2\,(6\,a+2\,b)-(12\,a+3\,c)
                = 4\,b-3\,c
                = &\\&
                = 2\,(0.46+2\,d+e)-(1.7+2\,e)
                = 0.92+4\,d-1.7
                \implies &\\&
                \implies
                b
                \cong \frac{
                    0.92
                    +4*\num{0.6658823528}
                    -1.7
                    +3*0.18
                }{4}
                \cong\num{0.6058823528}\cong\num{0.605128205128205}
            &
        \end{flalign*}
        \paragraph*{Balanço Energetico}
        \begin{flalign*}
            &
                \begin{cases}
                    \gamma_S
                    =  {
                           6*\gamma_{\ch{C}}
                        + 12*\gamma_{\ch{H}}
                        +  6*\gamma_{\ch{O}}
                    }
                    = 6*4+12*1+  6*(-2)
                    = 24
                    \\
                    \gamma_{\ch{O2}}
                    = 2*\gamma_{\ch{O}}
                    = -4
                    \\
                    \gamma_{\ch{NH3}}
                    = {
                        1*\gamma_{\ch{N}}
                        + 3*\gamma{\ch{H}}
                    } = 1*(-3)+3*1 = 0
                    \\
                    \gamma_{X}
                    = {
                           1   *\gamma_{\ch{C}}
                        +  1.7 *\gamma_{\ch{H}}
                        +  0.46*\gamma_{\ch{O}}
                        +  0.18*\gamma_{\ch{N}}
                    }
                    = 4.24
                    \\
                    \gamma_{\ch{CO2}}
                    = {
                          1*\gamma_{\ch{C}}
                        + 2*\gamma_{\ch{O}}
                    }
                    = 1*4+ 2*(-2) = 0
                    \\
                    \gamma_{\ch{H2O}}
                    = 2*\gamma_{\ch{H}}+\gamma_{\ch{O}}
                    = 0
                \end{cases}
                &\\[3ex]&
                a\,\gamma_{S}
                + b\,\gamma_{\ch{O2}}
                + c\,\gamma_{\ch{NH3}}
                = a\,24
                + b\,-4
                = &\\&
                = \gamma_{X}
                + d\,\gamma_{\ch{CO2}}
                + e\,\gamma_{\ch{H2O}}
                = 4.24
                \implies &\\&
                \implies
                b
                =(24*a-4.24)/4
                \cong(24*\num{0.2776470588}-4.24)/4
                \cong &\\&
                \cong\num{0.6058823528}
                \cong\num{0.605128205128205}
            &
        \end{flalign*}
    \end{questionBox}

    \begin{questionBox}2{ % Q1.2
        A batch culture of this organism initially contains 0.01\,\si{\gram\of{biomass}} and 20\,\si{\milli\mole\of{glucose}}. After a few hours of cultivation, the cells stopped growing. The total biomass in the culture is 1.0\,\si{\gram}. Estimate the final amount of glucose in the culture medium (in \si{\milli\mole}) and speculate on the likely cause of cell growth arrest.
    } % Q1.2
        \answer{}
        \begin{flalign*}
            &
                Y_{X/S}
                =\frac
                    {85\,\si{\gram\of{X}}}
                    {\si{\mole\of{S}}}
                =-\adv{X}{S}
                =\frac{X_1-X_0}{S_0-S_1}
                \implies &\\&
                \implies
                S_1/\si{\milli\mole}
                = S_0-\frac{X_1-X_0}{85*10^{-3}}
                = 20-\frac{1-0.01}{85*10^{-3}}
                \cong\num{8.352941176470588}
            &
        \end{flalign*}
        The non nule amount of glucose indicates that the arrest was due to the lack of another nutrient: \ch{NH3,O2}, micronutrients
    \end{questionBox}
\end{questionBox}

\begin{questionBox}1{ % Q2
    Consider an anaerobic fermentation by a yeast whose empirical biomass formula is as follows:
    \begin{center}
        \ch{CH_{1.7}O_{0.45}N_{0.15}} \((M_{w\,B} = 23.0\,\si{\gram\of{DCW}/\mole})\)
    \end{center}
    Carbon and nitrogen sources are glucose (\ch{C6H12O6}) and ammonium salts, respectively. Possible products of the growth reaction are biomass, ethanol (\ch{C2H6O}), carbon dioxide and water. Ethanol growth and formation depend on growing conditions.
} % Q2
    \answer{}
    \begin{center}\large
        \ch{
            a C6H12O6
            + b NH3
            -> 
            x CH_{1.7}O_{0.45}N_{0.15}
            + c C2H6O
            + d CO2
            + e H2O
        }
    \end{center}
    \begin{questionBox}2{ % Q2.1
        What is the maximum biomass yield coefficient (\ch{Y_{X/S}}\,\si{\gram\of{DCW}/\mole\of{glucose}}), and under what conditions is it obtainable?
    } % Q2.1
        \answer{}
        \(DCW=X\)
        % Y_{X/S\,\max} => no ethanol production
        \begin{flalign*}
            &
                Y_{X/S\,\max}
                = \frac
                    {x\si{\mole\of{X}}}
                    {a\,\si{\mole\of{S}}}
                \,\frac
                    {23.0\,\si{\gram\of{X}}}
                    {\si{\mole\of{X}}}
                ; &\\[3ex]&
                \text{Balanço energético:}&\\&
                \begin{cases}
                    \gamma_{\ch{NH3}}
                    = \gamma_{\ch{CO2}}
                    = \gamma_{\ch{H2O}}
                    = 0
                    \\
                    \gamma_{S}
                    = 6*4+12+6*(-2) = 24
                    \\
                    \gamma_{X}
                    = 4+1.7+0.45*(-2)+0.15*(-3)
                    = 4.35
                \end{cases}
                &\\&
                Y_{X/S\,\max}: c=0
                &\\&
                \therefore a\,\gamma_{S}=x\,\gamma_{X}
                \implies
                a = x\,24/4.35
                \cong x\,\num{0.18125000000034}
                \implies &\\&
                \implies
                \frac{Y_{X/S\,\max}}{
                    \si{\gram\of{X}/\mole\of{S}}
                }
                = x\,23/x\,\num{0.18125000000034}
                \cong\num{126.896551723899891}
            &
        \end{flalign*}
    \end{questionBox}

    \begin{questionBox}2{ % Q2.2
        What is the maximum ethanol yield coefficient (\ch{Y_{E/S}}\,\si{\mole\of{\ch{EtOH}}/\mole\of{glucose}}), and under what conditions is it obtainable?
    } % Q2.2
        \answer{}
        EtOH=E
        \begin{flalign*}
            &
                Y_{E/S\,\max}
                = \frac
                    {c\,\si{\mole\of{E}}}
                    {a\,\si{\mole\of{X}}}
                &\\[3ex]&
                \text{Balanço Energético}&\\&
                \begin{cases}
                       \gamma_{S}= 24
                    \\ \gamma_{X}= 4.35
                    \\ \gamma_{E}= 2*4+6-2=12
                \end{cases}
                &\\&
                Y_{E/S\,\max}\implies x=b=0;
                &\\&
                a\,\gamma_X=c\,\gamma_E
                \implies
                a
                =c\,\gamma_E/\gamma_X
                =c\,12/24=c/2
                \implies &\\[3ex]&
                \implies
                Y_{E/S\,\max}
                = \frac
                    {c\,\si{\mole\of{E}}}
                    {(c/2)\,\si{\mole\of{X}}}
                = 2\,\si{\frac{\mole\of{E}}{\mole\of{X}}}
            &
        \end{flalign*}
    \end{questionBox}
\end{questionBox}

\begin{questionBox}1{ % Q3
    Considere uma cultura em descontínuo de determinado microrganismo, cuja fórmula empírica da biomassa é a seguinte
    \begin{center}\large
        \ch{CH_{1.6}O_{0.55}N_{0.20}} (\(M_{w\,B}=25.2\,\si{\gram/\mole}\))
    \end{center}
    The culture medium contains 10\,\si{\milli\mole\of{S}} glucose (\ch{C6H12O6}) and ammonium sulfate in large excess. During the cultivation time there was an effective growth of 0.3\,\si{\gram} of dry weight and a total consumption of 15\,\si{\milli\mole\of{\ch{O2}}}.\\
    \textit{This organism does not excrete appreciable amounts of metabolites under growing conditions.}
} % Q3
    \answer{}
    \begin{center}
        \ch{
            a C6H12O6
            + b O2
            + c NH3
            -> 
            x CH_{1.6}O_{0.55}N_{0.20}
            + d CO2
            + e H2O
        }
    \end{center}
    \begin{questionBox}2{ % Q3.1
        Estimate the substrate yield coefficient, \(Y_{X/S}\) (\si{\gram\of{X}/\mole\of{S}}), and the final amount of glucose in the medium (\si{\milli\mole}).
    } % Q3.1
        \answer{}
        \begin{flalign*}
            &
                Y_{X/S}
                : Y_{X/S}
                = \frac
                    {x\,\si{\mole\of{X}}}
                    {a\,\si{\mole\of{S}}}
                \,\frac
                    {25.2\,\si{\gram\of{X}}}
                    {\si{\mole\of{X}}}
                = &\\[3ex]&
                S_1
                : Y_{X/S} 
                = -\adv{X}{S}
                = \frac{\adif{X}}{S_0-S_1};
                &\\[3ex]&
                \text{Balanço Energético}&\\&
                \begin{cases}
                    \gamma_S = 6*4+12-6*2 &= 24
                    \\
                    \gamma_X = 4+1.6-0.55*2-0.20*3 &= 3.9
                    \\
                    \gamma_{\ch{O2}} &= -4
                    \\
                    \gamma_{\ch{NH3}} 
                    = \gamma_{\ch{CO2}} 
                    = \gamma_{\ch{H2O}} 
                    &= 0
                \end{cases}
                &\\&
                a: a\,\gamma_S
                + b\,\gamma_{\ch{O2}}
                = a\,24
                - b\,4
                = &\\&
                = x\,\gamma_X
                = x\,3.9
                &\\[3ex]&
                a:b:
                Y_{X/\ch{O2}}
                = \frac
                    {x\,\si{\mole\of{X}}}
                    {b\,\si{\mole\of{\ch{O2}}}}
                \,\frac{25.2\,\si{\gram\of{X}}}{\si{\mole\of{X}}}
                = \adv{X}{\ch{O2}}
                = \frac
                    {0.3\,\si{\gram\of{X}}}
                    {15\E-3\,\si{\mole\of{\ch{O2}}}}
                \implies &\\&
                \implies
                b
                = \frac
                    {x*25.2*15\E-3}
                    {0.3}
                = x\,1.26
                \implies &\\&
                \implies
                a 
                = \frac{x\,3.9+b*4}{24}
                = \frac{x\,3.9+x\,1.26*4}{24}
                = x\,0.3725
                \implies &\\[3ex]&
                \implies
                \frac{Y_{X/S}}{\si{\gram\of{X}/\mole\of{S}}}
                = \frac{25.2\,x}{a}
                = \frac{25.2\,x}{x\,0.3725}
                \cong\num{67.651006711409398}
                \land &\\[3ex]&
                \land
                S_1/\si{\milli\mole}
                = S_0/\si{\milli\mole}-\frac{\adif{X}}{Y_{X/S}\E-3}/\si{\milli\mole}
                \cong &\\&
                \cong 
                10 - \frac{0.3}{\num{67.651006711409398}\E-3}
                \cong
                \num{5.565476190476191}
            &
        \end{flalign*}
    \end{questionBox}
    \begin{questionBox}2{ % Q3.2
        Estimate how much \ch{CO2} was produced (\si{\milli\mole}).
    } % Q3.2
        \answer{}
        \begin{flalign*}
            &
                X_{\ch{CO2}\,1}:
                Y_{\ch{CO2}/X}
                = \adv{X_{\ch{CO2}}}{X}
                = \frac{X_{\ch{CO2}\,1}-X_{\ch{CO2}\,0}}{\adif{X}}
                = \frac{X_{\ch{CO2}\,1}}{0.3}
                = &\\&
                = \frac
                    {d\,\si{\mole\of{\ch{CO2}}}}
                    {x\,\si{\mole\of{X}}}
                \,\frac
                    {\si{\mole\of{X}}}
                    {25.2\,\si{\gram\of{X}}}
                ; &\\[3ex]&
                \text{Balanço Mássico:}&\\&
                \begin{cases}
                    \ch{C}: a*6=x+d
                \end{cases}
                &\\&
                d
                =a*6-x
                =x\,0.3725*6-x
                =x\,1.235
                \implies &\\[3ex]&
                \implies
                \frac{X_{\ch{CO2}\,1}}{\si{\milli\mole\of{\ch{CO2}}}}
                = 0.3\frac{d\E3}{25.2\,x}
                = 0.3\frac{x\,1.235\E3}{25.2\,x}
                = &\\&
                = \num{14.702380952380953}
            &
        \end{flalign*}
    \end{questionBox}
\end{questionBox}

\begin{questionBox}1{ % Q4
    In a bacterial culture, pyruvate (\ch{C3H4O3}) is used as a carbon source for growth. The source of nitrogen is ammonia salts. The empirical formula for biomass is:
    \begin{center}\large
        \ch{CH_{1.8}O_{0.5}N_{0.17}} 
        (\(M_{w\,B} = 24.2\,\si{\gram\of{DCW}/\mole}\)).
    \end{center}
} % Q4
    \answer{}
    \begin{center}
        \ch{
            a C3H4O3
            + b NH3
            + c O2
            -> 
            x CH_{1.8}O_{0.5}N_{0.17}
            + d CO2
            + e H2O
        }
    \end{center}
    \begin{questionBox}2{ % Q4.1
        Based on the above information, estimate the maximum theoretical biomass yield per mole of pyruvate (\si{\gram\of{DCW}/\mole}).
    } % Q4.1
        \answer{}
        \begin{flalign*}
            &
                Y_{X/S\,\max}:
                Y_{X/S\,\max}
                = \frac
                    {x\,\si{\mole\of{X}}}
                    {a\,\si{\mole\of{S}}}
                \,\frac
                    {24.2\,\si{\gram\of{X}}}
                    {\si{\mole\of{X}}}
                ; &\\&
                Y_{X/S\,\max}
                \implies
                b=d=e=0
                ; &\\[3ex]&
                \text{Balanço Energético:}&\\&
                \begin{cases}
                    \gamma_{S}=3*4+4-3*2=10
                    \\
                    \gamma_{X}=4+1.8-0.5*2-0.17*3=4.29
                    \\
                    \gamma_{\ch{NH3}}
                    % = \gamma_{\ch{CO2}}
                    % = \gamma_{\ch{H2O}}
                    = 0
                \end{cases}
                &\\&
                a:
                a\,\gamma_{S}
                = a\,10
                = &\\&
                =x\,\gamma_{X}
                =x\,4.29
                \implies
                a=x\,0.429
                \implies &\\[3ex]&
                \implies
                \frac{Y_{X/S\,\max}}{
                    \si{\gram\of{X}/\mole\of{S}}
                }
                = \frac{x\,24.2}{a}
                = \frac{x\,24.2}{x\,0.429}
                \cong\num{56.41025641025641}
            &
        \end{flalign*}
    \end{questionBox}

    The culture described above is performed aerobically and excretion of metabolites into the extracellular medium was not detected. It was determined that 45\,\si{\milli\mole\of{\ch{CO2}}} is released for every \si{\gram\of{DCW}} of biomass produced.
    \begin{questionBox}2{ % Q4.2
        Estimate current biomass yield per mole of pyruvate (\si{\gram\of{DCW}/\mole}).
    } % Q4.2
        \answer{}
        \begin{flalign*}
            &
                Y_{X/S}:
                Y_{X/S}
                = \frac
                    {x\,24.2\,\si{\gram\of{X}}}
                    {a\,\si{\mole\of{S}}}
                ; &\\[3ex]&
                \text{Balanço Mássico:}
                \begin{cases}
                    \ch{C}:& a*3=x+d
                \end{cases}
                &\\[3ex]&
                a:d:
                Y_{\ch{CO2}/X}
                = \frac
                    {d\,\si{\mole\of{\ch{CO2}}}}
                    {x\,24.2\,\si{\mole\of{X}}}
                = \frac
                    {45\,\si{\milli\mole\of{\ch{CO2}}}}
                    {\si{\gram\of{X}}}
                \implies &\\&
                \implies
                d = x\,1.089
                \implies &\\[3ex]&
                \implies
                a
                =(x+d)/3
                =(x+x\,1.089)/3
                \cong x\,\num{0.696333333333333}
                \implies &\\[3ex]&
                \frac{Y_{X/S}}{
                    \si{\gram\of{X}/\mole\of{S}}
                }
                \cong \frac
                    {x\,24.2}
                    {x\,\num{0.696333333333333}}
                \cong\num{34.753470560076592}
            &
        \end{flalign*}
    \end{questionBox}
    \begin{questionBox}2{ % Q4.3
        Explain the difference in results obtained in the items a) and b).
    } % Q4.3
        \answer{}
        The second case uses the substrate only for growing whilas the second part is used on the cell respiration and maintenance thus the lower growth.
    \end{questionBox}
\end{questionBox}

\begin{questionBox}1{ % Q5
    The chemical reaction for glucose respiration is as follows:
    \begin{center}\large
        \ch{C6H12O6 + 6 O2 -> 6 CO2 + 6 H2O}
    \end{center}
    \textit{Candida utilis} yeast converts glucose to \ch{CO2} and \ch{H2O} as it grows. Its empirical formula is:
    \begin{center}\large
        \ch{CH_{1.84}O_{0.55}N_{0.2}} plus 5\% (\textit{w/w}) ash.
    \end{center}
    The substrate biomass yield is 0.5 (\textit{w/w}). The source of nitrogen is ammonia.
} % Q5
    \answer{}
    \begin{center}
        \ch{
            a C6H12O6
            + b O2
            + c NH3
            -> 
            x CH_{1.84}O_{0.55}N_{0.2}
            + d CO2
            + e H2O
        }
    \end{center}
    \begin{questionBox}2{ % Q5.1
        Formulate the free electron balance equation.
    } % Q5.1
        \answer{}
        \begin{flalign*}
            &
                \text{Balanço Energético:} &\\&
                \begin{cases}
                    \gamma_{S}=6*4+12-6*2=24
                    \\
                    \gamma_{X}=4+1.85-0.55*2-0.2*3=4.15
                    \\
                    \gamma_{\ch{O2}}=-4
                    \\
                    \gamma_{\ch{NH3}}
                    = \gamma_{\ch{CO2}}
                    = \gamma_{\ch{H2O}}
                    = 0
                \end{cases}
                &\\&
                a\,\gamma_S
                + b\,\gamma{\ch{O2}}
                = a\,24
                - b\,4
                = &\\&
                = x\,\gamma_X
                = x\,4.15
            &
        \end{flalign*}
    \end{questionBox}
    \begin{questionBox}2{ % Q5.2
        Assess oxygen requirements (oxygen/glucose yield) when growth occurs or when cells only breathe glucose (i.e., cell maintenance).
    } % Q5.2
        \answer{}
        \subsubquestion{}
        \begin{flalign*}
            &
                Y_{\ch{O2}/S}:
                Y_{\ch{O2}/S}
                = \frac
                    {b\,\si{\mole\of{\ch{O2}}}}
                    {a\,\si{\mole\of{S}}}
                = \frac
                    {((a\,24-x\,4.15)/4)\,\si{\mole\of{\ch{O2}}}}
                    {a\,\si{\mole\of{S}}}
                = &\\&
                = \frac
                    {(6-x\,1.0375/a)\,\si{\mole\of{\ch{O2}}}}
                    {\si{\mole\of{S}}}
                ; &\\[3ex]&
                a:
                Y_{X/S}
                = \frac
                    {x\,\si{\mole\of{X}}}
                    {a\,\si{\mole\of{S}}}
                \,\frac
                    {25.4663*1.05\,\si{\gram\of{X}}}
                    {\si{\mole\of{X}}}
                \,\frac
                    {\si{\mole\of{S}}}
                    {180.56\,\si{\gram\of{S}}}
                = \frac
                    {0.5\,\si{\gram\of{X}}}
                    {\si{\gram\of{S}}}
                \implies &\\&
                \implies
                a
                = \frac
                    {x\,25.4663*1.05}
                    {0.5*180.56}
                \cong x\,\num{0.296185367744794}
                \implies &\\[3ex]&
                \implies
                \frac{Y_{\ch{O2}/S}}{
                    \si{\mole\of{\ch{O2}}/\mole\of{S}}
                }
                = 6-\frac{x\,1.0375}{a}
                \cong &\\&
                \cong 6-\frac{x\,1.0375}{x\,\num{0.296185367744794}}
                \cong \num{2.497126080536313}
            &
        \end{flalign*}
        \subsubquestion{}
        \begin{flalign*}
            &
                Y_{\ch{O2}/S}:
                Y_{\ch{O2}/S}
                = \frac
                    {b\,\si{\mole\of{\ch{O2}}}}
                    {a\,\si{\mole\of{S}}}
                ; &\\[3ex]&
                \text{Balanço Energético:}&\\&
                \begin{cases}
                    \gamma_{S}=24
                    \\
                    \gamma_{\ch{O2}}=-4
                    \\
                    \gamma_{\ch{CO2}}
                    = \gamma_{\ch{H2O}}
                    = 0
                \end{cases}
                &\\&
                \text{Manteinance: } c=x=0;
                &\\[3ex]&
                b:
                a\,\gamma_S+b\,\gamma_{\ch{O2}}
                = a\,24-b\,4
                =0
                \implies
                b=a\,6
                \implies &\\[3ex]&
                \implies
                Y_{\ch{O2}/S}
                = \frac
                    {6\,\si{\mole\of{\ch{O2}}}}
                    {\si{\mole\of{S}}}
            &
        \end{flalign*}
    \end{questionBox}
    \begin{questionBox}2{ % Q5.3
        \textit{C. utilis} is also capable of using ethanol as a carbon source. Compare the maximum thermodynamic yields of ethanol growth and glucose growth.
    } % Q5.3
        \answer{}
        \begin{itemize}
            % \begin{multicols}{2}
                \item For glucose: 
                    \(S_0\), 
                    \ch{
                        a_0 C6H12O6
                        + c_0 NH3
                        -> 
                        x_0 CH_{1.84}O_{0.55}N_{0.2}
                    }
                \item for ethanol: \(S_1\),
                    \ch{
                        a_1 C2H6O
                        + c_1 NH3
                        -> 
                        x_1 CH_{1.84}O_{0.55}N_{0.2}
                    }
            % \end{multicols}
        \end{itemize}
        \begin{flalign*}
            &
                \frac{Y_{X/S_0}}{Y_{X/S_1}}:
                Y_{X/S_i}
                = \frac
                    {x_i\,\si{\mole\of{X}}}
                    {a_i\,\si{\mole\of{S_i}}}
                ; &\\[3ex]&
                \text{Balanço Energético:}
                \begin{cases}
                        \gamma_{S_0} = 24
                    \\  \gamma_{S_1} = 2*4+6-2 = 12
                    \\  \gamma_{X}=4.15
                    \\  \gamma_{\ch{NH3}}=0
                \end{cases}
                &\\&
                a_i:
                a_i\,\gamma_{S_i}=x_i\,\gamma_{X}
                \implies &\\[3ex]&
                \implies
                \frac{Y_{X/S_0}}{Y_{X/S_1}}
                = \frac{
                    \frac
                    {x_0\,\si{\mole\of{X}}}
                    {(x_0\,4.15/24)\,\si{\mole\of{S_0}}}
                }{
                    \frac
                    {x_1\,\si{\mole\of{X}}}
                    {(x_1\,4.15/12)\,\si{\mole\of{S_1}}}
                }
                = \frac
                    {2\,\si{\mole\of{S_1}}}
                    {\si{\mole\of{S_0}}}
            &
        \end{flalign*}
        Glucose can produce the double of biomass compared to ethanol, its more efficient
    \end{questionBox}
\end{questionBox}

\setcounter{question}{6}

\begin{questionBox}1{ % Q7
    A recombinant protein is produced using genetically modified Escherichia coli. It was found that protein formation is proportional to cell growth. Glucose and ammonia are used as carbon and nitrogen sources respectively. The empirical formulas for biomass and protein are \ch{CH_{1.77}O_{0.49}N_{0.24}} and \ch{CH_{1.55}O_{0.31}N_{0.25}} respectively. It was experimentally determined that the biomass to glucose yield is 0.48 (\si{w/w}) and that the protein to glucose yield is 0.096 (\si{w/w})
} % Q1.7
    \begin{questionBox}2{ % Q7.1
        Assess ammonia requirements.
    } % Q7.1
        \answer{}
        \begin{center}
            \ch{
                a C6H12O6
                + b O2
                + c NH3
                ->
                CH_{1.77}O_{0.49}N_{0.24}
                + d CH_{1.55}O_{0.31}N_{0.25}
                + e CO2
                + f H2O
            }
        \end{center}
        \begin{flalign*}
            &
                Y_{\ch{NH3/X}}
                = \frac{
                    c\,\si{\mole\of{\ch{NH3}}}
                }{
                    \si{\mole\of{X}}
                }
                \implies &\\[3ex]&
                \implies
                c = 0.24 + 0.25\,d
                ; &\\[3ex]&
                d\,\si{\mole\of{P}}
                = Y_{\ch{P/S}}
                \,a\,\si{\mole\of{S}}
                = Y_{\ch{P/S}}
                \,\frac{1\,\si{\mole\of{X}}}{Y_{\ch{X/S}}}
                = &\\&
                \left(
                    \frac{
                        0.096\,\si{\gram\of{P}}
                    }{
                        \si{\gram\of{S}}
                    }
                \right)
                \,\frac{
                    1\,\si{\mole\of{X}}
                }{
                    \left(
                        \frac{
                            \,\si{\gram\of{X}}
                        }{
                            \si{\gram\of{S}}
                        }
                    \right)
                }
                % 
                % 
                % 
                ; &\\[6ex]&
                Y_{\ch{P/S}}
                = \frac{d\,\si{\mole\of{P}}}{a\,\si{\mole\of{S}}}
                = \frac{
                    0.096\,\si{\gram\of{P}}
                }{
                    \si{\gram\of{S}}
                }
                \,\frac{
                    \si{\mole\of{P}}
                }{
                    \left(
                        \begin{aligned}
                                 1    & *12
                            \\ + 1.55 & *1
                            \\ + 0.31 & *16
                            \\ + 0.25 & *14
                        \end{aligned}
                        % 22.01
                    \right)\,\si{\gram\of{P}}
                }
                \,\frac{
                    \left(
                        \begin{aligned}
                                 6  &* 12
                            \\ + 12 &* 1
                            \\ + 6  &* 16
                        \end{aligned}
                        % 180
                    \right)
                    \,\si{\gram\of{S}}
                }{
                    \si{\mole\of{S}}
                }
            &
        \end{flalign*}
    \end{questionBox}

    \begin{questionBox}2{ % Q7.2
        Assess oxygen requirements.
    } % Q7.2
        \answer{}
        \begin{flalign*}
            &
                Y_{\ch{O2/X}}
                = \frac{
                    b\,\si{\mole\of{\ch{O2}}}
                }{
                    \si{\mole\of{X}}
                }
                % 
                % 
                % 
                \implies &\\[3ex]&
                \implies
                a*6 + b*2
                = 0.49 + d*0.31 + e*2 + f
                \implies &\\&
                \implies
                b 
                = (0.49 + d*0.31 + e*2 + f - a*6)/2
            &
        \end{flalign*}
    \end{questionBox}

    \begin{questionBox}2{ % Q7.3
        If the biomass yield to glucose were the same, what would be the ammonia and oxygen requirements for a wild strain of \textit{E. coli} that fails to synthesize the protein?
    } % Q7.3
        \answer{}
        \subsubquestion{Ammonia}
        \begin{flalign*}
            &
                Y_{\ch{NH3/X}}
                = \frac{
                    c\,\si{\mole\of{\ch{NH3}}}
                }{
                    \si{\mole\of{X}}
                }
                = \frac{
                    0.24\,\si{\mole\of{\ch{NH3}}}
                }{
                    \si{\mole\of{X}}
                }
            &
        \end{flalign*}
        \subsubquestion{Oxygen}
    \end{questionBox}
\end{questionBox}

\begin{questionBox}1{ % Q8
    \textit{Aerobacter aerogenes} is grown aerobically on glucose (\ch{C6H12O6}) or pyruvate (\ch{C3H4O3}) as a carbon source. The empirical formula for biomass is \ch{CH_{1.78}N_{0.24}O_{0.33}} (\(\text{MWB} = 22.5\,\si{\gram/\mole}\)) and its degree of reduction is \(= 4.4\). The following yields were experimentally determined
} % Q8
    \begin{center}
        \vspace{1ex}
        \begin{tabular}{l *{5}{C}}
            \toprule
            
                & \multicolumn{3}{L}{Y_{\ch{X/S}}}
                & \multicolumn{2}{L}{Y_{\ch{X/O2}}}
            
            \\\midrule
            
                Substrate
                & \si{\gram/\gram}
                & \si{\gram/\mole}
                & \si{\gram/\gram\of{\ch{C}}}
                & \si{\gram/\gram}
                & \si{\gram/\mole}
                \\
                Glucose
                & 0.40
                & 0.72
                & 1.01
                & 1.11
                & 35.5
                \\
                Pyruvato
                & 0.20
                & 17.9
                & 0.49
                & 0.48
                & 15.4
            
            \\\bottomrule
        \end{tabular}
        \vspace{2ex}
    \end{center}

    \begin{questionBox}2{ % Q8.1
        Compare the biomass yield per mole of glucose and per mole of pyruvate with their respective thermodynamic maxima. Which of the two substrates is more efficient with respect to biosynthesis? Justique.
    } % Q8.1
        \answer{}
        \begin{center}
            \ch{
                a C6H12O6
                + c NH3
                ->
                CH_{1.78}O_{0.33}N_{0.24}
                \\
                a C3H4O3
                + c NH3
                ->
                CH_{1.78}O_{0.33}N_{0.24}
            }
        \end{center}
        \begin{flalign*}
            &
                \frac{
                    \max{Y_{\ch{X/G}}}
                }{
                    \exp{Y_{\ch{X/G}}}
                }
                = \frac{
                    \left(
                        \frac{
                            \si{\mole\of{X}}
                        }{
                            a\,\si{\mole\of{G}}
                        }
                        \,\frac{
                            22.5\,\si{\gram\of{X}}
                        }{
                            \si{\mole\of{X}}
                        }
                    \right)
                }{
                }
                ; &\\[3ex]&
                Y_{\ch{X/P}}
                = \frac{
                    \si{\mole\of{X}}
                }{
                    a\,\si{\mole\of{P}}
                }
                \,\frac{
                    22.5\,\si{\gram\of{X}}
                }{
                    \si{\mole\of{X}}
                }
                ; &\\[6ex]&
                \gamma(\ch{C3H4O6})
                = 3*2+4*1-1*2
                = 8
            &
        \end{flalign*}
    \end{questionBox}

    \begin{questionBox}2{ % Q8.2
        Considering the case of growth on glucose, assess whether excretion of metabolites into the extracellular medium will be expected.
    } % Q8.2
        \answer{}
        \begin{center}
            \ch{
                a C6H12O6
                + c NH3
                ->
                CH_{1.78}O_{0.33}N_{0.24}
                + d CO2
                + e H2O
                \\
                a C3H4O3
                + c NH3
                ->
                CH_{1.78}O_{0.33}N_{0.24}
                + d CO2
                + e H2O
            }
        \end{center}
        \begin{flalign*}
            &
                \gamma(\ch{C6H12O6})\,a
                + \gamma(\ch{O2})\,b
                = (6*4+12*1-6*2)\,a
                - 4\,b
                = 24\,a- 4\,b
                = &\\&
                = \gamma(\text{biomassa})
                = 4.4
                % a)
                ; &\\[3ex]&
            &
        \end{flalign*}
    \end{questionBox}
\end{questionBox}

\end{document}
% !TEX root = ./CN_A-Slides_Anotacoes.2.tex
\providecommand\mainfilename{"./CN_A-Slides_Anotacoes.tex"}
\providecommand \subfilename{}
\renewcommand   \subfilename{"./CN_A-Slides_Anotacoes.2.tex"}
\documentclass[\mainfilename]{subfiles}

% \tikzset{external/force remake=true} % - remake all

\begin{document}

% \graphicspath{{\subfix{./.build/figures/CN_A-Slides_Anotacoes.2}}}
% \tikzsetexternalprefix{./.build/figures/CN_A-Slides_Anotacoes.2/graphics/}

\mymakesubfile{2}
[CN A]
{Interpolação e Aproximação Polinomial} % Subfile Title
{Interpolação e Aproximação Polinomial} % Part Title

\begin{sectionBox}1{Interpolação} % S
    
    Dado o conjunto \(\Omega\), se põe em questão existir um polinómio \textit{p} com menor grau possível que passa por todos os pontos

    \begin{BM}
        p: p_{(x_i)}=y_i\quad\forall\,i\in\myrange{0,n}
        \\[2ex]
        \Omega=\{
            (x_0,y_0),
            (x_1,y_1),
            \dots,
            (x_n,y_n),
        \}
        ;\\ 
        x_i\neq x_j\quad\forall\,\{i,j\}\in\mathbb{N}:i\neq j
    \end{BM}
    
\end{sectionBox}

\begin{sectionBox}2{Grau do polinomio} % S
    
    \begin{BM}
        \text{grau de } p\leq n
        \\
        p_{n\,(x)}
        = \sum_{i=0}^{n}{a_0\,x^i}
        \implies \\
        \implies
        S\equiv\begin{cases}
            \sum_{i=0}^{n}{a_i\,x_j^i}=y_j
            \\
            j\in\myrange{0,n}
        \end{cases}
    \end{BM}
    
\end{sectionBox}

\begin{sectionBox}2{Matriz de Vandermonde} % S
    
    Representação matricial das equações \textit{S}
    \begin{BM}
        V\,A=Y:
        \begin{cases}
            V\in\mathcal{M}_{n+1\times n+1}:v_{i,j}=x^{i}_{j}
            \\
            \{A,Y\}\in\mathcal{R}^n
        \end{cases}
    \end{BM}

    \begin{sectionBox}*3{Prova?} % S
        \begin{flalign*}
            &
                \myvert{V}
                = \prod_{\stackrel{i,j=1}{i>j}}^{n}{(x_i-x_j)}
                = \prod_{i=1}^{n-1}{\left(
                    \prod_{j=i+1}^{n}{(x_j-x_i)}
                \right)}
                \neq0
                &\\&
                :x_j-x_i\neq0\forall\,\{i,j\}\in\mathbb{N}:i\neq j
            &
        \end{flalign*}
    \end{sectionBox}
    
\end{sectionBox}

\begin{sectionBox}2{Funções de Lagrange} % S
    
    \begin{BM}
        L_{k\,(x)}
        = \left(
            \prod_{i=0}^{k-1}{\frac{x-x_i}{x_k-x_i}}
        \right)
        \,\left(
            \prod_{i=k+1}^{n}{\frac{x-x_i}{x_k-x_i}}
        \right)
        \\
        k\in\myrange{0,n}:
        \\[2ex]
        \left\{
            L_{i\,(x_j)}=\delta_{i,j}=\begin{cases}
                0,\quad& i\neq j
                \\
                1,\quad& i=j
            \end{cases}
        \right.
        ;\{i,j\}\in\myrange{0,n}
    \end{BM}

    As funções \(L_{k\,(x)}\) são \emph{funções base} pois tem-se
    \begin{BM}
        p_{n\,(x)} = \sum_{i=0}^{n}{L_{i\,(x)}\,y_i}
    \end{BM}
    
\end{sectionBox}

\begin{exampleBox}1{ % E
    Determine a expressão analítica do polinómio de Lagrange de grau \(\leq2, p_{2\,(x)}\), interpolador de f nos nodos \(\{0.2,0.5,1\}\).
    \begin{BM}
        f_{(x)}=1/x
    \end{BM}
} % E
    \answer{}
    \begin{flalign*}
        &
            p_{2\,(x)}
            = \sum_{i=0}^{2}{y_i\,L_{i\,(x)}}
            = \sum_{i=0}^{2}{
                f(x_i)
                \,\left(
                    \prod_{j=0}^{i-1}{\frac{x-x_j}{x_i-x_j}}
                    \,\prod_{j=i+1}^{n}{\frac{x-x_j}{x_i-x_j}}
                \right)
            }
            = &\\&
            = \left(
                \begin{aligned}
                    & % 0
                        f_{(x_0)}
                        \,\frac{(x-x_1)(x-x_2)}{(x_0-x_1)(x_0-x_2)}
                    &+\\+& % 1
                        f_{(x_1)}
                        \,\frac{(x-x_0)(x-x_2)}{(x_1-x_0)(x_1-x_2)}
                    &+\\+& % 2
                        f_{(x_2)}
                        \,\frac{(x-x_0)(x-x_1)}{(x_2-x_0)(x_2-x_1)}
                    &
                \end{aligned}
            \right)
            = &\\&
            = \left(
                \begin{aligned}
                    &
                        \frac{1}{0.2}
                        \,\frac{(x-0.5)(x-1)}{(0.2-0.5)(0.2-1)}
                    &+\\+& 
                        \frac{1}{0.5}
                        \,\frac{(x-0.2)(x-1)}{(0.5-0.2)(0.5-1)}
                    &+\\+& 
                        \frac{1}{1}
                        \,\frac{(x-0.2)(x-0.5)}{(1-0.2)(1-0.5)}
                    &
                \end{aligned}
            \right)
            = &\\&
            = 10\,x^2
            - 17\,x
            + 8
        &
    \end{flalign*}
\end{exampleBox}

\begin{sectionBox}1{Erro de Interpolação} % S
    
    \begin{BM}
        E_{n\,(x)} = g_{(x)}-p_{n\,(x)}=g_{(x)}-\widehat{g_{(x)}}
    \end{BM}

    \begin{BM}
        g_{(\tilde{x})} - p_{n\,(\tilde{x})}
        = \odv[order=n+1]{g_{(\gamma)}}{x}
        \,\frac{1}{(n+1)!}
        \,\prod_{i=0}^{n}(\tilde{x}-x_i)
        \\[2ex]
        \begin{cases}
            \gamma\in\myrange*{a,b};
            \\ g\in C^{n+1}(\myrange{a,b})
            \\ \Omega=\{(x_0,y_0),\dots,(x_n,y_n)\}
            \\ \{x_k\}_{k=0,1,\dots,n}\text{ um conjunto de nodos distintos entre si}
            \\ y_k=g(x_k), k=0,1,\dots,n
        \end{cases}
    \end{BM}
    
\end{sectionBox}

\end{document}

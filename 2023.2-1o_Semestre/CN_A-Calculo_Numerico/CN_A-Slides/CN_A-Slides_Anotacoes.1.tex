% !TEX root = ./CN_A-Slides_Anotacoes.1.tex
\providecommand\mainfilename{"./CN_A-Slides_Anotacoes.tex"}
\providecommand \subfilename{}
\renewcommand   \subfilename{"./CN_A-Slides_Anotacoes.1.tex"}
\documentclass[\mainfilename]{subfiles}

% \tikzset{external/force remake=true} % - remake all

\begin{document}

% \graphicspath{{\subfix{./.build/figures/CN_A-Slides_Anotacoes.1}}}
% \tikzsetexternalprefix{./.build/figures/CN_A-Slides_Anotacoes.1/graphics/}

\mymakesubfile{1}
[CN A]
{Teoria dos Erros} % Subfile Title
{Teoria dos Erros} % Part Title

\begin{sectionBox}1{Definindo Erros} % S
    
    \paragraph{Erros iniciais do problema:}
    erros iniciais independentes do processo de cálculo
    \begin{itemize}
        \item Relativos ao modelo matemático escolhido
        \item prevenientes dos dados iniciais
    \end{itemize}

    \paragraph{Erros que ocorrem durante a aplicação de métodos numéricos:}
    erros que ocorrem durante o processo de cálculo
    \begin{itemize}
        \item Erros de arredondamento
        \item Erros de truncatura
    \end{itemize}
    
\end{sectionBox}

\begin{sectionBox}2{Definições} % S
    
    \begin{BM}
        x\approx\hat{x}
        \implies
        \begin{cases}
            \text{Erro:}\quad&
            \varepsilon_x=x-\hat{x}
            \\
            \text{Erro absoluto:}\quad&
            \myvert{\varepsilon_x}=\myvert{x-\hat{x}}
            \\
            \text{Erro relativo:}\quad&
            r_x = \frac{\myvert{x-\hat{x}}}{\myvert{x}}
        \end{cases}
        \\
        \varepsilon_x
        \begin{cases}
            >0,\therefore\hat{x}\text{ é um valor aproximado por \textcolor{red\Light}{defeito}}
            \\
            <0,\therefore\hat{x}\text{ é um valor aproximado por \textcolor{green\Light}{excesso}}
        \end{cases}
    \end{BM}
    \paragraph{Erro relativo:} Mede a precisão do valor aproximado \(\hat{x}\)
    
\end{sectionBox}

\begin{sectionBox}2{Intervalo que contem \(x\)} % S
    
    \begin{BM}
        I_x=\myrange{\hat{x}-\eta_x,\hat{x}+\eta_x}
        : \myvert{x-\hat{x}}\leq\eta_x
        \\[2ex]
        \eta_x\begin{cases}
            0.5*10^{-i}
            \quad&
            :\hat{x}*10^i\in\mathbb{Z}
            \\
            \eta_x
            \quad&
            :\hat{x}\pm\eta_x
            \\
            y*10^{-i}
            \quad&
            :\hat{x}\,(y)\land\hat{x}*10^i\in\mathbb{Z}
        \end{cases}
    \end{BM}

    \begin{multicols}{2}
        \begin{exampleBox}1{ % E1
            12.3
        } % E1
            \answer{}
            \begin{flalign*}
                &
                    \myvert{x-\hat{x}}
                    \leq0.5*10^{-1}
                    =\eta_x
                    &\\&
                    \therefore
                    I_x
                    = \myrange{
                        12.3-0.05,
                        12.3+0.05
                    }
                    = &\\&
                    = \myrange{
                        12.25,12.35
                    }
                &
            \end{flalign*}
        \end{exampleBox}
        \begin{exampleBox}1{ % E2
            \(87.9\pm0.07\)
        } % E2
            \answer{}
            \begin{flalign*}
                &
                    \myvert{x-\hat{x}}
                    \leq0.07
                    = \eta_x
                    &\\&
                    \therefore
                    I_x
                    =\myvert{
                        87.9-0.07,
                        87.9+0.07
                    }
                    = &\\&
                    =\myvert{87.83,87.97}
                &
            \end{flalign*}
        \end{exampleBox}
        \begin{exampleBox}1{ % E3
            \(400.32\,(6)\)
        } % E3
            \answer{}
            \begin{flalign*}
                &
                    \myvert{x-\hat{x}}
                    \leq6*10^{-2}
                    = \eta_x
                    &\\&
                    \therefore
                    I_x
                    =\myrange{400.26,400.38}
                &
            \end{flalign*}
        \end{exampleBox}
    \end{multicols}
    \begin{exampleBox}1{ % E4
        \begin{BM}
            x=1/17
            \qquad
            \hat{x}=0.059
        \end{BM}
    } % E4
        \answer{}
        \sisetup{
            round-mode={places},% figures/places/unsertanty/none
            round-precision={3},
        }
        \subsubquestion{Erro Absoluto}
        \begin{flalign*}
            &
                \myvert{\varepsilon_x}
                = \myvert{x-\hat{x}}
                = \myvert{1/17-0.059}
                = 0.17647\dots\E-3
                \leq 0.177\E-3
            &
        \end{flalign*}
        \subsubquestion{Erro relativo}
        \begin{flalign*}
            &
                r_x
                = \frac{\myvert{x-\hat{x}}}{\myvert{x}}
                = \frac{\myvert{1/17-0.059}}{\myvert{1/17}}
                = 0.300\E-2
            &
        \end{flalign*}
    \end{exampleBox}
    
\end{sectionBox}

\begin{sectionBox}1{Significancia} % S
    
    Disemos que \(\hat{x}\) se aproxima de \textit{x} com, no mínimo \textit{k} \emph{casas decimais significativas} \(\iff\)
    \begin{BM}
        k\in\mathbb{N}_0
        : \myvert{\varepsilon_x}\leq0.5*10^{-k}
    \end{BM}

    Disemos que \(\hat{x}\) se aproxima de \textit{x} com, com \textit{n} \emph{algarismos significativos} \(\iff\)
    \begin{BM}
        n\in\mathbb{N}_0
        : \begin{cases}
            \myvert{\varepsilon_x}
            \leq0.5*10^{m+1-n}
            \\
            m\in\mathbb{Z}: 10^m\leq\myvert{x}<10^{m+1}
        \end{cases}
    \end{BM}
    
\end{sectionBox}

\begin{exampleBox}1{ % E
    Na determinação de \textit{x} obteve-se o resultado \num{0.001773(8)}.\\
} % E
    \answer{}
    \subsubexample{Casas decimais significativas}
    \begin{flalign*}
        &
            k
            :\myvert{x-\hat{x}}
            =\myvert{\varepsilon_x}
            \leq 0.8*10^{-5}
            = 0.5*10^{-4}
            \implies k=4
        &
    \end{flalign*}
    \subsubexample{Algarismos significativos}
    \begin{flalign*}
        &
            n:10^{-3}
            \leq\myvert{x}
            \approx\myvert{\hat{x}}
            <10^{-2}
            \implies &\\&
            \implies
            m+1-n
            = -2-n
            % = &\\&
            = -k
            = -4
            \implies &\\&
            \implies
            n=2
        &
    \end{flalign*}
\end{exampleBox}

\part*{Condicionamento de um problema}

\begin{sectionBox}1{Propagação de erro absoluto} % S
    
    \paragraph{Fórmula fundamental do cálculo dos erros}
    \begin{BM}
        \myvert{\varepsilon_{g_{(x)}}}
        \leq M_1\,\myvert{\varepsilon_x}
        : \myvert{\odv{g_{(z)}}{x}}\leq M_1,z\in V_{\delta\,(x)}
    \end{BM}

    \begin{sectionBox}*2{Desenvolvimento} % S
        Seja \textit{g} uma função diferenciável numa vizinhança \(V_{\delta\,(x)}\) de \textit{x}.\\
        Utilizando a fórmula de Taylor tem-se:
        \begin{flalign*}
            &
                g_{(x)} 
                = g_{(\hat{x})}
                + \odv{g_{(\xi)}}{x}(x-\hat{x})
                : \quad \xi\in\myrange{x,\hat{x}}
                \implies &\\&
                \implies
                \varepsilon_{g_{(x)}}
                = g_{(x)} - g_{(\hat{x})}
                = \odv{g_{(\xi)}}{x}(x-\hat{x})
                = M_1\,\varepsilon_x
                \implies &\\&
                \implies
                \myvert*{\varepsilon_{g_{(x)}}}
                \leq M_1\,\myvert{\varepsilon_x}
            &
        \end{flalign*}
    \end{sectionBox}
    
\end{sectionBox}

\begin{sectionBox}1{Propagação do erro relativo} % S
    
    \begin{BM}
        r_{g(x)}
        \approx
        C_{g\,(x)}\,r_x
    \end{BM}

    \begin{sectionBox}*3{Desenvolvimento} % S
        
        Considerando novamente \textit{g}, mas de classe \(C^2(V_{\delta\,(x)})\), aplicando a formula de Taylor:
        \begin{flalign*}
            &
                g_{(\hat{x})}
                = g_{(x)}
                + \odv{g_{(x)}}{x}(x-\hat{x})
                + \odv[order=2]{g_{(x)}}{x}\frac{(x-\hat{x})^2}{2}
                \underset{x\approx\hat{x}}{\approx}
                g_{(x)}
                + \odv{g_{(x)}}{x}(x-\hat{x})
                + \odv[order=2]{g_{(x)}}{x}\frac{0}{2}
                = &\\&
                = g_{(x)}
                + \odv{g_{(x)}}{x}(x-\hat{x})
                % \implies &\\&
                \implies
                \myvert{g_{(x)}-g_{(\hat{x})}}
                \approx
                \myvert{\odv{g_{(x)}}{x}}
                \myvert{x-\hat{x}}
                \implies &\\&
                \implies
                \frac{\myvert{g_{(x)}-g_{(\hat{x})}}}{\myvert{g_{(x)}}}
                = r_{g(x)}
                \approx
                \myvert{
                    \frac{x}{g_{(x)}}
                    \odv{g_{(x)}}{x}
                }
                \frac{\myvert{x-\hat{x}}}{\myvert{x}}
                = \myvert{
                    \frac{x}{g_{(x)}}
                    \odv{g_{(x)}}{x}
                }
                r_{x}
                = C_{g\,(x)}
                r_{x}
            &
        \end{flalign*}
    \end{sectionBox}

    \paragraph{Numero de condição de uma função \textit{g} num ponto \textit{x}}
    \begin{BM}
        C_{g\,(x)} 
        = \myvert{
            \frac{x}{g_{(x)}}
            \,\odv{g_{(x)}}{x}
        }
        ,\quad g_{(x)}\neq0
    \end{BM}
    
\end{sectionBox}

\begin{exampleBox}1{ % E
    \begin{BM}
        g_{1\,(x)} = x^2\quad (x>0).
    \end{BM}
} % E
    \vspace{-10ex}
    \begin{flalign*}
        &
            C_{g_1\,(x)}
            = \myvert{
                \frac{x}{g_{1\,(x)}}
                \,\odv{g_{1\,(x)}}{x}
            }
            = \myvert{
                \frac{x}{x^x}
                \,(x\,x^{x-1} + x^x\,\ln(x))
            }
            = \myvert{x\,(1+\ln(x))}
        &
    \end{flalign*}
\end{exampleBox}

\begin{sectionBox}1{Condicionamento de uma função/problema} % S
    
    Uma função diz-se:
    \begin{itemize}
        \item \textcolor{green\Light}{Bem condicionado:} Pequenas variações nos dados iniciais/parametros implicam em pequenas variações nos resultados
        \item \textcolor{red\Light}{Mal condicionado:} Situação inversa
    \end{itemize}
    \paragraph{Nota:} \emph{Independe} do método numérico utilizado
    
\end{sectionBox}

\begin{sectionBox}1{Estabilidade de um método numérico} % S
    
    Um método numérico diz-se
    \begin{itemize}
        \item \textcolor{green\Light}{Estável:} Acumulação dos erros não afeta significativamente o resultado final;
        \item \textcolor{red\Light}{Instável: } Caso contrário
    \end{itemize}
    
\end{sectionBox}

\begin{exampleBox}1{ % E
    Considere as duas funções matematicamente idênticas:
    \begin{BM}
        f_{(\theta)}
        = \frac{\tan^2\theta}{\theta^2}
        \qquad
        g_{(\theta)}
        = \frac{1-\cos(2\,\theta)}{\theta^2\,(1+\cos(2\,\theta))}
        \\
        \lim_{\theta\to0}{f_{(\theta)}}
        = \lim_{\theta\to0}{g_{(\theta)}}
        = 1
    \end{BM}
} % E
    \begin{center}
        \vspace{1ex}
        \sisetup{
            round-precision={9},
        }
        \begin{tabular}{*{3}{C}}
            \toprule
            
                \theta
                & f_{(\theta)}
                & g_{(\theta)}
            
            \\\midrule
            
               10^{-1} & \num{1.0067046422494887} & \num{1.006704642249489}
            \\ 10^{-2} & \num{1.0000666704446415} & \num{1.0000666704447514}
            \\ 10^{-3} & \num{1.0000006666670447} & \num{1.0000006666846255}
            \\ 10^{-4} & \num{1.0000000066666668} & \num{1.0000000039225287}
            \\ 10^{-5} & \num{1.0000000000666665} & \num{1.0000000828403706}
            \\ 10^{-6} & \num{1.0000000000006666} & \num{0.9999778782808777}
            \\ 10^{-7} & \num{1.0000000000000067} & \num{0.9992007221626502}
            \\ 10^{-8} & 1 & \num{1.1102230246251554}
            \\ 10^{-9} & 1 & 0
            \\ 10^{-10} & 1 & 0
            
            \\\bottomrule
        \end{tabular}
        \vspace{2ex}
    \end{center}
    \begin{BM}
        \therefore\begin{cases}
            f_{(\theta)}:&\text{\textcolor{green\Light}{estável}}
            \\
            g_{(\theta)}:&\text{\textcolor{red\Light}{instável}}
        \end{cases}
    \end{BM}
\end{exampleBox}

\end{document}
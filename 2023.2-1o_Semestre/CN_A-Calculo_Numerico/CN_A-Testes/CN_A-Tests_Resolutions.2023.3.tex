% !TEX root = ./CN_A-Tests_Resolutions.2023.3.tex
\documentclass[./CN_A-Tests_Resolutions.tex]{subfiles}

% \tikzset{external/force remake=true} % - remake all

\begin{document}

% \graphicspath{{\subfix{./.build/figures/CN_A-Tests_Resolutions.2023.3}}}
% \tikzsetexternalprefix{./.build/figures/CN_A-Tests_Resolutions.2023.3/graphics/}

\mymakesubfile{3}[CN\,A]
{Exame de Recurso 2023} % Subfile Title
{Exame de Recurso 2023} % Part Title

\begin{questionBox}1{} % Q1

  \begin{BM}
    f(x)=e^x-2
    ; \qquad
    g(x)=\cos(e^x-2)
    ; \qquad
    \alpha\in\myrange{0.5,1,5}
    \\
    \{x_{k}\}_{k\in\mathbb{N}}
    \text{ Sucessão gerada pela bisseçao convergente para } \alpha
  \end{BM}
  Qual o valor da iterada \(x_3\text{ e numero de iteradas }k\) para approx alpha 4 casas dec

  \answer{}

  \begin{gather*}
    \varepsilon
    = \myvert{I -\hat{I}}
    = \dots
    < \leq 0.5\E{-4}
  \end{gather*}

\end{questionBox}

\begin{questionBox}1{} % Q2

  Tabela
  \begin{center}
    \vspace{1ex}
    \begin{tabular}{C | *{4}{C}}
      x_i & -1 & 0 & 2 & 5
      \\\hline
      y_i & -4 & 3 & 5 & -22
    \end{tabular}
    \vspace{2ex}
  \end{center}
  Seja \(p_2(x)\) poli grau 2 q approx \(x_i,i=0,\dots,3\) por mínimos q.
  \begin{BM}
    \sum_{i=0}^3{(p_2(x_i)-y_i)^2}=0
  \end{BM}
  qual o valor de \(p_2(3)\)

  % \answer{}
  % \begin{flalign*}
  %     &
  %         p_2(3):
  %         \sum_{i=0}^3{(p_2(x_i)-y_i)^2}=0;
  %     &
  % \end{flalign*}

\end{questionBox}

\begin{questionBox}1{} % Q3

  \begin{BM}
    f(-x)+f(x)=2,\forall\,x\in\mathbb{R}
  \end{BM}
  approx dada por gauss simples com 2 pontos para \(I=_{-3}^{3}{f(x)\odif{x}}\)

  \answer{}

\end{questionBox}

\setcounter{question}{4}
\begin{questionBox}1{} % Q5

  \begin{BM}
    I=\int_0^1{f(x)\odif{x}}
  \end{BM}
  \begin{itemize}
    \item \(f(x)\) é poli de grau 4
    \item \(x^{(4)}=1/12\)
    \item 
  \end{itemize}
  Menor numero de subintervalos para dividir \(\myrange{0,1}\) e garantir 6 casas dec

  \answer{}

  \begin{gather*}
    n:\varepsilon<0.5\E{-6}
    \\
    % Erro de quadratura
    \varepsilon_I
    =I-\hat{I}_S
    = -\frac{h^5}{90}\,f^{(4)}(i)
    = -\frac{(1/2\,n)^5}{90}\,(1/12)
    \leq 0.5\E-6
    \implies \\
    \implies
    n
    = \ceil{(-90*12*0.5\E{-6})^{-1/5}/2}
    = \ceil{(-540\E{-6})^{-1/5}/2}
    = 5
  \end{gather*}

\end{questionBox}

\begin{questionBox}1{} % Q6

  Tabela
  \begin{center}
    \vspace{1ex}
    \begin{tabular}{C | *3{C}}
      x_i & -2 & -1 & 0
      \\\hline
      f(x_i) & 19 & 1 & 1
    \end{tabular}
    \vspace{2ex}
  \end{center}

\end{questionBox}

\begin{questionBox}2{} % Q6.1

  Poli de lag int de f para tabela approx \(f(0.5)\)

  \answer{}

  \begin{gather*}
    p_2(x):
    p_2(0.5)\approx f(0.5)
    ; \\[1ex]
    p_{2\,(x)}
    =\sum_{i=0}^{n}{y_i\,L_{i\,(x)}}
    = \\
    = \frac{
      (18\,x-2)(x+1)
    }{2}
  \end{gather*}

\end{questionBox}

\begin{questionBox}2{} % Q6.2

  \(\myvert{f^{(k)}}\leq(1/2)^k\,e^{-x/2},k=1,2,\dots\), det major p erro abs

  \answer{}

  \sisetup{round-precision={4}}
  \begin{gather*}
    f_{x^*}-p_{2\,(x^*)}
    = \frac{f^{x+1}_{(\xi)}}{(n+1)!}
    \prod_{i=0}^{n}{x^*-x_i}
  \end{gather*}

\end{questionBox}

\begin{questionBox}1{} % Q7

  \begin{center}
    \vspace{1ex}
    \begin{tabular}{C | *7{C}}
      x_i & -3 & -2 & -1 & 0 & 1 & 2 & 3
      \\\hline
      f(x_i)
      & 6 & 10 & 4 & 0 & -2 & -8 & -4
    \end{tabular}
    \vspace{2ex}
  \end{center}

\end{questionBox}

\begin{questionBox}2{} % Q7.1

  Utilizando a regra dos trap comp. \(\hat{I}_T\text{ de }I\int_{-3}^3{f(x)\odif{x}}\text{ com h=2}\)

  \answer{}

  \begin{gather*}
    h=\frac{3-(-3)}{n}=2
    \implies
    n=3
    ; \\[1ex]
  \end{gather*}

\end{questionBox}

\begin{questionBox}2{} % Q7.2

  regra Ponto médi, \(\hat{I}_{PM}\) \(n=3\)

\end{questionBox}

\begin{questionBox}1{} % Q8

  \begin{BM}
    f(x)=x^3-\sin(x), I=\myrange{0.6,1}
    \\
    y_{i+1}=y_n-\frac{f(y_n)}{f'(y_n)},n=0,1,\dots
  \end{BM}
  Sabendo que \(f'(x)\text{ e }f"(x)\) são func crescentes em I

\end{questionBox}

\begin{questionBox}2{} % Q8.1

  Verif a conv de \(y_n\) para \(\alpha\), partindo de \(y_0=1\)

  \answer{}

  \begin{gather*}
    \begin{cases}
      f (x)=& \,\,x^3-\sin(x)
      \\ f'(x)=& 3\,x^2-\cos(x)
      \\ f"(x)=& 6\,x+\sin(x)
    \end{cases}
    \\[1ex]
    \text{Condições de convergencia:}\\
    \begin{cases}
      \myvert{\alpha-x_n}
      \leq
      \frac{M_2}{2\,m_1}
      (x_n-x_{(n-1)})^2
      \\
      0<m_1<\myvert{f'(x)}_{\myrange{a,b}}
      \\
      M_2\geq \myvert{f"(x)}_{\myrange{a,b}}
    \end{cases}
    \\[1ex]
    0<m_1<
    f'(0.6)
    = 3\,(0.6)^2-\cos(0.6)
    \cong\num{0.254664385090322}
    \implies \\
    \implies
    m_1=0.25
    \\[1ex]
    M_2\geq 
    \myvert{f"(x)}_{\myrange{a,b}}
    =f"(1)
    =6*1+\sin(1)
    \cong\num{6.841470984807897}
    \implies \\
    \implies
    M_2=6.85
    \\[1ex]
    \myvert{\alpha-x_n}
    \leq \frac{6.85}{2*0.25}
    (x_n-x_{(n-1)})^2
    \implies \\
    \implies
    \begin{cases}
      y_0=1
      \\
      y_1
      =1+\frac{1^3-\sin(1)}{3*1^2-\cos(1)}
      \cong\num{1.064450609345331}
      \\
      y_2
      =1+\frac{
        \num{1.064450609345331}^3
        -\sin(\num{1.064450609345331})
      }{
        3*(\num{1.064450609345331})^2
        -\cos(\num{1.064450609345331})
      }
      \cong\num{1.11377420529712}
      \\
      y_3
      =1+\frac{
        \num{1.11377420529712}^3
        -\sin(\num{1.11377420529712})
      }{
        3*(\num{1.11377420529712})^2
        -\cos(\num{1.11377420529712})
      }
      \cong\num{1.147630745035126}
    \end{cases}
    \\
    \begin{cases}
      \myvert{\alpha-y_1}
      \leq 
      13.7(y_1-y_0)^2
      = 13.7(\num{1.064450609345331}-1)^2
      \cong\num{0.056908170316287}
      \\
      \myvert{\alpha-y_2}
      \leq 
      13.7(y_2-y_1)^2
      = 13.7(
        \num{1.11377420529712}
        -\num{1.064450609345331}
      )^2
      \cong\num{0.03332959451133}
      \\
      \myvert{\alpha-y_3}
      \leq 
      13.7(y_3-y_2)^2
      = 13.7(
        \num{1.147630745035126}
        -\num{1.11377420529712}
      )^2
      \cong\num{0.015703834377527}
    \end{cases}
    \\
    \therefore\text{ Converge para }\alpha
  \end{gather*}

\end{questionBox}

\begin{questionBox}2{} % Q8.2

  \(y_2\)

  \answer{}

  \sisetup{round-precision={6}}
  \begin{gather*}
    \begin{cases}
      y_0=1
      \\
      y_1
      =1+\frac{1^3-\sin(1)}{3*1^2-\cos(1)}
      \cong\num{1.064450609345331}
      \\
      y_2
      =1+\frac{
        \num{1.064450609345331}^3
        -\sin(\num{1.064450609345331})
      }{
        3*(\num{1.064450609345331})^2
        -\cos(\num{1.064450609345331})
      }
      \cong\num{1.11377420529712}
    \end{cases}
    \\[1ex]
    \myvert{\alpha-y_2}
    \leq 
    13.7(y_2-y_1)^2
    = 13.7(
      \num{1.11377420529712}
      -\num{1.064450609345331}
    )^2
    \cong \\
    \cong\num{0.03332959451133}
    <0.5\E{-2}
    \\
    \therefore 2 \text{ casas decimais}
  \end{gather*}

\end{questionBox}

\begin{questionBox}1{} % Q9

  \(A\,X=B\)
  \begin{BM}
    \begin{bmatrix}
      2 & -1
      \\ 1 & -4
    \end{bmatrix}
    \begin{bmatrix}
      x_1\\x_2
    \end{bmatrix}
    =\begin{bmatrix}
      2\\-2
    \end{bmatrix}
  \end{BM}

\end{questionBox}

\begin{questionBox}2{} % Q9.1

  Sucessão gerada pela jacobi converge para o sistema

  \answer{}

  \begin{gather*}
    \myVert{G_J}:
    G_J=-D^{-1}\,(L+U): \\
    D= \begin{bmatrix}
      2 & 0
      \\ 0 & -4
    \end{bmatrix}
    ; \quad
    U= \begin{bmatrix}
      0 & -1
      \\ 0 & 0
    \end{bmatrix}
    ; \quad
    L= \begin{bmatrix}
      0 & 0
      \\ 1 & 0
    \end{bmatrix}
    ; \\
    -D^{-1}
    = \begin{bmatrix}
      -(1/2) & 0
      \\ 0 & -(-1/4)
    \end{bmatrix}
    = \begin{bmatrix}
      -1/2 & 0
      \\0 & 1/4
    \end{bmatrix}
    \implies \\
    \implies 
    G_J
    =\begin{bmatrix}
      -1/2 & 0
      \\0 & 1/4
    \end{bmatrix}
    \begin{bmatrix}
      0 & -1
      \\ 1 & 0
    \end{bmatrix}
    = \begin{bmatrix}
      0 & 1/2
      \\1/4 & 0
    \end{bmatrix}
    \implies \\
    \implies
    \myVert{G_j}
    % = \\
    = \max(0.25,0.5)=0.5<1
    \\\therefore
    \text{Sistema converge}
  \end{gather*}

\end{questionBox}

\begin{questionBox}2{} % Q9.2

  \(X^{(2)}:X^{(0)}=[0\,0]^T\)

  \answer{}

  \begin{gather*}
    H_J=D^{-1}\,B
    =\begin{bmatrix}
      -1/2 & 0
      \\0 & 1/4
    \end{bmatrix}
    \begin{bmatrix}
      2\\-2
    \end{bmatrix}
    = \begin{bmatrix}
      -1\\-1/2
    \end{bmatrix}
    \implies \\
    \implies
    % X1
    X^{(1)}
    = G_J\,X^{(0)}+H_J
    = H_J
    \\[1ex]
    X^{(2)}
    = G_J\,X^{(1)}+H_J
    =\begin{bmatrix}
      0 & 1/2
      \\1/4 & 0
    \end{bmatrix}
    \begin{bmatrix}
      -1\\-1/2
    \end{bmatrix}
    + \begin{bmatrix}
      -1\\-1/2
    \end{bmatrix}
    =\begin{bmatrix}
      -5/4 \\ -3/4
    \end{bmatrix}
  \end{gather*}

\end{questionBox}

\begin{questionBox}2{} % Q9.3

  Menor dos maj para err abs ass ao \(X^{(2)}\)

  \answer{}

  \begin{gather*}
    \varepsilon_{abs}
  \end{gather*}

\end{questionBox}

\begin{questionBox}2{} % Q9.4

  qts iter para aprox c 4 casas dec

  \answer{}

  \begin{gather*}
    j:
    \myVert{X^{*}-X^{(j)}}<
    \myVert{G}_{\infty}^j
    \,\myVert{X^{(0)}}_{\infty}
    +\frac
    {\myVert{G}_{\infty}^j}
    {1-\myVert{G}_{\infty}}
    \,\myVert{H}
    = \\
    = \max{(1/4,1/2)}^j
    \,0
    +\frac
    {\max{(1/4,1/2)}^j}
    {1-\max{(1/4,1/2)}}
    \,\max{1,1/2}
    = \\
    = 1/2^{(j-1)}
    \leq0.5\E{-4}
    \implies \\
    \implies
    j
    =
    \floor{1+\log_{0.5}{0.5\E{-4}}}
    \cong\floor{\num{1.069990210709402}}
    =1
  \end{gather*}

\end{questionBox}

\end{document}

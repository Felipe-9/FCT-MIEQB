% !TEX root = ./CN_A-Tests_Resolutions.2022.3.tex
\documentclass[CN_A-Tests_Resolutions.tex]{subfiles}

% \tikzset{external/force remake=true} % - remake all

\begin{document}

% \graphicspath{{\subfix{./.build/figures/CN_A-Tests_Resolutions.2022.3}}}
% \tikzsetexternalprefix{./.build/figures/CN_A-Tests_Resolutions.2022.3/graphics/}

\mymakesubfile{3}[CN A]
{Exam 2022.3 Resolution} % Subfile Title
{Exam 2022.3 Resolution} % Part Title

\begin{questionBox}1m{} % Q1

  Considere o intervalo \([a,b]\), com \(a=x_0 < x_1 < \dots< x_{10}=b\) e uma função não linear, \textit{f}, da qual se conhece a seguinte tabela:
  \begin{center}
    \begin{tabular}{C | C C @{\quad\dots\quad} C}
      x
      & x_0
      & x_1
      & x_{10}
      \\\hline
      f(x)
      & y_0
      & y_1
      & y_{10}
    \end{tabular}
  \end{center}
  Pretende-se aproximar \textit{f} por uma função interpoladora nos pontos da tabela de forma a que esta tenha poucas ou nenhumas oscilações junto das extremidades do intervalo \([a,b]\). Para o efeito deve utilizar-se:
  \begin{enumerate}[label=\alph{enumi}.]
    \item O polinómio de \emph{Lagrange} interpolador de \textit{f} nos pontos da tabela.
    \item O polinómio do \emph{grau 2} que aproxima a função tabelada segundo o método dos mínimos quadrados.
    \item O polinómio de \emph{Newton} com diferenças divididas interpolador de \textit{f} nos pontos da tabela.
    \item O \emph{spline cúbico natural}, interpolador de \textit{f} dos pontos da tabela.
  \end{enumerate}

  \answer{d.}

\end{questionBox}

\begin{questionBox}1m{} % Q2
  Seja \(f(x)\in C^2{[1,3]}\) uma função que verifica \(f^{n}(x) = x^n/3, \forall\,x\in {[1,3]}\). Se pretendesse calcular um valor aproximado de \(I= \int_1^3{f(x)\,\odif{x}}\) utilizando a regra dos trapézios, qual o menor número de sub-intervalos de igual amplitude em que teria de dividir o intervalo \([1,3]\), por forma a garantir pelo menos 1 casa decimal significativa para do erro absoluto da aproximação?

  % tag = q2
  % n = n
  % x0 = 1
  % xn = 3

  \answer{\eqref{eq:q2 answer}}

  Finding subdivisions number \(n\)
  \begin{tcolorbox}
    \begin{gather}
      \abs{R_{n}}
      = \abs*{
        \frac{-h^3}{12}
        \,n\,f''(\xi)
      }
      = \mathText{\(h=(3-1)/n;\,f^{(n)}(x)=x^n/3\)}
      = \abs*{
        \frac{-((3-1)/n)^3}{12}
        \,n\,\xi^2/3
      }
      = \abs*{
        \frac{-2}{n^2\,9}\,\xi^2
      }
      = \frac{2}{n^2\,9}\,\xi^2
      \leq \mathText{\(k=1\) casa decimal}
      \leq 5\E{-1-1}
      = 5\E{-2}
      \implies
      n
      \geq
      \sqrt{
        \frac{2\,\xi^2}{9*5\E{-2}}
      }
      = \frac{2\,\xi\,\sqrt{10}}{3}
      \leq \mathText{\(\xi\in\myrange{1,3}\)}
      \leq \frac{2*3\,\sqrt{10}}{3}
      = 2\sqrt{10}
      \implies
      n 
      = \ceil{2\sqrt{10}} 
      = 7
      %
      \yesnumber\label{eq:q2 answer}
    \end{gather}
  \end{tcolorbox}

\end{questionBox}

\begin{questionBox}1m{} % Q3

  Considere a função \(f(x)=x/e^x\text{ e }S(x)\) o spline cúbico natural interpolador de \(f\) nos pontos \(0=x_0<x_1<x_2<x_3<x_4=1\). Qual o valor da expressão \(S(0)-S''(0)-2\,S(1)+2\,S''(1)\)?

  \answer{\eqref{eq:q3 answer}}

  Solving expression
  \begin{tcolorbox}
    \begin{gather}
      S(0)-S"(0)-2\,S(1)+2\,S"(1)
      = \mathText{spline natural: \(y''_0(x_0) = y''_n(x_n) = 0\)}
      = S(0)-2\,S(1)
      = \mathText{\(S_{k}(k) = f_k\)}
      = f(0) -2\,f(1)
      = \frac{0}{e^0} -2\,\frac{1}{e^1}
      = \frac{-2}{e}
      \yesnumber\label{eq:q3 answer}
    \end{gather}
  \end{tcolorbox}
\end{questionBox}

\begin{questionBox}1m{} % Q4

  Seja \(\alpha\) a raiz única da equação não linear \(f(x)=0\) no intervalo \([a,b]\). Considere as sucessões definidas por recorrência \(x_n=g_1(x_{n-1})\text{ e }y_n=g_2(y_{n-1}),n\in\mathbb{N}\), ambas convergentes para \(\alpha\), com \(g_1\text{ e }g_2\) duas funções definidas e continuas em \([a,b]\) tais que \(g_1(\alpha)=\alpha\text{ e }g_2(x)=x-f(x)/f'(x)\). Além disso tem-se \(0\neq\myvert{g'_1(\alpha)}<1\text{ e }g'_2(\alpha)=0\). Considerando que \(x_0=y_0\in{[a,b]}\), qual das opções seguintes é correta
  \begin{enumerate}[label=\alph{enumi}.]
    \item A sucessão \cemph{greenR1}{\(x_n\)} tem ordem de convergencia \cemph{greenR3}{\(p>1\)}
    \item A sucessão \cemph{greenR1}{\(x_n\)} converge mais rapidamente que \cemph{greenR4}{\(y_n\)}
    \item A sucessão \cemph{greenR2}{\(y_n\)} converge mais rapidamente que \cemph{greenR5}{\(x_n\)}
    \item A sucessão \cemph{greenR2}{\(y_n\)} tem ordem de convergencia \cemph{greenR3}{\(p>1\)}
  \end{enumerate}

  \answer{c.}

\end{questionBox}

\begin{questionBox}1m{} % Q5

  Considere a matrix \textit{A} do sistema de equações lineares \(A\,X=B\) com \(a\in\mathbb{R} \textit{ e } X,B\in\mathbb{R}^3\). De forma a garantir a vonvergência do método de Gauss-Siedel para a solução de \(A\,X=B\), qual dos seguintes valores pode ser assumido por \textit{a}?
  \begin{multicols}{2}
    \begin{BM}
      A=\begin{bmatrix}
        a/2 & -2 & 0
        \\ 0 & 7 & -a
        \\ 0 & -3 & a
      \end{bmatrix}
    \end{BM}
    \begin{enumerate}[label=\alph{enumi}.]
      % \begin{multicols}{2}
        \sisetup{round-mode=none}
        \item \num{4}
        \item \num{5.5}
        \item \num{-3}
        \item \num{8}
        % \end{multicols}
    \end{enumerate}
  \end{multicols}

  \answer{\eqref{eq:q5 answer}}

  % Metodo de Gauss-Stidel: A precisa ser diagonal estritamente dominante para o método convergir
  Finding \(a\) bounds
  \begin{tcolorbox}
    \begin{gather}
      a
      \begin{Bmatrix}
        \myvert{a/2} &> \myvert{-2}+0
        \\
        7 &> 0 + \myvert{-a}
        \\ 
        \myvert{a} &> 0 + \myvert{-3}
      \end{Bmatrix}
      =
      \begin{Bmatrix}
        \myvert{a} &> 4
        \\
        \myvert{a} &< 7
        \\ 
        \myvert{a} &> 3
      \end{Bmatrix}
      = 4<\abs{a}<7
      \shortintertext{Closest option}
      a = 5.5
      \label{eq:q5 answer}
    \end{gather}
  \end{tcolorbox}

\end{questionBox}

\begin{questionBox}1m{} % Q6
  Seja \textit{f} uma função da qual se conhece a seguinte tabela de valores
  \begin{center}
    \vspace{1ex}
    \begin{tabular}{C | *{3}{C}}
      x & -1 & 0 & 1
      \\\hline
      f(x) & 10 & 3 & 7
    \end{tabular}
    \vspace{2ex}
  \end{center}
\end{questionBox}

\begin{questionBox}2m{} % Q6.1
  
  \label{q6.1}
  Construa uma tabela de diferenças divididas e o correspondente polinómio de Newton interpolador de \textit{f} na tabela dada. Determine um valor aproximado de \(f(-0.5)\) (Não necessita apresentar o polinómio na forma simplificada).

  \answer{
    Tabela \ref{tab:q6.2 divdiff},
    \eqref{eq:q6.1 answer}
  }

  Construindo tabela de diferencas divididas
  \begin{tcolorbox}
    \begin{table}\centering
      \begin{tabular}{C C *{2}{C}}
        \toprule

        x_i & f(x_i) 
        & f[\cdot,\cdot]
        & f[\cdot,\cdot,\cdot]
        % f[x_{k+1},x_k]
        % \frac{\adif{f_{k+1}-f_k}}{x_{k+1}-x_k}

        \\\midrule
        -1 & 10
        & \multirow{2}{*}{\(-7\)}
        & \multirow{3}{*}{\(11/2\)}
        \\
        0 & 3
        & \multirow{2}{*}{\(4\)}
        \\
        1 & 7

        \\\bottomrule
      \end{tabular}
      \caption{Diferencas divididas \ref{q6.1}}
      \label{tab:q6.2 divdiff}
    \end{table}
  \end{tcolorbox}

  Calculando valor aproximado \(f(-0.5)\)
  \begin{tcolorbox}
    \begin{gather}
      f(-0.5) \approx p_2(-0.5)
      = 10 + (-0.5+1)(-7-0.5*11/2)
      = 5.125
      ; \\
      p_2(x)
      = f(x_0)
      + (x-x_0)
      \,f{[x_0,x_1]}
      + (x-x_0)(x-x_1)
      \,f{[x_0,x_1,x_2]}
      = \mathText{using table \ref{tab:q6.2 divdiff}}
      = 10
      + (x-(-1))
      \,-7
      + (x-(-1))(x-0)
      \,11/2
      = \\
      = 10 + (x+1)(-7+x\,11/2)
      % 
      \label{eq:q6.1 answer}
    \end{gather}
  \end{tcolorbox}

\end{questionBox}

\begin{questionBox}2m{} % Q6.2

  Sabendo que \(f^{(3)}(x) = 12\), determine um majorante do erro absoluto para a aproximação de \(f(-0.5)\) obtida em \ref{q6.1}.

  \answer{\eqref{eq:q6.2 answer}}

  Calculating error
  \begin{tcolorbox}
    \begin{gather}
      \abs{f(-0.5)-p_2(-0.5)}
      = \mathText{using \eqref{eq:q6.2 error formula}}
      = 2\abs{
        (-0.5-(-1))
        (-0.5-0)
        (-0.5-1)
      }
      = \\
      = 0.75
      % 
      \label{eq:q6.2 answer}
    \end{gather}
  \end{tcolorbox}

  Finding error formula
  \begin{tcolorbox}
    \begin{gather}
      \error_{abs}
      = \abs*{f(x^*)-p_2(x^*)}
      = \\
      = \abs*{
        (x^*-x_0)
        (x^*-x_1)
        (x^*-x_2)
        f^{(3)}(\theta)/3!
      }
      = \\
      = \abs*{
        (x^*-x_0)
        (x^*-x_1)
        (x^*-x_2)
        12/6
      }
      = \\
      = 2\abs*{
        (x^*-x_0)
        (x^*-x_1)
        (x^*-x_2)
      }
      \label{eq:q6.2 error formula}
    \end{gather}
  \end{tcolorbox}
  
\end{questionBox}


\begin{questionBox}1m{} % Q7

  Considere a seguinte tabela relativa a uma função \textit{f}
  \begin{center}
    \vspace{1ex}
    \begin{tabular}{C | *{5}{C}}
      x 
      & -1 & 0 & 1 & 2 & 3
      \\\hline
      f(x)
      & -2 & -5 & -6 & k & 22
    \end{tabular}
    \vspace{2ex}
  \end{center}
  Com \(k\in\mathbb{R}\text{ e }I=\int_{-1}^3{f(x)\,\odif{x}}\).
  Sabe-se que \textit{f} é uma função do tipo \(a\,x^4+b\,x^3+c\,x^2+d\,x+e\) em que \(a=1\text{ e }b,c,d,e\in\mathbb{R}\)

\end{questionBox}

\begin{questionBox}2m{} % Q7.1

  Determine uma aproximação de \textit{I} usando a regra de Simpson simples.

  \answer{\eqref{eq:q7.1 answer}}

  \begin{tcolorbox}
    \begin{gather}
      I\approx
      \hat{I}_S
      \approx \frac{h}{3}\,(f_0+4\,f_2+f_4)
      = \mathText{\(h=(3-(-1))/3=2\)}
      = \frac{2}{3}\,(-2+4\,(-6)+22)
      = -\frac{8}{3}
      \label{eq:q7.1 answer}
    \end{gather}
  \end{tcolorbox}

\end{questionBox}

\begin{questionBox}2m{} % Q7.2

  Determine uma aproximação de \textit{I} usando a regra de Simpson composta em função de \textit{k}.

  \answer{\eqref{eq:q7.2 answer}}

  Finding approximation
  \begin{tcolorbox}
    \begin{gather}
      I \approx \hat{I}
      = \frac{h}{3}\left(
        f(x_0)
        + 4\,\sum_{j=1}^{n  }{f(x_{2\,j-1})}
        + 2\,\sum_{j=1}^{n-1}{f(x_{2\,j  })}
        + f(x_{2\,n})
      \right)
      = \mathText{\(h=\frac{3-(-1)}{2\,n}=4/4=1\)}
      = \frac{1}{3}\left(
        f(x_0)
        + 4\,(f(x_1)+f(x_3))
        + 2\,(f(x_2))
        + f(x_{4})
      \right)
      = \\
      = \frac{1}{3}\left(
        -2
        + 4\,(-5+k)
        + 2\,(-6)
        + 22
      \right)
      = \\
      = k\,4/3-4
      \label{eq:q7.2 answer}
    \end{gather}
  \end{tcolorbox}
  
\end{questionBox}

\begin{questionBox}2m{} % Q7.3

  Usando as alineas anteriores determine \textit{k}

  \answer{\eqref{eq:q7.3 answer}}

  Encontrando valor de \(k\)
  \begin{tcolorbox}
    \begin{gather}
      I
      \approx \hat{I}_c + \error_{I,c}
      = \mathText{formula error simson composta e usando \eqref{eq:q7.2 answer}}
      = k\,4/3-4
      - n\,\frac{h^5}{90}\,f^{(4)}(\sigma)
      = \mathText{\( f^{(4)}(x) = 4! = 24 \)}
      = k\,4/3-4
      - 2\,\frac{1^5}{90}\,24
      = k\,4/3
      - \frac{68}{15}
      = \\
      = \hat{i}_s + \error_{I,s} 
      \approx \mathText{formula erro simpson simples e usando \eqref{eq:q7.1 answer}}
      \approx (-8/3)
      -\frac{h^5}{90}\,f^{(4)}(\xi)
      = \mathText{\( f^{(4)}(x) = 4! = 24 \)}
      = - 8/3
      -\frac{2^5}{90}\,24
      = -\frac{56}{5}
      \implies \\
      \implies
      k
      = \frac{3}{4}
      \left(
        -\frac{56*3}{5*3}
        +\frac{68}{15}
      \right)
      = \frac{3}{4}
      \left(
        -\frac{100}{15}
      \right)
      = -5
      \label{eq:q7.3 answer}
    \end{gather}
  \end{tcolorbox}

\end{questionBox}

% fixed point
% Ponto fixo
\begin{questionBox}1m{} % Q8
  Considere a equação \(1-x-\sin(x)=0\) a qual tem uma \underline{única solução} \(\alpha\) no intervalo \(\myrange{0.1,1}\).
\end{questionBox}

\begin{questionBox}2m{} % Q8.1

  Prove que \(\alpha\) é o ponto fixo de \(\varphi(x)=1-\sin(x)\)

  \answer{}

  \begin{tcolorbox}
    \begin{gather}
      \varphi(\alpha)-\alpha
      = 1-\sin(\alpha) - \alpha
      = \mathText{\alpha \in I}
      = 0
      \implies
      \varphi(\alpha) = \alpha, \alpha \in I
    \end{gather}
    \(\alpha\) é ponto fixo de \(\varphi(x)\)
  \end{tcolorbox}

\end{questionBox}

% convergence of sucession on fixed point
\begin{questionBox}2m{} % Q8.2

  Prove que a sucessão \(x_n=\varphi(x_{n-1}),n=1,2,\dots\) em que \(x_0=0.5\) converge para \(\alpha\).

  \answer{\(x_n\) converges to \(\alpha\)}

  Condições de convergencia do metodo do ponto fixo:
  \begin{enumerate}
    \item\label{enu:q8.2 cond 1} \(\varphi(x),\varphi'(x)\) é continua no intervalo \(I\)
    \item\label{enu:q8.2 cond 2} \(\varphi(x) \in I, \forall\,x \in I\)
    \item\label{enu:q8.2 cond 3} \(\abs{\varphi'(x)} \leq \lambda < 1, \forall\,x \in I\)
  \end{enumerate}

  Checking condition \ref{enu:q8.2 cond 3}
  \begin{tcolorbox}
    \begin{gather}
      \abs{\varphi'(x)}
      = \abs{-\cos(x)}
      = \mathText{\(x\in\myrange{0.1,1}\)}
      = \cos(x)
      \leq \cos(0.1)
      \cong \num{0.995004165278026}
      < 0.996 = \lambda < 1
      % 
      \label{eq:q8.2 lambda}
    \end{gather}
    condition \ref{enu:q8.2 cond 3} is verified
  \end{tcolorbox}

  Checking condition \ref{enu:q8.2 cond 1}
  \begin{tcolorbox}
    % Lembrar de usar rad
    \begin{gather}
      \varphi(1) 
      = 1-\sin(1) 
      \cong \num{0.158529015192103} \in \myrange{0.1,1}=I
      ; \\
      \varphi(0.1)
      = 1-\sin(0.1)
      \cong \num{0.900166583353172} \in \myrange{0.1,1}=I
      ; \\
      \varphi'(x)
      = -\cos(x)
      < \mathText{\(\cos(x)>0,\forall\,x\in\myrange{0.1,1}\)}
      < 0
      \implies \varphi(x_1)>\varphi(x_2)\,\forall\,(x_1,x_2)\in I:x_1<x_2
      \mathText{\(\varphi(x)\) é extritamente decrescente e \(\varphi'(x)\) é continua em \(I\)}
    \end{gather}
  \end{tcolorbox}

\end{questionBox}

\begin{questionBox}2m{} % Q8.3

  Determine \(x_2\). Quantas casas decimais significativas pode garantir para \(x_2\). Justifique

  \answer{\eqref{eq:q8.3 x2}}

  Finding \(x_2\) by sucession
  \begin{tcolorbox}
    \begin{gather}
      x_2 = \varphi(x_1)
      \cong \varphi(\num{0.520574461395797})
      = 1-\sin(\num{0.520574461395797})
      \cong \num{0.502621415553111}
      % 
      \label{eq:q8.3 x2}
      % 
      ; \\
      x_1 = \varphi(x_0)
      = \varphi(0.5)
      = 1-\sin(0.5)
      \cong \num{0.520574461395797}
      % 
      \label{eq:q8.3 x1}
    \end{gather}
  \end{tcolorbox}

  Finding uncertainty
  \begin{tcolorbox}
    \begin{gather}
      \error_{x_2}
      = \abs{\alpha-x_2}
      \leq \mathText{Error a posteriori}
      \leq \frac{\lambda}{1-\lambda}\abs{x_2-x_1}
      \cong \mathText{using 
        \eqref{eq:q8.2 lambda}
        \eqref{eq:q8.3 x2}
        \eqref{eq:q8.3 x1}
      }
      \cong \frac{0.996}{1-0.996}
      \,\abs{
        \num{0.502621415553111}
        - \num{0.520574461395797}
      }
      \cong \num{4.470308414828814}
      < 5\E{-1+1}
    \end{gather}
    Não tem casas decimais significativas
  \end{tcolorbox}

\end{questionBox}

\begin{questionBox}1m{} % Q9

  Considere o sistema de equações lineares \(A\,X=B\)
  \begin{BM}
    \begin{bmatrix}
      2 & -1 & 0
      \\ 3 & -5 & 1
      \\ 1 & -2 & 4
    \end{bmatrix}
    \begin{bmatrix}
      x_1\\x_2\\x_3
    \end{bmatrix}
    =\begin{bmatrix}
      1 \\ 2 \\ -1
    \end{bmatrix}
  \end{BM}
  \paragraph*{Nota:} Em todas as alíneas utilize 3 casas decimais convenientemente arredondadas.
  \sisetup{round-precision=3}

  \begin{questionBox}2b{} % Q9.1
    Obtenha a matriz de iteração para o método de Jacobi e com base nessa matriz verifique a convergencia da sucessão definida pelo mesmo método para a solução \(A\,X=B\)

    \answer{}

    Checando convergencia
    \begin{tcolorbox}
      \begin{gather}
        \norm{G_J}_{\infty}
        = \mathText{using \eqref{eq:Q9.1 GJ}}
        = \begin{Vmatrix}
          0     & 0.5 & 0   \\
          0.6   & 0   & 0.2 \\
          -0.25 & 0.5 & 0
        \end{Vmatrix}_{\infty}
        = \max(0.5,0.8,0.75)
        = 0.8 < 1
      \end{gather}
      Converge para a solução de \(A\,X=B\)
    \end{tcolorbox}

    Método de Jacobi
    \begin{tcolorbox}
      \begin{gather}
        G_J
        = -D^{-1}(L+U)
        = \mathText{using 
          \eqref{eq:Q9.1 DLU}
          \eqref{eq:Q9.1 -D-1}
        }
        = 
        \begin{bmatrix}
          -1/2 &  0 & 0
          \\ 0 & +1/5 & 0
          \\ 0 &  0 & -1/4
        \end{bmatrix}
        \left(
          \begin{bmatrix}
            0 &  0 & 0
            \\ 3 &  0 & 0
            \\ 1 & -2 & 0
          \end{bmatrix}
          + \begin{bmatrix}
            0 & -1 & 0
            \\ 0 &  0 & 1
            \\ 0 &  0 & 0
          \end{bmatrix}
        \right)
        = \\
        = \begin{bmatrix}
          -1/2 & 0    & 0 \\
          0    & +1/5 & 0 \\
          0    & 0    & -1/4
        \end{bmatrix}
        \begin{bmatrix}
          0    & -1   & 0 \\
          3    & 0    & 1 \\
          1    & -2   & 0
        \end{bmatrix}
        = \begin{bmatrix}
          0     & 0.5 & 0   \\
          0.6   & 0   & 0.2 \\
          -0.25 & 0.5 & 0
        \end{bmatrix}
        \label{eq:Q9.1 GJ}
      \end{gather}
    \end{tcolorbox}

    Matrizes do método de Jacobi
    \begin{tcolorbox}
      \begin{gather}
        \begin{bmatrix}
          2 & -1 & 0
          \\ 3 & -5 & 1
          \\ 1 & -2 & 4
        \end{bmatrix}
        \hspace{-.5em}
        \implies
        \hspace{-.5em}
        \begin{cases}
          D = \begin{bmatrix}
            2 &  0 & 0
            \\ 0 & -5 & 0
            \\ 0 &  0 & 4
          \end{bmatrix}
          ; 
          L
          = \begin{bmatrix}
            0 &  0 & 0
            \\ 3 &  0 & 0
            \\ 1 & -2 & 0
          \end{bmatrix}
          ;
          U  = \begin{bmatrix}
            0 & -1 & 0
            \\ 0 &  0 & 1
            \\ 0 &  0 & 0
          \end{bmatrix}
        \end{cases}
        \label{eq:Q9.1 DLU}
        ; \\
        -D^{-1}
        = -\begin{bmatrix}
          2 &  0 & 0
          \\ 0 & -5 & 0
          \\ 0 &  0 & 4
        \end{bmatrix}^{-1}
        = -\frac{1}{2-5+4}
        \begin{bmatrix}
          1/2 &  0 & 0
          \\ 0 & -1/5 & 0
          \\ 0 &  0 & 1/4
        \end{bmatrix}^{-1}
        = 
        \begin{bmatrix}
          -1/2 &  0 & 0
          \\ 0 & +1/5 & 0
          \\ 0 &  0 & -1/4
        \end{bmatrix}
        % 
        \label{eq:Q9.1 -D-1}
      \end{gather}
    \end{tcolorbox}

  \end{questionBox}

  \begin{questionBox}2b{} % Q9.2
    Considerando como aproximação inicial 
    \begin{BM}
      X^{(0)} = \begin{bmatrix}
        1 &  1 &  1
      \end{bmatrix}^T
    \end{BM}
    obtenha \(X^{(2)}\)
    \answer{\eqref{eq:Q9.2 answer}}

    Encontrando \(X^{(2)}\)
    \begin{tcolorbox}
      \begin{gather}
        X^{(2)}
        = G_J\,X^{(1)}+H_J
        = \mathText{using \eqref{eq:Q9.1 GJ}\eqref{eq:Q9.2 HJ}}
        = \dots
        = \begin{bmatrix}
          0.365 \\ 0.2 \\ -0.3
        \end{bmatrix}
        %
        \label{eq:Q9.2 answer}
        %
        ; \\
        X^{(1)}
        = G_J\,X^{(0)} + H_J
        = \begin{bmatrix}
          1 \\ 0.4 \\ 0
        \end{bmatrix}
      \end{gather}
    \end{tcolorbox}

    Encontrando \(H_J\)
    \begin{tcolorbox}
      \begin{gather}
        H_J = D^{-1}\,B 
        = \mathText{using \eqref{eq:Q9.1 -D-1}}
        = \dots
        = \begin{bmatrix}
          0.5 \\ -0.4 \\ -0.25
        \end{bmatrix}
        \label{eq:Q9.2 HJ}
      \end{gather}
    \end{tcolorbox}
  \end{questionBox}

  \begin{questionBox}2b{} % Q9.3
    Sabendo que 
    \begin{BM}[align]
      X^{(9)}&=\begin{bmatrix}
        0.372 \\ -0.270 \\ -472
      \end{bmatrix}
      ; &
      X^{(10)}&=\begin{bmatrix}
        0.365 \\ -0.271 \\ -0.478
      \end{bmatrix}
    \end{BM}
    São iteradas obtidas por aplicação do método de \emph{jacobi} com 3 casas decimais davidamente arredondadas, quantas casas decimais significativas pode garantir \(X^{(10)}\)? Justifique

    \answer{}

    \begin{gather}
      X^{(10)}-X^{(9)}
      =\begin{bmatrix}
        -0.007\\-0.001\\-0.006
      \end{bmatrix}
      ; \\[1ex]
      \myVert{X^{(10)}-X^{(9)}}_{\infty}
      =0.007
      ; \\[1ex]
      % Erro a posteriori
      \myVert{X^*-X^{(10)}}_{\infty}
      \leq
      \frac{\myVert{G}_{\infty}}{1-\myVert{G}_{\infty}}
      \myVert{X^{(10)}-X^{(9)}}_{\infty}
      = \\
      = \frac{0.8}{1-0.8}*0.007
      = 0.028<0.5\E1
      \\[1ex]
      \therefore
      \text{ Garante pelo menos 1 casa decimal}
    \end{gather}
  \end{questionBox}
  \begin{questionBox}2b{} % Q9.4
    Sabendo que a solução exata do sistema é
    \begin{BM}
      X^*
      =\begin{bmatrix}
        0.36 &  -0.28 &  -0.48
      \end{bmatrix}^{T}
    \end{BM}
    Determine o erro relativo associado a cada componente de \(X^{(10)}\)
    \answer{}
    \begin{gather}
      % Erro relativo
      r_{x^*}=\frac{\myvert{x^*-\tilde{x}}}{\myvert{x^*}}\
      \begin{cases}
        r_{x^*_1} 
        = \frac
        {\myvert{0.36-0.365}}
        {\myvert{0.36}}
        \\
        r_{x^*_2} 
        = \frac
        {\myvert{-0.28+0.271}}
        {\myvert{-0.28}}
        \\
        r_{x^*_3} 
        = \frac
        {\myvert{-0.48+0.478}}
        {\myvert{-0.48}}
      \end{cases}
    \end{gather}
  \end{questionBox}

\end{questionBox}

\begin{questionBox}2m{} % Q9.1

  Obtenha a matriz de iteração para o método de Jacobi e com base nessa matriz verifique a convergência da sucessão definida pelo mesmo método para a solução de \(A\,X=B\).

  \answer{}

  \begin{gather}
    \myVert{G_J}<1:
    G_J
    =-D^{-1}(L+U):\\
    % Método de jacobi
    D = \begin{bmatrix}
      2 & 0 & 0
      \\ 0 & -5 & 0
      \\ 0 & 0 & 4
    \end{bmatrix}
    ;\quad
    L = \begin{bmatrix}
      0 & 0 & 0
      \\ 3 & 0 & 0
      \\ 1 & -2 & 0
    \end{bmatrix}
    ;\quad
    U = \begin{bmatrix}
      0 & -1 & 0
      \\ 0 & 0 & 1
      \\ 0 & 0 & 0
    \end{bmatrix};
    \\
    D^{-1}=\begin{bmatrix}
      1/2 & 0 & 0
      \\ 0 & -1/5 & 0
      \\ 0 & 0 & 1/4
    \end{bmatrix}
    \implies \\
    \implies
    G_J
    = -\begin{bmatrix}
      1/2 & 0 & 0
      \\ 0 & -1/5 & 0
      \\ 0 & 0 & 1/4
    \end{bmatrix}
    \begin{bmatrix}
      0 & -1 & 0
      \\ 3 & 0 & 1
      \\ 1 & -2 & 0
    \end{bmatrix}
    = \\
    =\begin{bmatrix}
      0 & 1/2 & 0
      \\ 3/5 & 0 & 1/5
      \\ -1/4 & 1/2 & 0
    \end{bmatrix}
    \implies \\
    \implies
    \myVert{G_J}
    =\max(0.5,0.8,0.75)
    =0.8<1
    \therefore
    \text{o método converge}
  \end{gather}

\end{questionBox}

\begin{questionBox}2m{} % Q9.2

  Considerando como aproximação inicial \(X^{(0)} = \begin{bmatrix}1&1&1\end{bmatrix}^T\) obtenha \(X^{(3)}\).

  \answer{}

  \begin{gather}
    H_J=D^{-1}\,B
    =\begin{bmatrix}
      0.5\\-0.4\\-0.25
    \end{bmatrix}
    ; \\[1ex]
    % X1
    X^{(1)}
    =G_J\,X^{(0)}+H_J
    = \\
    =\begin{bmatrix}
      0 & 1/2 & 0
      \\ 3/5 & 0 & 1/5
      \\ -1/4 & 1/2 & 0
    \end{bmatrix}
    \begin{bmatrix}
      1\\1\\1
    \end{bmatrix}
    + \begin{bmatrix}
      0.5\\-0.4\\-0.25
    \end{bmatrix}
    =\begin{bmatrix}
      1\\0.4\\0
    \end{bmatrix}
    ; \\[1ex]
    % X2
    X^{(2)}
    =G_J\,X^{(1)}+H_J
    = \\
    =\begin{bmatrix}
      0 & 1/2 & 0
      \\ 3/5 & 0 & 1/5
      \\ -1/4 & 1/2 & 0
    \end{bmatrix}
    \begin{bmatrix}
      1\\0.4\\0
    \end{bmatrix}
    + \begin{bmatrix}
      0.5\\-0.4\\-0.25
    \end{bmatrix}
    =\begin{bmatrix}
      0.7\\0.2\\-0.3
    \end{bmatrix}
    ; \\[1ex]
    % X3
    X^{(3)}
    =G_J\,X^{(2)}+H_J
    = \\
    =\begin{bmatrix}
      0 & 1/2 & 0
      \\ 3/5 & 0 & 1/5
      \\ -1/4 & 1/2 & 0
    \end{bmatrix}
    \begin{bmatrix}
      0.7\\0.2\\-0.3
    \end{bmatrix}
    + \begin{bmatrix}
      0.5\\-0.4\\-0.25
    \end{bmatrix}
    =\dots
  \end{gather}

\end{questionBox}

\end{document}

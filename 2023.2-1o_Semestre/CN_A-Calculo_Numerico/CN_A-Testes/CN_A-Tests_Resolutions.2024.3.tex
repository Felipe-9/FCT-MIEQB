% !TEX root = ./CN_A-Tests_Resolutions.2024.3.tex
\documentclass["CN_A-Tests_Resolutions.tex"]{subfiles}

% \tikzset{external/force remake=true} % - remake all

\begin{document}

% \graphicspath{{\subfix{./.build/figures/CN_A-Tests_Resolutions.2024.3}}}
% \tikzsetexternalprefix{./.build/figures/CN_A-Tests_Resolutions.2024.3/graphics/}

\mymakesubfile{3}[CNA]
{Exam 2024.3 Resolution} % Subfile Title
{Exam 2024.3 Resolution} % Part Title

\begin{questionBox}1m{} % Q1
  Encotrar iesia iterada pelo met da bisexão
  \answer{a)}

\end{questionBox}

\begin{questionBox}1m{} % Q2
  Sucessõa jacobi
  \begin{BM}
    AX=B
    \\
    X_1^{(0)} = X_2^{(0)}
    \\
    \norm{G_1}_{\infty} = 4/5
    \\
    \norm{G_1}_1 = 7/6
    \\
    \norm{G_2}_{\infty} = 6/5
    \\
    \norm{G_2}_1 = 2
  \end{BM}

  \answer{a)}

  Convergencia
  \begin{tcolorbox}
    \begin{align}
      \norm{G_{1}}_{\infty} = 4/5 < 1 & \text{Converge}
      \\
      \norm{G_{2}}_{\infty} = 6/5 > 1 & \text{não Converge}
    \end{align}
  \end{tcolorbox}
  
\end{questionBox}

\begin{questionBox}1m{} % Q3
  Considere a tab
  \begin{center}
    \vspace{1ex}
    \begin{tabular}{C | *{4}{C}}

      x_i
      & -1 & 0 & 2 & 5
      \\\hline
      f(x_i)
      & -6 & 2 & 5 & 8

    \end{tabular}
  \end{center}

  \answer{d)}

  Encontrando \(p_1(-1)\)
  \begin{tcolorbox}
    % tag = q3
    % n = 1

    \answer{eq:q3 L}

    Lagrange's polynom
    \begin{tcolorbox}
      \begin{gather}
        p_{1}(x)
        = y_{1}\,l_1(x)
        + y_{3}\,l_3(x)
        = \mathText{using \eqref{eq:q3 li}}
        = \dots
        %
        \yesnumber\label{eq:q3 L}
      \end{gather}
    \end{tcolorbox}

    Finding singular \(l_i\)
    \begin{tcolorbox}
      \begin{gather}
        \begin{cases}
          l_1
          = 
          % 
          % 
          % 
          ; \\
          l_2
          = 
        \end{cases}
        %
        \yesnumber\label{eq:q3 li}
      \end{gather}
    \end{tcolorbox}
  \end{tcolorbox}

  Newton's polynom \(q_2\)
  \begin{tcolorbox}
    % tag = q3
    % n = 2
    \begin{tcolorbox}
      \begin{gather}
        q_{2\,(x)}
        = f(x_0)
        + (x-x_0)
        \,f\myrange{x_0,x_{1}}
        + ( x-x_0 ) ( x-x_1 )
        \,f\myrange{x_0,x_1,x_{3}}
        = \\
        = -6
        + (x-(-1))8
        + ( x-(-1) ) ( x-2 )
        \,\frac{34}{30}
        = \\
        = -48 + (x+1)
        + ( x+1 ) ( x-2 )
        \,\frac{34}{30}
        %
        \yesnumber\label{eq:q3 newtons polynom}
      \end{gather}
    \end{tcolorbox}

    Solving \(f[\cdot,\cdot]\)
    \begin{tcolorbox}
      \begin{gather}
        f\myrange{x_0,x_{1}}
        = \frac{2-(-6)}{0-(-1)}
        = 8
        ; \\
        f\myrange{x_0,x_1,x_{3}}
        = \frac{6/5-8}{5-(-1)}
        = \frac{34}{30}
        ;\\ 
        f\myrange{x_1,x_{3}}
        = \frac{8-2}{5-0}
        = 6/5
      \end{gather}
    \end{tcolorbox}
  \end{tcolorbox}
\end{questionBox}

\begin{questionBox}1m{} % Qindex
  \begin{BM}
    I=\int_0^1{f(x)\,\odif{x}}
    ; f(x) \in C^4([0,1])
    \\
    \abs{f^{(k)}(x)} \leq \sqrt[k]{\cos(x)+2^k-1}
    ,\forall\,x\in\mathbb{R}, k\in\mathbb{N}
  \end{BM}
  Qual o numero de app da regra de simp p garantir pelo menos 6 casas dec

  \answer{\eqref{eq:q4 answer} b)}
  % erro de quadratura
  \begin{tcolorbox}
    \begin{gather}
      \abs{ \varepsilon_i }
      = \abs*{
        -\frac{h^5}{90}\,f^{(4)}(\xi)
      }
      = \mathText{\(h=(1-0)/n=1/n\)}
      = \abs*{
        -\frac{(1/n)^5}{90}\,
      }
      \abs{ f^{(4)}(\xi) }
      = \abs*{
        -\frac{1}{90\,n^5}
      }
      \,\sqrt[4]{\cos(\xi)+2^{\xi}-1}
      = \mathText{\(n\in\mathbb{N}_0^+\)}
      \frac{1}{90\,n^5}
      \,\sqrt[4]{\cos(\xi)+2^{\xi}-1}
      \leq \mathText{\(k=6\) casas decimais}
      \leq 5\E{-1-6} = 5\E{-7}
      \implies
      n
      \geq 
      \left(
        \frac
        {\sqrt[4]{\cos(\xi)+2^{\xi}-1}}
        {90*5\E{-7}}
      \right)^{1/5}
      \leq \mathText{\(\xi\in\myrange{0,1}\)}
      \leq
      \left(
        \frac
        {\sqrt[4]{\cos(1)+2^{1}-1}}
        {90*5\E{-7}}
      \right)^{1/5}
      \cong \num{7.563760964934167}
      \implies
      n = \ceil{\num{7.563760964934167}} = 8
      \shortintertext{Closest alternative}
      n = 7
      \label{eq:q4 answer}
    \end{gather}
  \end{tcolorbox}

  \begin{tcolorbox}
    \begin{gather}
      \cos(\xi)+2^{\xi}
      \begin{cases}
        \cos(0)+2^{0} = 2
        \\
        \cos(1) + 2^{1}
        \cong \num{0.54030230586814}+2 
        = \num{2.54030230586814}
      \end{cases}
    \end{gather}
  \end{tcolorbox}
  
\end{questionBox}

\begin{questionBox}1m{} % Q5
  \answer{d)}
  % tag: q5
  % a = -6
  % b = 10

  \begin{gather}
    I_{G,2}
    = \int_{-6}^{10}{f(x)\,\odif{x}}
    = \mathText{Using simple gauss rule (\(n=2\))}
    = \frac{10-(-6)}{2}
    \,\int_{-1}^{1}{g(y)\,\odif{y}}
    = 8\,\int_{-1}^{1}{g(y)\,\odif{y}}
    = \mathText{Using \eqref{eq:q5-g(x)}}
  \end{gather}

  % g(x)
  \begin{gather}
    g(y)
    = f\left(
      \frac{b-a}{2}\,y
      +\frac{b+a}{2}
    \right)
    = f\left(
      \frac{10-(-6)}{2}\,y
      +\frac{10+(-6)}{2}
    \right)
    = f(8\,y +2)
    = 
    %
    \yesnumber\label{eq:q5-g(x)}
  \end{gather}

\end{questionBox}

\begin{questionBox}1m{} % Q6
  \begin{table}\centering
    \begin{tabular}{C | *{4}{C}}

      x_i
      & -2 &-1 & 0 & 1
      \\\hline
      f(x_i)
      & 19 & 1 & 1 & 1

    \end{tabular}
    \label{q6.1}
  \end{table}



  \begin{questionBox}2b{} % Q6.1
    Obtenha o pol de newton e aproxime \(f(0.01)\)

    \answer{}

    % tag = q6.1
    % n = 3

    \answer{\eqref{eq:q6.1 newtons polynom},\eqref{eq:q6.1 approx}}

    Construindo tabela de diferencas divididas
    \begin{tcolorbox}
      \begin{table}\centering
        \begin{tabular}{C C *{3}{C}}
          \toprule

          x_i & f(x_i) 
          & f[\cdot,\cdot]
          & f[\cdot,\cdot,\cdot]
          & f[\cdot,\cdot,\cdot,\cdot]
          % f[x_{k+1},x_k]
          % \frac{\adif{f_{k+1}-f_k}}{x_{k+1}-x_k}

          \\\midrule
          -2 & 19
          & \multirow{2}{*}{\(-18\)}
          & \multirow{3}{*}{\(9\)}
          & \multirow{4}{*}{\(-3\)}
          \\
          -1 & 1
          & \multirow{2}{*}{\(0\)}
          & \multirow{3}{*}{\(0\)}
          \\
          0 & 1
          & \multirow{2}{*}{\(0\)}
          \\
          1 & 1

          \\\bottomrule
        \end{tabular}
        \caption{Diferencas divididas \ref{q6.1}}
        \label{tab:q6.1 divdiff}
      \end{table}
    \end{tcolorbox}

    Newton's polynom
    \begin{tcolorbox}
      \begin{gather}
        p_{3\,(x)}
        = 19
        + \begin{pmatrix}
          + (x-x_0)\,f\myrange{x_0,x_{1}}
          \\
          + (x-x_0)(x-x_1)\,f\myrange{x_0,\dots,x_{2}}
          \\
          + (x-x_0)(x-x_1)(x-x_2)\,f\myrange{x_0,\dots,x_{3}}
        \end{pmatrix}
        = \mathText{using tabela \ref{tab:q6.1 divdiff}}
        = 19
        + \begin{pmatrix}
          + (x-(-2))\,18
          \\
          + (x-(-2))(x-(-1))\,9
          \\
          + (x-(-2))(x-(-1))(x-0)\,(-3)
        \end{pmatrix}
        = \\
        = 19
        + \begin{pmatrix}
          + (x+2)\,18
          \\
          + (x+2)(x+1)\,9
          \\
          + (x+2)(x+1)(x)\,(-3)
        \end{pmatrix}
        %
        \yesnumber\label{eq:q6.1 newtons polynom}
      \end{gather}
    \end{tcolorbox}

    approx para \(f(0.01)\)
    \begin{tcolorbox}
      \begin{gather}
        f(0.01)
        \cong 
        p_{3}(0.01)
        = \mathText{using \eqref{eq:q6.1 newtons polynom}}
        = 19
        + \begin{pmatrix}
          + (0.01+2)\,18
          \\
          + (0.01+2)(0.01+1)\,9
          \\
          + (0.01+2)(0.01+1)(0.01)\,(-3)
        \end{pmatrix}
        \cong \num{73.389997}
        \label{eq:q6.1 approx}
      \end{gather}
    \end{tcolorbox}
  \end{questionBox}

  \begin{questionBox}2b{} % Q6.2
    f(x) é pol ord 4, coeff \(x^4\) é 1, det maj erro abs
    \answer{}

    \sisetup{round-precision=4}

    \begin{tcolorbox}
      \begin{gather}
        \abs{\error_{p}}
        = \abs{
          f(x)
          - p(x)
        }
        = \mathText{using \eqref{eq:q6.1 newtons polynom}}
        = \abs*{
          a_0 + a_1\,x + a_2\,x^2 + a_3\,x^3 + x^4
          - 19
          - \begin{pmatrix}
            + (x+2)\,18
            \\
            + (x+2)(x+1)\,9
            \\
            + (x+2)(x+1)(x)\,(-3)
          \end{pmatrix}
        }
        \dots
      \end{gather}
    \end{tcolorbox}
  \end{questionBox}

  \begin{questionBox}2b{} % Q6.3
    \begin{BM}
      S(x)
      = \begin{cases}
        \frac{9}{2}\,x^3 + 27\,x^2 + \frac{63}{2}\,x + 10
        ,& x\in\myrange*l{-2,1}
        \\
        \frac{-9}{2}\,x^3+\frac{9}{2}\,x+1
        ,& x\in\myrange{-1,0}
      \end{cases}
    \end{BM}
    verif se é spline natural de f
    \answer{}
    \begin{tcolorbox}
      Pra ser spline natural
      \begin{enumerate}
        \item \(S_{k}(x_{k+1})=S_{k+1}{x_{k+1}}\)
        \item \(S'_{k}(x_{k+1})=S'_{k+1}{x_{k+1}}\)
        \item \(S''_{k}(x_{k+1})=S''_{k+1}{x_{k+1}}\)
        \item \(S''_{0}(x_{0})=S''_{n}{x_{n}}=0\)
      \end{enumerate}
      \begin{gather}
        S_1(-1) 
        = \frac{9}{2}\,(-1)^3 + 27\,(-1)^2 + \frac{63}{2}\,(-1) + 10
        = \frac{74-9-63}{2}
        = 1
        = \\
        = S_2(-1)
        = \frac{-9}{2}\,(-1)^3+\frac{9}{2}\,(-1)+1
        = \frac{9}{2}-\frac{9}{2}+1
        = 1
        ; \\
        S'_1(-1)
        = \frac{3*9}{2}\,(-1)^2 + 2*27\,(-1) + \frac{63}{2}
        = \frac{-4*27+3*9+63}{2}
        = -9
        = \\
        = S'_2(-1)
        = \frac{-3*9}{2}\,(-1)^2+\frac{9}{2}
        = \frac{-2*9}{2}
        = -9
        ; \\
        S''_1(-1)
        = \frac{9*3*2}{2}\,(-1) + 27*2*1
        = 27
        = \\
        = S''_2(-1)
        = \frac{-9*3*2}{2}\,(-1)
        = 27
        ;\\ 
        S''(-2)
        = \frac{9*3*2}{2}\,(-2) + 27*2*1
        = 0
        ; \\
        S''(0)
        = \frac{-9*3*2}{2}\,(0)
        = 0
      \end{gather}
      Todas as condições são verificadas, é spline cubico natural de f
    \end{tcolorbox}
  \end{questionBox}
\end{questionBox}

\begin{questionBox}1m{} % Q7
  \setlength\tabcolsep{1em}        % width
  \begin{center}
    \begin{tabular}{C | *{7}{C}}

      x_i
      & 0 & 0.5 & 1 & 1.5 & 2 & 2.5 & 3
      \\\hline
      f(x_i)
      & 5 & 6.1875 & 5 & f(1.5) & -3 & -5.3125 & -1

    \end{tabular}
  \end{center}

  \begin{questionBox}2b{} % Q7.1
    Sabendo a approx \(\hat{I}_{PM}\) de \(I=\int_0^3{f(x)\,\odif{x}}\) com \(h=1\) é 2.3125, obtenha \(f(1.5)\)
    \answer{}
    \begin{tcolorbox}
      \begin{gather}
        \int_{0}^{3}{f_{(x)}\,\odif{x}}
        = 2.3125
        \approx 
        h\,f_{\left(\frac{0+3}{2}\right)}
        = 1*f(1.5)
        \implies
        f(1.5)=2.3125
      \end{gather}
    \end{tcolorbox}
  \end{questionBox}

  \begin{questionBox}2b{} % Q7.2
    Regra de simp c 3 app obtenha approx \(\hat{I}_S\)
    \answer{}
    % integração
    \begin{gather}
      \int_{0}^{3}{f_{(x)}\,\odif{x}}
      \approx \\
      \approx \hat{I}
      = \frac{h}{3}\left(
        f(x_0)
        + 4\left(
          f(x_{1})
          + f(x_{3})
          + f(x_{5})
        \right)
        + 2(
          f(x_{2})
          + f(x_{4})
        )
        + f(x_{6})
      \right)
      = \mathText{\(h=(3-0)/3=1\)}
      = \frac{1}{3}\left(
        5
        + 4\left(
          6.1875
          + 2.3125
          + (-5.3125)
        \right)
        + 2(
          5
          + (-3)
        )
        + (-1)
      \right)
      = \\
      = \frac{1}{3}\left(
        5
        + 4\left(
          6.1875
          + 2.3125
          + (-5.3125)
        \right)
        + 2(
          5
          + (-3)
        )
        + (-1)
      \right)
      \cong
      \num{6.916666666666667}
      \label{eq:q7.2 answer}
    \end{gather}
  \end{questionBox}

  \begin{questionBox}2b{} % Q7.3
    Supondo \(f^{(4)}=0,\forall\,x\in\mathbb{R}\) det o val exato de \(I\)
    \answer{}



    % erro de quadratura
    \begin{gather}
      I
      = \hat{I}_S + \varepsilon_i
      = \hat{I}_S
      -n\,\frac{h^5}{90}\,f^{(4)}(\sigma)
      = \mathText{using \eqref{eq:q7.2 answer}}
      \cong \num{6.916666666666667}
      - 3*\frac{1^5}{90}*0
      = \num{6.916666666666667}
    \end{gather}

  \end{questionBox}
\end{questionBox}

\begin{questionBox}1m{} % Q8
  \begin{BM}[gather][\normalsize]
    x_{n+1} = F(x_n)
    \quad
    y_{n+1} = G(y_n)
    ;\quad
    \{x_0,y_0\} \in \myrange{0.1,1}, n=1,2,\dots
    \\
    F(x) = e^{\frac{x-4}{2}}
    \quad
    G(x) = 4+2\,\ln(x)
  \end{BM}

  \answer{}

  \begin{questionBox}2b{} % Qindex
    Prove que \(\alpha\) é raiz da eq \(2+\ln(x)-x/2\) se e so se é ponto fixo de \(F(x),G(x)\)
    \answer{}
    \begin{tcolorbox}
      \begin{gather}
        G(\alpha)
        = 4+2\,\ln(\alpha)
        = 2*(2+\ln(\alpha)-\alpha/2) + \alpha
        = \mathText{\(2+\ln(\alpha)-\alpha/2=0\)}
        = \alpha
        ; \\
        F(\alpha)
        = e^{\frac{\alpha-4}{2}}
        = e^{\frac{G(\alpha)-4}{2}}
        = e^{\frac{4+2\,\ln(\alpha)-4}{2}}
        = e^{\ln(\alpha)}
        = \alpha
      \end{gather}
      Assim \(\alpha\) é ponto médio de \(G(x)\) e \(f(x)\)
    \end{tcolorbox}
  \end{questionBox}

  \begin{questionBox}2b{} % Q8.2
    Convergencia de \(x_{i+1},y_{i+1}\) para \(\alpha\in I\), calss \(\alpha\) rel as func F e G, just
    \answer{}

    \begin{tcolorbox}
      Condições de convergencia do metodo do ponto fixo:
      \begin{enumerate}
        \item\label{enu:q8.2 cond 1} \(\varphi(x),\varphi'(x)\) é continua no intervalo \(I\)
        \item\label{enu:q8.2 cond 2} \(\varphi(x) \in I, \forall\,x \in I\)
        \item\label{enu:q8.2 cond 3} \(\abs{\varphi'(x)} \leq \lambda < 1, \forall\,x \in I\)
      \end{enumerate}
      \begin{gather}
        F'(x)
        = e^{\frac{x-4}{2}}
        \,(1/2)
        ; \\
        G'(x)
        = 2/x
        ; \\
        \abs{F'(x)}
        = e^{\frac{x-4}{2}}/2
        \leq \mathText{\(x\in\myrange{0.1,1}\)}
        \leq e^{\frac{1-4}{2}}/2
        = e^{-3/2}/2
        \cong \num{-2.240844535169032}
        \leq -2.24
        = \lambda < 1
        \label{eq:8.2 lambda}
        ; \\
        \abs{G'(x)}
        = \abs{2/x}
        = 2/x
        \leq \mathText{\(x\in\myrange{0.1,1}\)}
        \leq 2/0.1
        = 20 = \lambda > 1
      \end{gather}
      F converge e G diverge para \(\alpha\)
    \end{tcolorbox}

  \end{questionBox}
  
  \begin{questionBox}2b{} % Qindex
    \(x_0=1\) quantas casas decimais garante para \(x_2\)
    \answer{}

    Casas decimais
    \begin{tcolorbox}
      \begin{gather}
        \abs{\error_{x_2}}
        = \abs{\alpha-x_2}
        \leq \mathText{erro a posteriori}
        \leq \abs*{\frac{\lambda}{1-\lambda}}\abs{x_2-x_1}
        \cong \mathText{using \eqref{eq:8.2 lambda}}
        \cong \frac{-2.24}{1-(-2.24)}
        \abs*{
          \num{-69.466493969825554}
          -\num{-4.481689070338065}
        }
        \cong 
        \num{44.927766350262955}
        < 5\E{-1+3}
      \end{gather}
      Não pode se garantir nenhuma casa decimal
    \end{tcolorbox}

    \(x_2\)
    \begin{tcolorbox}
      \begin{gather}
        x_2 = F(x_1)
        \cong \\
        \cong F(\num{-4.481689070338065})
        = e^{\frac{\num{-4.481689070338065}-4}{2}}
        \cong
        \num{-69.466493969825554}
        ; \\
        x_1 = F(x_0)
        = F(1)
        = e^{\frac{1-4}{2}}
        = e^{-3/2}
        \cong \num{-4.481689070338065}
      \end{gather}
    \end{tcolorbox}
  \end{questionBox}



\end{questionBox}

% Q1: y(2.5) -> y(0.25)


\end{document}

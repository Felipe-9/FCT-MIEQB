% !TEX root = ./CN_A-Tests_Resolutions.2022.1.tex
\providecommand\mainfilename{"./CN_A-Tests_Resolutions.tex"}
\providecommand \subfilename{}
\renewcommand   \subfilename{"./CN_A-Tests_Resolutions.2022.1.tex"}
\documentclass[\mainfilename]{subfiles}

% \tikzset{external/force remake=true} % - remake all

\begin{document}

% \graphicspath{{\subfix{./.build/figures/CN_A-Tests_Resolutions.2022.1}}}
% \tikzsetexternalprefix{./.build/figures/CN_A-Tests_Resolutions.2022.1/graphics/}

\mymakesubfile{1}
[CN A]
{Teste 2022 Resolução} % Subfile Title
{Teste 2022 Resolução} % Part Title

\begin{questionBox}1{ % Q1
    Considere o seguinte integral impróprio:
    \begin{BM}
        I=\int_0^1{\frac{1}{\sqrt{1-x^2}}\,\odif{x}}
    \end{BM}
    O valor da aproximação dado pela regra de Gauss com 2 pontos simples arredondado com 6 casas decimais é:
} % Q1
    \answer{}
    \begin{flalign*}
        &
            I
            =1/2
            \,\int_{-1}^{1}{
                f(y/2+1/2)\,\odif{y}
            }
            =1/2
            \,\int_{-1}^{1}{
                \frac{1}{\sqrt{1-(y/2+1/2)^2}}
                \,\odif{y}
            }
        &
    \end{flalign*}
\end{questionBox}

\setcounter{question}{2}

\begin{questionBox}1{ % Q3
    Seja \(\alpha\in\myrange{0,1}\) a raiz única da equação não linear \(f(x)=0\), sendo \(f(x)\) uma função contínua em \(\myrange{0,1}\). Sabe-se que
    \begin{BM}
        f(0)>0
        \quad f(1/2)>0
        \quad f(3/4)<0
        \quad f(5/8)>0
        \quad f(1)<0
    \end{BM}
    Considere a sucessão gerada pelo método da bissecação para boter uma aproximação para \(\alpha\), então tem-se
} % Q3
    \answer{}
    \begin{flalign*}
        &
            \begin{cases}
                x_0=\frac{0+1}{2}=1/2
                \land f(1/2)>0\land f(1)<0
                \implies &\alpha\in\myrange{1/2,1}
                \\
                x_1=\frac{1/2+1}{2}=3/4
                \land f(3/4)<0
                \land f(1/2)>0
                \implies &\alpha\in\myrange{1/2,3/4}
                \\
                x_2=\frac{1/2+3/4}{2}=5/8
                \land f(5/8)>0
                \land f(3/4)<0
                \implies &\alpha\in\myrange{3/4,5/8}
            \end{cases}
            &\\&
            x_3=\frac{5/8+3/4}{2}=11/16
            \implies
            \myvert{\alpha-x_3}=\frac{1}{2^{3+1}}=0.0625
        &
    \end{flalign*}
\end{questionBox}

\setcounter{question}{3}

\begin{questionBox}1{ % Q4
    Considere a seguinte regra de integração numérica
    \begin{BM}
        \int_{-1}^{1}{
            f_{(x)}\,\odif{x}
        }
        = c_1\,f_{(-1/3)}
        + c_2\,f_{(0)}
        + c_3\,f_{(1/3)}
        ;\quad \{c_1,c_2,c_3\}\in\mathbb{R}
    \end{BM}
    Quais dos valores que \(c_1,c_2,c_3\) devem assumir para que a regra seja exata para polinómios básicos de grau o mais elevado possível?
} % Q4
    \answer{}
    \begin{flalign*}
        &
            \begin{cases}
                f(x)=x^0
                \\
                f(x)=x^1
                \\
                f(x)=x^2
            \end{cases}
            \implies &\\&
            \implies
            \begin{cases}
                c_1+c_2+c_3 
                = \int_{-1}^{1}{x^0\,\odif{x}}
                =2
                \\
                -c_1/3+c_2*0+c_3/3
                =\int_{-1}^{1}{x^1\,\odif{x}}
                = (1-1)/2=0
                \\
                c_1/9+c_2*0+c_3/9
                =\int_{-1}^{1}{x^2\,\odif{x}}
                = (1+1)/3=2/3
            \end{cases}
            \implies &\\&
            \implies
            \begin{cases}
                c_1+c_2+c_3=2
                \\
                -c_1+c_3=0
                \\
                c_1+c_3=18/3=6
            \end{cases}
            \implies
            \begin{cases}
                c_1=3
                \\
                c_2=-4
                \\
                c_3=3
            \end{cases}
        &
    \end{flalign*}
\end{questionBox}

\setcounter{question}{5}
\begin{questionBox}1{ % Q6
    Considere a equação não linear \(\sin(x)\,\cos(x)=x-1\). a qual tem \emph{uma única raiz \(\alpha\)} no intervalo [1,1.5].
} % Q6
    \begin{questionBox}2{ % Q6.1
        Verifique que \(\alpha\) é um ponto fixo da função \(\phi(x)=\cos(x)\,\sin(x)+1\).
    } % Q6.1
        \answer{}
        \begin{flalign*}
            &
                f(x)=0\implies
                \sin(\alpha)\cos(\alpha)-\alpha+1=0
                \implies
                \sin(\alpha)\cos(\alpha)+1=\alpha
                \implies &\\&
                \implies
                \phi(\alpha)
                = \sin(\alpha)\cos(\alpha)+1
                = \alpha
                &\\&
                \therefore
                x\text{ é ponto fixo de }\phi(x)\text{ em }\myrange{1,1.5}
            &
        \end{flalign*}
    \end{questionBox}
    \begin{questionBox}2{ % Q6.2
        Mostre que a sucessão
        \begin{BM}
            \begin{cases}
                x_0\in\myrange{1,1.5}
                \\
                x_{n+1}=\phi{x_n}\quad n=0,1,2,...
            \end{cases}
        \end{BM}
        converge para \(\alpha\) e classifique o ponto fixo \(\alpha\) justificando conveninentemente.
    } % Q6.2
        \answer{}
        \begin{flalign*}
            &
                % 1
                \phi(x)\text{ esta bem def e continua em }\mathbb{R}
                &\\[6ex]&
                % 2
                \begin{cases}
                    \phi(1)=\dots=1.4546\dots\in\myrange{1,1.5}
                    \phi(1.5)
                    =\dots=1.07056\in\myrange{1,1.5}
                \end{cases}
                ; &\\&
                \phi'(x)=\cos^2(x)-\sin^2(x)<0
                \quad\forall x\in\myrange{1,1.5}
                &\\&
                \therefore
                1<1.07056\leq\phi(x)\leq1.4546<1.5
                &\\&
                \therefore
                \phi(x)\in\myrange{1,1.5}\forall\,x\in\myrange{1,1.5}
                % 3
                &\\[3ex]&
                \myvert{\phi'(x)}
                = \dots 
                = \myvert{\cos(2\,x)}
                \leq\myvert{-0.99}
                =0.99=M<1
            &
        \end{flalign*}
    \end{questionBox}
    \begin{questionBox}2{ % Q6.3
        Considerando \(x_0=1\) e a sucessão definida em b. diga quantas casas decimais significativas pode garantir para a iterada \(n=700\). Justifique devidamente.
    } % Q6.3
    \end{questionBox}
\end{questionBox}

\setcounter{question}{7}

\begin{questionBox}1{ % Q8
    Seja \textit{S} a função definida por
    \begin{BM}
        S(x)
        = \begin{cases}
            a\,x^3+b\,x^2+x\,5/3-1,\quad& -1\leq x<0
            \\ 
            -2\,a\,x^3+b\,x^2+x\,5/3-1,\quad& 0\leq x<1
            \\ 
            a\,x^3-2\,b\,x^2+x\,41/3-5,\quad& 1\leq x<2
        \end{cases}
    \end{BM}
    e que passa nos pontos \((-1,y_0),(0,y_1),(1,y_2),(2,y_2)\). Determine as constantes reais \textit{a}, \textit{b} de forma que \textit{S(x)} seja spline cúbico interpolador e diga se \textit{S(x)} pode ser um spline natural?
} % Q8
    \answer{}
    \paragraph{Continuidade}
    \begin{flalign*}
        &
            S(x)
            &\\[3ex]&
            \lim_{x\to0^-}{S(x)}
            =\lim_{x\to0^+}{S(x)}
            &\\[3ex]&
             \lim_{x\to1^-}{S(x)}
            =\lim_{x\to1^+}{S(x)}
            &\\[6ex]&
            \odv[order=1]{S(x)}{x}=\dots
            &\\[3ex]&
             \lim_{x\to0^-}{\odv[order=1]{S(x)}{x}}
            =\lim_{x\to0^+}{\odv[order=1]{S(x)}{x}}
            &\\[3ex]&
             \lim_{x\to1^-}{\odv[order=1]{S(x)}{x}}
            =\lim_{x\to1^+}{\odv[order=1]{S(x)}{x}}
            &\\[6ex]&
            \odv[order=2]{S(x)}{x}=\dots
            &\\[3ex]&
             \lim_{x\to0^-}{\odv[order=2]{S(x)}{x}}
            =\lim_{x\to0^+}{\odv[order=2]{S(x)}{x}}
            &\\[3ex]&
             \lim_{x\to1^-}{\odv[order=2]{S(x)}{x}}
            =\lim_{x\to1^+}{\odv[order=2]{S(x)}{x}}
        &
    \end{flalign*}
\end{questionBox}

\begin{sectionBox}1{Erro quadratico} % S
    
    \begin{BM}
        (E_m)^2
        = \sum_0^m{
            (f_{(x_i)}-p_{m\,(x_i)})^2
        }
    \end{BM}
    
\end{sectionBox}

\end{document}
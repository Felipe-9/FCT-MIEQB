% !TEX root = ./CN_A-Tests_Resolutions.2023.1.tex
\documentclass[./CN_A-Tests_Resolutions.tex]{subfiles}

% \tikzset{external/force remake=true} % - remake all

\begin{document}

% \graphicspath{{\subfix{./.build/figures/CN_A-Tests_Resolutions.2023.1}}}
% \tikzsetexternalprefix{./.build/figures/CN_A-Tests_Resolutions.2023.1/graphics/}

\mymakesubfile{1}[CN A]
{Teste 2023.1 Resolução} % Subfile Title
{Teste 2023.1 Resolução} % Part Title

\begin{questionBox}1{} % Q1
  \begin{BM}
    x=1/13
    \qquad
    \bar{x}=0.0769
    \\
    g(x)=\frac{1}{2/25-x}
    ;\qquad
    r_{g(x)}\approx\myvert{\frac{x\,g'(x)}{g(x)}}r_x,\quad g(x)\neq 0
  \end{BM}

  \answer{c}

  \begin{gather*}
    \frac{r_{g(x)}}{r_x}
    \approx 
    \myvert{\frac{x\,g'(x)}{g(x)}}
    =
    \myvert{\cfrac{
        (1/13)
        \left(
          \,(-1)
          \,(2/25-1/13)^{-2}
          (-1)
        \right)
      }{
        (2/25-1/13)
    }}
    = 2640625
    \\
    \therefore\text{ é mal condicionada}
  \end{gather*}

\end{questionBox}

\begin{questionBox}1{} % Q2
  \begin{BM}
    \myrange{a,b}
    \quad a<x_0<x_1<\dots<x_n=b
    \\
    f_{(x_i)},i=1,\dots,n
    \quad
    n\geq3
    \\
    \begin{cases}
      p\text{ polinomio lagrange grau n}
      \\
      q\text{ polinomio 2 grau minimos quadrados}
      \\
      S\text{ spline cubico para } x_i
    \end{cases}
  \end{BM}

  \answer{c}

\end{questionBox}

\begin{questionBox}1{} % Q3

  Considere a tabela para f
  \begin{center}
    \vspace{1ex}
    \begin{tabular}{C || *{3}{C}}

      x_i
      & 0 & 1 & 3
      \\\hline
      f(x_i)
      & c & -1 & 2
    \end{tabular}
    \vspace{2ex}
  \end{center}
  \begin{BM}
    c\in\mathbb{R}\quad p_1(x)\text{ approx f por min quadr}
    :p_1(x_i)=f(x_i),i=0,1,2
  \end{BM}

  \answer{a}

  \begin{gather*}
    p_{1\,(x_i)} 
    = \alpha_0 + \alpha_1\,x_i
    \\[1ex]
    \begin{cases}
      \alpha_0
      +\alpha_1\,(x_0+x_1+x_2)
      = y_0+y_1+y_2
      \\
      \alpha_0\,(x_0+x_1+x_2)
      +\alpha_1\,(x_0^2+x_1^2+x_2^2)
      = x_0\,y_0
      + x_1\,y_1
      + x_2\,y_2
      \\
      \alpha_0\,(x_0^2+x_1^2+x_2^2)
      +\alpha_1\,(x_0^3+x_1^3+x_2^3)
      = x_0^2\,y_0
      + x_1^2\,y_1
      + x_2^2\,y_2
    \end{cases}
    = \\
    = \begin{cases}
      \alpha_0
      +\alpha_1\,(1+3)
      = c-1+2
      \\
      \alpha_0\,(1+3)
      +\alpha_1\,(1+9)
      = -1
      + 3*2
      \\
      \alpha_0\,(1+9)
      +\alpha_1\,(1+27)
      = 1*-1
      + 9*2
    \end{cases}
    \implies \\
    \implies
    \begin{cases}
      5/2 = c
      \\
      \alpha_1
      = 18/12
      = 3/2
      \\
      \alpha_0
      = 1.7-\alpha_1\,28/10
      = -5/2
    \end{cases}
  \end{gather*}

\end{questionBox}

\begin{questionBox}1{} % Q4

  \begin{BM}
    I=\int_{-1}^{1}{f(x)\,\odif{x}}
    \\
    \odv[order=2]{f(x)}{x}
    = C,\,\forall\,x\in\mathbb{R}
    ,C\neq0
  \end{BM}

  \answer{b}

  \begin{gather*}
    I_{T}
    \approx 
    \frac{h}{2}\,(f_0+f_1)
    = \frac{1}{2}\,(f_0+f_1)
    \\[1ex]
    I_{PM}
    \approx 
    h\,f_{\left(\frac{x_0+x_1}{2}\right)}
    = f_{\left(\frac{-1+1}{2}\right)}
    = f_0
  \end{gather*}

\end{questionBox}

\begin{questionBox}1{} % Q5

  \begin{BM}
    I=\int_a^b{f(x)\,\odif{x}}
    ;\qquad
    J=\int_a^b{p_2(x)\,\odif{x}}
    \\
    x=\{a,(a+b)/2,b\}
  \end{BM}

  \answer{b}

  \begin{gather*}
    I_S
    \approx \frac{(b-a)/2}{3}\,(f_0+4\,f_1+f_2)
    \\[1ex]
    I_T
    \approx \frac{(b-a)/2}{2}\,(f_0+f_1)
  \end{gather*}
  
\end{questionBox}

\begin{questionBox}1{} % Q6

  Considere a seguinte tabela de val da fun \textit{g}
  \begin{center}
    \vspace{1ex}
    \begin{tabular}{C || *{4}{C}}
      x_i & -2 & -1 & 4 & 5
      \\\hline
      g(x_i)
      & -1/3 & -1/24 & 8/3 & 1/8
    \end{tabular}
    \vspace{2ex}
  \end{center}

\end{questionBox}

\begin{questionBox}2{} % Q6.1

  Pol de N com diff div interp da tabela

  \answer{}

  \begin{gather*}
    p_{n\,(x)}
    = g_0
    + \sum_{i=0}^{n-1}{
      \left(\prod_{j=0}^{i}{x-x_j}\right)
      \,g_{[x_0,\dots,x_{i+1}]}
    }
    % = \\
    % = -1/3
    % + \sum_{i=0}^{n-1}{
    %     \left(\prod_{j=0}^{i}{x-x_j}\right)
    %     \,g_{[x_0,\dots,x_{i+1}]}
    % }
    % \\[1ex]
    % p_{n\,(x)}=\sum_{i=0}^{n}{y_i\,L_{i\,(x)}}
    % ;\\
    % L_{i\,(x)}
    % =\varphi_{i(x)}
    % =\prod_{j=0}^{i-1}{\frac{x-x_j}{x_i-x_j}}
    % \,\prod_{j=i+1}^{n}{\frac{x-x_j}{x_i-x_j}}
  \end{gather*}

\end{questionBox}

\begin{questionBox}1{} % Q7

  Considere o integral
  \begin{BM}
    I=\int_0^3{\ln(x^2)\odif{x}}
  \end{BM}

\end{questionBox}

\begin{questionBox}2{} % Q7.1
  Determine \(\hat{I}\) por ponto med comp h=1
  % \answer{}
  % \begin{flalign*}
  %     &
  %         % Integração
  %         \int_{a=x_0}^{b=x_1}{f_{(x)}\,\odif{x}}
  %         \approx h\,f_{\left(\frac{x_0+x_1}{2}\right)}
  %     &
  % \end{flalign*}
\end{questionBox}

\begin{questionBox}2{} % Q7.2

  \(\hat{I}_S,n=2\)
  \begin{gather*}
    \hat{I}_S
    \approx 
    \frac{h}{3}
    \,(f_0+4\,f_1+f_2)
    = \frac{(3-1)/2}{3}
    \,(\ln(0^2)+4\,\ln(1^2)+\ln(2^2))
  \end{gather*}

\end{questionBox}


\begin{questionBox}1{} % Q8

  Seja \textit{S} a função definida por
  \begin{BM}
    S\begin{cases}
      -x^3-6\,x^2-8\,x+2,\quad&-2\leq x< -1
      \\
      \alpha\,x^3+\beta\,x+4,\quad&-1\leq x<0
      \\
      -2\,x+4,\quad0\leq x<1
    \end{cases}
  \end{BM}
  Det \(\alpha\text{ e }\beta\) de forma que seja spline cubico interp, s é spline nat?
  
  \answer{}

  \begin{gather*}
    \lim_{x\to-1^-}{S(x)}
    = 1-6+8
    = 3
    = \lim_{x\to-1^+}{S(x)}
    = -\alpha-\beta+4
    \\[1ex]
    \lim_{x\to0^-}{S(x)}
    = 4
    = \lim_{x\to0^+}{S(x)}
    = 4
    \\[1ex]
    % ================== order=1 ================= %
    \odv[order=1]{S(x)}{x}
    =\begin{cases}
      -3\,x^2-12\,x-8,\quad&-2\leq x< -1
      \\
      3\,\alpha\,x^2+\beta,\quad&-1\leq x<0
      \\
      -2,\quad0\leq x<1
    \end{cases}
    \\[1ex]
    \lim_{x\to-1^-}{\odv[order=1]{S(x)}{x}}
    = -3+12-8 = 1
    = \lim_{x\to-1^+}{\odv[order=1]{S(x)}{x}}
    = 3\,\alpha+\beta
    \\[1ex]
    \lim_{x\to0^-}{\odv[order=1]{S(x)}{x}}
    = \beta
    = \lim_{x\to0^+}{\odv[order=1]{S(x)}{x}}
    = -2
    % ================== order=2 ================= %
    \\[1ex]
    \odv[order=2]{S(x)}{x}
    =\begin{cases}
      -6\,x-12,\quad&-2\leq x< -1
      \\
      6\,\alpha\,x,\quad&-1\leq x<0
      \\
      0,\quad0\leq x<1
    \end{cases}
    \\[1ex]
    \lim_{x\to-1^-}{\odv[order=2]{S(x)}{x}}
    = 6-12=-6
    = \lim_{x\to-1^+}{\odv[order=2]{S(x)}{x}}
    = -6\,\alpha
    \\[1ex]
    \lim_{x\to0^-}{\odv[order=2]{S(x)}{x}}
    = 0
    = 0
    \\[1ex]
    \implies
    \begin{cases}
      \beta=-2
      \\
      -\alpha-\beta+4
      = -\alpha+2+4
      =3
      \implies
      \alpha = 3
      \\
      1 
      \neq 3\,\alpha+\beta
      = 3*3-2
      = 7
    \end{cases}
    \\
    \therefore
    \text{não interpola nem é spline nat}
  \end{gather*}

\end{questionBox}

\end{document}

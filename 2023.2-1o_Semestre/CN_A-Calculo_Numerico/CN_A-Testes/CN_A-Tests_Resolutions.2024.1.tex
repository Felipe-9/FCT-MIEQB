% !TEX root = ./CN_A-Tests_Resolutions.2024.1.tex
\documentclass["CN_A-Tests_Resolutions.tex"]{subfiles}

% \tikzset{external/force remake=true} % - remake all

\begin{document}

% \graphicspath{{\subfix{./.build/figures/CN_A-Tests_Resolutions.2024.1}}}
% \tikzsetexternalprefix{./.build/figures/CN_A-Tests_Resolutions.2024.1/graphics/}

\mymakesubfile{1}[CN A]
{Resolução Teste 2024.1} % Subfile Title
{Resolução Teste 2024.1} % Part Title

\begin{questionBox}1{} % Q1

  Erro relativo para approx \(\hat{I}\)

  \answer{}

  \begin{gather*}
      r_{\hat{I}}
      = \frac{\myvert{\hat{I}-I}}{\myvert{I}}
      \leq \frac{\myvert{M}}{\myvert{I}}
      \therefore\text{ Alternativa d)}
    \end{gather*}

\end{questionBox}

\begin{questionBox}1{} % Q2

  \begin{itemize}
    \item \textit{f} def no intervalo \(\myrange{0,3}\)
    \item \(p_2(x)=2\,x^2\,3\,x\) Pol de grau \(\leq 2\) interp de f em \(\{0,1,2\}\)
    \item \(f\myrange{x_0,\dots,x_4}=4; x_3 = 3\)
  \item \(p_3(x)=?\) interp de f em \((x_i,f(x_i)),i=\{0,1,2,3\}\)
  \end{itemize}

  \answer{}

  \begin{gather*}
    % n: 3
    p_{3\,(x)}
    = f(x_0)
    + \sum_{i=0}^{3-1}{
      \left(\prod_{j=0}^{i}{x-x_j}\right)
      \,f\myrange{x_0,\dots,x_{i+1}}
    }
    = \\
    = p_{2\,(x)}
    + f[x_0,x_1,x_2,x_3]
    \,( x-x_0)
    \,( x-x_1)
    \,( x-x_2)
    = \\
    = 2\,x^2+3\,x
    + 4
    \,( x-0)
    \,( x-1)
    \,( x-2)
    = \\
    = 2\,x^2+3\,x
    +4(
      x^3-2\,x^2
      -x^2+2\,x
    )
    = \\
    =x^3*4
    -x^2*10
    +x^1*11
  \end{gather*}

  Resposta: aliena c)

  % \begin{center}
  %   \vspace{1ex}
  %   \begin{tabular}{C *{4}{C}}
  %     \toprule
  %
  %     x & f(x)
  %     & f\myrange{\cdot,\cdot}
  %     & f\myrange{\cdot,\cdot,\cdot}
  %     & f\myrange{\cdot,\cdot,\cdot,\cdot}
  %
  %     \\\midrule
  %
  %     1 & 
  %     & \multirow[c]{2}{*}{\(\)} % f[1,2]
  %     & \multirow[c]{3}{*}{\(\)} % f[1,2,3]
  %     & \multirow[c]{4}{*}{\(\)} % f[1,2,3,4]
  %     \\ 2 & 
  %     & \multirow[c]{2}{*}{\(\)} % f[2,3]
  %     & \multirow[c]{3}{*}{\(\)} % f[2,3,4]
  %     \\ 3 & 
  %     & \multirow[c]{2}{*}{} % f[3,4]
  %     \\ 4 & 
  %
  %     \\\bottomrule
  %   \end{tabular}
  %   \vspace{2ex}
  % \end{center}

\end{questionBox}

\begin{questionBox}1{} % Q3
  
  \begin{itemize}
    \item \(I=int_0^1{f(x)\,\odif{x}}\)
    \item \(I_T= 6.5\)
    \item \(I_{PM}=4.25\)
    \item \(I_{S}=?\)
  \end{itemize}

  \answer{}

  Resposta b)

  \begin{gather*}
    % integração
    \hat{i}
    \approx\int_{a=0}^{b=1}{f_{(x)}\,\odif{x}}
    =\frac{h}{3}\,(f_0+4\,f_1+f_2)
    = \frac{1}{3}\,(
      2\,\left(\frac{h}{2}(f(0)+f(1))\right)
      + 4\,( h\,f(0.5))
    )
    = \\
    = \frac{1}{3}\,(
      2\,I_T
      + 4\,I_{PM}
    )
    = \\
    = \frac{1}{3}\,(
      2*6.5
      + 4* 4.25
    )
    = 10
    % 
    % 
    % 
    ; \\[1ex]
    % integração
    I_{PM}
    \approx h\,f_{\left(\frac{0+1}{2}\right)}
    = h\,f(0.5)
    = 4.25
    % 
    % 
    % 
    ; \\[1ex]
    % integração
    I_T
    \approx \frac{h}{2}\,(f(0)+f(1))
    = 6.5
  \end{gather*}

\end{questionBox}

\begin{questionBox}1{} % Q4

  \begin{itemize}
    \item \(p_4(x)\)
    \item \(x=\{0,1,\dots,4\}\)
    \item S spline cubic de f em \numlist*{0;2;4}
    \item q é pol grau 4 por min quadrados
  \end{itemize}

  \answer{}

  Resposta alinea b)

\end{questionBox}

\begin{questionBox}1{} % Q5

  \begin{itemize}
    \item \(f'(0)=5\)
    \item \(f"(2)=-1\)
  \end{itemize}
  \begin{center}
    \vspace{1ex}
    \begin{tabular}{C || *{3}{C}}

      x_i    & 0 & 1 & 2
      \\\hline
      f(x_i) & 6 & 9 & 6
    \end{tabular}
    \vspace{2ex}
  \end{center}
  \begin{BM}
    S(x)
    = \begin{cases}
      -2\,x^2 + 5\,x + 6 ,\quad& 0 \leq x < 1
      \\ 4\,x^3 - 18\,x^2 + 23\,x ,\quad& 1 \leq x \leq 2
    \end{cases}
  \end{BM}

  \answer{}

  Resposta b)

  \begin{gather*}
    S(0) 
    = -2\,(0)^3 + 5*0 + 6
    = 6
    % 
    % 
    % 
    ; \\[1ex]
    S(1) 
    = -2\,(1)^3 + 5\,(1) 6
    = 9
    = \\
    = 4\,(1)^3 - 18*(1)^2 + 23*(1)
    = 9
    % 
    % 
    % 
    ; \\[1ex]
    S(2) 
    = 4\,(2)^3 - 18*(2)^2 + 23*(2)
    = 6
    % 
    % 
    % 
    \\[3ex]
    \therefore S\text{ interpola }f
    % 
    % 
    % 
    ; \\[1ex]
    \odv{S(x)}{x}
    \begin{cases}
      -4\,x + 5 ,\quad& 0 \leq x < 1
      \\ 12\,x^2 - 36\,x + 23 ,\quad& 1 \leq x \leq 2
    \end{cases}
    % 
    % 
    % 
    ; \\[1ex]
    \lim_{x\to 1^-}{\odv{S(x)}{x}}
    = -4*1 + 5
    = 1
    = \\
    \neq \lim_{x\to 1^+}{\odv{S(x)}{x}}
    = 12 - 36 + 23
    = -1
    ; \\
    S\text{ Não é spline}
  \end{gather*}

\end{questionBox}

\begin{questionBox}1{} % Q6

  Considere a tab com val f
  \begin{center}
    \vspace{1ex}
    \begin{tabular}{C || *{3}{C}}

      x_i    & -2  & 0  & 1
      \\\hline
      f(x_i) & -42 & -2 & 3

    \end{tabular}
    \vspace{2ex}
  \end{center}

\end{questionBox}

\begin{questionBox}2{} % Q6.1

  Dete o pol de lagrange inter da tab

  \answer{}

  \begin{gather*}
    % n: 3
    p_{3}(x)
    = \sum_{i=0}^{3}{ y_i\,L_{i}(x) }
    = -42*L_0
    - 2\,L_1
    + 3\,L_2
    = \\[1ex]
    = -42*\frac{1}{2}( x^2 -3\,x +2)
    - 2\,(-( x^2-2\,x))
    + 3\,\left(\frac{1}{2}( x^2-x )\right)
    % 
    % 
    % 
    ;\\[1ex]
    L_i(x)
    =\prod_{j=0}^{i-1}{\frac{x-x_j}{x_i-x_j}}
    \,\prod_{j=i+1}^{3}{\frac{x-x_j}{x_i-x_j}}
    % 
    % 
    % 
    ;\\[1ex]
    L_0(x)
    =\prod_{j=0}^{0-1}{\frac{x-x_j}{0-x_j}}
    \,\prod_{j=0+1}^{3}{\frac{x-x_j}{0-x_j}}
    = \frac{x-1}{0-1}
    \,\frac{x-2}{0-2}
    = \frac{1}{2}( x^2 -3\,x +2)
    % 
    % 
    % 
    ;\\[1ex]
    L_1(x)
    =\prod_{j=0}^{1-1}{\frac{x-x_j}{1-x_j}}
    \,\prod_{j=1+1}^{3}{\frac{x-x_j}{1-x_j}}
    = \frac{x-0}{1-0}
    \,\frac{x-2}{1-2}
    = -( x^2-2\,x)
    % 
    % 
    % 
    ;\\[1ex]
    L_2(x)
    =\prod_{j=0}^{2-1}{\frac{x-x_j}{2-x_j}}
    \,\prod_{j=2+1}^{3}{\frac{x-x_j}{2-x_j}}
    = \frac{x-0}{2-0}
    \,\frac{x-1}{2-1}
    = \frac{1}{2}( x^2-x )
  \end{gather*}

\end{questionBox}

\begin{questionBox}2{} % Q6.2

  Approx p \(f(-1)\) e majorante absoluto para f(-1)
  \begin{itemize}
    \item \(\myvert{f^{3}(x)}\leq 1, \forall\,x\in\myrange{-2,1}\)
  \end{itemize}

  \answer{}

  \begin{gather*}
    f(-1)
    \approx p_3(-1)
    = \\
    = -42*\frac{1}{2}( (-1)^2 -3\,(-1) +2)
    - 2\,(-( (-1)^2-2\,(-1)))
    + 3\,\left(\frac{1}{2}( (-1)^2-(-1) )\right)
    = \\
    = 57
    % 
    % 
    % 
    \\[1ex]
    \eta_{f(-1)} \leq 1
    % 
    % 
    % 
    \\[1ex]
    r_{f(-1)}
    = \frac
    { \myvert{f(-1)-p_3(-1)} }
    { \myvert{f(-1)} }
  \end{gather*}

\end{questionBox}

\begin{questionBox}2{} % Q6.3

  Pol de grau 1 por min quadrad

  \answer{}

  \begin{gather*}
    p_{1,(x)} 
    = \sum_{i=0}^1{\alpha_i\,x^i}
    % 
    % 
    % 
    ; \\[1ex]
    \begin{bmatrix}
      \sum_{i=0}^1{\left(
          \alpha_i\,\sum_{j=0}^{n}{x_{j}^{i+k}}
      \right)}
      = \sum_{i=0}^{n}{x_i^k\,y_i}
      \\
      k\in\myrange{0,1}
    \end{bmatrix}
    = \\
    = \begin{bmatrix}
      \left(
        \alpha_0\,\sum_{j=0}^{3}{x_{j}^{0+0}}
        +\alpha_1\,\sum_{j=0}^{3}{x_{j}^{1+0}}
      \right)
      = x_0^0\,y_0
      + x_1^0\,y_1
      + x_2^0\,y_2
      \\
      \left(
        \alpha_0\,\sum_{j=0}^{3}{x_{j}^{0+1}}
        +\alpha_1\,\sum_{j=0}^{3}{x_{j}^{1+1}}
      \right)
      = x_0^1\,y_0
      + x_1^1\,y_1
      + x_2^1\,y_2
    \end{bmatrix}
    = \\
    = \begin{bmatrix}
      \left(
        \alpha_0\,\sum_{j=0}^{3}{x_{j}^{0+0}}
        +\alpha_1\,\sum_{j=0}^{3}{x_{j}^{1+0}}
      \right)
      = x_0^0\,y_0
      + x_1^0\,y_1
      + x_2^0\,y_2
      \\
      \left(
        \alpha_0\,\sum_{j=0}^{3}{x_{j}^{0+1}}
        +\alpha_1\,\sum_{j=0}^{3}{x_{j}^{1+1}}
      \right)
      = x_0^1\,y_0
      + x_1^1\,y_1
      + x_2^1\,y_2
    \end{bmatrix}
  \end{gather*}

\end{questionBox}

\end{document}

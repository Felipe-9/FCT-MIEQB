% !TEX root = ./CN_A-Exercicios_Resolucoes.2.tex
\providecommand\mainfilename{"CN_A-Exercicios_Resolucoes.tex"}
\providecommand \subfilename{}
\renewcommand   \subfilename{"./CN_A-Exercicios_Resolucoes.2.tex"}
\documentclass[\mainfilename]{subfiles}

% \tikzset{external/force remake=true} % - remake all

\begin{document}

% \graphicspath{{\subfix{./.build/figures/CN_A-Exercicios_Resolucoes.2}}}
% \tikzsetexternalprefix{./.build/figures/CN_A-Exercicios_Resolucoes.2/graphics/}

\mymakesubfile{2}
[CN A]
{Interpolação e Aproximação Polinomial} % Subfile Title
{Interpolação e Aproximação Polinomial} % Part Title

\setcounter{question}{9}

\begin{questionBox}1{ % Q10
    Considere-se uma função real de variável real, \textit{g}, cujos valores se conhecem nos nodos \(x_0=-2 \text{, } x_1 =-1 \text{ e } x_2=1\):
} % Q10
    \begin{center}
        \vspace{1ex}
        \begin{tabular}{C || *{3}{C}}
            % \toprule
            
                x & -2 & -1 & 1
            
                % \\
            \\\midrule
            
                g(x)
                & \alpha & \beta & \gamma
                % & 1.409297 
                % & 0.900117
                % & 0.474453
                % & -0.506802
            
            % \\\bottomrule
        \end{tabular}
        \vspace{2ex}
    \end{center}
\end{questionBox}

\setcounter{question}{13}

\begin{questionBox}1{ % Q14
    Considere a função seccionalmente polinomial, \(S(x)\), definida por:
    \begin{BM}
        \begin{cases}
            x^2,\quad& -1\leq x<0
            \\
            a\,x^3+b\,x^2+c\,x
            ,\quad& 0\leq x<1
            \\
            2-x
            ,\quad& 1\leq x\leq 2
        \end{cases}
    \end{BM}
    Onde \(a,\ b\text{ e }c\) são constantes reais.
    \\
    Diga, justificando, se \(S(x)\) pode ser um spline cúbico.
} % Q14
    \answer{}
    \begin{flalign*}
        &
            \lim_{x\to0^-}{S(x)}
            = 0^2
            = 0
            = \lim_{x\to0^+}{S(x)}
            = a\,0^3+b\,0^2+c\,0
            = 0
            ; &\\[3ex]&
            \lim_{x\to1^-}{S(x)}
            = a\,1^3+b\,1^2+c\,1
            = a+b+c
            = \lim_{x\to1^+}{S(x)}
            = 2-1
            = 1
            \implies &\\&
            \implies
            a+b+c=1
            ; &\\[3ex]&
            \odv{S(x)}{x}
            = \begin{cases}
                2\,x,\quad& -1\leq x<0
                \\
                3\,a\,x^2+2\,b\,x+c
                ,\quad& 0\leq x<1
                \\
                -1
                ,\quad& 1\leq x\leq 2
            \end{cases}
            \implies &\\[3ex]&
            \implies
            \lim_{x\to0^-}{\odv{S(x)}{x}}
            = 2*0
            = 0
            = \lim_{x\to0^+}{\odv{S(x)}{x}}
            = 3\,a\,0^2+2\,b\,0+c
            = c
            \implies
            c=0
            ; &\\[3ex]&
            \lim_{x\to1^-}{\odv{S(x)}{x}}
            = 3\,a\,1^2+2\,b\,1+c
            = 3\,a+2\,b
            = \lim_{x\to1^+}{\odv{S(x)}{x}}
            = -1
            \implies &\\&
            \implies
            3\,a+2\,b
            =-1
            ; &\\[3ex]&
            \odv[order=2]{S(x)}{x}
            = \begin{cases}
                2,\quad& -1\leq x<0
                \\
                6\,a\,x+2\,b
                ,\quad& 0\leq x<1
                \\
                0
                ,\quad& 1\leq x\leq 2
            \end{cases}
            \implies &\\[3ex]&
            \implies
            \lim_{x\to0^-}{\odv[order=2]{S(x)}{x}}
            = 2
            = \lim_{x\to0^+}{\odv[order=2]{S(x)}{x}}
            = 6\,a\,0+2\,b
            = 2\,b
            \implies
            b=1
            ; &\\[3ex]&
            \lim_{x\to1^-}{\odv[order=2]{S(x)}{x}}
            = 6\,a\,1+2\,b
            = 6\,a+2
            = \lim_{x\to1^+}{\odv[order=2]{S(x)}{x}}
            =0
            \implies
            a=-1/3
            &\\[3ex]&
            \therefore\begin{cases}
                a+b+c=1
                \\
                c=0
                \\
                b=1
                \\
                6\,a+2\,b=0
                \implies 2=0
            \end{cases}
            % \implies &\\&
            \implies
            S(x)\text{ não pode ser spline}
        &
    \end{flalign*}
\end{questionBox}

\begin{questionBox}1{ % Q15
    Considere a seguinte tabela de valores de uma função \textit{g}
} % Q15
    \begin{center}
        \vspace{1ex}
        \begin{tabular}{C || *{4}{C}}
            % \toprule
            
                x & 1 & 2 & 4 & 8
            
            \\\hline
            % \\\midrule
            
                g(x)
                & -2 & 6 & 2 & 40
            
            % \\\bottomrule
        \end{tabular}
        \vspace{2ex}
    \end{center}
    Determine a expressão do spline cúbico natural, \(S(x)\), interpolador de \(g(x)\) nos pontos tabelados.
    \answer{}
    \begin{flalign*}
        &
            \left\{
                \begin{aligned}
                    h_0\,m_0
                    + 2\,(h_0+h_1)\,m_1
                    + h_1\,m_2
                    = 6\left(
                        \frac{y_2-y_1}{h_1}
                        -\frac{y_1-y_0}{h_0}
                    \right)
                    \\
                    h_1\,m_1
                    + 2\,(h_1+h_2)\,m_2
                    + h_2\,m_3
                    = 6\left(
                        \frac{y_3-y_2}{h_2}
                        -\frac{y_2-y_1}{h_1}
                    \right)
                    \\
                    m_0=m_3=0;
                    \\
                    h_i=x_{i+1}-x_i,\quad i\in\myrange{0,2}
                    \\
                    h_0=1,
                    h_1=2,
                    h_2=4
                    \\
                    y_0=- 2,
                    y_1=  6,
                    y_2=  2,
                    y_3= 40
                \end{aligned}
            \right\}
            = &\\&
            = \left\{
                \begin{aligned}
                    2*(1+2)\,m_1
                    + 2\,m_2
                    = 6\left(
                        \frac{2-6}{2}
                        -\frac{6+2}{1}
                    \right)
                    \\
                    2\,m_1
                    + 2\,(2+4)\,m_2
                    = 6\left(
                        \frac{40-2}{4}
                        -\frac{2-6}{2}
                    \right)
                \end{aligned}
            \right\}
            = &\\&
            = \left\{
                \begin{aligned}
                    6\,m_1
                    + 2\,m_2
                    = -60
                    \\
                    2\,m_1
                    + 12\,m_2
                    = 69
                \end{aligned}
            \right\}
            \implies &\\&
            \implies
            \begin{bmatrix}
                6 & 2
                \\ 2 & 12
            \end{bmatrix}
            \begin{bmatrix}
                m_1\\m_2
            \end{bmatrix}
            = \begin{bmatrix}
                -60\\69
            \end{bmatrix}
        &
    \end{flalign*}
    \begin{flalign*}
        &
            S_i(x)
            = -\frac{(x-x_{i+1})^3}{6\,h_i}
            \,m_i
            + \frac{(x-x_i)^3}{6\,h_i}
            \,m_{i+1}
            + &\\&
            + \left(
                f_i
                - \frac{h_i^2}{6}m_{i}
            \right)
            \,\frac{x_{i+1}-x}{h_i}
            + \left(
                f_{i+1}
                - \frac{h_i^2}{6}m_{i+1}
            \right)
            \,\frac{x-x_{i}}{h_i}
            ;&\\[6ex]&
            \begin{cases}
                h_0\,m_0
                + 2\,(h_0+h_1)\,m_1
                + h_1\,m_2
                = 6\left(
                    \frac{y_2-y_1}{h_1}
                    -\frac{y_1-y_0}{h_0}
                \right)
                \\
                h_1\,m_1
                + 2\,(h_1+h_2)\,m_2
                + h_2\,m_3
                = 6\left(
                    \frac{y_3-y_2}{h_2}
                    -\frac{y_2-y_1}{h_1}
                \right)
            \end{cases}
        &
    \end{flalign*}
\end{questionBox}

\setcounter{question}{18}

\begin{questionBox}1{ % Q19
    Considere a tabela de valores da função \textit{f}
} % Q19
    \begin{center}
        \vspace{1ex}
        \begin{tabular}{C || *{3}{C}}
            
                x_i
                & -3 & 0 & 2
            
            \\\hline
            
                f(x_i)
                & 2 & 4 & 12
        \end{tabular}
        \vspace{2ex}
    \end{center}
    \begin{questionBox}2{ % Q19.1
        Determine o polinómio de grau menor ou igual a 1 que melhor aproxima a função tabelada no sentido dos mínimos quadrados, no intervalo \(\myrange{-3,2}\)
    } % Q19.1
        \answer{}
        \begin{flalign*}
            &
                \begin{cases}
                    a_0\,\sum_{i=0}^{2}{1}
                    + a_1\,\sum_{i=0}^{2}{x_i}
                    = \sum_{i=0}^{2}{y_i}
                    \\
                    a_0\,\sum_{i=0}^{2}{x_i}
                    + a_1\,\sum_{i=0}^{2}{x_i^2}
                    = \sum_{i=0}^{2}{y_i\,x_i}
                \end{cases}
                = \begin{cases}
                    a_0\,(3)
                    + a_1\,(-1)
                    = 18
                    \\
                    a_0\,(-1)
                    + a_1\,(13)
                    = 18
                \end{cases}
                = &\\&
                = \begin{cases}
                    a_0 = 126/19
                    \\
                    a_1 = 36/19
                \end{cases}
                \implies &\\&
                \implies
                p_1(x)
                = 126/19 + x\,36/19
            &
        \end{flalign*}
    \end{questionBox}

    \begin{questionBox}2{ % Q19.2
        Mostre que
        \begin{BM}
            \sum_{i=0}^2{
                (
                    f(x_i)
                    -(\gamma_1\,x_i+\gamma_0)
                )^2
            }\geq 200/19
            ,\forall\,\gamma_0,\gamma_1\in\mathbb{R}
        \end{BM}
    } % Q19.2
        \answer{}
        \begin{flalign*}
            &
                \sum_0^2{
                    \left(
                        f(x_i)
                        - p_1(x_i)
                    \right)^2
                }
                = \dots
            &
        \end{flalign*}
    \end{questionBox}

    \begin{questionBox}2{ % Q19.3
        Seja \(p_2(x)\) o polinomio de grau menor ou igual a 2 interpolador de \textit{f} nos pontos tabelados.\\
        Justifique que a aproximação quadrática que melhor aproxima o conjunto de pontos 
        \(\left\{
            (-3,2),
            (0,4),
            (2,12),
        \right\}\)
        , no sentido dos mínimos quadrados, é o polinómio \(p_2\).
    } % Q19.3
        \answer{}
        \begin{center}
            \vspace{1ex}
            \begin{tabular}{*{4}{C}}
                \toprule
                
                    x & f(x) & f[,] & 
                
                \\\midrule
                
                    -3 & 2 & 
                    \multirow{2}{*}{
                        \(\frac{4-2}{0-(-3)}=2/3\)
                    }
                    & \multirow{3}{*}{
                        \(\frac{4-2/3}{2-(-3)}=2/3\)
                    }
                    \\
                    0 & 4 & 
                    \multirow{2}{*}{
                        \(\frac{12-4}{2-0}=4\)
                    }
                    \\
                    2 & 12
                
                \\\bottomrule
            \end{tabular}
            \vspace{2ex}
        \end{center}
        \begin{flalign*}
            &
                p_2(x)
                = 2
                + (x+3)\,2/3
                + (x+3)\,(x-0)\,2/3
                \text{ é polinómio de 2 grau}
            &
        \end{flalign*}
    \end{questionBox}
\end{questionBox}

\setcounter{question}{17}

\begin{questionBox}1{ % Q18
    A seguinte tabela represetna a população da China (em milhares de milhões de habitantes) arredondada a 5 dígitos:
} % Q18
    \begin{center}
        \vspace{1ex}
        \begin{tabular}{L || *{4}{C}}
            % \toprule
            
                t & 1990 & 2000 & 2010 & 2020
            
            \\\hline
            
                P(t) 
                & 1.1769,
                & 1.2906,
                & 1.3688,
                & 1.4393,
            
            % \\\bottomrule
        \end{tabular}
        \vspace{2ex}
    \end{center}
    Suponha que há uma relação linear entre a data \textit{t} (em anos) e a população \(P(t)\), isto é, que se verifica a relação \(p_1(t) = \alpha\,t + \beta\), onde \(\alpha\) e \(\beta\) são constantes reais (\(\alpha\neq0\)). Com base nestes dados, utilize o método dos mínimos quadrados para obter uma estimativa da população chinesa em 2015.
    \answer{}
    \begin{flalign*}
        &
            t\begin{cases}
                   0: 1990,
                \\ 1: 2000,
                \\ 2: 2010,
                \\ 3: 2020,
            \end{cases}
            ;&\\[3ex]&
            N\,C
            = \begin{bmatrix}
                \sum_{i=0}^{3}{t_i^0}
                & \sum_{i=0}^{3}{t_i^1}
                \\ \sum_{i=0}^{3}{t_i^1}
                &  \sum_{i=0}^{3}{t_i^2}
            \end{bmatrix}
            \,\begin{bmatrix}
                \beta\\alpha
            \end{bmatrix}
            = \begin{bmatrix}
                4
                & 8020
                \\ 8020
                & 16080600
            \end{bmatrix}
            \,\begin{bmatrix}
                \beta\\\alpha
            \end{bmatrix}
            = &\\[2ex]&
            = B
            = \begin{bmatrix}
                \sum_{i=0}^{3}{P(t_i)}
                \\ \sum_{i=0}^{3}{t_i\,P(t_i)}
            \end{bmatrix}
            = \begin{bmatrix}
                5.2756
                \\ 10581.905
            \end{bmatrix}
            ; &\\[3ex]&
            \therefore
            \begin{cases}
                \alpha\approx  0.00865
                \\ \beta\approx-16.0324
            \end{cases}
            ; &\\[3ex]&
            p_1(2015) \approx 1.39735
        &
    \end{flalign*}
\end{questionBox}

\begin{questionBox}1{ % Q19
    Considere a tabela de valores da função \textit{f}
} % Q19
    \begin{center}
        \vspace{1ex}
        \begin{tabular}{L || *{3}{C}}
            % \toprule
            
                x_i
                & -3
                & 0
                & 2
            
            \\\hline
            
                f(x_i)
                & 2
                & 4
                & 12
            
            % \\\bottomrule
        \end{tabular}
        \vspace{2ex}
    \end{center}
    \begin{questionBox}2{ % Q19.1
        Determine o polinómio de grau menor ou igual a 1 que melhor aproxima a função tabelada, no sentido dos mínimos quadrados, no intervalo \(\myrange{-3,2}\).
    } % Q19.1
        \answer{}
        \begin{flalign*}
            &
                p_i(x)= c_o+c_1\,x
                ; &\\[3ex]&
                N\,C
                = \begin{bmatrix}
                       \sum_{i=0}^{2}{x_i^0}
                    &  \sum_{i=0}^{2}{x_i^1}
                    \\ \sum_{i=0}^{2}{x_i^1}
                    &  \sum_{i=0}^{2}{x_i^2}
                \end{bmatrix}
                \,\begin{bmatrix}
                    c_0 & c_1
                \end{bmatrix}
                = \begin{bmatrix}
                       3
                    &  -1
                    \\ -1
                    &  13
                \end{bmatrix}
                \,\begin{bmatrix}
                    c_0 & c_1
                \end{bmatrix}
                = &\\[2ex]&
                = B
                = \begin{bmatrix}
                    \sum_{i=0}^{2}{f(x_i)}
                    \\ \sum_{i=0}^{2}{x_i\,f(x_i)}
                \end{bmatrix}
                = \begin{bmatrix}
                    18 \\ 18
                \end{bmatrix}
                % \implies &\\&
                \implies
                c\begin{cases}
                    0: 126/19
                    \\ 1:36/19
                \end{cases}
                &\\[3ex]&
                \therefore
                p_1(x)
                = 126/19 + x\,36/19
            &
        \end{flalign*}
    \end{questionBox}
    \begin{questionBox}2{ % Q19.2
        Mostre que
        \begin{BM}
            \sum_{i=0}^{2}{\left(
                f(x_i)
                - (\gamma_1\,x_i+\gamma_0)
            \right)^2}
            \geq 200/19,
            \quad\forall\,\gamma_0,\gamma_1\in\mathbb{R}
        \end{BM}
    } % Q19.2
        \answer{}
        \sisetup{
            % scientific / engineering / input / fixed
            exponent-mode           = fixed,
            round-mode              = places,        % figures/places/unsertanty/none
            round-precision         = 6,
        }
        \begin{flalign*}
            &
                % \sum_{i=0}^{2}{\left(
                %     f(x_i)
                %     - (\gamma_1\,x_i+\gamma_0)
                % \right)^2}
                % \geq 200/19
                p_1(x)\in\left\{
                    \gamma_1\,x+\gamma_0,\gamma_0\in\mathbb{R}
                \right\}
                % ; &\\&
                ;\quad
                p_1\text{ Minimiza o erro quadrático } E^2
                &\\[3ex]&
                E^2
                = \sum_{i=0}^{2}{\left(
                    f(x_i)-p_i(x_i)
                \right)^2}
                = \left(
                    \begin{aligned}
                        &
                            (2-\num{0.9473684210526315})^2
                        &+\\+&
                            (4-\num{6.631578947368421})^2
                        &+\\+&
                            (12-\num{10.421052631578947})^2
                        &
                    \end{aligned}
                \right)
                \cong \num{10.526315789473689}
            &
        \end{flalign*}
    \end{questionBox}
    \begin{questionBox}2{ % Q19.3
        Seja \(p_2(x)\) o polinómio de grau menor ou igual a 2 interpolador de \textit{f} nos pontos tabelados.\\
        Justifique que a aproximação quadrática que melhor aproxima o conjunto de pontos \(\{(-3,2),(0,4),(2,12)\}\), no sentido dos mínimos quadrados, é o polinomio \(p_2\)
    } % Q19.3
        \answer{}
        \begin{center}
            \vspace{1ex}
            \begin{tabular}{C | *{3}{C}}
                \toprule
                
                    x_i 
                    & f(x_i)
                    & f[.]
                    & f[..]
                
                \\\midrule
                
                    -3 & 2
                    & \multirow{2}{*}{2/3}
                    & \multirow{3}{*}{2/3}
                    \\
                    0 & 4 
                    & \multirow{2}{*}{4}
                    \\
                    2 & 12
                
                \\\bottomrule
            \end{tabular}
            \vspace{2ex}
        \end{center}
    \end{questionBox}
\end{questionBox}
\end{document}
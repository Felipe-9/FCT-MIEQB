% !TEX root = ./CN_A-Exercicios.3.tex
\providecommand\mainfilename{"./CN_A-Exercicios.tex"}
\providecommand \subfilename{}
\renewcommand   \subfilename{"./CN_A-Exercicios.3.tex"}
\documentclass[\mainfilename]{subfiles}

% \tikzset{external/force remake=true} % - remake all

\begin{document}

% \graphicspath{{\subfix{./.build/figures/CN_A-Exercicios.3}}}
% \tikzsetexternalprefix{./.build/figures/CN_A-Exercicios.3/graphics/}

\mymakesubfile{3}
[CN A]
{Areas?} % Subfile Title
{Areas?} % Part Title

\begin{questionBox}1{ % Q1
    \begin{BM}
        I = \int_{\pi/6}^{\pi/2}{
            e^{\sin{x}}
            \,\odif{x}
        }
    \end{BM}
} % Q1
    \answer{}
    \begin{questionBox}2{ % Q1.1
    } % Q1.1
        \answer{}
        \begin{flalign*}
            &
                g(x)
                = e^{\sin(x)}
                ; \hat{I}
                = h\,g((a+b)/2)
                = \frac{I}{3}
                \,g(\pi/3)=\pi/3\,e^{\sin(\pi/3)}
                \approx
                2.489652
                ; &\\&
                \myvert{\varepsilon}
                = \myvert{
                    \frac{h^3}{24}
                    \,g"(\gamma)
                }
                \leq\myvert{
                    \frac{(\pi/3)^3}{24}\,e
                }\leq\myvert{
                    0.0130068
                }
                , \gamma\in\myrange*{\pi/6,\pi/2}
                ; &\\[3ex]&
                g'(x) = \cos(x)\,e^{\sin{x}}
                &\\&
                g"(x)
                =-\sin(x)\,e^{\sin{x}}
                + \cos^2 (x)\,e^{\sin{x}}
                = e^{\sin{x}}
                \left(
                    \cos^2(x)
                    -\sin^2(x)
                \right)
            &
        \end{flalign*}
    \end{questionBox}

    \begin{questionBox}2{ % Q1.2
        Repita as regras mas para o ponto médio de Simpson
    } % Q1.2
    \end{questionBox}
\end{questionBox}

\begin{exampleBox}1{ % E1
    \begin{BM}
        I=\int_1^2{\ln{x}\,\odif{x}}
    \end{BM}
} % E1
    \answer{}
    \begin{flalign*}
        &
            \hat{I}
            = \frac{h}{3}\left(
                g(a)+4\,g((a+b)/2)+g(b)
            \right)
            = \frac{\frac{b-a}{2}}{3}\left(
                g(a)+4\,g((a+b)/2)+g(b)
            \right)
            = &\\&
            = \frac{\frac{2-1}{2}}{3}\left(
                g(1)+4\,g((1+2)/2)+g(2)
            \right)
            = \frac{1}{6}\left(
                0+4\,(0.405465)+(0.693147)
            \right)
            \approx &\\&
            \approx
            0.385835
            ; &\\[3ex]&
            \myvert{E}
            = \myvert{
                -\frac{h^2}{90}
                \,\odv[order=4]{g(\gamma)}{x}
            }
            = \myvert{
                -\frac{h^2}{90}
                \,\odv[order=4]{\ln(\gamma)}{x}
            }
            = \myvert{
                -\frac{h^2}{90}
                \,\odv[order=3]{1/\gamma}{x}
            }
            = \myvert{
                -\frac{h^2}{90}
                \,\odv[order=2]{-1/\gamma^2}{x}
            }
            = &\\&
            = \myvert{
                -\frac{h^2}{90}
                \,\odv[order=1]{2/\gamma^3}{x}
            }
            = \myvert{
                -\frac{h^2}{90}
                \,(-6/\gamma^4)
            }
            \leq
            = \myvert{
                -\frac{1/4}{90}
                \,6
            }
            \leq
            0.002084
        &
    \end{flalign*}
\end{exampleBox}

\begin{questionBox}1{ % Q2
    Considere o Integral
    \begin{BM}
        I = \int_{0.7}^{1.7}{
            \pi^x\,\odif{x}
        }
    \end{BM}
} % Q2
    \begin{questionBox}2{ % Q2.1
        Determina uma aproximação \(\hat{I}\), de \textit{I} ultizilando a regra dos trapézios compostos com \(h=0.25\).
        \\ Obtenha um majorante do erro absoluto cometido no cálculo do valor aproximado \(\hat{I}\)
    } % Q2.1
        \paragraph*{Nota:} Nos cálculos intermédios utilize 6 casas decimais, devidamente arrendondadas.
        \answer{}
        \begin{flalign*}
            &
                h=\frac{b-a}{2\,n}
                \implies
                n 
                = \frac{b-a}{2\,h}
                = \frac{1}{2*0.25}
                = 2
                \implies &\\[3ex]&
                \implies
                I_{S,2}
                = \frac{h}{3}
                \,\left(
                    f_{(x_0)}
                    + 4\,(f_{(x_1)}+f_{(x_3)})
                    + 2\,f_{(x_2)}
                    + f_{(x_4)}
                \right)
                = &\\&
                = \frac{0.25}{3}
                \,\left(
                    \pi^{0.7}
                    + 4\,(\pi^{.95} + \pi^{1.45})
                    + 2\,\pi^{1.2}
                    + \pi^{1.7}
                \right)
                \implies &\\[3ex]&
                \implies
                \myvert{I-I_{S,2}}
                \leq n\,\frac{h^5}{90}\,M_4
                = 2\,\frac{0.25^5}{90}*12.021728
                \cong\num{0.000260888194444}
            &
        \end{flalign*}
    \end{questionBox}
    \begin{questionBox}2{ % Q2.2
        Repita a alínea anterior para a regra de Simpson.
    } % Q2.2
    \end{questionBox}
    \begin{questionBox}2{ % Q2.3
        Quantos subintervalos teria que considerar se pretendesse calcular um valor aproxiumade de \textit{I} com um erro inferiror a \(10^{-6}\) usando
    } % Q2.3
        
        \begin{questionBox}3{ % Q2.3.1
            A regra do ponto médio
        } % Q2.3.1
        \end{questionBox}
        \begin{questionBox}3{ % Q2.3.2
            A regra dos trapézios.
        } % Q2.3.2
        \end{questionBox}
        \begin{questionBox}3{ % Q2.3.3
            A regra de Simpson.
        } % Q2.3.3
        \end{questionBox}
    \end{questionBox}
\end{questionBox}

\end{document}
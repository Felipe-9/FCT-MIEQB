% !TEX root = ./CNA-Notebook.6.tex
\documentclass["CNA-Notebook.tex"]{subfiles}

% \tikzset{external/force remake=true} % - remake all

\begin{document}

% \graphicspath{{\subfix{./.build/figures/CNA-Notebook.6}}}
% \tikzsetexternalprefix{./.build/figures/CNA-Notebook.6/graphics/}

\mymakesubfile{6}[CNA]
{Interpolation and Extrapolation} % Subfile Title
{Interpolation and Extrapolation} % Part Title

\setcounter{section}{3}
\begin{sectionBox}1m{Cubic spline interpolation} % Sindex
  A function composed of \cemph{greenA0}{low-order piecewise polynomials} is frequently referred to as a \cemph{greenA0}{spline}. A \cemph{greenA1}{data point where two splines are joined} is called a \cemph{greenA1}{knot}. Conventionally, a data interval is divided into more than one interval, where a \cemph{greenR2}{linear}, \cemph{greenR3}{quadratic}, or \cemph{greenR4}{cubic} polynomial is employed for each interval. In this respect, spline methods present a different approach than the polynomial interpolations covered so far. \par
  The simplest spline method is a \cemph{greenR2}{\textit{linear spline}}, where two data points are joined by a \cemph{greenR2}{line}\par
  The use of \cemph{greenR4}{cubic} polynomials is the most common choice in literature, The reason for this is that \cemph{greenR4}{cubic} splines can be joined in different ways to produce an overall interpolating curve. At this stage, we shall consider non-uniformly spaced data and will employ a \cemph{greenR4}{cubic} polynomial \cemph{greenR4}{\(y_k(x)\)}, which has a different set of coefficients for each interval, \(\myrange{x_k,x_{k+1}}\). Each \cemph{greenR4}{cubic} polynomial is then joined to its neighboring cubic polynomials at the \cemph{greenA1}{knots} by matching the slopes and curvatures \(y',y''\). cubic polynomials for each interval is written as
  \begin{BM}
    y(x) = \begin{bmatrix}
      \sum_{i=0}^{3}{c_{0,i}(x-x_i)^i}
      \\ 
      \sum_{i=0}^{3}{c_{1,i}(x-x_i)^i}
      \\ 
      \vdots
      \\
      \sum_{i=0}^{3}{c_{n,i}(x-x_i)^i}
    \end{bmatrix}
  \end{BM}
  The following conditions will then be imposed to determine the unknows
  \begin{BM}[align*]
    \text{Continuity of }& y(x) &
    y_{i-1}(x_i) 
    &= y_i(x_i) 
    = f_i
    & i\in\myrange{1,n-1}
    \\ 
    \text{Continuity of }& y'(x) &
    y'_{i-1}(x_i) 
    &= y'_i(x_i)
    & i\in\myrange{1,n-1}
    \\
    \text{Continuity of }& y''(x) &
    y''_{i-1}(x_i) 
    &= y''_i(x_i) 
    & i\in\myrange{1,n-1}
    \\
    \text{End conditions}
  \end{BM}
  From those conditions we can conclude
  \begin{BM}
    \begin{cases}
      y_{i}(x_{i+1})
      = c_{i,3}\adif{x}_i^3 + c_{i,2}\,\adif{x}_i^2 + c_{i,1}\,\adif{x}_i + c_0 
      = y_{i+1}
      = c_{i+1,0}
      \\
      y'_{i}(x_{i+1})
      = 3\,c_{i,3}\adif{x}^2 + 2\,c_{i,2}\adif{x} + c_{i,1} 
      = y'_{i+1}(x_{i+1})
      = c_{i+1,1}
      \\
      y''_{i}(x_{i+1}) 
      = 6\,c_{i,3}\adif{x} + 2\,c_{i,2} 
      = y''_{i+1}(x_{i+1}) 
      = 2\,c_{i+1,2}
    \end{cases}
  \end{BM}

  Merging the conditions and defining a few constructs we can organize all equations as follows
  \begin{BM}[gather*][\normalsize]
    \begin{Bmatrix}
      \begin{pmatrix}
        \adif{x_0}\,S_0 
        + \adif{x_{1}}\,S_{1+1}
        \\
        + 2\,(\adif{x_1}+\adif{x_0})\,S_{1} 
      \end{pmatrix}
      = 6\,\left(
        f\myrange{x_{1},x_{2}}
        - f\myrange{x_{0},x_{1}}
      \right)
      \\
      \vdots
      \\
      \begin{pmatrix}
        \adif{x_{k-1}}\,S_{k-1} 
        + \adif{x_{k}}\,S_{k+1}
        \\
        + 2\,(\adif{x_k}+\adif{x_{k-1}})\,S_{k} 
      \end{pmatrix}
      = 6\,\left(
        f\myrange{x_{k},x_{k+1}}
        - f\myrange{x_{k-1},x_{k}}
      \right)
      \\
      \vdots
      \\
      \begin{pmatrix}
        \adif{x_{n-1}}\,S_{n-1} 
        + \adif{x_{n}}\,S_{n+1}
        \\
        + 2\,(\adif{x_n}+\adif{x_{n-1}})\,S_{n} 
      \end{pmatrix}
      = 6\,\left(
        f\myrange{x_{n},x_{n+1}}
        - f\myrange{x_{n-1},x_{n}}
      \right)
    \end{Bmatrix}
  \end{BM}
  \begin{BM}[align*]
    S_{k+a} &= y''_k(x_{k+a})
    &
    \adif{x_k}&=x_{k+1}-x_{k}
    &
    f\myrange{x_{k},x_{k+1}} 
    &= \frac{f_{k+1}-f_{k}}{x_{k+1}-x_{k}}
  \end{BM}
  Up to the third condition we can achieve \(2(n-2)\) equations but we still have \(3\,n-5\) variables, with two more equations we can compleete a system of linear equations, those are the \emph{end conditions} and there are several spline formulations that are developed through the curvature estimations, we will consider only two cases: \cemph{greenA0}{natural splines} and \cemph{greenA2}{linear extrapolation end conditions}

  \begin{sectionBox}2b{Natural splines} % Sindex
    \begin{BM}
      S_0 = 0
      \quad S_n = 0
    \end{BM}
    Applying this two equations to the previous system gives
    \begin{BM}[gather*][\normalsize]
      \begin{bmatrix}
        b_1 & c_1
        \\
        a_2 & b_2 & c_2
        \\
        & a_3 & b_3 & c_3
        \\
        && \ddots & \ddots & \ddots
        \\
        &&& a_{n-2} & b_{n-2} & c_{n-2}
        \\
        &&&& a_{n-1} & b_{n-1} & c_{n-1}
      \end{bmatrix}
      \begin{bmatrix} S_1 \\ S_2 \\ S_3 \\ \vdots \\ S_{n-2} \\ S_{n-1} \end{bmatrix}
      =
      \begin{bmatrix} d_1 \\ d_2 \\ d_3 \\ \vdots \\ d_{n-2} \\ d_{n-1} \end{bmatrix}
    \end{BM}
    Where
    \begin{BM}
      a_k = \adif{x}_{k-1}
      \quad
      b_k = 2(\adif{x}_{k-1}+\adif{x}_{k})
      \quad
      c_k = \adif{x}_{k}
      \\
      d_k = 6\,(f\myrange{x_k,x_{k+1}} - f\myrange{x_{k-1},x_{k}})
    \end{BM}
  \end{sectionBox}
  \begin{sectionBox}2b{Linear extrpolation end condition} % Sindex
    This is one of the most frenquenty used end-conditions. the end point values are estimated by linear extrapolation.\par
    linear extrapolation for the knots \(1,n-1\) yelds
    \begin{BM}[align*][\normalsize]
      \frac{S_1-S_0}{\adif{x}_0} 
      &= \frac{S_2-S_1}{\adif{x_1}}
      &\hspace{-1em}\iff
      S_0
      &= S_1\,\left(1+\frac{\adif{x_1}}{\adif{x_2}}\right)
      - S_2\,\frac{\adif{x_1}}{\adif{x_2}}
      \\
      \frac{S_{n}-S_{n-1}}{\adif{x}_{n-1}} 
      &= \frac{S_{n-1}-S_{n-2}}{\adif{x_{n-2}}}
      &\hspace{-1em}\iff
      S_n
      &= S_{n-1}\,\left(1+\frac{\adif{x_{n-1}}}{\adif{x_{n-2}}}\right)
      - S_{n-2}\,\frac{\adif{x_{n-1}}}{\adif{x_{n-2}}}
    \end{BM}
  \end{sectionBox}
\end{sectionBox}

\end{document}

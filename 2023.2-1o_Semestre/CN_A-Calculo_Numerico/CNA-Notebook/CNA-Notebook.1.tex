% !TEX root = ./CNA-Notebook.1.tex
\documentclass["CNA-Notebook.tex"]{subfiles}

% \tikzset{external/force remake=true} % - remake all

\begin{document}

% \graphicspath{{\subfix{./.build/figures/CNA-Notebook.1}}}
% \tikzsetexternalprefix{./.build/figures/CNA-Notebook.1/graphics/}

\mymakesubfile{1}[CN A]
{Error Theory} % Subfile Title
{Error Theory} % Part Title


\begin{sectionBox}1m{Definitions} % Sindex

  \subsection*{Defining errors}
  Let \(Q\) and \(\tilde{Q}\) be the true and approximate values of a physical quantity.
  \begin{BM}[align*]
    \error_Q 
    = \adif{Q} 
    &= Q-\tilde{Q} 
    \tag{Error}\label{eq:s.1 error}
    \\
    \abs{\error_{Q}} 
    = \abs{\adif{Q}} 
    &= \abs{Q-\tilde{Q}} 
    \tag{Abolute Error}
    \\
    \relerror_{Q}
    = \abs*{\frac{\adif{Q}}{Q}} 
    &= \abs*{1-\frac{\tilde{Q}}{Q}} 
    \tag{Relative Error}\label{eq:s.1 relative error}
  \end{BM}
  \begin{BM}
    \error_Q \begin{cases}
      < 0 & \text{Approximating by \cemph[green]{excess}}
      \\
      > 0 & \text{Approximating by \cemph[red]{defect}}
    \end{cases}
  \end{BM}

  \begin{sectionBox}*2b{Categorizing errors} % Sindex
    There are two kinds of errors:
    \paragraph{Random/Precision errors} are caused by unpredictable changes in the environment, can be easily detected and analyzed by statistical methods.
    \paragraph{Systematic errors} are the result of how an experiment is conducted and can be identified by leading to results that are too high or too low.
  \end{sectionBox}

  % \paragraph{Erros iniciais do problema:}
  % erros iniciais independentes do processo de cálculo
  % \begin{itemize}
  %   \item Relativos ao modelo matemático escolhido
  %   \item prevenientes dos dados iniciais
  % \end{itemize}
  %
  % \paragraph{Erros que ocorrem durante a aplicação de métodos numéricos:}
  % erros que ocorrem durante o processo de cálculo
  % \begin{itemize}
  %   \item Erros de arredondamento
  %   \item Erros de truncatura
  % \end{itemize}

\end{sectionBox}

\begin{sectionBox}2m{Intervalo que contem \(x\)} % S
  \begin{BM}
    I_x=\myrange{\hat{x}-\eta_x,\hat{x}+\eta_x}
    : \myvert{x-\hat{x}}\leq\eta_x
    \\[2ex]
    \eta_x\begin{cases}
      0.5*10^{-i}
      \quad&
      :\hat{x}*10^i\in\mathbb{Z}
      \\
      \eta_x
      \quad&
      :\hat{x}\pm\eta_x
      \\
      y*10^{-i}
      \quad&
      :\hat{x}\,(y)\land\hat{x}*10^i\in\mathbb{Z}
    \end{cases}
  \end{BM}

  \begin{multicols}{2}
    \begin{exampleBox}*2b{ 12.3 } % E1
      \answer{}
      \begin{gather*}
        \myvert{x-\hat{x}}
        \leq0.5*10^{-1}
        =\eta_x
        \\
        \therefore
        I_x
        = \myrange{
          12.3-0.05,
          12.3+0.05
        }
        = \\
        = \myrange{
          12.25,12.35
        }
      \end{gather*}
    \end{exampleBox}
    \begin{exampleBox}*2b{ \(87.9\pm0.07\) } % E2
      \answer{}
      \begin{gather*}
        \myvert{x-\hat{x}}
        \leq0.07
        = \eta_x
        \\
        \therefore
        I_x
        =\myvert{
          87.9-0.07,
          87.9+0.07
        }
        = \\
        =\myvert{87.83,87.97}
      \end{gather*}
    \end{exampleBox}
    \begin{exampleBox}*2b{ \(400.32\,(6)\) } % E3
      \answer{}
      \begin{gather*}
        \myvert{x-\hat{x}}
        \leq6*10^{-2}
        = \eta_x
        \\
        \therefore
        I_x
        =\myrange{400.26,400.38}
      \end{gather*}
    \end{exampleBox}
  \end{multicols}

  \begin{exampleBox}*2b{\( 
      x=1/17 \qquad \hat{x}=0.059
  \) } % E4
    \answer{}

    \sisetup{
      round-mode={places},% figures/places/unsertanty/none
      round-precision={3},
    }
    \subsubquestion{Erro Absoluto}
    \begin{gather*}
      \myvert{\varepsilon_x}
      = \myvert{x-\hat{x}}
      = \myvert{1/17-0.059}
      = 0.17647\dots\E-3
      \leq 0.177\E-3
    \end{gather*}
    \subsubquestion{Erro relativo}
    \begin{gather*}
      r_x
      = \frac{\myvert{x-\hat{x}}}{\myvert{x}}
      = \frac{\myvert{1/17-0.059}}{\myvert{1/17}}
      = 0.300\E-2
    \end{gather*}
  \end{exampleBox}

\end{sectionBox}

% \begin{sectionBox}1m{Significancia} % S
%
%   Disemos que \(\hat{x}\) se aproxima de \textit{x} com, no mínimo \textit{k} \emph{casas decimais significativas} \(\iff\)
%   \begin{BM}
%     k\in\mathbb{N}_0
%     : \myvert{\varepsilon_x}\leq0.5*10^{-k}
%   \end{BM}
%   Disemos que \(\hat{x}\) se aproxima de \textit{x} com, com \textit{n} \emph{algarismos significativos} \(\iff\)
%   \begin{BM}
%     n\in\mathbb{N}_0
%     : \begin{cases}
%       \myvert{\varepsilon_x}
%       \leq0.5*10^{m+1-n}
%       \\
%       m\in\mathbb{Z}: 10^m\leq\myvert{x}<10^{m+1}
%     \end{cases}
%   \end{BM}
%
% \end{sectionBox}
%
% \begin{exampleBox}1m{} % E
%
%   Na determinação de \textit{x} obteve-se o resultado \num{0.001773(8)}.\\
%
%   \answer{}
%   \subsubexample{Casas decimais significativas}
%   \begin{gather*}
%     k
%     :\myvert{x-\hat{x}}
%     =\myvert{\varepsilon_x}
%     \leq 0.8*10^{-5}
%     = 0.5*10^{-4}
%     \implies k=4
%   \end{gather*}
%   \subsubexample{Algarismos significativos}
%   \begin{gather*}
%     n:10^{-3}
%     \leq\myvert{x}
%     \approx\myvert{\hat{x}}
%     <10^{-2}
%     \implies \\
%     \implies
%     m+1-n
%     = -2-n
%     % = \\
%     = -k
%     = -4
%     \implies \\
%     \implies
%     n=2
%   \end{gather*}
% \end{exampleBox}
%
% \part*{Condicionamento de um problema}
%
% \begin{sectionBox}1m{Propagação de erro absoluto} % S
%
%   \paragraph{Fórmula fundamental do cálculo dos erros}
%   \begin{BM}
%     \myvert{\varepsilon_{g_{(x)}}}
%     \leq M_1\,\myvert{\varepsilon_x}
%     : \myvert{\odv{g_{(z)}}{x}}\leq M_1,z\in V_{\delta\,(x)}
%   \end{BM}
%
%   \begin{sectionBox}*2b{Desenvolvimento} % S
%     Seja \textit{g} uma função diferenciável numa vizinhança \(V_{\delta\,(x)}\) de \textit{x}.\\
%     Utilizando a fórmula de Taylor tem-se:
%     \begin{gather*}
%       g_{(x)} 
%       = g_{(\hat{x})}
%       + \odv{g_{(\xi)}}{x}(x-\hat{x})
%       : \quad \xi\in\myrange{x,\hat{x}}
%       \implies \\
%       \implies
%       \varepsilon_{g_{(x)}}
%       = g_{(x)} - g_{(\hat{x})}
%       = \odv{g_{(\xi)}}{x}(x-\hat{x})
%       = M_1\,\varepsilon_x
%       \implies \\
%       \implies
%       \myvert*{\varepsilon_{g_{(x)}}}
%       \leq M_1\,\myvert{\varepsilon_x}
%     \end{gather*}
%   \end{sectionBox}
%
% \end{sectionBox}
%
% \begin{sectionBox}1m{Propagação do erro relativo} % S
%
%   \begin{BM}
%     r_{g(x)}
%     \approx
%     C_{g\,(x)}\,r_x
%   \end{BM}
%
%   \begin{sectionBox}*3b{Desenvolvimento} % S
%
%     Considerando novamente \textit{g}, mas de classe \(C^2(V_{\delta\,(x)})\), aplicando a formula de Taylor:
%     \begin{gather*}
%       g_{(\hat{x})}
%       = g_{(x)}
%       + \odv{g_{(x)}}{x}(x-\hat{x})
%       + \odv[order=2]{g_{(x)}}{x}\frac{(x-\hat{x})^2}{2}
%       \underset{x\approx\hat{x}}{\approx}
%       g_{(x)}
%       + \odv{g_{(x)}}{x}(x-\hat{x})
%       + \odv[order=2]{g_{(x)}}{x}\frac{0}{2}
%       = \\
%       = g_{(x)}
%       + \odv{g_{(x)}}{x}(x-\hat{x})
%       % \implies \\
%       \implies
%       \myvert{g_{(x)}-g_{(\hat{x})}}
%       \approx
%       \myvert{\odv{g_{(x)}}{x}}
%       \myvert{x-\hat{x}}
%       \implies \\
%       \implies
%       \frac{\myvert{g_{(x)}-g_{(\hat{x})}}}{\myvert{g_{(x)}}}
%       = r_{g(x)}
%       \approx
%       \myvert{
%         \frac{x}{g_{(x)}}
%         \odv{g_{(x)}}{x}
%       }
%       \frac{\myvert{x-\hat{x}}}{\myvert{x}}
%       = \myvert{
%         \frac{x}{g_{(x)}}
%         \odv{g_{(x)}}{x}
%       }
%       r_{x}
%       = C_{g\,(x)}
%       r_{x}
%     \end{gather*}
%   \end{sectionBox}
%
%   \paragraph{Numero de condição de uma função \textit{g} num ponto \textit{x}}
%   \begin{BM}
%     C_{g\,(x)} 
%     = \myvert{
%       \frac{x}{g_{(x)}}
%       \,\odv{g_{(x)}}{x}
%     }
%     ,\quad g_{(x)}\neq0
%   \end{BM}
%
% \end{sectionBox}
%
% \begin{exampleBox}1m{} % E
%
%   \begin{BM}
%     g_{1\,(x)} = x^2\quad (x>0).
%   \end{BM}
%   % \vspace{-10ex}
%
%   \answer{}
%
%   \begin{gather*}
%     C_{g_1\,(x)}
%     = \myvert{
%       \frac{x}{g_{1\,(x)}}
%       \,\odv{g_{1\,(x)}}{x}
%     }
%     = \myvert{
%       \frac{x}{x^x}
%       \,(x\,x^{x-1} + x^x\,\ln(x))
%     }
%     = \myvert{x\,(1+\ln(x))}
%   \end{gather*}
%
% \end{exampleBox}
%
% \begin{sectionBox}1m{Condicionamento de uma função/problema} % S
%
%   Uma função diz-se:
%   \begin{itemize}
%     \item \textcolor{\Emph{green}}{Bem condicionado:} Pequenas variações nos dados iniciais/parametros implicam em pequenas variações nos resultados
%     \item \textcolor{\Emph{red}}{Mal condicionado:} Situação inversa
%   \end{itemize}
%   \paragraph{Nota:} \emph{Independe} do método numérico utilizado
%
% \end{sectionBox}
%
% \begin{sectionBox}1m{Estabilidade de um método numérico} % S
%
%   Um método numérico diz-se
%   \begin{itemize}
%     \item \textcolor{\Emph{green}}{Estável:} Acumulação dos erros não afeta significativamente o resultado final;
%     \item \textcolor{\Emph{red}}{Instável: } Caso contrário
%   \end{itemize}
%
% \end{sectionBox}
%
% \begin{exampleBox}1m{} % E
%
%   Considere as duas funções matematicamente idênticas:
%   \begin{BM}
%     f_{(\theta)}
%     = \frac{\tan^2\theta}{\theta^2}
%     \qquad
%     g_{(\theta)}
%     = \frac{1-\cos(2\,\theta)}{\theta^2\,(1+\cos(2\,\theta))}
%     \\
%     \lim_{\theta\to0}{f_{(\theta)}}
%     = \lim_{\theta\to0}{g_{(\theta)}}
%     = 1
%   \end{BM}
%   
%   \begin{center}
%     \vspace{1ex}
%     \sisetup{round-precision={9}}
%     \begin{tabular}{*{3}{C}}
%       \toprule
%
%       \theta
%       & f_{(\theta)}
%       & g_{(\theta)}
%
%       \\\midrule
%
%       10^{-1} & \num{1.0067046422494887} & \num{1.006704642249489}
%       \\ 10^{-2} & \num{1.0000666704446415} & \num{1.0000666704447514}
%       \\ 10^{-3} & \num{1.0000006666670447} & \num{1.0000006666846255}
%       \\ 10^{-4} & \num{1.0000000066666668} & \num{1.0000000039225287}
%       \\ 10^{-5} & \num{1.0000000000666665} & \num{1.0000000828403706}
%       \\ 10^{-6} & \num{1.0000000000006666} & \num{0.9999778782808777}
%       \\ 10^{-7} & \num{1.0000000000000067} & \num{0.9992007221626502}
%       \\ 10^{-8} & 1 & \num{1.1102230246251554}
%       \\ 10^{-9} & 1 & 0
%       \\ 10^{-10} & 1 & 0
%
%       \\\bottomrule
%     \end{tabular}
%     \vspace{2ex}
%   \end{center}
%
%   \begin{BM}
%     \therefore\begin{cases}
%       f_{(\theta)}:&\text{\textcolor{\Emph{green}}{estável}}
%       \\
%       g_{(\theta)}:&\text{\textcolor{\Emph{red}}{instável}}
%     \end{cases}
%   \end{BM}
%
% \end{exampleBox}

\end{document}

% !TEX root = ./CN_A-Exercises_Resolutions.3.tex
\providecommand\mainfilename{"./CN_A-Exercises_Resolutions.tex"}
\providecommand \subfilename{}
\renewcommand   \subfilename{"./CN_A-Exercises_Resolutions.3.tex"}
\documentclass[\mainfilename]{subfiles}

% \tikzset{external/force remake=true} % - remake all

\begin{document}

% \graphicspath{{\subfix{./.build/figures/CN_A-Exercises_Resolutions.3}}}
% \tikzsetexternalprefix{./.build/figures/CN_A-Exercises_Resolutions.3/graphics/}

\mymakesubfile{3}
[CN A]
{Integração} % Subfile Title
{Integração} % Part Title

% \begin{questionBox}1{ % Q1
%     \begin{BM}
%         I = \int_{\pi/6}^{\pi/2}{
%             e^{\sin{x}}
%             \,\odif{x}
%         }
%     \end{BM}
% } % Q1
%     \answer{}
%     \begin{questionBox}2{ % Q1.1
%     } % Q1.1
%         \answer{}
%         \begin{flalign*}
%             &
%                 g(x)
%                 = e^{\sin(x)}
%                 ; \hat{I}
%                 = h\,g((a+b)/2)
%                 = \frac{I}{3}
%                 \,g(\pi/3)=\pi/3\,e^{\sin(\pi/3)}
%                 \approx
%                 2.489652
%                 ; &\\&
%                 \myvert{\varepsilon}
%                 = \myvert{
%                     \frac{h^3}{24}
%                     \,g"(\gamma)
%                 }
%                 \leq\myvert{
%                     \frac{(\pi/3)^3}{24}\,e
%                 }\leq\myvert{
%                     0.0130068
%                 }
%                 , \gamma\in\myrange*{\pi/6,\pi/2}
%                 ; &\\[3ex]&
%                 g'(x) = \cos(x)\,e^{\sin{x}}
%                 &\\&
%                 g"(x)
%                 =-\sin(x)\,e^{\sin{x}}
%                 + \cos^2 (x)\,e^{\sin{x}}
%                 = e^{\sin{x}}
%                 \left(
%                     \cos^2(x)
%                     -\sin^2(x)
%                 \right)
%             &
%         \end{flalign*}
%     \end{questionBox}
%
%     \begin{questionBox}2{ % Q1.2
%         Repita as regras mas para o ponto médio de Simpson
%     } % Q1.2
%     \end{questionBox}
% \end{questionBox}
%
% \begin{exampleBox}1{ % E1
%     \begin{BM}
%         I=\int_1^2{\ln{x}\,\odif{x}}
%     \end{BM}
% } % E1
%     \answer{}
%     \begin{flalign*}
%         &
%             \hat{I}
%             = \frac{h}{3}\left(
%                 g(a)+4\,g((a+b)/2)+g(b)
%             \right)
%             = \frac{\frac{b-a}{2}}{3}\left(
%                 g(a)+4\,g((a+b)/2)+g(b)
%             \right)
%             = &\\&
%             = \frac{\frac{2-1}{2}}{3}\left(
%                 g(1)+4\,g((1+2)/2)+g(2)
%             \right)
%             = \frac{1}{6}\left(
%                 0+4\,(0.405465)+(0.693147)
%             \right)
%             \approx &\\&
%             \approx
%             0.385835
%             ; &\\[3ex]&
%             \myvert{E}
%             = \myvert{
%                 -\frac{h^2}{90}
%                 \,\odv[order=4]{g(\gamma)}{x}
%             }
%             = \myvert{
%                 -\frac{h^2}{90}
%                 \,\odv[order=4]{\ln(\gamma)}{x}
%             }
%             = \myvert{
%                 -\frac{h^2}{90}
%                 \,\odv[order=3]{1/\gamma}{x}
%             }
%             = \myvert{
%                 -\frac{h^2}{90}
%                 \,\odv[order=2]{-1/\gamma^2}{x}
%             }
%             = &\\&
%             = \myvert{
%                 -\frac{h^2}{90}
%                 \,\odv[order=1]{2/\gamma^3}{x}
%             }
%             = \myvert{
%                 -\frac{h^2}{90}
%                 \,(-6/\gamma^4)
%             }
%             \leq
%             = \myvert{
%                 -\frac{1/4}{90}
%                 \,6
%             }
%             \leq
%             0.002084
%         &
%     \end{flalign*}
% \end{exampleBox}
%
% \begin{questionBox}1{ % Q2
%     Considere o Integral
%     \begin{BM}
%         I = \int_{0.7}^{1.7}{
%             \pi^x\,\odif{x}
%         }
%     \end{BM}
% } % Q2
%     \begin{questionBox}2{ % Q2.1
%         Determina uma aproximação \(\hat{I}\), de \textit{I} ultizilando a regra dos trapézios compostos com \(h=0.25\).
%         \\ Obtenha um majorante do erro absoluto cometido no cálculo do valor aproximado \(\hat{I}\)
%     } % Q2.1
%         \paragraph*{Nota:} Nos cálculos intermédios utilize 6 casas decimais, devidamente arrendondadas.
%         \answer{}
%         \begin{flalign*}
%             &
%                 h=\frac{b-a}{2\,n}
%                 \implies
%                 n 
%                 = \frac{b-a}{2\,h}
%                 = \frac{1}{2*0.25}
%                 = 2
%                 \implies &\\[3ex]&
%                 \implies
%                 I_{S,2}
%                 = \frac{h}{3}
%                 \,\left(
%                     f_{(x_0)}
%                     + 4\,(f_{(x_1)}+f_{(x_3)})
%                     + 2\,f_{(x_2)}
%                     + f_{(x_4)}
%                 \right)
%                 = &\\&
%                 = \frac{0.25}{3}
%                 \,\left(
%                     \pi^{0.7}
%                     + 4\,(\pi^{.95} + \pi^{1.45})
%                     + 2\,\pi^{1.2}
%                     + \pi^{1.7}
%                 \right)
%                 \implies &\\[3ex]&
%                 \implies
%                 \myvert{I-I_{S,2}}
%                 \leq n\,\frac{h^5}{90}\,M_4
%                 = 2\,\frac{0.25^5}{90}*12.021728
%                 \cong\num{0.000260888194444}
%             &
%         \end{flalign*}
%     \end{questionBox}
%     \begin{questionBox}2{ % Q2.2
%         Repita a alínea anterior para a regra de Simpson.
%     } % Q2.2
%     \end{questionBox}
%     \begin{questionBox}2{ % Q2.3
%         Quantos subintervalos teria que considerar se pretendesse calcular um valor aproxiumade de \textit{I} com um erro inferiror a \(10^{-6}\) usando
%     } % Q2.3
%         
%         \begin{questionBox}3{ % Q2.3.1
%             A regra do ponto médio
%         } % Q2.3.1
%         \end{questionBox}
%         \begin{questionBox}3{ % Q2.3.2
%             A regra dos trapézios.
%         } % Q2.3.2
%         \end{questionBox}
%         \begin{questionBox}3{ % Q2.3.3
%             A regra de Simpson.
%         } % Q2.3.3
%         \end{questionBox}
%     \end{questionBox}
% \end{questionBox}

\setcounter{question}{7}

\begin{questionBox}1{} % Q8
    \begin{BM}
      I = \int_{0}^{4}{f_{(x)}\,\odif{x}}
      ,\quad f_{(x)}\in C^n(\myrange{0,4})
      \\
      \myvert{f^{n}_{(x)}}\leq \frac{2^n}{n!}
      \quad\forall\,x\in\myrange{0,4}
      \land n\in\mathbb{N}
    \end{BM}
    \answer{}
    \sisetup{
      % scientific / engineering / input / fixed
      exponent-mode           = engineering,
      exponent-to-prefix      = false,          % 1000 g -> 1 kg
      round-mode              = places,        % figures/places/unsertanty/none
      round-precision         = 4,
    }
    \begin{flalign*}
        &
        \myvert{I-\hat{I}_S}
        \leq
        \myvert{
          -n\,\frac{h^5}{90}
          \,f^4_{(\theta)}
        }
        \leq 
        \myvert{
          -n\,\frac{\left(
              \frac{b-a}{2\,n}
          \right)^5}{90}
          \,\frac{2^4}{4!}
        }
        = &\\&
        = n\,\frac{\left(
            \frac{4-0}{2\,n}
        \right)^5}{90}
        \,\frac{2^4}{4!}
        = \frac{4^4}{2*n^4*3!*90}
        \leq 0.5\E-4
        \implies &\\&
        \implies
        n = \ceil{\num{8.29777303051304}}
        = 9
        &\\&
        \therefore 18
        % &\\&
        \text{ Numero de aplicações da regra de Simpson}
        &
    \end{flalign*}
\end{questionBox}

\begin{questionBox}{Questão de algum teste} % Q1
  Seja \(I=\int_{-1}^{1}{f(x)\,\odif{x}},\hat{I}_{P\,M,2}=5.85\) sua aproximação de I dada pela regra do ponto médio com \(n=2\) e \(\hat{I}_{T,2}=6.45\) a aproximação do \textit{I} pela regra de trapézios com \(n=3\). Qual o valor da aproximação por \textit{I} dadaa pela regrad e simpson com \(n=2\)
  \answer{}
  \begin{flalign*}
    &
    \hat{I}_{S}
    = \frac{h}{3}\left(
      f(x_0)
      + 4\,(
        f(x_1)
        + f(x_3)
      )
      + 2\,f(x_2)
      + f(x_4)
    \right)
    = &\\&
    = \frac{0.5}{3}\left(
      f(-1)
      + 4\,(
        f(-0.5)
        + f(0.5)
      )
      + 2\,f(0)
      + f(1)
    \right)
    = &\\&
    = \frac{0.5}{3}\left(
      4\,( 5.85)
      + ( 12.90)
    \right)
    \cong 6.05
    % 
    % 
    % 
    ; &\\[3ex]&
    x_i 
    = -1+h*i
    = \{
      -1.0,-0.5,0.0,0.5,1.0
    \}
    % 
    % 
    % 
    ; &\\[3ex]&
    h 
    = \frac{b-a}{2\,n}
    = \frac{1-(-1)}{2*2}
    = 0.5
    % 
    % 
    % 
    ; &\\[3ex]&
    \hat{I}_{PM,2}
    =h\left(
      f\left(\frac{-1+0}{2}\right)
      +f\left(\frac{0+1}{2}\right)
    \right)
    =
    f\left(\frac{-1}{2}\right)
    +f\left(\frac{1}{2}\right)
    = 5.85
    % 
    % 
    % 
    ;&\\[3ex]&
    \hat{I}_{T,2}
    = \frac{h}{2}\left(
      f(-1)+2\,f(0)+f(1)
    \right)
    = \frac{1}{2}\left(
      f(-1)+2\,f(0)+f(1)
    \right)
    =6.45
    \implies &\\&
    \implies
    f(-1)+2\,f(0)+f(1)
    =12.90
    &
  \end{flalign*}
\end{questionBox}

% \setcounter{question}{10}
%
% \begin{questionBox}1{ % Q11
%     Seja
%     \begin{BM}
%         I=\int_1^5{f_{(X)}\,\odif{x}}
%     \end{BM}
%     Considere a seguinte tabela da função \textit{f}, função polinomial de grau 2, da qual se sabe que \(f"_{(x)}=4\):
% } % Q11
%     \begin{center}
%         \vspace{1ex}
%         \begin{tabular}{C || *{5}{C}}
%             
%                 x 
%                 & 1 & 2 & 3 & 4 & 5
%                 
%             \\\hline
%                 f_{(X)}
%                 & -2 & -1 & 1 & \alpha & 9
%
%         \end{tabular}
%         \vspace{2ex}
%     \end{center}
%     \begin{questionBox}2{ % Q11.1
%         Recorrendo à regra dos trapézios, com duas aplicações, determine um valor aproximado de \textit{I} e o valor exato de \textit{I}.
%     } % Q11.1
%     \end{questionBox}
%     \begin{questionBox}2{ % Q11.2
%         Recorrendo à regra do ponto médio, com \(n = 2\), determine um valor aproximado de \textit{I} e o valor de \textit{I} comofunção de \(\alpha\).
%     } % Q11.2
%     \end{questionBox}
%     \begin{questionBox}2{ % Q11.3
%         Recorrendo às alíneas anteriores, determine o valor de \(\alpha\).
%     } % Q11.3
%     \end{questionBox}
%     \begin{questionBox}2{ % Q11.4
%         Utilize duas aplicações da regra de Simpson para determinar um valor aproximado de \textit{I}.
%     } % Q11.4
%     \end{questionBox}
% \end{questionBox}
%
\setcounter{question}{12}

\begin{questionBox}1{} % Q13
  Cosidere a seguinte tabela para a função \textit{f}:
  \begin{center}
    \vspace{1ex}
    \begin{tabular}{C || *{7}{C}}

      x
      & -3 & -2 & -1 & 0 & 1 &  2 &  3

      \\\hline

      f_{(x)}
      & 40 & 21 & 8 & 1 & 0 & 5 & 16

    \end{tabular}
    \vspace{2ex}
  \end{center}
  \answer{}
  \begin{flalign*}
    &
    h 
    = \frac{b-a}{n}
    = \frac{3-(-3)}{n}
    < 3
    % \implies &\\&
    \implies
    n > 2 \land n<4 
    \implies
    n=3
    \land 
    h = 2
    &
  \end{flalign*}
\end{questionBox}
\begin{questionBox}2{} % Q13.1
  Ultilizando a regra dos trapézios composta, obtenha uma proximação de \(\hat{I}_T\) de
  \begin{BM}
    I = \int_{-3}^{3}{f_{(x)}\,\odif{x}}
    ,\quad h<3\land n<4
  \end{BM}
  \answer{}
  \begin{flalign*}
    &
    \hat{I}_T
    = \frac{h}{2}
    \left(
      f_{(x_0)}
      + 2\,f_{(x_2)}
      + 2\,f_{(x_4)}
      + f_{(x_6)}
    \right)
    = \frac{2}{2}
    \left(
      f_{-3}
      + 2\,f_{-1}
      + 2\,f_{1}
      + f_{3}
    \right)
    = \left( 40 + 2*8 + 2*0 + 16 \right)
    = 72
    &
  \end{flalign*}
\end{questionBox}
\begin{questionBox}2{} % Q13.2
  \answer{}
  \begin{flalign*}
    &
    \hat{I}_{pm}
    = 2\,\left(
      f_{\left( \frac{-1+x_0}{2} \right)}
      + f_{\left( \frac{-1+x_4}{2} \right)}
      + f_{\left( \frac{1+x_6}{2} \right)}
    \right)
    = &\\&
    = 2\,\left(
      f_{(-3)}
      + f_{(0)}
      + f_{(2)}
    \right)
    = 2\,( 21 + 1 + 5 )
    % = &\\&
    = 54
    &
  \end{flalign*}
\end{questionBox}
\begin{questionBox}2{} % Q13.3
  \answer{}
  \begin{flalign*}
    &
    I - \hat{I}_{pm}
    = n\,\frac{h^3}{24}
    \,f"_{(\theta)}
    = 3\,\frac{2^3}{24}
    \,k
    = 6
    % \implies &\\&
    \implies
    k = 6
    % 
    % 
    % 
    ; &\\[3ex]&
    I-\hat{I}_{T}
    = -n\,\frac{h^3}{12}
    \,f"_{(\theta)}
    = -3\,\frac{2^3}{12}
    \,k
    = -3\,\frac{2^3}{12}
    \,6
    = - 12
    &
  \end{flalign*}
\end{questionBox}

\end{document}

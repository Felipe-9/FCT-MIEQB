% !TEX root = ./ERQ_I-Testes_Resolucoes.2023.1.tex
\providecommand\mainfilename{"./ERQ_I-Testes_Resolucoes.tex"}
\providecommand \subfilename{}
\renewcommand   \subfilename{"./ERQ_I-Testes_Resolucoes.2023.1.tex"}
\documentclass[\mainfilename]{subfiles}

% \tikzset{external/force remake=true} % - remake all

\begin{document}

% \graphicspath{{\subfix{./.build/figures/ERQ_I-Testes_Resolucoes.2023.1}}}
% \tikzsetexternalprefix{./.build/figures/ERQ_I-Testes_Resolucoes.2023.1/graphics/}

\mymakesubfile{1}
[ERQ I]
{Teste 1 2023 Resolução} % Subfile Title
{Teste 1 2023 Resolução} % Part Title

\begin{questionBox}1{ % Q1
    \begin{itemize}
        \item Fase liq
        \item \ch{A + B -> C}
        \item 3 reatores batch
        \item \(V=5\,\si{\metre^3}\)
        \item \(C_{A\,0}=C_{B\,0}=1\,\si{\M}\)
        \item \(t_{d}=2\,\si{\hour}\)
        \item \(M_A = 60\,\si{\gram/\mole}\)
        \item \(M_B = 130\,\si{\gram/\mole}\)
        \item Caso n res a): \(k=2.8\,\si{\deci\metre^3.\mole^{-1}.\hour^{-1}}\)
    \end{itemize}
} % Q1
    \begin{questionBox}2{ % Q1.1
        Lei cin
    } % Q1.1
        \answer{}
        \begin{flalign*}
            &
                -r_A
                = k\left(
                    C_{A}
                    \,C_{B}
                \right)
                = k\left(
                    C_{A\,0}(1-X)
                    \,C_{A\,0}(1-X)
                \right)
                = &\\&
                = k\,C_{A\,0}^2(1-X)^2
            &
        \end{flalign*}
    \end{questionBox}
    \begin{questionBox}2{ % Q1.2
        eq da curva \(X=f(t)\)
    } % Q1.2
        \answer{}
        \begin{flalign*}
            &
                X=f(t):&\\& 
                -r_A\,V
                = (k\,C_{A\,0}^2(1-X)^2)\,V
                = k\,C_{A\,0}^2(1-X)^2\,V
                = &\\&
                =\odv{N_A}{t}
                =\odv{(N_{A\,0}(1-X))}{t}
                =N_{A\,0}\odv{(1-X)}{t}
                \implies &\\[6ex]&
                \implies
                \int_0^t{
                    k\,C_{A\,0}^2(1-X)^2\,V
                    \,\odif{t}
                }
                = k\,C_{A\,0}^2\,V
                \,\int_0^t{
                    \,\odif{t}
                }
                = k\,C_{A\,0}^2\,V\,t
                = &\\[3ex]&
                = \int_{1}^{1-X}{
                    N_{A\,0}
                    \frac{\odif{(1-X)}}{(1-X)^2}
                }
                = &\\&
                = N_{A\,0}
                \,\int_{1}^{1-X}{
                    \frac{\odif{(1-X)}}{(1-X)^2}
                }
                % = &\\&
                = -N_{A\,0}
                \adif{(1-X)^{-1}}
                \big\vert_{1}^{1-X}
                = &\\&
                = N_{A\,0}\left(
                    \frac{1}{1-X}
                    -1
                \right)
                = \frac{N_{A\,0}}{1/X-1}
                \implies &\\[3ex]&
                \implies
                X
                = \left(
                    1+\frac{N_{A\,0}}{k\,C_{A\,0}^2\,V\,t}
                \right)^{-1}
                % = &\\&
                = \left(
                    1+\frac{C_{A\,0}\,V}{k\,C_{A\,0}^2\,V\,t}
                \right)^{-1}
                = &\\&
                = \left(
                    1+1/k\,C_{A\,0}\,t
                \right)^{-1}
            &
        \end{flalign*}
    \end{questionBox}
    \begin{questionBox}2{ % Q1.3
        const cine
    } % Q1.3
        \answer{}
        \begin{flalign*}
            &
                k:
                X=\left(
                    1+1/k\,C_{A\,0}\,t
                \right)^{-1}
                \implies &\\&
                \implies
                k
                = \left(
                    C_{A\,0}\,t\,(X^{-1}-1)
                \right)^{-1}
                \cong \left(
                    1*1.5*(0.8^{-1}-1)
                \right)^{-1}
                \cong &\\&
                \cong
                \SI{2.666666666666667}{\M^{-1}.\hour^{-1}}
            &
        \end{flalign*}
    \end{questionBox}
    \begin{questionBox}2{ % Q1.4
        \(t_{opt}\land X_{opt}\) (usando graf)
    } % Q1.4
        \answer{}
        traçando do ponto \((0,-t_d)\) até tangenciar o gráfico temos:
        \begin{BM}
            X_{opt} \cong 0.7
            \qquad
            t_{opt} \cong 1\,\si{\hour}
        \end{BM}
    \end{questionBox}
    \begin{questionBox}2{ % Q1.5
        Prod anual de C
        \begin{itemize}
            \item 24\,\si{\hour/\day}
            \item 330\,\si{\day/\year}
        \end{itemize}
    } % Q1.5
        \answer{}
        \begin{flalign*}
            &
                m_C
                = N_{C}\,M_C\,N_{batches}
                = N_{C}*130
                * (3*24*330/t_{batch})
                = &\\&
                = N_{C}
                \,3088800
                /(t_{opt}+t_d)
                = N_{C}
                \,3088800
                /(1+2)
                = N_{C}\,1029600
                ; &\\[3ex]&
                % N_B
                N_C
                = N_{A\,0}\,X
                = C_{A\,0}\,V*3\,X
                \cong &\\&
                \cong 1*5*0.7
                \cong 3.5
                \implies &\\[3ex]&
                \implies
                m_C \cong \SI{3.603600}{\tonne}
            &
        \end{flalign*}
    \end{questionBox}
    \begin{questionBox}2{ % Q1.6
        expl proc p det analit \(X_{opt}\land t_{opt}\)
    } % Q1.6
    \end{questionBox}
\end{questionBox}

\begin{questionBox}1{ % Q2
    \begin{itemize}
        \item \ch{A -> 3 B}
        % \item fase gas
        % \item vol const
        \item \(T_0=500\,\si{\celsius}=773.15\,\si{\kelvin}\)
        \item \(k=0.03\,\si{\min^{-1}}\)
    \end{itemize}
} % Q2
    \begin{BM}
        \int{\frac{1+a\,X}{1-X}\odif{X}}
        = -a\,X
        + (1+a)
        \,\ln\frac{1}{1-X}
    \end{BM}
    \begin{questionBox}2{ % Q2.1
        det P de conv
        \begin{itemize}
            \item batch
            \item vol const
            \item fase gas
            \item \(X=0.99\)
            \item Carreg A puro
            \item \(P_{0}=2\,\si{\atm}\)
        \end{itemize}
    } % Q2.1
        \answer{}
        \begin{flalign*}
            &
                P
                =P_0
                \,\frac{V_0}{V}
                \,\frac{T}{T_{0}}
                \,(1+\varepsilon\,X)
                =2\,(1+(-1+3)\,0.99)\,\si{\atm}
                \cong
                \SI{5.96}{\atm}
            &
        \end{flalign*}
    \end{questionBox}
    \begin{questionBox}2{ % Q2.2
        nas cond de a), qual o t para \(X=99\)
    } % Q2.2
        \answer{}
        \begin{flalign*}
            &
                t
                = C_{A\,0}
                \int_0^{X}{
                    \frac{\odif{X}}{-r_A}
                }
                = C_{A\,0}
                \int_0^{X}{
                    \frac{\odif{X}}{
                        k\,C_{A}
                    }
                }
                = C_{A\,0}
                \int_0^{X}{
                    \frac{\odif{X}}{
                        k\,(F_{A}/v)
                    }
                }
                = &\\&
                = C_{A\,0}
                \int_0^{X}{
                    \frac{
                        (v_0(1+\varepsilon\,X))
                    }{
                        k\,(F_{A\,0}(1-X))
                    }\odif{X}
                }
                = &\\&
                = \frac{
                    C_{A\,0}
                }{
                    k\,(F_{A\,0}/v_0)
                }
                \int_0^{X}{
                    \frac{
                        1+\varepsilon\,X
                    }{
                        1-X
                    }\odif{X}
                }
                = &\\&
                = \frac{
                    C_{A\,0}
                }{
                    k\,(C_{A\,0})
                }
                \adif{\left(
                    -\varepsilon\,X
                    + (1+\varepsilon)
                    \,\ln\frac{1}{1-X}
                \right)}\big\vert_0^X
                = &\\&
                = k^{-1}\left(
                    -\varepsilon\,X
                    + (1+\varepsilon)
                    \,\ln\frac{1}{1-X}
                \right)
                = &\\&
                = (0.03*60)^{-1}\left(
                    -2*0.99
                    + (1+2)
                    \,\ln\frac{1}{1-0.99}
                \right)
                \cong
                \SI{6.575283643313486}{\hour}
                % ================ 1st try =============== %
                % ================ 1st try =============== %
                % ================ 1st try =============== %
                % t
                % = C_{A\,0}
                % \int_0^{0.99}{
                %     \frac{\odif{X}}{-r_A}
                % }
                % = C_{A\,0}
                % \int_0^{0.99}{
                %     \frac{\odif{X}}{
                %         k\,C_{A}
                %     }
                % }
                % = &\\&
                % = C_{A\,0}
                % \int_0^{0.99}{
                %     \frac{\odif{X}}{
                %         k\,C_{A\,0}(1-X)
                %     }
                % }
                % % = &\\&
                % = \frac{C_{A\,0}}{k\,C_{A\,0}}
                % \int_0^{0.99}{
                %     \frac{\odif{X}}{
                %         (1-X)
                %     }
                % }
                % = &\\&
                % = -k^{-1}
                % \int_{1}^{1-0.99}{
                %     \frac{\odif{(1-X)}}{(1-X)}
                % }
                % % = &\\&
                % = -k^{-1}
                % \adif{\log(1-X)}
                % \big\vert_{1}^{0.01}
                % = &\\&
                % = -(0.03*60)^{-1}\,\log(0.01)
                % = 3.6\,\si{\hour}
            &
        \end{flalign*}
    \end{questionBox}
    \begin{questionBox}2{ % Q2.3
        Det o vol
        \begin{itemize}
            \item PFR
            \item \(v_A=100\,\si{\litre/\second}\)
            \item \(P=2\,\si{\atm}\)
        \end{itemize}
    } % Q2.3
        \answer{}
        \begin{flalign*}
            &
                V:
                \odif{V}
                = F_{A\,0}\,\frac{\odif{X}}{-r_A}
                = C_{A\,0}\,v_0
                \,\frac{\odif{X}}{
                    k\,(F_{A}/v)
                }
                = &\\&
                = C_{A\,0}\,v_0
                \,\frac{\odif{X}}{
                    \frac{
                        k\,(C_{A\,0}(1-X))
                    }{
                        (1+\varepsilon\,X)
                        \,(T/T_0)
                        \,(P_0/P)
                    }
                }
                = \frac{v_0}{k}
                \,\frac{\odif{X}}{
                    \frac{
                        1-X
                    }{
                        1+\varepsilon\,X
                    }
                }
                = \frac{v_0}{k}
                \,\frac{
                    1+\varepsilon\,X
                }{
                    1-X 
                }\odif{X}
                \implies &\\[3ex]&
                \implies
                V
                = \frac{v_0}{k}
                \,\int{
                    \frac{
                        1+\varepsilon\,X
                    }{
                        1-X 
                    }\odif{X}
                }
                = &\\&
                = \frac{v_0}{k}
                \adif{\left(
                    -\varepsilon\,X
                    + (1+\varepsilon)
                    \,\ln\frac{1}{1-X}
                \right)}
                \Bigg\vert_0^X
                = &\\&
                = \frac{v_0}{k}
                \left(
                    -\varepsilon\,X
                    + (1+\varepsilon)
                    \,\ln\frac{1}{1-X}
                \right)
                = &\\&
                = \frac{100}{(0.03/60)}
                \left(
                    -2*0.99
                    + (1+2)
                    \,\ln\frac{1}{1-0.99}
                \right)
                \cong
                \SI{2.367102111592854820822e6}{\litre}
            &
        \end{flalign*}
    \end{questionBox}
\end{questionBox}

\begin{questionBox}1{ % Q3
    Det numero de reatores
    \begin{itemize}
        \begin{multicols}{2}
            \item \ch{A -> B}
            \item bateria de R CSTR
            \item \(V_r=1\,\si{\metre^3}\)
            \item \(k=0.5\,\si{\hour^{-1}}\)
            \item \(C_{A\,0}=5\,\si{\M}\)
            \item \(v_0=1759\,\si{\deci\metre^3/\hour}\)
            \item \(X\geq 89\%\)
        \end{multicols}
    \end{itemize}
} % Q3
    \answer{}
    \begin{flalign*}
        &
            N_R
            = \ceil{V_R/V_r}
            = \ceil{1.15*V/1}
            = \ceil{1.15*(N_{A\,0}/C_{A\,0})}
            = &\\&
            = \ceil{1.15*(N_{A}/(1-X))/C_{A\,0}}
            ; &\\[6ex]&
            -r_A 
            = k\,C_{A}
            = k\,C_{A\,0}\,(1-X)
        &
    \end{flalign*}
\end{questionBox}

\end{document}
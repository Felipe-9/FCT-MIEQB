% !TEX root = ./ERQ_I-Testes_Resolucoes.2021.1.1.tex
\providecommand\mainfilename{"./ERQ_I-Testes_Resolucoes.tex"}
\providecommand \subfilename{}
\renewcommand   \subfilename{"./ERQ_I-Testes_Resolucoes.2021.1.1.tex"}
\documentclass[\mainfilename]{subfiles}

% \tikzset{external/force remake=true} % - remake all

\begin{document}

% \graphicspath{{\subfix{./.build/figures/ERQ_I-Testes_Resolucoes.2021.1.1}}}
% \tikzsetexternalprefix{./.build/figures/ERQ_I-Testes_Resolucoes.2021.1.1/graphics/}

\mymakesubfile{1}
[ERQ I]
{Teste 2021.1 Repescagem Resolução} % Subfile Title
{Teste 2021.1 Repescagem Resolução} % Part Title

\begin{questionBox}1{ % Q1
    A fugura mostra a curva cinética obtida em reator bath para a reação elementar, em fase líquida \ch{A + B <> C + D}. A reação é conduzida em reatores batch com volume de 5\,\si{\metre^3} cada, que são carregados com uma solução 1\,\si{\M} em \ch{A}. Determine, mostrando todos os calculos e tambem usando os gráfico:
} % Q1
    \paragraph*{Dados:}
    \begin{description}[
        % leftmargin=!,
        % labelwidth=\widthof{} % Longest item
    ]
        \item[Tempos mortos] 1.5\,\si{\hour}
        \item[Peso molecular]\ \vspace{-2ex}
        \begin{itemize}
            \begin{multicols}{3}
                \item \(C:\) 130;
                \item \(C_{A\,0}:\) 1\,\si{\M};
                \item \(C_{B\,0}:\) 2\,\si{\M}
            \end{multicols}
        \end{itemize} 
    \end{description}
    \begin{questionBox}2{ % Q1.1
        A experssão da lei cinética
    } % Q1.1
        \answer{}
        \begin{flalign*}
            &
                r
                = k\left(
                    C_{A}\,C_{B}
                    -\frac{C_{C}\,C_D}{k_e}
                \right)
            &
        \end{flalign*}
    \end{questionBox}
    \begin{questionBox}2{ % Q1.2
        O valor da converão de equilíbrio.
    } % Q1.2
        \answer{}
        O valor de X quando a curva está horizontal: 0.8
    \end{questionBox}
    \begin{questionBox}2{ % Q1.3
        O valor da constante de equilíbrio.
    } % Q1.3
        \answer{}
        \begin{flalign*}
            &
                r 
                = 0
                = k\left(
                    C_{A}\,C_{B}
                    -\frac{C_{C}\,C_D}{k_e}
                \right)
                = &\\&
                = k\left(
                    C_{A\,0}(1-X_e)
                    \,C_{A\,0}(\theta_B-X_e)
                    -\frac{
                        C_{A\,0}\,X_e
                        \,C_{A\,0}\,X_e
                    }{k_e}
                \right)
                = &\\&
                = k\,C_{A\,0}^2\left(
                    (1-X_e)
                    \,(\theta_B-X_e)
                    -\frac{X_e^2}{k_e}
                \right)
                \implies &\\&
                \implies
                (1-X_e)
                \,(\theta_B-X_e)
                -\frac{X_e^2}{k_e}
                = \theta_B-X_e
                -X_e\,\theta_B
                +X_e^2
                -\frac{X_e^2}{k_e}
                =0
            &
        \end{flalign*}
    \end{questionBox}
    \begin{questionBox}2{ % Q1.4
        Os valores do tempo ótimo e da converção ótima.
    } % Q1.4
        \answer{}
        \(\cong1\). No grafico conecta do ponto (-1.5,0) até tangenciar a reta, o valor T do ponto tangente é o tempo ótimo.
    \end{questionBox}
    \begin{questionBox}2{ % Q1.5
        O valor da constante cinética da reação direta
    } % Q1.5
        \answer{}
    \end{questionBox}
    \begin{questionBox}2{ % Q1.6
        O numero de reatores necessário a uma produção anual de \(C\) de 1000\,\si{\tonne}, supondo que a fabrica funciona 24\,\si{\hour} por dia e 330\,\si{\day/\year}, supondo que se pretende uma conversão correspondente a 99\,\si{\percent} da converão de equilíbrio.
    } % Q1.6

        \answer{}
        \begin{flalign*}
            &
                V_R
                = 1.15*V
                = 1.15*\frac{N_{A\,0}}{C_{A\,0}}
                = 1.15*\frac{N_C/X}{1}
                = 1.15*\frac{N_C}{X}
                = &\\&
                = 1.15*\frac{
                    \frac{1\e9}{130*\text{batch}_{\text{ano}}}
                }{0.99\,X_e}
                = 1.15*\frac{
                    \frac{
                        1\e9
                    }{
                        130
                        *\frac{24*330}{t_{\text{batch}}}
                    }
                }{0.792}
                % = &\\&
                = 1.15*\frac{
                    \frac{
                        1\e9
                    }{
                        130
                        *\frac{24*330}{t+t_d}
                    }
                }{0.792}
                = &\\&
                = 1.15*\frac{
                    \frac{
                        1\e9
                    }{
                        130
                        *\frac{24*330}{2+1.5}
                    }
                }{0.792}
                \cong
                \num{402.936008996615057}
            &
        \end{flalign*}
    \end{questionBox}
\end{questionBox}

\begin{questionBox}1{ % Q2
    A reaçõa elelemtar, em fase gasosa, \ch{2 A -> 3 B + C} é conduzida à temperatura de 473\,\si{\kelvin} e à pressão de 5\,\si{\atm} num reator PFR (\(k=0.4\,\si{\litre.\mole^{-1}.\second^{-1}}\)). Assumindo que o reagente A é alimentado puro ao reator, a um caudal volumétrico de 15\,\si{\litre/\second} e que se obtém uma conversão de 99\,\si{\percent}, Determine:
} % Q2
    \begin{BM}
        P\left(
            \frac{1+\varepsilon\,X}{1-X}
        \right)
        = \frac{(1+\varepsilon)^2}{1-X}
        - 2\,\varepsilon(1+\varepsilon)
        \,\ln\frac{1}{1-X}
        + \varepsilon^2\,X
        \\
        R=0.082\,\si{\litre.\atm.\kelvin^{-1}.\mole^{-1}}
    \end{BM}
    \begin{questionBox}2{ % Q2.1
        O volume do reator
    } % Q2.1
        \answer{}
        \begin{flalign*}
            &
                \int_{0}^{V}{\odif{V}}
                = V
                = \int_{0}^{X}{
                    \frac{F_{A\,0}}{-r_A}\odif{X}
                }
                = F_{A\,0}
                \,\int_{0}^{X}{
                    \frac{\odif{X}}{-r_A}
                }
                ; &\\[3ex]&
                r_A 
                = -k\,C_{A}^2
                = -k\,\left(
                    \frac{F_A}{v}
                \right)^2
                = -k\,\left(
                    \frac{F_{A\,0}\,(1-X)}{v}
                \right)^2
                = &\\&
                = -k\,\left(
                    \frac{
                        F_{A\,0}\,(1-X)
                    }{
                        v_0(1+\varepsilon\,X)
                    }
                \right)^2
                = -k\,C_{A\,0}^2\,\left(
                    \frac{
                        1-X
                    }{
                        1+\varepsilon\,X
                    }
                \right)^2
                ; &\\[3ex]&
                C_{A\,0}
                = \frac{P_{A\,0}}{R\,T}
                = \frac{P\,Y_{A\,0}}{R\,T}
                &\\[3ex]&
                \varepsilon
                = Y_{A\,0}\,\delta
                = Y_{A\,0}\,(-1+3/2+1/2)
                = Y_{A\,0}
                \implies &\\[3ex]&
                \implies
                V
                = F_{A\,0}
                \int_{0}^{X}{
                    \frac{\odif{X}}{
                        k\,C_{A\,0}^2
                        \left(
                            \frac{1-X}{
                                1+\varepsilon\,X
                            }
                        \right)^2
                    }
                }
                = &\\&
                = \frac{(v_0\,C_{A\,0})}{C_{A\,0}^2\,k}
                \int_{0}^{X}{
                    \left(
                        \frac{1+\varepsilon\,X}{1-X}
                    \right)^2
                    \,\odif{X}
                }
                = \frac{v_0}{C_{A\,0}\,k}
                \int_{0}^{X}{
                    \left(
                        \frac{1+\varepsilon\,X}{1-X}
                    \right)^2
                    \,\odif{X}
                }
                = &\\&
                = \frac{v_0}{C_{A\,0}\,k}
                \adif{\left(
                    \frac{(1+\varepsilon)^2}{1-X}
                    - 2\,\varepsilon(1+\varepsilon)
                    \,\ln\frac{1}{1-X}
                    + \varepsilon^2\,X
                \right)}\Bigg\vert_0^X
            &
        \end{flalign*}
    \end{questionBox}
    \begin{questionBox}2{ % Q2.2
        O valor do caudal volumétrico à saída do reator.
    } % Q2.2
        \answer{}
        \begin{flalign*}
            &
                v
                =v_0\,(1+\varepsilon\,X)
                =15\,(1+1*0.99)
                \cong\num{29.85}\,\si{\litre/\second}
            &
        \end{flalign*}
    \end{questionBox}
    \begin{questionBox}2{ % Q2.3
        O valor do caudal molar do produto B, à saída do reator.
    } % Q2.3
        \answer{}
        \begin{flalign*}
            &
                F_B
                = \frac{3}{2}\,F_{A\,0}\,X
                = \frac{3}{2}\,C_{A\,0}\,v_0\,X
                = \frac{3}{2}
                *\num{0.128978675645342}
                *15
                *0.99
            &
        \end{flalign*}
    \end{questionBox}
    \begin{questionBox}2{ % Q2.4
        Caso a reação seja conduzida num reator batch, a volume constante, nas mesmas condições de temperatura e pressão inicial, determine o valor da pressão à conversão de 99\,\si{\percent}
    } % Q2.4
        \answer{}
        \begin{flalign*}
            &
                \frac{P}{P_0}
                \,\frac{V}{V_0}
                = \frac{P}{P_0}
                =1+\varepsilon\,X\,\frac{T}{T_0}
                =1+\varepsilon\,X
                \implies &\\&
                \implies
                P
                = P_0\,(1+\varepsilon\,X)
                = 5\,(1+1*.99)
                = 9.95
            &
        \end{flalign*}
    \end{questionBox}
\end{questionBox}

\end{document}
% !TEX root = ./ERQ_I-Testes_Resolucoes.2020.1.tex
\providecommand\mainfilename{"./ERQ_I-Testes_Resolucoes.tex"}
\providecommand \subfilename{}
\renewcommand   \subfilename{"./ERQ_I-Testes_Resolucoes.2020.1.tex"}
\documentclass[\mainfilename]{subfiles}

% \tikzset{external/force remake=true} % - remake all

\begin{document}

\graphicspath{{\subfix{./.build/figures/ERQ_I-Testes_Resolucoes.2020.1}}}
% \tikzsetexternalprefix{./.build/figures/ERQ_I-Testes_Resolucoes.2020.1/graphics/}

\mymakesubfile{1}
[ERQ I]
{Teste 1 2020 Resolução} % Subfile Title
{Teste 1 2020 Resolução} % Part Title

\begin{questionBox}1{ % Q1
    The figure shows the kinetic curve obtained in a batch reactor, corresponding to the elemental liquid phase reaction \ch{2 A -> B}. The reaction is carried out in batch reactors of 5\,\si{\metre^3}, which are loaded with pure A.
} % Q1
    \begin{center}
        \includegraphics[width=.6\textwidth]{Screenshot 2023-11-08 at 18.33.05}
    \end{center}
    \paragraph*{Data:}
    \begin{itemize}
        \item \(t_d=120\,\si{\min}\). 
        \item Molecular weight of A: 58\,\si{\gram/\mole}.
        \item Molecular weight of B: 116\,\si{\gram/\mole}.
        \item Density of A: 0.791\,\si{\gram/\litre}. 
        \item If you were not able to solve b) use \(k=0.074\,\si{\deci\metre^3.\mole^{-1}.\hour^{-1}}\).
    \end{itemize}
    \begin{questionBox}2{ % Q1.1
        Write the expression of the rate law.
    } % Q1.1
        \begin{flalign*}
            &
                -r_A
                =k\,C_A^2
                =k\,(C_{A\,0}\,(1-X))^2
                =k\,C_{A\,0}^2\,(1-X)^2
            &
        \end{flalign*}
    \end{questionBox}
    \begin{questionBox}2{ % Q1.2
        Write the equation of the curve shown in the graphic.
    } % Q1.2
        \answer{}
        \begin{flalign*}
            &
                X=f(t):
                -r_A\,V
                =k\,C_{A\,0}^2\,(1-X)^2\,V
                = &\\&
                =-\odv{N_A}{t}
                =-\odv{(N_{A\,0}(1-X))}{t}
                =-N_{A\,0}\,\odv{(1-X)}{t}
                = &\\&
                =-C_{A\,0}\,V\,\odv{(1-X)}{t}
                \implies &\\[3ex]&
                \implies
                -\int_{1}^{1-X}{
                    \frac{\odif{(1-X)}}{(1-X)^2}
                }
                = &\\&
                =-(-1)
                \,\adif{(X^{-1})}
                \big\vert_{1}^{1-X}
                =1/(1-X)-1
                =\frac{1}{1/X-1}
                = &\\[3ex]&
                =\int_{0}^{t}{
                    k\,C_{A\,0}\,\odif{t}
                }
                =k\,C_{A\,0}
                \,\int_{0}^{t}{
                    \odif{t}
                }
                =k\,C_{A\,0}\,t
                \implies &\\[3ex]&
                \implies
                X
                =(1+1/k\,C_{A\,0}\,t)^{-1}
            &
        \end{flalign*}
    \end{questionBox}
    \begin{questionBox}2{ % Q1.3
        Evaluate the value of the kinetic constant. Use the graphic.
    } % Q1.3
        \answer{}
        \begin{flalign*}
            &
                k:
                X
                =(1+1/k\,C_{A\,0}\,t)^{-1}
                \implies &\\&
                \implies
                k 
                = \left(
                    (1/X-1)
                    \,C_{A\,0}
                    \,t
                \right)^{-1}
                = &\\&
                = \left(
                    (1/X-1)
                    \,(N_{A\,0}/V)
                    \,t
                \right)^{-1}
                = &\\&
                = \left(
                    (1/X-1)
                    \,(m_{A\,0}/M_{A}\,V)
                    \,t
                \right)^{-1}
                = &\\&
                = \left(
                    (1/X-1)
                    \,(\rho_{A}\,V/M_{A}\,V)
                    \,t
                \right)^{-1}
                = &\\&
                = \left(
                    (1/X-1)
                    \,(\rho_{A}/M_{A})
                    \,t
                \right)^{-1}
                \cong &\\&
                \cong \left(
                    (1/0.6-1)
                    \,(791/58)
                    \,1.9
                \right)^{-1}
                \,\si{\litre/\mole.\hour}
                \cong &\\&
                \cong
                \SI{57.888083039457}{\milli\litre/\mole.\hour}
            &
        \end{flalign*}
    \end{questionBox}
    \begin{questionBox}2{ % Q1.4
        Calculate the optimal conversion and the optimal reaction time.
    } % Q1.4
        \answer{}
        Traça uma reta entre \((0,-t_d)\) e o gráfico, o ponto tangente é o optimo:
        \begin{BM}
            X_{opt}\cong 0.68
            \qquad
            t_{opt}\cong 2.3\,\si{\hour}
        \end{BM}
    \end{questionBox}
    \begin{questionBox}2{ % Q1.5
        If the plant works 24\,\si{\hour} day and 330\,\si{\day/\year}, determine the number of reactors needed for an annual production of B of 1500\,\si{\tonne}. Use the conversion calculated in d) but if you were not able to, use any value at your choice.
    } % Q1.5
        \answer{}
        \center{\ch{A-> 1/2 B}}
        \begin{flalign*}
            &
                N_R
                =\ceil{V_R/5\,\si{\metre^2}}
                % 1.15 é margem de erro
                =\ceil{1.15\,V/5\,\si{\metre^2}}
                =\left\lceil
                    \frac{
                        1.15
                        \,(N_{A\,0}/C_{A\,0})
                    }{5\,\si{\metre^2}}
                \right\rceil
                ; &\\[3ex]&
                N_{A\,0}\,X/2
                % /2 pela conversão 1 A -> 1/2 B
                = N_B
                % N_B é em mol quantos
                % 1 tonne = 1e6 gram
                = \frac{1500*10^6}{
                    M_B*N_{batch}
                }
                = \frac{1500*10^6}{
                    116*\frac{
                        330*24
                    }{
                        t_{batch}
                    }
                }
                = &\\&
                = \frac{1500*10^6}{
                    \frac{
                        116*330*24
                    }{
                        t_{opt}+t_{d}
                    }
                }
                = \frac{1500*10^6}{
                    \frac{
                        116*330*24
                    }{
                        2.3+2
                    }
                }
                \cong
                \SI{7.020637408568443e3}{\mole}
                \implies &\\[3ex]&
                \implies
                N_{A\,0}
                \cong
                2*\num{7.020637408568443e3}/0.68
                \cong
                \num{20.648933554613068}
                \implies &\\[3ex]&
                \implies
                N_R
                \cong
                \left\lceil
                    \frac{
                        1.15*\num{20.648933554613068}
                    }{
                        (\rho_A/M_A)
                    }
                \right\rceil
                = \left\lceil
                    \frac{
                        1.15*\num{20.648933554613068}
                    }{
                        5*(0.791*10^{3}/58)
                    }
                \right\rceil
                \cong &\\&
                \cong
                \ceil{\num{0.348238651856559}}
                = 1
            &
        \end{flalign*}
    \end{questionBox}
\end{questionBox}

\begin{questionBox}1{ % Q2
    A reacção de 1ª ordem \ch{A -> 2 B}, em fase líquida, é conduzida num sistema de reactores contínuos, sendo uma solução de A com a concentração de 0.1\,\si{\mole/\deci\metre^3} alimentada a um caudal volumétrico de 10\,\si{\deci\metre^3/\min}. Sabendo-se que a constante cinética à temperatura da reacção é \(k = 0.02\,\si{\min^{-1}}\), e que se pretende obter uma conversão final de 70\%, determine mostrando todos os cálculos:
} % Q2
    \begin{questionBox}2{ % Q2.1
        O volume de um único reactor CSTR.
    } % Q2.1
    \end{questionBox}
    \begin{questionBox}2{ % Q2.2
        O número de reactores CSTR de 1\,\si{\metre^3} de volume, associados em série.
    } % Q2.2
    \end{questionBox}
\end{questionBox}

\end{document}
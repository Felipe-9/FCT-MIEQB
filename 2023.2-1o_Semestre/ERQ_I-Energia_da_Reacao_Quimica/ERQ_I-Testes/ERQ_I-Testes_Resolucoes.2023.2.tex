% !TEX root = ./ERQ_I-Testes_Resolucoes.2023.2.tex
\providecommand\mainfilename{"./ERQ_I-Testes_Resolucoes.tex"}
\providecommand \subfilename{}
\renewcommand   \subfilename{"./ERQ_I-Testes_Resolucoes.2023.2.tex"}
\documentclass[\mainfilename]{subfiles}

% \tikzset{external/force remake=true} % - remake all

\begin{document}

\graphicspath{{\subfix{./.build/figures/ERQ_I-Testes_Resolucoes.2023.2}}}
% \tikzsetexternalprefix{./.build/figures/ERQ_I-Testes_Resolucoes.2023.2/graphics/}

\mymakesubfile{2}
[ERQ\,I]
{Teste 2023.2 Resolução} % Subfile Title
{Teste 2023.2 Resolução} % Part Title

\begin{questionBox}1{ % Q1
    \begin{itemize}
        \begin{multicols}{2}
            \item \ch{A<>B}
            \item fase gásosa
            \item Reator tubular adiabático
            \item A\,(10\%)
            \item \(T_0=200\,\si{\celsius}=473.15\,\si{\kelvin}\)
            \item \(v_0=10\,\si{\deci\metre^3/\second}=600\,\si{\litre/\min}\)
            \item gráfico \(X\times T\)
            \item \(C_{p\,A}=C_{p\,B}=5\,\si{\calorie/\mole.\kelvin}\)
            \item \(C_{p\,I}=12\,\si{\calorie/\mole.\kelvin}\)
            \item \(k_{d\,(473\,\si{\kelvin})}=1.17\,\si{\min^{-1}}\)
            \item \(K_{e\,(473\,\si{\kelvin})}=40\)
            \item \(Ea=20\,\si{\kilo\calorie/\mole}\)
            \item \(R=\SI{1.987204258640832}{\calorie.\mole^{-1}.\kelvin^{-1}}\)
        \end{multicols}
    \end{itemize}
} % Q1
    \begin{figure}\centering
        \includegraphics[width=.8\textwidth]{IMG_5691-dark.png}
        % \includegraphics[width=.8\textwidth]{IMG_5691-bright.png}
    \end{figure}
    \begin{questionBox}2{ % Q1.1
        Endotermica/exotermica
    } % Q1.1
        \answer{}
        Exotérmica pois \(X\propto T^{-1}\) ou seja diminui conforme a temperatura almenta.
    \end{questionBox}
    \begin{questionBox}2{ % Q1.2
        Calor de reação
    } % Q1.2
        \answer{}
        \begin{flalign*}
            &
                \adif{H_R}:
                K_{e\,(T)}
                =K_{e\,(T_R)}
                \,\exp{\left(
                    -\frac{\adif{H_R}}{R}
                    (T^{-1}-T_R^{-1})
                \right)}
                \implies &\\&
                \implies
                \adif{H_R}
                = \frac{-R}{(T^{-1}-T_R^{-1})}
                \,\ln\frac{K_{e\,(T)}}{K_{e\,(T_R)}}
                ; &\\[3ex]&
                K_{e\,(T)}:
                K_{e\,(T)}
                = \frac
                    {p_{p\,B}}
                    {p_{p\,A}}
                = \frac
                    {C_{p\,B}\,R\,T}
                    {C_{p\,A}\,R\,T}
                = \frac
                    {C_{p\,A_0}\,X}
                    {C_{p\,A_0}\,(1-X)}
                =\frac{1}{1/X-1}
                \implies &\\&
                \implies
                X=\frac{1}{1+1/K_e}
                \implies
                X=0.5
                \begin{cases}
                    K_e=1
                    \\
                    T\cong 619\,\si{\kelvin}
                \end{cases}
                &\\[3ex]&
                \therefore
                {\color{Emph}
                    \adif{H_R}
                }
                = \frac{-R}{(T^{-1}-T_R^{-1})}
                \,\ln\frac{K_{e\,(T)}}{K_{e\,(T_R)}}
                \cong \frac{
                    -\num{1.987204258640832}
                }{
                    (619^{-1}-473^{-1})
                }
                \,\ln\frac{1}{40}
                \cong &\\&
                \cong
                {\color{Emph}
                    \SI{-14.700628636232281423}{\kilo\calorie/\mole}
                }
            &
        \end{flalign*}
    \end{questionBox}
    \begin{questionBox}2{ % Q1.3
        \(X_{eq}\land T_{eq}\)
    } % Q1.3
        \answer{}
        \begin{flalign*}
            &
                % X_{(770\,\si{\kelvin})}
                % =\frac{(C_{p\,A}+\theta_I\,C_{p\,I})(T-T_0)}{-\adif{H_R}}
                % =\frac{(C_{p\,A}+\frac{Y_{I\,0}}{Y_{A\,0}}\,C_{p\,I})(T-T_0)}{-\adif{H_R}}
                % \cong &\\&
                % \cong\frac{
                %     \left(
                %         5+\frac{0.9}{0.1}\,12
                %     \right)
                %     (770-473)
                % }{\num{14.700628636232281423e3}}
                % \cong\dots
                {\color{Emph}
                    X_{(520\,\si{\kelvin})}
                }
                =\frac{(C_{p\,A}+\theta_I\,C_{p\,I})(T-T_0)}{-\adif{H_R}}
                =\frac{(C_{p\,A}+\frac{Y_{I\,0}}{Y_{A\,0}}\,C_{p\,I})(T-T_0)}{-\adif{H_R}}
                \cong &\\&
                \cong\frac{
                    \left(
                        5+\frac{0.9}{0.1}\,12
                    \right)
                    (520-473)
                }{\num{14.700628636232281423e3}}
                \cong
                {\color{Emph}
                    \num{0.360124055304131}
                }
                &\\&
                \begin{cases}
                    X_0=0;& T_0=473
                    \\
                    X_1\cong\num{0.360124055304131}
                    ;& T_1=520
                \end{cases}
            &
        \end{flalign*}
        \begin{BM}\color{Emph}
            X_{eq}\cong 0.79
            \quad\land\quad
            T_{eq}\cong 546\,\si{\kelvin}
        \end{BM}
        % Houve algum erro ao calcular o \(X_{(770\,\si{\kelvin})}\) portanto para poder prosseguir assumo o valor de 0.6
    \end{questionBox}
    \begin{questionBox}2{ % Q1.4
        Volume do reator para \(X_1=95\%\,X_e\)
    } % Q1.4
        \answer{}
        \begin{flalign*}
            &
                V
                =\int_{0}^{X}{\frac{F_{A\,0}\odif{X}}{-r_A}}
                =\int_{0}^{X}{\frac{C_{A\,0}\,v_0\odif{X}}{-r_A}}
                ; &\\[3ex]&
                -r_A
                = k(C_A-C_B/K_e)
                = &\\&
                =k\left(
                    \left(
                        \frac{C_{A\,0}(1-X)}{1+\varepsilon\,X}
                        \frac{T_0}{T}
                    \right)
                    -\left(
                        \frac{C_{A\,0}\,X}{1+\varepsilon\,X}
                        \frac{T_0}{T}
                    \right)
                    /K_e
                \right)
                = &\\&
                = k\,\left(
                    \frac{
                        C_{A\,0}(1-X(1-1/K_e))
                    }{1+\varepsilon\,X}
                    \frac{T_0}{T}
                \right)
                = &\\&
                = k\,\left(
                    \frac{
                        C_{A\,0}(1-X(1-1/K_e))
                    }{1+y_{A\,0}\,\delta\,X}
                    \frac{T_0}{T}
                \right)
                = &\\&
                = k\,\left(
                    \frac{
                        C_{A\,0}(1-X(1-1/K_e))
                    }{1+y_{A\,0}\,(1-1)\,X}
                    \frac{T_0}{T}
                \right)
                = &\\&
                = k\,C_{A\,0}(1-X(1-K_e))
                \,\frac{T_0}{T}
                \implies &\\[3ex]&
                \implies
                V
                =\int_{0}^{X}{
                    \frac
                        {C_{A\,0}\,v_0\odif{X}}
                        {
                            k\,C_{A\,0}(1-X(1-K_e))
                            \,\frac{T_0}{T}
                        }
                }
                = &\\&
                =\int_{0}^{X}{
                    \frac
                        {v_0\,\odif{X}}
                        {
                            k\,(1-X(1-K_e))
                            \,\frac{T_0}{T}
                        }
                }
                = &\\&
                =\int_{0}^{.95*.79}{
                    \frac
                        {600\,\odif{X}}
                        {
                            k\,(1-X(1-K_e))
                            \,\frac{473}{T}
                        }
                }
                = &\\&
                \cong\int_{0}^{0.7505}{
                    \frac
                        {\num{1.268498942917548}\,\odif{X}}
                        {
                            k\,(1-X(1-K_e))/T
                        }
                }
                &\\[3ex]&
                \text{Simpson:}&\\&
                {\color{Emph}
                    f(X)
                    = \frac
                    {\num{1.268498942917548}\,\odif{X}}
                    {
                        k\,(1-X(1-K_e))/T
                    }
                }
                ; &\\[3ex]&
                {\color{Emph}T}:
                X
                =\frac{(C_{p\,A}+\theta_I\,C_{p\,I})(T-T_0)}{-\adif{H_R}}
                \implies &\\&
                \implies 
                T
                =T_0-\frac
                    {X\,\adif{H_R}}
                    {C_{p\,A}+\theta_I\,C_{p\,I}}
                \cong 
                473-\frac
                    {-X\,\num{14.700628636232281423e3}}
                    {5+\frac{.9}{.1}\,12}
                \cong &\\&
                \cong {\color{Emph}
                    473+X\,\num{130.094058727719292}
                }
                ; &\\[3ex]&
                % Arrhenios
                {\color{Emph}
                    k_{(T)}
                }
                =k_{(T_R)}
                \,\exp{\left(
                    -\frac{Ea}{R}(T^{-1}-T_R^{-1})
                \right)}
                = &\\&
                \cong 
                1.17
                \,\exp{\left(
                    -\frac
                        {20\E{3}}
                        {\num{1.987204258640832}}
                        (T^{-1}-473^{-1})
                \right)}
                \cong &\\&
                \cong {\color{Emph}
                    1.17
                    \,\exp{\left(
                        -\num{10.064390669975314e3}
                        (T^{-1}-473^{-1})
                    \right)}
                }
                ; &\\[3ex]&
                % Vant Hoff
                {\color{Emph}
                    K_{e\,(T)}
                }
                = K_{e\,(T_R)}
                \,\exp{\left(
                    -\frac{\adif{H}}{R}
                    \,(T^{-1}-T_R^{-1})
                \right)}
                \cong &\\&
                \cong
                 40
                \,\exp{\left(
                    \frac
                        {\num{14.700628636232281423e3}}
                        {\num{1.987204258640832}}
                    \,(T^{-1}-473^{-1})
                \right)}
                \cong &\\&
                \cong {\color{Emph}
                    40
                    \,\exp{\left(
                        \num{7.397643484463404028e3}
                        \,(T^{-1}-473^{-1})
                    \right)}
                }
                &\\[3ex]&
                h=\frac{0.7505}{2}=0.37525
                \begin{cases}
                    X_0=0\\X_1=0.37525\\X_2=0.7505
                \end{cases}
                &\\[3ex]&
                \therefore
                V=\frac{h}{3}\left(
                    f_{(X_0)}
                    + 4\,f_{(X_1)}
                    + f_{(X_2)}
                \right)
            &
        \end{flalign*}
        Encontramos com os 3 pontos de X os valores para T, \(k_{(T)}\), \(K_{e\,(T)}\) que podem ser usados para encontrar \(f(X)\) que finalmente são usados para encontrar \textit{V}
    \end{questionBox}
    \begin{questionBox}2{ % Q1.5
        \(Y_{A\,0}\) para \(X_{eq}=90\%\)
    } % Q1.5
        \answer{}
        \begin{flalign*}
            &
                Y_{A\,0}: 
                -X_{(T)}\,\adif{H_R}
                = \left(C_{p\,A}+\theta_I\,C_{p\,I}\right)(T-T_0)
                = &\\&
                = \left(C_{p\,A}+\frac{Y_{I\,0}}{Y_{A\,0}}\,C_{p\,I}\right)(T-T_0)
                = &\\&
                =\left(C_{p\,A}+\frac{1-Y_{A\,0}}{Y_{A\,0}}\,C_{p\,I}\right)(T-T_0)
                ; &\\&
                T_{eq}
                \cong
                473+0.9*\num{130.094058727719292}
                \cong 
                \num{590.08465285494737}
                \implies &\\&
                \implies
                Y_{A\,0}
                = \left(
                    1+\frac{
                        \frac{-X_{(T)}\,\adif{H_R}}{T-T_0}
                        -C_{p\,A}
                    }{
                        C_{p\,I}
                    }
                \right)^{-1}
                = &\\&
                = \left(
                    1+\frac{
                        \frac{0.9*\num{14.700628636232281423e3}}{\num{590.08465285494737}-473}
                        -8
                    }{
                        12
                    }
                \right)^{-1}
                \cong
                \num{0.102564102564103}
            &
        \end{flalign*}
    \end{questionBox}
\end{questionBox}

\begin{questionBox}1{ % Q2
    \begin{itemize}
        \begin{multicols}{2}
            \item \ch{A->B}
            \item CSTR não adiabático
            \item \(V=6\,\si{\metre^3}\)
            \item Em estado estacionário
            \item \(v_0=6\,\si{\litre/\min}=360\,\si{\litre/\hour}\)
            \item \(Y_{A\,0}=0.1\)
            \item \(\adif{H_R}=-100.7\,\si{\kilo\joule/\mole}\)
            \item \(C_{p\,A}=C_{p\,B}=34.6\,\si{\joule/\mole.\kelvin}\)
            \item \(C_{p\,I}=75.4\,\si{\joule/\mole.\kelvin}\)
            \item \(R=\SI{8.314462618}{\joule.\mole^{-1}.\kelvin^{-1}}\)
            \item \(C_{A\,0}=1\,\si{\M}\)
            \item \(T_0=298\,\si{\kelvin}\)
            \item \(k_{(298\,\si{\kelvin})}=0.0227\,\si{\hour^{-1}}\)
        \end{multicols}
        \item com parede envolvida até 85\% com agua a \(100\,\si{\celsius}=373\,\si{\kelvin}\)
    \end{itemize}
} % Q2
    \begin{questionBox}2{ % Q2.1
        Equações das curvas
    } % Q2.1
        \answer{}
        \begin{flalign*}
            &
                G_{(T)}=-\adif{H_R}\,X
                &\\[3ex]&
                X:
                r_{(T)}
                = G_{(T)}
                = \frac{Ua}{F_{A\,0}}
                \,(T-T_0)
                +\sum{\theta_i\,C{p\,i}(T-T_0)}
                = &\\&
                = \frac{
                    \left(
                        \frac{Q^\cdot}{T_0-T}
                    \right)
                }{F_{A\,0}}
                \,(T-T_0)
                +\frac{1-Y_{A\,0}}{Y_{A\,0}}
                \,C_{p\,B}(T-T_0)
                = &\\&
                = \frac{
                    -Q^\cdot
                }{C_{A\,0}\,v_0}
                +(1/Y_{A\,0}-1)
                \,C_{p\,A}\,X(T-T_0)
                \implies &\\&
                \implies 
                X
                = \frac{
                    G_{(T)}
                    +\frac{
                        Q^\cdot
                    }{C_{A\,0}\,v_0}
                }{(1/Y_{A\,0}-1)\,C_{p\,A}(T-T_0)}
                \implies &\\[3ex]&
                \implies
                G_{(T)}
                = &\\&
                =-\adif{H_R}
                \,\left(
                    \frac{
                        Q^\cdot
                    }{C_{A\,0}\,v_0}
                \right)
                \,\left(
                    (1/Y_{A\,0}-1)\,C_{p\,A}(T-T_0)
                    +\adif{H_R}
                \right)^{-1}
                = &\\&
                =-100.7\E{3}
                \,\left(
                    \frac{
                        Q^\cdot
                    }{1*360}
                \right)
                \,\left(
                    (1/0.1-1)\,1(T-298)
                    -100.7\E{3}
                \right)^{-1}
                \cong &\\&
                \cong 
                \frac{
                    -\num{31.080246913580244}
                    \,Q^\cdot
                }{
                    T-\num{11.486888888888889e3}
                }
            &
        \end{flalign*}
        Por não conseguir calcular \(Q^{\cdot}\) deixei como variável, se mostra necessário uma vez que a equação final deve ser uma reta, \(Q^\cdot \propto T^2\) o que nao verificou
    \end{questionBox}
    \setcounter{subquestion}{3}
    \begin{questionBox}2{ % Q2.4
        Ea usando gráfico
    } % Q2.4
    \end{questionBox}
\end{questionBox}
\end{document}
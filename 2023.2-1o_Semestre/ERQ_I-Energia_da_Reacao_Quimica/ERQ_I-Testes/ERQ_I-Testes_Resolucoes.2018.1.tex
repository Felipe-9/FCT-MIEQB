% !TEX root = ./ERQ_I-Testes_Resolucoes.2018.1.tex
\providecommand\mainfilename{"./ERQ_I-Testes_Resolucoes.tex"}
\providecommand \subfilename{}
\renewcommand   \subfilename{"./ERQ_I-Testes_Resolucoes.2018.1.tex"}
\documentclass[\mainfilename]{subfiles}

% \tikzset{external/force remake=true} % - remake all

\begin{document}

\graphicspath{{\subfix{./.build/figures/ERQ_I-Testes_Resolucoes.2018.1}}}
% \tikzsetexternalprefix{./.build/figures/ERQ_I-Testes_Resolucoes.2018.1/graphics/}

\mymakesubfile{1}
[ERQ I]
{Teste 1 Resolução} % Subfile Title
{Teste 1 Resolução} % Part Title

\begin{questionBox}1{ % Q1
    The reversible reaction \ch{A<>B} is carried out in a battery of two CSTR reactors arranged in series, it being known that the second reactor is double the volume of the first reactor. Reagent A is fed to the reactor battery in a concentration of 0.5\,\unit{\molar} at a volumetric flow rate of 20\,\unit{\litre/\min}. The direct and reverse reactions are elementary and the values of the direct reaction kinetic constant and the equilibrium constant are 0.15\,\unit{\min^{-1}} and 28, respectively.
} % Q1
    \begin{questionBox}2{ % Q1.1
        Derive the expression of the rate law.
    } % Q1.1
        \answer{}
        \begin{flalign*}
            &
                -r_{A}
                = k\,\left(
                    \ch{[A]}
                    -\ch{[B]}/k_e
                \right)
                = k\,\left(
                    \ch{[A]}_0\,(1-X)
                    -\ch{[A]}_0\,X/k_e
                \right)
                = &\\&
                = k\,\ch{[A]}_0\left(
                    1-X\,(1+1/k_e)
                \right)
            &
        \end{flalign*}
    \end{questionBox}
    \begin{questionBox}2{ % Q1.2
        For each of the reactors, derive the expressions that relate the volume of the reactor to the conversion.
    } % Q1.2
        \answer{}
        \begin{flalign*}
            &
                % =============== 1 Reator =============== %
                \text{1 Reator:}&\\&
                \tau_1
                =V_1/v
                ; &\\[3ex]&
                0
                = F_{A\,0}-F_A+r_{A\,1}\,V_1
                = F_{A\,0}
                - F_{A\,0}\,(1-X_1)
                + r_{A\,1}\,V_1
                = &\\&
                = F_{A\,0}\,X_1
                + r_{A\,1}\,V_1
                \implies &\\&
                \implies
                0
                = F_{A\,0}\,X_1/v
                + r_{A\,1}\,V/v
                = C_{A\,0}\,X_1
                + r_{A\,1}\,\tau_1
                \implies &\\&
                \implies
                \tau_1
                =\frac{C_{A\,0}\,X_1}{-r_{A\,1}}
                =\frac{C_{A\,0}\,X_1}{
                    k\,C_{A\,0}\,\left(
                        1-X_1\,(1+1/k_e)
                    \right)
                }
                = &\\&
                =\frac{1}{
                    k\,\left(
                        1/X_1-1-1/k_e
                    \right)
                }
                &\\[3ex]&
                % =============== 2 Reator =============== %
                \text{2 Reator:}&\\&
                \tau_2=V_2/v
                ; &\\[3ex]&
                0
                = F_{A\,1}
                - F_{A\,2}
                + r_{A\,2}\,V_2
                = &\\&
                = F_{A\,0}\,(1-X_1)
                - F_{A\,0}\,(1-X_2)
                + r_{A\,2}\,V_2
                = &\\&
                = F_{A\,0}\,(X_1-X_2)
                + r_{A\,2}\,V_2
                \implies &\\[1.5ex]&
                \implies
                0
                = F_{A\,0}\,(X_1-X_2)/v
                + r_{A\,2}\,V_2/v
                = &\\&
                = C_{A\,0}\,(X_1-X_2)
                + r_{A\,2}\,\tau_2
                \implies &\\[1.5ex]&
                \implies
                \tau_2
                = \frac{
                    C_{A\,0}\,(X_1-X_2)
                }{-r_{A\,2}}
                = \frac{
                    C_{A\,0}\,(X_1-X_2)
                }{
                    k\,C_{A\,0}\,\left(
                        1-X_2\,(1+1/k_e)
                    \right)
                }
                = &\\&
                = \frac{
                    X_1-X_2
                }{
                    k\,\left(
                        1-X_2\,(1+1/k_e)
                    \right)
                }
            &
        \end{flalign*}
    \end{questionBox}
    \begin{questionBox}2{ % Q1.3
        Determine the value of the equilibrium conversion.
    } % Q1.3
        \answer{}
        \begin{flalign*}
            &
                X_e
                =\frac{\ch{[B]}_e}{\ch{[A]}_0}
                =1-\frac{\ch{[A]}_e}{\ch{[A]}_0}
                ; &\\[3ex]&
                \frac{\ch{[B]}_e}{\ch{[A]}_e}
                = \frac{\ch{[A]}_0\,X_e}{\ch{[A]}_0\,(1-X_e)}
                = \frac{1}{1/X_e-1}
                = k_e
                \implies &\\&
                \implies
                X_e
                = \frac{1}{1/k_e+1}
                = \frac{1}{1/28+1}
                \cong
                \num{0.96551724137931}
            &
        \end{flalign*}
    \end{questionBox}
    \begin{questionBox}2{ % Q1.4
        Knowing that the conversion at the exit of the 2nd reactor corresponds to 99\% of the equilibrium conversion, determine the conversion at the exit of the 1st reactor.
    } % Q1.4
        \answer{}
        \begin{flalign*}
            &
                \frac{1}{
                    k\,\left(
                        1/X_1-1-1/k_e
                    \right)
                }
                =\tau_1
                = &\\&
                =\tau_2/2
                =\frac{
                    X_1-X_2
                }{
                    2\,k\,\left(
                        1-X_2\,(1+1/k_e)
                    \right)
                }
                % = &\\&
                % =-\frac{
                %     X_1-(X_e\,0.99)
                % }{
                %     2\,k\,\left(
                %         1-(X_e\,0.99)\,(1+1/k_e)
                %     \right)
                % }
                \implies &\\&
                \implies
                0=
                \left(
                    \begin{aligned}
                        &
                            X_1^2\,\left(
                                1+1/k_e
                            \right)
                        &+\\-&
                            X_1\,\left(
                                1
                                +X_2\left(
                                    1+5/k_e+4
                                \right)
                            \right)
                        &+\\+&
                            X_2
                        &
                    \end{aligned}
                \right)
                = &\\&
                = \left(
                    \begin{aligned}
                        &
                            X_1^2\,\left(
                                1+1/k_e
                            \right)
                        &+\\-&
                            X_1\,\left(
                                1
                                +X_e\,0.99\left(
                                    1+5/k_e+4
                                \right)
                            \right)
                        &+\\+&
                            X_e\,0.99
                        &
                    \end{aligned}
                \right)
                \cong &\\&
                \cong \left(
                    \begin{aligned}
                        &
                            X_1^2\,\left(
                                1+1/28
                            \right)
                        &+\\-&
                            X_1\,\left(
                                1
                                +\num{0.96551724137931}
                                *0.99\left(
                                    1+5/28+4
                                \right)
                            \right)
                        &+\\+&
                            \num{0.96551724137931}
                            *0.99
                        &
                    \end{aligned}
                \right)
                \cong &\\&
                \cong 
                % X_1 linear coeff was wrong before
                \left(
                    \begin{aligned}
                        &
                            X_1^2\,\num{1.035714285714286}
                        &+\\-&
                            X_1\,\num{2.009854}
                        &+\\+&
                            \num{0.955862068965517}
                        &
                    \end{aligned}
                \right)
                \begin{cases}
                      \num{1.1064029467593446}
                    \\\emph{\num{0.8341457428958271}}
                \end{cases}
                % &\\&
                \therefore 
                X_1\cong\num{0.8341457428958271}
            &
        \end{flalign*}
    \end{questionBox}
    \begin{questionBox}2{ % Q1.5
        Determine the volumes of the reactors.
    } % Q1.5
        \answer{}
        \begin{flalign*}
            &
                V_i=v\,\tau_i;
                &\\[3ex]&
                \text{1 Reator}&\\&
                V_1
                =v\,\tau_1
                =v\,\left(
                    \frac{1}{
                        k\,\left(
                            1/X_1-1-1/k_e
                        \right)
                    }
                \right)
                \cong &\\&
                \cong
                \frac{20}{
                    0.15\,\left(
                        1/\num{0.8341457428958271}
                        -1
                        -1/28
                    \right)
                }
                \cong
                \num{817.409261406201231}
                &\\[3ex]&
                \text{2 Reator}&\\&
                V_2
                =v\,\tau_2
                =v\,2\,\tau_1
                =2\,V_1
                \cong 2\,\num{817.409261406201231}
                \cong \num{1634.818522812402461}
            &
        \end{flalign*}
    \end{questionBox}
\end{questionBox}

\begin{questionBox}1{ % 
    The figure shows the kinetic curve obtained in a batch reactor, corresponding to the elemental liquid phase reaction \ch{2 A -> B}. The reaction is carried out in batch reactors of 5\,\unit{\metre^3}, which are loaded with pure A.
} % Q2
    \begin{center}
        \includegraphics[width=.6\textwidth]{Screenshot 2023-11-08 at 18.33.05}
    \end{center}
    \paragraph*{Data:}
    \begin{itemize}
        \item \(t_d=120\,\unit{\min}\). 
        \item Molecular weight of A: 58\,\unit{\gram/\mole}.
        \item Molecular weight of B: 116\,\unit{\gram/\mole}.
        \item Density of A: 0.791\,\unit{\gram/\litre}. 
        \item If you were not able to solve b) use \(k=0.074\,\unit{\deci\metre^3.\mole^{-1}.\hour^{-1}}\).
    \end{itemize}
    \begin{questionBox}2{ % Q2.1
        Write the expression of the rate law.
    } % Q2.1
        \begin{flalign*}
            &
                -r_A
                =k\,C_A^2
                =k\,(C_{A\,0}\,(1-X))^2
                =k\,C_{A\,0}^2\,(1-X)^2
            &
        \end{flalign*}
    \end{questionBox}
    \begin{questionBox}2{ % Q2.2
        Write the equation of the curve shown in the graphic.
    } % Q2.2
        \answer{}
        \begin{flalign*}
            &
                X=f(t):
                -r_A\,V
                =k\,C_{A\,0}^2\,(1-X)^2\,V
                = &\\&
                =-\odv{N_A}{t}
                =-\odv{(N_{A\,0}(1-X))}{t}
                =-N_{A\,0}\,\odv{(1-X)}{t}
                = &\\&
                =-C_{A\,0}\,V\,\odv{(1-X)}{t}
                \implies &\\[3ex]&
                \implies
                -\int_{1}^{1-X}{
                    \frac{\odif{(1-X)}}{(1-X)^2}
                }
                = &\\&
                =-(-1)
                \,\adif{(X^{-1})}
                \big\vert_{1}^{1-X}
                =1/(1-X)-1
                =\frac{1}{1/X-1}
                = &\\[3ex]&
                =\int_{0}^{t}{
                    k\,C_{A\,0}\,\odif{t}
                }
                =k\,C_{A\,0}
                \,\int_{0}^{t}{
                    \odif{t}
                }
                =k\,C_{A\,0}\,t
                \implies &\\[3ex]&
                \implies
                X
                =(1+1/k\,C_{A\,0}\,t)^{-1}
            &
        \end{flalign*}
    \end{questionBox}
    \begin{questionBox}2{ % Q2.3
        Evaluate the value of the kinetic constant. Use the graphic.
    } % Q2.3
        \answer{}
        \begin{flalign*}
            &
                k:
                X
                =(1+1/k\,C_{A\,0}\,t)^{-1}
                \implies &\\&
                \implies
                k 
                = \left(
                    (1/X-1)
                    \,C_{A\,0}
                    \,t
                \right)^{-1}
                = &\\&
                = \left(
                    (1/X-1)
                    \,(N_{A\,0}/V)
                    \,t
                \right)^{-1}
                = &\\&
                = \left(
                    (1/X-1)
                    \,(m_{A\,0}/M_{A}\,V)
                    \,t
                \right)^{-1}
                = &\\&
                = \left(
                    (1/X-1)
                    \,(\rho_{A}\,V/M_{A}\,V)
                    \,t
                \right)^{-1}
                = &\\&
                = \left(
                    (1/X-1)
                    \,(\rho_{A}/M_{A})
                    \,t
                \right)^{-1}
                \cong &\\&
                \cong \left(
                    (1/0.6-1)
                    \,(791/58)
                    \,1.9
                \right)^{-1}
                \,\unit{\litre/\mole.\hour}
                \cong &\\&
                \cong
                \qty{57.888083039457}{\milli\litre/\mole.\hour}
            &
        \end{flalign*}
    \end{questionBox}
    \begin{questionBox}2{ % Q2.4
        Calculate the optimal conversion and the optimal reaction time.
    } % Q2.4
        \answer{}
        Traça uma reta entre \((0,-t_d)\) e o gráfico, o ponto tangente é o optimo:
        \begin{BM}
            X_{opt}\cong 0.68
            \qquad
            t_{opt}\cong 2.3\,\unit{\hour}
        \end{BM}
    \end{questionBox}
    \begin{questionBox}2{ % Q2.5
        If the plant works 24\,\unit{\hour} day and 330\,\unit{\day/\year}, determine the number of reactors needed for an annual production of B of 1500\,\unit{\tonne}. Use the conversion calculated in d) but if you were not able to, use any value at your choice.
    } % Q2.5
        \answer{}
        \center{\ch{A-> 1/2 B}}
        \begin{flalign*}
            &
                N_R
                =\ceil{V_R/5\,\unit{\metre^2}}
                % 1.15 é margem de erro
                =\ceil{1.15\,V/5\,\unit{\metre^2}}
                =\left\lceil
                    \frac{
                        1.15
                        \,(N_{A\,0}/C_{A\,0})
                    }{5\,\unit{\metre^2}}
                \right\rceil
                ; &\\[3ex]&
                N_{A\,0}\,X/2
                % /2 pela conversão 1 A -> 1/2 B
                = N_B
                % N_B é em mol quantos
                % 1 tonne = 1e6 gram
                = \frac{1500*10^6}{
                    M_B*N_{batch}
                }
                = \frac{1500*10^6}{
                    116*\frac{
                        330*24
                    }{
                        t_{batch}
                    }
                }
                = &\\&
                = \frac{1500*10^6}{
                    \frac{
                        116*330*24
                    }{
                        t_{opt}+t_{d}
                    }
                }
                = \frac{1500*10^6}{
                    \frac{
                        116*330*24
                    }{
                        2.3+2
                    }
                }
                \cong
                \qty{7.020637408568443e3}{\mole}
                \implies &\\[3ex]&
                \implies
                N_{A\,0}
                \cong
                2*\num{7.020637408568443e3}/0.68
                \cong
                \num{20.648933554613068}
                \implies &\\[3ex]&
                \implies
                N_R
                \cong
                \left\lceil
                    \frac{
                        1.15*\num{20.648933554613068}
                    }{
                        (\rho_A/M_A)
                    }
                \right\rceil
                = \left\lceil
                    \frac{
                        1.15*\num{20.648933554613068}
                    }{
                        5*(0.791*10^{3}/58)
                    }
                \right\rceil
                \cong &\\&
                \cong
                \ceil{\num{0.348238651856559}}
                = 1
            &
        \end{flalign*}
    \end{questionBox}
\end{questionBox}

\end{document}
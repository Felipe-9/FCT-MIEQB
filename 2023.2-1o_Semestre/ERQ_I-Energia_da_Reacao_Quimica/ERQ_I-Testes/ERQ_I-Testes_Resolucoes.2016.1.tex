% !TEX root = ./ERQ_I-Testes_Resolucoes.2016.1.tex
\providecommand\mainfilename{"./ERQ_I-Testes_Resolucoes.tex"}
\providecommand \subfilename{}
\renewcommand   \subfilename{"./ERQ_I-Testes_Resolucoes.2016.1.tex"}
\documentclass[\mainfilename]{subfiles}

% \tikzset{external/force remake=true} % - remake all

\begin{document}

% \graphicspath{{\subfix{./.build/figures/ERQ_I-Testes_Resolucoes.2016.1}}}
% \tikzsetexternalprefix{./.build/figures/ERQ_I-Testes_Resolucoes.2016.1/graphics/}

\mymakesubfile{1}
[ERQ I]
{Teste 1 2016 Resolução} % Subfile Title
{Teste 1 2016 Resolução} % Part Title

\begin{questionBox}1{ % Q1
    A reação reversível \ch{A <> B} é conduzida numa bateria de dois reactores CSTR iguais. O 1º Teste reagente A é alimentado à bateria de reactores numa concentração de 1\,\unit{\M}, a um caudal volumétrico de 10\,\unit{\litre/\min}. As reacções directa e inversa são elementares e os valores da constante cinética da reacção directa e da conversão de equilíbrio são respectivamente 0.05\,\unit{\min^{-1}} e 96\%.
} % Q1
    \begin{questionBox}2{ % Q1.1
        Deduza a expressão da lei cinética.
    } % Q1.1
        \answer{}
        \begin{flalign*}
            &
                -r_{A}
                = k\,(C_{A}-C_{B}/k_e)
                = &\\&
                = k\,(C_{A\,0}(1-X)-C_{A\,0}\,X/k_e)
                = &\\&
                = k\,C_{A\,0}\,(
                    1 - X\,(1+1/k_e)
                )
            &
        \end{flalign*}
    \end{questionBox}
    \begin{questionBox}2{ % Q1.2
        Para cada um dos reactores deduza as expressões que relacionam o volume do reactor com a conversão.
    } % Q1.2
        \answer{}
        \begin{flalign*}
            &
                V_i(X)
                : 0
                = F_{A\,i\,i}
                - F_{A\,i\,o}
                + r_{A\,i}\,V_i
                \implies &\\[6ex]&
                % Reator 2
                i=1
                \implies &\\&
                \implies 
                0
                = F_{A\,1\,i}
                - F_{A\,1\,o}
                + r_{A\,1}\,V_1
                = &\\&
                = F_{A\,1\,i}
                - F_{A\,1\,i}(1-X_1)
                + (
                    - k\,C_{A\,1\,i}\,(
                        1 - X_1\,(1+1/k_e)
                    )
                )\,V_1
                = &\\&
                = F_{A\,1\,i}\,X_1
                - k\,C_{A\,1\,i}\,(
                    1 - X\,(1+1/k_e)
                )\,V_1
                = &\\&
                = v_{1\,i}\,C_{A\,1\,i}\,X_1
                - k\,C_{A\,1\,i}\,(
                    1 - X\,(1+1/k_e)
                )\,V_1
                \implies &\\[3ex]&
                \implies
                V_1
                = \frac{
                    v_{A\,1\,i}
                    \,X_1
                }{
                    k\,(
                        1 - X_1\,(1+1/k_e)
                    )
                }
                % = &\\&
                = \frac{
                    v_{A\,1\,i}
                }{
                    k\,(
                        1/X_1 - 1 + 1/k_e
                    )
                }
                ;&\\[6ex]&
                % Reator 2
                i=2
                \implies &\\&
                \implies
                0 
                = F_{A\,2\,i}
                - F_{A\,2\,o}
                + r_{A\,2}\,V_2
                = &\\&
                = F_{A\,1\,o}
                - F_{A\,1\,i}\,(1-X_2)
                + (
                    k\,C_{A\,1\,i}\,(
                        1 - X_2\,(1+1/k_e)
                    )
                )\,V_2
                = &\\&
                = F_{A\,1\,i}\,(1-X_1)
                - F_{A\,1\,i}\,(1-X_2)
                + k\,C_{A\,1\,i}\,(
                    1 - X_2\,(1+1/k_e)
                )\,V_2
                = &\\&
                = F_{A\,1\,i}\,(X_2-X_1)
                + k\,C_{A\,1\,i}\,(
                    1 - X_2\,(1+1/k_e)
                )\,V_2
                = &\\&
                = C_{A\,1\,i}\,v_{1\,i}\,(X_2-X_1)
                + k\,C_{A\,1\,i}\,(
                    1 - X_2\,(1+1/k_e)
                )\,V_2
                \implies &\\[3ex]&
                \implies
                V_2
                = \frac{
                    v_{1\,i}\,(X_2-X_1)
                }{
                    k\,(
                        1 - X_2(1 + 1/k_e)
                    )
                }
            &
        \end{flalign*}
    \end{questionBox}
    \begin{questionBox}2{ % Q1.3
        Determine o valor da constante de equilíbrio.
    } % Q1.3
        \answer{}
        \begin{flalign*}
            &
                k_e
                = \frac{C_{B\,e}}{C_{A\,e}}
                = \frac{C_{A\,0}\,X_e}{C_{A\,0}(1-X_e)}
                = (1/X_e-1)^{-1}
                = (1/0.96-1)^{-1}
                = 24
            &
        \end{flalign*}
    \end{questionBox}
    \begin{questionBox}2{ % Q1.4
        Sabendo que a conversão à saída do 2º reactor corresponde a 90\% da conversão de equilíbrio, determine a conversão à saída do 1º reactor.
    } % Q1.4
        \answer{}
        \begin{flalign*}
            &
                X_1:
                V_2
                = \frac{
                    v_{1\,i}\,(X_2-X_1)
                }{
                    k\,(
                        1 - X_2(1 + 1/k_e)
                    )
                }
                = &\\&
                = V_1
                = \frac{
                    v_{1\,i}
                }{
                    k\,(
                        1/X_1 - 1 + 1/k_e
                    )
                }
                \implies &\\[3ex]&
                \implies
                0
                = \left(
                    \begin{aligned}
                        &
                            X_1^2\,(1 - 1/k_e)
                        &+\\+&
                            X_1\,2\,(X_2/k_e- 1)
                        &+\\+&
                            X_2
                        &
                    \end{aligned}
                \right)
                = &\\&
                = \left(
                    \begin{aligned}
                        &
                            X_1^2\,(1 - 1/24)
                        &+\\+&
                            X_1\,2\,(0.9*0.96/24 - 1)
                        &+\\+&
                            X_2
                        &
                    \end{aligned}
                \right)
                = &\\&
                = \left(
                    \begin{aligned}
                        &
                            X_1^2\,\num{0.9583333333}
                        &+\\-&
                            X_1\,1.928
                        &+\\+&
                            0.864
                        &
                    \end{aligned}
                \right)
                \begin{cases}
                    \num{1.4841091729140812}
                    \\\emph{\num{0.7473723085674003}}
                \end{cases}
            &
        \end{flalign*}
    \end{questionBox}
    \begin{questionBox}2{ % Q1.5
        Determine o volume dos reactores.
    } % Q1.5
        \answer{}
        \begin{flalign*}
            &
                V
                = \frac{
                    v_{1\,i}
                }{
                    k\,(
                        1/X_1 - 1 + 1/k_e
                    )
                }
                \cong \frac{
                    10
                }{
                    0.05\,(
                        1/\num{0.7473723085674003} - 1 + 1/24
                    )
                }
                \cong
                \qty{526.748449804993057}{\litre}
            &
        \end{flalign*}
    \end{questionBox}
\end{questionBox}

\begin{questionBox}1{ % Q2
    A reacção elementar, em fase gasosa, \ch{2 A -> 3 B + C} é conduzida à temperatura de 493\,\unit{\kelvin} e à pressão de 7\,\unit{\atm} num reactor PFR (\(k = 0.45\,\unit{\litre.\mole^{-1}.\second^{-1}}\)). Assumindo que o reagente A é alimentado puro ao reactor, a um caudal volumétrico de 15\,\unit{\litre/\second} e que se obtém uma conversão de 90\%, determine:
} % Q2
    \begin{BM}
        P\left(
            \frac{1+\varepsilon\,X}{1-X}
        \right)^2
        = \frac{(1+\varepsilon)^2}{1-X}
        - 2\,\varepsilon(1+\varepsilon)
        \,\ln(1-X)^{-1}
        +\varepsilon^2\,X
    \end{BM}
    \begin{questionBox}2{ % Q2.1
        O valor da velocidade de reacção à entrada do reactor.
    } % Q2.1
        \answer{}
        \centering{\ch{A -> 3/2 B + 1/2 C}}
        \begin{flalign*}
            &
                -r_{A\,0}:
                -r_A
                = k\,C_{A}^2
                = k\,\left(
                    \frac{F_{A}}{v}
                \right)^2
                = &\\&
                = k\,\left(
                    \frac{
                        C_{A\,0}(1-X)
                    }{
                        (1+\varepsilon\,X)
                        \,(T/T_0)
                        \,(P_0/P)
                    }
                \right)^2
                = &\\&
                = k\,\left(
                    \frac{
                        (P_{A\,0}/R\,T)(1-X)
                    }{
                        (1+\varepsilon\,X)
                        \,(1)
                        \,(1)
                    }
                \right)^2
                = &\\&
                = k
                \,\frac{
                    (P_{A\,0}/R\,T)^2
                    \,(1-X)^2
                }{
                    (1+\varepsilon\,X)^2
                }
                \implies &\\[3ex]&
                \implies
                -r_{A\,0}
                = k\,\left(
                    \frac{P_{A\,0}}{R\,T}
                \right)^2
                \cong
                0.45\,\left(
                    \frac{7}{
                        \num{0.082057366080960}
                        *493
                    }
                \right)^2
                \cong &\\&
                \cong
                \qty{1.3473474661319e-2}{\mole.\litre^{-1}.\second^{-1}}
            &
        \end{flalign*}
    \end{questionBox}
    \begin{questionBox}2{ % Q2.2
        O volume do reactor.
    } % Q2.2
        \answer{}
        \begin{flalign*}
            &
                V:
                \odif{V}
                =F_{A\,0}\,\frac{\odif{X}}{-r_A}
                =C_{A\,0}
                \,v_0
                \,\frac{\odif{X}}{
                    k
                    \,\frac{
                        C_{A\,0}^2
                        \,(1-X)^2
                    }{
                        (1+\varepsilon\,X)^2
                    }
                }
                ; &\\&
                \varepsilon 
                = y_{A\,0}\,\delta
                = - 1 + 3/2 + 1/2 = 1
                \implies &\\[3ex]&
                \implies
                \int_{0}^{V}{\odif{V}}
                =V
                = &\\&
                = 
                \int_{0}^{X}{
                    \frac{
                        v_0
                    }{
                        k\,C_{A\,0}
                    }
                    \,\frac{
                        (1+\varepsilon\,X)^2
                    }{
                        \,(1-X)^2
                    }\,\odif{X}
                }
                = &\\&
                = 
                \frac{
                    v_0
                }{
                    k\,C_{A\,0}
                }
                \adif{\left(
                    \frac{(1+\varepsilon)^2}{1-X}
                    - 2\,\varepsilon(1+\varepsilon)
                    \,\ln(1-X)^{-1}
                    +\varepsilon^2\,X
                \right)}
                \Bigg\vert_{0}^{X}
                = &\\&
                = 
                \frac{
                    v_0
                }{
                    k\,C_{A\,0}
                }
                \left(
                    % X
                    \frac{(1+\varepsilon)^2}{1-X}
                    - 2\,\varepsilon(1+\varepsilon)
                    \,\ln(1-X)^{-1}
                    +\varepsilon^2\,X
                    % 0
                    - (1+\varepsilon)^2
                \right)
                = &\\&
                = 
                \frac{
                    v_0
                }{
                    k\,C_{A\,0}
                }
                \left(
                    (1+\varepsilon)^2
                    \frac{1}{1/X-1}
                    - 2\,\varepsilon(1+\varepsilon)
                    \,\ln(1-X)^{-1}
                    +\varepsilon^2\,X
                \right)
                \cong &\\&
                \cong 
                \frac{
                    15
                }{
                    0.45
                    *\num{0.173034836963345}
                }
                \left(
                    4
                    \frac{1}{1/0.9-1}
                    % 36
                    - 4
                    \,\ln(1-0.9)^{-1}
                    % 9.210340371976183
                    +0.9
                \right)
                % 27.689659628023817
                \cong &\\&
                \cong
                \qty{5334.12040295089687}{\litre}
            &
        \end{flalign*}
    \end{questionBox}
    \begin{questionBox}2{ % Q2.3
        O valor do caudal volumétrico à saída do reactor.
    } % Q2.3
        \answer{}
        \begin{flalign*}
            &
                v
                =v_0\,(1+\varepsilon\,X)
                =15\,(1+0.9)
                =2.85\,\unit{\litre/\second}
            &
        \end{flalign*}
    \end{questionBox}
    \begin{questionBox}2{ % Q2.4
        O valor do caudal molar do produto B, à saída do reactor.
    } % Q2.4
        \answer{}
        \begin{flalign*}
            &
                F_B
                = F_{A\,0}\,X\,3/2\
                = &\\&
                = C_{A\,0}\,v_0\,X\,3/2
                \cong &\\&
                \cong
                \num{0.173034836963345}
                * 15
                * 0.9 * 3/2
                \cong &\\&
                \cong
                \qty{3.503955448507736}{\mole/\second}
            &
        \end{flalign*}
    \end{questionBox}
    \begin{questionBox}2{ % Q2.5
        Caso a reacção seja conduzida num reactor batch, a volume constante, nas mesmas condições de temperatura e pressão inicial, determine o valor da pressão à conversão de 90\%
    } % Q2.5
        \answer{}
        \begin{flalign*}
            &
                P:
                % Eq geral de reações gasosas
                \frac{P_0}{P}
                \,\frac{T}{T_0}
                (1+\varepsilon\,X)
                = \frac{P_0}{P}
                (1+\varepsilon\,X)
                = &\\&
                = \frac{V}{V_0}
                = 1
                \implies &\\[3ex]&
                \implies
                P
                = P_0(1+\varepsilon\,X)
                = 7\,(1+0.9)
                = 13.3\,\unit{\atm}
            &
        \end{flalign*}
    \end{questionBox}
\end{questionBox}

\end{document}
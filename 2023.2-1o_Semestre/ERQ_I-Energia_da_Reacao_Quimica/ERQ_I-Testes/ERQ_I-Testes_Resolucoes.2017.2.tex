% !TEX root = ./ERQ_I-Testes_Resolucoes.2017.2.tex
\providecommand\mainfilename{"./ERQ_I-Testes_Resolucoes.tex"}
\providecommand \subfilename{}
\renewcommand   \subfilename{"./ERQ_I-Testes_Resolucoes.2017.2.tex"}
\documentclass[\mainfilename]{subfiles}

% \tikzset{external/force remake=true} % - remake all

\begin{document}

\graphicspath{{\subfix{./.build/figures/ERQ_I-Testes_Resolucoes.2017.2}}}
% \tikzsetexternalprefix{./.build/figures/ERQ_I-Testes_Resolucoes.2017.2/graphics/}

\mymakesubfile{2}
[ERQ\,I]
{Teste 2017.2 Resolução} % Subfile Title
{Teste 2017.2 Resolução} % Part Title



\begin{questionBox}1{ % Q1
    A reacção reversível \ch{A<>B} é conduzida, na fase gasosa, num reactor tubular adiabático. O reagente A (30\%) e um inerte são alimentados, à temperatura de 773\,\si{\kelvin}, a um caudal volumétrico de 100\,\si{\litre/\min}. A figura representa a variação da conversão de equilíbrio com a temperatura.
    \begin{figure}\centering
        \includegraphics[width=.6\textwidth]{test1-cutout.png}
        % Curva X\times T
    \end{figure}
} % Q1
    \paragraph*{Dados:}
    \begin{itemize}
        \vspace{-3ex}
        \begin{multicols}{2}
            \item \(C_{p\,A}=C_{p\,B}=10\,\si{\calorie/\mole.\kelvin}\)
            \item \(C_{p\,I}=12\,\si{\calorie/\mole.\kelvin}\)
            \item \(K_{e\,(773\,\si{\kelvin})}=30\)
            \item \(Ea=25\,\si{\kilo\calorie/\mole}\)
            \item \(R=\SI{1.987204258640832}{\calorie.\mole^{-1}.\kelvin^{-1}}\)
            \item \(\adif{H_R}=20\,\si{\kilo\calorie/\mole}\)
        \end{multicols}
        \item Constante cinética da reação direta \(k_{(773\,\si{\kelvin})}=8.57\,\si{\min^{-1}}\)
    \end{itemize}
    \begin{questionBox}2{ % Q1.1
        Determine, usando o gráfico, o valor do calor de reação
    } % Q1.1
        \answer{}
        % Calor da reação     => Lei de Vant Hoff
        % Energia de ativação => Lei de Arrhenius
        \begin{flalign*}
            &
                \adif{H}:
                k_{e\,(T)}
                =k_{e\,(T_R)}
                \,\exp{\left(
                    -\frac{\adif{H}}{R}
                    \,(T^{-1}-T_R^{-1})
                \right)}
                &\\[3ex]&
                % Grafico X\times t, arranjar forma de relacionar X com k_e
                % (reação reversivel)
                k_e
                =\frac{\sum{p_{1\,E}}}{\sum{p_{0\,E}}}
                =\frac{p_{B\,E}}{p_{A\,E}}
                = &\\&
                % por ser fase gasosa usa gases perfeitos
                = \frac{C_{B\,E}\,R\,T}{C_{A\,E}\,R\,T}
                = \frac{C_{A\,0}\,X}{C_{A\,0}(1-X)}
                = \frac{1}{-1+1/X}
                \implies &\\&
                \implies
                X = \frac{1}{1+1/K_e}
                \implies
                X=0.5
                \begin{cases}
                    K_e=1\\T\cong 610\,\si{\kelvin}
                \end{cases}
                % Pega o valor de da temperatura que X=0.5=>k_e=1=>t=(grafico)
                % Com os dois valordes de ke e t aplica a lei de vant hoff e retira o Delta H_r
                \implies &\\[3ex]&
                \implies
                \adif{H}
                = -\frac{R}{(610^{-1}-773^{-1})}
                \,\ln\frac{k_{e\,(610\,\si{\kelvin})}}{k_{e\,(773\,\si{\kelvin})}}
                \cong &\\&
                \cong -\frac{\num{1.987204258640832}}{(610^{-1}-773^{-1})}
                \,\ln\frac{1}{30}
                \cong
                \SI{19.552219755329109539}{\kilo\calorie/\mole}
                % Caso n seja dado ke faz esse processo duas vezes
            &
        \end{flalign*}
    \end{questionBox}
    \begin{questionBox}2{ % Q1.2
        Determine a conversão de equilíbrio e a correspondente temperatura de equilíbrio.
        % X_{eq}; T_{eq}
    } % Q1.2
        \answer{}
        \begin{flalign*}
            &
                % T0=773k=>X=0
                % Escolhe outra temperatura qualquer
                % T1=520k=>X=?
                X_{1\,(520\,\si{\kelvin})}
                =\frac{C_{p\,A}+\theta_{I}\,C_{p\,I}(T-T_0)}{-\adif{H_R}}
                =\frac{C_{p\,A}+\frac{Y_{I\,0}}{Y_{A\,0}}\,C_{p\,I}(T-T_0)}{-\adif{H_R}}
                \cong &\\&
                \cong\frac{(10+\frac{0.7}{0.3}\,12)(520-773)}{-\num{19.552219755329109539e3}}
                \cong\num{0.491708876041025}
                &\\[3ex]&
                \begin{cases}
                    T_0=773\,\si{\kelvin};& X_0=0
                    \\
                    T_1=520\,\si{\kelvin};& X_1=\num{0.491708876041025}
                \end{cases}
                % Junta os dados com o theta e a eq acima e acha o segundo ponto, traça no gráfico interecptando a reta e esse ponto é o Xeq e o Teq
            &
        \end{flalign*}
        \begin{figure}\centering
            \includegraphics[width=.6\textwidth]{test1-cutout2.png}
        \end{figure}
        \begin{BM}
            X_{eq}\cong 0.29
            \land
            T_{eq}\cong 580\,\si{\kelvin}
        \end{BM}

    \end{questionBox}
    \begin{questionBox}2{ % Q1.3
        Calcule o volume do reactor, necessário a uma conversão de 95\% da conversão de equilíbrio.
        % Vol do reator para \(X=95\%\,X_{eq}\)
    } % Q1.3
        \answer{}
        \begin{flalign*}
            &
                V:
                % X
                % =.95\,X_{eq}
                % \cong .95*2.9
                % =2.755;
                % &\\[3ex]&
                V
                =F_{A\,0}
                \,\int_{0}^{X}{\frac{\odif{X}}{-r_A}}
                =C_{A\,0}\,v_0
                \,\int_{0}^{X}{\frac{\odif{X}}{-r_A}}
                ; &\\[3ex]&
                % faz a lei cinética usando a constante de eq
                -r_A
                =k(C_A-C_B/K_e)
                = &\\&
                =k\left(
                    \left(
                        \frac{C_{A\,0}(1-X)}{1+\varepsilon\,X}
                        \frac{T_0}{T}
                    \right)
                    -\left(
                        \frac{C_{A\,0}\,X}{1+\varepsilon\,X}
                        \frac{T_0}{T}
                    \right)
                    /K_e
                \right)
                = &\\&
                = k\,\left(
                    \frac{
                        C_{A\,0}(1-X(1-1/K_e))
                    }{1+\varepsilon\,X}
                    \frac{T_0}{T}
                \right)
                = &\\&
                = k\,\left(
                    \frac{
                        C_{A\,0}(1-X(1-1/K_e))
                    }{1+y_{A\,0}\,\delta\,X}
                    \frac{T_0}{T}
                \right)
                = &\\&
                = k\,\left(
                    \frac{
                        C_{A\,0}(1-X(1-1/K_e))
                    }{1+y_{A\,0}\,(1-1)\,X}
                    \frac{T_0}{T}
                \right)
                = &\\&
                = k\,C_{A\,0}(1-X(1-K_e))
                \frac{T_0}{T}
                \implies &\\[3ex]&
                \implies
                V
                =\int_0^{.95*.29}{
                    \frac{v_0}{k\,\frac{T_0}{T}(1-X\,(1-1/K_e))}
                    \,\odif{X}
                }
                = &\\&
                =\int_0^{0.2755}{
                    \frac{100}{k\,\frac{773}{T}(1-X\,(1-1/K_e))}
                    \,\odif{X}
                }
                % Integral dificil de calcular usa lei de simpson
                % x=> pela lei de arrhenius 3 pontos
                % t=>pelo balanço (alinha anterior) 3 pontos
                % k_e=>lei de arrhenius em 3 pontos
                % encontra f(X_i) para os 3 pontos e substitui na formula
                ; &\\[3ex]&
                % Balanço de energia
                {\color{Emph}T}:
                X
                =\frac{(C_{p\,A}+\theta_I\,C_{p\,I})(T-T_0)}{-\adif{H_R}}
                \implies &\\&
                \implies 
                T
                =T_0-\frac
                    {X\,\adif{H_R}}
                    {C_{p\,A}+\theta_I\,C_{p\,I}}
                \cong
                773
                - \frac
                    {X\,\num{19.552219755329109539e3}}
                    {10+\frac{0.7}{0.3}\,12}
                \cong &\\&
                \cong
                    {\color{Emph}
                        773-X\,\num{5.1453209882445e2}
                    }
                ; &\\[3ex]&
                % Arrhenios
                {\color{Emph}
                    k_{(T)}
                }
                =k_{(T_R)}
                \,\exp{\left(
                    -\frac{Ea}{R}(T^{-1}-T_R^{-1})
                \right)}
                \cong &\\&
                \cong
                8.57
                \,\exp{\left(
                    -\frac{25\E{3}}{\num{1.987204258640832}}
                    (T^{-1}-773^{-1})
                \right)}
                \cong &\\&
                \cong {\color{Emph}
                    8.57
                    \,\exp{\left(
                        -\num{12.580488337469142e3}
                        (T^{-1}-773^{-1})
                    \right)}
                }
                ; &\\[3ex]&
                % Vant Hoff
                {\color{Emph}
                    K_{e\,(T)}
                }
                = K_{e\,(T_R)}
                \,\exp{\left(
                    -\frac{\adif{H}}{R}
                    \,(T^{-1}-T_R^{-1})
                \right)}
                \cong &\\&
                \cong 
                30\,\exp{\left(
                    -\frac
                        {\num{19.552219755329109539e3}}
                        {\num{1.987204258640832}}
                    \,(T^{-1}-773^{-1})
                \right)}
                \cong &\\&
                \cong {\color{Emph}
                    30\,\exp{\left(
                        -\num{9.839058904142065e3}
                        \,(T^{-1}-773^{-1})
                    \right)}
                }
                ; &\\[3ex]&
                h
                =\frac{X_1-X_0}{2}
                =\frac{0.2755}{2}
                =0.13775
                ; \qquad
                X_1=0.13775
            &
        \end{flalign*}
        \begin{center}
            \vspace{1ex}
            \begin{tabular}{*{5}{C}}
                \toprule
                
                    X & T & k & K_e & f_{(X)}
                    % Usa tabela para calcular os Fs de X para aplicar no simpson
                
                \\\midrule
                    % T:   5.1453209882445e2
                    % k:  12.580488337469142e3
                    % k_e: 9.839058904142065e3
                
                    0 & 773 & 8.57 & 30 & \num{11.668611435239207}
                    \\
                    0.13775 
                    & \num{702.123203386932013}
                    & \num{44.307713036601854}
                    & \num{108.427749331313463}
                    \\
                    \dots
                
                \\\bottomrule
            \end{tabular}
            \vspace{2ex}
        \end{center}
        \begin{flalign*}
            &
                V
                =\frac{h}{3}\left(
                    f_{(X_0)}
                    + 4\,f_{(X_1)}
                    + f_{(X_2)}
                \right)
                = &\\&
                =\frac{0.13775}{3}\left(
                    f_{(X_0)}
                    + 4\,f_{(X_1)}
                    + f_{(X_2)}
                \right)
                =\dots
            &
        \end{flalign*}
    \end{questionBox}
\end{questionBox}

\begin{questionBox}1{ % Q2
    A reacção elementar em fase líquida, \ch{A -> B}, é conduzida num reactor CSTR adiabático, de 1\,\si{\metre^3}, a funcionar em estado estacionário. A alimentação ao reactor, a um caudal volumétrico de 20\,\si{\litre/\min} é constituída por A (10\,\si{\mole\percent}) e um inerte I. A figura mostra a curva de geração de calor.
    \begin{figure}\centering
        \includegraphics[width=.6\textwidth]{test2.png}
        % Grafico G_{(T)}\times t
    \end{figure}
} % Q2
    \paragraph*{Dados:}
    \begin{itemize}
        \begin{multicols}{2}
            \vspace{-2ex}
            \item \(\adif{H_R}=-25\,\si{\kilo\calorie/\mole}\)
            \item \(C_{p\,A}=C_{p\,B}=8\,\si{\calorie/\mole.\kelvin}\)
            \item \(C_{p\,I}=18\,\si{\calorie/\,mole.\kelvin}\)
            \item \(R=\SI{1.987204258640832}{\calorie.\mole^{-1}.\kelvin^{-1}}\)
        \end{multicols}
    \end{itemize}
    Determine:

    \begin{questionBox}2{ % Q2.1
        O valor da temperatura da corrente de saída, correspondente a uma conversão de 90\%
    } % Q2.1
        \answer{}
        % Gt=>Calor gerado
        % Grafico G(t)\times temperatura
        \begin{flalign*}
            &
                T:
                G_{(T)}
                =-\adif{H_{R\,T}}\,X
                =25\E{3}*.90
                \cong
                \SI{22.5e3}{\calorie/\mole}
                % Busca o t no gráfico a partir do G_{(T)}
                \implies &\\&
                \implies
                T_{(22.5\E{3})}\cong 370\,\si{\kelvin}
            &
        \end{flalign*}
    \end{questionBox}
    \begin{questionBox}2{ % Q2.2
        O valor da temperatura da alimentação, nas condições da alínea a).
    } % Q2.2
        \begin{flalign*}
            &
                T_0:
                R_{(T)}
                =\left(
                    C_{p\,A}+\theta_I\,C_{p\,I}
                \right)
                (T-T_0)
                = G_{(T)}
                \implies &\\&
                \implies 
                T_0
                = T-\frac{G_{(T)}}{C_{p\,A}+\theta_I\,C_{p\,I}}
                \cong
                370
                -\frac{22.5\E{3}}{8+\frac{0.9}{0.1}\,18}
                \cong
                \SI{237.647058823529412}{\kelvin}
                % Estado estácionário: G_{(T)}=R_{(T)} 
            &
        \end{flalign*}
    \end{questionBox}
    \begin{questionBox}2{ % Q2.3
        Os valores das temperaturas de ignição e extinção.
    } % Q2.3
        \answer{}
        \begin{flalign*}
            &
                \begin{cases}
                    T_0\cong\SI{237.647058823529412}{\kelvin}
                    ;& G_{(T_0)}=0
                    \\
                    T_1\cong370\,\si{\kelvin}
                    ;& G_{(T_1)}=22.5\E{3}
                \end{cases}
                &\\&
                G_{(T)}
                =m\,T+b
                =m\,T+(-m\,T_0)
                =m\,(T-T_0)
                = &\\&
                =\adv{22.5\E3}{370-\num{237.647058823529412}}\,(T-\num{237.647058823529412})
                =170\,T-40400
                \implies &\\&
                \implies
                \begin{cases}
                    T_2=250;& G_{(T_2)}=2100
                \end{cases}
            &
        \end{flalign*}
        \begin{figure}\centering
            \includegraphics[width=.6\textwidth]{test2.1.png}
            % Une no grafico uma reta do t entrada e t de saída
            % traça paralelamente para tangenciar a curva por cima e baixo,
            % a tangente inferior vai ser a t de ignição e a superior a de extinção
        \end{figure}
        \begin{BM}
            T_{\text{ignição}}\cong 319\,\si{\kelvin}
            \quad\land\quad
            T_{\text{extinção}}\cong 370\,\si{\kelvin}
        \end{BM}
    \end{questionBox}
    \begin{questionBox}2{ % Q2.4
        A composição da alimentação, nas condições da alínea a), para uma temperatura da alimentação de 298\,\si{\kelvin}.
    } % Q2.4
        \begin{flalign*}
            &
                Y_{A\,0}:
                R_{(T)}
                = G_{(T)}
                = \left(
                    C_{p\,A}+\theta_I\,C_{p\,I}
                \right)
                (T-T_0)
                = &\\&
                = \left(
                    C_{p\,A}
                    +\frac{1-Y_{A\,0}}{Y_{A\,0}}
                    \,C_{p\,I}
                \right)
                (T-T_0)
                \implies &\\&
                \implies
                Y_{A\,0}
                =
                \left(
                    1+\frac{
                        \frac{G_{(T)}}{T-T_0}
                        -C_{p\,A}
                    }{
                        C_{p\,I}
                    }
                \right)^{-1}
                = \left(
                    1+\frac{
                        \frac{22.5\E{3}}{370-298}
                        -8
                    }{
                        18
                    }
                \right)^{-1}
                \cong &\\&
                \cong
                \num{0.055813953488372}
            &
        \end{flalign*}
    \end{questionBox}
\end{questionBox}

\end{document}
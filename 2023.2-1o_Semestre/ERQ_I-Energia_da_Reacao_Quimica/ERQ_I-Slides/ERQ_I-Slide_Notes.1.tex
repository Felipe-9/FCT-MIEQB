% !TEX root = ./ERQ_I-Slides_Anotacoes.1.tex
\providecommand\mainfilename{"./ERQ_I-Slides_Anotacoes.tex"}
\providecommand \subfilename{}
\renewcommand   \subfilename{"./ERQ_I-Slides_Anotacoes.1.tex"}
\documentclass[\mainfilename]{subfiles}

% \tikzset{external/force remake=true} % - remake all

\begin{document}

\graphicspath{{\subfix{./.build/figures/ERQ_I-Slides_Anotacoes.1}}}
% \tikzsetexternalprefix{./.build/figures/ERQ_I-Slides_Anotacoes.1/graphics/}

\mymakesubfile{1}
[ERQ I]
{Slide: Chemical Reactors} % Subfile Title
{Slide: Chemical Reactors} % Part Title

\begin{sectionBox}*{} % S
    
    \begin{itemize}
        \item Ideal reactors dimensioning
        \item Isothermic and non-isothermic operations
        \item Catalysts preparation and characterization
    \end{itemize}
    
\end{sectionBox}


\begin{sectionBox}1{General equaltion of molar balance} % S
    
    \begin{center}\large
        \ch{
            a A + b B + \dots -> d D + e E + \dots
        }
    \end{center}
    \begin{BM}
        F_{i\,0} - F_{i\,1} + F_{i\,P} = \odv{N_i}{t}
    \end{BM}
    \paragraph*{Molar balance to the limitant reagent \ch{A}}
    \begin{description}[
        leftmargin=!,
        labelwidth=\widthof{\(\adif{F}/\si{\mole\of{A}}\)} % Longest item
    ]
        \begin{multicols}{2}
            \item[\(F_{A\,0}/\si{\mole\of{A}/\litre\of{0}.\hour}\)] input
            \item[\(F_{A\,1}/\si{\mole\of{A}/\litre\of{1}.\hour}\)] output
            \item[\(F_{A\,P}/\si{\mole\of{A}/\litre.\hour}\)] produced
            \item[\(\odv{N_A}{t}/\si{\mole\of{A}/\litre.\hour}\)] Acumulated
        \end{multicols}
    \end{description}
    
\end{sectionBox}

\begin{sectionBox}1{Reaction rate} % S
    
    \begin{BM}
        r_{A}= V^{-1}\,\odv{N_A}{t}
        \qquad
        r'_{A}= W^{-1}\,\odv{N_A}{t}
        \qquad
        r"_{A}= S^{-1}\,\odv{N_A}{t}
        \\
        F_{A\,P}=\int_V{r_A\,\odif{V}}
    \end{BM}

\end{sectionBox}

\begin{sectionBox}1{Ideal reactors} % S

    \begin{multicols}{2}
        
        \begin{sectionBox}*2{\emph{CSTR}: Continuous stirred-tank reactor} % S
            
            Homogenous mix trought \emph{every point} of the reactor
            \begin{figure}\centering
                \includegraphics[width=1\textwidth]{Cstr.png}
            \end{figure}
            
        \end{sectionBox}
    
        \begin{sectionBox}*2{\emph{PFR}: Plug flow reactor model} % S
            
            Homogeneous mix trought a \emph{disk section}
            \begin{figure}\centering
                \includegraphics[width=1\textwidth]{Pipe-PFR.svg.png}
            \end{figure}
            
        \end{sectionBox}
    \end{multicols}

    \paragraph*{Batch:} Similar to CSTR but descontinous (no input/output)
    
\end{sectionBox}

\begin{sectionBox}1{How to balance each reactor} % S
    
    \paragraph*{General equation}
    \begin{BM}
        F_{i\,0}-F_{i\,1}+F_{i\,P}=\odv{N_i}{t}
    \end{BM}
    \paragraph*{\emph{X}: Conversion}
    \begin{BM}
        X
        =1-N_{i\,1}/N_{i\,0}
        =1-F_{i\,1}/F_{i\,0}
    \end{BM}

    \subsection{Continuous reactors}
    \paragraph*{Steady State:}
    Maintaining constant all functioning conditions (input current, temperature, pressure, \dots), after a certain time the reactor reaches a steady state where all the output parameters (caudal, concentration, temperature, \dots) become constant in time.
    \paragraph*{Spatial time:}
    \begin{BM}
        \tau=\frac{V\,\text{(reactor volume)}}{v\,\text{(volumetric caudal)}}
    \end{BM}
    
\end{sectionBox}

\begin{sectionBox}2{Batch balance} % S
    
    \begin{figure}\centering
        \includegraphics[height=15ex]{batch.png}
    \end{figure}

    \begin{BM}
        r_i\,V=\odv{N_i}{t}
        \iff
        t=C_{i\,0}\,\int_{0}^{X}{\frac{\odif{X}}{-r_A}}
        ;\\[2ex]
        \left(\small
            \begin{aligned}
                F_{i\,0} = F_{i\,1} = 0
                &\,\text{(No input/output)}
                \\
                \int_V{r_i\,\odif{V}}=r_i\,V
                &\,\text{(Continously agitated)}
            \end{aligned}
        \right)
    \end{BM}
    \begin{flalign*}
        &
            F_{i\,0}+F_{i\,1}+F_{i\,P}
            = F_{i\,P}
            = \int_V\,r_i\,\odif{V}
            = r_i\int_V\,\odif{V}
            = {\color{Emph}
                r_i\,V
                = \odv{N_i}{t}
            }
            ; &\\[3ex]&
            X=1-N_{i\,1}/N_{i\,0}
            \implies
            N_{i\,1}=N_{i\,0}\,(1-X)
            \implies &\\[3ex]&
            \implies
            \odv{N_i}{t}
            = \odv{N_{i\,0}\,(1-X)}{t}
            = -N_{i\,0}\,\odv{X}{t}
            r_i\,V
            \implies &\\&
            \implies
            \int_0^t{\odif{t}}
            = {\color{Emph}t}
            = \int_0^X{
                \frac
                    {-N_{i\,0}}
                    {r_i\,V}
                \,\odif{X}
            }
            = \frac{N_{i\,0}}{V}
            \int_0^X{
                \frac
                    {\odif{X}}
                    {-r_i}
            }
            = {\color{Emph}
                C_{i\,0}
                \int_0^X{
                    \frac
                        {\odif{X}}
                        {-r_i}
                }
            }
        &
    \end{flalign*}
\end{sectionBox}

\begin{sectionBox}2{CSTR balance} % S
    
    \begin{figure}\centering
        \includegraphics[height=15ex]{cstr2.png}
    \end{figure}
    \begin{BM}
        \tau=C_{i\,0}\,\frac{X}{-r_i}
        ; \\[2ex]
        \left(\small
            \begin{aligned}
                \odv{N_i}{t}=0
                &\,\text{(Steady state)}
                \\
                \int_V{r_A\,\odif{V}}=r_A\,V
                &\,\text{(Continously agitated)}
            \end{aligned}
        \right)
    \end{BM}
    \begin{flalign*}
        &
            F_{i\,0}-F_{i\,1}+F_{i\,P}
            = F_{i\,0}-(F_{i\,0}(1-X))+(r_i\,V)
            = F_{i\,0}\,X+r_i\,V
            = (v_0\,C_{i\,0})\,X+r_i\,V
            = &\\&
            = \odv{N_A}{t}
            = 0
            % \implies &\\&
            \implies
            \frac{V}{v_0}
            ={\color{Emph}
                \tau
                =C_{i\,0}\,\frac{X}{-r_i}
            }
        &
    \end{flalign*}
    
\end{sectionBox}

\begin{sectionBox}2{PFR Balance} % S
    
    \begin{figure}\centering
        \includegraphics[height=15ex]{pfr.png}
    \end{figure}
    \begin{BM}
        \tau=C_{i\,0}\int_{0}^{X}{\frac{\odif{X}}{-r_A}}
        ; \\[2ex]
        \left(
            \begin{aligned}
                \text{steady state elementar disk}
            \end{aligned}
        \right)
    \end{BM}
    \begin{flalign*}
        &
            F_{i\,j}-F_{i\,(j+1)}+F_{i}
            = F_{i\,j}-(F_{i\,j}+\odif{F_{i\,j}})+(r_{i}\,\odif{V})
            = -\odif{F_{i\,j}}+r_{i}\,\odif{V}
            = &\\&
            = -\odif{F_{i\,0}(1-X)}+r_{i}\,\odif{V}
            = F_{i\,0}\,\odif{X}+r_{i}\,\odif{V}
            = {\color{Emph}
                (C_{i\,0}\,v_0)\,\odif{X}+r_{i}\,\odif{V}
            }
            = &\\&
            % 
            =\odv{N_i}{t}
            ={\color{Emph}
                0
            }
            \implies &\\&
            \implies
            \frac{1}{v_0}\int_0^V{\odif{V}}
            \frac{V}{v_0}
            = {\color{Emph}
                \tau
            }
            = \int_0^X{
                C_{i\,0}
                \frac{\odif{X}}{-r_i}
            }
            = {\color{Emph}
                C_{i\,0}\int_0^X{
                    \frac{\odif{X}}{-r_i}
                }
            }
        &
    \end{flalign*}
    
\end{sectionBox}

\end{document}
% !TEX root = ./ERQ_I-Exercicios_Resolucoes.1.tex
\providecommand\mainfilename{"./ERQ_I-Exercicios_Resolucoes.tex"}
\providecommand \subfilename{}
\renewcommand   \subfilename{"./ERQ_I-Exercicios_Resolucoes.1.tex"}
\documentclass[\mainfilename]{subfiles}

% \tikzset{external/force remake=true} % - remake all

\begin{document}

\graphicspath{{\subfix{./.build/figures/ERQ_I-Exercicios_Resolucoes.1}}}
% \tikzsetexternalprefix{./.build/figures/ERQ_I-Exercicios_Resolucoes.1/graphics/}

\mymakesubfile{1}
[ERQ I]
{Exercicios} % Subfile Title
{Exercicios} % Part Title

\begin{questionBox}1{ % Q1
    A figura mostra a variação de \(-r_a^{-1}\) com \textit{X}, para uma reação isotérmica:
    \vspace{2ex}
    \begin{center}
        \Large\ch{A -> B}
    \end{center}
    \begin{figure}\centering
        \includegraphics[width=.8\textwidth]{Screenshot 2023-09-25 at 15.37.10}
    \end{figure}
    Consiste um sistema de reatores em que um CSTR e um PFR estão associados em série:
    \begin{figure}\centering
        \includegraphics[width=.8\textwidth]{Screenshot 2023-09-25 at 15.44.14}
    \end{figure}
    Admitindo que o reagente A é alimentado a um caudal volumétrico de 5\,\unit{\metre^3/\hour}, a uma concentração de 0.001\,\unit{\molar}, determine os volumes de ambos os reatores:/
} % Q1
\end{questionBox}

\end{document}
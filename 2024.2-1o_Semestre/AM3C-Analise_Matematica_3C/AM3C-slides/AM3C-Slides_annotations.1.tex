% !TEX root = ./AM3C-Slides_annotations.1.tex
\documentclass["AM3C-Slides_annotations.tex"]{subfiles}

% \tikzset{external/force remake=true} % - remake all

\begin{document}

% \graphicspath{{\subfix{./.build/figures/AM3C-Slides_annotations.1}}}
% \tikzsetexternalprefix{./.build/figures/AM3C-Slides_annotations.1/graphics/}

\mymakesubfile{1}[AM 3C]
{Equações Diferenciais Ordinárias (EDO)} % Subfile Title
{Equações Diferenciais Ordinárias (EDO)} % Part Title

\begin{sectionBox}1{EDO de Primeira Ordem} % S1
  \begin{BM}
    F(x,y(x),y'(x)) = 0
  \end{BM} 
  \textit{F} é definida num conjunto aberto \(D\subset\mathbb{R}^3\).
  Dado um intervalo aberto \(I\subset\mathbb{R}\), Diz-se que uma função \(\phi:I\to\mathbb{R}\) diferenciavel em \textit{I} é uma solução da equação diferencial acima se:
  \begin{enumerate}
    \item \((x,\phi(x),\phi'(x))\in D, \quad \forall\,x\in I\)
    \item \(F(x,\phi(x),\phi'(x)) = 0, \quad \forall\,x\in I\)
  \end{enumerate}

  \paragraph*{Ordem de uma equação diferencial} é a ordem da derivada mais elevada referida na equação

  \begin{multicols}{2}
    \begin{exampleBox}1{} % E1.1
      A equação
      \begin{BM}
        y' - \frac{y}{x} = x\,e^x
      \end{BM}
      é de primeira ordem e as funções
      \begin{BM}[align*]
        y(x) &= c\,x + x\,e^x & c\in\mathbb{R}
      \end{BM}
      são soluções em \(\myrange*{0,\infty}\) desta equação.
    \end{exampleBox}

    \begin{exampleBox}1{} % E1.2
      A equação
      \begin{BM}
        y" + 4\,y = 0
      \end{BM}
      é de segunda ordem e as funções
      \begin{BM}[align*]
        y(x) &= c_1\,\cos{2\,x} 
        \\   &+ c_2\,\sin{2\,x},
        \\ & c_1,c_2\in\mathbb{R}
      \end{BM}

      São soluções em \(\mathbb{R}\) desta equação
    \end{exampleBox}
  \end{multicols}

  \subsection*{Forma normal}
  \begin{BM}
    y'(x) = f(x,y(x))
  \end{BM}
  Com \textit{f} definida no conjunto aberto \(A\subset\mathbb{R}^2\).
  As equações de primeira ordem na forma normal admitem uma interpretação geométricas relativamente simples e que \emph{permite ter uma ideia aproxiamada dos gráficos das soluções} destas esquações.

  \subsection*{Campo de direções da equação}
  Com uma equação diferencial de primeira ordem na forma normal definida no conjunto aberto \(A\subset\mathbb{R}^2\), se a cada ponto \((x,y)\) de \textit{A} se associar a direção das retas de declive igual a \(f(x,y)\), se obtem aquilo a que usualmente se chama de campo de direções da equação.

  \begin{exampleBox}1{Campo de direções da equação} % E1.3
    \begin{BM}
      y'=-y
    \end{BM}
    \begin{center}
      % \tikzset{external/remake next=true}
      % \pgfplotsset{height=7cm, width= .6\textwidth}
      \begin{tikzpicture}
      \begin{axis}
        [
          % xmajorgrids = true,
          % legend pos={south east},
          xtick distance=1,
          domain=-3:3,
          y domain = -3:3,
          % ymin=-3.5,ymax=3.5,
          % xmin=-3.5,xmax=3.5,
          view={0}{90},
          % xlabel={},
          % ylabel={},
          % axis lines={center}, % left|right|center|box,
        ]

        % Plot from equation
        \addplot3[
          % opacity=1,
          opacity=0.2,
          Graph,
          samples = 12,
          quiver = {
            u={1},
            v={-y},
            scale arrows=0.25,
          },
          -stealth,
        ]{0};
      \end{axis}
      \end{tikzpicture}
    \end{center}
    
  \end{exampleBox}
\end{sectionBox}

\begin{sectionBox}1{Equação autonoma} % S2
  Uma EDO em que não aparece explicitamente a variável independente.
  Se for \textit{y} a função icógnita e \textit{x} a variável independente, uma equa;cão diferencial autónoma de primeira ordem é uma equação da forma \(F(y,y')=0\) ou na forma normal:
  \begin{BM}
    \odv{y}{x}=f(y)
  \end{BM}
  \paragraph*{Pontos de equilíbrio (críticos ou estacionários)} são os zeros da função 
  \begin{BM}
    f(c)=0 \implies y(x)=c \text{ é solução de } f(x)=\odv{y}{x}
  \end{BM}
  \(y(x)=c\) chama-se solução de equilíbrio (ou estacionária)


  \subsection*{Classificação dos pontos de equilíbrio (Eq autónomas)}
  Prestando atenção nos limites:
  \begin{BM}
    f(c)=0
  \end{BM}
  \begin{align*}
       x &\to +\infty          && \implies y(x) \to c \implies c \text{ é um ponto de eq estável}
    \\ x &\to -\infty          && \implies y(x) \to c \implies c \text{ é um ponto de eq instável}
    \\ x &\to -\infty \land x \to +\infty && \implies y(x) \to c \implies c \text{ é um ponto de eq semiestável}
  \end{align*}
\end{sectionBox}

\begin{exampleBox}1{Pontos de equilíbrio} % E4
  Considere-se a equação autónoma
  \begin{BM}
    \odv{y}{x}=y(a-b\,y); a,b\in\mathbb{R}^+
  \end{BM}
  Pontos de equilíbrio:
  \begin{gather*}
    c=y:
    y(a-b\,y) = 0
    \begin{cases}
      y=0\\ y=\frac{a}{b}
    \end{cases}
    \\
    \therefore y(x)=0\lor y(x)=a/b
    \end{gather*}
  Podemos prever o comportamento da equalção pela seguinte tabela
  \begin{center}
    \vspace{1ex}
    \def\-{\cellC{Graph33}{-}}
    \def\+{\cellC{Graph32}{+}}
    \begin{tabular}{*{4}{C}}
      \toprule

      \multirow[c]{2}{*}{\(y\)} 
      & \multicolumn{3}{C}{sign}
      \\\cline{2-4}
      & y & a-b\,y & y(a-b\,y)

      \\\midrule

      y<0     & \- & \+ & \-
      \\ 0<y<a/b & \+ & \+ & \+
      \\ a/b<y   & \+ & \- & \-

      \\\bottomrule
    \end{tabular}
    \vspace{2ex}
  \end{center}
  Se desenharmos um grafico das soluções de equilíbrio
  \begin{center}
    % \tikzset{external/remake next=true}
    % \pgfplotsset{height=7cm, width= .6\textwidth}
    \begin{tikzpicture}
      \begin{axis}
        [
          % xmajorgrids = true,
          legend pos={south east},
          domain=-3:3,
          ymax=2,
          ymin=-1,
          xmax=3,
          xlabel={\(x/b\)},
          ylabel={\(y*b/a\)},
          axis lines={center}, % left|right|center|box,
          view={0}{90},
        ]
        % Legends
        % \addlegendimage{empty legend}
        % \addlegendentry[Graph]{\(
        %   {\color{Graph}x}
        %   {\color{GraphC}x^2}
        % \)}
        \addplot3[
            Graph!20!background,
            opacity=1,
            -stealth,
            samples=16,
            % domain=-2:2,
            quiver={
              u={1},
              v={y*(1-y)},
              scale arrows=0.150,
            },
          ](x,y,0);

        % Plot from equation
        \addplot[
            smooth,
            thick,
            Graph,
            samples = 2,
          ]{ 1 }
          node[above,pos=0.6,opacity=1]{\(y(x)=a/b\)};
        \addplot[
            smooth,
            thick,
            Graph,
            samples = 2,
          ]{ 0 }
          node[above,pos=0.6,opacity=1]{\(y(x)=0\)};

        \draw[-,ultra thick,Graph33] (0,-1) -- node[right]{-} (0,0);
        \draw[-,ultra thick,Graph32] (0, 0) -- node[right]{+} (0,1);
        \draw[-,ultra thick,Graph33] (0, 1) -- node[right]{-} (0,2);


        % y(a-by)
        % \frac{y}{b/a}(a-b\,y*a/b)
        % \frac{y}{b}(1-y)

        % English
        % 19:00 % marcação
        % 808 204 020 % geral


      \end{axis}
    \end{tikzpicture}
  \end{center}
  Podemos ver que as tres regiões divididas pelos dois pontos de equilíbrio tem um comportamento: \(R_1\) Decrescente, \(R_2\) Crescente e \(R_3\) Decrescente\\
  Seja \(y(x)=0\) a solução que verifica a condição inicial \(y(0)=y_0\):
  \begin{align*}
    y_0 &< 0
    && \begin{cases}
      x \to -\infty &\implies y(x) \to 0
      \\ x \to +\infty &\implies y(x) \to -\infty
    \end{cases}
    \\
    0 < y_0 &< a/b
    && \begin{cases}
      x \to -\infty &\implies y(x) \to 0
      \\ x \to +\infty &\implies y(x) \to a/b
      % y(x) \to_{+} 0 \impliedby x \to -\infty
      % \\ y(x) \to_{-} a/b \impliedby x \to +\infty
    \end{cases}
    \\
    a/b < y_0 &
    && \begin{cases}
      x \to -\infty &\implies y(x) \to +\infty
      \\ x \to +\infty &\implies y(x) \to a/b
      % y(x) \to_{+} a/b \impliedby x \to +\infty
    \end{cases}
  \end{align*}
  Podemos dizer que \(y(x)=0\) é um ponto de equilíbrio instável e que \(y(x)=a/b\) é um ponto de equilíbrio estável
\end{exampleBox}
\begin{exampleBox}1{Equilíbrio semiestável} % E6
  A equação autónoma
  \begin{BM}
    \odv{y}{x} = (y-1)^2
  \end{BM}
  tem \(y=1\) como único ponto de equilíbrio. Observando a reta fase, verifica-se que qualquer solução \(y(x)\) em qualquer um dos intervalos \(\myrange*{-\infty,1}\text{ e }\myrange*{1,+\infty}\) é crescente
  \begin{center}
    % \tikzset{external/remake next=true}
    % \pgfplotsset{height=7cm, width= .6\textwidth}
    \begin{tikzpicture}
      \begin{axis}
        [
          % xmajorgrids = true,
          % legend pos={south east},
          domain=-3:3,
          ymax=2,
          ymin=-1,
          % xmax=3,
          % xlabel={\(x/b\)},
          % ylabel={\(y*b/a\)},
          axis lines={center}, % left|right|center|box,
          view={0}{90},
        ]
        % Legends
        % \addlegendimage{empty legend}
        % \addlegendentry[Graph]{\(
        %   {\color{Graph}x}
        %   {\color{GraphC}x^2}
        % \)}
        \addplot3[
            Graph!20!background,
            opacity=1,
            -stealth,
            samples=24,
            % y domain=-0.5:2,
            quiver={
              u={1},
              v={(y-1)^2},
              scale arrows=0.150,
            },
          ](x,y,0);

        % Plot from equation
        \addplot[
            smooth,
            thick,
            Graph,
            % domain=-2:2,
            samples=2,
          ] {1}
          node[above,pos=0.6,opacity=1]{\(y(x)=1\)};

      \end{axis}
    \end{tikzpicture}
  \end{center}
  Podemos caracterizar esse ponto como \emph{ponto de equilíbrio semiestável}
\end{exampleBox}

\begin{sectionBox}*2{Soluções implicitas e explicitas} % S
  \begin{multicols}{2}
    \begin{sectionBox}*2b{Soluções explicitas}
      \begin{BM}
        y = f(x)
      \end{BM}
      \textit{y} isolado
    \end{sectionBox}

    \begin{sectionBox}*2b{Soluções implicitas}
      \begin{BM}
        G(x,y)=0
      \end{BM}
      Define implicitamente uma função \(y(x)\) solução da equação.
    \end{sectionBox}
  \end{multicols}

  \vspace{-3ex}

  \begin{exampleBox}1m{} % E
    Da equação
    \begin{BM}
      y^2\,y'=x^2
    \end{BM}
    Podemos tirar a solução de duas formas
    \begin{multicols}{2}
      \begin{minipage}{0.4\textwidth}
        Da forma implicita
        \begin{BM}
          y^3 -x^3 -8 = 0
        \end{BM}
      \end{minipage}

      \begin{minipage}{0.4\textwidth}
        Da forma explicita
        \begin{BM}
          y = \varphi(x) = \sqrt[3]{8+x^3}
        \end{BM}
      \end{minipage}
    \end{multicols}
  \end{exampleBox}
\end{sectionBox}

\begin{sectionBox}*2{Famílias de Soluções} % S
  \begin{BM}
    G(x,y,z) = 0
  \end{BM}
  Tal como sucede no cálculo da primitiva de uma função, em que aparece uma constante \textit{c} de integração, \emph{quando se resolve uma EDO de primeira ordem, geralmente obtém-se com solução uma expressão contendo uma constante (ou parâmetro) \textit{c}, e que representa um conjunto de soluções} a que se chamará família de soluções a um parâmetro.

  \paragraph*{Soluções particulares} são obtidas quando atribuimos valores ao parametro da familia de soluções

  \paragraph*{Solucções singulares} nem sempre existem mas existem, não podem ser obtidas atribuindo um valor a constante \textit{c}

  \paragraph*{Integral Geral} Uma família de soluções que define todas as soluções de uma EDO para um intervalo \textit{I}
\end{sectionBox}

\begin{sectionBox}1{Equação Linear de Primeira Ordem} % S2
  \begin{BM}
    y' = f(x,y)
    \iff y' + p(x)y = q(x)
  \end{BM}
  Com \(p(x)\) e \(q(x)\) funções contínuas num intervalo aberto \(I \subseteq \mathbb{R}\)

  \begin{exampleBox}*1{Exemplo} % E
    \begin{flalign*}
      y' + 2\,x\,y = x^3
      \begin{cases}
        p(x) = 2\,x
        \\ q(x) = x^3
      \end{cases}
    \end{flalign*}
  \end{exampleBox}

  \subsection*{Equação linear homgénea}
  Equação linear em que \(q(x) = 0\), quando em uma equação linear completa (\(q(x) \neq 0 \land p(x) \neq 0\)) substituirmos \(q(x)\) por 0, obtemos a equação linear homogénea associada.
\end{sectionBox}

\begin{sectionBox}*2{Solução geral de equações lineáres de primeira ordem} % S
  \begin{BM}
    y' + a(x)\,y = b(x)
  \end{BM}

  % tag  = 3.w
  % a(x) = a(x)
  % b(x) = b(x)
  % y    = y
  % x    = x

  General solution
  \begin{gather*}
    y
    = \frac{c_0}{\varphi(x)}
    + \frac{1  }{\varphi(x)}
    \,\int{b(x)\,\varphi(x)\,\odif{x}}
    %
    %
    %
    = \mathText{using 
      \eqref{eq:3.w phi_x}
      \eqref{eq:3.w prim}
    }
    = \dots
    %
    \yesnumber\label{eq:3.w answer}
  \end{gather*}

  Finding \(\varphi(x)\)
  \begin{gather*}
    \varphi(x) 
    = \exp{\left(
        \int{a(x)\,\odif{x}}
    \right)}
    = \dots
    %
    \yesnumber\label{eq:3.w phi_x}
  \end{gather*}

  Integrating
  \begin{gather*}
    \prim_x{\left(
        b(x)\,\varphi(x)
    \right)}
    = \mathText{using \eqref{eq:3.w phi_x}}
    = \dots
    %
    \yesnumber\label{eq:3.w prim}
  \end{gather*}
  \vspace{-3ex}
  \begin{sectionBox}*3m{Demonstração} % Sindex
    \begin{gather*}
      y' + p(x)\,y = q(x)
      \implies
      \left(
        y' + p(x)\,y 
      \right)\,\varphi(x)
      = \\
      = y'\,\exp{\left(\int{p(x)\,\odif{x}}\right)}
      + p(x)\,y\,\exp{\left( \int{p(x)\,\odif{x}} \right)}
      = \left(
        y\,\exp{\int{p(x)\,\odif{x}}}
      \right)'
      = \\
      = q(x)\,\varphi(x)
      = q(x)\,\exp{\int{p(x)\,\odif{x}}}
      \implies \\
      \implies y\,\exp{\int{p(x)\,\odif{x}}}
      = c+\int{
        q(x)\,\exp{\left(\int{p(x)\,\odif{x}}\right)}
        \,\odif{x}
      }
      \implies \\
      \implies y
      = \frac{c}{\exp{\int{p(x)\,\odif{x}}}}
      + \frac{1}{\exp{\int{p(x)\,\odif{x}}}}
      \,\int{
        q(x)\,\exp{\left(\int{p(x)\,\odif{x}}\right)}
        \,\odif{x}
      }
      = \\
      = \frac{c}{\varphi(x)}
      + \frac{1}{\varphi(x)}
      \,\int{q(x)\,\varphi(x)\,\odif{x}}
    \end{gather*}
  \end{sectionBox}

\end{sectionBox}

\begin{exampleBox}1{} % E
  Considere a equação
  \begin{BM}
    y' + (1-1/x)\,y = 2\,x 
    ,\quad x < 0
  \end{BM}
  Encontre a solução para a equação acima e a equação homgénea associada

  \answer{}

  % tag  = 7
  % a(x) = (1-1/x)
  % b(x) = 2\,x
  % y    = y
  % x    = x
  \begin{gather*}
    y
    = \frac{c_0}{\varphi(x)}
    + \frac{1  }{\varphi(x)}
    \,\int{2\,x\,\varphi(x)\,\odif{x}}
    %
    %
    %
    = \mathText{using \eqref{eq:7-phi_x}}
    = \frac{c_0}{(e^{x}\,c_2/x)}
    + \frac{1  }{(e^{x}\,c_2/x)}
    \,\int{2\,x\,(e^{x}\,c_2/x)\,\odif{x}}
    %
    %
    %
    = \mathText{using \eqref{eq:7-prim}}
    = c_3\,x\,e^{-x}
    + \frac{x  }{e^{x}\,c_2}
    \,2\,c_2\,e^{x}
    = c_3\,x\,e^{-x}
    + 2\,x
  \end{gather*}

  % φ(x)
  \begin{gather*}
    \varphi(x) 
    = \exp{\left(
      \int{(1-1/x)\,\odif{x}}
    \right)}
    = \exp{
      x
      - \ln{x}
      + c_1
    }
    = e^{x}\,c_2/x
    %
    \yesnumber\label{eq:7-phi_x}
  \end{gather*}

  % prim
  \begin{gather*}
    P\left(
      2\,x\,\varphi(x)
    \right)
    = \mathText{using \eqref{eq:7-phi_x}}
    = P\left(
      2\,x\,e^{x}\,c_2/x
    \right)
    = 2\,c_2\,e^{x}
    %
    \yesnumber\label{eq:7-prim}
  \end{gather*}

\end{exampleBox}

\begin{exampleBox}1{} % E8
  Na investigação de um homicídio, é, muitas vezes importante estimar o instante em que a morte ocorreu. A partir de observações experimentais, a lei de arrefecimento de Newton estabelece, com uma exatidão satisfatória, que a taxa de variação da temperatura \(T(t)\) de um corpo em arrefecimento é proporcional à diferença entre a temperatura desse corpo e a temperatura constante \(T_a\) do meio ambiente, isto é:
  \begin{BM}
    \odv{T}{t} = -k\,(T-T_a)
    %
    \yesnumber\label{eq:8-dTdt}
  \end{BM}
  Suponhamos que duas horas depois a temperatura é novamente medida e o valor encontrado é \(T_1 = \qty*{23}{\celsius}\). O crime parece ter ocorrido durante a madrugada e corpo foi encontrado pela manhã bem cedo, pelas 6 horas e 17 minutos. A perícia então faz a suposição adicional de que a temperatura do meio ambiente entre a hora da morte e a hora em que o cadáver foi encontrado se manteve mais ou menos constante nos 20°C. A perícia sabe também que a temperatura normal de um ser humano vivo é de 37°C. Vejamos como, com os dados considerados, a perícia pode determinar a hora em que ocorreu o crime.

  \answer{}

  \begin{gather*}
    t:T(t) = 37
    %
    \yesnumber\label{eq:8-T=37}
    \cong \mathText{using \eqref{eq:8 t(t)}}
    \cong 10\,e^{-t\,\num{0.601986402162968}}
    + 20
    \implies
    t 
    \cong -\ln(1.7)/\num{0.601986402162968}
    \cong \qty{-0.881462187776328}{\hour}
    \cong \qty{-52.887731266579699}{\minute}
  \end{gather*}
  Desenvolvendo \eqref{eq:8-dTdt}
  \begin{gather*}
    \odv{T}{t} 
    = -k\,(T-T_a)
    = -k\,T+k\,T_a
    = -k\,T+k\,20
    \implies
    \odv{T}{t}
    + k\,T
    = k\,20
    %
    \yesnumber\label{eq:8-dTdt 2}
  \end{gather*}

  Encontrando \(T(t)\) a partir de \eqref{eq:8-dTdt 2}

  % tag  = 8
  % a(x) = k
  % b(x) = k\,20
  % y    = T
  % x    = t
  \begin{gather*}
    T
    = \frac{c_0}{\varphi(t)}
    + \frac{1  }{\varphi(t)}
    \,\int{k\,20\,\varphi(t)\,\odif{t}}
    %
    %
    %
    = \mathText{using 
      \eqref{eq:8 phi_x}
      \eqref{eq:8-prim}
    }
    = \frac{c_0}{c_1\,e^{k\,t}}
    + \frac{1  }{c_1\,e^{k\,t}}
    k\,20\,c_1\,\left(
      c_2+e^{k\,t}/k
    \right)
    = (c_0/c_1)\,e^{-k\,t}
    + c_2\,k\,20\,e^{-k\,t}
    + 20
    = \\
    = c_3\,e^{-k\,t}
    + 20
    %
    \yesnumber\label{eq:8 T(t c_3 k)}
    \cong \mathText{using \eqref{eq:8 c_3}\eqref{eq:8 k}}
    \cong 10\,e^{-t\,\num{0.601986402162968}}
    + 20
    %
    \yesnumber\label{eq:8 t(t)}
  \end{gather*}

  % k
  \begin{gather*}
    T(2) = 23
    = \mathText{using \eqref{eq:8 T(t c_3 k)}\eqref{eq:8 c_3}}
    = 10\,e^{-k\,2}
    + 20
    \implies
    k 
    = -\ln(0.3)/2
    \cong \num{0.601986402162968}
    %
    \yesnumber\label{eq:8 k}
  \end{gather*}

  % c_3
  \begin{gather*}
    T(0) = 30
    = \mathText{using \eqref{eq:8 T(t c_3 k)}}
    = c_3\,e^{-k*0}
    + 20
    = c_3
    + 20
    \implies
    c_3 = 10
    % 
    \yesnumber\label{eq:8 c_3}
  \end{gather*}

  % φ(x)
  \begin{gather*}
    \varphi(t) 
    = \exp{\left(
      \int{k\,\odif{t}}
    \right)}
    = c_1\,e^{ k\,t }
    %
    \yesnumber\label{eq:8 phi_x}
  \end{gather*}
  % prim
  \begin{gather*}
    P\left(
      k\,20\,\varphi(t)
    \right)
    = \mathText{using \eqref{eq:8 phi_x}}
    = P\left(
      k\,20\,c_1\,e^{k\,t}
    \right)
    = k\,20\,c_1\,\left(
      c_2+e^{k\,t}/k
    \right)
    %
    \yesnumber\label{eq:8-prim}
  \end{gather*}

\end{exampleBox}

\begin{sectionBox}1{Método de Variação das constantes Eq Diff de ordem 1} % S
  Um metodo alternativo para resolver a mesma equação diferencial linear de primeira ordem
  \begin{BM}
    y' + a(x)\,y = b(x)
  \end{BM}

  Solução geral
  \begin{gather*}
    y 
    = \frac{C(x)}{\varphi(x)}
    %
    \yesnumber\label{eq:4 y(C(x))}
    = \mathText{using
      \eqref{eq:4 C(x)}
      \eqref{eq:4 phi_x}
    }
    = \dots
  \end{gather*}

  Finding \(C(x)\)
  \begin{gather*}
    y' + a(x)\,y 
    = \mathText{using \eqref{eq:4 y(C(x))}}
    = \mdif{x}{\left(
        \frac{C(x)}{\varphi(x)}
    \right)}
    + a(x)
    \,\frac{C(x)}{\varphi(x)}
    = b(x)
    \implies C'(x) = \dots
    \implies
    C(x) = \dots
    %
    \yesnumber\label{eq:4 C(x)}
  \end{gather*}

  Finding \(\varphi(x)\)
  \begin{gather*}
    \varphi(x)
    = \exp{\left(
        \prim_x{\left(
            a(x)
        \right)}
    \right)}
    = \dots
    %
    \yesnumber\label{eq:4 phi_x}
  \end{gather*}

  Podemos resolver a equação homogênea associada \(y_h\) substituir \(c_0 \to c_0(x)\) e aplicar \(y=c_0(x)/\varphi(x)\) na equação linear original, dessa forma podemos obter \(c_0(x)\) e por sequencia \(y=c_0(x)/\varphi{x}\)

  \section*{Método usando solução particular}
  \begin{BM}
    y 
    = \frac{c_0}{\varphi(x)}
    + \frac{1  }{\varphi(x)}
    \,\int{q(x)\,\varphi(x)\,\odif{x}}
    = y_h + y_i
  \end{BM}
  \begin{itemize}
    \item \(y_h\) é a solução da equação homogênea associada
    \item \(y_i\) é uma solução particular
  \end{itemize}
  Mesmo \(y_i\) aparecer como uma solução particular em que \(c_0=1\), por estarmos trabalhando com uma solução arbitrária, isso não impede de ser qualquer solução particular, da no mesmo ao final das contas
\end{sectionBox}

\begin{exampleBox}1{} % E

  \begin{BM}
    y' - \frac{2x}{x^2+1}\,y = 1
  \end{BM}
  Encontre a solução geral usando o método de variação das constantes
  
  \answer{}

  Solução geral
  \begin{gather*}
    y
    = \frac{C(x)}{\varphi(x)}
    %
    \yesnumber\label{eq:e.9 y(C(x))}
    = \mathText{using
      \eqref{eq:e.9 C(x)}
      \eqref{eq:e.9 phi_x}
    }
    = \frac{
      c_1\,(
      \arctan{x}+c_2
    )    
    }{
      \frac{c_1}{x^2+1}
    }
    = \left(x^2+1\right)
    \,\left(
      \arctan{x}
      + c_2
    \right)
  \end{gather*}

  Finding \(C(x)\)
  \begin{gather*}
    y'-\frac{2\,x}{x^2+1}
    = \mathText{using 
      \eqref{eq:e.9 y(C(x))}
      \eqref{eq:e.9 phi_x}
    }
    = \mdif{x}{\left(
        \frac{C(x)}{\frac{c_1}{x^2+1}}
    \right)}
    - \frac{2\,x}{x^2+1}
    \,\frac{C(x)}{\frac{c_1}{x^2+1}}
    % = \\
    = \frac{1}{c_1}\left(
      C'(x)
      (x^2+1)
      + C(x)\,2\,x
    \right)
    - \frac{C(x)\,2\,x}{c_1}
    = \\
    = C'(x)\,\frac{x^2+1}{c_1}
    = 1
    % \implies \\
    \implies
    C'(x) 
    = \frac{c_1}{x^2+1}
    \implies \\
    \implies
    C(x)
    = \prim_x{\left(
      \frac{c_1}{x^2+1}
    \right)}
    = c_1\,(
      \arctan{x}+c_2
    )    
    %
    \yesnumber\label{eq:e.9 C(x)}
  \end{gather*}

  Finding \(\varphi(x)\)
  \begin{gather*}
    \varphi(x)
    = \exp{\left(
        \prim_x{\left(
            -\frac{2\,x}{x^2+1}
        \right)}
    \right)}
    = \exp{\left(
        -\int{\left(
            \frac{\odif{x^2+1}}{x^2+1}
        \right)}
    \right)}
    = \\
    = \exp{\left(
        -(
          \ln(x^2+1)+c_0
        )
    \right)}
    = 
    \frac{c_1}{x^2+1}
    %
    \yesnumber\label{eq:e.9 phi_x}
  \end{gather*}


  \answer{\eqref{eq:9.2 answer}}

  % tag  = 9
  % a(x) = \left(\frac{-2\,x}{x^2+1}\right)
  % b(x) = 1
  % y    = y
  % x    = x

  \begin{gather*}
    y
    = \frac{c_0}{\varphi(x)}
    + \frac{1  }{\varphi(x)}
    \,\int{1\,\varphi(x)\,\odif{x}}
    %
    %
    %
    = \mathText{using 
      \eqref{eq:9.2 phi_x}
      \eqref{eq:9.2 prim}
    }
    = \frac{c_0}{\frac{1}{c_2\,(x^2+1)}}
    + \frac{1  }{\frac{1}{c_2\,(x^2+1)}}
      (c_3 + \arctan{x})/c_2
    = (x^2+1)(c_5 + \arctan{x})
    %
    \yesnumber\label{eq:9.2 answer}
  \end{gather*}

  % φ(x)
  \begin{gather*}
    \varphi(x) 
    = \exp{\left(
      \int{\left(\frac{-2\,x}{x^2+1}\right)\,\odif{x}}
    \right)}
    = \exp{\left(
        -\int{\frac{\odif{(x^2+1)}}{x^2+1}}
    \right)}
    = \\
    = \exp{-\left(
        c_1 + \ln{(x^2+1)}
    \right)}
    = \frac{1}{c_2\,(x^2+1)}
    %
    \yesnumber\label{eq:9.2 phi_x}
  \end{gather*}

  % prim
  \begin{gather*}
    P\left(
      1\,\varphi(x)
    \right)
    = \mathText{using \eqref{eq:9.2 phi_x}}
    = P\left(
      \frac{1}{c_2\,(x^2+1)}
    \right)
    = (c_3 + \arctan{x})/c_2
    %
    \yesnumber\label{eq:9.2 prim}
  \end{gather*}

\end{exampleBox}

\begin{sectionBox}1{Equação de Bernoulli e a equação de Riccati} % S5
  São equações não lineares que, após mudanças de variáveis apropriadas, se transformam em equações lineares:

  \subsection{Eq de Bernoulli}
  \begin{BM}
    \shortintertext{A atribuição}
    y = z^{1/(1-k)}
    \shortintertext{Transforma a eq diferencial}
    y' + a(x)\,y = b(x)\,y^k
    \implies
    z' + (1-k)\,a(x)\,z = (1-k)\,b(x)
    \shortintertext{onde \(z\) pode ser encontrado por}
    z 
    = \frac{c_0}{\varphi(x)}
    + \frac{1  }{\varphi(x)}
    \,\int{(1-k)\,b(x)\,\varphi(x)\,\odif{x}}
    ; \\
    \varphi(x)
    = \exp{\left(
        \int{(1-k)\,a(x)\,\odif{x}}
    \right)}
  \end{BM}
  Quando encontramos uma EDO que possa ser escrita na forma acima, podemos realizar a substituiçãp de \(z=y^{1-k}\) transformando a EDO em uma equação linear, assim podemos encontrar a solução geral para \textit{z} que pode ser substituida para encontrar a solução de \textit{y} que é a equação original.

  \begin{sectionBox}*3m{workflow} % S
    \begin{BM}
      y' + a(x)\,y = b(x)\,y^k
    \end{BM}

    Solução geral
    \begin{gather*}
      y = z^{1/(1-k)}
      = \left(
        \frac{c_0}{\varphi(x)}
        + \frac{1}{\varphi(x)}
        \,\int{(1-k)\,b(x)\,\varphi(x)\,\odif{x}}
      \right)^{1/(1-k)}
      = \mathText{using 
        \eqref{eq:w 5.1 phi_x}
        \eqref{eq:w 5.1 prim}
      }
      = \dots
    \end{gather*}

    Encontrando \(\varphi(x)\)
    \begin{gather*}
      \varphi(x)
      = \exp{\left(
          \int{(1-k)\,a(x)\,\odif{x}}
      \right)}
      = \dots
      %
      \yesnumber\label{eq:w 5.1 phi_x}
    \end{gather*}

    Resolvendo integral
    \begin{gather*}
      \int{(1-k)\,b(x)\,\varphi(x)\,\odif{x}}
      = \mathText{using \eqref{eq:w 5.1 phi_x}}
      = \dots
      %
      \yesnumber\label{eq:w 5.1 prim}
    \end{gather*}
  \end{sectionBox}

  \subsection{Eq de Riccati}
  \begin{BM}
    \shortintertext{A substituição}
    y = y_1 + 1/z
    \shortintertext{Transforma a eq diferencial}
    y' + a(x)\,y = b(x) + c(x)\,y^2
    \implies \\
    \implies z' + (2\,c(x)\,y_1-a(x))z = -c(x)
    \shortintertext{onde \(z\) pode ser encontrado por}
    z
    = \frac{c_0}{\varphi(x)}
    + \frac{  1}{\varphi(x)}
    \,\int{
      -c(x)\,\varphi(x)\,\odif{x}
    }
    ; \\
    \varphi(x)
    = \exp{\left(
        \int{
          \left(
            2\,c(x)\,y_1
            -a(x)
          \right)
          \,\varphi(x)
          \,\odif{x}
        }
    \right)}
  \end{BM}
\end{sectionBox}

\begin{exampleBox}1{Eq de Bernoulli} % E
  Considere o problma de valores iniciais (PVI)
  \begin{BM}[align*]
    y' - x\,y = x\,y^3 
    ,\quad 
    y(0) = 1
  \end{BM}

  \answer{\eqref{eq:e.10 answer}}

  % tag= e.10
  % a(x) = (-x)
  % b(x) = x
  % k    = 3

  Substituição de Bernoulli
  \begin{gather*}
    y = z^{1/(1-3)}
    \implies
    y' - x\,y = x\,y^3 
    % \implies \\
    \implies
    z' + (1-k)\,(-x)\,z = (1-k)\,3
  \end{gather*}

  Solução geral
  \begin{gather*}
    y 
    = z^{1/(1-3)}
    = \left(
      \frac{c_0}{\varphi(x)}
      +\frac{1 }{\varphi(x)}
      \,\int{
        (1-3)\,x
        \,\varphi(x)
        \,\odif{x}
      }
    \right)^{-1/2}
    = \mathText{using
      \eqref{eq:e.10 phi_x}
      \eqref{eq:e.10 prim}
    }
    = \left(
      \frac{c_0}{c_2\,e^{x^2}}
      +\frac{1 }{c_2\,e^{x^2}}
      (-c_2)
      \,\left(
        e^{x^2}+c_3
      \right)
    \right)^{-1/2}
    = \left(
      c_5\,e^{-x^2}
      -1
    \right)^{-1/2}
    %
    \yesnumber\label{eq:e.10 y(x c_5)}
    = \mathText{using \eqref{eq:e.10 c_5}}
    = \left(
      2\,e^{-x^2}
      -1
    \right)^{-1/2}
    %
    \yesnumber\label{eq:e.10 answer}
  \end{gather*}

  Encontrando \(c_5\)
  \begin{gather*}
    y(0)=1
    = \mathText{using \eqref{eq:e.10 y(x c_5)}}
    = \left(
      c_5\,e^{-0^2}
      -1
    \right)^{-1/2}
    \implies \\
    \implies
    c_5 = 1+1^2 = 2
    %
    \yesnumber\label{eq:e.10 c_5}
  \end{gather*}

  Encontrando \(\varphi(x)\)
  \begin{gather*}
    \varphi(x)
    = \exp{\left(
        \prim_x\left(
          (1-3)\,(-x)
        \right)
    \right)}
    = \exp{\left(
        2(c_1+x^2/2)
    \right)}
    = c_2\,e^{x^2}
    % 
    \yesnumber\label{eq:e.10 phi_x}
  \end{gather*}

  Integrando
  \begin{gather*}
    \int{
      (1-3)\,x
      \,\varphi(x)
      \,\odif{x}
    }
    = \mathText{using \eqref{eq:e.10 phi_x}}
    = -\int{
      2\,x
      \,c_2\,e^{x^2}
      \,\odif{x}
    }
    = -c_2
    \,\int{
      e^{x^2}
      \,\odif{\left( x^2 \right)}
    }
    = -c_2
    \,\left(
      e^{x^2}+c_3
    \right)
    % 
    \yesnumber\label{eq:e.10 prim}
  \end{gather*}
\end{exampleBox}

\begin{exampleBox}1{Eq Bernoulli} % E
  Suponhamos que numa comunidade constituida por \textit{N} individuos
  \begin{itemize}
    \item \(y(t)\) representa o número de intectados pelo vírus da gripe A
    \item \(x(t)=N-y(t)\) representa a população não infectada.
  \end{itemize}
  Considere-se que o vírus se propaga pelo contacto entre infectados e não infectados e que a propagação é proporcional ao número de contactos entre estes dois grupos. Suponhamos também que os elementos dos dois grupos se relacionam livremente entre si de modo que o número de contactos entre infectados e não infectados é proporcional ao produto de \(x(t)\) por \(y(t)\) isto é
  \begin{BM}
    k\,x(t) = k\,(N-y(t))\,y(t)
  \end{BM}
  em que \textit{k} é a constante de proporcionalidade.
  se \(y_0\) é o numero inicial de infectados, o número de infectados \(y(t)\) no instante \(t\) é a solução PVI
  \begin{BM}[align*]
    y' &= k\,(N-y)\,y 
    ;& k &> 0
    ;& y(0) = y_0
  \end{BM}

  Incompleta:
  \begin{gather*}
      y: 
      y' = k\,(N-y)\,y 
      \implies
      y' - N\,k\,y = -k\,y^2 
      % 
      % 
      % 
      ; \\[1ex]
      y 
      = z^{-1}
      = \left(
        c\,e^{-N\,k\,t}
        + \frac{1}{N\,t}
      \right)^{-1}
      = \dots
      = \frac{N\,y_0}
      { (N-y_0)\,e^{-N\,k\,t} + y_0 }
      % 
      % 
      % 
      ; \\[1ex]
      c: 
      y(0)^{-1}
      = (z(0))
      = c\,e^{-N\,k *0}
      + \frac{1}{N *0}
      = y_0^{-1}
      % 
      % 
      % 
      ; \\[1ex]
      % a(t) = (-N\,k)
      % b(t) = (-k)
      % k    = 2
      z=y^{1-2}  = 1/y
      \implies \\
      \implies
      z' + N\,k\,z = k
      % a(t) = N\,k
      % b(t) = k
      % y    = z
      % x    = t
      z
      = \frac{c_0}{\varphi(t)}
      + \frac{1  }{\varphi(t)}
      \,\int{k\,\varphi(t)\,\odif{t}}
      %
      %
      %
      = \\[1ex]
      = \frac{c_0}{c_2\,e^{N\,k\,t}}
      + \frac{1  }{c_2\,e^{N\,k\,t}}
      \,\int{k\,c_2\,e^{N\,k\,t}\,\odif{t}}
      % 
      = e^{-N\,k\,t}
      \,\frac{c_0}{c_2}
      + e^{-N\,k\,t}
      \,\frac{k\,c_2}{c_2}
      \,\frac{e^{N\,k\,t}}{N\,k\,t}
      = c\,e^{-N\,k\,t}
      + \frac{1}{N\,t}
      ;\\
      c = c_0/c_2
      %
      %
      %
      ; \\[1ex]
      \varphi(t) 
      = \exp{\left(
        \int{N\,k\,\odif{t}}
      \right)}
      = \exp{\left(
        N\,k\,t+c_1
      \right)}
      = c_2\,e^{N\,k\,t}
      ; \\ c_2 = e^{c_1}
    \end{gather*}
\end{exampleBox}

\begin{exampleBox}1{Eq Riccati} % E11

  Determine a solução do PVI
  \begin{BM}
    y' - y = -2\,x + \frac{1}{2\,x^2}\,y^2
    ,\quad y(1) = -2
    ,\quad x > 0
  \end{BM}
  Sabendo que a equação admite a solução \(y=2\,x\)

  % tag  = e.12
  % a(x) = (-1)
  % b(x) = (-2\,x)
  % c(x) = \frac{1}{2\,x^2}
  % y_1(x) = 2\,x

  \answer{\eqref{eq:e.12 answer}}

  Riccati substitution
  \begin{gather*}
    y = 2\,x + z^{-1}
    \implies
    y' + (-1)\,y = (-2\,x) + \frac{1}{2\,x^2}\,y^2
    \implies \\
    \implies
    z' + \left(2\,\frac{1}{2\,x^2}\,2\,x - (-1)\right)\,z 
    = z' + (1+2/x)\,z 
    = -\frac{1}{2\,x^2}
  \end{gather*}

  General solution
  \begin{gather*}
    y
    = 2\,x + z^{-1}
    = 2\,x + \left(
      \frac{c_0}{\varphi(x)}
      + \frac{1}{\varphi(x)}
      \,\prim_x{\left(
        -\frac{1}{2\,x^2}
        \,\varphi(x)
      \right)}
    \right)^{-1}
    = \mathText{using
      \eqref{eq:e.12 phi_x}
      \eqref{eq:e.12 prim}
    }
    = 2\,x + \left(
      \frac{c_0}{e^x\,x^2\,c_3}
      + \frac{1}{e^x\,x^2\,c_3}
      \frac{-c_3}{2}\,\left( e^x+c_4 \right)
    \right)^{-1}
    = 2\,x + \left(
      \frac{c_6}{e^{x}\,x^{2}}
      - \frac{1}{2\,x^2}
    \right)^{-1}
    %
    \yesnumber\label{eq:e.12 y(x c_6)}
    = \mathText{using \eqref{eq:e.12 c_6}}
    = 2\,x + \left(
      \frac{e/4}{e^{x}\,x^{2}}
      - \frac{1}{2\,x^2}
    \right)^{-1}
    %
    \yesnumber\label{eq:e.12 answer}
  \end{gather*}

  Finding \(c_6\)
  \begin{gather*}
    y(1)=-2
    = \mathText{using \eqref{eq:e.12 y(x c_6)}}
    = 2*1 + \left(
      \frac{c_6}{e^{1}\,1^{2}}
      - \frac{1}{2*1^2}
    \right)^{-1}
    = 2 + \left(
      \frac{c_6}{e}
      - \frac{1}{2}
    \right)^{-1}
    \implies \\
    \implies
    c_6
    = e\left(
      (-2-2)^{-1}
      + \frac{1}{2}
    \right)
    = \frac{e}{4}
    % 
    \yesnumber\label{eq:e.12 c_6}
  \end{gather*}

  Finding \(\varphi(x)\)
  \begin{gather*}
    \varphi(x)
    = \exp{\left(
      \prim_x{(
        1+2/x
      )}
    \right)}
    = \exp{\left(
      x
      +c_1
      +2\,(
        c_2 + \ln{x}
      )
    \right)}
    = 
    e^x\,x^2\,c_3
    %
    \yesnumber\label{eq:e.12 phi_x}
  \end{gather*}

  Integrating
  \begin{gather*}
    \prim_x{\left(
      -\frac{1}{2\,x^2}
      \,\varphi(x)
    \right)}
    = \mathText{using \eqref{eq:e.12 phi_x}}
    = \prim_x{\left(
      -\frac{1}{2\,x^2}
      \,e^x\,x^2\,c_3
    \right)}
    = -\frac{c_3}{2}
    \,\prim_x{\left(
      e^x
    \right)}
    = 
    -\frac{c_3}{2}\,\left( e^x+c_4 \right)
    %
    \yesnumber\label{eq:e.12 prim}
  \end{gather*}
\end{exampleBox}

\begin{sectionBox}1{Operador de Derivação \(\mdif{}\)} % S
  \begin{BM}
    \mdif[n]{x} = \odv[n]{}{x}
    \\
    \mdif[k]{x} :C^n(I) \to C^{n-k}(I)
    \\ \mdif[k]{x} : y \to y^{(k)}=\odv[k]{y}{x}
  \end{BM}
\end{sectionBox}

\begin{sectionBox}1{Equação Diferencial Linear de ordem \textit{n}} % S
  \begin{BM}[align*]
    \sum_{i=0}^n{
      a_i\,\mdif[i]{x}(y)
    } 
    = \left(\sum_{i=0}^n{
        a_i\,\mdif[i]{x}
    }\right)\,y
    = P\,y
    = f(x)
  \end{BM}

  \begin{itemize}
    \item \(a_n\) é o Coeficiente lider
    \item Forma normal é quando esta escrita de forma que \(a_n=1\)
  \end{itemize}

  \begin{exampleBox}*1m[-4ex]{Example} % Eindex
    \begin{BM}
      \mdif[3]{x}(y) + x^2\,\mdif[2]{x}(y) - 5\,x\,\mdif[1]{x}(y) + y = x\,\cos(x)
    \end{BM}
    está escrita na forma normal
  \end{exampleBox}
\end{sectionBox}

\begin{sectionBox}*2{Operador P} % Sindex

  \begin{BM}
    P = \mdif[n]{x} + \sum_{i=0}^{n-1}{
      a_{i}\,\mdif[i]{x}
    }
  \end{BM}

  \subsection*{Linearidade}
  Dadas duas funções \(y_1,y_2 \in C^n(I)\) e \(\alpha,\beta\) numeros reais
  \begin{BM}
    P(\alpha\,y_1 + \beta\,y_2) = \alpha\,P\,y_1 + \beta\,P\,y_2
  \end{BM}

  \subsection*{Espaço Solução da equação}
  \begin{BM}
    \nuc(P) : A = \{  y \in C^n(I) : P\,y = 0 \}
  \end{BM}
  O conjunto á é nucleo do operador P, sendo portanto um subespaço de \(C^n(I)\). Este subespaço é designado por espaço solução da equação

  \subsection*{Teorema: Solução que satisfaz \(P\,y=0\)}
  \begin{BM}
    y=\varphi(x) : \mdif[i]{x}\varphi(x_0) = \alpha_i
    \\ x_0 \in I
    \land \alpha_i \in \mathbb{R}\quad \forall\,i
  \end{BM}
  Dado um \(x_0\) no intervalo aberto \textit{I} e constantes reais arbitrarias \(\alpha\), existe uma e só uma função que satisfaz \(P\,y=0\)

  \subsection*{Finidade da dimensão de \(\nuc(P)\)}
  \begin{BM}
    \dim(\nuc(P)) = n \impliedby P = \mdif[n]{x} + \sum_{i=0}^{n-1}{a_i\,\mdif[i]{x}}
  \end{BM}m
  Sendo o espaço solução da equação \(P\,y = 0\) (\(\nuc(P)\)) um subespaço do espaço liear \(C^n(I)\), Não limitado a ter dimenção infinita, a dimensão do nucleo de \textit{P} deve ser \textit{n} (limitado).

  \subsection*{Solução trivial}
  \begin{BM}
    \alpha_i=0\quad\forall\,i 
    \impliedby \\
    \impliedby
    \sum_{i=0}^{n}{\alpha_i\,y_i(x)} = 0 
    : \{y\}\text{ é linearmente idependente}
  \end{BM}

  \section*{Sistema fundamental de soluções de \(P\,y = 0\)}
  \begin{BM}
    y = \sum_{i=1}^{n}{c_i\,y_i}
  \end{BM}
  \begin{itemize}
    \item \(\{y_i\,\forall\,i\}\) é um sistema fundamental de soluções de \(P\,y = 0\)
    \item \(c_i\,\forall\,i\) são constantes arbitrárias que consituem a sua solução (ou integral) geral
  \end{itemize}
  Quaisquer \textit{n} soluções linearmente idependentes de \(P\,y=0\) que constituem uma base de \(\nuc(P)\)

\end{sectionBox}

\begin{sectionBox}1{Abaixando a ordem de uma EDO} % Sindex
  \begin{BM}
    z(x) : y = \varphi(x)\,\int{(z)\,\odif{x}}
    ;\\ P\,y=0
  \end{BM}
  \begin{itemize}
    \item \(\varphi(x)\) é uma solução particular da equação linear homogenea de ordem \textit{n} (\(P\,y = 0\))
  \end{itemize}
\end{sectionBox}

\begin{exampleBox}1{Baixamento de grau de uma Eq lin homogenea} % Eindex
  Determine a solução geral da equação
  \begin{BM}
    y'' + y'/x-y/x^2 = 0
    ,\quad x>0
  \end{BM}
  Sabendo que \(\varphi(x) = x\) é uma solução.
  
  \answer{}

  Solução geral
  \begin{gather*}
    y
    = \varphi(x)\,\prim_x{(z)}
    = x\,\prim_x{(z)}
    % 
    \yesnumber\label{eq:e.13 y->z}
    = \mathText{using \eqref{eq:e.13 z(x)}}
    = x\,\prim_x{\left(
        \frac{c_3}{x^3} 
    \right)}
    = x\,c_3\,\left(\frac{1}{2\,x^2}+c_4\right)
    = \frac{c_5}{x}+x\,c_6
  \end{gather*}

  Substitution \(y \to z\)
  \begin{gather*}
    y'' + y'/x-y/x^2 = 0
    \implies \mathText{using
      \eqref{eq:e.13 y->z}
      \eqref{eq:e.13 Dx(y)}
      \eqref{eq:e.13 D2x(y)}
    }
    \implies
    (
      2\,z + x\,z'
    )
    + \frac{1}{x}(
      \prim_x{(z)} + x\,z
    )
    - \frac{1}{x^2}(
      x\,\prim_x{(z)} 
    )
    = \\
    = 2\,z 
    + x\,z'
    + \prim_x{(z)}/x
    + z
    - \prim_x{(z)}/x
    % = \\
    = x\,z' + 3\,z 
    = 0
    \implies
    z' + \frac{3}{x}\,z = 0
    % 
    \yesnumber\label{eq:e.13 eq z}
  \end{gather*}
  
  Finding \(\mdif[1]{x}{y},\mdif[2]{x}{y}\)
  \begin{gather*}
    \mdif{x}{y}
    = \mathText{using \eqref{eq:e.13 y->z}}
    = \mdif{x}{\varphi(x)}
    \,\prim_x{(z)}
    + \varphi(x)\,z
    = \mdif{x}{\varphi(x)}
    \,\prim_x{(z)}
    + \varphi(x)\,z
    = 
    \prim_x{(z)} + x\,z
    % 
    \yesnumber\label{eq:e.13 Dx(y)}
    ;\\
    \mdif[2]{x}{y}
    = \mathText{using \eqref{eq:e.13 Dx(y)}}
    = \mdif{x}{(\prim_x{(z)} + x\,z)}
    = 
    2\,z + x\,z'
    % 
    \yesnumber\label{eq:e.13 D2x(y)}
  \end{gather*}

  Solving \eqref{eq:e.13 eq z}
  \begin{gather*}
    z
    = c_0\,(\varphi_z(x))^{-1}
    = c_0\,\left(
      \exp{\left(
          \int{(3/x)\,\odif{x}}
      \right)}
    \right)^{-1}
    = c_0\,\left(
      e^{3\,(\ln{(x)}+c_1)}
    \right)^{-1}
    = c_0\,(
      c_2\,x^3
    )^{-1}
    = \\
    = 
    \frac{c_3}{x^3} 
    %
    \yesnumber\label{eq:e.13 z(x)}
  \end{gather*}
\end{exampleBox}

\begin{sectionBox}1{Wronskiano: check dependencia linear} % Sindex
  \begin{BM}[align*]
    W(f_1,f_2,\dots,f_n)(x)
    &= \det(w)
    ;&
    w \in \mathcal{M}_{n,m}
    : w_{i,j} &= \mdif[j]{x}\,f_i
  \end{BM}
  \vspace{-3ex}
  \begin{BM}
    W(f_1,f_2,\dots,f_n)(x)
    \begin{cases}
         = 0 &\text{ Linear depedent}
      \\ \neq 0 &\text{ Linear independent}
    \end{cases}
  \end{BM}

\end{sectionBox}

\begin{sectionBox}1{Método de variação das constantes abitrárias para equação linear de ordem \textit{n}} % Sindex
  \begin{BM}[gather*]
      % a_0(x) = a_1(x)
      % a_1(x) = a_1(x)
      % a_2(x) = a_2(x)
      % a_3(x) = a_3(x)
      % f(x)   = f(x)
      % y_1(x) = y_1(x)
      % y_2(x) = y_1(x)
      % y_3(x) = y_1(x)
      y:
      \begin{pmatrix}
          a_1(x)
        \\ + a_1(x)\,\mdif[1]{x}
        \\ + a_2(x)\,\mdif[2]{x}
        \\ + a_3(x)\,\mdif[3]{x}
      \end{pmatrix}
      \,y
      = f(x)
      %
      %
      %
      \\[1ex]
      y
      = c_1(x)\,y_1(x)
      + c_2(x)\,y_2(x)
      + c_3(x)\,y_3(x)
      %
      %
      %
      \\[1ex]
      \begin{Bmatrix}
        {
            c_1'(x)\,\mdif[0]{x}y_1(x) 
          + c_2'(x)\,\mdif[0]{x}y_2(x)
          + c_3'(x)\,\mdif[0]{x}y_3(x)
        } = 0
        \\ {
            c_1'(x)\,\mdif[1]{x}\,y_1(x) 
          + c_2'(x)\,\mdif[1]{x}\,y_2(x)
          + c_3'(x)\,\mdif[1]{x}\,y_3(x)
        } = 0
        \\ {
            c_1'(x)\,\mdif[2]{x}\,y_1(x) 
          + c_2'(x)\,\mdif[2]{x}\,y_2(x)
          + c_3'(x)\,\mdif[2]{x}\,y_3(x)
        } = \frac{f(x)}{a_3(x)}
      \end{Bmatrix}
  \end{BM}
\end{sectionBox}

\begin{exampleBox}1{Metodo das var const arb} % Eindex
  Considere a equação
  \begin{BM}
    y" + 9\,y = 1/\cos(3\,x)
    ;\quad x\in\myrange*{-\pi/6,\pi/6}
  \end{BM}
  % As funções \(\cos{(3\,x)}\text{ e }\sin{(3\,x)}\) são duas soluções linearmente idependentes da equação homogénea
  % \begin{BM}
  %   y'' + 9\,y = 0
  % \end{BM}
  % Pelo que seu integral geral será dado por
  % \begin{BM}
  %   y = c_1\,\cos{(3\,x)} + c_2\,\sin{(3\,x)}
  %   ;\quad c_1,c_2\in\mathbb{R}
  % \end{BM}
  Utilizemos o método da variação das constantes arbitrárias para determinar o integral geral da equação completa

  \answer{}
  % tag    = 14
  % a_0(x) = 9
  % a_1(x) = 0
  % a_2(x) = 1
  % f(x)   = \frac{1}{\cos{(3\,x)}}

  Solução geral
  \begin{gather*}
    y
    = \mathText{using \eqref{eq:e.14 y_h}}
    = c_1(x)\,y_1(x)
    + c_2(x)\,y_2(x)
    = 
    \mathText{using
      \eqref{eq:e.14.2 y_1 y_2}
      \eqref{eq:e.14.2 c_1}
      \eqref{eq:e.14.2 c_2}
    }
    = \left(
      \cos{(3\,x)}
    \right)
    \,\left(
      -\ln{(\cos{(3\,x)})}-c_3
    \right)
    + \left(
      \sin{(3\,x)}
    \right)
    \,\left(
      x/3 + c_4
    \right)
  \end{gather*}

  Solução da equação homogenea análoga
  \begin{sectionBox}*3m{Solução da equação homogenea análoga} % Sindex
    \begin{BM}
      y'' + 9\,y = 0
    \end{BM}

    % tag  = 14.1
    % P    = \mdif[2] + 9

    % \answer{\eqref{eq:e.14.1 answer}}

    General Solution
    \begin{gather*}
      y 
      = \varphi(x)\,\prim_x{(z)}
      = \,\prim_x{(z)}
      %
      \yesnumber\label{eq:e.14.1 y->z}
      %
      = \mathText{using \eqref{eq:e.14.1 z}}
      = \,\prim_x{\left(
        z
      \right)}
      %
      \yesnumber\label{eq:e.14 y_h}
    \end{gather*}

    Substitution \(y \to z\)
    \begin{gather*}
      P(x)\,y 
      = \mathText{using \eqref{eq:e.14.1 y->z}}
      = \mdif[2] + 9
      \,(
        y
      )
      = \mathText{using 
        \eqref{eq:e.14.1 D1x(y)}
        \eqref{eq:e.14.1 D2x(y)}
      }
      = 0
      %
      \yesnumber\label{eq:e.14.1 eq z}
    \end{gather*}

    Finding \(
      \mdif{x}{y},
      \mdif[2]{x}{y}
    \)
    \begin{gather*}
      \mdif{x}{y}
      = \mdif{x}{(
        \,\prim_x{(z)}
      )}
      = \mdif{x}{(
        \,\prim_x{(z)}
      )}
      %
      \yesnumber\label{eq:e.14.1 D1x(y)}
      %
      %
      %
      ;\\
      \mdif[2]{x}{y}
      = \mathText{using \eqref{eq:e.14.1 D1x(y)}}
      = \mdif{x}{\left(
        \mdif{x}{y}
      \right)}
      %
      \yesnumber\label{eq:e.14.1 D2x(y)}
      % %
      % %
      % %
      % ;\\
      % \mdif[3]{x}{y}
      % = \mathText{using \eqref{eq:e.14.1 D2x(y)}}
      % = \mdif{x}{\left(
      %   \mdif[2]{x}{y}
      % \right)}
      % %
      % \yesnumber\label{eq:e.14.1 D3x(y)}
    \end{gather*}

    Solving \eqref{eq:e.14.1 eq z}
    \begin{gather*}
      z
      %
      \yesnumber\label{eq:e.14.1 z}
    \end{gather*}
    
  \end{sectionBox}

  % y_1 y_2
  \begin{gather*}
    \begin{cases}
      y_1 = \cos{(3\,x)}
      \\
      y_2 = \sin{(3\,x)}
    \end{cases}
    \yesnumber\label{eq:e.14.2 y_1 y_2}
  \end{gather*}

  % c_1,c_2
  \begin{gather*}
    c_1(x) 
    = \int{c_1'(x)\,\odif{x}}
    = \mathText{using \eqref{eq:e.14.2 c_1'}}
    = \int{\left(
        3\,\frac{\sin{(3\,x)}}{\cos{(3\,x)}}
    \right)\,\odif{x}}
    = -\int{
      \frac{\odif{\cos{(3\,x)}}}{\cos{(3\,x)}}
    }
    = -\ln{(\cos{(3\,x)})}-c_3
    % 
    \yesnumber\label{eq:e.14.2 c_1}
    %
    %
    ; \\[1ex]
    c_2(x) 
    = \int{c_2'(x)\,\odif{x}}
    = \mathText{using \eqref{eq:e.14.2 c_2'}}
    = \int{( 1/3)\,\odif{x}}
    = x/3 + c_4
    % 
    \yesnumber\label{eq:e.14.2 c_2}
  \end{gather*}

  % c_1',c_2'
  \begin{gather*}
    \mathText{using \eqref{eq:e.14.2 eq_sytem}}
    c_1'(x)
    = \frac{1}{W(y_1,y_2)}
    \,\begin{vmatrix}
      0 
      &  \mdif[0]{x}y_2
      \\ \frac{1}{\cos{(3\,x)}}
      &  \mdif[1]{x}y_2
    \end{vmatrix}
    = 
    \mathText{using 
      \eqref{eq:e.14.2 w}
      \eqref{eq:e.14.2 D_x(y_2)}
    }
    = \frac{1}{3}
    \,\begin{vmatrix}
      0 
      &  \sin{(3\,x)}
      \\ \frac{1}{\cos{(3\,x)}}
      &  3\,\cos{(3\,x)}
    \end{vmatrix}
    = 3\,\frac{\sin{(3\,x)}}{\cos{(3\,x)}}
    \yesnumber\label{eq:e.14.2 c_1'}
    %
    %
    %
    \\[3ex]
    \mathText{using \eqref{eq:e.14.2 eq_sytem}}
    c_2'(x)
    = \frac{1}{W(y_1,y_2)}
    \,\begin{vmatrix}
      \mdif[0]{x}y_1
      &  0 
      \\ \mdif[1]{x}y_1
      &  \frac{1}{\cos{(3\,x)}}
    \end{vmatrix}
    =
    \mathText{using 
      \eqref{eq:e.14.2 w}
      \eqref{eq:e.14.2 D_x(y_1)}
    }
    = \frac{1}{3}
    \,\begin{vmatrix}
      \cos{(3\,x)}
      &  0 
      \\ -3\,\sin{(3\,x)}
      &  \frac{1}{\cos{(3\,x)}}
    \end{vmatrix}
    = 1/3
    % 
    \yesnumber\label{eq:e.14.2 c_2'}
  \end{gather*}

  % Wronskiano
  \begin{gather*}
    W(y_1,y_2)
    = \det\begin{bmatrix}
      \mdif[0]{x}\,y_1
      &  \mdif[0]{x}\,y_2
      \\ \mdif[1]{x}\,y_1
      &  \mdif[1]{x}\,y_2
    \end{bmatrix}
    =
    \mathText{using 
      \eqref{eq:e.14.2 D_x(y_1)}
      \eqref{eq:e.14.2 D_x(y_2)}
    }
    = \det\begin{bmatrix}
      \cos{(3\,x)}
      &  \sin{(3\,x)}
      \\ -3\,\sin{(3\,x)}
      &  +3\,\cos{(3\,x)}
    \end{bmatrix}
    = 3\,\cos^2{(3\,x)}
    +3\,\sin^2{(3\,x)}
    = 3
    % 
    \yesnumber\label{eq:e.14.2 w}
  \end{gather*}

  % Sistema de equações da regra de crammer
  \begin{gather*}
    \begin{Bmatrix}
      {
        c_1'(x)\,\mdif[0]{x}\,y_1(x) 
        + c_2'(x)\,\mdif[0]{x}\,y_2(x)
      } &=& 0
      \\ {
        c_1'(x)\,\mdif[1]{x}\,y_1(x) 
        + c_2'(x)\,\mdif[1]{x}\,y_2(x)
      } &=& \frac{1}{\cos{(3\,x)}}
    \end{Bmatrix}
    \yesnumber\label{eq:e.14.2 eq_sytem}
  \end{gather*}

  % y_1',y_2'
  \begin{gather*}
    \mdif[1]{x}\,y_1
    = \mdif[1]{x}\,\cos{(3\,x)}
    = -3\,\sin{(3\,x)}
    \yesnumber\label{eq:e.14.2 D_x(y_1)}
    %
    %
    ; \\[1ex]
    \mdif[1]{x}\,y_2
    = \mdif[1]{x}\,\sin{(3\,x)}
    = +3\,\cos{(3\,x)}
    \yesnumber\label{eq:e.14.2 D_x(y_2)}
  \end{gather*}

\end{exampleBox}

\end{document}

% !TEX root = ./AM3C-Slides_annotations.1.tex
\documentclass["AM3C-Slides_annotations.tex"]{subfiles}

% \tikzset{external/force remake=true} % - remake all

\begin{document}

% \graphicspath{{\subfix{./.build/figures/AM3C-Slides_annotations.1}}}
% \tikzsetexternalprefix{./.build/figures/AM3C-Slides_annotations.1/graphics/}

\mymakesubfile{1}[AM 3C]
{Equações Diferenciais Ordinárias (EDO)} % Subfile Title
{Equações Diferenciais Ordinárias (EDO)} % Part Title

\begin{sectionBox}1{EDO de Primeira Ordem} % S1
  \begin{BM}
    F(x,y(x),y'(x)) = 0
  \end{BM} 
  \textit{F} é definida num conjunto aberto \(D\subset\mathbb{R}^3\).
  Dado um intervalo aberto \(I\subset\mathbb{R}\), Diz-se que uma função \(\phi:I\to\mathbb{R}\) diferenciavel em \textit{I} é uma solução da equação diferencial acima se:
  \begin{enumerate}
    \item \((x,\phi(x),\phi'(x))\in D, \quad \forall\,x\in I\)
    \item \(F(x,\phi(x),\phi'(x)) = 0, \quad \forall\,x\in I\)
  \end{enumerate}

  \paragraph*{Ordem de uma equação diferencial} é a ordem da derivada mais elevada referida na equação

  \begin{multicols}{2}
    \begin{exampleBox}1{} % E1.1
      A equação
      \begin{BM}
        y' - \frac{y}{x} = x\,e^x
      \end{BM}
      é de primeira ordem e as funções
      \begin{BM}[align*]
        y(x) &= c\,x + x\,e^x & c\in\mathbb{R}
      \end{BM}
      são soluções em \(\myrange*{0,\infty}\) desta equação.
    \end{exampleBox}

    \begin{exampleBox}1{} % E1.2
      A equação
      \begin{BM}
        y" + 4\,y = 0
      \end{BM}
      é de segunda ordem e as funções
      \begin{BM}[align*]
        y(x) &= c_1\,\cos{2\,x} 
        \\   &+ c_2\,\sin{2\,x},
        \\ & c_1,c_2\in\mathbb{R}
      \end{BM}

      São soluções em \(\mathbb{R}\) desta equação
    \end{exampleBox}
  \end{multicols}

  \subsection*{Forma normal}
  \begin{BM}
    y'(x) = f(x,y(x))
  \end{BM}
  Com \textit{f} definida no conjunto aberto \(A\subset\mathbb{R}^2\).
  As equações de primeira ordem na forma normal admitem uma interpretação geométricas relativamente simples e que \emph{permite ter uma ideia aproxiamada dos gráficos das soluções} destas esquações.

  \subsection*{Campo de direções da equação}
  Com uma equação diferencial de primeira ordem na forma normal definida no conjunto aberto \(A\subset\mathbb{R}^2\), se a cada ponto \((x,y)\) de \textit{A} se associar a direção das retas de declive igual a \(f(x,y)\), se obtem aquilo a que usualmente se chama de campo de direções da equação.

  \begin{exampleBox}1{Campo de direções da equação} % E1.3
    \begin{BM}
      y'=-y
    \end{BM}
    \begin{center}
      % \tikzset{external/remake next=true}
      % \pgfplotsset{height=7cm, width= .6\textwidth}
      \begin{tikzpicture}
      \begin{axis}
        [
          % xmajorgrids = true,
          % legend pos={south east},
          xtick distance=1,
          domain=-3:3,
          y domain = -3:3,
          % ymin=-3.5,ymax=3.5,
          % xmin=-3.5,xmax=3.5,
          view={0}{90},
          % xlabel={},
          % ylabel={},
          % axis lines={center}, % left|right|center|box,
        ]

        % Plot from equation
        \addplot3[
          % opacity=1,
          opacity=0.2,
          Graph,
          samples = 12,
          quiver = {
            u={1},
            v={-y},
            scale arrows=0.25,
          },
          -stealth,
        ]{0};
      \end{axis}
      \end{tikzpicture}
    \end{center}
    
  \end{exampleBox}
\end{sectionBox}

\begin{sectionBox}1{Equação autonoma} % S2
  Uma EDO em que não aparece explicitamente a variável independente.
  Se for \textit{y} a função icógnita e \textit{x} a variável independente, uma equa;cão diferencial autónoma de primeira ordem é uma equação da forma \(F(y,y')=0\) ou na forma normal:
  \begin{BM}
    \odv{y}{x}=f(y)
  \end{BM}
  \paragraph*{Pontos de equilíbrio (críticos ou estacionários)} são os zeros da função 
  \begin{BM}
    f(c)=0 \implies y(x)=c \text{ é solução de } f(x)=\odv{y}{x}
  \end{BM}
  \(y(x)=c\) chama-se solução de equilíbrio (ou estacionária)


  \subsection*{Classificação dos pontos de equilíbrio (Eq autónomas)}
  Prestando atenção nos limites:
  \begin{BM}
    f(c)=0
  \end{BM}
  \begin{align*}
       x &\to +\infty          && \implies y(x) \to c \implies c \text{ é um ponto de eq estável}
    \\ x &\to -\infty          && \implies y(x) \to c \implies c \text{ é um ponto de eq instável}
    \\ x &\to -\infty \land x \to +\infty && \implies y(x) \to c \implies c \text{ é um ponto de eq semiestável}
  \end{align*}
\end{sectionBox}

\begin{exampleBox}1{Pontos de equilíbrio} % E4
  Considere-se a equação autónoma
  \begin{BM}
    \odv{y}{x}=y(a-b\,y); a,b\in\mathbb{R}^+
  \end{BM}
  Pontos de equilíbrio:
  \begin{flalign*}
    &
    c=y:
    y(a-b\,y) = 0
    \begin{cases}
      y=0\\ y=\frac{a}{b}
    \end{cases}
    &\\&
    \therefore y(x)=0\lor y(x)=a/b
    &
  \end{flalign*}
  Podemos prever o comportamento da equalção pela seguinte tabela
  \begin{center}
    \vspace{1ex}
    \def\-{\cellC{Graph33}{-}}
    \def\+{\cellC{Graph32}{+}}
    \begin{tabular}{*{4}{C}}
      \toprule

      \multirow[c]{2}{*}{\(y\)} 
      & \multicolumn{3}{C}{sign}
      \\\cline{2-4}
      & y & a-b\,y & y(a-b\,y)

      \\\midrule

      y<0     & \- & \+ & \-
      \\ 0<y<a/b & \+ & \+ & \+
      \\ a/b<y   & \+ & \- & \-

      \\\bottomrule
    \end{tabular}
    \vspace{2ex}
  \end{center}
  Se desenharmos um grafico das soluções de equilíbrio
  \begin{center}
    % \tikzset{external/remake next=true}
    % \pgfplotsset{height=7cm, width= .6\textwidth}
    \begin{tikzpicture}
      \begin{axis}
        [
          % xmajorgrids = true,
          legend pos={south east},
          domain=-3:3,
          ymax=2,
          ymin=-1,
          xmax=3,
          xlabel={\(x/b\)},
          ylabel={\(y*b/a\)},
          axis lines={center}, % left|right|center|box,
          view={0}{90},
        ]
        % Legends
        % \addlegendimage{empty legend}
        % \addlegendentry[Graph]{\(
        %   {\color{Graph}x}
        %   {\color{GraphC}x^2}
        % \)}
        \addplot3[
            Graph!20!background,
            opacity=1,
            -stealth,
            samples=16,
            % domain=-2:2,
            quiver={
              u={1},
              v={y*(1-y)},
              scale arrows=0.150,
            },
          ](x,y,0);

        % Plot from equation
        \addplot[
            smooth,
            thick,
            Graph,
            samples = 2,
          ]{ 1 }
          node[above,pos=0.6,opacity=1]{\(y(x)=a/b\)};
        \addplot[
            smooth,
            thick,
            Graph,
            samples = 2,
          ]{ 0 }
          node[above,pos=0.6,opacity=1]{\(y(x)=0\)};

        \draw[-,ultra thick,Graph33] (0,-1) -- node[right]{-} (0,0);
        \draw[-,ultra thick,Graph32] (0, 0) -- node[right]{+} (0,1);
        \draw[-,ultra thick,Graph33] (0, 1) -- node[right]{-} (0,2);


        % y(a-by)
        % \frac{y}{b/a}(a-b\,y*a/b)
        % \frac{y}{b}(1-y)

        % English
        % 19:00 % marcação
        % 808 204 020 % geral


      \end{axis}
    \end{tikzpicture}
  \end{center}
  Podemos ver que as tres regiões divididas pelos dois pontos de equilíbrio tem um comportamento: \(R_1\) Decrescente, \(R_2\) Crescente e \(R_3\) Decrescente\\
  Seja \(y(x)=0\) a solução que verifica a condição inicial \(y(0)=y_0\):
  \begin{align*}
    y_0 &< 0
    && \begin{cases}
      x \to -\infty &\implies y(x) \to 0
      \\ x \to +\infty &\implies y(x) \to -\infty
    \end{cases}
    \\
    0 < y_0 &< a/b
    && \begin{cases}
      x \to -\infty &\implies y(x) \to 0
      \\ x \to +\infty &\implies y(x) \to a/b
      % y(x) \to_{+} 0 \impliedby x \to -\infty
      % \\ y(x) \to_{-} a/b \impliedby x \to +\infty
    \end{cases}
    \\
    a/b < y_0 &
    && \begin{cases}
      x \to -\infty &\implies y(x) \to +\infty
      \\ x \to +\infty &\implies y(x) \to a/b
      % y(x) \to_{+} a/b \impliedby x \to +\infty
    \end{cases}
  \end{align*}
  Podemos dizer que \(y(x)=0\) é um ponto de equilíbrio instável e que \(y(x)=a/b\) é um ponto de equilíbrio estável
\end{exampleBox}
\begin{exampleBox}1{Equilíbrio semiestável} % E6
  A equação autónoma
  \begin{BM}
    \odv{y}{x} = (y-1)^2
  \end{BM}
  tem \(y=1\) como único ponto de equilíbrio. Observando a reta fase, verifica-se que qualquer solução \(y(x)\) em qualquer um dos intervalos \(\myrange*{-\infty,1}\text{ e }\myrange*{1,+\infty}\) é crescente
  \begin{center}
    % \tikzset{external/remake next=true}
    % \pgfplotsset{height=7cm, width= .6\textwidth}
    \begin{tikzpicture}
      \begin{axis}
        [
          % xmajorgrids = true,
          % legend pos={south east},
          domain=-3:3,
          ymax=2,
          ymin=-1,
          % xmax=3,
          % xlabel={\(x/b\)},
          % ylabel={\(y*b/a\)},
          axis lines={center}, % left|right|center|box,
          view={0}{90},
        ]
        % Legends
        % \addlegendimage{empty legend}
        % \addlegendentry[Graph]{\(
        %   {\color{Graph}x}
        %   {\color{GraphC}x^2}
        % \)}
        \addplot3[
            Graph!20!background,
            opacity=1,
            -stealth,
            samples=24,
            % y domain=-0.5:2,
            quiver={
              u={1},
              v={(y-1)^2},
              scale arrows=0.150,
            },
          ](x,y,0);

        % Plot from equation
        \addplot[
            smooth,
            thick,
            Graph,
            % domain=-2:2,
            samples=2,
          ] {1}
          node[above,pos=0.6,opacity=1]{\(y(x)=1\)};

      \end{axis}
    \end{tikzpicture}
  \end{center}
  Podemos caracterizar esse ponto como \emph{ponto de equilíbrio semiestável}
\end{exampleBox}

\begin{sectionBox}*2{Soluções implicitas e explicitas} % S
  \begin{multicols}{2}
    \begin{sectionBox}*2b{Soluções explicitas}
      \begin{BM}
        y = f(x)
      \end{BM}
      \textit{y} isolado
    \end{sectionBox}

    \begin{sectionBox}*2b{Soluções implicitas}
      \begin{BM}
        G(x,y)=0
      \end{BM}
      Define implicitamente uma função \(y(x)\) solução da equação.
    \end{sectionBox}
  \end{multicols}

  \vspace{-3ex}

  \begin{exampleBox}1m{} % E
    Da equação
    \begin{BM}
      y^2\,y'=x^2
    \end{BM}
    Podemos tirar a solução de duas formas
    \begin{multicols}{2}
      \begin{minipage}{0.4\textwidth}
        Da forma implicita
        \begin{BM}
          y^3 -x^3 -8 = 0
        \end{BM}
      \end{minipage}

      \begin{minipage}{0.4\textwidth}
        Da forma explicita
        \begin{BM}
          y = \varphi(x) = \sqrt[3]{8+x^3}
        \end{BM}
      \end{minipage}
    \end{multicols}
  \end{exampleBox}
\end{sectionBox}

\begin{sectionBox}*2{Famílias de Soluções} % S
  \begin{BM}
    G(x,y,z) = 0
  \end{BM}
  Tal como sucede no cálculo da primitiva de uma função, em que aparece uma constante \textit{c} de integração, \emph{quando se resolve uma EDO de primeira ordem, geralmente obtém-se com solução uma expressão contendo uma constante (ou parâmetro) \textit{c}, e que representa um conjunto de soluções} a que se chamará família de soluções a um parâmetro.

  \paragraph*{Soluções particulares} são obtidas quando atribuimos valores ao parametro da familia de soluções

  \paragraph*{Solucções singulares} nem sempre existem mas existem, não podem ser obtidas atribuindo um valor a constante \textit{c}

  \paragraph*{Integral Geral} Uma família de soluções que define todas as soluções de uma EDO para um intervalo \textit{I}
\end{sectionBox}

\begin{sectionBox}1{Equação Linear de Primeira Ordem} % S2
  \begin{BM}
    y' = f(x,y)
    \iff y' + p(x)y = q(x)
  \end{BM}
  Com \(p(x)\) e \(q(x)\) funções contínuas num intervalo aberto \(I \subseteq \mathbb{R}\)

  \begin{exampleBox}*1{Exemplo} % E
    \begin{flalign*}
      y' + 2\,x\,y = x^3
      \begin{cases}
        p(x) = 2\,x
        \\ q(x) = x^3
      \end{cases}
    \end{flalign*}
  \end{exampleBox}

  \subsection*{Equação linear homgénea}
  Equação linear em que \(q(x) = 0\), quando em uma equação linear completa (\(q(x) \neq 0 \land p(x) \neq 0\)) substituirmos \(q(x)\) por 0, obtemos a equação linear homogénea associada.
\end{sectionBox}

\begin{sectionBox}*2{Solução geral de equações lineáres de primeira ordem} % S
  \begin{BM}
    y' + p(x)\,y = q(x)
    \implies y 
    = \frac{c}{\varphi(x)}
    + \frac{1}{\varphi(x)}
    \,\int{\varphi(x)\,q(x)\,\odif{x}}
    \\
    \varphi(x) = \exp{\int{p(x)\,\odif{x}}}
    \\[3ex]
    y 
    = \frac{c}{\varphi(x)}
    : q(x) = 0 \text{ (eq Homogênea)}
  \end{BM}
  \vspace{-3ex}
  \begin{sectionBox}*3m{Demonstração} % Sindex
    \begin{flalign*}
      &
        y' + p(x)\,y = q(x)
        \implies
        \left(
          y' + p(x)\,y 
        \right)\,\varphi(x)
        = &\\&
        = y'\,\exp{\left(\int{p(x)\,\odif{x}}\right)}
        + p(x)\,y\,\exp{\left( \int{p(x)\,\odif{x}} \right)}
        = \left(
          y\,\exp{\int{p(x)\,\odif{x}}}
        \right)'
        = &\\&
        = q(x)\,\varphi(x)
        = q(x)\,\exp{\int{p(x)\,\odif{x}}}
        \implies &\\&
        \implies y\,\exp{\int{p(x)\,\odif{x}}}
        = c+\int{
          q(x)\,\exp{\left(\int{p(x)\,\odif{x}}\right)}
          \,\odif{x}
        }
        \implies &\\&
        \implies y
        = \frac{c}{\exp{\int{p(x)\,\odif{x}}}}
        + \frac{1}{\exp{\int{p(x)\,\odif{x}}}}
        \,\int{
          q(x)\,\exp{\left(\int{p(x)\,\odif{x}}\right)}
          \,\odif{x}
        }
        = &\\&
        = \frac{c}{\varphi(x)}
        + \frac{1}{\varphi(x)}
        \,\int{q(x)\,\varphi(x)\,\odif{x}}
      &
    \end{flalign*}
  \end{sectionBox}
\end{sectionBox}

\begin{exampleBox}1{} % E
  Considere a equação
  \begin{BM}[align*]
    y' + (1-1/x)\,y &= 2\,x ,& x &< 0
  \end{BM}
  Encontre a solução para a equação acima e a equação homgénea associada
  \answer{}
  \begin{flalign*}
    &
      % a(x) = (1-1/x)
      % b(x) = (2\,x)
      y
      = \frac{c_0}{\varphi(x)}
      + \frac{1}{\varphi(x)}
      \,\int{(2\,x)\,\varphi(x)\,\odif{x}}
      % 
      % 
      % 
      = &\\[3ex]&
      = c_0\,\frac{x}{c_3\,e^x}
      +\,\frac{c_3\,x}{e^x}
      \,\int{(2\,x)\,\frac{e^x}{c_3\,x}\,\odif{x}}
      = &\\&
      = \frac{c_0}{c_3}
      \,\frac{x}{e^x}
      + \frac{x}{e^x}
      \,2
      \,\int{e^x\,\odif{x}}
      = &\\&
      = \frac{c_0}{c_3}
      \,\frac{x}{e^x}
      + \frac{2\,x}{e^x}
      \,(c_4 + e^x)
      = c\,\frac{x}{e^x}
      + 2\,x
      % 
      % 
      % 
      ;&\\& 
      c = \frac{c_0}{c_3}+2\,c_4
      %
      %
      %
      ; &\\[3ex]&
      \varphi(x) 
      = \exp{\left(
        \int{(1-1/x)\,\odif{x}}
      \right)}
      = \exp{\left(
          c_1+x
          -(c_2+\ln{x})
      \right)}
      = &\\&
      = c_3\,\frac{\exp{x}}{x}
      = c_3\,\frac{e^x}{x}
      % 
      % 
      % 
      ; &\\& c_3 = \exp{(c_1-c_2)}
    &
  \end{flalign*}

  \subsection*{Equação homgénea}
  \begin{flalign*}
    &
      % a(x) = (1-1/x)
      % b(x) = 0
      % \varphi(x) = c_3\,\frac{e^x}{x}
      y' + (1-1/x)\,y = 0
      % \implies &\\&
      \implies
      y
      = \frac{c_0}{\varphi(x)}
      %
      %
      %
      % = &\\[3ex]&
      = c_0\,\frac{x}{c_3\,e^x}
      = c_5\,\frac{x}{e^x}
      % 
      % ; &\\& 
      ; c_5=c_0/c_3
    &
  \end{flalign*}
\end{exampleBox}

\begin{exampleBox}1{} % E8
   Na investigação de um homicídio, é, muitas vezes importante estimar o instante em que a morte ocorreu. A partir de observações experimentais, a lei de arrefecimento de Newton estabelece, com uma exatidão satisfatória, que a taxa de variação da temperatura \(T(t)\) de um corpo em arrefecimento é proporcional à diferença entre a temperatura desse corpo e a temperatura constante \(T_a\) do meio ambiente, isto é:
   \begin{BM}
     \odv{T}{t} = -k\,(T-T_a)
   \end{BM}
   Suponhamos que duas horas depois a temperatura é novamente medida e o valor encontrado é \(T_1 = \qty*{23}{\celsius}\). O crime parece ter ocorrido durante a madrugada e corpo foi encontrado pela manhã bem cedo, pelas 6 horas e 17 minutos. A perícia então faz a suposição adicional de que a temperatura do meio ambiente entre a hora da morte e a hora em que o cadáver foi encontrado se manteve mais ou menos constante nos 20°C. A perícia sabe também que a temperatura normal de um ser humano vivo é de 37°C. Vejamos como, com os dados considerados, a perícia pode determinar a hora em que ocorreu o crime.
   \answer{}
   \begin{flalign*}
     &
       t:T(t) = 37
       \implies &\\[3ex]&
       \implies
       T(t)
       = c\,e^{-k\,t} + 20
       = &\\[3ex]&
       \cong 10\,e^{-\num{0.601986402162968}*t} + 20
       = 37
       \implies &\\&
       \implies
       t
       \cong -\frac{1}{\num{0.601986402162968}}\ln(17/10)
       \cong \qty{-0.881462187776328}{\hour}
       \cong \qty{-52.8877312665797}{\min}
       % 
       % 
       % 
       ; &\\[3ex]&
       T(0) 
       = c\,e^{-k*0} + 20
       = 30
       \implies c=30-20=10
       % 
       % 
       % 
       ; &\\[3ex]&
       T(2)
       = c\,e^{-k*2} + 20
       = 10\,e^{-k*2} + 20
       = 23
       \implies &\\&
       \implies
       k = -0.5\ln(3/10)
       \cong \num{0.601986402162968}
       % 
       % 
       % 
       ; &\\[3ex]&
       \odv{T}{t} = -k\,(T-20)
       % \implies &\\&
       \implies T' + k\,t = k\,20
       \implies &\\&
       \implies
       % a(x) = (k)
       % b(x) = k\,20
       y
       = \frac{c_0}{\varphi(t)}
       + \frac{1  }{\varphi(t)}
       \,\int{k\,20\,\varphi(t)\,\odif{t}}
       %
       %
       %
       = &\\[3ex]&
       = \frac{c_0}{c_2\,e^{k\,t}}
       + \frac{1  }{c_2\,e^{k\,t}}
       \,\int{k\,20\,c_2\,e^{k\,t}\,\odif{t}}
       = \frac{c_0}{c_2}\,e^{-k\,t}
       + \frac{1  }{c_2}\,e^{-k\,t}
       \frac{k\,20\,c_2}{k}
       \,\int{e^{k\,t}\,\odif{(k\,t)}}
       = &\\&
       = \frac{c_0}{c_2}\,e^{-k\,t}
       + 20\,c_3\,e^{-k\,t}
       + 20\,e^{-k\,t}\,e^{k\,t}
       = &\\&
       = c\,e^{-k\,t}
       + 20
       ; &\\&
       c = \frac{c_0}{c_2}+20\,c_3
       %
       %
       %
       ; &\\[3ex]&
       \varphi(t) 
       = \exp{\left(
         \int{(k)\,\odif{t}}
       \right)}
       = \exp{\left(
           k\,t+c_1
       \right)}
       = e^{k\,t}\,c_2
       ; &\\&
       c_2=e^{c_1}
     &
   \end{flalign*}
\end{exampleBox}

\begin{sectionBox}1{Método de Variação das constantes} % S
  \begin{BM}
    y = \frac{c_0(x)}{\varphi(x)}
    : \left(\frac{c_0(x)}{\varphi(x)}\right) 
    + p(x)\left(\frac{c_0(x)}{\varphi(x)}\right)
    = q(x)
    \\
    y_h' + p(x)\,y_h = q(x) 
    \iff y_h = \frac{c_0}{\varphi{x}} 
    \implies y = \frac{c_0(x)}{\varphi{x}}
  \end{BM}
  Podemos resolver a equação homogênea associada \(y_h\) substituir \(c_0 \to c_0(x)\) e aplicar \(y=c_0(x)/\varphi(x)\) na equação linear original, dessa forma podemos obter \(c_0(x)\) e por sequencia \(y=c_0(x)/\varphi{x}\)

  \section*{Método usando solução particular}
  \begin{BM}
    y 
    = \frac{c_0}{\varphi(x)}
    + \frac{1  }{\varphi(x)}
    \,\int{q(x)\,\varphi(x)\,\odif{x}}
    = y_h + y_i
  \end{BM}
  \begin{itemize}
    \item \(y_h\) é a solução da equação homogênea associada
    \item \(y_i\) é uma solução particular
  \end{itemize}
  Mesmo \(y_i\) aparecer como uma solução particular em que \(c_0=1\), por estarmos trabalhando com uma solução arbitrária, isso não impede de ser qualquer solução particular, da no mesmo ao final das contas
\end{sectionBox}

\begin{exampleBox}1{} % E
  \begin{BM}
    y' - \frac{2x}{x^2+1}\,y = 1
  \end{BM}
  Encontre a solução geral usando o método de variação das constantes
  \answer{}
  \begin{flalign*}
    &
      y: y' - \frac{2x}{x^2+1}\,y = 1
      % 
      % 
      % 
      \implies &\\[3ex]&
      \implies
      y = c_0(x)\,(x^2+1)
      = &\\&
      = (\arctan(x)+c_2)\,(x^2+1)
      % 
      % 
      % 
      ; &\\[3ex]&
      y' - \frac{2x}{x^2+1}\,y
      = (c_0(x)\,(x^2+1))' 
      - \frac{2x}{x^2+1}\,(c_0(x)\,(x^2+1)) 
      = &\\&
      = c_0'(x)\,(x^2+1)
      + c_0(x)\,2\,x
      - 2\,x\,c_0(x)
      % = &\\&
      = c_0'(x)\,(x^2+1)
      = 1
      \implies &\\&
      \implies c_0'(x) = \frac{1}{x^2+1}
      \implies \mathemph{
        c_0(x)
        = \int{\frac{1}{x^2+1}\,\odif{x}}
        = \arctan{x}+c_1
      }
      % 
      % 
      % 
      ; &\\[3ex]&
      y' + \frac{2x}{x^2+1}\,y = 0
      % \implies &\\&
      \implies
      % a(x) = \frac{2x}{x^2+1}
      % b(x) = 0
      y
      = \frac{c_0}{\varphi(x)}
      %
      %
      %
      % = &\\[3ex]&
      = \mathemph{
        c_0\,(x^2+1)
      }
      %
      %
      %
      ; &\\[3ex]&
      \varphi(x) 
      = \exp{\left(
        \int{\frac{2x}{x^2+1}\,\odif{x}}
      \right)}
      = \frac{1}{x^2+1}
      &
  \end{flalign*}
\end{exampleBox}

\begin{sectionBox}1{Equação de Bernoulli e a equação de Riccati} % S5
  São equações não lineares que, após mudanças de variáveis apropriadas, se transformam em equações lineares:

  \subsection{Eq de Bernoulli}
  \begin{BM}
    y' + a(x)\,y = b(x)\,y^k
    ; \\
    z=y^{1-k}
    \implies z' + (1-k)\,a(x)\,z = (1-k)\,b(x)
  \end{BM}
  Quando encontramos uma EDO que possa ser escrita na forma acima, podemos realizar a substituiçãp de \(z=y^{1-k}\) transformando a EDO em uma equação linear, assim podemos encontrar a solução geral para \textit{z} que pode ser substituida para encontrar a solução de \textit{y} que é a equação original.

  \subsection{Eq de Ricatti}
  \begin{BM}
    y' + a(x)\,y = b(x) + c(x)\,y^2
    ; \\
    y(x)=y_1(x) + \frac{1}{z(x)}
    \implies z' + (2\,c(x)\,y_1-a(x))z = -c(x)
  \end{BM}
\end{sectionBox}

\begin{exampleBox}1{Eq de Bernoulli} % E
  Considere o problma de valores iniciais (PVI)
  \begin{BM}[align*]
    y' - x\,y &= x\,y^3 ,& y(0) &= 1
  \end{BM}
  \begin{flalign*}
    &
      % a(x) = (-x)
      % b(x) = (x)
      % k    = 3
      y: y' + -x\,y = x\,y^3
      %
      %
      %
      ; &\\[3ex]&
      y 
      = z^{-1/2}
      = \frac{1}{\sqrt{
          c_{z,0}\,e^{-x^2}-1
      }}
      = \frac{1}{\sqrt{
          2\,e^{-x^2}-1
      }}
      %
      %
      %
      ; &\\[3ex]&
      c_{z,0} :
      y(0)
      = (z(0))^{-1/2}
      = (c_{z,0}\,e^{-0^2}-1)^{-1/2}
      = (c_{z,0}-1)^{-1/2}
      = 1
      \implies &\\&
      \implies
      c_{z,0} = 2
      %
      %
      %
      ; &\\[3ex]&
      z=y^{1-3} = y^{-2}
      \implies &\\&
      \implies
      z' + 2\,x\,z = -2\,x     
      % 
      % 
      % 
      &\\&
      % a(x) = (2\,x)
      % b(x) = -2
      % y    = z
      % x    = x
      z
      = \frac{c_{z,0}}{\varphi_{z}(x)}
      + \frac{1  }{\varphi_{z}(x)}
      \,\int{-2\,\varphi_{z}(x)\,\odif{x}}
      %
      %
      %
      = &\\[3ex]&
      = \frac{c_{z,0}}{e^{x^2}}
      + \frac{-2     }{e^{x^2}}
      \,\int{e^{x^2}\,\odif{x}}
      = c_{z,0}\,e^{-x^2} - 1
      %
      %
      %
      ; &\\[3ex]&
      \varphi_{z}(x) 
      = \exp{\left(
        \int{(2\,x)\,\odif{x}}
      \right)}
      = e^{ x^2 }
    &
  \end{flalign*}
\end{exampleBox}

\begin{exampleBox}1{Eq Bernoulli} % E
  Suponhamos que numa comunidade constituida por \textit{N} individuos
  \begin{itemize}
    \item \(y(t)\) representa o número de intectados pelo vírus da gripe A
    \item \(x(t)=N-y(t)\) representa a população não infectada.
  \end{itemize}
  Considere-se que o vírus se propaga pelo contacto entre infectados e não infectados e que a propagação é proporcional ao número de contactos entre estes dois grupos. Suponhamos também que os elementos dos dois grupos se relacionam livremente entre si de modo que o número de contactos entre infectados e não infectados é proporcional ao produto de \(x(t)\) por \(y(t)\) isto é
  \begin{BM}
    k\,x(t) = k\,(N-y(t))\,y(t)
  \end{BM}
  em que \textit{k} é a constante de proporcionalidade.
  se \(y_0\) é o numero inicial de infectados, o número de infectados \(y(t)\) no instante \(t\) é a solução PVI
  \begin{BM}[align*]
    y' &= k\,(N-y)\,y 
    ;& k &> 0
    ;& y(0) = y_0
  \end{BM}
  Incompleta:
  \begin{flalign*}
    &
      y: 
      y' = k\,(N-y)\,y 
      \implies
      y' - N\,k\,y = -k\,y^2 
      % 
      % 
      % 
      ; &\\[3ex]&
      y 
      = z^{-1}
      = \left(
        c\,e^{-N\,k\,t}
        + \frac{1}{N\,t}
      \right)^{-1}
      = \dots
      = \frac{N\,y_0}
      { (N-y_0)\,e^{-N\,k\,t} + y_0 }
      % 
      % 
      % 
      ; &\\[3ex]&
      c: 
      y(0)^{-1}
      = (z(0))
      = c\,e^{-N\,k *0}
      + \frac{1}{N *0}
      = y_0^{-1}
      % 
      % 
      % 
      ; &\\[3ex]&
      % a(t) = (-N\,k)
      % b(t) = (-k)
      % k    = 2
      z=y^{1-2}  = 1/y
      \implies &\\&
      \implies
      z' + N\,k\,z = k
      % a(t) = N\,k
      % b(t) = k
      % y    = z
      % x    = t
      z
      = \frac{c_0}{\varphi(t)}
      + \frac{1  }{\varphi(t)}
      \,\int{k\,\varphi(t)\,\odif{t}}
      %
      %
      %
      = &\\[3ex]&
      = \frac{c_0}{c_2\,e^{N\,k\,t}}
      + \frac{1  }{c_2\,e^{N\,k\,t}}
      \,\int{k\,c_2\,e^{N\,k\,t}\,\odif{t}}
      % 
      = e^{-N\,k\,t}
      \,\frac{c_0}{c_2}
      + e^{-N\,k\,t}
      \,\frac{k\,c_2}{c_2}
      \,\frac{e^{N\,k\,t}}{N\,k\,t}
      = c\,e^{-N\,k\,t}
      + \frac{1}{N\,t}
      ;&\\&
      c = c_0/c_2
      %
      %
      %
      ; &\\[3ex]&
      \varphi(t) 
      = \exp{\left(
        \int{N\,k\,\odif{t}}
      \right)}
      = \exp{\left(
        N\,k\,t+c_1
      \right)}
      = c_2\,e^{N\,k\,t}
      ; &\\& c_2 = e^{c_1}
    &
  \end{flalign*}
\end{exampleBox}

\begin{exampleBox}1{Eq Ricatti} % E11
  Determine a solução do PVI
  \begin{BM}[align*]
    y' - y &= -2\,x + \frac{1}{2\,x^2}\,y^2
    ,& y(1) &= -2
    ,& x &> 0
  \end{BM}
  Sabendo que a equação admite a solução \(y=2\,x\)
  \answer{}
  \begin{flalign*}
    &
      % a(x) = (-1)
      % b(x) = (-2\,x)
      % c(x) = \frac{1}{2\,x^2}
      % y_1(x) = (-2)
      y'
      + (-1)\,y
      = (-2\,x)
      + \frac{1}{2\,x^2}\,y^2
      %
      %
      %
      ; &\\[3ex]&
      y(x)
      = y_1(x)
      + z^{-1}
      = -2 + z^{-1}
      = -2 + (
        \frac{c-e^x}{2\,x^2\,e^x}
      )^{-1}
      = \mathemph{
        -2 + \frac{2\,x^2\,e^x}{c-e^x}
      }
      %
      %
      %
      ; &\\[3ex]&
      z: 
      z' + \left(
        2
        \,\frac{1}{2\,x^2}
        \,(-2)
        - (-1)
      \right)\,z
      z' + \left(
        1-\frac{2}{x^2}
      \right)\,z
      = - \frac{1}{2\,x^2}
      % 
      % 
      % 
      &\\[3ex]&
      z = \dots = \frac{c-e^x}{2\,x^2\,e^x}
    &
  \end{flalign*}
\end{exampleBox}

\begin{sectionBox}1{Operador de Derivação \(\mdif{}\)} % S
  \begin{BM}
    \mdif[n]{x} = \odv[n]{}{x}
    \\
    \mdif[k]{x} :C^n(I) \to C^{n-k}(I)
    \\ \mdif[k]{x} : y \to y^{(k)}=\odv[k]{y}{x}
  \end{BM}
\end{sectionBox}

\begin{sectionBox}1{Equação Diferencial Linear de ordem \textit{n}} % S
  \begin{BM}[align*]
    \sum_{i=0}^n{
      a_i\,\mdif[i]{x}(y)
    } 
    = \left(\sum_{i=0}^n{
        a_i\,\mdif[i]{x}
    }\right)\,y
    = P\,y
    = f(x)
  \end{BM}

  \begin{itemize}
    \item \(a_n\) é o Coeficiente lider
    \item Forma normal é quando esta escrita de forma que \(a_n=1\)
  \end{itemize}

  \begin{exampleBox}*1m[-4ex]{Example} % Eindex
    \begin{BM}
      \mdif[3]{x}(y) + x^2\,\mdif[2]{x}(y) - 5\,x\,\mdif[1]{x}(y) + y = x\,\cos(x)
    \end{BM}
    está escrita na forma normal
  \end{exampleBox}
\end{sectionBox}

\begin{sectionBox}*2{Operador P} % Sindex

  \begin{BM}
    P = \mdif[n]{x} + \sum_{i=0}^{n-1}{
      a_{i}\,\mdif[i]{x}
    }
  \end{BM}

  \subsection*{Linearidade}
  Dadas duas funções \(y_1,y_2 \in C^n(I)\) e \(\alpha,\beta\) numeros reais
  \begin{BM}
    P(\alpha\,y_1 + \beta\,y_2) = \alpha\,P\,y_1 + \beta\,P\,y_2
  \end{BM}

  \subsection*{Espaço Solução da equação}
  \begin{BM}
    \nuc(P) : A = \{  y \in C^n(I) : P\,y = 0 \}
  \end{BM}
  O conjunto á é nucleo do operador P, sendo portanto um subespaço de \(C^n(I)\). Este subespaço é designado por espaço solução da equação

  \subsection*{Teorema: Solução que satisfaz \(P\,y=0\)}
  \begin{BM}
    y=\varphi(x) : \mdif[i]{x}\varphi(x_0) = \alpha_i
    \\ x_0 \in I
    \land \alpha_i \in \mathbb{R}\quad \forall\,i
  \end{BM}
  Dado um \(x_0\) no intervalo aberto \textit{I} e constantes reais arbitrarias \(\alpha\), existe uma e só uma função que satisfaz \(P\,y=0\)

  \subsection*{Finidade da dimensão de \(\nuc(P)\)}
  \begin{BM}
    \dim(\nuc(P)) = n \impliedby P = \mdif[n]{x} + \sum_{i=0}^{n-1}{a_i\,\mdif[i]{x}}
  \end{BM}m
  Sendo o espaço solução da equação \(P\,y = 0\) (\(\nuc(P)\)) um subespaço do espaço liear \(C^n(I)\), Não limitado a ter dimenção infinita, a dimensão do nucleo de \textit{P} deve ser \textit{n} (limitado).

  \subsection*{Solução trivial}
  \begin{BM}
    \alpha_i=0\quad\forall\,i 
    \impliedby \\
    \impliedby
    \sum_{i=0}^{n}{\alpha_i\,y_i(x)} = 0 
    : \{y\}\text{ é linearmente idependente}
  \end{BM}

  \section*{Sistema fundamental de soluções de \(P\,y = 0\)}
  \begin{BM}
    y = \sum_{i=1}^{n}{c_i\,y_i}
  \end{BM}
  \begin{itemize}
    \item \(\{y_i\,\forall\,i\}\) é um sistema fundamental de soluções de \(P\,y = 0\)
    \item \(c_i\,\forall\,i\) são constantes arbitrárias que consituem a sua solução (ou integral) geral
  \end{itemize}
  Quaisquer \textit{n} soluções linearmente idependentes de \(P\,y=0\) que constituem uma base de \(\nuc(P)\)

\end{sectionBox}

\begin{sectionBox}1{Abaixando a ordem de uma EDO} % Sindex
  \begin{BM}
    z(x) : y = \varphi(x)\,\int{(z)\,\odif{x}}
    ;\\ P\,y=0
  \end{BM}
  \begin{itemize}
    \item \(\varphi(x)\) é uma solução particular da equação linear homogenea de ordem \textit{n} (\(P\,y = 0\))
  \end{itemize}
\end{sectionBox}

\begin{exampleBox}1{Baixamento de grau de uma Eq lin homogenea} % Eindex
  Determine a solução geral da equação
  \begin{BM}[align*]
    P\,y &= 0
    ;& P &= (\mdif[2]{x}+\frac{1}{x^2}\mdif[1]{x}-\frac{1}{x^2})
    ;& x &> 0
  \end{BM}
  Sabendo que \(\varphi(x) = x\) é uma solução.
  \answer{}
  \begin{flalign*}
    &
      % P = \mdif[2]{x}+\frac{1}{x}\mdif[1]{x}-\frac{1}{x^2}
      % φ(x) = x
      P\,y
      = \left(
        \mdif[2]{x}+\frac{1}{x}\mdif[1]{x}-\frac{1}{x^2}
      \right)
      \,y
      = 0
      %
      %
      %
      ; &\\[3ex]&
      y 
      = \varphi(x)\,\int{z(x)\,\odif{x}}
      = x\,\int{
        \frac{c}{x^3}\odif{x}
      }
      = x
      \,c
      \,\left(
        c_1
        + \frac{x^{-2}}{-2}
      \right)
      = x \,c_2
      + \frac{c_3}{x}
      %
      %
      %
      ; &\\[3ex]&
      z = \frac{c}{x^3}
      ; &\\& c \in \mathbb{R}
      %
      %
      %
      ; &\\[3ex]&
      P\,y
      = \left(
        \mdif[2]{x}+\frac{1}{x}\mdif[1]{x}-\frac{1}{x^2}
      \right)
      \,\left(
        x\,\int{ z(x)\,\odif{x} }
      \right)
      = &\\& 
      = 
      (2\,z + x\,z')
      +\frac{1}{x}
      \,\left(
        \int{z(x)\,\odif{x}}
        + x\,z
      \right)
      -\frac{1}{x^2}
      \,\left(
        x\,\int{ z(x)\,\odif{x} }
      \right)
      = &\\& 
      = 3\,z + x\,z'
      = 0
      %
      %
      %
      ; &\\[3ex]&
      \mdif[1]{x}{y}
      = \mdif[1]{x}{\left(
        \varphi(x)
        \,\int{z(x)\,\odif{x}}
      \right)}
      = \mdif[1]{x}{\left(
        x
        \,\int{z(x)\,\odif{x}}
      \right)}
      = \int{z(x)\,\odif{x}}
      + x\,z
      %
      %
      %
      ; &\\[1ex]&
      \mdif[2]{x}{y}
      = \mdif[2]{x}{\left(
        \varphi(x)
        \,\int{z(x)\,\odif{x}}
      \right)}
      = z + z + x\,z'
      = 2\,z + x\,z'
    &
  \end{flalign*}
\end{exampleBox}

\begin{sectionBox}1{Wronskiano: check dependencia linear} % Sindex
  \begin{BM}[align*]
    W(f_1,f_2,\dots,f_n)(x)
    &= \det(w)
    ;&
    w \in \mathcal{M}_{n,m}
    : w_{i,j} &= \mdif[j]{x}\,f_i
  \end{BM}
  \vspace{-3ex}
  \begin{BM}
    W(f_1,f_2,\dots,f_n)(x)
    \begin{cases}
         = 0 &\text{ Linear depedent}
      \\ \neq 0 &\text{ Linear independent}
    \end{cases}
  \end{BM}

\end{sectionBox}

\begin{sectionBox}1{Método de variação das constantes abitrárias para equação linear de ordem \textit{n}} % Sindex
  \begin{BM}[flalign*]
    &
      % a_0(x) = a_1(x)
      % a_1(x) = a_1(x)
      % a_2(x) = a_2(x)
      % a_3(x) = a_3(x)
      % f(x)   = f(x)
      % y_1(x) = y_1(x)
      % y_2(x) = y_1(x)
      % y_3(x) = y_1(x)
      y:
      \begin{pmatrix}
          a_1(x)
        \\ + a_1(x)\,\mdif[1]{x}
        \\ + a_2(x)\,\mdif[2]{x}
        \\ + a_3(x)\,\mdif[3]{x}
      \end{pmatrix}
      \,y
      = f(x)
      %
      %
      %
      &\\[3ex]&
      y
      = c_1(x)\,y_1(x)
      + c_2(x)\,y_2(x)
      + c_3(x)\,y_3(x)
      %
      %
      %
      &\\[3ex]&
      \begin{Bmatrix}
        {
            c_1'(x)\,\mdif[0]{x}y_1(x) 
          + c_2'(x)\,\mdif[0]{x}y_2(x)
          + c_3'(x)\,\mdif[0]{x}y_3(x)
        } = 0
        \\ {
            c_1'(x)\,\mdif[1]{x}\,y_1(x) 
          + c_2'(x)\,\mdif[1]{x}\,y_2(x)
          + c_3'(x)\,\mdif[1]{x}\,y_3(x)
        } = 0
        \\ {
            c_1'(x)\,\mdif[2]{x}\,y_1(x) 
          + c_2'(x)\,\mdif[2]{x}\,y_2(x)
          + c_3'(x)\,\mdif[2]{x}\,y_3(x)
        } = \frac{f(x)}{a_3(x)}
      \end{Bmatrix}
    &
  \end{BM}
\end{sectionBox}

\begin{exampleBox}1{Metodo das var const arb} % Eindex
  \begin{BM}
    y" + 9\,y = 1/\cos(3\,x)
  \end{BM}
  \answer{}
  \begin{flalign*}
    &
      % a_0(x) = 9
      % a_1(x) = 0
      % a_2(x) = 1
      % f(x)   = \frac{1}{\cos(3\,x)}
      % y_1(x) = \cos(3\,x)
      % y_2(x) = \sin(3\,x)
      y:
      \begin{pmatrix}
          9
        \\ + 0\,\mdif[1]{x}
        \\ + 1\,\mdif[2]{x}
      \end{pmatrix}
      \,y
      = \frac{1}{\cos(3\,x)}
      %
      %
      %
      &\\[3ex]&
      y
      = c_1(x)\,y_1(x)
      + c_2(x)\,y_2(x)
      = c_1(x)\,\cos(3\,x)
      + c_2(x)\,\sin(3\,x)
      %
      %
      %
      ; &\\[3ex]&
      % Regra de Crammer
      c_1(x) 
      = \int{c_1'(x)\,\odif{x}}
      %
      %
      ; &\\[1ex]&
      c_2(x) 
      = \int{c_2'(x)\,\odif{x}}
      %
      %
      %
      &\\[3ex]&
      c_1'(x)
      = \frac{1}{W(\cos(3\,x),\sin(3\,x))}
      \,\begin{vmatrix}
        0 
        &  \mdif[0]{x}\sin(3\,x)
        \\ \frac{\frac{1}{\cos(3\,x)}}{1}
        &  \mdif[1]{x}\sin(3\,x)
      \end{vmatrix}
      %
      %
      %
      &\\[3ex]&
      c_2'(x)
      = \frac{1}{W(\cos(3\,x),\sin(3\,x))}
      \,\begin{vmatrix}
           \mdif[0]{x}\cos(3\,x)
        &  0 
        \\ \mdif[1]{x}\cos(3\,x)
        &  \frac{\frac{1}{\cos(3\,x)}}{1}
      \end{vmatrix}
      %
      %
      %
      &\\[3ex]&
      % Wronskiano
      W(\cos(3\,x),\sin(3\,x))
      = \det\begin{bmatrix}
           \mdif[0]{x}\,\cos(3\,x)
        &  \mdif[0]{x}\,\sin(3\,x)
        \\ \mdif[1]{x}\,\cos(3\,x)
        &  \mdif[1]{x}\,\sin(3\,x)
      \end{bmatrix}
      %
      %
      %
      &\\[3ex]&
      \begin{Bmatrix}
        {
            c_1'(x)\,\mdif[0]{x}y_1(x) 
          + c_2'(x)\,\mdif[0]{x}y_2(x)
        } = 0
        \\ {
            c_1'(x)\,\mdif[1]{x}\,y_1(x) 
          + c_2'(x)\,\mdif[1]{x}\,y_2(x)
        } = \frac{\frac{1}{\cos(3\,x)}}{1}
      \end{Bmatrix}
      = &\\&
      = \begin{Bmatrix}
        {
            c_1'(x)\,\mdif[0]{x}\cos(3\,x) 
          + c_2'(x)\,\mdif[0]{x}\sin(3\,x)
        } = 0
        \\ {
            c_1'(x)\,\mdif[1]{x}\,\cos(3\,x)
          + c_2'(x)\,\mdif[1]{x}\,\sin(3\,x)
        } = \frac{\frac{1}{\cos(3\,x)}}{1}
      \end{Bmatrix}
      = &\\&
      = \begin{Bmatrix}
        {
            c_1'(x)\,\mdif[0]{x}\cos(3\,x) 
          + c_2'(x)\,\mdif[0]{x}\sin(3\,x)
        } = 0
        \\ {
            c_1'(x)\,\mdif[1]{x}\,\cos(3\,x)
          + c_2'(x)\,\mdif[1]{x}\,\sin(3\,x)
        } = \frac{\frac{1}{\cos(3\,x)}}{1}
      \end{Bmatrix}
      %
      %
      %
      ; &\\[3ex]&
      \mdif[1]\,\cos(3\,x)
      = \mdif[1]\,\cos(3\,x)
      %
      %
      ; &\\[1ex]&
      \mdif[1]\,\sin(3\,x)
      = \mdif[1]\,\sin(3\,x)
    &
  \end{flalign*}
\end{exampleBox}

\end{document}

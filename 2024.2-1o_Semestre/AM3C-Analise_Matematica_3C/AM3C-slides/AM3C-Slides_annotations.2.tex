% !TEX root = ./AM3C-Slides_annotations.2.tex
\documentclass["AM3C-Slides_annotations.tex"]{subfiles}

% \tikzset{external/force remake=true} % - remake all

\begin{document}

% \graphicspath{{\subfix{./.build/figures/AM3C-Slides_annotations.2}}}
% \tikzsetexternalprefix{./.build/figures/AM3C-Slides_annotations.2/graphics/}

\mymakesubfile{2}[AM3C]
{Laplace Transform} % Subfile Title
{Laplace Transform} % Part Title 

\begin{sectionBox}*1m{Tables} % Sindex
  
  \begin{sectionBox}*2b{Proprierties} % Sindex
    \begin{align*}
      e^{a\,t}\,\sin{w\,t}
      &\xrightarrow{\Lapl} 
      \frac{w}{(s-a)^2+w^2}
      % 
      \\
      % 
      e^{a\,t}\,\cos{w\,t}
      &\xrightarrow{\Lapl} 
      \frac{s-a}{(s-a)^2+w^2}
      % 
      &
      % 
      t
      &\xrightarrow{\Lapl} 
      \frac{n!}{s^{n+1}}
      % 
      \\
      % 
      t^n\,e^{a\,t}, n\in\mathbb{N}^+
      &\xrightarrow{\Lapl} 
      \frac{n!}{(s-a)^{n+1}}
      % 
      &
      % 
      t\,\sin{w\,t}
      &\xrightarrow{\Lapl} 
      \frac{2\,s\,w}{(s^2+w^2)^2}
      % 
      \\
      % 
      t\,\cos{w\,t}
      &\xrightarrow{\Lapl} 
      \frac{s^2-w^2}{(s^2+w^2)^2}
      % 
      &
      % 
      t\,\sinh{w\,t}
      &\xrightarrow{\Lapl} 
      \frac{2\,s\,w}{(s^2-w^2)^2}
      % 
      \\
      % 
      t\,\cosh{w\,t}
      &\xrightarrow{\Lapl} 
      \frac{s^2+w^2}{(s^2-w^2)^2}
      % 
      &
      % 
      \frac{\sin{w\,t}}{t}
      &\xrightarrow{\Lapl} 
      \frac{\pi}{2}-\tan^{-1}{s/w}
      % 
      \\
      % 
      a\,f + b\,g 
      &\xrightarrow{\Lapl} 
      a\,F(s) + b\,G(s)
      % 
      &
      % 
      f(\lambda\,t) 
      &\xrightarrow{\Lapl} 
      \frac{1}{\lambda}F\left(\frac{s}{\lambda}\right)
      % 
      \\
      % 
      \Heavi(t-\tau)\,f(t-\tau) 
      &\xrightarrow{\Lapl} 
      e^{-s\,\tau}\,F(s)
      % 
      &
      % 
      e^{-\lambda\,t}\,f(t) 
      &\xrightarrow{\Lapl} 
      F(s+\lambda)
      % 
      \\
      % 
      f(t)/t 
      &\xrightarrow{\Lapl} 
      \int_{s}^{\infty}{F(p)\,\odif{p}}
      % 
      &
      % 
      (f \cdot g)(t) 
      &\xrightarrow{\Lapl} 
      F(s)\,G(s)
    \end{align*}
    Derivative transform
    \begin{align*}
      \Lapl(f')
      &=  s\,\Lapl(f) - f(0)
      \\
      \Lapl(\mdif[n]{t}f(t))
      &= s^n\,\Lapl(f)
      - \sum_{k=0}^{n-1}{
        s^{n-1-k}\,\mdif[k]{t}{f}(0)
      }
    \end{align*}
    Translations
    \begin{align*}
      \Lapl(e^{a\,t}\,f(t)) &= F(s-a)
      & \eqref{eq:Lapl(e^(a t) f(t))}
      ; \\
      \Lapl-1(F(s-a)) &= e^{a\,t}\,f(t)
      & \eqref{eq:Lapl-1(F(s-a))}
      ;\\
      \Lapl(f(t-a)\,\Heavi(t-a)) = e^{-a\,s}\,F(s)
      & \eqref{eq:Lapl(f(t-a) Heavi(t-a))}
      ; \\
      \Lapl^{-1}(e^{-a\,s}\,F(s)) = f(t-a)\,\Heavi(t-a)
      & \eqref{eq:Lapl-1(e^(-a s) F(s))}
    \end{align*}
  \end{sectionBox}

  \begin{sectionBox}*2b{Basic transforms} % Sindex
    \begin{align*}
      \Lapl(1)           & = 1/s ,                 & s & > 0;             &
      \Lapl(e^{a\,t})    & = \frac{1}{s-a} ,       & s & > a;            \\
      \Lapl(\cos(w\,t))  & = \frac{s}{s^2+w^2} ,   & s & > 0;             &
      \Lapl(\cosh(w\,t)) & = \frac{s}{s^2-w^2} ,   & s & > \max{(-w,w)}; \\
      \Lapl(\sin(w\,t))  & = \frac{w}{s^2+w^2} ,   & s & > 0;             &
      \Lapl(\sinh(w\,t)) & = \frac{w}{s^2-w^2} ,   & s & > \max{(-w,w)}; \\
    \end{align*}
    \begin{align*}
      \Lapl(t^n)         & = \frac{n!}{{s^{n+1}}}, & s & >0\land n\in\mathbb{N}^+;
    \end{align*}
  \end{sectionBox}

  \begin{sectionBox}*2b{Inverse Transforms} % Sindex
    \begin{align*}
      \Lapl^{-1}{\left(\frac{1}{(s-a)(s-b)}\right)} &= \frac{e^{a\,t}-e^{b\,t}}{a-b}
      ,& a \neq b \land s> \max{(a,b)}
    \end{align*}
  \end{sectionBox}

  \begin{sectionBox}*2b{Derivative Transforms} % Sindex
    \begin{align*}
      \Lapl(f') &= s\,\Lapl(f) - f(0)
      ; & s>\rho:\rho\text{ is the exponential order \ref{exponential order} of } f(x)
    \end{align*}
  \end{sectionBox}
\end{sectionBox}

\begin{sectionBox}*2m{Solving notable transforms} % Sindex

  \begin{sectionBox}*3b{Auxiliar demonstrations} % Sindex
    \begin{gather*}
      \prim_t{(e^{a\,t}\,\cos(b\,t))}
      =\frac{1}{b}\prim_t{(e^{a\,t}\,(b\,\cos(b\,t)))}
      = \mathText{\(
          \prim_t(u\,v')
          = u\,v - \prim_t(u'\,v)
          \begin{cases}
            u = e^{a\,t}
            \\
            v = \sin(b\,t)
          \end{cases}
      \)}
      = \frac{1}{b}
      (
        e^{a\,t}\,\sin(b\,t)
        - a\,\prim_t{\left(e^{a\,t}\,\sin(b\,t)\right)}
      )
      = \\
      = \frac{1}{b}
      \left(
        e^{a\,t}\,\sin(b\,t)
        + \frac{a}{b}\,\prim_t{\left(e^{a\,t}\,(-b\,\sin(b\,t))\right)}
      \right)
      = \mathText{\(
          \prim_t(u\,v')
          = u\,v - \prim_t(u'\,v)
          \begin{cases}
            u = e^{a\,t}
            \\
            v = \cos(e^{b\,t})
          \end{cases}
      \)}
      = \\
      = \frac{1}{b}
      \left(
        e^{a\,t}\,\sin(b\,t)
        + \frac{a}{b}\,\left(
          e^{a\,t}\,\cos(b\,t)
          - a\,\prim_t{(e^{a\,t}\,\cos(b\,t))}
        \right)
      \right)
      = \\
      =
      \frac{1}{b}\,e^{a\,t}\,\sin(b\,t)
      + \frac{a}{b^2}\,e^{a\,t}\,\cos(b\,t)
      - \frac{a^2}{b^2}\,\prim_t{(e^{a\,t}\,\cos(b\,t))}
      \implies \\
      \implies
      \prim_t{(e^{a\,t}\,\cos(b\,t))}
      = e^{a\,t}
      \,\frac{
        a\,\cos(b\,t)
        + b\,\sin(b\,t) 
      }{
        a^2+b^2
      }
      % 
      \yesnumber\label{eq:prim e^at cos(bt)}
    \end{gather*}

    \begin{gather*}
      \prim_t{\left(e^{a\,t}\,\sin(b\,t)\right)}
      = - \frac{1}{b}\prim_t{\left(e^{a\,t}\,(-b\,\sin(b\,t))\right)}
      = \mathText{\(
          \prim_t(u\,v')
          = u\,v - \prim_t(u'\,v)
          \begin{cases}
            u = e^{a\,t}
            \\
            v = \cos(b\,t)
          \end{cases}
      \)}
      = - \frac{1}{b}
      \left(
        e^{a\,t}\,\cos(b\,t)
        - a\,\prim_t{\left(e^{a\,t}\,\cos(b\,t)\right)}
      \right)
      = \\
      = - \frac{1}{b}
      \left(
        e^{a\,t}\,\cos(b\,t)
        - \frac{a}{b}\,\prim_t{\left(e^{a\,t}\,(b\,\cos(b\,t))\right)}
      \right)
      = \mathText{\(
          \prim_t(u\,v')
          = u\,v - \prim_t(u'\,v)
          \begin{cases}
            u = e^{a\,t}
            \\
            v = \sin(b\,t)
          \end{cases}
      \)}
      = - \frac{1}{b}
      \left(
        e^{a\,t}\,\cos(b\,t)
        - \frac{a}{b}\,\left(
          e^{a\,t}\,\sin(b\,t)
          - a\,\prim_t{\left(e^{a\,t}\,\sin(b\,t)\right)}
        \right)
      \right)
      = \\
      = 
      - \frac{1}{b}\,e^{a\,t}\,\cos(b\,t)
      + \frac{a}{b^2}\,e^{a\,t}\,\sin(b\,t)
      - \frac{a^2}{b^2}\,\prim_t{\left(e^{a\,t}\,\sin(b\,t)\right)}
      \implies \\
      \implies
      \prim_t{\left(e^{a\,t}\,\sin(b\,t)\right)}
      =
      e^{a\,t}
      \,\frac{
        a\,\sin(b\,t)
        - b\,\cos(b\,t)
      }{
        a^2+b^2
      }
      % 
      \yesnumber\label{eq:prim e^at sin(bt)}
    \end{gather*}
  \end{sectionBox}

  \subsection*{Transforms}

  \begin{sectionBox}*3b{\(\Lapl(1)\)} % Sindex
    \begin{gather*}
      \Lapl(1) 
      = \int_0^{\infty}{(
          e^{-s\,t}\,1\,\odif{t}
      )}
      = \lim_{k\to\infty}{
        \int_0^k{(
            e^{-s\,t}\,1\,\odif{t}
        )}
      }
      = \lim_{k\to\infty}{
        \left(
          -\frac{1}{s}\,e^{-s\,t}
        \right)
        \Big\vert_0^k
      }
      = \\
      = \lim_{k\to\infty}{ \left(
          -\frac{1}{s}\,e^{-s\,k}
          +\frac{1}{s}\,e^{-s\,0}
      \right) }
      = \frac{1}{s}
      %
      \yesnumber\label{eq:Lapl(1)}
    \end{gather*}
  \end{sectionBox}

  \begin{sectionBox}*3b{\(\Lapl\left(e^{a\,t}\right)\)} % Sindex
    \begin{gather*}
      \Lapl(e^{a\,t})
      = \lim_{k\to\infty}{\left(
        \int_{0}^{k}{
          e^{-s\,t}
          \,e^{a\,t}
          \,\odif{t}
        }
      \right)}
      = \lim_{k\to\infty}{\left(
        \int_{0}^{k}{
          e^{(a-s)\,t}
          \,\odif{t}
        }
      \right)}
      = \lim_{k\to\infty}{
        \left(
          \frac{e^{(a-s)\,t}}{a-s}
        \right)\Big\vert_0^k
      }
      = \\
      = \lim_{k\to\infty}{\left(
          \frac{e^{(a-s)\,k}}{a-s}
          - \frac{e^{(a-s)\,0}}{a-s}
      \right)}
      = \mathText{\(s>a\)}
      = -\frac{1}{a-s}
      = \frac{1}{s-a}
      %
      \yesnumber\label{eq:Lapl(e^(a t))}
      %
      ; \\
      \Lapl^{-1}(1/(s-a))
      = e^{a\,t}
      % 
      \yesnumber\label{eq:Lapl-1(1/(s-a))}
    \end{gather*}
  \end{sectionBox}

  \begin{sectionBox}*3b{\(\Lapl(\cos(w\,t))\)} % Sindex
    \begin{gather*}
      \Lapl(\cos(w\,t))
      = \lim_{k\to\infty}{\left(
          \int_0^k{
            e^{-s\,t}
            \,\cos(w\,t)
            \,\odif{t}
          }
      \right)}
      = \mathText{using \eqref{eq:prim e^at cos(bt)}}
      = \lim_{k\to\infty}{
        \left(
          e^{-s\,t}
          \,\frac{
            -s\,\cos(w\,t)
            + w\,\sin(w\,t) 
          }{
            s^2+w^2
          }
        \right)\Big\vert_0^k
      }
      = \\
      = \lim_{k\to\infty}{
        \left(
          e^{-s\,k}
          \,\frac{
            -s\,\cos(w\,k)
            + w\,\sin(w\,k) 
          }{
            s^2+w^2
          }
          - e^{-s\,0}
          \,\frac{
            -s\,\cos(w\,0)
            + w\,\sin(w\,0) 
          }{
            s^2+w^2
          }
        \right)
      }
      = \mathText{\(s>0\)}
      = \frac{ s }{ s^2+w^2 }
      % 
      \yesnumber\label{eq:Lapl(cos(w t))}
      % 
      ; \\
      \Lapl^{-1}{\left(
          \frac{ s }{ s^2+w^2 } 
      \right)}
      = \cos{w\,t}
      % 
      \yesnumber\label{eq:Lapl-1(s/(s^2+w^2))}
    \end{gather*}
  \end{sectionBox}

  \begin{sectionBox}*3b{\(\Lapl(\sin(w\,t))\)} % Sindex
    \begin{gather*}
      \Lapl(\sin(w\,t))
      = \lim_{k\to\infty}{\left(
          \int_0^k{
            e^{-s\,t}
            \,\sin(w\,t)
            \,\odif{t}
          }
      \right)}
      = \mathText{using \eqref{eq:prim e^at sin(bt)}}
      = \lim_{k\to\infty}{
        \left(
          e^{-s\,t}
          \,\frac{
            - s\,\sin(w\,t)
            - w\,\cos(w\,t)
          }{
            s^2+w^2
          }
        \right)\Big\vert_0^k
      }
      = \\
      = \lim_{k\to\infty}{
        \left(
          e^{-s\,k}
          \,\frac{
            - s\,\sin(w\,k)
            - w\,\cos(w\,k)
          }{
            s^2+w^2
          }
          - e^{-s\,0}
          \,\frac{
            - s\,\sin(w\,0)
            - w\,\cos(w\,0)
          }{
            s^2+w^2
          }
        \right)
      }
      = \mathText{\(s>0\)}
      = \frac{ w }{ s^2+w^2 }
      % 
      \yesnumber\label{eq:Lapl(sin(w t))}
    \end{gather*}
  \end{sectionBox}

  \begin{sectionBox}*3b{\(\Lapl{(\cosh(a\,t))}\)} % Sindex
    \begin{gather*}
      \Lapl(\cosh(a\,t))
      = \Lapl{(
          \frac{1}{2} \left( e^{a\,t} + e^{-a\,t} \right)
      )}
      = \frac{1}{2}\left(
        \Lapl{\left(e^{a\,t}\right)}
        + \Lapl{\left(e^{-a\,t}\right)}
      \right)
      = \mathText{using \eqref{eq:Lapl(e^(a t))}}
      = \frac{1}{2}\left(
        \frac{1}{s-a} +\frac{1}{s+a}
      \right)
      = \frac{s}{s^2-a^2}
      % 
      \yesnumber\label{eq:Lapl(cosh(a t))}
    \end{gather*}
  \end{sectionBox}

  \begin{sectionBox}*3b{\(\Lapl{(\sinh(a\,t))}\)} % Sindex
    \begin{gather*}
      \Lapl(\cosh(a\,t))
      = \Lapl{(
          \frac{1}{2} \left( e^{a\,t} - e^{-a\,t} \right)
      )}
      = \frac{1}{2}\left(
        \Lapl{\left(e^{a\,t}\right)}
        - \Lapl{\left(e^{-a\,t}\right)}
      \right)
      = \mathText{using \eqref{eq:Lapl(e^(a t))}}
      = \frac{1}{2}\left(
        \frac{1}{s-a} -\frac{1}{s+a}
      \right)
      = \frac{a}{s^2-a^2}
      % 
      \yesnumber\label{eq:Lapl(sinh(a t))}
    \end{gather*}
  \end{sectionBox}

  \subsection*{Inverse Laplace transform}

  \begin{sectionBox}*3b{\(\Lapl^{-1}\left(1/(s-a)(s-b)\right)\)}
    \begin{gather*}
      \Lapl^{-1}\left(
        \frac{1}{(s-a)(s-b)}
      \right)
      = \Lapl^{-1}\left(
        \frac{1}{a-b}
        \left(
          \frac{a-b+s-s}{(s-a)(s-b)}
        \right)
      \right)
      = \\
      = \frac{1}{a-b}
      \,\Lapl^{-1}\left(
        \left(
          \frac{s-b}{(s-a)(s-b)}
          - \frac{s-a}{(s-a)(s-b)}
        \right)
      \right)
      = \\
      = \frac{1}{a-b}
      \left(
        \Lapl^{-1}\left( \frac{1}{s-a} \right)
        - \Lapl^{-1}\left( \frac{1}{s-b} \right)
      \right)
      = \mathText{using 
        \eqref{eq:Lapl(e^(a t))}
        \(\land\, s>\max{(a,b)} \land a \neq b\)
      }
      = \frac{ e^{a\,t}-e^{b\,t} }{a-b}
      %
      \yesnumber\label{eq:Lapl-1(1/(s-a)(s-b))}
    \end{gather*}
  \end{sectionBox}

  \begin{sectionBox}*3b{\(\Lapl(t^n)\)} % Sindex

    \begin{gather*}
      \Lapl(t^n)
      = \lim_{k=\infty}{
        \int_0^k\left(
          e^{-s\,t}
          \,t^n
          \,\odif{t}
        \right)
      }
      = \mathText{\(
        \prim(u\,v')   
        = u\,v
        - \prim(u'\,v)
        \begin{cases}
          u = t^n
          \\
          v = -e^{-s\,t}/s
        \end{cases}
      \)}
      = \lim_{k=\infty}{
        \left(
          - \left(
            \frac{t^n\,e^{-s\,t}}{s}
          \right)\Bigg\vert_0^k
          - \int_0^k\left(
            \frac{e^{-s\,t}}{-s}
            \,n\,t^{n-1}
            \,\odif{t}
          \right)
        \right)
      }
      = \\
      = \lim_{k=\infty}{
        \left(
          - \frac{k^n\,e^{-s\,k}}{s}
          + \frac{0^n\,e^{-s\,0}}{s}
          - \int_0^k\left(
            \frac{e^{-s\,t}}{-s}
            \,n\,t^{n-1}
            \,\odif{t}
          \right)
        \right)
      }
      = \mathText{\(s>0\)}
      = \frac{n}{s}
      \lim_{k=\infty}{
        \left(
          \int_0^k\left(
            e^{-s\,t}
            \,t^{n-1}
            \,\odif{t}
          \right)
        \right)
      }
      = \frac{n}{s}
      \,\Lapl\left(t^{n-1}\right)
      = \frac{n}{s}
      \,\frac{n-1}{s}
      \,\Lapl\left(t^{n-2}\right)
      = \\
      = \prod_{i=0}^{n-1}{\left(
          \frac{n-i}{s}
      \right)}
      \,\Lapl(t^{n-n})
      = \frac{n!}{s^{n}}
      \,\Lapl(1)
      = \mathText{using \eqref{eq:Lapl(1)}}
      = \frac{n!}{s^{n}}
      \,\frac{1}{s}
      = \frac{n!}{s^{n+1}}
      % 
      \yesnumber\label{eq:Lapl(t^n)}
    \end{gather*}
  \end{sectionBox}
  
\end{sectionBox}

\begin{sectionBox}1m{Introduction} % Sindex
  
  \begin{BM}
    \Lapl{f(x)} = F(x)
  \end{BM}

  Let \(f(t)\) be a function of the real variable \(t\), for all \(t\in\mathbb{R}\); the values of \(f(t)\) may be either real or complex, although in our applications they will be real. The function \(f\) is said to be differetiable at tpnly finitely many points of \(I\), and all its points of discotinuity are jumps (i.e. there are right and left limits of the function at those points).

  \subsection*{Exploring the existence of the transform}
  We now introduce a class of functions for which the transformatio will be defined. We assume that the following three conditions are satisfied:\label{laplace conditions}
  \begin{enumerate}[label=(\arabic{enumi})]
    \item\label{laplace cond.1} \(t=0 \implies f(t)=0\)
    \item\label{laplace cond.2} \(f\) is piecewise differentialbe
    \item\label{laplace cond.3} there exist real numbers \(M,\rho\) such that
      \begin{BM}
        \myvert{f(t)} \leq M\,e^{\rho\,t}
        \quad \forall\,t\in\mathbb{R}
      \end{BM}
  \end{enumerate}
  \paragraph{note:} here \(\rho\) is said to be the \textit{exponential order} of \(f\) \label{exponential order}

  \begin{exampleBox}*2b{Checking if transform exists} % Eindex
    \begin{BM}[align*]
      \cosh{t}; && \sinh{t}; && t^n
    \end{BM}
    \answer{}
    \begin{gather*}
      \myvert{\cosh{t}} 
      = \myvert{\frac{e^t+e^{-t}}{2}}
      = \frac{e^t+e^{-t}}{2}
      \leq \frac{e^t+e^{t}}{2}
      = e^t
      ;\\
      \myvert{\sinh{t}}
      = \myvert{\frac{e^t-e^{-t}}{2}}
      \leq \frac{1}{2}\left(
        \myvert{e^t} + \myvert{e^{-t}}
      \right)
      = \frac{1}{2}\left(
        e^t + e^{-t}
      \right)
      \leq \frac{1}{2}\left(
        e^t + e^{t}
      \right)
      = e^t
      ; \\
      \myvert{t^n}
      = n!\,\frac{t^n}{n!}
      \leq n!\sum_{i=0}^{\infty}{
        \frac{t^i}{i!}
      }
      = n!\,e^t
    \end{gather*}
  \end{exampleBox}
  
  \begin{sectionBox}2b{The Heaviside function} % Sindex
    \begin{BM}
      \Heavi(t) = \begin{cases}
        0, & t < 0
        \\
        1, & t \geq 0
      \end{cases}
      % 
      \yesnumber\label{eq:heaviside function}
    \end{BM} 
    
    \begin{center}
      \pgfplotsset{
        width={.5\linewidth},
        height={15ex},
      }
      \begin{tikzpicture}
        \begin{axis}
          [
            minor tick=false,
            ytick={0,0.5,1},
            % legend pos={north west},
            axis on top,
            axis lines=left,
            ylabel={\(\Heavi(t)\)},
            xlabel={\(t\)},
          ]

          \addplot[
              const plot, no marks, thick
            ] coordinates {
              ( 0,0.0)
              ( 1,1.0)
              ( 2,1.0)
            };

        \end{axis}
      \end{tikzpicture}
    \end{center}

    Like any bounded function, this satisfies condition \ref{laplace cond.3} with \(\rho=0\).  
    Any function \(\phi(t)\) that fails to satisfy conditions \ref{laplace cond.1}, but does satisfy conditions \ref{laplace cond.2} and \ref{laplace cond.3}, then the function \(f(t)=\Heavi(t)\,\phi(t)\) will satisfy all three conditions.
    for Example
    \begin{BM}[align*]
      \Heavi(t)\,\sin{w\,t}
      ,&& \Heavi(t)\,t^n
      ,&& \Heavi(t)\,e^{a\,t}
    \end{BM}
    For simplicity we usually omit the factor \(\Heavi(t)\)

    \begin{sectionBox}*2b{} % Sindex
      \begin{BM}
        \tilde{f}(t) = f(t-a)\,\Heavi(t-a)
      \end{BM}
    \end{sectionBox}
  \end{sectionBox}


  \subsection*{Uses for the Heaviside function}
  Let \(f(t)\) be a function on the interval \(t\geq0\), and let \(f_1(t)\) be a ``piece'' of \(f(t)\) on the interval \(\myrange r{a,b}, a \geq 0\), that is
  \begin{BM}
    f_1(t)
    \begin{cases}
      f(t), & t\in\myrange r{a,b}
      \\
      0, & c.c.
    \end{cases}
  \end{BM}
  To set the value of \(f_1(t)\) to zero for \(t<0\), we multiply \(f(t)\) by \(\Heavi(t-a)\). To get zero for \(t \geq b\) we can substrac from \(f(t)\) the values \(f(t)\) as \(t \geq b\), that is subtract \(\Heavi(t-b)\,f(t)\). thus
  \begin{BM}
    f_1(t) = (\Heavi(t-a)-\Heavi(t-b))f(t)
  \end{BM}

  \begin{center}
    \pgfplotsset{
      width={.5\linewidth},
      height={15ex},
    }
    \begin{tikzpicture}
      \begin{axis}
        [
          minor tick=false,
          xtick=\empty,
          ytick=\empty,
          % legend pos={north west},
          % hide axis,
          axis on top,
          axis lines=left,
          ylabel={\(f_1(t)/f(t)\)},
          xlabel={\(t\)},
        ]

        \addplot[
            const plot, no marks, thick, Graph
          ] coordinates {
            ( 0.0,0.0)
            ( 0.3,1.0)
            ( 0.8,0.0)
            ( 1.0,0.0)
          };
          \node[above,rotate=90] at (0.3,0.5) {a};
          \node[above,rotate=90] at (0.8,0.5) {b};

      \end{axis}
    \end{tikzpicture}
  \end{center}

\end{sectionBox}

\begin{exampleBox}1m{} % Eindex
  Using the Heaviside function, write down the piecewise definition of the function
  \begin{BM}
    f(t)
    \begin{cases}
      0,   & 0\leq t< 2 \\
      3\,t & 2\leq t< 4, \\
      2,   & t \geq 4
    \end{cases}
  \end{BM}

  \answer{}
  \begin{tcolorbox}
    \begin{gather*}
      f(t) = 
      \begin{pmatrix}
        + 3\,t(\Heavi(t-2)-\Heavi(t-4))
        \\
        + 2\,(\Heavi(t-4))
      \end{pmatrix}
    \end{gather*}
  \end{tcolorbox}
\end{exampleBox}

\begin{sectionBox}1m{Laplace Transform of the Derivative} % Sindex
  
  \subsection*{For the first derivative}
  Suppose that \(f(x)\) follows all three laplace conditions \ref{laplace conditions} and has exponential order \ref{exponential order} \(\gamma\)
  \begin{align*}
    \Lapl(f') = s\,\Lapl(f) - f(0), \quad s>\gamma
  \end{align*}

  \subsection*{For the n-th derivative}
  Suppose that \(\mdif[i]{t}{f}\,\forall\,i\) follows all three laplace conditions \ref{laplace conditions} and has exponential order \ref{exponential order} \(\gamma\)
  \begin{BM}
    \Lapl\left(\mdif[n]{t}{f}\right) 
    = s^n\,\Lapl(f) 
    - \sum_{i=1}^{n}{
      s^{n-i}\,\mdif[i-1]{t}{f}(0)
    }
    \yesnumber\label{eq:laplace transform derivative}
  \end{BM}


\end{sectionBox}

\begin{exampleBox}1m{} % Eindex
  Find the transforms using the derivative method
  \begin{BM}[align*]
    t^n; && \sin{w\,t} && \sin^2{t}
  \end{BM}

  \answer{}

  Solving for \(t^n\)
  \begin{gather*}
    \Lapl(\mdif[n+1]{t}{t^n})
    = \Lapl(0) = 0
    = \mathText{using \eqref{eq:laplace transform derivative}}
    = s^{n+1}\,\Lapl(t^n)
    - \sum_{i=1}^{n+1}{
      s^{n-i}\,\mdif[i-1]{t}{t^n}(0)
    }
    = s^{n+1}\,\Lapl(t^n)
    - n!
    \implies
    \Lapl(t^n)
    = \frac{n!}{s^{n+1}}
  \end{gather*}

  Solving for \(\sin{w\,t}\)
  \begin{gather*}
    \Lapl(\mdif[2]{t}\sin{w\,t})
    = \Lapl(
      -w^2\,\sin(w\,t)
    )
    = -w^2\,\Lapl(\sin(w\,t))
    = \mathText{using \eqref{eq:laplace transform derivative}}
    = s^2\,\Lapl(\sin{w\,t})
    - s\,\sin{w * 0}
    - w\,\cos{w*0}
    \implies
    \Lapl(\sin{w\,t}) = \frac{w}{s^2+w^2}
  \end{gather*}

  Solving for \(\sin^2(t)\)
  \begin{gather*}
    \Lapl(
      \mdif{t}{\sin^2{t}}
    )
    = \Lapl(
      2\,\sin(t)\,\cos(t)
    )
    = \Lapl(
      \sin(2\,t)
    )
    = \mathText{using \eqref{eq:Lapl(sin(w t))}}
    = \frac{2}{s^2+4}
    = \mathText{using \eqref{eq:laplace transform derivative}}
    = s\,\Lapl(\sin^2{t})-\sin(2*0)
    = s\,\Lapl(\sin^2{t})
    \implies
    \Lapl(\sin^2{t})
    = \frac{2/s}{s^2+4}
  \end{gather*}
\end{exampleBox}

\begin{exampleBox}1m{Applying to differential equations} % Eindex
  Considere o PVI
  \begin{BM}
    y'' + 4\,y' +3\,y = 0, \quad y(0)=3, y'(0)=1
  \end{BM}

  \answer{}

  Finding general solution
  \begin{tcolorbox}
    \begin{gather}
      y 
      = \Lapl^{-1}{Y}
      = \mathText{using \eqref{eq:e.3 Y}}
      = \Lapl^{-1}{\left(
          \frac{5}{s+1} + \frac{-2}{s+3}
      \right)}
      = 5\Lapl(1/(s+1)) -2\Lapl(1/(s+3))
      = \mathText{using \eqref{eq:Lapl(e^(a t))}}
      = 5\,e^{-1\,t} -2\,e^{-3\,t}
    \end{gather}
  \end{tcolorbox}

  Checking existence of Laplace transform
  \begin{tcolorbox}
    \begin{gather*}
      y_1 = 5\,e^{-t}
      \\
      y_2 = -2\,e^{-3\,t}
      \\
      s>\max{(-1,-3)} = -1
      \\
      \lambda =\max{(-1,-3)} = -1
    \end{gather*}
    Both follow the conditions \ref{laplace conditions} with greater exponential being \(-1\)
  \end{tcolorbox}

  Finding \(Y\)
  \begin{tcolorbox}
    \begin{gather*}
      \Lapl(
        y'' + 4\,y' +3\,y
      )
      = \Lapl(y'')
      + 4\,\Lapl(y')
      + 3\,\Lapl(y)
      = \\
      = s^2\,\Lapl(y) - s\,y(0) - y'(0)
      + 4\,(s\,\Lapl(y) - y(0))
      + 3\,\Lapl(y)
      = \\
      = s^2\,\Lapl(y) 
      + 4\,s\,\Lapl(y) 
      - 13 - s\,3
      + 3\,\Lapl(y)
      = 0
      \implies \\
      \implies
      \Lapl(y) 
      = \frac{13 + s\,3}{ s^2 + 4\,s + 3 }
      = \mathText{\(
          s
          = \frac{
            - 4
            \pm \sqrt{
              4^2
              -4*1*3
            }
          } {2\,1}
          = - 2 \pm 1
      \)}
      = \frac{13 + s\,3}{(s+1)(s+3)}
      = \frac{A}{s+1}
      + \frac{B}{s+3}
      %
      \yesnumber\label{eq:e.3 Y constants}
      %
      = \mathText{using \eqref{eq:e.3 const of Y}}
      = 
      \frac{5}{s+1} + \frac{-2}{s+3}
      % 
      \yesnumber\label{eq:e.3 Y}
    \end{gather*}
  \end{tcolorbox}

  Finding constants in \eqref{eq:e.3 Y constants}
  \begin{tcolorbox}
    \begin{gather*}
      13 + s\,3
      = A(s+3)
      + B(s+1)
      = (A+B)s+3\,A+B
      \implies \\
      \implies
      \begin{cases}
        B  = 13-3\,A = 13-15 = -2
        \\
        A + (13-3\,A) = 3 \implies A=10/2=5
      \end{cases}
      \yesnumber\label{eq:e.3 const of Y}
    \end{gather*}
  \end{tcolorbox}
\end{exampleBox}

\begin{sectionBox}1m{Laplace transform of an Integral} % Sindex
  \begin{BM}
    \Lapl{\left(
        \int_0^t{f(x)\,\odif{x}}
    \right)}
    = \frac{1}{s}\,\Lapl(f(t))
    % 
    \yesnumber \label{eq:Lapl(int_0^t(f))}
    % 
    ; \\
    \Lapl^{-1}(f(t)/s)
    = \int_0^t{\left(f(x)\,\odif{x}\right)}
    % 
    \yesnumber\label{eq:Lapl-1(F(s)/s)}
  \end{BM}
\end{sectionBox}

\begin{exampleBox}1m{} % Eindex
  Find the inverse Laplace transform of
  \begin{BM}
    F(s) 
    = \frac{1}{s(s^2+w^2)}
  \end{BM}

  \answer{}

  \begin{tcolorbox}
    \begin{gather*}
      \Lapl^{-1}{(F(s))}
      = \Lapl^{-1}\left(
        \frac{1}{s(s^2+w^2)}
      \right)
      = \Lapl^{-1}\left(
        \frac{1}{s}
        \,\frac{1}{w}
        \,\frac{w}{s^2+w^2}
      \right)
      = \mathText{using \eqref{eq:Lapl(sin(w t))}}
      = \Lapl^{-1}\left(
        \frac{1}{s}
        \frac{1}{w}
        \Lapl{(
            \sin{w\,t}
        )}
      \right)
      = \mathText{using \eqref{eq:Lapl(int_0^t(f))}}
      = \frac{1}{w}
      \,\int_0^t{(
          \sin{w\,x}\odif{x}
      )}
      = \frac{1}{w}
      \,(
        -\cos(w\,x)/w
      )\Bigg\vert_0^t
      = \\
      = \frac{1}{w^2}
      \,(
        -\cos(w\,t)
        +\cos(w\,0)
      )
      = \frac{1-\cos{w\,t}}{w^2}
    \end{gather*}
  \end{tcolorbox}
\end{exampleBox}

\begin{sectionBox}1m{Translação da variavel \(s\)} % Sindex
  \begin{BM}
    \Lapl(f(t))
    =F(s), \quad s\in\myrange*{\gamma,\infty}
    \implies
    \Lapl(e^{a\,t}\,f(t))
    = F(s-a), \quad s\in\myrange*{a+\gamma,\infty}
    %
    \yesnumber\label{eq:Lapl(e^(a t) f(t))}
    %
    ; \\
    \Lapl^{-1}(F(s-a)) = e^{a\,t}\,f(t)
    \yesnumber\label{eq:Lapl-1(F(s-a))}
  \end{BM}
\end{sectionBox}

\begin{exampleBox}1m{} % Eindex
  Consider the problem with inital values
  \begin{BM}
    y'' + 2\,y' + 5\,y = 0, \quad y(0)=2,\quad y'(0)=-4
  \end{BM}
  Find the general solution

  \answer{}
  
  General solution for \(y\)
  \begin{tcolorbox}
    \begin{gather}
      y = \Lapl^{-1}{Y}
      = \mathText{using \eqref{eq:e.4 Y}}
      = \Lapl^{-1}{\left(
          2\,\frac{s + 1}{
            (s + 1)^2 + 2^2
          }
          - \frac{2}{
            (s + 1)^2 + 2^2
          }
      \right)}
      = \\
      = 2\,\Lapl^{-1}{\left(
          \frac{s + 1}{ (s + 1)^2 + 2^2 }
      \right)}
      - \Lapl^{-1}{\left(
          \frac{2}{ (s + 1)^2 + 2^2 }
      \right)}
      = \mathText{using \eqref{eq:Lapl-1(s/(s^2+w^2))}}
      = 2\,e^{-t}\,\cos(2\,t)
      - e^{-t}\,\sin(2\,t)
    \end{gather}
  \end{tcolorbox}

  Finding \(Y\)
  \begin{tcolorbox}
    \begin{gather*}
      \Lapl\left(
        y'' + 2\,y' + 5\,y
      \right)
      = \Lapl( y'' )
      +2\,\Lapl( y' )
      + 5\,\Lapl( y )
      = \mathText{using \eqref{eq:laplace transform derivative}}
      = s^2\,Y-s\,y(0)-y'(0)
      +2\,(s\,Y-y(0))
      + 5\,Y
      = \\
      = s^2\,Y
      - s\,(2)
      - (-4)
      + 2\,s\,Y
      - 2\,(2)
      + 5\,Y
      = 0
      \implies \\
      \implies
      = 
      Y
      = \frac{s\,2}{
        s^2 + 2\,s + 5
      }
      = \frac{2\,(s + 1 - 1)}{
        (s + 1)^2 + 4
      }
      = 2\,\frac{s + 1}{
        (s + 1)^2 + 2^2
      }
      - \frac{2}{
        (s + 1)^2 + 2^2
      }
      % 
      \yesnumber\label{eq:e.4 Y}
    \end{gather*}
  \end{tcolorbox}

\end{exampleBox}

\begin{sectionBox}1m{Translation of the variable \(t\)} % Sindex
  \begin{BM}
    \Lapl(f(t)) = F(s), s\in\myrange*{\gamma,\infty}
    \implies
    \Lapl\left(
      f(t-a)\,\Heavi(t-a)
    \right) = e^{-a\,s}\,F(s)
    % 
    \yesnumber\label{eq:Lapl(f(t-a) Heavi(t-a))}
    % 
    ; \\
    \Lapl^{-1}(e^{-a\,s}\,F(s))
    = f(t-a)\,\Heavi(t-a)
    % 
    \yesnumber\label{eq:Lapl-1(e^(-a s) F(s))}
  \end{BM}
\end{sectionBox}

\begin{exampleBox}1m{} % Eindex
  \begin{BM}
    F(s) 
    = \frac{e^{-3\,s}}{s^3}
  \end{BM}

  \answer{}
  

  \begin{tcolorbox}
    \begin{gather*}
      \Lapl^{-1}(F(s))
      = \Lapl^{-1}{\left(
          \frac{e^{-3\,s}}{s^3}
      \right)}
      = \Lapl^{-1}{\left(
          \frac{e^{-3\,s}}{2}
          \,\frac{2}{s^3}
      \right)}
      = \mathText{using \eqref{eq:Lapl(t^n)}}
      = \frac{1}{2}
      \,\Lapl^{-1}{\left(
          e^{-3\,s}
          \Lapl(t^2)
      \right)}
      = \mathText{using \eqref{eq:Lapl-1(e^(-a s) F(s))}}
      = 
      \frac{
        (t-3)^2\,\Heavi(t-3)
      }{2}
    \end{gather*}
  \end{tcolorbox}
\end{exampleBox}

\begin{exampleBox}1m{} % Eindex
  Find the laplace transform of the function
  \begin{BM}
    f(t)
    = \begin{cases}
      1, & 0<t<\pi \\
      0, & \pi<t<2\,\pi \\
      \sin{t}, & t>2\,pi
    \end{cases}
  \end{BM}

  \answer{}

  Solving lagplace transform of \(f\)
  \begin{tcolorbox}
    \begin{gather*}
      \Lapl(f(t))
      = \Lapl(
        1(\Heavi(t-0)-\Heavi(t-\pi))
        + \sin(t)\Heavi(t-2\,\pi)
      )
      = \\
      = \Lapl( 1)
      - \Lapl( \Heavi(t-\pi) )
      + \Lapl( \sin(t)\Heavi(t-2\,\pi))
      = \mathText{using \eqref{eq:Lapl(sin(w t))}}
      = \frac{1}{s}
      - \frac{e^{-\pi\,s}}{s}
      + \frac{e^{-2\,\pi\,s}}{s^2+1}
    \end{gather*}
  \end{tcolorbox}

\end{exampleBox}

\begin{exampleBox}1m{} % Eindex
  Consider the problem of initial values
  \begin{BM}
    y'' + 3\,y' +2\,y = r(t)
    , \quad y(0) = 0, y'(0) = 0
    ; \\
    r(t) = \begin{cases}
      1, & 0<t<1
      \\
      0, & t>1
    \end{cases}
  \end{BM}

  \answer{}

  Finding \(y\)
  \begin{tcolorbox}
    \begin{gather*}
      y = \Lapl^{-1}{Y}
      = \mathText{using \eqref{eq:e.8 Y}}
      = \Lapl^{-1}{\left(
          (1-e^{-s})\left(
            \frac{1/2}{s}
            + \frac{-1}{s+1}
            + \frac{1/2}{s+2}
          \right)
      \right)}
      = \\
      = \Lapl^{-1}{\left(
          \frac{1/2}{s}
          + \frac{-1}{s+1}
          + \frac{1/2}{s+2}
          -e^{-s}\left(
            \frac{1/2}{s}
            + \frac{-1}{s+1}
            + \frac{1/2}{s+2}
          \right)
      \right)}
      = \mathText{using 
        \eqref{eq:Lapl-1(1/(s-a))}
        \eqref{eq:Lapl-1(e^(-a s) F(s))}
      }
      = \frac{1}{2}
      - e^{-1\,t}
      + \frac{1}{2}\,e^{-2\,t}
      -\Heavi(t-1)\left(
        \frac{1}{2}
        -e^{-1\,t}
        + \frac{1}{2}\,e^{-2\,t}
      \right)
    \end{gather*}
  \end{tcolorbox}

  
  \begin{tcolorbox}
    \begin{gather*}
      \Lapl(
        y'' + 3\,y' +2\,y 
      )
      = \Lapl( y'')
      + 3\,\Lapl( y' )
      + 2\,\Lapl( y )
      = \mathText{using \eqref{eq:laplace transform derivative}}
      = s^2\,Y - s\,y(0) - y'(0)
      + 3\,(s\,Y - y(0))
      + 2\,Y
      = Y( s^2 + 3\,s + 2)
      = \\
      = \Lapl(r(t))
      = \Lapl(
        \Heavi(t-0)
        -\Heavi(t-1)
      )
      = \Lapl( 1 -\Heavi(t-1))
      = \frac{1}{s}
      - \frac{e^{-1\,s}}{s}
      \implies \\
      \implies
      Y
      = \frac{1-e^{-s}}{
        s(s^2 + 3\,s + 2)
      }
      = \frac{1-e^{-s}}{
        s(s(s + 1)+ 2(s + 1))
      }
      = \\
      = \frac{1-e^{-s}}{
        s(s+2)(s+1)
      }
      = (1-e^{-s})\left(
        \frac{A}{s}
        + \frac{B}{s+1}
        + \frac{C}{s+2}
      \right)
      %
      \yesnumber\label{eq:e.8 Y const}
      %
      = \mathText{using \eqref{eq:e.8 const of Y}}
      = (1-e^{-s})\left(
        \frac{1/2}{s}
        + \frac{-1}{s+1}
        + \frac{1/2}{s+2}
      \right)
      \yesnumber\label{eq:e.8 Y}
    \end{gather*}
  \end{tcolorbox}

  Finding constants in \eqref{eq:e.8 Y const}
  \begin{tcolorbox}
    \begin{gather*}
      1
      = (A+C+B)\,s^2
      + s\,(3\,A+2\,B+C)
      + A\,2
      \implies \\
      \begin{cases}
        A = 1/2
        \\
        C = -B-A = -B-1/2 = -(-1)-1/2 = 1/2
        \\
        3\,(1/2)+2\,B+(-B-1/2) = 0
        \implies
        B = -1
      \end{cases}
      \yesnumber\label{eq:e.8 const of Y}
    \end{gather*}
  \end{tcolorbox}


\end{exampleBox}

\end{document}

% !TEX root = ./AM3C.exercises_resolutions.2024.1.tex
\documentclass["./AM3C.exercises_resolutions.2024.tex"]{subfiles}

% \tikzset{external/force remake=true} % - remake all

\begin{document}

\graphicspath{{\subfix{.build/figures/AM3C.exercises_resolutions.2024.1}}}
% \tikzsetexternalprefix{./figures/AM3C.exercises_resolutions.2024.1/graphics/}

\mymakesubfile{1}
[AM 3C]
{Equações Diferenciais Ordinárias} % Subfile Title
{Equações Diferenciais Ordinárias} % Part Title

\begin{questionBox}1{ % MARK: Q1
	Verifique que cada uma das funções indicadas é solução da equação diferencial considerada.
	} % Q1
\end{questionBox}

\begin{questionBox}2{ % MARK: Q1.1
	\begin{BM}
		y(x) = e^{2\,x}\,\cos(3\,x), y"-4\,y'+13\,y=0
	\end{BM}
	} % Q1.1
	\answer{}
	\begin{flalign*}
		 &
		0
		= y" - 4\,y' + 13\,y
		= (
		e^{2\,x}\,\cos{ (3\,x) }
		)" - 4\,(
		e^{2\,x}\,\cos{ (3\,x) }
		)'
		+ 13\,y
		= (
		e^{2\,x}(-\sin{ (3\,x) }\,3)
		+ 2\,e^{2\,x}\,\cos{ (3\,x) }
		)'
		- 4\,(
		e^{2\,x}(-\sin{ (3\,x) }\,3)
		+ 2\,e^{2\,x}\,\cos{ (3\,x) }
		)
		+ 13\,y
		= \dots
		 &
	\end{flalign*}
\end{questionBox}
\begin{questionBox}2{ % MARK: Q1.2
	\begin{BM}
		y(x)
		= e^{-x^2}\int_0^x{e^{t^2}\,\odif{t}}
		+ c_1\,e^{-x^2}
		,\quad \odv{y}{x}+2\,x\,y=1
	\end{BM}
	} % Q1.2
	\begin{BM}
		\odv{}{x}
		\int_{a(x)}^{b(x)}{
		f(t)\,\odif{t}
		}
		= b'(x)\,f(b(x))
		- a'(x)\,f(a(x))
	\end{BM}
	\begin{flalign*}
		  &
		1
		= \odv{y}{x} + 2\,x\,y
		= \odv{\left(
			e^{-x^2}\int_0^x{e^{t^2}\,\odif{t}}
			+ c_1\,e^{-x^2}
			\right)}{x}
		+ 2\,x\,\left(
		e^{-x^2}\int_0^x{e^{t^2}\,\odif{t}}
		+ c_1\,e^{-x^2}
		\right)
		= \left(
		(
		e^{-x^2}\,(-2\,x)
		\,\int_0^x{e^{t^2}\,\odif{t}}
		)
		+ (
		e^{-x^2}
		(
		1*e^{x^2}
		-0*(e^{0^2}\,\odif{0})
		)
		)
		+ c_1\,(
		e^{-x^2}
		\,(-2\,x)
		)
		\right)
		+ 2\,x\,(
		e^{-x^2}\int_0^x{e^{t^2}\,\odif{t}}
		+ c_1\,e^{-x^2}
		)
		= & \\&
		= \dots
		  &
	\end{flalign*}
\end{questionBox}

\begin{questionBox}1{ % MARK: Q2
	Mostre que a equação \(2\,x^2\,y - y^2 + 1 = 0\) define implicitamente uma solução da equação diferencial
	\begin{BM}
		(x^2-y)\,\odv{y}{x}+2\,x\,y = 0
	\end{BM}
	Determine explicitamente a solução que verifica a condição \(y(0)=1\).
	} % Q2
	\answer{}
	% \begin{flalign*}
	%   &
	%     0
	%     = (x^2-y)\,\odiv{y}{x}+2\,x\,y
	%     ; &\\[3ex]&
	%     \odv{}{x}\left(2\,x^2\,y - y^2 + 1\right)
	%     = \left(
	%       2*2\,x\,y
	%       + 2\,x^2\,y'
	%     \right)
	%     - \left(
	%       2\,y\,\odiv{y}{x}
	%     \right)
	%     = 4\,x\,y
	%     + 2\,\odiv{y}{x}(
	%       x^2-y
	%     )
	%     = 0
	%     \implies &\\&
	%     \implies
	%     2\,x\,y
	%     + odiv{y}{x}(
	%       x^2-y
	%     )
	%     = 0
	%     % \odiv{y}{x} = \frac{-4\,x\,y}{2\,(x^2-y)}
	%     % 
	%     % 
	%     % 
	%     ; &\\[3ex]&
	%     \text{Equação de 2º grau em }y:
	%     &\\&
	%     \dots
	%   &
	% \end{flalign*}
\end{questionBox}

\begin{questionBox}1{ % MARK: Q3
	Determine o valores de \(k\) para os quais:
	} % Q3
	\begin{questionBox}2{ % MARK: Q3.1
		\(y(x)=\exp{(k\,x)}\) é solução da equação
		\begin{BM}
			y"-y'+6\,y=0
		\end{BM}
		} % Q3.1
	\end{questionBox}
	\begin{questionBox}2{ % MARK: Q3.2
		\(y(x)=x^k\) é solução da equação
		\begin{BM}
			x\,y"+2\,y'=0.
		\end{BM}
		} % Q3.2
		\answer{}
		\begin{flalign*}
			         &
			0
			= x\,y"+2\,y'
			= x\,(
			x^k
			)"+2\,(
			x^k
			)'
			= x\,(
			k\,\,(k-1)\,x^(k-2)
			)+2\,(
			k\,x^(k-1)
			)
			= x^{k-1}\,k\,(
			k+1
			)
			\implies & \\&
			\implies
			k = - 1 \lor k = 0
			         &
		\end{flalign*}
	\end{questionBox}
\end{questionBox}

\begin{questionBox}1{ % Q4
	% Em cada uma das seguintes equaçoes aut´onomas determine os pontos de equil´ıbrio e represente o respetivo retrato de fase. Classifique os pontos cr´ıticos e represente graficamente os diferentes tipos de solu¸c˜oes em cada uma das regi˜oes determinadas pelas solu¸c˜oes de equil´ıbrio.
	} % Q4
\end{questionBox}

\begin{questionBox}2{ % Q4.1
	\begin{BM}
		\odv{y}{x}
		= y^2-3\,y
	\end{BM}
	} % Q4.1
	\answer{}
	\begin{flalign*}
		 &
		\odv{y}{x}
		= y^2-3\,y
		= y(y-3)
		= 0
		 &
	\end{flalign*}
	\begin{center}
		\vspace{1ex}
		\begin{tabular}{L *5{C}}
			\toprule

			 & \dots & 0 & \dots & 3 & \dots
			\\\midrule

			y
			 & -     & 0 & +     & + & +
			\\ y-3
			 & -     & - & -     & 0 & +
			\\\midrule
			y(y-3)
			 & +     & 0 & -     & 0 & +

			\\\bottomrule
		\end{tabular}
		\vspace{2ex}
	\end{center}
\end{questionBox}

\part*{Equações Diferenciais lienares de primeira ordem e Equações Redutíveis a Lineares de Primeira ordem}

\setcounter{question}{5}

\begin{questionBox}2{ % Q5.1
	Determine a solução geral da equação diferencial linear homogénea de primeira ordem
	\begin{BM}
		\odv{y}{x}+2\,x\,y=0
	\end{BM}
	} % Q5.1
	\answer{}
	\begin{flalign*}
		         &
		\odv{y}{x}+2\,x\,y=0
		\implies & \\&
		\implies
		y
		= \gamma{(x)}^{-1}
		\,C
		+ \gamma{(x)}^{-1}
		\,\int{\exp{a(x)}*b(x)\,\odif{x}}
		= \left(
		\int{\exp{(2\,x)}}
		\right)^{-1}
		\,C
		+ \gamma{(x)}^{-1}
		\,\int{\exp{a(x)}*0\,\odif{x}}
		= \exp{(-x^2)}
		\,C
		         &
	\end{flalign*}
\end{questionBox}

\begin{questionBox}1{ % Q6
	Determine a solu¸c˜ao geral das seguintes equa¸c˜oes diferenciais lineares de primeira
	ordem:
	} % Q6
\end{questionBox}

\begin{questionBox}2{ % Q6.1
	\begin{BM}
		\odv{y}{x}-y\,\tan{x}=\cos{x}
		, \quad
		x\in\myrange*{-\pi/2,\pi/2}
	\end{BM}
	} % Q6.1
\end{questionBox}

\begin{questionBox}2{ % Q6.2
	\begin{BM}
		y^2\,\odif{x}
		- (2\,x\,y+3)\,\odif{y}=0
	\end{BM}
	considere x como fun¸c˜ao inc´ognita e y vari´avel independente
	} % Q6.2
	\answer{}
	\begin{flalign*}
		         &
		y^2\,\odif{x}
		- (2\,x\,y+3)\,\odif{y}=0
		\implies & \\&
		x \text{ como função de } y
		         & \\&
		\implies
		y^2\,\odv{x}{y}
		- (2\,x\,y+3)=0
		\implies
		\odv{x}{y}
		=
		2\,x\,y^{-1}
		+ 3\,y^{-2}
		\implies & \\&
		\implies
		         &
	\end{flalign*}
\end{questionBox}

\begin{questionBox}1{ % Q7
	Determine a solução geral da equação
	\begin{BM}
		\odv{z}{x}
		+ (x-x^{-1})\,z
		=0
		,\quad
		x>0
	\end{BM}
	e utilizando o m´etodo da varia¸c˜ao das constantes arbitr´arias determine a solu¸c˜ao
	geral de
	\begin{BM}
		\odv{z}{x}
		+ (x-x^{-1})\,z
		= - x^2
	\end{BM}
	} % Q7
	\answer{}
	\begin{flalign*}
		         &
		\odv{z}{x}
		+ (x-x^{-1})\,z
		=0
		,\quad
		x>0
		\implies & \\&
		\implies
		z
		=c\,\gamma^{-1}(x)
		=c\,\exp{-\int{(x-x^{-1})\,\odif{x}}}
		= \dots
		=c\,\exp{-x^2/2}
		         &
	\end{flalign*}
	\begin{flalign*}
		  &
		\text{Variação das constantes arbitrárias}
		  & \\&
		\odv{z}{x}
		+ (x-x^{-1})\,z
		= - x^2
		\implies
		z(x)
		= c(x)\,x\,\exp{-x^2/2}
		; & \\&
		z'(x)
		= c'(x)\,x\,exp{-x^2/2}
		+c(x)\left(
		\exp{(-x^2/2)}
		-x^2\exp{(-x^2/2)}
		\right)
		  &
	\end{flalign*}
\end{questionBox}

\setcounter{question}{8}
\setcounter{subquestion}{1}

\begin{questionBox}2{ % Q8.2
	Utilizando a substituição definida por \(y=e^{4\,x}\), determine a solução geral da equação:
	\begin{BM}
		x\,\odv[order=2]{y}{x}
		-(4\,x+1)\,\odv{y}{x}
		+4\,y
		= 0
	\end{BM}
	} % Q8.2
	\begin{flalign*}
		         &
		( x\,D-(4\,x+1)\,D^2+4 )\,y
		=Py=0
		\implies & \\&
		\implies
		y=e^{4\,x}\,\int{z}
		\implies & \\&
		\implies
		% 
		% 
		% 
		;        & \\[3ex]&
		\odv{y}{x}
		=\odv{}{x}\left(
		e^{4\,x}
		\right)
		\int{z}
		+ \odv{}{x}\left(
		\int{z}
		\right)
		=4\,x\,e^{4\,x}
		\,\int{z}
		+ z\,e^{4\,x}
		% 
		% 
		% 
		;        & \\[3ex]&
		\odv[order=2]{y}{x}
		= \odv{}{x}\left(
		4\,x\,e^{4\,x}
		\,\int{z}
		+ z\,e^{4\,x}
		\right)
		=\dots
		% 
		% 
		% 
		         & \\[3ex]&
		\implies
		x\,e^{4\,x}
		\,\left(
		16\,\int{z}
		+8\,z
		\odv{z}{x}
		\right)
		-(4\,x+1)
		\,e^{4\,x}
		\left(
		4\,\int{z} +z
		\right)
		+4\,e^{4\,x}\,\int{z}
		=0
		\implies & \\&
		\implies
		\dots
		\implies & \\&
		\implies
		\odv{z}{x}+(4-1/x)\,z=0
		\implies & \\&
		\implies
		\dots
		z
		=\frac{C}{
			\exp{(\int{(4-1/x)})}
		}
		= \dots
		= C\,x\,e^{-4\,x}
		\implies & \\&
		\implies
		y
		=e^{4\,x}\,\int{C\,x\,e^{-4\,x}}
		= C\,e^{4\,x}\,\left(
		-\frac{x\,e^{-4\,x}}{4}
		-\frac{e^{-4\,x}}{16}
		+k
		\right)
		= \dots
		=        & \\&
		= -\frac{c}{4}\left(
		x+1/4
		\right)
		+c\,k\,e^{4\,x}
		= c_1\left(
		x+1/4
		\right)
		+c_2\,e^{4\,x}
		\quad\forall\,c_1,c_2\in\mathbb{R}
		         &
	\end{flalign*}
\end{questionBox}

\setcounter{question}{9}
\setcounter{subquestion}{2}

\begin{questionBox}2{
	}
	\answer{}
	\begin{flalign*}
		 &
		\odv{z}{x}
		+\frac{z}{2\,\sqrt{x}}
		=-\frac{z^3}{2}
		\implies
		z^{-3}\,\odv{z}{x}
		+\frac{z^{-1}}{2\,\sqrt{x}}
		=-\frac{1}{2}
		 &
	\end{flalign*}
\end{questionBox}

\setcounter{question}{10}
\setcounter{subquestion}{1}

\begin{questionBox}2{ % Q10.2
	\begin{BM}
		y'+x\,y^2-(2\,x^2+1)\,y+x^3+x-1=0
	\end{BM}
	} % Q10.2
	\answer{}
	\begin{flalign*}
		         &
		y'+x\,y^2-(2\,x^2+1)\,y+x^3+x-1=0
		\implies & \\&
		\implies
		y'
		+y\,\left(
		-(2\,x^2+1)
		\right)
		=\left(
		-x^3-x+1
		\right)
		+y^2\,\left(
		-x
		\right)
		         &
	\end{flalign*}

\end{questionBox}

\setcounter{question}{13}

\begin{questionBox}1{ % MARK:Q14
	Determine o integral geral de cada uma das seguintes equações diferenciaia lienares de coeficientes constantes:
	} % Q14
\end{questionBox}
\begin{questionBox}2{ % MARK:Q14.1
	\begin{BM}
		D^3\,(D+1)^2((D-5)^2+16)\,y=0
		,\quad\left(
		\text{em que} D=\odv{}{x}
		\right)
	\end{BM}
	} % Q14.1
	\answer{}
	\begin{flalign*}
		         &
		D^3\,(D+1)^2((D-5)^2+16)\,y=0
		\implies & \\&
		\implies
		\text{Raizes:}
		\begin{pmatrix}
			0 & 3
			\\ -1 & 2
			\\ 5\pm4\,i & 1
		\end{pmatrix}
		\implies & \\&
		\implies
		y
		= c_0
		+ c_1\,x
		+ c_2\,x^2
		+ c_3\,e^{-x}
		+ c_4\,x\,e^{-x}
		+ c_5\,e^{-5\,x}\,\cos{4\,x}
		+ c_6\,e^{-5\,x}\,\sin{4\,x}
		         &
	\end{flalign*}
\end{questionBox}

\begin{questionBox}2{ % MARK:Q14.2
	\begin{BM}
		(D^4-1)\,y=x^3-x+2
		,\quad\left(
		\text{em que }
		D=\odv{}{x}
		\right)
	\end{BM}
	} % Q14.2
	\answer{}
	\begin{flalign*}
		         &
		y=y_0+y_1
		=        & \\&
		= \left(
		c_1\,e^{x}
		+c_2\,e^{-x}
		+c_3\,\cos{x}
		+c_4\,\sin{x}
		\right)
		+ \left(
		x^3-x+2
		\right)
		%
		%
		%
		;        & \\[3ex]&
		y_0\text{ é solução de }
		(D^4-1)\,y=0
		\implies & \\&
		\implies
		(\alpha^4-1)
		= (\alpha^2+1)(\alpha^2-1)
		\implies & \\&
		\implies
		\text{Raizes:}
		\left\{
		\begin{tabular}{cc}
			\text{Raizes} & \text{Multiplicidade}
			\\1 & 1
			\\ -1 & 1
			\\ \pm i & 1
		\end{tabular}
		\right.
		         & \\&
		\implies
		y_0
		=c_1\,e^{x}
		+c_2\,e^{-x}
		+c_3\,\cos{x}
		+c_4\,\sin{x}
		%
		%
		%
		;        & \\[4ex]&
		y_1\text{ é \(x^p\) vezes um polinomio de mesmo grau que }
		x^3-x+2
		         & \\&
		p\text{ é a multiplicidade da raiz}\alpha=0
		\implies
		\implies & \\&
		y_1
		=x^p\left(
		a_0
		+ a_1\,x
		+ a_2\,x^2
		+ a_3\,x^3
		\right)
		\implies & \\&
		\implies
		(D^4+1)\,y_1
		= 0+y_1
		= y_1
		= x^3-x+2
		         &
	\end{flalign*}
\end{questionBox}

\begin{questionBox}2{ % MARK:Q14.3
	\begin{BM}
		y"+y=\cos{x}
	\end{BM}
	} % Q14.3
	\answer{}
	\begin{flalign*}
		         &
		y"+y
		= (D^2+1)\,y
		=\cos{x}
		\implies & \\&
		\implies
		y=y_0+y_1
		= \left(
		c_1\,\cos{x}
		+c_2\,\sin{x}
		\right)
		+ \left(
		\frac{x}{2}\,\sin{x}
		\right)
		%  
		%  
		%  
		;        & \\[3ex]&
		\alpha^2+1=0
		\implies & \\&
		\implies
		\begin{tabular}{lc}
			\text{Raizes}
			 & \pm 1
			\\ \text{Multiplicidade}
			 & 1
		\end{tabular}
		         & \\&
		\implies
		y_0
		=c_1\,\cos{x}
		+c_2\,\sin{x}
		%
		%
		%
		         & \\[3ex]&
		y_1
		= x^p\,\left(
		r\,e^{a\,x}\,\cos{(b\,x)}
		+ s\,e^{a\,x}\,\sin{(b\,x)}
		\right)
		= x\,\left(
		r\,\cos{x}
		+ s\,\sin{x}
		\right)
		\implies & \\&
		\implies
		\odv{y_1}{x}
		= r\,\cos{x}
		+ s\,\sin{x}
		= x\,\left(
		- r\,\sin{x}
		+ s\,\cos{x}
		\right)
		\implies & \\&
		\implies
		\odv[order=2]{y_1}{x}
		= \dots
		\implies & \\&
		\implies
		(D^2+1)y_1
		= -2\,r\,\sin{x}
		+ 2\,s\,\cos{x}
		= \cos{x}
		\implies
		s=1/2\land r=0
		\implies & \\&
		\implies
		y_1=\frac{x}{2}\,\sin{x}
		         &
	\end{flalign*}
\end{questionBox}

\setcounter{question}{14}
\begin{questionBox}1{ % MARK:Q15
	Determine a sol geral da equação diferencial linear homogénea de coeficientes consntantes
	\begin{BM}
		\mdif*[order=2]{y}
		-4\,\mdif*{y}
		+5\,y
		=0
	\end{BM}
	Utilizando uma das substituições \(x=e^t\) ou \(y=x^{-1}\,\int{z\,\odif{x}}\) determine a solução geral da equação
	\begin{BM}
		\mdif*[order=2]{y}
		-4\,\mdif*{y}
		+5\,y
		= \frac
		{e^{2\,x}}
		{\sin{x}}
	\end{BM}
	} % Q15

\end{questionBox}

\begin{questionBox}1{ % MARK:Q16
	Ultilizando uma das substituições \(x=e^t\) ou \(y=x^{-1}\,\int{z\,\odif{x}}\) determine a solução da equação geral
	\begin{BM}
		2\,x^2
		\,\odv[order=2]{y}{x}
		+7\,x
		\,\odv{y}{x}
		+3\,x
		=0
	\end{BM}
	com \(x>0\). Determine anda a solução geral da equação
	\begin{BM}
		\odv[order=2]{y}{x}
		+\frac{7}{2\,x}
		\,\odv{y}{x}
		+\frac{3}{2\,x^2}
		\,y
		= 0.5\,x^{-3/2}
	\end{BM}
	} % Q16
	\answer{}
	% \begin{flalign*}
	% 	 &
	% 	ee
	% 	 &
	% \end{flalign*}
	\begin{center}
		\includegraphics[width=.4\textwidth]{16.1.jpeg}
		\includegraphics[width=.4\textwidth]{16.2.jpeg}
		\includegraphics[width=.4\textwidth]{16.3.jpeg}
		\includegraphics[width=.4\textwidth]{16.4.jpeg}
	\end{center}
\end{questionBox}

\end{document}

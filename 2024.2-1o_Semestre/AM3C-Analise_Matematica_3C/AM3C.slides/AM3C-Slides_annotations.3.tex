% !TEX root = ./AM3C-Slides_annotations.3.tex
\documentclass["AM3C-Slides_annotations.tex"]{subfiles}

% \tikzset{external/force remake=true} % - remake all

\begin{document}

% \graphicspath{{\subfix{./.build/figures/AM3C-Slides_annotations.3}}}
% \tikzsetexternalprefix{./.build/figures/AM3C-Slides_annotations.3/graphics/}

\mymakesubfile{3}[AM3C]
{Series} % Subfile Title
{Series} % Part Title

\begin{sectionBox}1{Notas:} % S1
  Inicialmente queremos saber se as series convergem ou divergem,
  aprender os diversos casos espeçificos que nos provam limites espeçificos
  \begin{center}
    \vspace{1ex}
    \setlength\tabcolsep{3mm}        % width
    % \renewcommand\arraystretch{1.25} % height
    \begin{tabular}{l L C C}
      \toprule

        \multirow[c]{2}{*}{Critérios} 
        & \multirow[c]{2}{*}{Teste} 
        & \multicolumn{2}{c}{Casos}
        \\\cline{3-4} & 
        & \multicolumn{1}{l}{Divergente}
        & \multicolumn{1}{l}{Convergente}

      \\\midrule

        Comparação
        & \lim_{n\to\infty}{\frac{b_n}{a_n}}
        & > 1 & < 1
        \\ A'lamberk
        & \lim_{n\to\infty}{\frac{a_{n+1}}{a_n}}
        & > 1 & < 1
        \\ Raiz 
        & \limsup_{n\to\infty}{\sqrt[n]{a_n}}
        & > 1 & < 1
        \\ Integral
        & \int_1^{\infty}{f(x)\,\odif{x}}: f(n)=a_n
        & > 1 & < 1

      \\\bottomrule
    \end{tabular}
    \vspace{2ex}
  \end{center}
  \subsection*{Critério do Integral}
  \begin{BM}
    \int_1^{\infty}{f(x)\,\odif{x}} : f(n) = n
  \end{BM}
  \begin{itemize}
    \item \(f(x)\) é continua em \(\myrange r{1,\infty}\)
    \item Testar \textit{f} para saber se converge ou diverge, \(a_n\) segue o mesmo comportamento
    \item \(\odv{f(x)}{x}\) Aponta con/divergencia de \textit{f}
  \end{itemize}
\end{sectionBox}

\end{document}

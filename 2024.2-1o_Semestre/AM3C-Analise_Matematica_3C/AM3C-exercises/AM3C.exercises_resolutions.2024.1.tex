% !TEX root = ./AM3C.exercises_resolutions.2024.1.tex
\documentclass["./AM3C.exercises_resolutions.2024.tex"]{subfiles}

% \tikzset{external/force remake=true} % - remake all

\begin{document}

\graphicspath{{\subfix{.build/figures/AM3C.exercises_resolutions.2024.1}}}
% \tikzsetexternalprefix{./figures/AM3C.exercises_resolutions.2024.1/graphics/}

\mymakesubfile{1}[AM 3C]
{Equações Diferenciais Ordinárias} % Subfile Title
{Equações Diferenciais Ordinárias} % Part Title

\begin{questionBox}1{} % Q1
  Verifique que cada uma das funções indicadas é solução da equação diferencial considerada.
\end{questionBox}

\begin{questionBox}2{} % Q1.1

  \begin{BM}
    y(x) = e^{2\,x}\,\cos(3\,x)
    ,\quad
    y''-4\,y'+13\,y=0
  \end{BM}

  \answer{}

  \begin{gather*}
    0
    = y'' - 4\,y' + 13\,y
    = \mathText{Using 
      \eqref{eq:1.1-y'}
      \eqref{eq:1.1-y''}
    }
    = e^{2\,x}(
      -5\,\cos(3\,x)
      -12\,\sin(3\,x)
    )
    - 4\,\left(
      e^{2\,x}(
        2\,\cos(3\,x) -3\,\sin(3\,x)
      )
    \right)
    + 13\,y
    = \\
    = e^{2\,x}(
      -5\,\cos(3\,x)
      -12\,\sin(3\,x)
      -8\,\cos(3\,x) 
      +12\,\sin(3\,x)
    )
    + 13\,y
    = \\
    = e^{2\,x}(
      -13\,\cos(3\,x)
    )
    + 13\,y
    \implies \\
    \implies
    y = e^{2\,x}\,\cos(3\,x)
  \end{gather*}

  %y'
  \begin{gather*}
    y'
    = \odv{}{x}
    \left( e^{2\,x}\,\cos(3\,x) \right)
    = 2\,e^{2\,x}\,\cos{(3\,x)}
    - e^{2\,x}\,\sin{(3\,x)}\,3
    = \\
    = e^{2\,x}(
      2\,\cos(3\,x) -3\,\sin(3\,x)
    )
    % 
    \yesnumber\label{eq:1.1-y'}
  \end{gather*}
  % y''
  \begin{gather*}
    y''
    = \odv{y'}{x}
    = \mathText{Using \eqref{eq:1.1-y'}}
    = \odv{}{x}\left(
      e^{2\,x}(
        2\,\cos(3\,x) -3\,\sin(3\,x)
      )
    \right)
    = \\
    = 2\,e^{2\,x}(
      2\,\cos(3\,x) -3\,\sin(3\,x)
    )
    + e^{2\,x}(
      -6\,\sin(3\,x)
      -9\,\cos(3\,x)
    )
    = \\
    = e^{2\,x}(
      -5\,\cos(3\,x)
      -12\,\sin(3\,x)
    )
    % 
    \yesnumber\label{eq:1.1-y''}
  \end{gather*}

\end{questionBox}

\begin{questionBox}2{} % Q1.2

  \begin{BM}
    y(x)
    = e^{-x^2}\int_0^x{e^{t^2}\,\odif{t}}
    + c_1\,e^{-x^2}
    ,\quad \odv{y}{x}+2\,x\,y=1
  \end{BM}


  \answer{}

  \begin{gather*}
    y
    = \frac{1}{2\,x}\left(
      1-\odv{y}{x}
    \right)
    = \mathText{using \eqref{eq:1.2-y'}}
    = \frac{1}{2\,x}\left(
      1
      - \left(
        -2\,x\,e^{-x^2}\left(
          c_1
          + \int_{0}^{x}{e^{t^2}\,\odif{t}}
        \right)
        + x 
      \right)
    \right)
    \\
    = \frac{1}{2\,x}
    + e^{-x^2}\left(
      c_1
      + \int_{0}^{x}{e^{t^2}\,\odif{t}}
    \right)
    - 1/2 
  \end{gather*}
  Portanto \(y(x)\) não é solução da edo

  \begin{gather*}
    \odv{y}{x}
    = \odv{}{x}
    \left(
      e^{-x^2}\int_0^x{e^{t^2}\,\odif{t}}
      + c_1\,e^{-x^2}
    \right)
    = \\
    = -2\,x\,e^{-x^2}\,\int_{0}^{x}{e^{t^2}\,\odif{t}}
    + e^{-x^2}\,\odv{}{x}\left(
      \int_{0}^{x}{e^{t^2}\,\odif{t}}
    \right)
    - c_1\,2\,x\,e^{-x^2}
    = \mathText{using \eqref{eq:1.2-ddx(Pf)}}
    = -2\,x\,e^{-x^2}\left(
      c_1
      + \int_{0}^{x}{e^{t^2}\,\odif{t}}
    \right)
    + e^{-x^2}(
      x\,e^{x^2}
      -0*e^{0^2}
    )
    = \\
    = -2\,x\,e^{-x^2}\left(
      c_1
      + \int_{0}^{x}{e^{t^2}\,\odif{t}}
    \right)
    + x 
    % 
    \yesnumber\label{eq:1.2-y'}
  \end{gather*}
  % ddx(Pf)
  \begin{gather*}
    \odv{}{x}
    \left(
      \int_{a(x)}^{b(x)}{
        f(t)\,\odif{t}
      }
    \right)
    = \odv{}{x}\left(
      P(f(b(x)))
      -P(f(a(x)))
    \right)
    = b'(x)\,f(b(x))
    - a'(x)\,f(a(x))
    %
    \yesnumber\label{eq:1.2-ddx(Pf)}
  \end{gather*}

\end{questionBox}

\begin{questionBox}1{} % Q2

  Mostre que a equação \(2\,x^2\,y - y^2 + 1 = 0\) define implicitamente uma solução da equação diferencial
  \begin{BM}
    (x^2-y)\,\odv{y}{x}+2\,x\,y = 0
  \end{BM}
  Determine explicitamente a solução que verifica a condição \(y(0)=1\).

  \answer{}

  Verificando a funcção implicita
  \begin{gather*}
    \odv{}{x}
    \left(
      2\,x^2\,y - y^2 + 1
    \right)
    = 2*2\,x\,y
    +2\,x^2\,y'
    -2\,y\,y'
    = 4\,x\,y
    +\left(2\,x^2-2\,y\right)\,y'
    = \mathText{using \(a=b \implies a'=b'\)}
    = \odv{0}{x}
    = 0
    \implies
    2\,x\,y
    +\left(x^2-y\right)\,y'
    = 0
  \end{gather*}

  Encontrando solução explicita em que \(y(0)=1\)
  \\\dots

\end{questionBox}

\begin{questionBox}1{} % Q3
  Determine o valores de \(k\) para os quais:
\end{questionBox}

\begin{questionBox}2{} % Q3.1

  \(y(x)=e^{(k\,x)}\) é solução da equação
  \begin{BM}
    y''-y'+6\,y=0
  \end{BM}

  \answer{}
  \begin{gather*}
    y''-y'+6\,y
    = \odv[order=2]{}{x}\left(
      e^{k\,x}
    \right)
    - \odv{}{x}\left(
      e^{k\,x}
    \right)
    + 6\,x
    = k^2\,e^{k\,x}
    - k\,e^{k\,x}
    + 6\,e^{k\,x}
    = 0
  \end{gather*}

  \begin{gather*}
    k^2-k+6 = 0
    \lor e^{k\,x}=0
    % \implies \\
    \implies
    k
    = \frac{1\pm\sqrt{1-4*1*6}}{2*1}
    = \frac{1\pm\sqrt{-23}}{2*1}
  \end{gather*}

  \begin{gather*}
    e^{k\,x}=0
    \implies k\to-\infty
  \end{gather*}

  Não existe \(k\) exceto \((k=-\infty\land x>0)\land(k=\infty\land x<0)\)

\end{questionBox}

\begin{questionBox}2{} % Q3.2

  \(y(x)=x^k\) é solução da equação
  \begin{BM}
    x\,y''+2\,y'=0.
  \end{BM}

  \answer{}

  \begin{gather*}
    0
    = x\,y''+2\,y'
    = x\,(
      x^k
    )''+2\,(
      x^k
    )'
    = x\,(
      k\,\,(k-1)\,x^(k-2)
    )+2\,(
      k\,x^(k-1)
    )
    = x^{k-1}\,k\,(
      k+1
    )
    \implies \\
    \implies
    k = - 1 \lor k = 0
  \end{gather*}
\end{questionBox}

\begin{questionBox}1{} % Q4

  Em cada uma das seguintes equaçoes autónomas determine os pontos de equilíbrio e represente o respetivo retrato de fase. Classifique os pontos críticos e represente graficamente os diferentes tipos de soluções em cada uma das regiões determinadas pelas soluções de equilíbrio.

\end{questionBox}

\begin{questionBox}2{} % Q4.1

  \begin{BM}
    \odv{y}{x}
    = y^2-3\,y
  \end{BM}

  \answer{}

  \begin{gather*}
    \odv{y}{x}
    = y^2-3\,y
    = y(y-3)
    = 0
  \end{gather*}
  \begin{center}
    \vspace{1ex}
    \begin{tabular}{L *5{C}}
      \toprule

      & \dots & 0 & \dots & 3 & \dots
      \\\midrule

      y
      & -     & 0 & +     & + & +
      \\ y-3
      & -     & - & -     & 0 & +
      \\\midrule
      y(y-3)
      & +     & 0 & -     & 0 & +

      \\\bottomrule
    \end{tabular}
    \vspace{2ex}
  \end{center}

\end{questionBox}

\part*{Equações Diferenciais lienares de primeira ordem e Equações Redutíveis a Lineares de Primeira ordem}

\setcounter{question}{5}
\begin{questionBox}2{} % Q5.1

  Determine a solução geral da equação diferencial linear homogénea de primeira ordem
  \begin{BM}
    \odv{y}{x}+2\,x\,y=0
  \end{BM}

  \answer{}

  \begin{gather*}
    \odv{y}{x}+2\,x\,y=0
    \implies \\
    \implies
    y
    = \gamma{(x)}^{-1}
    \,C
    + \gamma{(x)}^{-1}
    \,\int{\exp{a(x)}*b(x)\,\odif{x}}
    = \left(
      \int{\exp{(2\,x)}}
    \right)^{-1}
    \,C
    + \gamma{(x)}^{-1}
    \,\int{\exp{a(x)}*0\,\odif{x}}
    = \exp{(-x^2)}
    \,C
  \end{gather*}
\end{questionBox}

\begin{questionBox}1{} % Q6
  Determine a solução geral das seguintes equações diferenciais lineares de primeira ordem:
\end{questionBox}

\begin{questionBox}2{} % Q6.1

  \begin{BM}
    \odv{y}{x}-y\,\tan{x}=\cos{x}
    , \quad
    x\in\myrange*{-\pi/2,\pi/2}
  \end{BM}

\end{questionBox}

\begin{questionBox}2{} % Q6.2
  \begin{BM}
    y^2\,\odif{x}
    - (2\,x\,y+3)\,\odif{y}=0
  \end{BM}
  considere x como fun¸c˜ao inc´ognita e y vari´avel independente

  \answer{}

  \begin{gather*}
    y^2\,\odif{x}
    - (2\,x\,y+3)\,\odif{y}=0
    \implies \\
    x \text{ como função de } y
    \\
    \implies
    y^2\,\odv{x}{y}
    - (2\,x\,y+3)=0
    \implies
    \odv{x}{y}
    =
    2\,x\,y^{-1}
    + 3\,y^{-2}
    \implies \\
    \implies
  \end{gather*}

\end{questionBox}

\begin{questionBox}1{} % Q7

  Determine a solução geral da equação
  \begin{BM}
    \odv{z}{x}
    + (x-x^{-1})\,z
    =0
    ,\quad
    x>0
  \end{BM}
  e utilizando o m´etodo da varia¸c˜ao das constantes arbitr´arias determine a solu¸c˜ao
  geral de
  \begin{BM}
    \odv{z}{x}
    + (x-x^{-1})\,z
    = - x^2
  \end{BM}

  \answer{}

  \begin{gather*}
    \odv{z}{x}
    + (x-x^{-1})\,z
    =0
    ,\quad
    x>0
    \implies \\
    \implies
    z
    =c\,\gamma^{-1}(x)
    =c\,\exp{-\int{(x-x^{-1})\,\odif{x}}}
    = \dots
    =c\,\exp{-x^2/2}
  \end{gather*}

  \begin{gather*}
    \text{Variação das constantes arbitrárias}
    \\
    \odv{z}{x}
    + (x-x^{-1})\,z
    = - x^2
    \implies
    z(x)
    = c(x)\,x\,\exp{-x^2/2}
    ; \\
    z'(x)
    = c'(x)\,x\,exp{-x^2/2}
    +c(x)\left(
      \exp{(-x^2/2)}
      -x^2\exp{(-x^2/2)}
    \right)
  \end{gather*}

\end{questionBox}

\setcounter{question}{8}
\setcounter{subquestion}{1}
\begin{questionBox}2{} % Q8.2

  Utilizando a substituição definida por \(y=e^{4\,x}\), determine a solução geral da equação:
  \begin{BM}
    x\,\odv[order=2]{y}{x}
    -(4\,x+1)\,\odv{y}{x}
    +4\,y
    = 0
  \end{BM}

  \answer{}

  \begin{gather*}
    ( x\,D-(4\,x+1)\,D^2+4 )\,y
    =Py=0
    \implies \\
    \implies
    y=e^{4\,x}\,\int{z}
    \implies \\
    \implies
    % 
    % 
    % 
    ;        \\[3ex]
    \odv{y}{x}
    =\odv{}{x}\left(
      e^{4\,x}
    \right)
    \int{z}
    + \odv{}{x}\left(
      \int{z}
    \right)
    =4\,x\,e^{4\,x}
    \,\int{z}
    + z\,e^{4\,x}
    % 
    % 
    % 
    ;        \\[3ex]
    \odv[order=2]{y}{x}
    = \odv{}{x}\left(
      4\,x\,e^{4\,x}
      \,\int{z}
      + z\,e^{4\,x}
    \right)
    =\dots
    % 
    % 
    % 
    \\[3ex]
    \implies
    x\,e^{4\,x}
    \,\left(
      16\,\int{z}
      +8\,z
      \odv{z}{x}
    \right)
    -(4\,x+1)
    \,e^{4\,x}
    \left(
      4\,\int{z} +z
    \right)
    +4\,e^{4\,x}\,\int{z}
    =0
    \implies \\
    \implies
    \dots
    \implies \\
    \implies
    \odv{z}{x}+(4-1/x)\,z=0
    \implies \\
    \implies
    \dots
    z
    =\frac{C}{
      \exp{(\int{(4-1/x)})}
    }
    = \dots
    = C\,x\,e^{-4\,x}
    \implies \\
    \implies
    y
    =e^{4\,x}\,\int{C\,x\,e^{-4\,x}}
    = C\,e^{4\,x}\,\left(
      -\frac{x\,e^{-4\,x}}{4}
      -\frac{e^{-4\,x}}{16}
      +k
    \right)
    = \dots
    =        \\
    = -\frac{c}{4}\left(
      x+1/4
    \right)
    +c\,k\,e^{4\,x}
    = c_1\left(
      x+1/4
    \right)
    +c_2\,e^{4\,x}
    \quad\forall\,c_1,c_2\in\mathbb{R}
  \end{gather*}
\end{questionBox}

\setcounter{question}{9}
\setcounter{subquestion}{2}
\begin{questionBox}2{} % Q9.3

  \answer{}

  \begin{gather*}
    \odv{z}{x}
    +\frac{z}{2\,\sqrt{x}}
    =-\frac{z^3}{2}
    \implies
    z^{-3}\,\odv{z}{x}
    +\frac{z^{-1}}{2\,\sqrt{x}}
    =-\frac{1}{2}
  \end{gather*}
\end{questionBox}


\begin{questionBox}1{} % Q10
  Considere as seguintes eq dif de prim ordem não lin
\end{questionBox}

\setcounter{subquestion}{1}
\begin{questionBox}2{} % Q10.2

  \begin{BM}
    y'+x\,y^2-(2\,x^2+1)\,y+x^3+x-1=0
    \\ y_1(x) = x
  \end{BM}

  \answer{}

  \begin{gather*}
    y'+x\,y^2-(2\,x^2+1)\,y+x^3+x-1=0
    \implies \\
    \implies
    y'
    +y\,\left(
      -(2\,x^2+1)
    \right)
    =\left(
      -x^3-x+1
    \right)
    +y^2\,\left(
      -x
    \right)
  \end{gather*}

\end{questionBox}

\setcounter{question}{13}
\begin{questionBox}1{} % Q14

  Determine o integral geral de cada uma das seguintes equações diferenciaia lienares de coeficientes constantes:

\end{questionBox}

\begin{questionBox}2{} % Q14.1

  \begin{BM}
    D^3\,(D+1)^2((D-5)^2+16)\,y=0
    ,\quad\left(
      \text{em que} D=\odv{}{x}
    \right)
  \end{BM}
  \answer{}

  \begin{gather*}
    D^3\,(D+1)^2((D-5)^2+16)\,y=0
    \implies \\
    \implies
    \text{Raizes:}
    \begin{pmatrix}
      0 & 3
      \\ -1 & 2
      \\ 5\pm4\,i & 1
    \end{pmatrix}
    \implies \\
    \implies
    y
    = c_0
    + c_1\,x
    + c_2\,x^2
    + c_3\,e^{-x}
    + c_4\,x\,e^{-x}
    + c_5\,e^{-5\,x}\,\cos{4\,x}
    + c_6\,e^{-5\,x}\,\sin{4\,x}
  \end{gather*}
\end{questionBox}

\begin{questionBox}2{} % Q14.2

  \begin{BM}
    (D^4-1)\,y=x^3-x+2
    ,\quad\left(
      \text{em que }
      D=\odv{}{x}
    \right)
  \end{BM}

  \answer{}

  \begin{gather*}
    y=y_0+y_1
    =        \\
    = \left(
      c_1\,e^{x}
      +c_2\,e^{-x}
      +c_3\,\cos{x}
      +c_4\,\sin{x}
    \right)
    + \left(
      x^3-x+2
    \right)
    %
    %
    %
    ;        \\[3ex]
    y_0\text{ é solução de }
    (D^4-1)\,y=0
    \implies \\
    \implies
    (\alpha^4-1)
    = (\alpha^2+1)(\alpha^2-1)
    \implies \\
    \implies
    \text{Raizes:}
    \left\{
      \begin{tabular}{cc}
        \text{Raizes} & \text{Multiplicidade}
        \\1 & 1
        \\ -1 & 1
        \\ \pm i & 1
      \end{tabular}
    \right.
    \\
    \implies
    y_0
    =c_1\,e^{x}
    +c_2\,e^{-x}
    +c_3\,\cos{x}
    +c_4\,\sin{x}
    %
    %
    %
    ;        \\[4ex]
    y_1\text{ é \(x^p\) vezes um polinomio de mesmo grau que }
    x^3-x+2
    \\
    p\text{ é a multiplicidade da raiz}\alpha=0
    \implies
    \implies \\
    y_1
    =x^p\left(
      a_0
      + a_1\,x
      + a_2\,x^2
      + a_3\,x^3
    \right)
    \implies \\
    \implies
    (D^4+1)\,y_1
    = 0+y_1
    = y_1
    = x^3-x+2
  \end{gather*}

\end{questionBox}

\begin{questionBox}2{} % Q14.3

  \begin{BM}
    y''+y=\cos{x}
  \end{BM}

  \answer{}

  \begin{gather*}
    y''+y
    = (D^2+1)\,y
    =\cos{x}
    \implies \\
    \implies
    y=y_0+y_1
    = \left(
      c_1\,\cos{x}
      +c_2\,\sin{x}
    \right)
    + \left(
      \frac{x}{2}\,\sin{x}
    \right)
    %  
    %  
    %  
    ;        \\[3ex]
    \alpha^2+1=0
    \implies \\
    \implies
    \begin{tabular}{lc}
      \text{Raizes}
      & \pm 1
      \\ \text{Multiplicidade}
      & 1
    \end{tabular}
    \\
    \implies
    y_0
    =c_1\,\cos{x}
    +c_2\,\sin{x}
    %
    %
    %
    \\[3ex]
    y_1
    = x^p\,\left(
      r\,e^{a\,x}\,\cos{(b\,x)}
      + s\,e^{a\,x}\,\sin{(b\,x)}
    \right)
    = x\,\left(
      r\,\cos{x}
      + s\,\sin{x}
    \right)
    \implies \\
    \implies
    \odv{y_1}{x}
    = r\,\cos{x}
    + s\,\sin{x}
    = x\,\left(
      - r\,\sin{x}
      + s\,\cos{x}
    \right)
    \implies \\
    \implies
    \odv[order=2]{y_1}{x}
    = \dots
    \implies \\
    \implies
    (D^2+1)y_1
    = -2\,r\,\sin{x}
    + 2\,s\,\cos{x}
    = \cos{x}
    \implies
    s=1/2\land r=0
    \implies \\
    \implies
    y_1=\frac{x}{2}\,\sin{x}
  \end{gather*}
\end{questionBox}

\setcounter{question}{14}
\begin{questionBox}1{} % Q15

  Determine a sol geral da equação diferencial linear homogénea de coeficientes consntantes
  \begin{BM}
    \mdif*[order=2]{y}
    -4\,\mdif*{y}
    +5\,y
    =0
  \end{BM}
  Utilizando uma das substituições \(x=e^t\) ou \(y=x^{-1}\,\int{z\,\odif{x}}\) determine a solução geral da equação
  \begin{BM}
    \mdif*[order=2]{y}
    -4\,\mdif*{y}
    +5\,y
    = \frac
    {e^{2\,x}}
    {\sin{x}}
  \end{BM}

\end{questionBox}

\begin{questionBox}1{} % Q16

  Ultilizando uma das substituições \(x=e^t\) ou \(y=x^{-1}\,\int{z\,\odif{x}}\) determine a solução da equação geral
  \begin{BM}
    2\,x^2
    \,\odv[order=2]{y}{x}
    +7\,x
    \,\odv{y}{x}
    +3\,x
    =0
  \end{BM}
  com \(x>0\). Determine anda a solução geral da equação
  \begin{BM}
    \odv[order=2]{y}{x}
    +\frac{7}{2\,x}
    \,\odv{y}{x}
    +\frac{3}{2\,x^2}
    \,y
    = 0.5\,x^{-3/2}
  \end{BM}

  \answer{}

  \begin{gather*}
    y = c_1\,x^{-3/2} + c_2\,x^{-1} + \frac{\sqrt{x}}{6}
    \\[1ex]
    \mdif[2]{x}{y} 
    + \frac{7}{2\,x}
    \,\mdif[1]{x}{y}
    + \frac{3}{2\,x^2}
    \,y
    = x^{-3/2}/2
  \end{gather*}
  \begin{center}
    \includegraphics[width=.4\textwidth]{16.1.jpeg}
    \includegraphics[width=.4\textwidth]{16.2.jpeg}
    \includegraphics[width=.4\textwidth]{16.3.jpeg}
    \includegraphics[width=.4\textwidth]{16.4.jpeg}
  \end{center}

\end{questionBox}

\setcounter{question}{28}
\setcounter{subquestion}{1}

\begin{questionBox}2{} % Q28.2

  A equação
  \begin{BM}
    (x\,y^2 + x^2\,y^2 + 3)\,\odif{x} + x^2\,y\,\odif{y} = 0
  \end{BM}
  Admite fatores que são apenas funções de x. det um desses fatores integrantes e o integral geral dessa equação, Determine a solução particular que passa pelo ponto (1,0)

  \answer{}

  \begin{gather*}
    % fatores integ
    \varphi(x) = c\,e^{2\,x}
    \\
    % um fator integ
    \varphi_i(x) = e^{2\,x}.
    \\
    % ingegral geral
    e^{2\,x}(x^2\,y^2+3) = e
    \\
    % Solução particular
    e^{2\,x}(x^2\,y^2+3) = 3\,e^2
  \end{gather*}
  \begin{center}
    \includegraphics[width=.4\textwidth]{28.2.1.jpeg}
    \includegraphics[width=.4\textwidth]{28.2.2.jpeg}
    \includegraphics[width=.4\textwidth]{28.2.3.jpeg}
    \includegraphics[width=.4\textwidth]{28.2.4.jpeg}
  \end{center}

  \begin{gather*}
    \dots \\
    \rho{x}=e^{2\,x} \text{ é um fator integrante}
  \end{gather*}
\end{questionBox}

\setcounter{question}{31}
\begin{questionBox}2{} % Q31.1

  \begin{BM}
    \begin{cases}
      \odv[order=2]{x}{t}
      +x
      +\odv{y}{t}
      +2\,y
      =t
      \\
      -\odv[order=3]{x}{t}
      +3\,\odv[order=2]{x}{t}
      -\odv{x}{t}
      +3\,x
      -\odv[order=2]{y}{t}
      =-3\,t-1
    \end{cases}
  \end{BM}

  \answer{}

  \begin{center}
    \includegraphics[width=.4\textwidth]{31.1.1.jpeg}
    \includegraphics[width=.4\textwidth]{31.1.2.jpeg}
    \includegraphics[width=.4\textwidth]{31.1.3.jpeg}
    \includegraphics[width=.4\textwidth]{31.1.4.jpeg}
    \includegraphics[width=.4\textwidth]{31.1.5.jpeg}
  \end{center}

  \begin{gather*}
    \begin{bmatrix}
      D^2+1              & D+2 &|& T
      \\ -D^3+3\,D^2-D+3 & -D^2 &|& 3t-1
    \end{bmatrix}
    % \implies \\
    \implies
    \begin{bmatrix}
      D^2+1 & D+2 &|& T
      \\ 0  & -D -6 &|& 0
    \end{bmatrix}
  \end{gather*}
\end{questionBox}

\end{document}

% !TEX root = ./AM3C.exercises_resolutions.2024.3.tex
\documentclass["AM3C.exercises_resolutions.2024.tex"]{subfiles}

% \tikzset{external/force remake=true} % - remake all

\begin{document}

\graphicspath{{\subfix{./.build/figures/AM3C.exercises_resolutions.2024.3}}}
% \tikzsetexternalprefix{./.build/figures/AM3C.exercises_resolutions.2024.3/graphics/}

\mymakesubfile{3}[AM3C]
{Sucessões} % Subfile Title
{Sucessões} % Part Title

\setcounter{question}{3}
\setcounter{subquestion}{3}
\begin{questionBox}2{} % Q3.4
  \answer{}
  \begin{center}
    \includegraphics[width=.8\textwidth]{Q3.4.jpeg}
  \end{center}
\end{questionBox}

\setcounter{question}{4}
\setcounter{subquestion}{1}

\begin{questionBox}2{} % Q4.2
  \answer{}
  \begin{center}
    \includegraphics[width=.8\textwidth]{Q4.2.jpeg}
  \end{center}
\end{questionBox}


\setcounter{question}{5}
\setcounter{subquestion}{2}

\begin{questionBox}2{} % Q5.3
  \answer{}
  \begin{center}
    \includegraphics[width=.8\textwidth]{Q5.3.jpeg}
  \end{center}
\end{questionBox}

\setcounter{question}{6}
\setcounter{subquestion}{1}

\begin{questionBox}2{} % Q6.2
  Estudar a convergencia de:
  \begin{BM}
    \int_{n=1}^{\infty}{ \frac
      {\sqrt{n+1} - \sqrt{n}}
      {\sqrt{}}
    }
  \end{BM}
  \answer{}
  \begin{center}
    \includegraphics[width=.8\textwidth]{Q6.2.jpeg}
  \end{center}
  \begin{flalign*}
    &
      \sum_{n=1}^{\infty}{ \frac
        {\sqrt{n+1} - \sqrt{n}}
        {\sqrt{n^2+n}}
      }
      % 
      % 
      % 
      &\\[3ex]&
      \frac
      {\sqrt{n+1} - \sqrt{n}}
      {\sqrt{n^2+1}}
      = \frac
      {\sqrt{n+1} - \sqrt{n}}
      {\sqrt{n}\,\sqrt{n+1}}
      = \frac{1}{\sqrt{n}}
      - \frac{1}{\sqrt{n+1}}
      = a_n - a_{n+k} : a_n = \frac{1}{\sqrt{n}} \land k=1
      % 
      % 
      % 
      &\\[3ex]&
      \text{Serie telescópica:} &\\&
      \sum_{n=1}^{\infty}{(a_n - a_{n+k})}
      \text{ Converge se } a_n\to0
    &
  \end{flalign*}
\end{questionBox}

\setcounter{question}{7}
\setcounter{subquestion}{1}

\begin{questionBox}2{} % Q7.2
  Estudar o qudro de divergencia de
  \begin{BM}
    \sum_{n=1}^{\infty}{
      (-1)^{n}
      \,\frac{n^2}{(n+1)(n+2)}
    }
  \end{BM}
  \answer{}
  \begin{center}
    \includegraphics[width=.8\textwidth]{Q7.2.jpeg}
  \end{center}
\end{questionBox}

\setcounter{question}{9}
\setcounter{subquestion}{1}

\begin{questionBox}2{} % Q9.2
  Use o critério do integral para estudar a convergencia de
  \begin{BM}
    \sum_{n=1}^{\infty}{
      \frac{1}{n\,\sqrt{\log{n}-1}}
    }
  \end{BM}
  \answer{}
  \begin{flalign*}
    &
      \sum_{n=1}^{\infty}{
        \frac{1}{n\,\sqrt{\log{n}-1}}
      }
      % 
      % 
      % 
      &\\[3ex]&
      \text{Converge se} &\\&
      \int_3^{\infty}{
        \frac{\odif{x}}{x\,\sqrt{\log{x}-1}}
      }
      = \int_3^{\infty}{
        2\,\left(
          \sqrt{\log(x)-1}
        \right)'
        \,\odif{x}
      }
      = 2\,\left(
        \sqrt{\log(\infty)-1}
        - \sqrt{\log(3)-1}
      \right)
      = + \infty
      &\\&
      \therefore \text{Não converge}
      % 
      % 
      % 
      &\\[3ex]&
      \odv{}{x}
      \sqrt{\log(x)-1}
      = \dots
      = \frac{1}{2}
      \,\frac{1}{\sqrt{\log{x}-1}}
    &
  \end{flalign*}
  \begin{center}
    \includegraphics[width=.8\textwidth]{Q9.2.jpeg}
  \end{center}
\end{questionBox}

\setcounter{question}{10}
\setcounter{subquestion}{2}

\begin{questionBox}2{} % Q10.3
  \begin{BM}
    \sum_{n=3}^{\infty}{
      \frac{1}{\log(n)}
    }
  \end{BM}
  \answer{}
  Comparar com \(1/n\)
  usando o criterio da comparação
\end{questionBox}


\end{document}

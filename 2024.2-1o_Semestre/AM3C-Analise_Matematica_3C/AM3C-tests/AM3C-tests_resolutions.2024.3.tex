% !TEX root = ./AM3C-tests_resolutions.2024.3.tex
\documentclass["AM3C-tests_resolutions.tex"]{subfiles}

% \tikzset{external/force remake=true} % - remake all

\begin{document}

% \graphicspath{{\subfix{./.build/figures/AM3C-tests_resolutions.2024.3}}}
% \tikzsetexternalprefix{./.build/figures/AM3C-tests_resolutions.2024.3/graphics/}

\mymakesubfile{3}
{Exam 2024.3 Resolution} % Subfile Title
{Exam 2024.3 Resolution} % Part Title

\group{}

\begin{questionBox}1m{} % 1Q1
  \begin{BM}
    \odv{y}{x} +2\,\sin(2\,x)\,y = \sin(2\,x)
    ; y(\pi/2) = 3/2
  \end{BM} 

  % tag  = 1q1
  % a(x) = 2\,\sin(2\,x)
  % b(x) = \sin(2\,x)
  % y    = y
  % x    = x

  % y' + y( 
  %   2\,\sin(2\,x) 
  % )
  % = \sin(2\,x)

  \answer{\eqref{eq:1q1 answer}}

  General solution
  \begin{tcolorbox}
    \begin{gather*}
      y
      = \frac{c_0}{\varphi(x)}
      + \frac{1  }{\varphi(x)}
      \,\prim_x{\left(
        \sin(2\,x)\,\varphi(x)
      \right)}
      %
      %
      %
      = \mathText{using 
        \eqref{eq:1q1 phi_x}
        \eqref{eq:1q1 prim}
      }
      = \frac{c_0}{c_2\,e^{( -\cos(2\,x) )}}
      + \frac{1  }{c_2\,e^{( -\cos(2\,x) )}}
      \,c_2\,\left(c_3+e^{-\cos(2\,x)}\right)
      = \\
      = 
      e^{\cos(2\,x)}\, \left(
        \frac{c_0}{c_2}
        +c_3
      \right)
      + 1
      = c_4\,e^{\cos(2\,x)} + 1
      %
      \yesnumber\label{eq:1q1 y const}
      %
      = \mathText{using \eqref{eq:1q1 const of y}}
      = (e/2)\,e^{\cos(2\,x)} + 1
      = e^{1+\cos(2\,x)}/2 + 1
      ; \shortintertext{Closest option:}
      e^{\cos(2\,x)+1}+1
      %
      \yesnumber\label{eq:1q1 answer}
    \end{gather*}
  \end{tcolorbox}

  Finding constants in \eqref{eq:1q1 y const}
  \begin{tcolorbox}
    \begin{gather*}
      y(\pi/2)
      = 3/2
      = \mathText{using \eqref{eq:1q1 y const}}
      = c_4\,e^{-1} + 1
      \implies
      c_4 = e^1\left(
        \frac{3}{2}-\frac{2}{2}
      \right)
      = e/2
      %
      \yesnumber\label{eq:1q1 const of y}
    \end{gather*}
  \end{tcolorbox}

  Finding \(\varphi(x)\)
  \begin{tcolorbox}
    \begin{gather*}
      \varphi(x) 
      = \exp{\left(
        \prim_x{\left(
          2\,\sin(2\,x)
        \right)}
      \right)}
      = \exp{\left(
          c_1-\cos(2\,x)
      \right)}
      = c_2\,e^{( -\cos(2\,x) )}
      %
      \yesnumber\label{eq:1q1 phi_x}
    \end{gather*}
  \end{tcolorbox}

  Integrating
  \begin{tcolorbox}
    \begin{gather*}
      \prim_x{\left(
        \sin(2\,x)\,\varphi(x)
      \right)}
      = \mathText{using \eqref{eq:1q1 phi_x}}
      = \prim_x{\left(
        \sin(2\,x)
        c_2\,\exp{( -\cos(2\,x) )}
      \right)}
      = \mathText{\(
        \mdif{x}{\left(e^{-\cos(2\,x)}\right)}
        = e^{-\cos(2\,x)}
        \,(2\,\sin(2\,x))
      \)}
      = c_2\,(c_3+e^{-\cos(2\,x)})
      %
      \yesnumber\label{eq:1q1 prim}
    \end{gather*}
  \end{tcolorbox}
\end{questionBox}

\begin{questionBox}1m{} % Qindex
  \begin{BM}
    (\mdif[3]{x} + \mdif[2]{x})\,y = -4
  \end{BM}

  % tag   = 1q2
  % P     = \mdif[3]{x} + \mdif[2]{x}

  \answer{\eqref{eq:1q2 answer}}

  General solution for \(y\)
  \begin{tcolorbox}
    \begin{gather*}
      y
      = y_h 
      + \bar{y}
      = \mathText{using 
        \eqref{eq:1q2 bar y}
        \eqref{eq:1q2 mapped roots for y_h}
      }
      = e^{+0\,x}\,(c_0+c_1\,x)
      + e^{-1\,x}\,(c_2)
      - 2\,x^2
      %
      \yesnumber\label{eq:1q2 answer}
    \end{gather*}
  \end{tcolorbox}

  % solve bar y here
  % tag   = 1q2
  % i     = 
  % P     = \mdif[3]{x} + \mdif[2]{x}
  % alpha = 0
  % Pk    = -4
  % k     = 0

  % y
  % (\mdif[3]{x} + \mdif[2]{x})
  % = e^{0 x}
  % ( -4 )

  Finding \(\bar{y}\)
  \begin{tcolorbox}
    \begin{gather*}
      \bar{y}
      = x^p
      \,Q_0(x)
      = x^p
      \,\sum_{i=0}^{0}{
        \rho_i\,x^i
      }
      = x^2\,\rho_0
      %
      \yesnumber\label{eq:1q2 bar y (const)}
      = \mathText{using \eqref{eq:1q2 const of bar y}}
      = -2\,x^2
      %
      \yesnumber\label{eq:1q2 bar y}
    \end{gather*}
  \end{tcolorbox}

  Finding constants of \eqref{eq:1q2 bar y (const)}
  \begin{tcolorbox}
    \begin{gather*}
      \bar{y}\,P
      = x^2\,\rho_0
      \,\left(
        \mdif[3]{x} + \mdif[2]{x}
      \right)
      = 2\,\rho_0
      = \\
      = -4
      \implies \rho_0 = -2
      %
      \yesnumber\label{eq:1q2 const of bar y}
    \end{gather*}
  \end{tcolorbox}

  Mapping roots of \eqref{eq:1q2 roots of y_h} to solution
  \begin{tcolorbox}
    \begin{gather*}
      \begin{cases}
        r_0 = r_1 = 0
        \implies
        e^{+0\,x}\,(c_0+c_1\,x)
        \\
        r_2 = -1
        \implies
        e^{-1\,x}\,(c_2)
      \end{cases}
      %
      \yesnumber\label{eq:1q2 mapped roots for y_h}
    \end{gather*}
  \end{tcolorbox}

  Roots for characteristic equation for \(y_h\)
  \begin{tcolorbox}
    \begin{gather*}
      P
      = \mdif[3]{x} + \mdif[2]{x}
      \implies \mathText{\(\mdif[i]{x} \to r^i\)}
      \implies
      r^3+r^2
      = r^2(r+1)
      = 0
      \implies
      \begin{cases}
        r_0 = r_1 = 0
        \\
        r_2 = -1
      \end{cases}
      %
      \yesnumber\label{eq:1q2 roots of y_h}
    \end{gather*}
  \end{tcolorbox}
\end{questionBox}

\begin{questionBox}1m{} % 1Q3
  \begin{BM}
    y' + \frac{1}{x}\,y = -2\,x^5\,y^4
    ,\quad x>0
  \end{BM}

  % tag  = 1q3.2
  % a(x) = \frac{1}{x}
  % b(x) = (-2)\,x^5
  % k    = 4

  \answer{\eqref{eq:1q3.2 answer}}

  General solution
  \begin{tcolorbox}
    \begin{gather*}
      y 
      = z^{-1/3}
      %
      \yesnumber\label{eq:1q3.2 y->z}
      %
      = \\
      = \left(
        \frac{c_0}{\varphi(x)}
        + \frac{1}{\varphi(x)}
        \,\prim_x{(
          -3
          \,(-2)\,x^5
          \,\varphi(x)
        )}
      \right)^{-1/3}
      = \mathText{using
        \eqref{eq:1q3.2 phi_x}
        \eqref{eq:1q3.2 prim}
      }
      = \left(
        \frac{c_0}{c_1\,x^{-3}}
        + \frac{1}{c_1\,x^{-3}}
        \,6\,c_1(c_2+x^3/3)
      \right)^{-1/3}
      = \\
      = \left(
        x^3\left(
          \frac{c_0}{c_1}
          +6\,c_2
        \right)
        + 2
      \right)^{-1/3}
      = \\
      = \left(
        x^3\,c_3 + 2
      \right)^{-1/3}
      ; \shortintertext{Closest option}
        \frac{1}{\sqrt[3]{2\,x^6+c\,x^3}}
      %
      \yesnumber\label{eq:1q3.2 answer}
    \end{gather*}
  \end{tcolorbox}

  Bernoulli's substitution
  \begin{tcolorbox}
    \begin{gather*}
      y' + \frac{1}{x}\,y = (-2)\,x^5\,y^{4}
      \implies \mathText{using \eqref{eq:1q3.2 y->z}}
      \implies
      z' + -3\frac{1}{x}\,z = -3\,(-2)\,x^5
    \end{gather*}
  \end{tcolorbox}

  Finding \(\varphi(x)\)
  \begin{tcolorbox}
    \begin{gather*}
      \varphi(x)
      = \exp{\left(
        \prim_x{(
          -3\,\frac{1}{x}
        )}
      \right)}
      = \exp{\left(
        -3\,(c_0+\ln{x})
      \right)}
      = c_1\,x^{-3}
      %
      \yesnumber\label{eq:1q3.2 phi_x}
    \end{gather*}
  \end{tcolorbox}

  Integrating
  \begin{tcolorbox}
    \begin{gather*}
      \prim_x{\left(
        -3
        \,(-2)\,x^5
        \,\varphi(x)
      \right)}
      = \mathText{using \eqref{eq:1q3.2 phi_x}}
      = \prim_x{\left(
        -3\,(-2)\,x^5
        c_1\,x^{-3}
      \right)}
      = 6\,c_1\prim_x{\left(
        x^2
      \right)}
      = 
      6\,c_1(c_2+x^3/3)
      %
      \yesnumber\label{eq:1q3.2 prim}
    \end{gather*}
  \end{tcolorbox}

  % tag  = 1q3
  % a(x) = \frac{1}{x}
  % b(x) = -2\,x^5
  % k    = 4

  \answer{\eqref{eq:1q3 answer}}

  General solution
  \begin{tcolorbox}
    \begin{gather*}
      y 
      = z^{-1/3}
      %
      \yesnumber\label{eq:1q3 y->z}
      %
      = \\
      = \left(
        \frac{c_0}{\varphi(x)}
        + \frac{1}{\varphi(x)}
        \,\prim_x{(
            (-3)\,(-2)\,x^5
          \,\varphi(x)
        )}
      \right)^{-1/3}
      = \mathText{using
        \eqref{eq:1q3 phi_x}
        \eqref{eq:1q3 prim}
      }
      = \left(
        \frac{c_0}{ c_1\,x^{-3} }
        + \frac{1}{ c_1\,x^{-3} }
        6\,c_1( c_2 + x^3/3)
      \right)^{-1/3}
      = \left(
        x^3\left(
          \frac{c_0}{c_1}
          +6\,c_2
        \right) + 2
      \right)^{-1/3}
      = \left(
        x^3\,c_3 + 2
      \right)^{-1/3}
      %
      \yesnumber\label{eq:1q3 answer}
    \end{gather*}
  \end{tcolorbox}

  Bernoulli's substitution
  \begin{tcolorbox}
    \begin{gather*}
      y' + \frac{1}{x}\,y = -2\,x^5\,y^{4}
      \implies \mathText{using \eqref{eq:1q3 y->z}}
      \implies
      z' + -3\frac{1}{x}\,z = (-3)\,(-2)\,x^5
    \end{gather*}
  \end{tcolorbox}

  Finding \(\varphi(x)\)
  \begin{tcolorbox}
    \begin{gather*}
      \varphi(x)
      = \exp{\left(
        \prim_x{(
          -3
          \,\frac{1}{x}
        )}
      \right)}
      = \exp{\left(
        -3\,\prim_x{(
          \,\frac{1}{x}
        )}
      \right)}
      = \exp{\left(
          -3\,(c_0+\ln{x})
      \right)}
      = \\
      = 
      c_1\,x^{-3}
      %
      \yesnumber\label{eq:1q3 phi_x}
    \end{gather*}
  \end{tcolorbox}

  Integrating
  \begin{tcolorbox}
    \begin{gather*}
      \prim_x{\left(
          (-3)*(-2)\,x^5
        \,\varphi(x)
      \right)}
      = \mathText{using \eqref{eq:1q3 phi_x}}
      = \prim_x{\left(
        6\,x^5\,c_1\,x^{-3}
      \right)}
      = 
      6\,c_1( c_2 + x^3/3)
      %
      \yesnumber\label{eq:1q3 prim}
    \end{gather*}
  \end{tcolorbox}
\end{questionBox}

\begin{questionBox}1m{} % 1Q4
  As series converge?
  \begin{BM}[align*]
    \sum_{n=1}^{\infty}&{
      \frac{
        \sqrt[3]{n^2}+\sqrt[3]{n^5}
      }{
        \sqrt{n^3}+\sqrt{n^5}
      }
    }
    \yesnumber\label{eq:1q4 serie 1}
    ; \\
    \sum_{n=1}^{\infty}&{
      \frac{1}{\sqrt{n}}
      -\frac{1}{\sqrt{n+1}}
    }
    \yesnumber\label{eq:1q4 serie 2}
    ; \\
    \sum_{n=1}^{\infty}&{
      \frac{
        1*3*5*\dots*(2\,n-1)
      }{
        n!
      }
    }
    \yesnumber\label{eq:1q4 serie 3}
  \end{BM}

  \answer{Todas convergem}

  Finding convergence for \eqref{eq:1q4 serie 1}
  \begin{tcolorbox}
    \begin{gather*}
      \frac{
        \sqrt[3]{n^2}+\sqrt[3]{n^5}
      }{
        \sqrt{n^3}+\sqrt{n^5}
      }
      = 
      \left(
        \frac{
          \sqrt[3]{n^2}
        }{
          \sqrt{n^3}+\sqrt{n^5}
        }
      \right)
      + 
      \left(
        \frac{
          \sqrt[3]{n^5}
        }{
          \sqrt{n^3}+\sqrt{n^5}
        }
      \right)
      = \\
      = 
      \left(
        \frac{
          \sqrt{n^3}+\sqrt{n^5}
        }{
          \sqrt[3]{n^2}
        }
      \right)^{-1}
      + 
      \left(
        \frac{
          \sqrt{n^3}+\sqrt{n^5}
        }{
          \sqrt[3]{n^5}
        }
      \right)^{-1}
      = \\ 
      = 
      \left(
        n^{
          \frac{3}{2}
          -\frac{2}{3}
        }
        + n^{
          \frac{5}{2}
          -\frac{2}{3}
        }
      \right)^{-1}
      + 
      \left(
        n^{
          \frac{3}{2}
          - \frac{5}{3}
        }
        + n^{
          \frac{5}{2}
          - \frac{5}{3}
        }
      \right)^{-1}
      = \\ 
      = 
      \left(
        n^{
          \frac{5}{6}
        }
        + n^{
          \frac{11}{6}
        }
      \right)^{-1}
      + 
      \left(
        n^{
          \frac{-1}{6}
        }
        + n^{
          \frac{5}{6}
        }
      \right)^{-1}
      = \\ 
      = 
      \frac{1}{
        n^{ \frac{6}{6} }
        \,n^{-1/6}
        + n^{ \frac{12}{6} }
        \,n^{-1/6}
      }
      + 
      \frac{1}{
        n^{ \frac{-1}{6} }
        + n^{ \frac{6}{6} }
        \,n^{-1/6}
      }
      = \\ 
      = 
      \frac{n^{1/6-1}}{
        1 + n
      }
      + 
      \frac{n^{1/6}}{
        1 + n
      }
      = \\ 
      = 
      \frac{n^{1/6}}{n+1}
      ( n^{-1}+1)
    \end{gather*}
    Converge
  \end{tcolorbox}

  Verificando convergencia de \eqref{eq:1q4 serie 2}
  \begin{tcolorbox}
    \begin{gather*}
      \frac{1}{\sqrt{n}}
      -\frac{1}{\sqrt{n+1}}
    \end{gather*}
    Converge
  \end{tcolorbox}

  Verificando convergencia de \eqref{eq:1q4 serie 3}
  \begin{tcolorbox}
    \begin{gather*}
      \frac{
        1*3*5*\dots*(2\,n-1)
      }{
        n!
      }
      =
      \frac{
        \prod_{i=0}^{n}{2\,i+1}
      }{
        \prod_{i=0}^n{i}
      }
      = \prod_{i=0}^{n}{2+1/i}
    \end{gather*}
    converge
  \end{tcolorbox}

\end{questionBox}

\begin{questionBox}1m{} % 1Q5
  \begin{BM}
    \begin{cases}
      (\mdif{y}-2)\,x
      + (\mdif[2]{x} + 3\,\mdif{x}) y = e^{2\,t}
      \\
      (5\,\mdif[2]{y} -12\,\mdif{y} + 4)\,x
      + (5\,\mdif[3]{x}+13\,\mdif[2]{x}-7\,\mdif{x}-3)\,y
      = 8\,e^{2\,t}
    \end{cases}
  \end{BM}

  \answer{}
  \begin{tcolorbox}
    \begin{gather*}
      \begin{cases}
        (\mdif{y}-2)x + (\mdif[2]{x}+3\mdif{x})y = e^{2\,t}
        \\
        (\mdif{x}+3)y = 8\,e^{2\,t}
      \end{cases}
    \end{gather*}
  \end{tcolorbox}
\end{questionBox}

\begin{questionBox}1m{Laplace} % 1Q6
  \begin{BM}[align*]
    f(t) &= t\,e^{-t}
    ;&
    g(t) &= \Heavi(t-1)\,e^{-t}
    ;&
    h(t) &= e^{-2\,t}\,\cos(2\,t)
  \end{BM}

  \answer{}

  Solving \(f\)
  \begin{tcolorbox}
    \begin{gather*}
      \Lapl(f(t))
      = \Lapl\left(
        t\,e^{-t}
      \right)
      = \frac{1}{(s+1)^{2}}
    \end{gather*}
  \end{tcolorbox}

  Solving \(g\)
  \begin{tcolorbox}
    \begin{gather*}
      \Lapl(g)
      = \Lapl\left(
        \Heavi(t-1)\,e^{-t}
      \right)
      = \frac{e^{-(s+1)}}{s+1}
    \end{gather*}
  \end{tcolorbox}

  Solving \(h\)
  \begin{tcolorbox}
    \begin{gather*}
      \Lapl(h)
      = \Lapl(
        e^{-2\,t}\,\cos(2\,t)
      )
      = \frac{s+2}{(s+2)^2+2^2}
    \end{gather*}
  \end{tcolorbox}

\end{questionBox}

\begin{questionBox}1m{} % 1Q7
  \answer{}
  \begin{tcolorbox}
    \begin{gather*}
      \frac{4}{\pi}\sum_{n=1}^{\infty}{
        \frac{1}{2\,n-1}
        \sin((2\,n-1)\,\pi\,x)
      }
    \end{gather*}
  \end{tcolorbox}
\end{questionBox}

\begin{questionBox}1m{} % 1Q8
  \begin{BM}
    \pdv[2]{u}{x}
    - \pdv{u}{x,t}
    ; y=t;z=x+t
  \end{BM}
  \answer{}
  \begin{tcolorbox}
    \begin{gather*}
      \pdv[2]{u}{y}
      +\pdv{u}{y,z}
    \end{gather*}
  \end{tcolorbox}

  \begin{tcolorbox}
    \begin{gather*}
      \pdv[2]{u}{x}
      - \pdv{u}{x,t}
      =
      \pdv[2]{u}{z}
      \pdv[2]{z}{x}
      - \left(
        \pdv{u}{z}
        \pdv{z}{x}
      \right)
      \pdv{}{z}
      \pdv{z}{t}
    \end{gather*}
  \end{tcolorbox}
\end{questionBox}

\group{}

\begin{questionBox}1m{} % Qindex
  Det a sol geral da eq lin hom de coef const
  \begin{BM}
    \odv[4]{y}{x}
    -4\,\odv[3]{y}{x}
    +13\,\odv[2]{y}{x}
    =0
  \end{BM}
  Sabendo
  \begin{BM}
    \odv[4]{y}{x}
    -4\,\odv[3]{y}{x}
    +13\,\odv[2]{y}{x}
    =x^4+x-2
  \end{BM}
  Admite sol part \(\bar{y}=x^p\,Q_k(x)\) diga just. os val de p e k
  
  % tag   = 2q1
  % P     = \mdif[4]{x}-4\mdif[3]{x}+13\mdif[2]{x}

  \answer{\(p=2,k=4\)}

  General solution for \(y\)
  \begin{tcolorbox}
    \begin{gather*}
      y
      = y_h 
      + \bar{y}
      = \mathText{using 
        \eqref{eq:2q1 bar y}
        \eqref{eq:2q1 mapped roots for y_h}
      }
      = c_0+c_1\,x
      + e^{2\,x}
      \begin{pmatrix}
        + \cos(3\,x)\,c_2
        \\
        + \sin(3\,x)\,c_3
      \end{pmatrix}
      + \begin{pmatrix}
        x^2\,\rho_0
        \\+ x^3\,\rho_1
        \\+ x^4\,\rho_2
        \\+ x^5\,\rho_3
        \\+ x^6\,\rho_4
      \end{pmatrix}
      %
      \yesnumber\label{eq:2q1 answer}
    \end{gather*}
  \end{tcolorbox}

  % solve bar y here
  % tag   = 2q1
  % i     = 
  % P     = \mdif[4]{x}-4\mdif[3]{x}+13\mdif[2]{x}
  % alpha = 0
  % Pk    = x^4+x-2
  % k     = 4

  % y
  % (\mdif[4]{x}-4\mdif[3]{x}+13\mdif[2]{x})
  % = e^{0 x}
  % ( x^4+x-2 )

  Finding \(\bar{y}\)
  \begin{tcolorbox}
    \begin{gather*}
      \bar{y}
      = x^2
      \,Q_4(x)
      = x^2
      \,\sum_{i=0}^{4}{
        \rho_i\,x^i
      }
      = x^2
      \,\begin{pmatrix}
        x^0\,\rho_0
        \\+ x^1\,\rho_1
        \\+ x^2\,\rho_2
        \\+ x^3\,\rho_3
        \\+ x^4\,\rho_4
      \end{pmatrix}
      %
      \yesnumber\label{eq:2q1 bar y (const)}
      = \mathText{using \eqref{eq:2q1 const of bar y}}
      = \begin{pmatrix}
        x^2\,\rho_0
        \\+ x^3\,\rho_1
        \\+ x^4\,\rho_2
        \\+ x^5\,\rho_3
        \\+ x^6\,\rho_4
      \end{pmatrix}
      %
      \yesnumber\label{eq:2q1 bar y}
    \end{gather*}
  \end{tcolorbox}

  Finding constants of \eqref{eq:2q1 bar y (const)}
  \begin{tcolorbox}
    \begin{gather*}
      \bar{y}\,P
      = x^2
      \,\begin{pmatrix}
        x^0\,\rho_0
        \\+ x^1\,\rho_1
        \\+ x^2\,\rho_2
        \\+ x^3\,\rho_3
        \\+ x^4\,\rho_4
      \end{pmatrix}
      \,\left(
        \mdif[4]{x}-4\mdif[3]{x}+13\mdif[2]{x}
      \right)
      = \\
      = \begin{pmatrix}
        13*2*1\,\rho_0
        \\+ 
        -4*3*2*1\,\rho_1
        +13*3*2\,x^1\,\rho_1
        \\+ 
        4*3*2*1\,\rho_2
        -4*4*3*2\,x^1\,\rho_2
        +13*4*3\,x^2\,\rho_2
        \\+ 
        5*4*3*2\,x^1\,\rho_3
        -4*5*4*3\,x^2\,\rho_3
        +13*5*4\,x^3\,\rho_3
        \\+ 
        6*5*4*3\,x^2\,\rho_4
        -4*6*5*4\,x^3\,\rho_4
        +13*6*5\,x^4\,\rho_4
      \end{pmatrix}
      % = \\
      % = \begin{pmatrix}
      %   +26\,\rho_0
      %   -18\,\rho_1
      %   +18\,\rho_2
      %   \\
      %   +78\,x^1\,\rho_1
      %   -96\,x^1\,\rho_2
      %   +120\,x^1\,\rho_3
      %   \\
      %   +156\,x^2\,\rho_2
      %   -240\,x^2\,\rho_3
      %   +360\,x^2\,\rho_4
      %   \\
      %   +260\,x^3\,\rho_3
      %   -480\,x^3\,\rho_4
      %   \\
      %   +390\,x^4\,\rho_4
      % \end{pmatrix}
      = \\
      = \begin{pmatrix}
        + 13*2*1\,\rho_0
        - 4*3*2*1\,\rho_1
        + 4*3*2*1\,\rho_2
        \\
        x\left(
        + 13*3*2\,\rho_1
        - 4*4*3*2\,\rho_2
        + 5*4*3*2\,\rho_3
        \right)
        \\ 
        x^2\left(
          + 13*4*3\,\rho_2
          - 4*5*4*3\,\rho_3
          + 6*5*4*3\,\rho_4
        \right)
        \\ 
        x^3(
          + 13*5*4\,\rho_3
          - 4*6*5*4\,\rho_4
        )
        \\ 
        + x^4\,13*6*5\,\rho_4
      \end{pmatrix}
      = \\
      = x^4+x-2
      \implies \\
      \begin{cases}
        \rho_4 = 1/13*6*5)
        \\
        \rho_3 
        = \frac{4*6\,\rho_4}{13}
        = \frac{4*6/13*6*5}{13}
        = 4/13*13*5
        \\
          + 13*4*3\,\rho_2
          =\frac{1}{13}\left(
            1
            - (1/13)
            + (4*4/13*13)
          \right)
          \dots
      \end{cases}
      %
      \yesnumber\label{eq:2q1 const of bar y}
    \end{gather*}
  \end{tcolorbox}

  Mapping roots of \eqref{eq:2q1 roots of y_h} to solution
  \begin{tcolorbox}
    \begin{gather*}
      \begin{cases}
        r_0 = r_1 = 0
        \implies
        c_0+c_1\,x
        ;\\
        r_3 = 2 \pm i\,3
        \implies
        e^{2\,x}
        \begin{pmatrix}
          + \cos(3\,x)\,c_2
          \\
          + \sin(3\,x)\,c_3
        \end{pmatrix}
      \end{cases}
      %
      \yesnumber\label{eq:2q1 mapped roots for y_h}
    \end{gather*}
  \end{tcolorbox}

  Roots for characteristic equation for \(y_h\)
  \begin{tcolorbox}
    \begin{gather*}
      P
      = \mdif[4]{x}-4\mdif[3]{x}+13\mdif[2]{x}
      \implies \mathText{\(\mdif[i]{x} \to r^i\)}
      \implies
      r^4-4\,r^3+13\,r^2
      = r^2(r^2-4\,r+13)
      = 0
      \implies \\
      \implies
      \begin{cases}
        r_0 = r_1 = 0
        \\
        p = 2
        \\
        r_3
        = \frac{
          - (-4)
          \pm \sqrt{
            -4^2
            -4*1*13
          }
        } {2*1}
        = 2 \pm i\,3
      \end{cases}
      %
      \yesnumber\label{eq:2q1 roots of y_h}
    \end{gather*}
  \end{tcolorbox}
\end{questionBox}

\begin{questionBox}1m{} % Qindex
  \begin{BM}
    (3\,y+20\,x/y)\odif{x}
    + (2\,x-6\,y/x^2)\odif{y}
    = 0
    \\
    \phi(x,y) = x^2\,y
    \\
    x=1\implies y=2
  \end{BM}

  \answer{\eqref{eq:2q2 f}}

  Transformando em equação exata
  \begin{tcolorbox}
    \begin{gather*}
      \phi(x,y)
      (u(x,y)dx +v(x,y)dy)
      = x^2\,y
      \left(
        (3\,y+20\,x/y)\odif{x}
        + (2\,x-6\,y/x^2)\odif{y}
      \right)
      = \\
      = (3\,x^2\,y^2+20\,x^2)\odif{x}
      + (2\,x^3\,y-6\,y^2)\odif{y}
    \end{gather*}
  \end{tcolorbox}

  % tag = 2q2
  % u = 3\,x^2\,y^2+20\,x^2
  % v = 2\,x^3\,y-6\,y^2

  \answer{\eqref{eq:2q2 f}}

  Finding general solution \(f(x)\)
  \begin{tcolorbox}
    \begin{gather*}
      f(x)
      = \prim_x{u(x)}
      + \prim_y{v(y)}
      = \prim_x{\left(
          3\,x^2\,y^2+20\,x^2
      \right)}
      + \prim_y{\left(
          2\,x^3\,y-6\,y^2
      \right)}
      = \\
      = 
      3\,y^2\,(c_0 + x^3/3)
      + 20\,(c_1 + x^3/3)
      + 2\,x^3\,(c_2+y^2/2)
      - 6\,(c_3+y^3/3)
      = 0
      \implies \\
      \implies
      3\,y^2\,c_0 
      + y^2\,x^3(4)
      + x^3(
        20/3 + 2\,c_2
      )
      - 2\,y^3
      = 6\,c_3
      - 20\,c_1 
      %
      \yesnumber\label{eq:2q2 f const}
      %
      = \mathText{using \eqref{eq:2q2 const of f}}
      3\,y^2\,\left(
        \frac{1}{12}
        \left(
          - 20/3 
          + 6\,c_3
          - 20\,c_1 
          - 2\,c_2
        \right)
      \right)
      + y^2\,x^3(4)
      + x^3(
        20/3 + 2\,c_2
      )
      - 2\,y^3
      = \\
      = 6\,c_3
      - 20\,c_1 
      \yesnumber\label{eq:2q2 f}
    \end{gather*}
  \end{tcolorbox}

  finding constants in \eqref{eq:2q2 f const}
  \begin{tcolorbox}
    \begin{gather*}
      3\,(2)^2\,c_0 
      + (2)^2\,(1)^3(4)
      + (1)^3(
        20/3 + 2\,c_2
      )
      - 2\,(2)^3
      = 6\,c_3
      - 20\,c_1 
      \implies\\
     \implies
      c_0 
      = 
      \frac{1}{12}
      \left(
        - 20/3 
        + 6\,c_3
        - 20\,c_1 
        - 2\,c_2
      \right)
      \yesnumber\label{eq:2q2 const of f}
    \end{gather*}
  \end{tcolorbox}

\end{questionBox}

\setcounter{group}{3}
\group{}

\begin{questionBox}1m{laplace} % Qindex
  \begin{BM}
    y'' + 36\,y = \delta(t-\pi/6)
    ; y(0)=1, y'(0)=1
  \end{BM}
  \answer{}

  solving for y
  \begin{tcolorbox}
    \begin{gather*}
      y = \Lapl^{-1}{Y}
      = \mathText{using \eqref{eq:4q1 Y}}
      = \Lapl^{-1}{(
          \frac{1}{s^2+6^2}
          \left(
            e^{\pi/6} + 1 +s
          \right)
      )}
      = \frac{1}{6}
      \Lapl^{-1}{(
          \frac{6}{s^2+6^2}
          \left(
            e^{\pi/6} + 1 +s
          \right)
      )}
      = \\
      = \frac{1}{6}
      \sin{w\,t}
      \Lapl^{-1}{(
          \left(
            e^{\pi/6} + 1 +s
          \right)
      )}
      = \dots
    \end{gather*}
  \end{tcolorbox}

  Finding Y
  \begin{tcolorbox}
    \begin{gather*}
      \Lapl(y'') + 36\,\Lapl(y)
      = s^2\,Y - s\,y(0) - y'(0)
      + 36\,Y
      = s^2\,Y - s\,1 - 1
      + 36\,Y
      = \\
      = \Lapl(\delta(t-\pi/6))
      = e^{\pi/6}
      \implies 
      = Y 
      = \frac{1}{s^2+6^2}
      \left(
        e^{\pi/6} + 1 +s
      \right)
      \yesnumber\label{eq:4q1 Y}
    \end{gather*}
  \end{tcolorbox}

\end{questionBox}


\end{document}

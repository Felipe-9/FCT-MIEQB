% !TEX root = ./AM3C-tests_resolutions.2023.1.3.tex
\documentclass["AM3C-tests_resolutions.tex"]{subfiles}

% \tikzset{external/force remake=true} % - remake all

\begin{document}

% \graphicspath{{\subfix{./.build/figures/AM3C-tests_resolutions.2023.1.3}}}
% \tikzsetexternalprefix{./.build/figures/AM3C-tests_resolutions.2023.1.3/graphics/}

\mymakesubfile{3}[AM3C]
{Exam 2023.1.3 Resolution} % Subfile Title
{Exam 2023.1.3 Resolution} % Part Title

\group{}

\begin{questionBox}1{} % Q1
  A equação diferencial linear de primeira ordem
  \begin{BM}
    \odv{y}{x}
    + \frac{1}{4}\,x\,y
    = \frac{1}{4}\,x^3
  \end{BM}
  Com acondição \(y(0) = -4\) tem como solução:
  \answer{}
  % eq   = 1.1
  % a(x) = \left(\frac{1}{4}\,x\right)
  % b(x) = \left(\frac{1}{4}\,x^3\right)
  % y    = y
  % x    = x
  \begin{gather*}\label{eq:1.1-y}
    y
    = \frac{c_0}{\varphi(x)}
    + \frac{1  }{\varphi(x)}
    \,\int{\left(\frac{1}{4}\,x^3\right)\,\varphi(x)\,\odif{x}}
    %
    %
    %
    = 
    \usingref{eq:1.1-phi_x}
    = \frac{c_0}{\left(c_1\,e^{x^2/8}\right)}
    + \frac{1  }{\left(c_1\,e^{x^2/8}\right)}
    \,\int{\left(\frac{1}{4}\,x^3\right)\,\left(c_1\,e^{x^2/8}\right)\,\odif{x}}
    = \\ 
    = \frac{c_0}{\left(c_1\,e^{x^2/8}\right)}
    + \frac{1  }{e^{x^2/8}}
    \,\int{\left(\frac{1}{4}\,x^3\right)\,\left(e^{x^2/8}\right)\,\odif{x}}
    = 
    \usingref{eq:1.1-prim}
    = c_2\,e^{-x^2/8}
    + \frac{1}{e^{x^2/8}}
    \,\left(
      \left(x^2-8\right)\,e^{x^2/8}
    \right)
    = \\ 
    \yesnumber
    = c_2\,e^{-x^2/8}
    + x^2-8
    = \\ 
    \usingref{eq:1.1-c_2}
    = 4\,e^{-x^2/8} + x^2 - 8
    % 
    % 
    % 
    \\[3ex]
    c_2 = c_0/c_1
  \end{gather*}
  % solving y(0) = -4
  \begin{gather*}\label{eq:1.1-c_2}
    % \usingref{eq:1.1-y}
    y(0)
    = c_2\,e^{-0^2/8}+ 0^2 - 8
    = c_2 - 8
    = - 4
    \implies c_2 = 4
    \yesnumber
  \end{gather*}
  % φ(x)
  \begin{gather*} \label{eq:1.1-phi_x}
    \varphi(x) 
    = \exp{\left(
      \int{\frac{1}{4}\,x\,\odif{x}}
    \right)}
    = \exp{\left(
      \frac{1}{4} \left( \frac{x^2}{2} + c \right)
    \right)}
    = \exp{\left( \frac{c}{4} \right)}
    \,\exp{\left(
      \frac{x^2}{8}
    \right)}
    = c_1\,e^{ \frac{x^2}{8} }
    \yesnumber
    ; \\
    c_1 = e^{c/4}
  \end{gather*}
  % prim
  \begin{gather*}
    P\left(
      \left(\frac{1}{4}\,x^3\right)\,\left(e^{x^2/8}\right)
    \right)
    = 
    P\left(
      \left( x^2 \right)
      \,\left(
        e^{x^2/8}
        \,\frac{x}{4}
      \right)
    \right)
    = 
    P\left(
      \left( x^2 \right) % u
      \,\left( e^{x^2/8} \right)' % v'
    \right)
    = \\
    = 
    % u\,v
    x^2\,P\left(
      \left( e^{x^2/8} \right)'
    \right)
    % P(v\,u')
    - P\left(
      P\left(
        \odv{}{x}\left( e^{x^2/8} \right)
      \right)
      \,\odv{x^2}{x}
    \right)
    = \\
    = 
    x^2\,e^{x^2/8} 
    - P\left(
      e^{x^2/8}
      \,2\,x
    \right)
    = 
    x^2\,e^{x^2/8} 
    - 8
    \,P\left(
      e^{x^2/8}
      \,x/4
    \right)
    = \\
    = 
    \left(x^2-8\right)
    \,e^{x^2/8} 
    \yesnumber\label{eq:1.1-prim}
  \end{gather*}
\end{questionBox}

\begin{questionBox}1{} % Q2
  A equação diferencial
  \begin{BM}
    3\,x\,y^2\,\odif{x} + 4\,x^2\,y\,\odif{y} = 0
  \end{BM}
  admite um fator integrante da forma \(\varphi(x,y) = x\,y^k \), em que \textit{k} é uma constante real. Encontre \textit{k}
  \answer{}
  \begin{flalign*}
    & \notag
      k:\varphi(x,y)=x\,y^k
      \implies &\\& \notag
      \implies 
      (x\,y^k)\,3\,x\,y^2\,\odif{x}
      + (x\,y^k)\,4\,x^2\,y\,\odif{y}
      = 0
      \implies &\\& \notag{}
      \implies
      \pdv{}{y}(
        (x\,y^k)\,3\,x\,y^2
      )
      = \pdv{}{y}(
        3\,x^2\,y^{2+k}
      )
      = (2+k)\left(3\,x^2\,y^{1+k}\right)
      = &\\& \notag
      = \pdv{}{x}(
        (x\,y^k)\,4\,x^2\,y
      )
      = \pdv{}{x}(
        4\,x^3\,y^{1+k}
      )
      = 3\left(
        4\,x^2\,y^{1+k}
      \right)
      \implies &\\&
      \implies
      k
      = (12-6)/3
      = 2
    &
  \end{flalign*}
\end{questionBox}

\begin{questionBox}1{} % Q3
  Designando \(\odv{y}{x}\) por \textit{p} a solução geral da equação de Lagrange
  \begin{BM}
    y = -2\,x\,\odv{y}{x} + \frac{1}{2}\,\left(\odv{y}{x}\right)^2
  \end{BM}
  na forma paramétrica é\dots
  \answer{}
  \begin{flalign*}
    &
    \text{Equações paramétricas:}
    \begin{cases}
      x(y') = x(p)
      \\
      y(y') = y(p) = -2\,x\,p + \frac{1}{2}\,p^2
    \end{cases}
    &\\ \shortintertext{Using \eqref{eq:1.3-x_p-solution}} &
    x(p)
    = \frac{c_0}{p^{2/3}}
    + \,\frac{1}{5}\,p
    &
  \end{flalign*}
  % solving diff equation
  \begin{questionBox}*1{Solving \eqref{eq:1.3-x_p}}
    % eq   = 1.3
    % a(x) = \frac{2}{3\,p}
    % b(x) = \frac{1}{3}
    % y    = x
    % x    = p
    \begin{flalign*}\label{eq:1.3-x_p-solution}
      & 
      x
      = \frac{c_0}{\varphi(p)}
      + \frac{1  }{\varphi(p)}
      \,\int{\frac{1}{3}\,\varphi(p)\,\odif{p}}
      %
      %
      %
      = &\\[3ex] \shortintertext{Using \eqref{eq:1.3-phi_x}} &
      = \frac{c_0}{p^{2/3}}
      + \frac{1  }{p^{2/3}}
      \,\int{\frac{1}{3}\,p^{2/3}\,\odif{p}}
      %
      %
      %
      = &\\ \shortintertext{Using \eqref{eq:1.3-prim}} &
      = \frac{c_0}{p^{2/3}}
      + \frac{1  }{p^{2/3}}
      \,\frac{1}{5}\,p^{5/3}
      = \frac{c_0}{p^{2/3}}
      + \,\frac{1}{5}\,p
      % 
      \yesnumber
      & 
    \end{flalign*}

    % φ(x)
    \begin{flalign*}\label{eq:1.3-phi_x}
      & 
      \varphi(p) 
      = \exp{\left(
        \int{\frac{2}{3\,p}\,\odif{p}}
      \right)}
      = \exp{\left(
        \frac{2}{3}
        \ln{ p }
      \right)}
      = p^{2/3}
      %
      \yesnumber
      & 
    \end{flalign*}

    % prim
    \begin{flalign*}\label{eq:1.3-prim}
      & 
      P\left(
        \frac{1}{3}\,p^{2/3}
      \right)
      = \frac{1}{3}
      \,\frac{3}{5}
      \,p^{5/3}
      = \frac{1}{5}\,p^{5/3}
      %
      \yesnumber
      & 
    \end{flalign*}
  \end{questionBox}
  \begin{flalign*} \label{eq:1.3-x_p}
    &
    p
    = \odv{y}{x}
    = \odv{}{x} \left(
      -2\,x\,p + \frac{1}{2}\,p^2
    \right)
    = -2\,\left(
      x\,\odv{p}{x} 
      + p
    \right)
    + \frac{1}{2}
    \,2
    \,p
    \,\odv{p}{x}
    \implies &\\&
    \implies 
    3\,p
    = 
    \left(
      -2\,x
      + p
    \right)
    \,\odv{p}{x} 
    % \implies &\\&
    \implies
    3\,p
    \,\odv{x}{p}
    = 
    -2\,x
    + p
    \implies &\\&
    \implies
    \odv{x}{p}
    +\frac{2\,x}{3\,p}
    = 
    \frac{1}{3} 
    % 
    \yesnumber
    &
  \end{flalign*}
\end{questionBox}

\begin{questionBox}1{} % Q4
  Um sistema equivalente ao seguinte sistema de equações diferenciais lineares de coeficientes constantes é\dots
  \begin{BM}
    \begin{cases}
      (D-2)\,x + (D^2 + 3\,D)\,y = t + 1
      \\ \begin{pmatrix*}[r]
        (5\,D^2 - 12\,D + 4)\,x 
        \\ + (5\,D^3 + 13\,D^2 - 7\,D - 3)\,y
      \end{pmatrix*}
      = -2\,t + 4
    \end{cases}
  \end{BM}
  (\(D\) designa o operador de derivação a ordem \textit{t})
  \answer{}
  \begin{flalign*}
    (D-2)\,x + (D^2 + 3\,D)\,y = t + 1
    \\ \begin{pmatrix*}[r]
      (5\,D^2 - 12\,D + 4)\,x 
      \\ + (5\,D^3 + 13\,D^2 - 7\,D - 3)\,y
    \end{pmatrix*}
  \end{flalign*}
\end{questionBox}
\end{document}

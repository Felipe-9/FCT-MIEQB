% !TEX root = ./AM3C-tests_resolutions.2023.1.3.tex
\documentclass["AM3C-tests_resolutions.tex"]{subfiles}

% \tikzset{external/force remake=true} % - remake all

\begin{document}

% \graphicspath{{\subfix{./.build/figures/AM3C-tests_resolutions.2023.1.3}}}
% \tikzsetexternalprefix{./.build/figures/AM3C-tests_resolutions.2023.1.3/graphics/}

\mymakesubfile{3}[AM3C]
{Exam 2023.1.3 Resolution} % Subfile Title
{Exam 2023.1.3 Resolution} % Part Title

\group{}

\begin{questionBox}1{} % Q1
  A equação diferencial linear de primeira ordem
  \begin{BM}
    \odv{y}{x}
    + \frac{1}{4}\,x\,y
    = \frac{1}{4}\,x^3
  \end{BM}
  Com acondição \(y(0) = -4\) tem como solução:
  \answer{}
  % eq   = 1.1
  % a(x) = \left(\frac{1}{4}\,x\right)
  % b(x) = \left(\frac{1}{4}\,x^3\right)
  % y    = y
  % x    = x
  \begin{flalign*}\label{eq:1.1-y}
    & 
    y
    = \frac{c_0}{\varphi(x)}
    + \frac{1  }{\varphi(x)}
    \,\int{\left(\frac{1}{4}\,x^3\right)\,\varphi(x)\,\odif{x}}
    %
    %
    %
    = &\\[3ex]& 
    & \text{Using \eqref{eq:1.1-phi_x}} \\& 
    = \frac{c_0}{\left(c_1\,e^{x^2/8}\right)}
    + \frac{1  }{\left(c_1\,e^{x^2/8}\right)}
    \,\int{\left(\frac{1}{4}\,x^3\right)\,\left(c_1\,e^{x^2/8}\right)\,\odif{x}}
    = &\\& 
    = \frac{c_0}{\left(c_1\,e^{x^2/8}\right)}
    + \frac{1  }{e^{x^2/8}}
    \,\int{\left(\frac{1}{4}\,x^3\right)\,\left(e^{x^2/8}\right)\,\odif{x}}
    = &\\& 
    & \text{Using \eqref{eq:1.1-prim}} \\& 
    = c_2\,e^{-x^2/8}
    + \frac{1}{e^{x^2/8}}
    \,\left(
      \left(x^2-8\right)\,e^{x^2/8}
    \right)
    = &\\& 
    % \tagthiseq{}
    \yesnumber
    = c_2\,e^{-x^2/8}
    + x^2-8
    = &\\& 
    & \text{Using \eqref{eq:1.1-c_2}} \\& 
    = 4\,e^{-x^2/8} + x^2 - 8
    % 
    % 
    % 
    &\\[3ex]& 
    c_2 = c_0/c_1
    & 
  \end{flalign*}
  % solving y(0) = -4
  \begin{flalign}\label{eq:1.1-c_2}
    & \notag
      & \text{Using \eqref{eq:1.1-y}} \\&
      y(0)
      = c_2\,e^{-0^2/8}+ 0^2 - 8
      = c_2 - 8
      = - 4
      \implies c_2 = 4
    &
  \end{flalign}
  % φ(x)
  \begin{flalign} \label{eq:1.1-phi_x}
    & 
      \varphi(x) 
      = \exp{\left(
        \int{\frac{1}{4}\,x\,\odif{x}}
      \right)}
      = \exp{\left(
        \frac{1}{4} \left( \frac{x^2}{2} + c \right)
      \right)}
      = \exp{\left( \frac{c}{4} \right)}
      \,\exp{\left(
        \frac{x^2}{8}
      \right)}
      % \notag
      % = &\\&
      = c_1\,e^{ \frac{x^2}{8} }
      ; &\\& \notag
      c_1 = e^{c/4}
    & 
  \end{flalign}
  % prim
  \begin{flalign}\label{eq:1.1-prim}
    & 
    P\left(
      \left(\frac{1}{4}\,x^3\right)\,\left(e^{x^2/8}\right)
    \right)
    = 
    P\left(
      \left( x^2 \right)
      \,\left(
        e^{x^2/8}
        \,\frac{x}{4}
      \right)
    \right)
    = 
    P\left(
      \left( x^2 \right) % u
      \,\left( e^{x^2/8} \right)' % v'
    \right)
    \notag{}
    = &\\&
    = 
    % u\,v
    x^2\,P\left(
      \left( e^{x^2/8} \right)'
    \right)
    % P(v\,u')
    - P\left(
      P\left(
        \odv{}{x}\left( e^{x^2/8} \right)
      \right)
      \,\odv{x^2}{x}
    \right)
    \notag{}
    = &\\&
    = 
    x^2\,e^{x^2/8} 
    - P\left(
      e^{x^2/8}
      \,2\,x
    \right)
    = 
    x^2\,e^{x^2/8} 
    - 8
    \,P\left(
      e^{x^2/8}
      \,x/4
    \right)
    \notag{}
    = &\\&
    = 
    \left(x^2-8\right)
    \,e^{x^2/8} 
    & 
  \end{flalign}
\end{questionBox}

\begin{questionBox}1{} % Q2
  A equação diferencial
  \begin{BM}
    3\,x\,y^2\,\odif{x} + 4\,x^2\,y\,\odif{y} = 0
  \end{BM}
  admite um fator integrante da forma \(\varphi(x,y) = x\,y^k \), em que \textit{k} é uma constante real. Encontre \textit{k}
  \answer{}
  \begin{flalign*}
    & \notag
      k:\varphi(x,y)=x\,y^k
      \implies &\\& \notag
      \implies 
      (x\,y^k)\,3\,x\,y^2\,\odif{x}
      + (x\,y^k)\,4\,x^2\,y\,\odif{y}
      = 0
      \implies &\\& \notag{}
      \implies
      \pdv{}{y}(
        (x\,y^k)\,3\,x\,y^2
      )
      = \pdv{}{y}(
        3\,x^2\,y^{2+k}
      )
      = (2+k)\left(3\,x^2\,y^{1+k}\right)
      = &\\& \notag
      = \pdv{}{x}(
        (x\,y^k)\,4\,x^2\,y
      )
      = \pdv{}{x}(
        4\,x^3\,y^{1+k}
      )
      = 3\left(
        4\,x^2\,y^{1+k}
      \right)
      \implies &\\&
      \implies
      k
      = (12-6)/3
      = 2
    &
  \end{flalign*}
\end{questionBox}
\end{document}

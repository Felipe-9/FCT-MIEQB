% !TEX root = ./AM3C-tests_resolutions.2024.2.tex
\documentclass["AM3C-tests_resolutions.tex"]{subfiles}

% \tikzset{external/force remake=true} % - remake all

\begin{document}

% \graphicspath{{\subfix{./.build/figures/AM3C-tests_resolutions.2024.2}}}
% \tikzsetexternalprefix{./.build/figures/AM3C-tests_resolutions.2024.2/graphics/}

\mymakesubfile{2}[AM3C]
{Teste 2024.2 Resolução} % Subfile Title
{Teste 2024.2 Resolução} % Part Title

\group{}

\begin{questionBox}1{} % Q1
  Avalie a convergencia das seguintes séries numéricas
  \begin{BM}
    \sum_{n=1}^{\infty}{
      \frac{
        \sqrt{n+2}
        - \sqrt{n+1}
      }{
        \sqrt{n}
      }
    }
    ;\quad
    \sum_{n=1}^{\infty}{
      \frac{
        (-1)^n\,\cos(n\,\pi)
      }{
        \sqrt{n^2+1}
      }
    }
    \\
    \sum_{n=1}^{\infty}{
      \frac{1}{n!}
      \prod_{k=1}^{n}{(2\,k-1)}
    }
  \end{BM}
  \answer{}
  Using {
    \eqref{eq:1.1-sqrt},
    \eqref{eq:1.1-(-1)^n},
    \eqref{eq:1.1-cos(n*pi)},
    \eqref{eq:1.1-prod(2n-1)}
  } \(\implies\) B e C divergentes e A convergente

  \begin{gather*}
    \frac{
      \sqrt{n+2}
      - \sqrt{n+1}
    }{
      \sqrt{n}
    }
    = \sqrt{\frac{ n+2 }{ n }}
    - \sqrt{\frac{ n+1 }{ n }}
    = \sqrt{ 1+2/n }
    - \sqrt{ 1+1/n }
    \yesnumber\label{eq:1.1-sqrt}
    % 
    % 
    % 
    ;\\[1ex]
    (-1)^n
    = \begin{cases}
      +1 &\quad n=2*m
      \\
      -1 &\quad n=2*m-1
    \end{cases}
    \yesnumber\label{eq:1.1-(-1)^n}
    % 
    % 
    % 
    ;\\[1ex]
    \cos(n\},\pi)
    = \begin{cases}
      +1 &\quad n=2*m
      \\
      -1 &\quad n=2*m-1
    \end{cases}
    \yesnumber\label{eq:1.1-cos(n*pi)}
    % 
    % 
    % 
    ;\\[1ex]
    \frac{
      \prod_{k=1}^{n}{(2\,k-1)}
    }{n!}
    = \frac{
      \prod_{k=1}^{n}{(2\,k-1)}
    }{
      \prod_{k=1}^n{k}
    }
    = \prod_{k=1}^{n}{
      \frac{ 2\,k-1 }{ k }
    }
    = \prod_{k=1}^{n}{
      2-1/k
    }
    \yesnumber\label{eq:1.1-prod(2n-1)}
  \end{gather*}
\end{questionBox}

\begin{questionBox}1{} % Q2
  Considere a série de funções redútivel e avalie sua convergencia
  \begin{BM}
    \sum_{n=1}^{\infty}{\left(
      \frac{1}{n\,x+2}
      - \frac{1}{(n+1)\,x+2}
    \right)}
    ,\quad x\in\myrange{0,1}
  \end{BM}
  \answer{}
  Usando \eqref{eq:1.2-ans} A série converge uniformemente para a função identicamente nula 
  \begin{gather*}
    \sum_{n=1}^{\infty}{\left(
      \frac{1}{n\,x+2}
      - \frac{1}{(n+1)\,x+2}
    \right)}
    = \sum_{n=1}^{\infty}{\left(
      \frac{
        n\,x+2 + x
        - (n\,x+2)
      }{
        (n\,x+2)^2
        +(n\,x+2)\,x
      }
    \right)}
    = \\
    = \sum_{n=1}^{\infty}{\left(
      \frac{
        x
      }{
        (
          n^2\,x^2
          + 4\,n\,x
          + 4
        )
        + (n\,x^2+2\,x)
      }
    \right)}
    = \\
    = \sum_{n=1}^{\infty}{\left(
      \frac{
        1
      }{
        n^2\,x
        + 4\,n
        + x + 2 + 4/x
      }
    \right)}
    \yesnumber\label{eq:1.2-ans}
  \end{gather*}
\end{questionBox}

\begin{questionBox}1{} % Q3
  A equação diferencial \(x^2\,y"+y"-4\,x\,y'+6\,y=0\) admite uma solução da forma \(y=\sum_{n=0}^{\infty}{a_n\,x^n}\). Então a sucessão (\(a_n\)) verifica a relação de recorrencia\dots
  \answer{}
  \begin{gather*}
    a_{n+2} = \frac{n-3}{n+1}\,a_n,\quad n=2,3,\dots
  \end{gather*}
\end{questionBox}

\begin{questionBox}1{} % Q4
  A função \(f(x)\) é par, periódica de período \(p=2\) e no intervalo \(\myrange*{0,1}\) está definida por \(f(x)=x\). Então a sua série de Fourrier é:
  \answer{}
  \begin{gather*}
    \frac{1}{2}
    - \frac{2}{\pi}
    \sum_{n=1}^{\pi}{
      \frac{\cos(n\,\pi\,x)}{ n^2 }
    }
  \end{gather*}
\end{questionBox}

\begin{questionBox}1{} % Q5
  A equação com derivadas parciais
  \begin{BM}
    \pdv{u}{y} = y\,\pdv{u}{x}
  \end{BM}
  com as condições \(u(1,1)=e\) e \(u(0,0)=1\) admite uma solução da forma \(u(x,y)=X(x)\,Y(y)\). Então \(u(1,0)=?\)
  \answer{}
  \begin{gather*}
    u(0,1) = \sqrt{e}
  \end{gather*}
\end{questionBox}

\group{}

\begin{questionBox}1{} % Q1
  Estude a convergencia da série:
  \begin{BM}
    \sum_{n=1}^{\infty}{
      \frac{n^n\,n!}{(2\,n)!}
    }
  \end{BM}
  \answer{}
  \begin{gather*}
    \frac{n^n\,n!}{(2\,n)!}
    = \frac{
      \prod_{k=1}^{n}{n}
      \,\prod_{k=1}^{k}
    }{
      \prod_{k=1}^{2\,n}{k}
    }
    = \frac{
      \prod_{k=1}^{n}{n}
      \,\prod_{k=1}^{n}{k}
    }{
      \prod_{k=1}^{n}{k}
      \prod_{k=n+1}^{2\,n}{k}
    }
    = \frac{
      \prod_{k=1}^{n}{n}
    }{
      \prod_{k=n+1}^{2\,n}{k}
    }
    = \prod_{n+1}^{2\,n}{
      \frac{n}{k}
    }
    % = \prod_{n+1}^{2\,n}{
    %   \frac{1}{k/n}
    % }
    % = \\
    % = \prod_{n+1}^{2\,n}{
    %   \frac{1}{1+k/n}
    % }
    = \\
    = \frac{n}{n+1}
    * \frac{n}{n+2}
    * \dots
    * \frac{n}{2\,n-1}
    * \frac{1}{2}
    = \\
    = \frac{1}{1+1/n}
    * \frac{1}{1+2/n}
    * \dots
    * \frac{1}{1+(n-1)/n}
    * \frac{1}{2}
    = \prod_{k=1}^{n}{
      \frac{1}{1+k/n}
    }
    \implies \\ 
    \implies
    \lim_{n\to\infty}{
      \frac{n^2\,n!}{(2\,n)!}
    }
    = \lim_{n\to\infty}{
      \prod_{k=1}^{n}{
        \frac{1}{1+k/n}
      }
    }
    \approx \left(\frac{1}{2}\right)^{\infty} = 0
  \end{gather*}
  A série converge uniformemente para zero
\end{questionBox}

\begin{questionBox}1{} % Q2
  Determine o intervalo de convêrgencia da série de potências
  \begin{BM}
    \sum_{n=0}^{\infty}{
      \frac{(-1)^n}{4^n}
      \,x^{2\,n}
    }
    ,\qquad
    \text{sug: Faça } t=x^2
  \end{BM}
  Determine a soma de série no ponto \(x=-1\). Designando por \(f(x)\) a função que a série representa no seu intervalo de convergencia, determine, justificando, o valor de \(f^{(21)}(0)\)
  \answer{}
  \begin{gather*}
    \frac{(-1)^n}{4^n}
    \,x^{2\,n}
    = \frac{(-1)^n}{4^n}
    \,t^{n}
    = (-1)^n\,(t/4)^{n}
    \\
    \begin{Bmatrix}
      \pm\infty\text{(diverge)} &\quad: t>4
      \\\pm 1 &\quad: t=4
      \\ 0 &\quad: t<4
    \end{Bmatrix}
    = \begin{Bmatrix}
      \pm\infty\text{(diverge)} &\quad: \myvert{x}>2
      \\\pm 1 &\quad: \myvert{x}=2
      \\ 0 &\quad: \myvert{x}<2
    \end{Bmatrix}
  \end{gather*}
  
  \begin{gather*}
    \sum_{n=0}^{\infty}{
      \frac{(-1)^n}{4^n}
      \,(-1)^{2\,n}
    }
    = \sum_{n=0}^{\infty}{
      \frac{(-1)^n}{4^n}
      \,1^{n}
    }
    = \sum_{n=0}^{\infty}{
      \frac{(-1)^n}{4^n}
    }
    = \sum_{k=0}^{\infty}{
      -\frac{1}{4^{2\,k+1}}
    }
    + \sum_{k=0}^{\infty}{
      \frac{1}{4^{2\,k}}
    }
    = \\
    = \sum_{k=0}^{\infty}{
      \frac{-1+4}{4*4^{2\,k}}
    }
    = \frac{3}{4}
    \,\sum_{k=0}^{\infty}{
      \frac{1}{16^{k}}
    }
  \end{gather*}

  \begin{gather*}
    f^{(21)}(x)
    = \frac{(-1)^{21}}{4^{21}}\,0^{2\,n}
    = \frac{-1}{4^{21}}
  \end{gather*}
\end{questionBox}

\group{}

\begin{questionBox}1{} % Q1
  Acerca de uma função periódica \(f(x)\) sabe-se que é contínua, tem derivada seccionalmente contínua e tem por série de Fourrier
  \begin{BM}
    \frac{2}{\pi}
    + \sin(2\,\pi\,x)
    -\frac{4}{\pi}
    \,\sum_{n=1}^{\infty}{
      \frac{1}{(2\,n-1)(2\,n+1)}
      \,\cos(2\,n\,\pi\,x)
    }
  \end{BM}
  Diga justificando, a pariaridade e o período positivo mínimo da função \(f(x)\). Sabendo que no ponto \(x=1/2\) a função toma o valor um, determine a soma da série
  \begin{BM}
    \sum_{n=1}^{\infty}{
      \frac{ (-1)^n }{
        (2\,n-1)
        (2\,n+1)
      }
    }
  \end{BM}
\end{questionBox}

\group{} % G4

\begin{questionBox}1{} % Q1
  Seja  \(f(x,y)\) uma solução da equação com derivadas parciais
  \begin{BM}
    x^2\,\pdv[order=2]{f}{x}
    + y^2\,\pdv[order=2]{f}{y}
    + x\,\pdv{f}{x}
    + y\,\pdv{f}{y}
    = 0
  \end{BM}
  e seja \(g(s,t) = f(x,y)\) em que \(x=e^s\) e \(y=e^t\). Mostre que \(g\) é solução da equação de Laplace, isto é, satisfaz a equação
  \begin{BM}
    \pdv[order=2]{g}{s}
    +\pdv[order=2]{g}{t}
    = 0 
  \end{BM}
\end{questionBox}

\end{document}

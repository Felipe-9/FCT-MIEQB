% !TEX root = ./AM3C-tests_resolutions.2024.1.tex
\documentclass["AM3C-tests_resolutions.tex"]{subfiles}

% \tikzset{external/force remake=true} % - remake all

\begin{document}

% \graphicspath{{\subfix{./.build/figures/AM3C-tests_resolutions.2024.1}}}
% \tikzsetexternalprefix{./.build/figures/AM3C-tests_resolutions.2024.1/graphics/}

\mymakesubfile{1}[AM 3C]
{Teste 2024.1 Resolucão} % Subfile Title
{Teste 2024.1 Resolução} % Part Title

\group{}

\begin{questionBox}1{edo ord 1} % Q1
  A equação dif lin ord 1
  \begin{BM}[align*]
    \odv{y}{x} + \frac{\cos(x)}{\sin(x)}\,y
    &= -x
    ;& x &\in \myrange*{0,\pi}
  \end{BM}
  \answer{}
  \begin{BM}
    y = \frac{c}{\sin(x)} - \frac{\cos(x)}{\sin(x)}
  \end{BM}
  \begin{flalign*}
    &
      \left(
        \mdif[1]{x} + \frac{\cos(x)}{\sin(x)}
      \right)\,y = -x
      % 
      % 
      % 
      &\\[3ex]&
      % a(x) = \frac{\cos(x)}{\sin(x)}
      % b(x) = -1
      % y    = y
      % x    = x
      y
      = \frac{c_0}{\varphi(x)}
      + \frac{1  }{\varphi(x)}
      \,\int{-1\,\varphi(x)\,\odif{x}}
      %
      %
      %
      = &\\[3ex]&
      = \frac{c_0}{c_2}
      + \frac{1  }{c_2}
      \,\int{-1\,c_2\,\odif{x}}
      %
      %
      %
      ; &\\[3ex]&
      \varphi(x) 
      = \exp{\left(
        \int{\frac{\cos(x)}{\sin(x)}\,\odif{x}}
      \right)}
      = e^{1/2 + c_1}
      = c_2
      %
      %
      %
      &\\[3ex]&
      \odif{\cos(x)} = -\sin(x)\,\odif{x}
      % 
      % 
      % 
      &\\[1ex]&
      \odif{\left(\frac{1}{\sin(x)}\right)}
      = \frac{1}{-2\,\sin^2(x)} (\cos(x))
      \,\odif{x}
      \implies &\\&
      \implies
      \odif(x)
      = \frac{-2\,\sin^2(x)}{\cos(x)}
      \,\odif{\left(\frac{1}{\sin(x)}\right)}
      %
      %
      %
      &\\[3ex]&
      \int{u\,v'\,\odif{x}}
      = u\,v + \int{u'\,v\,\odif{x}}
      %
      %
      %
      &\\[3ex]&
      \int{
        \frac{1}{\sin(x)}
        \,\cos(x)
        \,\odif{x}
      }
      = \int{
        \frac{1}{\sin(x)}
        \,\odv{}{x}{\left(\sin(x)\right)}
        \,\odif{x}
      }
      = &\\&
      = \frac{\sin(x)}{\sin(x)}
      + \int{
        \odv{}{x}{\left(\frac{1}{\sin(x)}\right)}
        \,\sin(x)
        \,\odif{x}
      }
      = 1
      + \int{
        \left(
          -\frac{\cos(x)}{\sin^2(x)}
        \right)
        \,\sin(x)
        \,\odif{x}
      }
      \implies &\\&
      \implies
      \int{\frac{\cos(x)}{\sin(x)}\,\odif{x}}
      = 1/2
    &
  \end{flalign*}
\end{questionBox}

\begin{questionBox}1{bernoulli} % Q2
  A solução da eq de bernoulli
  \begin{BM}
    \odv{y}{x} + y = \frac{1}{y'}
  \end{BM}
  que satisfaz a condição \(y(0)=2\) é
  \answer{}
  \begin{BM}
    y = \sqrt{e^{-2\,x}\,3 + 1}
  \end{BM}
  \begin{flalign*}
    &
      % a(x) = 1
      % b(x) = 1
      % k    = -1
      y' + y = y^{-1}
      %
      %
      %
      &\\[3ex]&
      y
      = \sqrt{z}
      = \sqrt{
        e^{-2\,x}\,3 + 1
      }
      % 
      % 
      % 
      ; &\\[3ex]&
      c: y(0)
      = \sqrt{
        z(0)
      }
      = \sqrt{
        e^{-2* 0}\,c + 1
      }
      = \sqrt{ c + 1 }
      = 2
      \implies
      c = 4-1 = 3
      % 
      % 
      % 
      ; &\\[3ex]&
      z=y^{2} 
      \implies &\\&
      \implies
      z' + (1+1)1\,z = z' + 2\,z 
      = (1+ 1) = 2
      &\\[3ex]&
      % 
      % a(x) = 2
      % b(x) = 2
      % y    = z
      % x    = x
      z
      = \frac{c_0}{\varphi(x)}
      + \frac{1  }{\varphi(x)}
      \,\int{2\,\varphi(x)\,\odif{x}}
      %
      %
      %
      = &\\[3ex]&
      = \frac{c_0}{c_2\,e^{2\,x}}
      + \frac{1  }{c_2\,e^{2\,x}}
      \,\int{2\,c_2\,e^{2\,x}\,\odif{x}}
      = &\\&
      = e^{-2\,x}\,\frac{c_0}{c_2}
      + \frac{2\,c_2}{c_2\,e^{2\,x}}
      \left(
        \frac{e^{2\,x}}{2} + c_3
      \right)
      = e^{-2\,x}\,\frac{c_0}{c_2}
      + 1
      + \frac{2}{e^{2\,x}}\,c_3
      = e^{-2\,x}\,c + 1
      %
      %
      %
      ; &\\[3ex]&
      \varphi(x) 
      = \exp{\left(
        \int{2\,\odif{x}}
      \right)}
      = \exp{\left(
          2\,x + c_1
      \right)}
      = c_2\,e^{2\,x}
    &
  \end{flalign*}
\end{questionBox}

\begin{questionBox}1{fator int} % Q3
  A equação differencial
  \begin{BM}
    (5\,x\,y^2 - 2\,y)\,\odif{x} + (3\,x^2\,y-x)=0
  \end{BM}
  Admite um fator integrante na forma \(\phi(x,y)=x^m\,y^n\), com \(m,n \in \mathbb{N}\). então
  \answer{}
  \begin{BM}
    m=3,n=2
  \end{BM}
\end{questionBox}

\begin{questionBox}1{met var const arb} % Q4
  A eq dif hom
  \begin{BM}
    x\,y" + x^2\,y' + 4\,y = x^3
  \end{BM}
  Tem como solução geral a função \(y(x) = c_1(x)\,y_1(x) + c_1(x)\,y_2(x)\). \dots
  \answer{}
  \begin{flalign*}
    &
      % a_0(x) = 4
      % a_1(x) = x^2
      % a_2(x) = x
      % f(x)   = x^3
      % y_1(x) = y_1(x)
      % y_2(x) = y_2(x)
      y:
      \begin{pmatrix}
          4
        \\ + x^2\,\mdif[1]{x}
        \\ + x\,\mdif[2]{x}
      \end{pmatrix}
      \,y
      = x^3
      %
      %
      %
      &\\[3ex]&
      \begin{Bmatrix}
        {
            c_1'(x)\,\mdif[0]{x}y_1(x) 
          + c_2'(x)\,\mdif[0]{x}y_2(x)
        } = 0
        \\ {
            c_1'(x)\,\mdif[1]{x}\,y_1(x) 
          + c_2'(x)\,\mdif[1]{x}\,y_2(x)
        } = \frac{x^3}{x} = x^2
      \end{Bmatrix}
    &
  \end{flalign*}
\end{questionBox}

\begin{questionBox}1{Transf laplace} % Q5
  \begin{itemize}
    \item \(f(t)\) def em ord ate 2 em \(\mathbb{R}^+_0\)
  \end{itemize}
  \answer{}
  \begin{BM}
    (s+1)^2\,F(s+1) - s + 1 - s\,F'(s) - F(s)
  \end{BM}
\end{questionBox}

\group{}

\begin{questionBox}*1{} % II Q
  Det sol ger da eq lin hom de coef const.
  \begin{BM}
    \odv[2]{y}{x}
    + \odv{y}{x}
    -y\,6 = 0
  \end{BM}
  \answer{}
  \begin{flalign*}
    &
    % P = \mdif[2]{x} + \mdif[1]{x} -6
      % φ(x) = \varphi(x)
      P\,y
      = \left(
        \mdif[2]{x} + \mdif[1]{x} -6
      \right)
      \,y
      = 0
      %
      %
      %
      ; &\\[3ex]&
      y 
      = \varphi(x)\,\int{z(x)\,\odif{x}}
      = \varphi(x)\,\int{
        z(x)
        \,\odif{x}
      }
      %
      %
      %
      ; &\\[3ex]&
      P\,y
      = \left(
        \mdif[2]{x} + \mdif[1]{x} -6
      \right)
      \,\left(
        \varphi(x)\,\int{
          z(x)\,\odif{x}
        }
      \right)
      % = &\\&
      = 0
      %
      %
      %
      ; &\\[3ex]&
      \mdif[1]{x}{y}
      = \mdif[1]{x}{\left(
        \varphi(x)
        \,\int{z(x)\,\odif{x}}
      \right)}
      %
      %
      %
      ; &\\[3ex]&
      \mdif[2]{x}{y}
      = \mdif[2]{x}{\left(
        \varphi(x)
        \,\int{z(x)\,\odif{x}}
      \right)}
    &
  \end{flalign*}
  
\end{questionBox}

\begin{questionBox}*1{met ver const arb} % II Q2
  util o met da var das const arb det a sol ger da eq n homog
  \begin{BM}
    \odv[2]{y}{x}
    + \odv{y}{x}
    -y\,6 
    = -5\,e^{2\,x}\,\cos(x)
  \end{BM}
  \answer{}
  \begin{flalign*}
    &
      % a_0(x) = -6
      % a_1(x) = 1
      % a_2(x) = 1
      % f(x)   = \left(-5\,e^{2\,x}\,\cos(x)\right)
      % y_1(x) = y_1(x)
      % y_2(x) = y_2(x)
      y:
      \begin{pmatrix}
          -6
        \\ + 1\,\mdif[1]{x}
        \\ + 1\,\mdif[2]{x}
      \end{pmatrix}
      \,y
      = \left(-5\,e^{2\,x}\,\cos(x)\right)
      %
      %
      %
      &\\[3ex]&
      y
      = c_1(x)\,y_1(x)
      + c_2(x)\,y_2(x)
      %
      %
      %
      ; &\\[3ex]&
      % Regra de Crammer
      c_1(x) 
      = \int{c_1'(x)\,\odif{x}}
      %
      %
      ; &\\[1ex]&
      c_2(x) 
      = \int{c_2'(x)\,\odif{x}}
      %
      %
      %
      &\\[3ex]&
      c_1'(x)
      = \frac{1}{W(y_1(x),y_2(x))}
      \,\begin{vmatrix}
        0 
        &  \mdif[0]{x}y_2(x)
        \\ -5\,e^{2\,x}\,\cos(x)
        &  \mdif[1]{x}y_2(x)
      \end{vmatrix}
      %
      %
      %
      &\\[3ex]&
      c_2'(x)
      = \frac{1}{W(y_1(x),y_2(x))}
      \,\begin{vmatrix}
           \mdif[0]{x}y_1(x)
        &  0 
        \\ \mdif[1]{x}y_1(x)
        &  -5\,e^{2\,x}\,\cos(x)
      \end{vmatrix}
      %
      %
      %
      &\\[3ex]&
      % Wronskiano
      W(y_1(x),y_2(x))
      = \det\begin{bmatrix}
           \mdif[0]{x}\,y_1(x)
        &  \mdif[0]{x}\,y_2(x)
        \\ \mdif[1]{x}\,y_1(x)
        &  \mdif[1]{x}\,y_2(x)
      \end{bmatrix}
      %
      %
      %
      &\\[3ex]&
      \begin{Bmatrix}
        {
            c_1'(x)\,\mdif[0]{x}y_1(x) 
          + c_2'(x)\,\mdif[0]{x}y_2(x)
        } &=& 0
        \\ {
            c_1'(x)\,\mdif[1]{x}\,y_1(x) 
          + c_2'(x)\,\mdif[1]{x}\,y_2(x)
        } &=& -5\,e^{2\,x}\,\cos(x)
      \end{Bmatrix}
      %
      %
      %
      ; &\\[3ex]&
      \mdif[1]{x}\,y_1(x)
      = \mdif[1]{x}\,y_1(x)
      %
      %
      ; &\\[1ex]&
      \mdif[1]{x}\,y_2(x)
      = \mdif[1]{x}\,y_2(x)
    &
  \end{flalign*}
\end{questionBox}

\group{}
\setcounter{question}{1}

\begin{questionBox}2{} % III Q1.1
  Det todas as sol de cliraut
  \begin{BM}
    y = x\,\odv{y}{x} - \left(\odv{y}{x}\right)^3
  \end{BM}
\end{questionBox}

\begin{questionBox}2{question} % III Q1.2
  mud de var \(x=1/t\) resolva
  \begin{BM}
    y = -x\,\odv{y}{x} + x^6\,\left(\odv{y}{x}\right)^3
    \qquad x>0
  \end{BM}
\end{questionBox}

\setcounter{group}{4}
\group{}

\begin{questionBox}1{} % III
  
\end{questionBox}

\end{document}

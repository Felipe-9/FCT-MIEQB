% !TEX root = ./AM3C-tests_resolutions.2024.1.tex
\documentclass["AM3C-tests_resolutions.tex"]{subfiles}

% \tikzset{external/force remake=true} % - remake all

\begin{document}

% \graphicspath{{\subfix{./.build/figures/AM3C-tests_resolutions.2024.1}}}
% \tikzsetexternalprefix{./.build/figures/AM3C-tests_resolutions.2024.1/graphics/}

\mymakesubfile{1}[AM3C]
{Teste 2024.1 Resolucão} % Subfile Title
{Teste 2024.1 Resolução} % Part Title

\group{}

\begin{questionBox}1{} % Q1
  A equação diferencial linear de primeira ordem
  \begin{BM}
    \odv{y}{x} + \frac{\cos(x)}{\sin(x)}\,y
    = -x
    ;\quad x \in \myrange*{0,\pi}
  \end{BM}
  Tem como solução geral
  \begin{itemize}[label=\Box]
    \begin{multicols}{3}
      \item \(y = \frac{c}{\sin x} - x\,\frac{\cos x}{\sin x} +1\)
      \item \(y = \frac{c}{\cos x} - x\,\frac{\cos x}{\sin x} +1\)
      \item \(y = \frac{c}{\sin x} + x\,\frac{\sin x}{\cos x} -1\)
      \item \(y = \frac{c}{\cos x} + x\,\frac{\cos x}{\sin x} -1\)
      \item \(y = \frac{c}{\sin x} - x\,\frac{\cos x}{\sin x}\)
      \item \(y = \frac{c}{\cos x} + x\,\frac{\sin x}{\cos x}\)
    \end{multicols}
  \end{itemize}

  \answer{}

  % tag  = 1.1
  % a(x) = \frac{\cos{x}}{\sin{x}}
  % b(x) = (-x)
  % y    = y
  % x    = x
  \begin{gather*}
    y
    = \frac{c_0}{\varphi(x)}
    + \frac{1  }{\varphi(x)}
    \,\int{(-x)\,\varphi(x)\,\odif{x}}
    %
    %
    %
    = \mathText{Using \eqref{eq:1.1-phi_x}}
    = \frac{c_0}{C_2\,\sin{x}}
    + \frac{1  }{C_2\,\sin{x}}
    \,\int{(-x)\,C_2\,\sin{x}\,\odif{x}}
    %
    %
    %
    = \mathText{Using \eqref{eq:1.1-prim}}
    = \frac{c_0}{C_2\,\sin{x}}
    + \frac{1  }{C_2\,\sin{x}}
    \, C_2\,(
      x\,\cos{x}
      - \sin{x}
      - C_3
    )
    = \\
    = \frac{C_4}{\sin{x}}
    + x\,\frac{\cos{x}}{\sin{x}}
    - 1
    - \frac{C_3}{\sin{x}}
    % = \\
    = \frac{C_5}{\sin{x}}
    + x\,\frac{\cos{x}}{\sin{x}}
    - 1
  \end{gather*}
  % φ(x)
  \begin{gather*}
    \varphi(x) 
    = \exp{\left(
      \int{\frac{\cos{x}}{\sin{x}}\,\odif{x}}
    \right)}
    = \exp{\left(
        \int{\frac{\odif{(\sin{x})}}{\sin{x}}}
    \right)}
    = \exp{\left(
        \ln{(\sin{x})}+C_1
    \right)}
    = C_2\,\sin{x}
    %
    \yesnumber\label{eq:1.1-phi_x}
  \end{gather*}
  % prim
  \begin{gather*}
    P\left(
      -x\,C_2\,\sin{x}
    \right)
    = C_2\,P\left(
      x\,(-\sin{x})
    \right)
    = C_2\,P\left(
      x\,\odif{\cos{x}}
    \right)
    = \mathText{using \(P(u\,v')=u\,v-P(u'\,v)\)}
    = C_2\,(
      x\,\cos{(x)}
      - P(\cos{x})
    )
    = C_2\,(
      x\,\cos{(x)}
      - \sin{x}
      - C_3
    )
    %
    \yesnumber\label{eq:1.1-prim}
  \end{gather*}

\end{questionBox}

\begin{questionBox}1{} % Q2
  A solução da equação de Bernoulli
  \begin{BM}
    \odv{y}{x} + y = \frac{1}{y}
  \end{BM}
  que satisfaz a condição \(y(0)=2\), é:
  \begin{itemize}[label=\square]
    \begin{multicols}{3}
      \item \(y=\sqrt{e^{+2\,x}+3}\)
      \item \(y=\sqrt{e^{-2\,x}+3}\)
      \item \(y=\sqrt{3\,e^{+2\,x}+1}\)
      \item[\blacksquare]\(y=\sqrt{3\,e^{-2\,x}+1}\)
      \item \(y=\sqrt{2\,e^{+2\,x}+2}\)
      \item \(y=\sqrt{2\,e^{-2\,x}+2}\)
    \end{multicols}
  \end{itemize}

  \answer{}

  \begin{gather*}
    y
    = \sqrt{z}
    % 
    \yesnumber\label{eq:1.2-y=z}
    = \mathText{Using \eqref{eq:1.2.0.1-z(x)}}
    = \sqrt{
      c_6\,e^{-2\,x}
      +1
    }
    \yesnumber\label{eq:1.2-y(x c)}
    = \mathText{Using \eqref{eq:1.2-c_6}}
    = \sqrt{3\,e^{-2\,x}+1}
  \end{gather*}
  % y(0)
  Using \eqref{eq:1.2-y(x c)}
  \begin{gather*}
    y(0)
    = \sqrt{
      c_6\,e^{-2*0}
      + 1
    }
    = \sqrt{
      c_6
      + 1
    }
    = 2
    \implies
    c_6=3
    \yesnumber\label{eq:1.2-c_6}
  \end{gather*}

  % tag  = 1.2
  % a(x) = 1
  % b(x) = 1
  % k    = (-1)
  \begin{gather*}
    y' + y = \frac{1}{y}
    \implies \mathText{Using \eqref{eq:1.2-y=z}}
    \implies 
    z' + 2\,z = 2
    \yesnumber\label{eq:1.2-z(z')}
  \end{gather*}

  \begin{questionBox}*3m{Solving \(z\)} % Q
    % tag  = 1.2.0.1
    % a(x) = 2
    % b(x) = 2
    % y    = z
    % x    = x
    Using \eqref{eq:1.2-z(z')}
    \begin{gather*}
      z
      = \frac{c_0}{\varphi(x)}
      + \frac{1  }{\varphi(x)}
      \,\int{2\,\varphi(x)\,\odif{x}}
      %
      %
      %
      = \mathText{Using \eqref{eq:1.2.0.1-phi_x}}
      = \frac{c_0}{c_2\,e^{2\,x}}
      + \frac{1  }{c_2\,e^{2\,x}}
      \,\int{2\,c_2\,e^{2\,x}\,\odif{x}}
      = c_4\,e^{-2\,x}
      + \frac{1  }{c_2\,e^{2\,x}}
      \,\int{2\,c_2\,e^{2\,x}\,\odif{x}}
      %
      %
      %
      = \mathText{Using \eqref{eq:1.2.0.1-prim}}
      = c_4\,e^{-2\,x}
      + \frac{1  }{c_2\,e^{2\,x}}
      \,c_2\,(e^{2\,x}+c_3)
      = c_4\,e^{-2\,x}
      + 1
      + c_5\,e^{-2\,x}
      = c_6\,e^{-2\,x}
      + 1
      %
      \yesnumber\label{eq:1.2.0.1-z(x)}
    \end{gather*}
    % φ(x)
    \begin{gather*}
      \varphi(x) 
      = \exp{\left(
        \int{2\,\odif{x}}
      \right)}
      = \exp{2\,(x+c_1)}
      = c_2\,e^{2\,x}
      %
      \yesnumber\label{eq:1.2.0.1-phi_x}
    \end{gather*}
    % prim
    \begin{gather*}
      P\left(
        2\,\varphi(x)
      \right)
      \mathText{Using \eqref{eq:1.2.0.1-phi_x}}
      = P\left(
        2\,c_2\,e^{2\,x}
      \right)
      = c_2\,(e^{2\,x}+c_3)
      %
      \yesnumber\label{eq:1.2.0.1-prim}
    \end{gather*}  
  \end{questionBox}

  % \answer{}
  %
  % \begin{BM}
  %   y = \sqrt{e^{-2\,x}\,3 + 1}
  % \end{BM}
  % \begin{flalign*}
  %   &
  %     % a(x) = 1
  %     % b(x) = 1
  %     % k    = -1
  %     y' + y = y^{-1}
  %     %
  %     %
  %     %
  %     &\\[3ex]&
  %     y
  %     = \sqrt{z}
  %     = \sqrt{
  %       e^{-2\,x}\,3 + 1
  %     }
  %     % 
  %     % 
  %     % 
  %     ; &\\[3ex]&
  %     c: y(0)
  %     = \sqrt{
  %       z(0)
  %     }
  %     = \sqrt{
  %       e^{-2* 0}\,c + 1
  %     }
  %     = \sqrt{ c + 1 }
  %     = 2
  %     \implies
  %     c = 4-1 = 3
  %     % 
  %     % 
  %     % 
  %     ; &\\[3ex]&
  %     z=y^{2} 
  %     \implies &\\&
  %     \implies
  %     z' + (1+1)1\,z = z' + 2\,z 
  %     = (1+ 1) = 2
  %     &\\[3ex]&
  %     % 
  %     % a(x) = 2
  %     % b(x) = 2
  %     % y    = z
  %     % x    = x
  %     z
  %     = \frac{c_0}{\varphi(x)}
  %     + \frac{1  }{\varphi(x)}
  %     \,\int{2\,\varphi(x)\,\odif{x}}
  %     %
  %     %
  %     %
  %     = &\\[3ex]&
  %     = \frac{c_0}{c_2\,e^{2\,x}}
  %     + \frac{1  }{c_2\,e^{2\,x}}
  %     \,\int{2\,c_2\,e^{2\,x}\,\odif{x}}
  %     = &\\&
  %     = e^{-2\,x}\,\frac{c_0}{c_2}
  %     + \frac{2\,c_2}{c_2\,e^{2\,x}}
  %     \left(
  %       \frac{e^{2\,x}}{2} + c_3
  %     \right)
  %     = e^{-2\,x}\,\frac{c_0}{c_2}
  %     + 1
  %     + \frac{2}{e^{2\,x}}\,c_3
  %     = e^{-2\,x}\,c + 1
  %     %
  %     %
  %     %
  %     ; &\\[3ex]&
  %     \varphi(x) 
  %     = \exp{\left(
  %       \int{2\,\odif{x}}
  %     \right)}
  %     = \exp{\left(
  %         2\,x + c_1
  %     \right)}
  %     = c_2\,e^{2\,x}
  %   &
  % \end{flalign*}
\end{questionBox}

\begin{questionBox}1{} % Q3
  A equação differencial
  \begin{BM}
    (5\,x\,y^2 - 2\,y)\,\odif{x} + (3\,x^2\,y-x)\,\odif{y}=0
  \end{BM}
  Admite um fator integrante na forma \(\phi(x,y)=x^m\,y^n\), com \(m,n \in \mathbb{N}\). Então:
  \begin{itemize}[label=\square]
    \begin{multicols}{3}
      \item \(m=3,n=2\)
      \item \(m=1,n=1\)
      \item \(m=2,n=2\)
      \item \(m=2,n=1\)
      \item \(m=1,n=3\)
      \item \(m=3,n=1\)
    \end{multicols}
  \end{itemize}
  \answer{}
  \begin{BM}
    m=3,n=2
  \end{BM}

  \answer{}

  % f1:  (5\,x\,y^2-2\,y)
  % f2:  (3\,x^2\,y-x)
  % φ:   x^m\,y^n
  \begin{gather*}
    (5\,x\,y^2-2\,y)\,\odif{x}
    + (3\,x^2\,y-x)\,\odif{y}
    = 0
    ;\quad
    % ;\\
    \varphi(x,y)
    = x^m\,y^n
    %
    %
    %
    \implies \\
    \implies
    \left(x^m\,y^n\right)
    \left(5\,x\,y^2-2\,y\right)
    \,\odif{x}
    + 
    \left(x^m\,y^n\right)
    \left(3\,x^2\,y-x\right)
    \,\odif{y}
    = 0
    %
    %
    %
    \implies \\
    \implies
    \pdv{}{y}
    \left(x^m\,y^n\right)
    \left(5\,x\,y^2-2\,y\right)
    = (x^m\,y^{n-1}\,n)
    \left(5\,x\,y^2-2\,y\right)
    = \\
    =
    \pdv{}{x}
    \left(x^m\,y^n\right)
    \left(3\,x^2\,y-x\right)
    = 
    (m\,x^{m-1}\,y^n)
    \left(3\,x^2\,y-x\right)
  \end{gather*}

\end{questionBox}

\begin{questionBox}1{} % Q4
  A equação diferencial linear homogénea
  \begin{BM}
    x\,y'' + x^2\,y' + 4\,y = 0,\quad x>0
  \end{BM}
  Tem como solução geral a função \(y(x) = c_1\,y_1(x) + c_2\,y_2(x)\). Então a equação não homogénea
  \begin{BM}
    x\,y'' + x^2\,y' + 4\,y = x^3
  \end{BM}
  admite como solução geral a função \(y(x) = c_1(x)\,y_1(x) + c_2(x)\,y_2(x)\), onde as funções \(c_1(x),c_2(x)\) são determinadas a partir do sistema
  \begin{itemize}[label=\square]
    \begin{multicols}{2}
      \item[\blacksquare] \(\begin{cases}
        c_1'(x)\,y_1 + c_2'(x)y_2 = 0
        \\
        c_1'(x)\,y_1' + c_2'(x)y_2' = x^2
      \end{cases}\)
      \item \(\begin{cases}
        c_1'(x)\,y_1 + c_2'(x)y_2 = 0
        \\
        c_1'(x)\,y_1' + c_2'(x)y_2' = x
      \end{cases}\)
      \item \(\begin{cases}
        c_1'(x)\,y_1 + c_2'(x)y_2 = 0
        \\
        c_1'(x)\,y_1' + c_2'(x)y_2' = 1
      \end{cases}\)
      \item \(\begin{cases}
        c_1(x)\,y_1 + c_2(x)y_2 = 0
        \\
        c_1'(x)\,y_1' + c_2'(x)y_2' = x^2
      \end{cases}\)
      \item \(\begin{cases}
        c_1(x)\,y_1 + c_2(x)y_2 = 0
        \\
        c_1'(x)\,y_1' + c_2'(x)y_2' = x
      \end{cases}\)
      \item \(\begin{cases}
        c_1(x)\,y_1 + c_2(x)y_2 = 0
        \\
        c_1'(x)\,y_1' + c_2'(x)y_2' = 1
      \end{cases}\)
    \end{multicols}
  \end{itemize}
  \answer{}
  \begin{flalign*}
    &
      % a_0(x) = 4
      % a_1(x) = x^2
      % a_2(x) = x
      % f(x)   = x^3
      % y_1(x) = y_1(x)
      % y_2(x) = y_2(x)
      y:
      \begin{pmatrix}
          4
        \\ + x^2\,\mdif[1]{x}
        \\ + x\,\mdif[2]{x}
      \end{pmatrix}
      \,y
      = x^3
      %
      %
      %
      &\\[3ex]&
      \begin{Bmatrix}
        {
            c_1'(x)\,\mdif[0]{x}y_1(x) 
          + c_2'(x)\,\mdif[0]{x}y_2(x)
        } = 0
        \\ {
            c_1'(x)\,\mdif[1]{x}\,y_1(x) 
          + c_2'(x)\,\mdif[1]{x}\,y_2(x)
        } = \frac{x^3}{x} = x^2
      \end{Bmatrix}
    &
  \end{flalign*}
\end{questionBox}

\begin{questionBox}1{} % Q5
  Acerca de uma função \(f(x)\) definida e com derivadas até à segunda ordem em \(\mathbb{R}^+_0\) sabe-se que admite transformada de Laplace \(F(s)\), que \(f(0)=1,f'(0)=-2\). Então a trasnformada de Laplace da função
  \begin{BM}
    e^{-t}\,f''(t)+t\,f'(t)
  \end{BM}
  é:
  \begin{itemize}[label=\square]
    \begin{multicols}{2}
      \item \((s+1)^2\,F(s+1)-s+2+s\,F'(s)\)\phantom{\(-F(s)\)}
      \item \((s+1)^2\,F(s+1)-s+1-s\,F'(s)-F(s)\)
      \item \((s+1)^2\,F(s+1)-s+1+s\,F'(s)+F(s)\)
      \item \(s^2\,F(s)-s+1+s\,F'(s)-F(s)\)\phantom{\(+1+1\)}
      \item \(s^2\,F(s)-s+1+s\,F'(s+1)-F(s+1)\)
      \item \(s^2\,F(s)-s+1+s\,F'(s+1)+F(s+1)\)
    \end{multicols}
  \end{itemize}
  \answer{}
  \begin{BM}
    (s+1)^2\,F(s+1) - s + 1 - s\,F'(s) - F(s)
  \end{BM}
\end{questionBox}

\group{}

\begin{questionBox}*1{} % II Q
  Determine a solução geral da equação diferencial linear homogénea e de coeficientes constantes
  \begin{BM}
    \odv[2]{y}{x}
    + \odv{y}{x}
    -y\,6 = 0
  \end{BM}
  \answer{}
  \begin{flalign*}
    &
    % P = \mdif[2]{x} + \mdif[1]{x} -6
      % φ(x) = \varphi(x)
      P\,y
      = \left(
        \mdif[2]{x} + \mdif[1]{x} -6
      \right)
      \,y
      = 0
      %
      %
      %
      ; &\\[3ex]&
      y 
      = \varphi(x)\,\int{z(x)\,\odif{x}}
      = \varphi(x)\,\int{
        z(x)
        \,\odif{x}
      }
      %
      %
      %
      ; &\\[3ex]&
      P\,y
      = \left(
        \mdif[2]{x} + \mdif[1]{x} -6
      \right)
      \,\left(
        \varphi(x)\,\int{
          z(x)\,\odif{x}
        }
      \right)
      % = &\\&
      = 0
      %
      %
      %
      ; &\\[3ex]&
      \mdif[1]{x}{y}
      = \mdif[1]{x}{\left(
        \varphi(x)
        \,\int{z(x)\,\odif{x}}
      \right)}
      %
      %
      %
      ; &\\[3ex]&
      \mdif[2]{x}{y}
      = \mdif[2]{x}{\left(
        \varphi(x)
        \,\int{z(x)\,\odif{x}}
      \right)}
    &
  \end{flalign*}
  
\end{questionBox}

\begin{questionBox}*1{met ver const arb} % II Q2
  Utilizando o método da variação das constantes arbitrárias, determine a solução geral da equação não homogénea
  \begin{BM}
    \odv[2]{y}{x}
    + \odv{y}{x}
    -y\,6
    = -5\,e^{2\,x}\,\cos{x}
  \end{BM}

  \answer{}

  \begin{flalign*}
    &
      % a_0(x) = -6
      % a_1(x) = 1
      % a_2(x) = 1
      % f(x)   = \left(-5\,e^{2\,x}\,\cos(x)\right)
      % y_1(x) = y_1(x)
      % y_2(x) = y_2(x)
      y:
      \begin{pmatrix}
          -6
        \\ + 1\,\mdif[1]{x}
        \\ + 1\,\mdif[2]{x}
      \end{pmatrix}
      \,y
      = \left(-5\,e^{2\,x}\,\cos(x)\right)
      %
      %
      %
      &\\[3ex]&
      y
      = c_1(x)\,y_1(x)
      + c_2(x)\,y_2(x)
      %
      %
      %
      ; &\\[3ex]&
      % Regra de Crammer
      c_1(x) 
      = \int{c_1'(x)\,\odif{x}}
      %
      %
      ; &\\[1ex]&
      c_2(x) 
      = \int{c_2'(x)\,\odif{x}}
      %
      %
      %
      &\\[3ex]&
      c_1'(x)
      = \frac{1}{W(y_1(x),y_2(x))}
      \,\begin{vmatrix}
        0 
        &  \mdif[0]{x}y_2(x)
        \\ -5\,e^{2\,x}\,\cos(x)
        &  \mdif[1]{x}y_2(x)
      \end{vmatrix}
      %
      %
      %
      &\\[3ex]&
      c_2'(x)
      = \frac{1}{W(y_1(x),y_2(x))}
      \,\begin{vmatrix}
           \mdif[0]{x}y_1(x)
        &  0 
        \\ \mdif[1]{x}y_1(x)
        &  -5\,e^{2\,x}\,\cos(x)
      \end{vmatrix}
      %
      %
      %
      &\\[3ex]&
      % Wronskiano
      W(y_1(x),y_2(x))
      = \det\begin{bmatrix}
           \mdif[0]{x}\,y_1(x)
        &  \mdif[0]{x}\,y_2(x)
        \\ \mdif[1]{x}\,y_1(x)
        &  \mdif[1]{x}\,y_2(x)
      \end{bmatrix}
      %
      %
      %
      &\\[3ex]&
      \begin{Bmatrix}
        {
            c_1'(x)\,\mdif[0]{x}y_1(x) 
          + c_2'(x)\,\mdif[0]{x}y_2(x)
        } &=& 0
        \\ {
            c_1'(x)\,\mdif[1]{x}\,y_1(x) 
          + c_2'(x)\,\mdif[1]{x}\,y_2(x)
        } &=& -5\,e^{2\,x}\,\cos(x)
      \end{Bmatrix}
      %
      %
      %
      ; &\\[3ex]&
      \mdif[1]{x}\,y_1(x)
      = \mdif[1]{x}\,y_1(x)
      %
      %
      ; &\\[1ex]&
      \mdif[1]{x}\,y_2(x)
      = \mdif[1]{x}\,y_2(x)
    &
  \end{flalign*}
\end{questionBox}

\group{}

\setcounter{question}{1}

\begin{questionBox}2{} % 3 Q1.1
  \label{qbox:3.1.1}
  Determine todas as soluções da equação de Clairaut
  \begin{BM}
    y = x\,\odv{y}{x} - \left(\odv{y}{x}\right)^3
  \end{BM}
\end{questionBox}

\begin{questionBox}2{} % 3 Q1.2
  Utilizando a mudança de variável definida por \(x=1/t\), resolva a equação
  \begin{BM}
    y = -x\,\odv{y}{x} + x^6\,\left(\odv{y}{x}\right)^3
    ,\quad x>0
  \end{BM}
  \paragraph*{Sug:} Após a mudança de variável utilize (\ref{qbox:3.1.1}).
\end{questionBox}

\group{}

\begin{questionBox}*1{} % 4 Q1
  Utilize a transformada de Laplace para resolver o problema de valores iniciais
  \begin{BM}
    y''+y'+y\,5/2 = \fdif{( t-2 )}
    ,\quad y(0)=0
    ,\quad y'(0)=1
  \end{BM}
  \paragraph*{Sug:} tenha em contra que \(s^2+s+5/2=(s+1/2)^2+9/4\).
\end{questionBox}

\group{}

\begin{questionBox}*1{} % 5 Q1
  Considere a equação diferencial lienar de ordem \(n\) e coeficientes constantes
  \begin{BM}
    \left(
      \mdif[n]{x}
      + \sum_{k=0}^{n-1}{
        \alpha_{k}\,\mdif[k]{x}
      }
    \right)\,y
    % \left(
    %   \mdif[n]{x}
    %   + \alpha_{n-1}\,\mdif[n-1]{x}
    %   + \alpha_{n-2}\,\mdif[n-2]{x}
    %   + \dots
    %   + \alpha_{1}\,\mdif[1]{x}
    % \right)y
    = e^{\alpha\,x}
    , \alpha\in\mathbb{R}
    \yesnumber\label{eq:5.1-enunciado}
  \end{BM}
  Seja \(P(r)=r_n+\sum_{k=0}^{n-1}{a_k\,r^{k}}\). admitimos que \(r=\alpha\) é raiz da equação \(P(r)=0\) com grau de multiplicidade um. Justifique \(P'(\alpha)=0\).
\end{questionBox}

\begin{questionBox}*1{} % 5 Q2
  A equação \eqref{eq:5.1-enunciado} tem uma solução particular da forma
  \begin{BM}
    \bar{y}=\frac{c}{2\,P'(\alpha)}\,x\,e^{\alpha\,x}
    ,\quad c\in\mathbb{R}
  \end{BM}
  Sabendo que
  \begin{BM}
    \mdif[k]{x}\,(x\,e^{\alpha\,x})
    = k\,\alpha^{k-1}\,e^{\alpha\,x}
    + \alpha^{k}\,x\,e^{\alpha\,x}
    ,\quad \forall\,k\in\mathbb{N}
    ,\quad \mdif[k]{x}=\odv[k]{}{x}
  \end{BM}
  Determine, justificando detalhadamente, o valor de \emph{c}.
\end{questionBox}

\end{document}

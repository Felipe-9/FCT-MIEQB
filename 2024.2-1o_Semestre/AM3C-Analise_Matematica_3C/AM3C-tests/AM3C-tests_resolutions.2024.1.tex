% !TEX root = ./AM3C-tests_resolutions.2024.1.tex
\documentclass["AM3C-tests_resolutions.tex"]{subfiles}

% \tikzset{external/force remake=true} % - remake all

\begin{document}

% \graphicspath{{\subfix{./.build/figures/AM3C-tests_resolutions.2024.1}}}
% \tikzsetexternalprefix{./.build/figures/AM3C-tests_resolutions.2024.1/graphics/}

\mymakesubfile{1}[AM3C]
{Teste 2024.1 Resolucão} % Subfile Title
{Teste 2024.1 Resolução} % Part Title

\group{}

\begin{questionBox}1m{} % Q1
  A equação diferencial linear de primeira ordem
  \begin{BM}
    \odv{y}{x} + \frac{\cos(x)}{\sin(x)}\,y
    = -x
    ;\quad x \in \myrange*{0,\pi}
  \end{BM}
  Tem como solução geral
  \begin{itemize}[label=\Box]
    \begin{multicols}{3}
      \item[\blacksquare] \(y = \frac{c}{\sin x} - x\,\frac{\cos x}{\sin x} +1\)
      \item \(y = \frac{c}{\cos x} - x\,\frac{\cos x}{\sin x} +1\)
      \item \(y = \frac{c}{\sin x} + x\,\frac{\sin x}{\cos x} -1\)
      \item \(y = \frac{c}{\cos x} + x\,\frac{\cos x}{\sin x} -1\)
      \item \(y = \frac{c}{\sin x} - x\,\frac{\cos x}{\sin x}\)
      \item \(y = \frac{c}{\cos x} + x\,\frac{\sin x}{\cos x}\)
    \end{multicols}
  \end{itemize}

  % tag  = 1
  % a(x) = \frac{\cos{x}}{\sin{x}}
  % b(x) = -x
  % y    = y
  % x    = x

  \answer{\eqref{eq:q.1.1 answer}}

  General solution
  \begin{tcolorbox}
    \begin{gather*}
      y
      = \frac{c_0}{\varphi(x)}
      + \frac{1  }{\varphi(x)}
      \,\prim_x{\left(
          -x\,\varphi(x)
      \right)}
      %
      %
      %
      = \mathText{using 
        \eqref{eq:q.1.1 phi_x}
        \eqref{eq:q.1.1 prim}
      }
      = \frac{c_0}{ c_1\,\sin{x} }
      + \frac{1  }{ c_1\,\sin{x} }
      \,\left(
        -c_1\,(
          x\,\cos{x} - (c_2+\sin{x})
        )
      \right)
      = \\
      = \frac{c_0}{c_1\,\sin{x}}
      + \frac{c_2}{\sin{x}}
      - x\,\frac{\cos{x}}{\sin{x}}
      + 1
      = \frac{c_4}{\sin{x}}
      - x\,\frac{\cos{x}}{\sin{x}}
      + 1
      %
      \yesnumber\label{eq:q.1.1 answer}
    \end{gather*}
  \end{tcolorbox}

  Finding \(\varphi(x)\)
  \begin{tcolorbox}
    \begin{gather*}
      \varphi(x) 
      = \exp{\left(
          \prim_x{\left(
              \frac{\cos{x}}{\sin{x}}\,\odif{x}
          \right)}
      \right)}
      = \mathText{\(\odif{(\sin{x})} = \cos{x}\odif{x}\)}
      = \exp{\left(
          \int{\left(
              \frac{\odif{(\sin{x})}}{\sin{x}}
          \right)}
      \right)}
      = \exp{\left(
          c_0 + \ln{(\sin{x})}
      \right)}
      = c_1\,\sin{x} 
      %
      \yesnumber\label{eq:q.1.1 phi_x}
    \end{gather*}
  \end{tcolorbox}

  Integrating
  \begin{tcolorbox}
    \begin{gather*}
      \prim_x{\left(
          -x\,\varphi(x)
      \right)}
      = \mathText{using \eqref{eq:q.1.1 phi_x}}
      = \prim_x{\left(
          -x\,c_1\,\sin{x} 
      \right)}
      = \mathText{\(
          \prim_x{(u\,v')}=u\,v-\prim_x{(u'\,v)}
          \begin{cases}
            u=x
            \\
            v=\cos{x}
          \end{cases}
      \)}
      = x\,\cos{x}
      - \prim_x{(
          \cos{x}
      )}
      = -c_1\,(
        x\,\cos{x} - (c_2+\sin{x})
      )
      %
      \yesnumber\label{eq:q.1.1 prim}
    \end{gather*}
  \end{tcolorbox}
\end{questionBox}

\begin{questionBox}1m{} % Q2

  A solução da equação de Bernoulli
  \begin{BM}
    \odv{y}{x} + y = \frac{1}{y}
  \end{BM}
  que satisfaz a condição \(y(0)=2\), é:

  \begin{itemize}[label=\square]
    \begin{multicols}{3}
      \item \(y=\sqrt{e^{+2\,x}+3}\)
      \item \(y=\sqrt{e^{-2\,x}+3}\)
      \item \(y=\sqrt{3\,e^{+2\,x}+1}\)
      \item[\blacksquare]\(y=\sqrt{3\,e^{-2\,x}+1}\)
      \item \(y=\sqrt{2\,e^{+2\,x}+2}\)
      \item \(y=\sqrt{2\,e^{-2\,x}+2}\)
      \end{multicols}
  \end{itemize}

  % tag  = q.1.2
  % a(x) = 1
  % b(x) = 1
  % k    = {-1}

  % y' + y(
  %   1
  % )
  % = y^{-1} (
  %   1
  % )

  \answer{\eqref{eq:q.1.2 answer}}

  General solution
  \begin{tcolorbox}
    \begin{gather*}
      y 
      = z^{1/2}
      %
      \yesnumber\label{eq:q.1.2 y->z}
      %
      = \\
      = \left(
        \frac{c_0}{\varphi(x)}
        + \frac{1}{\varphi(x)}
        \,\prim_x{(
            2*1\,\varphi(x)
        )}
      \right)^{1/2}
      = \mathText{using
        \eqref{eq:q.1.2 phi_x}
        \eqref{eq:q.1.2 prim}
      }
      = \left(
        \frac{c_0}{c_1\,e^{2\,x}}
        + \frac{1}{c_1\,e^{2\,x}}
        \,c_1\,e^{2\,x}
      \right)^{1/2}
      = \left(
        \frac{c_0}{c_1\,e^{2\,x}}
        + 1
      \right)^{1/2}
      = \left( c_2\,e^{-2\,x} + 1 \right)^{1/2}
      %
      \yesnumber\label{eq:q.1.2 y const}
      %
      = \mathText{using \eqref{eq:q.1.2 c2}}
      = \left( 3\,e^{-2\,x} + 1 \right)^{1/2}
      %
      \yesnumber\label{eq:q.1.2 answer}
    \end{gather*}
  \end{tcolorbox}

  Finding \(c_2\)
  \begin{tcolorbox}
    \begin{gather*}
      y(0) = 2
      = \mathText{using \eqref{eq:q.1.2 y const}}
      = \left( c_2\,e^{-2*0} + 1 \right)^{1/2}
      \implies
      c_2 = 4 - 1 = 3
      %
      \yesnumber\label{eq:q.1.2 c2}
    \end{gather*}
  \end{tcolorbox}

  Bernoulli's substitution
  \begin{tcolorbox}
    \begin{gather*}
      y' + y = y^{-1}
      \implies \mathText{using \eqref{eq:q.1.2 y->z}}
      \implies
      z' + 2\,z = 2
    \end{gather*}
  \end{tcolorbox}

  Finding \(\varphi(x)\)
  \begin{tcolorbox}
    \begin{gather*}
      \varphi(x)
      = \exp{\left(
          \prim_x{(2)}
      \right)}
      = 
      c_1\,e^{2\,x}
      %
      \yesnumber\label{eq:q.1.2 phi_x}
    \end{gather*}
  \end{tcolorbox}

  Integrating
  \begin{tcolorbox}
    \begin{gather*}
      \prim_x{\left(
          2*1\,\varphi(x)
      \right)}
      = \mathText{using \eqref{eq:q.1.2 phi_x}}
      = 2\,c_1\,e^{2\,x}/2 
      = c_1\,e^{2\,x}
      %
      \yesnumber\label{eq:q.1.2 prim}
    \end{gather*}
  \end{tcolorbox}
\end{questionBox}

\begin{questionBox}1m{} % Q3
  A equação differencial
  \begin{BM}
    (5\,x\,y^2 - 2\,y)\,\odif{x} + (3\,x^2\,y-x)\,\odif{y}=0
  \end{BM}
  Admite um fator integrante na forma \(\phi(x,y)=x^m\,y^n\), com \(m,n \in \mathbb{N}\). Então:
  \begin{itemize}[label=\square]
    \begin{multicols}{3}
      \item \(m=3,n=2\)
      \item \(m=1,n=1\)
      \item \(m=2,n=2\)
      \item \(m=2,n=1\)
      \item \(m=1,n=3\)
      \item[\blacksquare] \(m=3,n=1\)
    \end{multicols}
  \end{itemize}

  % tag = q.1.3
  % u = 5\,x\,y^2-2\,y
  % v = 3\,x^2\,y-x

  \answer{\eqref{eq:q.1.3 answer}}

  Finding \(m,n\)
  \begin{tcolorbox}
    \begin{gather*}
      \pdv{u\,\varphi(x)}{y}
      = \pdv{}{y}{\left(
          x^m\,y^n
          \,\left(
            5\,x\,y^2-2\,y
          \right)
      \right)}
      = \pdv{}{y}{\left(
          5\,x^{m+1}\,y^{n+2}-2\,y^{n+1}\,x^m
      \right)}
      = \\
      = 5\,x^{m+1}\,(n+2)\,y^{n+1}-2\,(n+1)\,y^{n}\,x^m
      = \\
      = \pdv{v\,\varphi(x)}{x}
      = \pdv{}{x}{\left(
          x^m\,y^n\,\left(
            3\,x^2\,y-x
          \right)
      \right)}
      = \pdv{}{x}{\left(
          3\,x^{2+m}\,y^{n+1}-x^{m+1}\,y^n
      \right)}
      = \\
      = 3\,(2+m)\,x^{1+m}\,y^{n+1}
      - (m+1)\,x^{m}\,y^n
      \implies \\
      \implies
      \begin{cases}
        2(n+1)=m+1
        \implies
        m = 2\,n+1 = 3
        \\
        5\,(n+2)=3(m+2)
        \implies
        n = 1
      \end{cases}
      %
      \yesnumber\label{eq:q.1.3 answer}
    \end{gather*}
  \end{tcolorbox}
\end{questionBox}

\begin{questionBox}1m{} % Q4
  A equação diferencial linear homogénea
  \begin{BM}
    x\,y'' + x^2\,y' + 4\,y = 0,\quad x>0
  \end{BM}
  Tem como solução geral a função \(y(x) = c_1\,y_1(x) + c_2\,y_2(x)\). Então a equação não homogénea
  \begin{BM}
    x\,y'' + x^2\,y' + 4\,y = x^3
  \end{BM}
  admite como solução geral a função \(y(x) = c_1(x)\,y_1(x) + c_2(x)\,y_2(x)\), onde as funções \(c_1(x),c_2(x)\) são determinadas a partir do sistema
  \begin{itemize}[label=\square]
    \begin{multicols}{2}
      \item[\blacksquare] \(\begin{cases}
        c_1'(x)\,y_1 + c_2'(x)y_2 = 0
        \\
        c_1'(x)\,y_1' + c_2'(x)y_2' = x^2
      \end{cases}\)
      \item \(\begin{cases}
        c_1'(x)\,y_1 + c_2'(x)y_2 = 0
        \\
        c_1'(x)\,y_1' + c_2'(x)y_2' = x
      \end{cases}\)
      \item \(\begin{cases}
        c_1'(x)\,y_1 + c_2'(x)y_2 = 0
        \\
        c_1'(x)\,y_1' + c_2'(x)y_2' = 1
      \end{cases}\)
      \item \(\begin{cases}
        c_1(x)\,y_1 + c_2(x)y_2 = 0
        \\
        c_1'(x)\,y_1' + c_2'(x)y_2' = x^2
      \end{cases}\)
      \item \(\begin{cases}
        c_1(x)\,y_1 + c_2(x)y_2 = 0
        \\
        c_1'(x)\,y_1' + c_2'(x)y_2' = x
      \end{cases}\)
      \item \(\begin{cases}
        c_1(x)\,y_1 + c_2(x)y_2 = 0
        \\
        c_1'(x)\,y_1' + c_2'(x)y_2' = 1
      \end{cases}\)
    \end{multicols}
  \end{itemize}

  % tag    = q.1.4
  % a_0(x) = 4
  % a_1(x) = x^2
  % a_2(x) = x
  % f(x)   = x^3

  %  y:
  %  \begin{pmatrix}
  %         4
  %    \\ + x^2\,\mdif[1]{x}
  %    \\ + x\,\mdif[2]{x}
  %  \end{pmatrix}
  %  \,y
  %  = x^3

  \answer{\eqref{eq:q.1.4 eq_sytem}}

  Crammers equation system
  \begin{tcolorbox}
    \begin{gather*}
      \begin{Bmatrix}
        {
            c_1'(x)\,\mdif[0]{x}\,y_1(x) 
          + c_2'(x)\,\mdif[0]{x}\,y_2(x)
        } &=& 0
        \\ {
            c_1'(x)\,\mdif[1]{x}\,y_1(x) 
          + c_2'(x)\,\mdif[1]{x}\,y_2(x)
        } &=& \frac{x^3}{x} = x^2
      \end{Bmatrix}
      %
      \yesnumber\label{eq:q.1.4 eq_sytem}
    \end{gather*}
  \end{tcolorbox}

\end{questionBox}

\begin{questionBox}1m{} % Q5
  Acerca de uma função \(f(x)\) definida e com derivadas até à segunda ordem em \(\mathbb{R}^+_0\) sabe-se que admite transformada de Laplace \(F(s)\), que \(f(0)=1,f'(0)=-2\). Então a trasnformada de Laplace da função
  \begin{BM}
    e^{-t}\,f''(t)+t\,f'(t)
  \end{BM}
  é:
  \begin{itemize}[label=\square]
    \begin{multicols}{2}
      \item \((s+1)^2\,F(s+1)-s+2+s\,F'(s)\)\phantom{\(-F(s)\)}
      \item \((s+1)^2\,F(s+1)-s+1-s\,F'(s)-F(s)\)
      \item \((s+1)^2\,F(s+1)-s+1+s\,F'(s)+F(s)\)
      \item \(s^2\,F(s)-s+1+s\,F'(s)-F(s)\)\phantom{\(+1+1\)}
      \item \(s^2\,F(s)-s+1+s\,F'(s+1)-F(s+1)\)
      \item \(s^2\,F(s)-s+1+s\,F'(s+1)+F(s+1)\)
    \end{multicols}
  \end{itemize}
  \answer{}
  \begin{BM}
    (s+1)^2\,F(s+1) - s + 1 - s\,F'(s) - F(s)
  \end{BM}
\end{questionBox}

\group{} % G2

\begin{questionBox}*1m{} % II Q
  Determine a solução geral da equação diferencial linear homogénea e de coeficientes constantes
  \begin{BM}
    \odv[2]{y}{x}
    + \odv{y}{x}
    -y\,6 = 0
  \end{BM}

  \answer{}

  % tag   = q.2.1
  % P     = \mdif[2]{x} + \mdif{x} - 6

  \answer{\eqref{eq:q.2.1 answer}}

  General solution for \(y\)
  \begin{tcolorbox}
    \begin{gather*}
      y
      = \mathText{using 
        \eqref{eq:q.2.1 mapped roots for y}
      }
      = e^{+2\,x}\,c_0
      + e^{-3\,x}\,c_1
      %
      \yesnumber\label{eq:q.2.1 answer}
    \end{gather*}
  \end{tcolorbox}

  Mapping roots of \eqref{eq:q.2.1 roots of y} to solution
  \begin{tcolorbox}
    \begin{gather*}
      \begin{cases}
        r_0 = +2
        \implies
        e^{+2\,x}\,c_0
        ;\\
        r_1 = -3
        \implies
        e^{-3\,x}\,c_1
      \end{cases}
      %
      \yesnumber\label{eq:q.2.1 mapped roots for y}
    \end{gather*}
  \end{tcolorbox}

  Roots for characteristic equation for \(y\)
  \begin{tcolorbox}
    \begin{gather*}
      P
      = \mdif[2]{x} + \mdif{x} - 6
      \implies \mathText{\(\mdif[i]{x} \to r^i\)}
      \implies
      r^2 + r - 6 = 0
      \implies
      % a = 1
      % b = 1
      % c = -6
      % 1 x^2 + 1 x + -6 = 0
      r
      = \frac{
        - 1 \pm \sqrt{ 1 -4*-6 }
      } {2}
      = \frac{ - 1 \pm 5 } {2}
      \implies \\
      \implies
      \begin{cases}
        r_0 = +2
        \\
        r_1 = -3
      \end{cases}
      %
      \yesnumber\label{eq:q.2.1 roots of y}
    \end{gather*}
  \end{tcolorbox}
\end{questionBox}

\begin{questionBox}*1m{} % II Q2

  Utilizando o método da variação das constantes arbitrárias, determine a solução geral da equação não homogénea
  \begin{BM}
    \odv[2]{y}{x}
    + \odv{y}{x}
    -y\,6
    = -5\,e^{2\,x}\,\cos{x}
  \end{BM}

  % tag    = q.2.2
  % a_0(x) = -6
  % a_1(x) = 1
  % a_2(x) = 1
  % f(x)   = -5\,e^{2\,x}\,\cos{x}
  % y_1(x) = y_1(x)
  % y_2(x) = y_2(x)

  %  y:
  %  \begin{pmatrix}
  %         -6
  %    \\ + 1\,\mdif[1]{x}
  %    \\ + 1\,\mdif[2]{x}
  %  \end{pmatrix}
  %  \,y
  %  = -5\,e^{2\,x}\,\cos{x}

  \answer{\eqref{eq:q.2.2 answer}}

  General solution
  \begin{tcolorbox}
    \begin{gather*}
      y
      = \mathText{using \eqref{eq:q.2.1 answer}}
      = c_1(x)\,e^{+2\,x}
      + c_2(x)\,e^{-3\,x}
      = \mathText{using 
        \eqref{eq:q.2.2 c_1}
        \eqref{eq:q.2.2 c_2}
      }
      = \left( c_3+\sin{x} \right)\,e^{+2\,x}
      + \left(
        \frac{e^{5\,x}}{26}
        \,(5\,\cos{x}+\sin{x})
      \right)\,e^{-3\,x}
      = \\
      = e^{2\,x}
      \,\left(
        \frac{5}{26}\,\cos{x}
        + \frac{27}{26}\sin{x}
        + c_3
      \right)
      %
      \yesnumber\label{eq:q.2.2 answer}
    \end{gather*}
  \end{tcolorbox}

  \(y_1,y_2\) comes from \eqref{eq:q.2.1 answer} which is the homogeneous analogus equation for \(y\)
  \begin{tcolorbox}
    \begin{gather*}
      \begin{cases}
        y_1 = e^{+2\,x}
        \\
        y_2 = e^{-3\,x}
      \end{cases}
      %
      \yesnumber\label{eq:q.2.2 y1 y2}
    \end{gather*}
  \end{tcolorbox}

  Finding \(c_1,c_2\)
  \begin{tcolorbox}
    \begin{gather*}
      c_1(x) 
      = \prim_x{(c_1'(x))}
      = \mathText{Using \eqref{eq:q.2.2 c_1'}}
      = \prim_x{(
          -\cos{x}
      )}
      = c_3+\sin{x}
      %
      \yesnumber\label{eq:q.2.2 c_1}
      %
      %
      ; \\[1ex]
      c_2(x) 
      = \prim_x{(
          c_2'(x)
      )}
      = \mathText{Using \eqref{eq:q.2.2 c_2'}}
      = \prim_x{(
        e^{5\,x}\,\cos{x}
      )}
      = \mathText{\(
          \prim_x{(u\,v')}
          = u\,v
          - \prim_x{(u'\,v)}
          \begin{cases}
            u = e^{5\,x}
            \\
            v = \sin{x}
          \end{cases}
      \)}
      = e^{5\,x}\,\sin{x}
      - \prim_x{(
        5\,e^{x}\,\sin{x}
      )}
      = e^{5\,x}\,\sin{x}
      + 5\,\prim_x{(
        e^{5\,x}\,(-\sin{x})
      )}
      = \mathText{\(
          \prim_x{(u\,v')}
          = u\,v
          - \prim_x{(u'\,v)}
          \begin{cases}
            u = e^{5\,x}
            \\
            v = \cos{x}
          \end{cases}
      \)}
      = e^{5\,x}\,\sin{x}
      + 5\,(
        e^{5\,x}\,\cos{x}
        -\prim_x{(
            5\,e^{5\,x}\,\cos{x}
        )}
      )
      = e^{5\,x}\,\sin{x}
      + 5\,e^{5\,x}\,\cos{x}
      - 25\,\prim_x{(
          e^{5\,x}\,\cos{x}
      )}
      \implies \\
      \implies
      c_1 = \prim_x{(e^{5\,x}\,\cos{x})}
      = \frac{e^{5\,x}}{26}
      \,(5\,\cos{x}+\sin{x})
      %
      \yesnumber\label{eq:q.2.2 c_2}
    \end{gather*}
  \end{tcolorbox}

  solving \(\mdif[1]{x}{(c_1,c_2)}\)
  \begin{tcolorbox}
    \begin{gather*}
      \mathText{Using \eqref{eq:q.2.2 eq_sytem}}
      c_1'(x)
      = \frac{1}{W(y_1,y_2)}
      \,\begin{vmatrix}
        0 
        &  \mdif[0]{x}y_2
        \\ \frac{-5\,e^{2\,x}\,\cos{x}}{1}
        &  \mdif[1]{x}y_2
      \end{vmatrix}
      = \mathText{using 
        \eqref{eq:q.2.2 w}
        \eqref{eq:q.2.2 D_x(y_2)}
      }
      = \frac{1}{- 5\,e^{-x}}
      \,\begin{vmatrix}
        0 
        &  e^{-3\,x}
        \\ -5\,e^{2\,x}\,\cos{x}
        &  -3\,e^{-3\,x}
      \end{vmatrix}
      = \frac{
        5\,e^{-x}\,\cos{x}
      }{- 5\,e^{-x}}
      = -\cos{x}
      %
      \yesnumber\label{eq:q.2.2 c_1'}
      %
      %
      %
      ; \\[2ex]
      \mathText{Using \eqref{eq:q.2.2 eq_sytem}}
      c_2'(x)
      = \frac{1}{W(y_1,y_2)}
      \,\begin{vmatrix}
           \mdif[0]{x}y_1
        &  0 
        \\ \mdif[1]{x}y_1
        &  \frac{-5\,e^{2\,x}\,\cos{x}}{1}
      \end{vmatrix}
      = \mathText{using 
        \eqref{eq:q.2.2 w}
        \eqref{eq:q.2.2 D_x(y_1)}
      }
      = \frac{1}{- 5\,e^{-x}}
      \,\begin{vmatrix}
            e^{+2\,x}
        &   0 
        \\  2\,e^{+2\,x}
        &   -5\,e^{2\,x}\,\cos{x}
      \end{vmatrix}
      = \frac
      {-5\,e^{4\,x}\,\cos{x}}
      {-5\,e^{-x}}
      = e^{5\,x}\,\cos{x}
      %
      \yesnumber\label{eq:q.2.2 c_2'}
    \end{gather*}
  \end{tcolorbox}

  Solving Wronskiano
  \begin{tcolorbox}
    \begin{gather*}
      W(y_1,y_2)
      = \det\begin{bmatrix}
           y_1
        &  y_2
        \\ \mdif[1]{x}\,y_1
        &  \mdif[1]{x}\,y_2
      \end{bmatrix}
      = \mathText{using 
        \eqref{eq:q.2.2 D_x(y_1)}
        \eqref{eq:q.2.2 D_x(y_2)}
      }
      = \det\begin{bmatrix}
           +e^{+2\,x}
        &  +e^{-3\,x}
        \\ +2\,e^{+2\,x}
        &  -3\,e^{-3\,x}
      \end{bmatrix}
      = - 3\,e^{-x}
      - 2\,e^{-x}
      = - 5\,e^{-x}
      %
      \yesnumber\label{eq:q.2.2 w}
    \end{gather*}
  \end{tcolorbox}

  Crammers equation system
  \begin{tcolorbox}
    \begin{gather*}
      \begin{Bmatrix}
        {
            c_1'(x)\,\mdif[0]{x}\,y_1(x) 
          + c_2'(x)\,\mdif[0]{x}\,y_2(x)
        } &=& 0
        \\ {
            c_1'(x)\,\mdif[1]{x}\,y_1(x) 
          + c_2'(x)\,\mdif[1]{x}\,y_2(x)
        } &=& \frac{-5\,e^{2\,x}\,\cos{x}}{1}
      \end{Bmatrix}
      %
      \yesnumber\label{eq:q.2.2 eq_sytem}
    \end{gather*}
  \end{tcolorbox}

  Solving \(\mdif[1]{x}{(y_1,y_2)}\)
  \begin{tcolorbox}
    \begin{gather*}
      \mdif[1]{x}\,y_1
      = \mdif[1]{x}{\left(
          e^{+2\,x}
      \right)}
      = + 2\,e^{+2\,x}
      %
      \yesnumber\label{eq:q.2.2 D_x(y_1)}
      %
      %
      ; \\[1ex]
      \mdif[1]{x}\,y_2
      = \mdif[1]{x}{\left(
          e^{-3\,x}
      \right)}
      = -3\,e^{-3\,x}
      %
      \yesnumber\label{eq:q.2.2 D_x(y_2)}
    \end{gather*}
  \end{tcolorbox}

\end{questionBox}

\group{} % G3

\setcounter{question}{1}

\begin{questionBox}2m{} % 3 Q1.1
  \label{qbox:3.1.1}
  Determine todas as soluções da equação de Clairaut
  \begin{BM}
    y = x\,\odv{y}{x} - \left(\odv{y}{x}\right)^3
  \end{BM}

  \answer{\eqref{eq:q.3.1 answer}}

  General solution
  \begin{tcolorbox}
    \begin{gather*}
      y = \sum_i{y_i}
      = \mathText{using
        \eqref{eq:q.3.1 y0}
        \eqref{eq:q.3.1 y1 y2}
      }
      = \begin{cases}
        x\,c-c^3 
        &\quad \text{General solution}
        \\
        \pm 2\,(x/3)^{3/2} 
        &\quad \text{Singular solutions}
      \end{cases}
      %
      \yesnumber\label{eq:q.3.1 answer}
    \end{gather*}
  \end{tcolorbox}

  Finding \(y_i\)
  \begin{tcolorbox}
    \begin{gather*}
      y_0
      = \mathText{using 
        \eqref{eq:q.3.1 y to p}
        \eqref{eq:q.3.1 p_i}
        \(p = c\)
      }
      = x\,c - c^3
      % 
      \yesnumber\label{eq:q.3.1 y0}
      % 
      % 
      ; \\
      y_1
      = \mathText{using 
        \eqref{eq:q.3.1 y to p}
        \eqref{eq:q.3.1 p_i}
        \(p = \pm\sqrt{x/3}\)
      }
      = x\,(\pm\sqrt{x/3}) - (\pm\sqrt{x/3})^3
      = \pm x\,(x/3)^{1/2}
      \pm (-(x/3)^{3/2})
      = \\
      = \pm (x/3)^{1/2}(
        x - x/3
      )
      = \pm x\,(x/3)^{1/2}(3/3 - 1/3)
      = \pm 2\,(x/3)^{3/2}
      % 
      \yesnumber\label{eq:q.3.1 y1 y2}
      % 
    \end{gather*}
  \end{tcolorbox}

  Finding \(p\)
  \begin{tcolorbox}
    \begin{gather*}
      y'
      = \mdif{x}{y}
      = p
      = \mathText{using \eqref{eq:q.3.1 y to p}}
      = \mdif{x}{\left(
        x\,p - p^3
      \right)}
      = p + x\,\mdif{x}{p} - 3\,p^2\,\mdif{x}{p}
      \implies
      (x - 3\,p^2)\,\mdif{x}{p} = 0
      \implies \\
      \implies
      \begin{cases}
        \mdif{x}{p} = 0 \implies p = c
        \\
        p = \pm\sqrt{x/3}
      \end{cases}
      % 
      \yesnumber\label{eq:q.3.1 p_i}
    \end{gather*}
  \end{tcolorbox}

  Clairut's substitution
  \begin{tcolorbox}
    \begin{gather*}
      y 
      = x\,\odv{y}{x} - \left(\odv{y}{x}\right)^3
      = \mathText{\(\mdif[i]{x}{y} = \mdif[i-1]{x}{p}\)}
      = x\,p - p^3
      %
      \yesnumber\label{eq:q.3.1 y to p}
    \end{gather*}
  \end{tcolorbox}
\end{questionBox}

\begin{questionBox}2m{} % 3 Q1.2
  Utilizando a mudança de variável definida por \(x=1/t\), resolva a equação
  \begin{BM}
    y = -x\,\odv{y}{x} + x^6\,\left(\odv{y}{x}\right)^3
    ,\quad x>0
  \end{BM}
  \paragraph*{Sug:} Após a mudança de variável utilize (\ref{qbox:3.1.1}).

  \answer{\eqref{eq:q.3.2 answer}}

  General solution
  \begin{tcolorbox}
    \begin{gather*}
      y 
      = \mathText{using \eqref{eq:q.3.2 y to t}}
      = \odv{y}{t}\,t
      - \left(\odv{y}{t}\right)^3
      = \mathText{using \eqref{eq:q.3.1 answer}}
      = \begin{cases}
        t\,c-c^3
        \\
        \pm 2(t/3)^{3/2}
      \end{cases}
      %
      \yesnumber\label{eq:q.3.2 answer}
    \end{gather*}
  \end{tcolorbox}

  Variable change \(x=1/t\)
  \begin{tcolorbox}
    \begin{gather*}
      y 
      = -x\,\odv{y}{x} + x^6\,\left(\odv{y}{x}\right)^3
      = \mathText{\(x=1/t\)}
      = -(1/t)\,\odv{y}{t}\,\odv{t}{x}
      + (1/t)^6\,\left(
        \odv{y}{t}\,\odv{t}{x}
      \right)^3
      = \mathText{\(\odv{t}{x} = -1/x^2 = -1/(1/t)^2 = -t^2\)}
      = -(1/t)\,\odv{y}{t}\,(-t^2)
      + (1/t)^6\,\left(
        \odv{y}{t}(-t^2)
      \right)^3
      = \odv{y}{t}\,t
      - \left(\odv{y}{t}\right)^3
      % 
      \yesnumber\label{eq:q.3.2 y to t}
    \end{gather*}
  \end{tcolorbox}
\end{questionBox}

\group{} % G4

\begin{questionBox}*1m{} % 4 Q1
  Utilize a transformada de Laplace para resolver o problema de valores iniciais
  \begin{BM}
    y''+y'+y\,5/2 = \fdif{( t-2 )}
    ,\quad y(0)=0
    ,\quad y'(0)=1
  \end{BM}
  \paragraph*{Sug:} tenha em contra que \(s^2+s+5/2=(s+1/2)^2+9/4\).
\end{questionBox}

\group{} % G5

\begin{questionBox}*1m{} % 5 Q1
  Considere a equação diferencial lienar de ordem \(n\) e coeficientes constantes
  \begin{BM}
    \left(
      \mdif[n]{x}
      + \sum_{k=0}^{n-1}{
        \alpha_{k}\,\mdif[k]{x}
      }
    \right)\,y
    % \left(
    %   \mdif[n]{x}
    %   + \alpha_{n-1}\,\mdif[n-1]{x}
    %   + \alpha_{n-2}\,\mdif[n-2]{x}
    %   + \dots
    %   + \alpha_{1}\,\mdif[1]{x}
    % \right)y
    = e^{\alpha\,x}
    , \alpha\in\mathbb{R}
    \yesnumber\label{eq:5.1-enunciado}
  \end{BM}
  Seja \(P(r)=r^n+\sum_{k=0}^{n-1}{a_k\,r^{k}}\). admitimos que \(r=\alpha\) é raiz da equação \(P(r)=0\) com grau de multiplicidade um. Justifique \(P'(\alpha)=0\).

  \answer{}

  Solving \(P'(\alpha)=0\)
  \begin{tcolorbox}
    \begin{gather*}
      P'(\alpha)
      = 
    \end{gather*}
  \end{tcolorbox}

  Finding \(P'(r)\)
  \begin{tcolorbox}
    \begin{gather*}
      P'(r)
      = \mathText{using \eqref{eq:q.5.1 P(r)}}
      = \mdif{r}{\left(
        (r-\alpha)
      \right)}
      = 1
    \end{gather*}
  \end{tcolorbox}

  Finding \(P(r)\)
  \begin{tcolorbox}
    \begin{gather*}
      P
      = \mdif[n]{x}
      + \sum_{k=0}^{n-1}{a_k\,\mdif[k]{x}}
      \implies \mathText{\(\mdif[n]{x} \to r^n\)}
      \implies
      P(r)
      = r^n
      + \sum_{k=0}^{n-1}{a_k\,r^k}
      = r^n
      + a_0
      + \sum_{k=1}^{n-1}{a_k\,r^k}
      % 
      \yesnumber\label{eq:q.5.1 P(r) 1}
      % 
      = \mathText{Since \(\alpha\) is the only root with multiplicity 1 we can write \(P(r)\) as follows}
      = r-\alpha
      %
      \yesnumber\label{eq:q.5.1 P(r) 2}
      %
      \implies \\
      \implies
      \begin{cases}
        a_0 = -\alpha
        \\
        n=1
      \end{cases}
      %
      \yesnumber\label{eq:q.5.1 n a0}
    \end{gather*}
  \end{tcolorbox}

  Mapping \(P(r)\) roots to general solution of \(y_h\)
  \begin{tcolorbox}
    \begin{gather*}
      r_0 = 0 \implies e^{0\,x}\,c_0
      %
      \yesnumber\label{eq:q.5.1 mapped roots}
    \end{gather*}
  \end{tcolorbox}

  \answer{}

  % tag   = q.5.1
  % i     = 
  % P     = \mdif[n]{x} + \sum_{k=0}^{n-1}{a_k\,\mdif[k]{x}}
  % alpha = \alpha
  % Pk    = 1
  % k     = 0

  % y
  % (\mdif[n]{x} + \sum_{k=0}^{n-1}{a_k\,\mdif[k]{x}})
  % = e^{\alpha x}
  % ( 1 )

  Finding \(\bar{y}\)
  \begin{tcolorbox}
    \(p=1\) from given roots of \(P(r)\)
    \begin{gather*}
      \bar{y}
      = x^p
      \,e^{\alpha\,x}
      \,Q_0(x)
      = x^1
      \,e^{\alpha\,x}
      \,\sum_{i=0}^{0}{
        \rho_i\,x^i
      }
      = x^1\,e^{\alpha\,x}\,\rho_0
      %
      \yesnumber\label{eq:q.5.1 bar y (const)}
      = \mathText{using \eqref{eq:q.5.1 const of bar y}}
      = x^p
      \,e^{\alpha\,x}
      \,\rho_0
      %
      \yesnumber\label{eq:q.5.1 bar y}
    \end{gather*}
  \end{tcolorbox}

  Finding constants of \eqref{eq:q.5.1 bar y (const)}
  \begin{tcolorbox}
    \begin{gather*}
      \bar{y}\,P
      = x^1
      \,\rho_0
      \,\left(
        \mdif[n]{x} + \sum_{k=0}^{n-1}{a_k\,\mdif[k]{x}}
      \right)
      = a_0\,x + a_1\,\rho_0
      = \\
      = 1
      \implies \\
      \begin{cases}
        %
        & \implies \rho_0 = 
        \\
        %
        & \implies \rho_1 = 
      \end{cases}
      %
      \yesnumber\label{eq:q.5.1 const of bar y}
    \end{gather*}
  \end{tcolorbox}

\end{questionBox}

\begin{questionBox}*1m{} % 5 Q2
  A equação \eqref{eq:5.1-enunciado} tem uma solução particular da forma
  \begin{BM}
    \bar{y}=\frac{c}{2\,P'(\alpha)}\,x\,e^{\alpha\,x}
    ,\quad c\in\mathbb{R}
  \end{BM}
  Sabendo que
  \begin{BM}
    \mdif[k]{x}\,(x\,e^{\alpha\,x})
    = k\,\alpha^{k-1}\,e^{\alpha\,x}
    + \alpha^{k}\,x\,e^{\alpha\,x}
    ,\quad \forall\,k\in\mathbb{N}
    ,\quad \mdif[k]{x}=\odv[k]{}{x}
  \end{BM}
  Determine, justificando detalhadamente, o valor de \emph{c}.
\end{questionBox}

\end{document}

% !TEX root = ./FT_II-Exercises_Resolutions.2.tex
\providecommand\mainfilename{"./FT_II-Exercises_Resolutions.tex"}
\providecommand \subfilename{}
\renewcommand   \subfilename{"./FT_II-Exercises_Resolutions.2.tex"}
\documentclass[\mainfilename]{subfiles}

% \tikzset{external/force remake=true} % - remake all

\begin{document}

% \graphicspath{{\subfix{./.build/figures/Exercicios_Resolucoes}}}
% \tikzsetexternalprefix{./.build/figures/Exercicios_Resolucoes/}

\mymakesubfile{1}
[FT II]
{Exercicios: Difusão em estado pseudo-estacionário} % Subfile Title
{Difusão em estado pseudo-estacionário} % Part Title


\begin{questionBox}1{ % Q1
    Uma camada de água com 1\,\unit{\milli\metre} de espessura é mantida a 20\,\unit{\celsius} em contacto com o ar seco a 1\,\unit{\atm}. Admitindo que a evaporação se dá por difusão molecular através de uma camada de ar estagnado com 5\,\unit{\milli\metre} de espessura, calcule o tempo necessário para que a água evapore completamente. O coeficiente de difusão de água no ar é 0.26\,\unit{\centi\metre^2.\second^{-1}} e a pressão de vapor da água a 20\,\unit{\celsius} é 0.0234\,\unit{\atm}.
} % Q1
    
\end{questionBox}

\begin{questionBox}1{ % Q2
    Calcule o tempo necessário para sublimar completamente uma esfera de naftleno (\ch{C10H8}) cujo diâmetro inicial é 1\,\unit{\centi\metre}. A esfera está colocada numa quantidade “infinita” de ar a 318\,\unit{\kelvin}.
} % Q2
    \begin{BM}
        p_{*\,(\text{naftaleno})} 
        = 0.106\,\unit{\atm}
        \qquad
        \rho_{(\text{naftaleno})} 
        = 1140\,\unit{\kilo\gram/\metre^3}
        \\
        \mathscr{D}_{\text{Naft,Ar}}
        = 6.9*10^{-7}\,\unit{\metre^2/\second}
    \end{BM}

    \answer{}

    \begin{flalign*}
        &   
            \text{C. Fronteira}
            \begin{cases}
                r=r_0 &\quad y_A=y_{A,*}
                \\
                r=\infty &\quad y_A=0
            \end{cases}
            &\\[1.5ex]&
            % ===================== t ==================== %
            Q_A
            = -C_{A,L}\,\odv{V}{t}
            = N_{A,r}\,S_r
            = N_{A,r}\,4\,\pi\,r^2
            \implies &\\&
            \implies
            -C_{A,L}\,\odif{V}
            = -C_{A,L}\,\odif{\pi\,r^3\,4/3}
            = -C_{A,L}\,\pi\,(4/3)\,3\,r^2\odif{r}
            = -C_{A,L}\,\pi\,4\,r^2\odif{r}
            = &\\[1.5ex]&
            = N_{A,r}\,4\,\pi\,r^2
            \,\odif{t}
            \implies &\\&
            \implies
            -C_{A,L}\odv{r}{t}
            = N_{A,r}
            ; &\\[3ex]&
            % ============================================ %
            % ============================================ %
            % ============================================ %
            % \implies &\\&
            % \implies
            % \int_{0}^{t}{
            %     Q_A\,\odif{t}
            % }
            % % = &\\&
            % = Q_A
            % \,\int_{0}^{t}{\odif{t}}
            % % = &\\&
            % = (N_{A,r}\,S_r)
            % \,\adif{t}\big\vert_{0}^{t}
            % = &\\&
            % = N_{A,r}\,4\,\pi\,r^2\,t
            % = \left(
            %     y_A(N_{A,r}+N_{B,r})
            %     - \frac{P\,\mathscr{D}_{A,B}}{R\,T}
            %     \,\odv{y_A}{r}
            % \right)
            % \,4\,\pi\,r^2\,t
            % = \left(
            %     y_A(N_{A,r}+N_{B,r})
            %     - \frac{P\,\mathscr{D}_{A,B}}{R\,T}
            %     \,\odv{y_A}{r}
            % \right)
            % \,4\,\pi\,r^2\,t
            % ============================================ %
            % = &\\[1.5ex]&
            % = \int_{r_0}^{0}{-C_{A,L}\,\odif{V}}
            % = -C_{A,L}\,\int_{r_0}^{0}{\odif{(4/3)\pi\,r^3}}
            % = -C_{A,L}\,\int_{r_0}^{0}{(4/3)\pi\,3\,r^2\odif{r}}
            % = &\\&
            % = -C_{A,L}\,4\pi
            % \,\int_{r_0}^{0}{r^2\odif{r}}
            % % = &\\&
            % = -C_{A,L}\,4\pi
            % \,\adif{(r^3/3)}\big\vert_{r_0}^{0}
            % = C_{A,L}\,\pi\,r_0^3\,4/3
            % \implies &\\&
            % \implies
            % t 
            % = \frac{
            %     C_{A,L}\,\pi\,r_0^3\,4/3
            % }{
            %     N_{A,r}\,4\,\pi\,r^2
            % }
            % = \frac{
            %     C_{A,L}\,r_0^3
            % }{
            %     N_{A,r}\,3\,r^2
            % }
            % ================== N_{A,r} ================= %
            N_{A,r}
            = y_A(N_{A,r} + N_{B,r})
            -\frac{P\,\mathscr{D}_{A,B}}{R\,T}
            \odv{y_A}{r}
            = y_A\,N_{A,R}
            -\frac{P\,\mathscr{D}_{A,B}}{R\,T}
            \odv{y_A}{r}
            \implies &\\&
            \implies
            N_{A,r}\,\odif{r}
            = -\frac{P\,\mathscr{D}_{A,B}}{R\,T}
            \frac{\odif{y_A}}{1-y_{A}}
            \implies &\\&
            \implies
            \int_{r_0}^{\infty}{
                N_{A,r}\,\odif{r}
            }
            = \int_{r_0}^{\infty}{
                \frac{N_{A,r_0}\,S_0}{S_r}
                \,\odif{r}
            }
            = N_{A,r_0}
            \,\int_{r_0}^{\infty}{
                \frac{
                    4\,\pi\,r_0^2
                }{
                    4\,\pi\,r^2
                }
                \,\odif{r}
            }
            % = &\\&
            = N_{A,r_0}\,r_0^2
            \,\int_{r_0}^{\infty}{
                \frac{\odif{r}}{r^2}
            }
            = &\\&
            = N_{A,r_0}\,r_0^2
            \,\adif{(-r^{-1})}\big\vert_{r_0}^{\infty}
            = N_{A,r_0}\,r_0^{2}
            \left(
                r_0^{-1}-0
            \right)
            = N_{A,r_0}\,r_0
            = &\\[3ex]&
            = \int_{y_{A,*}}^{0}{
                -\frac{P\,\mathscr{D}_{A,B}}{R\,T}
                \frac{\odif{y_A}}{1-y_{A}}
            }
            = \frac{P\,\mathscr{D}_{A,B}}{R\,T}
            \,\int_{y_{A,*}}^{0}{
                \frac{\odif{(1-y_A)}}{1-y_{A}}
            }
            % = &\\&
            = \frac{P\,\mathscr{D}_{A,B}}{R\,T}
            \,\adif{\ln{(1-y_A)}}\big\vert_{y_{A,*}}^{0}
            = &\\&
            = \frac{P\,\mathscr{D}_{A,B}}{R\,T}
            \,\ln{(1-y_{A,*})^{-1}}
            \implies &\\&
            \implies
            N_{A,r_0}
            = \frac{P\,\mathscr{D}_{A,B}}{R\,T\,r_0}
            \,\ln{(1-y_{A,*})^{-1}}
            \implies &\\[3ex]&
            % ===================== t ==================== %
            \implies
            -C_{A,L}\odv{r_0}{t}
            = \frac{P\,\mathscr{D}_{A,B}}{R\,T\,r_0}
            \,\ln{(1-y_{A,*})^{-1}}
            % \therefore 
            % t
            % = \frac{
            %     C_{A,L}\,r_0^3
            % }{
            %     \left(
            %         \frac{P\,\mathscr{D}_{A,B}}{R\,T\,r_0}
            %         \,\ln{(1-y_{A,*})^{-1}}
            %     \right)
            %     \,3\,r_0^2
            % }
            % = &\\&
            % = \frac{
                % C_{A,L}\,r_0^2
            % }{
                % \frac{P\,\mathscr{D}_{A,B}}{R\,T}
            %     \,\ln{(1-y_{A,*})^{-1}}
            %     \,3
            % }
            % = &\\&
            % = \frac{
            %     C_{A,L}
            %     \,R\,T
            % }{
            %     P\,\mathscr{D}_{A,B}
            %     \,3
            % }
            % \frac{
            %     r_0^2
            % }{
            %     \,\ln{(1-y_{A,*})^{-1}}
            % }
        &
    \end{flalign*}
\end{questionBox}

% \begin{questionBox}1{ % Q3
%     Foi usada uma célula de Arnold para medir o coeficiente de difusão do clorofórmio em ar a 25\,\unit{\celsius} e à pressão de 1\,\unit{\atm}. A massa específica do clorofórmio é 1.485\,\unit{\gram/\centi\metre^3} e a pressão de vapor é 200\,\unit{\mmHg}. No tempo \(t=0\) a superfície do clorofórmio líquido situava-se a 7.4\,\unit{\centi\metre} do topo do tubo e após 10 horas a superfície do líquido desceu 0.44\,\unit{\centi\metre}. Se a concentração de clorofórmio for nula no topo do tubo, qual será o valor do coeficiente de difusão do clorofórmio em ar?
% } % Q3
% \end{questionBox}

% \begin{questionBox}1{ % Q4
%     Uma gota de água com geometria de hemisfério repousa numa superfície plana. O diâmetro do hemisfério da gota de água é reduzido de 0.6\,\unit{\centi\metre} até 0.125\,\unit{\centi\metre}, por evaporação através de difusão molecular num filme estagnado de azoto com 0.5\,\unit{\centi\metre} de espessura. O teor de vapor de água no seio da fase gasosa de azoto é nulo. A pressão de vapor de água à temperatura do ensaio (25\,\unit{\centi\metre}) é de \(1.013*10^4\,\unit{\pascal}\) e a pressão total do sistema é de \(1.013*10^5\,\unit{\pascal}\). A esta pressão e temperatura o coeficiente de difusão da água em azoto é de \(2.1*10^{-5}\,\unit{\metre^2/\second}\).
% } % Q4
%     \begin{questionBox}2{ % Q4.1
%         Calcule o tempo necessário para o processo acima descrito se a espessura do filme de azoto for constante.
%     } % Q4.1
%     \end{questionBox}

%     \begin{questionBox}2{ % Q4.2
%         Repita o cálculo anterior no caso da espessura do filme de azoto ocupar o espaço deixado livre pela água evaporada.
%     } % Q4.2
%     \end{questionBox}
% \end{questionBox}

% \begin{questionBox}1{ % Q5
%     Uma partícula de carvão queima no ar a 1145\,\unit{\kelvin} e o processo é limitado pela difusão de \ch{O2} em sentido oposto ao do \ch{CO} formado à superfície. Se o carvão for considerado como uma esfera de carbono puro com uma massa específica de 1280\,\unit{\kilo\gram/\metre^3} e com um diâmetro inicial de 0.015\,\unit{\centi\metre}:
% } % Q5

%     \begin{BM}
%         \mathscr{D}_{\ch{O2}-\text{mistura}}
%         = 10^{-4}\,\unit{\metre^2/\second}
%     \end{BM}

%     \begin{questionBox}2{ % Q5.1
%         Calcule o tempo que a partícula demora a arder completamente
%     } % Q5.1
%     \end{questionBox}

%     \begin{questionBox}2{ % Q5.2
%         Calcule o tempo que a partícula demora a arder completamente
%     } % Q5.2
%     \end{questionBox}
% \end{questionBox}

\end{document}
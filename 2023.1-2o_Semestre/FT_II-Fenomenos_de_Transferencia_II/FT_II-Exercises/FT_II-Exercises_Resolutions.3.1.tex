% !TEX root = ./FT_II-Exercises_Resolutions.3.tex
\providecommand\mainfilename{"./FT_II-Exercises_Resolutions.tex"}
\providecommand \subfilename{}
\renewcommand   \subfilename{"./FT_II-Exercises_Resolutions.3.tex"}
\documentclass[\mainfilename]{subfiles}

% \tikzset{external/force remake=true} % - remake all

\begin{document}

% \graphicspath{{\subfix{./.build/figures/FT_II-Exercises_Resolutions.3}}}

\mymakesubfile{1}
[FT II]
{Exercicios: Transferencia de massa entre fases} % Subfile Title
{Transferencia de massa entre fases} % Part Title

\setcounter{question}{1}

\begin{questionBox}1{ % Q2
    Água com cloro usada no branqueamento da pasta de papel é obtida por adsorção de cloro gasoso em água numa coluna de enchimento a 293\,\unit{\kelvin} e 1\,\unit{\atm}. Num dado ponto da coluna a pressão de cloro no gás é 100\,\unit{\mmHg} e a concentração de cloro no liquido é de 1\,\unit{\kilo\gram/\metre^3}
} % Q2
    Se 80\% da resistência à transferencia de massa estiver na fase líquida, calcule:

    \begin{center}
        \vspace{1ex}
        \setlength\tabcolsep{4mm}        % width
        % \renewcommand\arraystretch{1.25} % height
        \begin{tabular}{L *{6}{C}}
            \toprule
            
                P_{\text{cloro}} (\unit{\mmHg})
                & 5 & 10 & 30 & 50 & 100 & 150
                \\
                C_{\text{cloro}} (\unit{\gram/\litre})
                & 0.438 & 0.575 & 0.937 & 1.21 & 1.773 & 2.27
            
            \\\bottomrule
        \end{tabular}
        \vspace{2ex}
    \end{center}

    \begin{questionBox}2{ % Q2.1
        As composições interfaciais.
    } % Q2.1
            \begin{flalign*}
                &
                    0.8 
                    = \frac{
                        C_{a,i}-C_{a,l}
                    }{
                        C_{a,*}-C_{a,l}
                    }
                    % \implies &\\&
                    \implies
                    C_{a,i}
                    = 0.8\left(
                        C_{a,*}-C_{a,l}
                    \right)
                    + c_{a,l}
                    = 0.8(1.773-1) + 1
                    = 1.62
                &
            \end{flalign*}
    \end{questionBox}

    \begin{questionBox}2{ % Q2.2
        As composições de equilíbrio.
    } % Q2.2
        body
    \end{questionBox}


\end{questionBox}

\end{document}
% !TEX root = ./FT_II-Exercises_Resolutions.1.tex
\providecommand\mainfilename{"./FT_II-Exercises_Resolutions.tex"}
\providecommand \subfilename{}
\renewcommand   \subfilename{"./FT_II-Exercises_Resolutions.1.tex"}
\documentclass[\mainfilename]{subfiles}

% \tikzset{external/force remake=true} % - remake all

\begin{document}

% \graphicspath{{\subfix{./.build/figures/FT_II-Exercises_Resolutions.1}}}
% \tikzsetexternalprefix{./.build/figures/FT_II-Exercises_Resolutions.1/}

\mymakesubfile{1}
[FT II]
{Exercicios -- Difusão em estado estacionário} % Subfile Title
{Difusão em estado estacionário} % Part Title

\begin{questionBox}1{ % Q1
    Um componente A difunde-se através de uma camada em repouso de um componente B de espessura Z. A pressão parcial de A num dos lados da camada é \(p_{A,1}\) e no outro lado \(p_{A,2} < p_{A,1}\).
} % Q1
    Mostre que o fluxo máximo possível de A através dessa camada é dado por:
    \begin{BM}
        N_{A,\max}
        = \frac{\mathcal{D}\,P}{R\,T\,Z}
        \,\ln\frac{P}{P-p_{A,1}}
        ;\qquad
        P:\text{Pressão total}
    \end{BM}

    \answer{}
    \begin{flalign*}
        &
            N_{A,z}
            = y_A(N_{A,z}+N_{B,z})
            -\frac{P\,\mathcal{D}_{A,B}}{R\,T}
            \,\odv{y_A}{z}
            = y_A\,N_{A,z}
            -\frac{P\,\mathcal{\mathscr{D}_{A,B}}}{R\,T}
            \,\odv{y_A}{z}
            \implies &\\[3ex]&
            \implies
            y_A\,N_{A,z}-N_{A,z}
            = N_{A,z}(y_A-1)
            % = &\\[3ex]&
            =
            \frac{P\,\mathcal{\mathscr{D}_{A,B}}}{R\,T}
            \,\odv{y_A}{z}
            \implies &\\[3ex]&
            \implies
            \int_{0}^{z}{N_{A,z}\,\odif{z}}
            = N_{A,z}\,\int_{0}^{z}{\odif{z}}
            = N_{A,z}\,\adif{z}\big\vert_{0}^{z}
            = N_{A,z}\,z
            = &\\[1.5ex]&
            =\int_{y_{A,1}}^{y_{A,2}}{
                \frac{P\,\mathcal{\mathscr{D}_{A,B}}}{R\,T}
                \,\frac{\odif{y_A}}{y_A-1}
            }
            =
            \frac{P\,\mathcal{\mathscr{D}_{A,B}}}{R\,T}
            \int_{y_{A,1}}^{y_{A,2}}{
                \frac{\odif{y_A}}{y_A-1}
            }
            =
            \frac{P\,\mathcal{\mathscr{D}_{A,B}}}{R\,T}
            \int_{y_{A,1}}^{y_{A,2}}{
                \frac{\odif{(y_A-1)}}{y_A-1}
            }
            = &\\&
            =
            \frac{P\,\mathcal{\mathscr{D}_{A,B}}}{R\,T}
            \adif{\left(
                \ln(y_{A}-1)
            \right)}\big\vert_{y_{A,1}}^{y_{A,2}}
            =
            \frac{P\,\mathcal{\mathscr{D}_{A,B}}}{R\,T}
            \ln\frac{y_{A,2}-1}{y_{A,1}-1}
            =
            \frac{P\,\mathcal{\mathscr{D}_{A,B}}}{R\,T}
            \ln\frac{p_{A,2}/P-1}{p_{A,1}/P-1}
            = &\\&
            =
            \frac{P\,\mathcal{\mathscr{D}_{A,B}}}{R\,T}
            \ln\frac{p_{A,2}-P}{p_{A,1}-P}
            \implies &\\[3ex]&
            \implies
            N_{a,z}
            = \frac{P\,\mathcal{\mathscr{D}_{A,B}}}{R\,T\,z}
            \ln\frac{p_{A,2}-P}{p_{A,1}-P}
            \underset{p_{A,2}=0}{=}
            \frac{P\,\mathcal{\mathscr{D}_{A,B}}}{R\,T\,z}
            \ln\frac{-P}{p_{A,1}-P}
            =\frac{P\,\mathcal{\mathscr{D}_{A,B}}}{R\,T\,z}
            \ln\frac{P}{P-p_{A,1}}
        &
    \end{flalign*}
    O fluxo é maximo quando \(p_{A,2}=0\)
\end{questionBox}

\begin{questionBox}1{ % Q2
    Moldou-se naftaleno sob a forma de um cilindro de raio \(r_1\) que se deixou sublimar no ar em repouso. Mostre que a velocidade de sublimação é dada por:
} % Q2
    \begin{BM}
        Q
        =\frac{2\,\pi\,L\,\mathcal{D}\,P}{R\,T}
        \,\frac{
            \ln\left(
                \frac
                {1-y_{A,2}}
                {1-y_{A,*}}
            \right)
        }{
            \ln(r_2/r_1)
        }
    \end{BM}

    \begin{description}[
        leftmargin=!,
        labelwidth=\widthof{} % Longest item
    ]
        \item[\(y_{A,*}\)] Fração molar correspondente a pressão de vapor do naftaleno
        \item[\(y_{A,2}\)] Fração molar correspondente a \(r_2\)
    \end{description}

    \subsubquestion{Explique o que sucede à  velocidade de sublimação quando \(r_2\) se torna muito grande}

    \answer{}
    ar em repouso\(\iff \bar{N}_{B,r}=0\)
    \begin{flalign*}
        &
            N_{A,r}
            = y_{A}(N_{A,r}+N_{B,r})
            - \frac{P\,\mathscr{D}_{A,B}}{R\,T}
            \,\odv{y_{A,r}}{r}
            = y_{A}N_{A,r}
            - \frac{P\,\mathscr{D}_{A,B}}{R\,T}
            \,\odv{y_A}{r}
            \implies &\\&
            \implies
            N_{A,r}(y_{A,r}-1)
            = \frac{P\,\mathscr{D}_{A,B}}{R\,T}
            \,\odv{y_{A,r}}{r}
            \implies
            \frac{N_{A,1}\,r_1}{r}
            = \frac{P\,\mathscr{D}_{A,B}}{R\,T}
            \,\frac{1}{y_{A,r}-1}
            \,\odv{y_{A,r}}{r}
            \implies &\\[3ex]&
            \implies
            \int_{r_1}^{r_2}{
                N_{A,1}\,r_1
                \frac{\odif{r}}{r}
            }
            = N_{A,1}\,r_1
            \int_{r_1}^{r_2}{
                \frac{\odif{r}}{r}
            }
            = N_{A,1}\,r_1
            \adif{\ln(r)}\big\vert_{r_1}^{r_2}
            = N_{A,1}\,r_1
            \ln\frac{r_2}{r_1}
            % ============================================ %
            = &\\[3ex]&
            = \int_{y_{A,1}}^{y_{A,2}}{
                \frac{P\,\mathscr{D}_{A,B}}{R\,T}
                \,\frac{\odif{y_{A,r}}}{y_{A,r}-1}
            }
            = 
            \frac{P\,\mathscr{D}_{A,B}}{R\,T}
            \int_{y_{A,1}}^{y_{A,2}}{
                \frac{\odif{y_{A,r}}}{y_{A,r}-1}
            }
            = 
            \frac{P\,\mathscr{D}_{A,B}}{R\,T}
            \int_{y_{A,1}}^{y_{A,2}}{
                \frac{\odif{(y_{A,r}-1)}}{y_{A,r}-1}
            }
            = &\\&
            = 
            \frac{P\,\mathscr{D}_{A,B}}{R\,T}
            \adif{\ln(y_{A,r}-1)}
            \big\vert_{y_{A,1}}^{y_{A,2}}
            = 
            \frac{P\,\mathscr{D}_{A,B}}{R\,T}
            \ln\frac{y_{A,2}-1}{y_{A,1}-1}
            ;&\\[3ex]&
            Q
            =N_{A,S}\,S
            =N_{A,1}\,2\,\pi\,r_1\,L
            \implies
            N_{A,1}\,r_1
            =\frac{Q}{2\,\pi\,L}
            \implies &\\[3ex]&
            \implies 
            \frac{Q}{2\,\pi\,L}
            \ln\frac{r_2}{r_1}
            = 
            \frac{P\,\mathscr{D}_{A,B}}{R\,T}
            \ln\frac{y_{A,2}-1}{y_{A,1}-1}
            \implies &\\&
            \implies
            Q
            = 
            \frac{P\,\mathscr{D}_{A,B}\,2\,\pi\,L}{R\,T}
            \,\frac
            {\ln\left(
                \frac{y_{A,2}-1}{y_{A,1}-1}
            \right)}
            {\ln(r_2/r_1)}
            = 
            \frac{2\,\pi\,L\,\mathscr{D}_{A,B}\,P}{R\,T}
            \,\frac
            {\ln\left(
                \frac{1-y_{A,2}}{1-y_{A,1}}
            \right)}
            {\ln(r_2/r_1)}
        &
    \end{flalign*}

    \answer{(i)}
    A velocidade de sublimação é constante por se tratar de um cilindro.

\end{questionBox}

\begin{questionBox}1{ % Q3
    Um tubo com 1\,\unit{\centi\metre} de diâmetro e 20\,\unit{\centi\metre} de comprimento está cheio com uma mistura de \ch{CO2} e \ch{H2} a uma pressão total de 2\,\unit{\atm} e a uma temperatura de 0\,\unit{\celsius}. O coeficiente de difusão do \ch{CO2-H2} nestas condições é 0.275\,\unit{\centi\metre^2.\second^{-1}}. Se a pressão parcial do \ch{CO2} for 1.5\,\unit{\atm} num dos lados do tubo e 0.5\,\unit{\atm} no outro extremo
} % Q3

    \begin{questionBox}2{ % Q3.1
        calcule a velocidade de difusão para Contradifusão equimolar
    } % Q3.1
        \begin{BM}
            N_{\ch{CO2}}=-N_{\ch{H2}}
        \end{BM}
    \end{questionBox}

    \begin{questionBox}2{ % Q3.2
        calcule a velocidade de difusão para A seguinte relação entre os fluxos 
    } % Q3.1
        \begin{BM}
            N_{\ch{H2}} = -0.75\,N_{\ch{CO2}}
        \end{BM}
    \end{questionBox}
\end{questionBox}

\end{document}

% !TEX root = ./FT_II-Exercises_Resolutions.extras.tex
\providecommand\mainfilename{"./FT_II-Exercises_Resolutions.tex"}
\providecommand \subfilename{}
\renewcommand   \subfilename{"./FT_II-Exercises_Resolutions.extras.tex"}
\documentclass[\mainfilename]{subfiles}

% \tikzset{external/force remake=true} % - remake all

\begin{document}

% \graphicspath{{\subfix{./figures/FT_II-Exercises_Resolutions.extras}}}
% \tikzsetexternalprefix{./figures/FT_II-Exercises_Resolutions.extras/graphics/}

\mymakesubfile{5}
[FT II]
{Exerices extras} % Subfile Title
{Exerices extras} % Part Title

\begin{questionBox}1{ % MARK: Q1
    A composição molar de uma mistura gasosa a \qty*{273}{\kelvin} e \qty*{1.5e5}{\Pa} é:
} % Q1
    \begin{center}
        \vspace{1ex}
        \begin{tabular}{*{4}{C}}
            \toprule

                    \ch{O2}
                &   \ch{CO}
                &   \ch{CO2}
                &   \ch{N2}
                \\  7\%
                &   10\%
                &   15\%
                &   68\%
            
            \\\bottomrule
        \end{tabular}
        \vspace{2ex}
    \end{center}
    Determine:

    \begin{questionBox}2{ % MARK: Q1.1
        A composição em percentagem mássica
    } % Q1.1
        \answer{}
        \begin{flalign*}
            &
                X_{m,i}
                =\frac{m_i}{m_{total}}
                =\frac{m_i}{\sum{m_{i}}}
                =\frac
                {X_{N,i}\,M_i\,N_{total}}
                {\sum{X_{N,i}\,M_i\,N_{total}}}
                =\frac
                {X_{N,i}\,M_i}
                {\sum{X_{N,i}\,M_i}}
                ; &\\&
                \frac{m_{total}}{N_{total}}
                \cong .07*32+.10*30+.15*44+.68*28
                = 30.88
                ; &\\&
                \begin{cases}
                    X_{m,\ch{O2}}
                    & = .07*32/30.88
                    \cong\qty{7.25388601}{\percent}
                    \\
                    X_{m,\ch{CO}}
                    & = .10*30/30.88
                    \cong\qty{9.715025907}{\percent}
                    \\
                    X_{m,\ch{CO2}}
                    & = .15*44/30.88
                    \cong\qty{21.37305699}{\percent}
                    \\
                    X_{m,\ch{N2}}
                    & = .68*28/30.88
                    \cong \qty{61.65803109}{\percent}
                \end{cases}
            &
        \end{flalign*}
    \end{questionBox}
    \begin{questionBox}2{ % MARK: Q1.2
        A massa específica da mistura gasosa
    } % Q1.2
        \answer{}
        \begin{flalign*}
            &
                \rho
                = \frac{m}{V}
                = \frac{m}{\frac{N\,R\,T}{P}}
                \cong \frac{
                    30.88
                    * N
                    * 1.5\E{5}
                }{
                    N
                    * \num{8.31446261815324}
                    * 273
                }
                \cong
                \qty{2.040665013032626461}{\kilo\gram/\m^3}
            &
        \end{flalign*}
    \end{questionBox}
\end{questionBox}

\begin{questionBox}1{ % MARK: Q2
    Determine o coeficiente de difusão do \ch{CO} numa mistura gasosa cuja composição é:
    \begin{BM}
                y_{\ch{O2}}=0.20
        ;\quad  y_{\ch{N2}}=0.70
        ;\quad  y_{\ch{CO}}=0.10
    \end{BM}
    A mistura está à temperatura de \qty*{298}{\K} e à pressão de \qty*{2}{\atm}. Os coeficientes de difusão do \ch{CO} em oxigénio e azoto são:
    \begin{BM}
        \mathscr{D}_{\ch{CO,O2}}
        (\qty*{273}{\K},\qty*{1}{\atm})
        = \qty*{1.85e-5}{\m^2/\s}
        \\
        \mathscr{D}_{\ch{CO,N2}}
        (\qty*{288}{\K},\qty*{1}{\atm})
        = \qty*{1.92e-5}{\m^2/\s}
    \end{BM}
} % Q2
    \answer{}
    \begin{flalign*}
        &
            \mathscr{D}_{\ch{CO},M}
            = \frac{
                \sum_{j=2}^{n}{y_{j}}
            }{
                \sum_{i=2}^{n}{y_i/\mathscr{D}_{\ch{CO},i}}
            }
            % 
            % 
            % 
            ; &\\[3ex]&
            \mathscr{D}_{(
                \ch{CO,N2}
                ,\qty*{298}{\K}
                ,\qty*{2}{\atm}
            )}
            =
            \mathscr{D}_{(
                \ch{CO,N2}
                ,\qty*{288}{\K}
                ,\qty*{1}{\atm}
            )}
            \,\frac{1}{2}
            \,\left(
                \frac{298}{288}
            \right)^{3/2}
            % 0.52626643134963
            % \cong &\\&
            \cong
            \qty{1.01043154819129e-5}{\m^2/\s}
            % 
            % 
            % 
            ; &\\[1ex]&
            \mathscr{D}_{(
                \ch{CO,O2}
                ,\qty*{298}{\K}
                ,\qty*{2}{\atm}
            )}
            =
            \mathscr{D}_{(
                \ch{CO,O2}
                ,\qty*{273}{\K}
                ,\qty*{1}{\atm}
            )}
            \,\frac{1}{2}
            \,\left(
                \frac{298}{273}
            \right)^{3/2}
            % 0.570230483093418
            % \cong &\\&
            \cong
            \qty{1.054926393722823e-5}{\m^2/\s}
            % 
            % 
            % 
            ; &\\[3ex]&
            \mathscr{D}_{\ch{CO},mist}
            \cong
            \frac{
                0.2+0.7
            }{
                0.2/\num{1.054926393722823e-5}
                +0.7/\num{1.01043154819129e-5}
            }
            \cong
            \qty{1.01999185289502e-5}{\m^2/\s}
        &
    \end{flalign*}
\end{questionBox}

\begin{questionBox}1{ % MARK: Q3
    Um componente A difunde-se através de uma camada em repouso de um componente B de espessura Z. A pressão parcial de A num dos lados da camada é \(P_{A,1}\) e no outro lado \(P_{A,2}<P_{A,1}\) Mostre que o fluxo máximo possível de Aatravés dessa camada é dado por:
    \begin{BM}
        N_{A,\max}
        = \frac{\mathscr{D}\,P}{R\,T\,z}
        \,\ln{\frac{P}{P-P_{A,1}}}
    \end{BM}
} % Q3
    \answer{}
    \begin{flalign*}
        &
            N_{A,\max,z}
            = \frac
            {c\,\mathscr{D}_{A,B}}
            {\Theta\,\eta_d\,z}
            \ln{\frac
                {1-\Theta\,y_{A,2}}
                {1-\Theta\,y_{A,1}}
            }
            = \frac
            {
                \frac{P}{R\,t}
                \,\mathscr{D}_{A,B}
            }
            {\Theta\,\eta_d\,z}
            \ln{\frac
                {1-\Theta\,y_{A,2}}
                {1-\Theta\,y_{A,1}}
            }
            % 
            % 
            % 
            ; &\\[3ex]&
            N_{A,\max,z}\implies y_{A,2}=0
            % 
            % 
            % 
            ; &\\[3ex]&
            \Theta
            = 1+N_{B,z}/N_{A,z}
            = 1
            % 
            % 
            % 
            ; &\\[3ex]&
            \therefore
            N_{A,\max,z}
            = \frac
            {P\,\mathscr{D}_{A,B}}
            {R\,T\,(1)\,z}
            \ln{\frac
                {1}
                {1-y_{A,1}}
            }
            = &\\&
            = \frac
            {P\,\mathscr{D}_{A,B}}
            {R\,T\,z}
            \ln{\frac
                {1}
                {1-P_{A,1}/P}
            }
            = \mathemph{
                \frac
                {P\,\mathscr{D}_{A,B}}
                {z\,R\,T}
                \ln{\frac
                    {P}
                    {P-P_{A,1}}
                }
            }
        &
    \end{flalign*}
\end{questionBox}

\begin{questionBox}1{ % MARK: Q4
    Moldou-se naftaleno sob aforma de um cilindro de raio \(R_1\), que se deixou sublimar no ar em repouso. Mostre que a velocidade de sublimação é dada por:
    \begin{BM}
        Q
        = \frac
        {2\,\pi\,L\,\mathscr{D}\,P}
        {
            R\,T
            \,\ln{\frac{R_2}{R_1}}
        }
        \,\ln{\frac
            {1-y_{A,2}}
            {1-y^*_{A}}
        }
    \end{BM}
    Sendo \(y^*_A\) a fração molar corresponedene à pressão de vapor do naftaleno e \(y_{A,2}\) a fração molar correspondente ao \(R_2\).
} % Q4
    \answer{}
    \begin{flalign*}
        &
            Q
            = N_{A,R_1}\,S_{R_1}
            = \frac
            {c\,\mathscr{D}_{A,B}}
            {\Theta\,R_1\,\ln{(R_2/R_1)}}
            \,\ln{\frac
                {1-\Theta\,y_{A,2}}
                {1-\Theta\,y_{A,1}}
            }
            \,(2\,\pi\,R_1\,L)
            = &\\&
            = \frac{
                \left(\frac{P}{R\,T}\right)
                \,\mathscr{D}_{A,B}
                \,2\,\pi\,L
            }
            {\Theta\,\ln{(R_2/R_1)}}
            \,\ln{\frac
                {1-\Theta\,y_{A,2}}
                {1-\Theta\,y_{A,1}}
            }
            % 
            % 
            % 
            ; &\\[3ex]&
            \Theta
            = 1+N_{B}/N_{A}
            = 1+0/N_{A}
            = 1
            % 
            % 
            % 
            ; &\\[3ex]&
            \therefore
            Q
            % = &\\&
            = \frac
            {
                P
                \,\mathscr{D}_{A,B}
                \,2\,\pi\,L
            }
            {R\,T\,\ln{(R_2/R_1)}}
            \,\ln{\frac
                {1-y_{A,2}}
                {1-y^*_{A}}
            }
        &
    \end{flalign*}
\end{questionBox}

\begin{questionBox}2{ % MARK: Q4.1
    Explique o que sucede à velocidade de sublimação quando \(R_2\) se torna muito grande.
} % Q4.1
    \answer{}
    \begin{flalign*}
        &
            \lim_{R_2\to\infty}{Q}
            = \lim_{R_2\to\infty}{
                \frac
                {
                    P\,\mathscr{D}_{A,B}\,2\,\pi\,L
                }
                {R\,T\,\ln{(R_2/R_1)}}
                \,\ln\frac
                {1-y_{A,2}}
                {1-y^*_{A}}
            }
            = &\\&
            = 
            \frac
            {
                P\,\mathscr{D}_{A,B}\,2\,\pi\,L
            }
            {
                R\,T
                \,\lim_{R_2\to\infty}{
                    \ln{(R_2/R_1)}
                }
            }
            \,\ln\frac
            {1-y_{A,2}}
            {1-y^*_{A}}
            = &\\&
            = 
            \frac
            {
                P\,\mathscr{D}_{A,B}\,2\,\pi\,L
            }
            {
                R\,T
                \infty
            }
            \,\ln\frac
            {1-y_{A,2}}
            {1-y^*_{A}}
            =0
        &
    \end{flalign*}
\end{questionBox}
\begin{questionBox}2{ % MARK: Q4.2
    E se a geometria for esférica
} % Q4.2
    \answer{}
    \begin{flalign*}
        &
            Q
            = N_{A,R_1}\,S_{R_1}
            = \frac
            {c\,\mathscr{D}_{A,B}}
            {\Theta\,R_1(1-R_1/R_2)}
            \,\ln{\frac
                {1-\Theta\,y_{A,2}}
                {1-\Theta\,y_{A,1}}
            }
            \,(4\,\pi\,R_1^2)
            = &\\&
            = \frac
            {
                \left(
                    \frac{P}{R\,T}
                \right)
                \,\mathscr{D}_{A,B}
                \,4\,\pi
            }
            {R_1^{-1}-R_2^{-1}}
            \,\ln{\frac
                {1-y_{A,2}}
                {1-y^*_{A}}
            }
            % 
            % 
            % 
            ; &\\[3ex]&
            \lim_{R_2\to\infty}{Q}
            = \lim_{R_2\to\infty}{
                \frac
                {
                    P\,\mathscr{D}_{A,B}
                    \,4\,\pi
                }
                {R\,T\,(R_1^{-1}-R_2^{-1})}
                \,\ln\frac
                {1-y_{A,2}}
                {1-y^*_{A}}
            }
            = &\\&
            = 
            \frac
            {
                P\,\mathscr{D}_{A,B}
                \,4\,\pi
            }
            {
                R\,T
                \,\lim_{R_2\to\infty}{
                    {(R_1^{-1}-R_2^{-1})}
                }
            }
            \,\ln\frac
            {1-y_{A,2}}
            {1-y^*_{A}}
            = &\\&
            = 
            \frac
            {
                P\,\mathscr{D}_{A,B}
                \,4\,\pi
            }
            {
                R\,T
                (R_1^{-1})
            }
            \,\ln\frac
            {1-y_{A,2}}
            {1-y^*_{A}}
        &
    \end{flalign*}
\end{questionBox}

\begin{questionBox}1{ % MARK: Q5
    Um tubo com \qty*{1}{\cm} de diâmetro e \qty*{20}{\cm} de comprimento está cheio com uma mistura de \ch{CO2} e \ch{H2} a uma pressão total de \qty*{2}{\atm} e a uma temperatura de \qty*{0}{\celsius}. O coeficiente de difusão do \ch{CO2-H2} nestas condições é \qty*{0.275}{\cm^2/\s}. Se a pressão parcial do \ch{CO2} for \qty*{1.5}{\atm} num dos lados do tubo e \qty*{0.5}{\atm} no outro extremo, calcule a velocidade de difusão para:
} % Q5
\end{questionBox}

\begin{questionBox}2{ % MARK: Q5.1
    Contradifusão equimolar (\(N_{\ch{CO2}}=-N_{\ch{H2}}\))
} % Q5.1
    \answer{}
    \begin{flalign*}
        &
            \begin{cases}
                A: \ch{CO2}
                \\ B: \ch{H2}
            \end{cases}
            &\\[3ex]&
            Q
            = N_{A,z_1}\,S_{z_1}
            = -\frac
            {c\,\mathscr{D}_{A,B}}
            {\adif{z}}
            \,(y_{A,1}-y_{A,0})
            (\pi\,d^2/4)
            = &\\&
            = -\frac
            {
                \frac{P}{R\,T}
                \,\mathscr{D}_{A,B}
                \,\pi\,d^2/4
            }
            {\adif{z}}
            \,\frac{P_{A,1}-P_{A,0}}{P}
            = &\\&
            = -\frac
            {
                \mathscr{D}_{A,B}
                \,\pi\,d^2/4
            }{R\,T\,\adif{z}}
            \,(P_{A,1}-P_{A,0})
            = &\\&
            \cong -\frac
            {
                0.275
                \,\pi\,1^2/4
            }{
                \num{8.20573660809596e1}
                *273.15
                *20
            }
            \,(0.5-1.5)
            \cong\qty
            {2.409038864139341e-7}
            {\mol/\s}
        &
    \end{flalign*}
\end{questionBox}

\begin{questionBox}2{ % MARK: Q5.2
    A seguinte relação entre os fluxos \(N{\ch{H2}} = -0.75\,N_{\ch{CO2}}\)
} % Q5.2
    \answer{}
    \begin{flalign*}
        &
            \begin{cases}
                A: \ch{CO2}
                \\ B: \ch{H2}
            \end{cases}
            &\\[3ex]&
            Q
            = N_{A,z_1}\,S_{z_1}
            = \frac
            {c\,\mathscr{D}_{A,B}}
            {\Theta\,\adif{z}}
            \,\ln\frac
            {1-\Theta\,y_{A,1}}
            {1-\Theta\,y_{A,0}}
            \,S_{z_1}
            = &\\&
            = \frac
            {
                \left(
                    \frac{P}{R\,T}
                \right)
                \,\mathscr{D}_{A,B}
            }
            {\Theta\,\adif{z}}
            \,\ln\frac
            {1-\Theta\,P_{A,1}/P}
            {1-\Theta\,P_{A,0}/P}
            \,(\pi\,d^2/4)
            % 
            % 
            % 
            ; &\\[3ex]&
            \Theta
            = 1+N_{B}/N_A
            = 1-0.75\,N_{A}/N_A
            = 0.25
            % 
            % 
            % 
            ; &\\[3ex]&
            \therefore
            Q
            = \frac
            {
                P
                \,\mathscr{D}_{A,B}
                \,\pi\,d^2/4
            }
            {
                R\,T\,0.25\,\adif{z}
            }
            \,\ln{
                \frac
                {P-0.25\,P_{A,1}}
                {P-0.25\,P_{A,0}}
            }
            \cong &\\&
            = \frac
            {
                2
                * .275
                * \pi
                * 1^2/4
            }
            {
                \num{8.20573660809596e1}
                * 273.15
                * 0.25
                * 20
            }
            % 3.854462182622945e-6
            \,\ln{
                \frac
                {2-0.25*0.5}
                {2-0.25*1.5}
            }
            % 0.143100843640673
            \cong\qty
            {5.51576790114414e-7}
            {\mol/\s}
        &
    \end{flalign*}
\end{questionBox}

\end{document}
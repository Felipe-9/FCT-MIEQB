% !TEX root = ./FT_II-Exercises_Resolutions.3.tex
\providecommand\mainfilename{"./FT_II-Exercises_Resolutions.tex"}
\providecommand \subfilename{}
\renewcommand   \subfilename{"./FT_II-Exercises_Resolutions.3.tex"}
\documentclass[\mainfilename]{subfiles}

% \tikzset{external/force remake=true} % - remake all

\begin{document}

% \graphicspath{{\subfix{./figures/FT_II-Exercises_Resolutions.3}}}
% \tikzsetexternalprefix{./figures/FT_II-Exercises_Resolutions.3/graphics/}

\mymakesubfile{3}
[FT II]
{Exercicios 3: Difusão em estado transiente (?)} % Subfile Title
{Exercicios 3: Difusão em estado transiente (?)} % Part Title

\begin{questionBox}1{ % MARK: Q1
    Experiências de transferência de calor permitiram obter uma correlação para o coeficiente de transferência de calor, \textit{h}, para um cilindro colocado numa corrente de água.
    \begin{BM}
        Nu
        = \left(
            0.506
            \,Re^{0.5}
            + 0.00141
            \,Re
        \right)
        \,Pr^{1/3}
    \end{BM}
} % Q1

    \paragraph*{Dados:}
    \begin{itemize}
        \begin{multicols}{2}
            \item Solubilidade \ch{NaCl}: \qty*{6}{\mole/\L}
            \item \(\rho_{agua}=\qty*{1e3}{\kg/\m^3}\)
            \item \(\mu_{agua}=\qty*{1e-3}{\N.\s/\m^3}\)
            \item \(\mathscr{D}=\qty*{1.6e-9}{\m^2/\s}\)
        \end{multicols}
        \item Analogia de Chilton--Colburn \(j_H=j_D\)
        \begin{BM}
            \frac{h}{\rho\,u\,C\,p}
            \,Pr^{2/3}
            = \frac{k_C}{u}
            \,Sc^{2/3}
        \end{BM}
    \end{itemize}
    \begin{BM}
        Nu=\frac{h\,d}{k}
        ; \quad
        Pr=\frac{\mu\,C\,p}{j}
        ; \quad
        Sc=\frac{\mu}{\rho\,\mathscr{D}}
        ; \quad
        Re=\frac{\rho\,u\,d}{\mu}
    \end{BM}

    \begin{questionBox}2{ % MARK: Q1.1
        Utilizando a analogia de Chilton--Colburn calcule o coeficiente de transferência de massa para um cilindro de \ch{NaCl} com \qty*{1.5}{\cm} de diâmetro e \qty*{10}{\cm} de comprimento. A água a \qty*{300}{\kelvin} tem uma velocidade de \qty*{10}{\m/\s}.
    } % Q1.1
        \answer{}
        \begin{flalign*}
            &
                \frac{h}{\rho\,v\,c_P}
                \,Pr^{2/3}
                = \frac{k_C}{v}
                \,Sc^{2/3}
                \implies &\\&
                \implies
                k_C
                = \frac{h}{\rho\,c_P}
                \,\left(
                    \frac{Pr}{Sc}
                \right)^{2/3}
                = &\\&
                = \frac{h}{\rho\,c_P}
                \,\left(
                    \frac
                    {\frac{\mu\,c_P}{K}}
                    {\frac{\mu}{\rho\,\mathscr{D}}}
                \right)^{2/3}
                % 
                % 
                % 
                ; &\\[3ex]&
                h: &\\&
                Nu
                = \frac{K}{h}
                \,0.015
                = &\\&
                = \left(
                    0.506
                    \,Re^{0.5}
                    + 0.00141
                    \,Re
                \right)
                \,Pr^{1/3}
                = &\\&
                = \left(
                    0.506
                    \,Re^{0.5}
                    + 0.00141
                    \,Re
                \right)
                \,(C_p/K)^{1/3}
                % 
                % 
                % 
                ; &\\[3ex]&
                Re:&\\&
                Re
                = \frac{\rho\,v\,d}{\mu}
                = \frac{1\E{3}*10*1.5\E{-2}}{1\E{-3}}
                = 1.5\E{5}
            &
        \end{flalign*}
    \end{questionBox}

    \begin{questionBox}2{ % MARK: Q1.2
        A velocidade de dissolução do cilindro.
    } % Q1.2
        \answer{}
        \begin{flalign*}
            &
                W
                = N_A\,A
                = k_C(c_{A,s}-C_{A,0})
                \,(\pi\,d\,L+2\,\pi\,d^2/4)
                = k_C\,c_{A,s}
            &
        \end{flalign*}
        \begin{flalign*}
            &
                w=k_c\,A\,c^*
                ; &\\[3ex]&
                A
                = \pi\,d\,L
                + 2\,\pi\,r^2
            &
        \end{flalign*}
    \end{questionBox}

    \begin{questionBox}2{ % MARK: Q1.3
        Seria possível usar a analogia de Reynolds neste caso? Justifique a sua resposta.
    } % Q1.3
        \answer{}
        Condições não conferem: \(Sc\neq1\)
    \end{questionBox}

    \begin{questionBox}2{ % MARK: Q1.4
        A velocidade de dissolução se usar um prisma com uma secção quadrada com \qty*{1.5}{\cm} de lado e \qty*{10}{\cm} de comprimento.
    } % Q1.4
        \answer{}
        \begin{flalign*}
            &
                w
                = k_c\,A\,c^*
                ; &\\[3ex]&
                A
                = 2\,A_b+4\,A_L
                = 2\,(1.5)^2\E{-4}
                + 4\,(1.5*10)^2\E{-4}
            &
        \end{flalign*}
    \end{questionBox}

\end{questionBox}

\begin{questionBox}1{ % MARK: Q2
    Foram obtidas as seguintes correlações para o coeficiente de transferência de calor em condutas cilíndricas:
    \begin{BM}
        Nu
        = 0.023
        \,Re^{0.8}
        \,Pr^{0.33},
        \quad
        Re > 1\E4
        \quad
        Re\,Pr\,\frac{d}{L} < 17
        : Nu=4.1
    \end{BM}
    Faz-se passar ar a \qty*{20}{\celsius} a uma velocidade média igual a \qty*{30}{\m/\s} por uma conduta com \qty*{2.5}{\cm} de diâmetro (\textit{d}) e \qty*{2}{\m} de comprimento (\textit{L}), cuja superfície interna está revestida com um componente \textit{A}. Utilizando a analogia de Chilton-Colburn, determine:
} % Q2
    \paragraph*{Dados:}
    \begin{itemize}
        \begin{multicols}{2}
            \item \(P^*_{A,\qty*{20}{\celsius}}=\qty*{4.0}{\mmHg}\)
            \item \(\mu_{Ar}(\qty*{20}{\celsius}) = \qty*{1.74e-5}{\newton.\second/\metre^2}\)
            \item \(\rho_{Ar}(\qty*{20}{\celsius}) = \qty*{1.164e-5}{\kilo\gram/\metre^3}\)
            \item \(\mathscr{D}_{A,Ar}=\qty*{6.2e-6}{\m^2/\second}\)
            \item \(k^T_{Ar,\qty*{20}{\celsius}}=\qty*{0.0251}{\joule/\second.\metre.\kelvin}\)
        \end{multicols}
    \end{itemize}
\end{questionBox}

\begin{questionBox}2{ % MARK: Q2.1
    Coeficiente de transferencia de massa
} % Q2.1
    \answer{}
    \begin{flalign*}
        &
            \frac{h}{\rho\,v\,c_P}
            \,Pr^{2/3}
            = \frac{k_c}{v}
            \,Sc^{2/3}
            % 
            % 
            % 
            ; &\\[3ex]&
            Nu
            = \frac{h\,d}{k}
            = 0.023
            \,Re^{0.8}
            \,Pr^{0.33}
            = 0.023
            \,\left(
                \frac{\rho\,d\,v}{\mu}
            \right)^{0.8}
            \,\left(
                \frac{\mu\,c_P}{k^T}
            \right)^{0.33}
        &
    \end{flalign*}
\end{questionBox}

\begin{questionBox}2{ % MARK: Q2.2
    A velocidade de sublimação e a concentração de A à saída da conduta.
} % Q2.2
    \answer{}
    \begin{flalign*}
        &
            c_A\,v\,\frac{\pi\,d^2}{4}
            + k_C(c_{A,s}-c_A)
            \,\pi\,d\,
        &
    \end{flalign*}
\end{questionBox}

\begin{questionBox}2{ % MARK: Q2.3
    Seria possível usar a analogia de Reynolds neste caso? Justifique a sua resposta.
} % Q2.3
\end{questionBox}

\begin{questionBox}2{ % MARK: Q2.4
    A velocidade de sublimação se a conduta tiver uma secção quadrada com \qty*{2.5}{cm} de lado. Indique todos os passos necessários.
} % Q2.4
    body
\end{questionBox}

\end{document}
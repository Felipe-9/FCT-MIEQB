% !TEX root = ./FT_II-Annotations.2.tex
\providecommand\mainfilename{"./FT_II-Annotations.tex"}
\providecommand \subfilename{}
\renewcommand   \subfilename{"./FT_II-Annotations.2.tex"}
\documentclass[\mainfilename]{subfiles}

% \tikzset{external/force remake=true} % - remake all

\begin{document}

% \graphicspath{{\subfix{./.build/figures/}}}

\mymakesubfile{1}
[FT II]
{Anotações} % Subfile Title
{Anotações} % Part Title

\begin{exampleBox}1{ % E1
} % E1
    
    Ao absorver \ch{NH3} (de uma mistura com ar) em água numa coluna de encgimento a 60\,\unit{\celsius} e 3\,\unit{\atm}, os coeficientes individuais observados foram:

    \begin{BM}
        k_l = 1.1\,\unit{\metre/\hour}
        ; \quad
        k_g = 0.25
        \,\unit{\frac{\mole}{\hour\,\metre^2\,\atm}}
    \end{BM}

    A pressão parcial de \ch{NH3} no equilibrio em soluções diluidas é dada por \(p_{\ch{NH3}}=0.25\,c_{\ch{NH3}}\). Determine os valores dos coeficientes:

    \begin{exampleBox}2{ % E1.1
        \(k_y\)
    } % E1.1
        \begin{flalign*}
            &
                k_y\,(y_a-y_{a,i})
                = k_g\,(p_{a}-p_{a,i})
                = k_g\,(p\,y_{a}-p\,y_{a,i})
                = k_g\,p\,(y_{a}-y_{a,i})
                \implies &\\&
                \implies
                k_y 
                = k_g\,p
                = 0.25*3=0.75\,\unit{\mole/\hour\,\metre^2}
            &
        \end{flalign*}
    \end{exampleBox}

    \begin{exampleBox}2{ % E1.2
        \(k_c\) (gás-conc.molares)
    } % E1.2
        \begin{flalign*}
            &
                k_c\,(c_a-c_{a,i})
                = k_g\,(p_{a}-p_{a,i})
                = k_g\,(c_{a}\,R\,T-c_{a,i}\,R\,T)
                = k_g\,R\,T(c_{a}-c_{a,i})
                \implies &\\&
                \implies
                k_c
                = k_g\,R\,T
                = 0.25 
                * \num{8.20573660809596e-5}
                (60+273.15)
                \cong 
                \qty{683.435287746792269e-5}{\metre/\hour}
            &
        \end{flalign*}
    \end{exampleBox}

    \begin{exampleBox}2{ % E1.3
        \(K_g\)
    } % E1.3
        \begin{flalign*}
            &
                K_{g}
                = (p_a-p_{a,*})^{-1}
                = \left(
                    (p_a-p_{a,i})
                    + (p_{a,i}-p_{a,*})
                \right)^{-1}
                = &\\&
                = \left(
                    (k_g^{-1})
                    + (H(c_{a,i}-c_{a,l}))
                \right)^{-1}
                % = &\\&
                = \left(
                    k_g^{-1}
                    + H/k_l
                \right)^{-1}
                = &\\&
                = \left(
                    0.25^{-1}
                    + 0.25/1.1
                \right)^{-1}
                \cong
                \qty{0.236559139784946}{
                    \frac{\mole}{\hour\,\metre^2\,\atm}
                }
            &
        \end{flalign*}
    \end{exampleBox}

    \begin{exampleBox}2{ % E1.4
        \(K_y\)
    } % E1.4
    \end{exampleBox}

    \begin{exampleBox}2{ % E1.5
        \(K_l\)
    } % E1.5
    \end{exampleBox}

    \begin{exampleBox}2{ % E1.6
        Se um ponto da coluna \(p_{\ch{NH3}} = 0.03 \,\unit{\atm},\ C_{\ch{NH3}}=0.05\,\unit{\mole/\metre}\), qual o fluxo de absorção do \ch{NH3}?
    } % E1.6
        \begin{flalign*}
            &
                N_a
                = K_g\,(p_{a,g}-p_{a,*})
                = K_g\,(p_{a,g}-m\,c_{a,L})
                = &\\&
                = \num{0.236559139784946}
                \,(0.03-0.25*0.05)
                \cong \qty{4.139784946236555e-3} {\mole/\hour\,\metre^2}
            &
        \end{flalign*}
    \end{exampleBox}

    \begin{exampleBox}2{ % E1.7
        Quais os valores das composições interfaciais?
    } % E1.7
        \begin{flalign*}
            &
                N_a
                = k_g(p_{a,g}-p_{a,i})
                % \implies &\\&
                \implies
                p_{a,i}
                = p_{a,g}-N_a/k_g
                \cong &\\&
                \cong 0.03
                -\num{4.139784946236555e-3}
                /0.25
                \cong
                \num{1.344086021505378e-2}
            &\\[3ex]&
                c_{a,i}
                = p_{a,i}/H
                \cong \num{1.344086021505378e-2}/0.25
                \cong \num{5.376344086021512e-2}
            &
        \end{flalign*}
    \end{exampleBox}

    \begin{exampleBox}2{ % E1.8
        Qual a resistência exercida em cada fase?
    } % E1.8
        \begin{flalign*}
            &
                r_l
                = K_g/k_g
                \cong \num{0.236559139784946}/0.25
                \cong \qty{94.6236559139784}{\percent}
            &\\&
                r_g
                = 1-r_l
                \cong
                \qty{5.3763440860216}{\percent}
            &
        \end{flalign*}
    \end{exampleBox}
    
\end{exampleBox}

\end{document}
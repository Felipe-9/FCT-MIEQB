% !TEX root = ./FT_II-Annotations.1.tex
\providecommand\mainfilename{"./FT_II-Annotations.tex"}
\providecommand \subfilename{}
\renewcommand   \subfilename{"./FT_II-Annotations.1.tex"}
\documentclass[\mainfilename]{subfiles}

% \tikzset{external/force remake=true} % - remake all

\begin{document}

% \graphicspath{{\subfix{./.build/figures/}}}

\mymakesubfile{1}
[FT II]
{Anotações} % Subfile Title
{Anotações} % Part Title

\begin{questionBox}1{ % Q1
} % Q1
    \begin{flalign*}
        &
            D_{\ch{CO\to Mist},298\unit{\kelvin}}
            = \left(
                \frac{y_2+y_3}{
                    \frac{y_2}{D_{1\to2,298\unit{\kelvin},2\unit{\atm}}}
                   +\frac{y_3}{D_{1\to3,298\unit{\kelvin},2\unit{\atm}}}
                }
            \right)^{-1}
            = &\\&
            = \left(
                \frac{y_2+y_3}{
                     \frac{y_2}{
                        D_{1\to2,278\unit{\kelvin},1\unit{\atm}}
                        *\left(\frac{298}{278}\right)^{3/2}
                        /2
                    }
                    +\frac{y_3}{D_{1\to3,288\unit{\kelvin},1\unit{\atm}}}
                }
            \right)^{-1}
        &
    \end{flalign*}
\end{questionBox}

\begin{questionBox}1{ % Q2
    Um componente A difunde-se através de uma camada em repouso de um componente B de espessura Z. A pressão parcial de A num dos lados da camada é \(P_{A,1}\) e no outro lado \(P_{A,2}<P_{A,1}\).
} % Q2
    Mostre que o fluxo máximo possível de A através dessa camada e dado por:
    \begin{BM}
        \max{N_{A}}
        = \frac{D\,P_t}{R\,T\,Z}
        \,\ln\frac{P_t}{P_t-P_{A,1}}
    \end{BM}

    \begin{flalign*}
        &
        N_{A}
            = - C\,\mathscr{D}
            \,\odv{y_{A}}{z}
            +y_A(N_A+N_B)
            = - C\,\mathscr{D}
            \,\odv{y_{A}}{z}
            +y_A\,N_A
            \implies &\\&
            \implies
            \int_{0}^{Z}{N_{A}\,\odif{z}}
            = N_{A}\adif{Z}\big\vert_{0}^{Z}
            = N_{A}\,Z
            = &\\&
            = - \int_{y_{A,1}}^{y_{A,2}}{
                C\,\mathscr{D}
                \,\odif{y_{A}}
            }
            = - C\,\mathscr{D}
            \,\int_{y_{A,1}}^{y_{A,2}}{
                \frac{\odif{y_{A}}}{(1-y_A)}
            }
            % = &\\&
            = \frac{P}{R\,T}
            \,\mathscr{D}
            \,\int_{1-y_{A,1}}^{1-y_{A,2}}{
                \frac{\odif{(1-y_{A})}}{(1-y_A)}
            }
            = &\\&
            = -\frac{P\,\mathscr{D}}{R\,T}
            \,\int_{1-y_{A,2}}^{1-y_{A,1}}{
                \frac{\odif{(1-y_{A})}}{(1-y_A)}
            }
            % = &\\&
            = -\frac{P\,\mathscr{D}}{R\,T}
            \adif{\ln(1-y_A)}\big\vert_{y_{A,2}}^{y_{A,1}}
            % = &\\&
            = -\frac{P\,\mathscr{D}}{R\,T}
            \adif{\ln(1-y_A)}\big\vert_{y_{A,2}}^{y_{A,1}}
            = &\\&
            = \frac{P\,\mathscr{D}}{R\,T}
            \ln{\frac{
                1-y_{A,2}
            }{
                1-y_{A,1}
            }}
            = \frac{P\,\mathscr{D}}{R\,T}
            \ln{\frac{
                1-P_{A,2}/P
            }{
                1-P_{A,1}/P
            }}
            = \frac{P\,\mathscr{D}}{R\,T}
            \ln{\frac{
                P-P_{A,2}
            }{
                P-P_{A,1}
            }}
            = \frac{P\,\mathscr{D}}{R\,T}
            \ln{\frac{
                P
            }{
                P-P_{A,1}
            }}
            \implies &\\&
            \implies 
            \max{N_{A}}
            = \frac{D\,P_t}{R\,T\,Z}
            \,\ln\frac{P_t}{P_t-P_{A,1}}
        &
    \end{flalign*}
\end{questionBox}

\begin{questionBox}1{ % Q1
    Moldou-se naftaleno sob a forma de um cilindro de raon \(R_1\) que se deixou sublimar no ar em repouso.
} % Q1
    Mostre que a velocidade de sublimação é dada por:
    \begin{BM}
        Q
        = \frac{2\,\pi\,L\,\mathscr{D}\,P}{R\,T}
        \,\ln\frac{1-y_{A,2}}{1-y_{A,1}}
        \,\ln^{-1}\frac{R_2}{R_1}
    \end{BM}

    Sendo \(y_{A,*}\) a fração molar correspondente à presão de vapor do naftaleno e \(y_{A,2}\) a fração molar correspondente a \(R_2\)

    \begin{flalign*}
        &
            N_{A}
            = -C\,\mathscr{C}
            \,\odv{y_A}{r}
            + y_{A}(N_A+N_B)
            = -C\,\mathscr{C}
            \,\odv{y_A}{Z}
            + y_{A}(N_A+0)
            \implies &\\&
            \implies
            \int_{r_1}^{r_2}{
                N_{A}
                \,\odif{r}
            }
            = \int_{r_1}^{r_2}{
                \frac{Q}{2\,\pi\,r\,L}
                \,\odif{r}
            }
            % = &\\&
            = \frac{Q}{2\,\pi\,L}
            \,\int_{r_1}^{r_2}{
                \frac{\odif{r}}{r}
            }
            % = &\\&
            = \frac{Q}{2\,\pi\,L}
            \,\adif{\ln{r}}\big\vert_{r_1}^{r_2}
            = &\\&
            = \frac{Q}{2\,\pi\,L}\,\ln\frac{r_2}{r_1}
            = &\\&
            = -C\,\mathscr{D}
            \,\int_{y_{A,1}}^{y_{A,2}}{
                \frac{\odif{y_A}}{1-y_A}
            }
            = \frac{P}{R\,T}
            \,\mathscr{D}
            \,\ln\frac{1-y_{A,2}}{1-y_{A,1}}
            \implies &\\&
            \implies
            Q
            = \frac{P\,2\,\pi\,L\,\mathscr{D}}{R\,T}
            \,\ln\frac{1-y_{A,2}}{1-y_{A,1}}
            \,\ln^{-1}\frac{r_2}{r_1}
        &
    \end{flalign*}

    \begin{questionBox}3{ % Q1.0.1
        Explique o que sucede à velocidade de sublimação quando \(R_2\) se torna muito grade
    } % Q1.0.1
        \begin{flalign*}
            &
                \lim_{r_2\to\infty}{Q}
                = \lim_{r_2\to\infty}{
                    \frac{P\,2\,\pi\,L\,\mathscr{D}}{R\,T}
                    \,\ln\frac{1-y_{A,2}}{1-y_{A,1}}
                    \,\ln^{-1}\frac{r_2}{r_1}
                }
                = &\\&
                = \frac{P\,2\,\pi\,L\,\mathscr{D}}{R\,T}
                \,\ln\frac{1-y_{A,2}}{1-y_{A,1}}
                \,\lim_{r_2\to\infty}{
                    \ln^{-1}\frac{r_2}{r_1}
                }
                = 0
            &
        \end{flalign*}
    \end{questionBox}

    \begin{questionBox}2{ % Q1.1
        E se a geometria fosse esférica?
    } % Q1.1
        Mesma coisa só q com sup esférica
        quando r2 max q tende a um minimo
    \end{questionBox}
\end{questionBox}

\sisetup{
    % scientific / engineering / input / fixed
    exponent-mode           = engineering,
    exponent-to-prefix      = false,          % 1000 g -> 1 kg
    % exponent-product        = *,             % x * 10^y
    % fixed-exponent          = 0,
    round-mode              = places,        % figures/places/unsertanty/none
    round-precision         = 2,
    % round-minimum           = 0.01, % <x => 0
    output-exponent-marker  = {\,\mathrm{E}},
}

\begin{questionBox}1{ % Q2
    um tubo com 1\,\unit{\centi\metre} de diâmetro e 20\,\unit{\centi\metre} de comprimento está cehio com uma mistura de \ch{CO2} e \ch{H2} a uma pressão total de 2\,\unit{\atm} e a uma temperatura de 0\,\unit{\celsius} nessas condições é 0.275\,\unit{\centi\metre^2/\second}. Se a pressão parcial do \ch{CO2} for 1.5\,\unit{\atm} num dos lados do tubo é 0.5\,\unit{\atm} no outro extremo.
} % Q2
    Calcule a velocidade de difusão para:

    \begin{questionBox}3{ % Q2.0.1
        Contradifusão equimolar 
        (\(N_{\ch{CO2}} = - N_{\ch{H2}}\))
    } % Q2.0.1
        \begin{flalign*}
            &
                N_{\ch{CO2}}
                = -C\,\mathscr{D}
                \,\odv{y_{\ch{CO2}}}{z}
                + y_{\ch{CO2}}(
                    N_{CO2}
                    + N_{H2}
                )
                = -\frac{P}{R\,T}\,\mathscr{D}
                \,\odv{y_{\ch{CO2}}}{z}
                \implies &\\&
                \implies
                \int_{0}^{z}{
                    N_{\ch{CO2}}
                    \odif{z}
                }
                = N_{\ch{CO2}}
                \,\int_{0}^{z}{
                    \odif{z}
                }
                = N_{\ch{CO2}}\,Z
                = &\\&
                = -\int{
                    \frac{P\,\mathscr{D}}{R\,T}
                    \,\odif{y_{\ch{CO2}}}
                }
                = -\frac{P\,\mathscr{D}}{R\,T}
                (y_{\ch{CO2},2}-y_{\ch{CO2},1})
                = -\frac{P\,\mathscr{D}}{R\,T}
                \frac{(P_{\ch{CO2},2}-P_{\ch{CO2},1})}{P}
                \implies &\\&
                \implies
                N_{\ch{CO2}}
                = -\frac{P\,\mathscr{D}}{R\,T\,Z}
                \frac{(P_{\ch{CO2},2}-P_{\ch{CO2},1})}{P}
                = &\\&
                = -\frac{
                    2\E{5}
                    *0.275
                    *10^{-4}
                }{`'
                    \num{8.314462618}
                    *273.15
                    *20*10^{-2}
                }
                = &\\&
                = -\frac{
                    2*0.275
                }{
                    \num{8.314462618}
                    *273.15
                    *20
                }
                *10^{2}
                \frac{1.5-0.5}{2}
                \cong
                \num{6.05434699567257e-4}
            &
        \end{flalign*}
    \end{questionBox}

    \begin{questionBox}3{ % Q2.0.2
        A seguinte relação entre os fluxos 
        (\(N_{\ch{H2}} = -0.75\,N_{\ch{CO2}}\))
    } % Q2.0.2
        \begin{flalign*}
            &
                N_{\ch{CO2}}
            &
        \end{flalign*}
    \end{questionBox}
\end{questionBox}

\end{document}
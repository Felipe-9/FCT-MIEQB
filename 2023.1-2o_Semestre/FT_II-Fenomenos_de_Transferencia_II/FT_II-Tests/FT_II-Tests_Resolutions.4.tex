% !TEX root = ./FT_II-Tests_Resolutions.4.tex
\providecommand\mainfilename{"./FT_II-Tests_Resolutions.tex"}
\providecommand \subfilename{}
\renewcommand   \subfilename{"./FT_II-Tests_Resolutions.4.tex"}
\documentclass[\mainfilename]{subfiles}

% \tikzset{external/force remake=true} % - remake all

\begin{document}

% \graphicspath{{\subfix{./.build/figures/FT_II-Tests_Resolutions.4}}}
% \tikzsetexternalprefix{./.build/figures/FT_II-Tests_Resolutions.4/graphics/}

\mymakesubfile{4}
[FT II]
{Exame época especial} % Subfile Title
{Exame época especial} % Part Title

\begin{questionBox}1{ % Q1
    A reação de isomeriszaç!ão de \ch{A} (\ch{n A -> A_n}) ocorre na superfici de part de um cat.
} % Q1
    \begin{itemize}
        \item Sobre o cat existe uma camada estagnada de gás de espessura \(\delta\)
        \item Através de qual \(A\text{ e }A_n\) se difundem 
        \item e no exterior dessa camada a fração molar de \ch{A} é \(y_{A,0}\)
        \item Se a reação cinética for instantanea
        
    \end{itemize}
    Desenvolva expressões para a velocidade de isomerização de A em função das prop da fase gasosa envolvente.

    \begin{questionBox}2{ % Q1.1
        Para um cat com geometria plana com comprimento \textit{L} e largura \textit{W}
    } % Q1.1
        \begin{flalign*}
            &
                N_{A,z} = -n\,N_{A_n,z}
                &\\[3ex]&
                \begin{cases}
                    z_0 = \delta & y_{A,0}
                    \\
                    z_1 = L & y_{A,1} = 0
                \end{cases}
                &\\[3ex]&
                N_{A,z}\,A
                = \frac{c\,D_{A,B}}{\Theta\,\adif{z}}
                \,\ln\frac
                {1-\Theta\,y_{A,1}}
                {1-\Theta\,y_{A,0}}
                \,\left(
                    \pi\,(W/2)^2
                \right)
                = &\\&
                = \frac{c\,D_{A,B}}{(1+N_{A_n}/N_A)\,(L-\delta)}
                \,\ln\frac
                {1-(1+N_{A_n}/N_A)\,y_{A,1}}
                {1-(1+N_{A_n}/N_A)\,y_{A,0}}
                \,\pi\,W^2/4
                = &\\&
                = \frac{
                    \left(
                        \frac{p}{R\,T}
                    \right)\,D_{A,B}
                }{
                    (1-1/n)\,(L-\delta)
                }
                \,\ln\frac
                {1-(1-1/n)\,y_{A,1}}
                {1-(1-1/n)\,y_{A,0}}
                \,\pi\,W^2/4
                % ======================================== %
                % ======================================== %
                % ======================================== %
                % N_{A,z} = n\,N_{A_n,z}
                % &\\[3ex]&
                % N_{A,z}
                % = y_{A,z}(
                %     N_{A,z}+N_{B,z}
                % )
                % - \frac{D_{A,B}}
                % &\\[6ex]&
                % N_{A} = n\,N_{A_n}
                % ; &\\[3ex]&
                % N_{A,z}
                % =\frac{c\,D_{A,B}}{\Theta\,\adif{z}}
                % \,\ln\frac
                % {1-\Theta\,y_{A,1}}
                % {1-\Theta\,y_{A,0}}
                % = &\\&
                % = \frac{c\,D_{A,B}}{
                %     \left(
                %         1+N_{A_n}/N_{A}
                %     \right)
                %     (L-\delta)
                % }
                % \,\ln\frac
                % {1-\left(1+N_{A_n}/N_{A}\right)\,y_{A,1}}
                % {1-\left(1+N_{A_n}/N_{A}\right)\,y_{A,0}}
                % = &\\&
                % = \frac{c\,D_{A,B}}{
                %     \left(1+1/n\right)
                %     (L-\delta)
                % }
                % \,\ln\frac
                % {1-\left(1+1/n\right)\,y_{A,1}}
                % {1-\left(1+1/n\right)\,y_{A,0}}
            &
        \end{flalign*}
    \end{questionBox}

    \begin{questionBox}2{ % Q1.2
        Para um cat com geometria esférica, com raio \(R\)
    } % Q1.2
        \answer{}
        \begin{flalign*}
            &
                \begin{cases}
                    r_0=\delta & y_{A,0}
                    \\ 
                    r_1=R & y_{A,1}
                \end{cases}
                &\\[3ex]&
                N_{A,r}\,A
                = -\frac{c\,D_{A,B}}{\Theta\,r_0(1-r_0/r_1)}
                \,\ln\frac
                {1-\Theta\,y_{A,1}}
                {1-\Theta\,y_{A,0}}
                \,4\,\pi\,R^2
                = &\\&
                = -\frac{c\,D_{A,B}}{(1+N_{A_n}/N_{A})\,\delta(1-\delta/R)}
                \,\ln\frac
                {1-(1+N_{A_n}/N_{A})\,y_{A,1}}
                {1-(1+N_{A_n}/N_{A})\,y_{A,0}}
                = &\\&
                = -\frac{c\,D_{A,B}}{(1-1/n)\,\delta(1-\delta/R)}
                \,\ln\frac
                {1-(1-1/n)\,y_{A,1}}
                {1-(1-1/n)\,y_{A,0}}
            &
        \end{flalign*}
    \end{questionBox}
\end{questionBox}

%   ,ad8888ba,     ad888888b,
%  d8"'    `"8b   d8"     "88
% d8'        `8b          a8P
% 88          88       ,d8P"
% 88          88     a8P"
% Y8,    "88,,8P   a8P'
%  Y8a.    Y88P   d8"
%   `"Y8888Y"Y8a  88888888888

\begin{questionBox}1{ % Q2
    A permeabilidade de uma membrana de PDMS foi det exp a 30\,\unit{\celsius}.
} % Q2
    \begin{itemize}
        \item Sabendo que a representação de P em função do tempo, para tempos elevados é uma reta com declive igual a 20.
    \end{itemize}

    Determine a perm da membrana

    \begin{BM}
        p=\left(
            \frac{A\,R\,T\,p_0}{V\,L}
        \right)
        \left(
            S\,D_{A,B}\,t
            -S\,L^2/6
        \right)
    \end{BM}

    \begin{itemize}
        \item \(A=6*10^{-4}\,\unit{\metre^2}\)
        \item \(p_0=1*10^5\,\unit{\pascal}\)
        \item \(V=50\,\unit{\centi\metre^3}\)
        \item \(R=\qty{8.20573660809596e-5}{\metre^3.\atm.\mole^{-1}.\kelvin^{-1}}\)
        \item Espessura da membrana \(L=2\,\unit{\centi\metre}\)
    \end{itemize}

    \answer{}
    \begin{flalign*}
        &
            p
            =\left(
                \frac{
                    A
                    \,R
                    \,T
                    \,p_0
                }{
                    V\,L
                }
            \right)
            \left(
                S
                \,D_{A,B}
                \,t
                -S\,L^2/6
            \right)
            = &\\&
            =\left(
                \frac{
                    6*10^{-4}
                    *\num{8.20573660809596e-5}
                    *(30+273.15)
                    *p_0
                }{
                    (50\E-6)
                    *2\E-2
                }
            \right)
            \left(
                S
                \,D_{A,B}
                \,t
                -S\,L^2/6
            \right)
            ; &\\&
            \lim_{t>>}{p}
            \cong 20\,t
            \cong \frac{A\,R\,T\,S\,D_{A,B}}{V\,L}t
            \implies &\\[3ex]&
            \implies
            D_{A,B}
            \cong 
            \frac{20\,V\,L}{A\,R\,T\,S}
            = &\\&
            = \frac{
                20
                *(50\E-6)
                *(2\E-2)
            }{
                (6\E-4)
                *\num{8.20573660809596e-5}
                *(30+273.15)
                *(4\,\pi\,R^2)
            }
            = &\\&
            = \frac{
                20
                *(50\E-6)
                *(2\E-2)
            % 2000e-8
            }{
                (6\E-4)
                *\num{8.20573660809596e-5}
                *(30+273.15)
                *(4\,\pi\,R^2)
            }
            % \,\frac
            % {\Omega_{\mathscr{D},T_1}}
            % {\Omega_{\mathscr{D},T_2}}
        &
    \end{flalign*}
    
\end{questionBox}

%   ,ad8888ba,     ad888888b,
%  d8"'    `"8b   d8"     "88
% d8'        `8b          a8P
% 88          88       aad8"
% 88          88       ""Y8,
% Y8,    "88,,8P          "8b
%  Y8a.    Y88P   Y8,     a88
%   `"Y8888Y"Y8a   "Y888888P'

\begin{questionBox}1{ % Q3
    Efet transf de massa de ac benz a 20\,\unit{\celsius}
} % Q3
    \begin{itemize}
        \item Fazendo passar agua a velocidade média de 2.5\,\unit{\metre/\second} 
        \item através de um tubo de ac benz com 2\,\unit{\centi\metre} de diâmetro e 1\,\unit{\metre} de comprimento, 
        \item tendo-se obtido uma velocidade de dissolução de ácido benz de \(2.1\E-5\)\unit{\kilo\gram/\second}
    \end{itemize}

    \begin{questionBox}2{ % Q3.1
        Calcule a cencentração de ácdio benz à saída de um tubo de igual diâmetro, mas com 10\,\unit{\metre} de comprimento dse fizer passar água a uma vel quatro vezes superiror
    } % Q3.1
        \answer{}
        \begin{flalign*}
            &
                \begin{cases}
                    z_0=0 & y_{0}=0
                    \\
                    z_1=10 & y_{1}
                \end{cases}
                &\\&
                21\E-2=
                N_{A,z}\,A
                = \frac{c\,D_{A,B}}{\Theta\,\adif{z}}
                \,\ln\frac
                {1-\Theta\,y_{A,1}}
                {1-\Theta\,y_{A,0}}
                \,\pi\,(d/2)^2
                = &\\&
                = \frac{c\,D_{A,B}}{10}
                \,\ln{(1-y_{A,1})}
                \,\pi\,(d/2)^2
                = &\\&
                = \frac{
                    (2.65\E3)
                    *8.8\E{-10}
                }{10}
                \,\ln{(1-y_{A,1})}
                \,\pi\,(2\E-2/2)^2
                \implies &\\&
                \implies
                y_{A,1}
                = 1-\exp\left(
                    \frac{
                        21\E-2
                    }{
                        \frac{
                            (2.65\E3)
                            *8.8\E{-10}
                        }{10}
                        \,\pi\,(2\E-2/2)^2
                        % 0.000000000073262
                    }
                    % 2866426933.902060384139201
                \right)
                \cong
                % ======================================== %
                % ======================================== %
                % ======================================== %
                &\\[6ex]&
                \ln\frac
                {c_{A,s}-c_{A,0}}
                {c_{A,s}-c_{A,l}}
                = \frac{4\,k_c\,L}{d\,v}
                = &\\[3ex]&
                =
                \ln\frac
                {c_{A,s}-0}
                {c_{A,s}-\left(
                    2.1\E-2
                    \,\frac{\unit{\kilo\gram\of{Ac}}}{\unit{\second}}
                    \,\frac{\unit{\second}}{2.5\,\unit{\metre\of{z}}}
                    \,\pi\,(d/2)^2
                \right)}
                = \frac{4\,k_c\,L}{d\,v}
                ; &\\[3ex]&
                W
                =k_c\,\pi\,d\,l\,c_{*}
                \implies
                k_c
                =\frac{w}{\pi\,d\,l\,c_{*}}
                =\frac{w}{\pi\,d\,l\,y_{*}/A}
                =\frac
                {w\,4\,\pi\,r^2}
                {\pi\,d\,l\,y_{*}}
            &
        \end{flalign*}
    \end{questionBox}

    \begin{questionBox}2{ % Q3.2
        Que métodos poderá usar para calcular coeficientes de transferência de massa? quais os mais utilizados?
    } % Q3.2
        \answer{}
        Encontrar o coeficiente usando os dados experimentais e usar a proporcionalidade dentre coeficientes para transferir a uma nova situação variando pressão e temperatura
        \begin{BM}
            \frac
            {D_{A,B,T_2\,\unit{\kelvin},P_2\,\unit{\atm}}}
            {D_{A,B,T_1\,\unit{\kelvin},P_1\,\unit{\atm}}}
            = 
            \,\frac{P_1}{P_2}
            \,\left(
                \frac{T_2}{T_1}
            \right)^{3/2}
            % \,\frac
            % {\Omega_{\mathscr{D},T_1}}
            % {\Omega_{\mathscr{D},T_2}}
        \end{BM}
    \end{questionBox}
\end{questionBox}

\begin{questionBox}1{ % Q4
    Agua com cloro para branqueamento da pasta de papel é obtida por absorção de cloro gasoso em água numa coluna de enchimento a 293\,\unit{\kelvin} e 1\,\unit{\atm}.
} % Q4
    Num dado ponto da coluna a pressão de cloro no gás é de 75\,\unit{\mmHg} e a concentração de cloro é no liq é 0.75\,\unit{\gram/\litre}.

    Se 80\,\unit{\percent} da resistencia à transferencia de massa estiver na fase líquida, calcule:

    \begin{questionBox}2{ % Q4.1
        As composições interfaciais
    } % Q4.1
        \answer{}
        \begin{flalign*}
            &
                \begin{cases}
                    z_0 & y_{0,z}=0\,\unit{\gram/\litre}
                    \\
                    z_1=z & y_{1,z} = 75\,\unit{\mmHg}
                \end{cases}
            &
        \end{flalign*}
    \end{questionBox}
\end{questionBox}

\end{document}
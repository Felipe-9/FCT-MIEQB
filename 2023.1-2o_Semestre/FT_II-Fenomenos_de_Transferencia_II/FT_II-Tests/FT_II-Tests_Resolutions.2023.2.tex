% !TEX root = ./FT_II-Tests_Resolutions.2023.2.tex
\providecommand\mainfilename{"./FT_II-Tests_Resolutions.tex"}
\providecommand \subfilename{}
\renewcommand   \subfilename{"./FT_II-Tests_Resolutions.2023.2.tex"}
\documentclass[\mainfilename]{subfiles}

% \tikzset{external/force remake=true} % - remake all

\begin{document}

% \graphicspath{{\subfix{./figures/FT_II-Tests_Resolutions.2023.2}}}
% \tikzsetexternalprefix{./figures/FT_II-Tests_Resolutions.2023.2/graphics/}

\Large

\mymakesubfile{2}
[FT II]
{Test 2023.2 Resolution} % Subfile Title
{Test 2023.2 Resolution} % Part Title

\begin{questionBox}1{ % MARK: Q1
    Pretende-se limpar um tubo cilíndrico com \qty*{5}{\cm} de diâmetro e \qty*{120}{\m} de comprimento cuja superfície interior se encontra revestida de ácido benzóico. Para isso faz-se circular água a \qty*{25}{\celsius} no interior do tubo a uma velocidade \qty*{5}{\m/\s}.
} % Q1
    \paragraph*{Dados:}
    \begin{itemize}
        \item \(M(ac. benzoico) = \qty*{122}{\g/\mole}\)
        \item \(\mathscr{D}_{ac.ben,agua}=\qty*{1.0e-5}{\cm^2/\s}\)
        \item Solubilidade Ac Benzoico: \qty*{3e-3}{\g/\cm^3}
        \item \(C_f=7.9\E{-2}\,Re^{-.25}\)
        \item \(Sc=\mu/\rho\,\mathscr{D}_{A,B}=1000\)
        \item \(Re=\rho\,d\,V/\mu\)
        \item Analogia de Reynolds: \(k_C/V=C_f/2\)
        \item Analogia de Chilton-Coulburn: \(Sc^{2/3}\,k_C/v=C_f/2\)
        \item \(
            \ln{\frac
                {C_{A,S}-C_{A,0}}
                {C_{A,S}-C_{A,L}}
            }
            = \frac{4\,L}{d}
            \,\frac{k_C}{v}
        \)
        \item \(W=v\,(\pi\,d^2/4)(C_{A,L}-C_{A,0}); C_{A,S}=C^*\land v:\text{Velocidade}\)
    \end{itemize}
\end{questionBox}

\begin{questionBox}2{ % MARK: Q1.1
    Calcule o coeficiente de transferência de massa, escolhendo a analogia mais adequada. Justifique.
} % Q1.1
    \answer{}
    \begin{flalign*}
        &
            Sc=1000\neq1
            \implies
            \text{Chilton-Colburn é a mais adequada}
            % 
            % 
            % 
            &\\[3ex]&
            \implies
            k_C
            = \frac{C_f\,v}{2\,Sc^{2/3}}
            = 
            \frac{
                \left(
                    7.9\E{-2}
                    \,Re^{-.25}
                \right)
                \,v
            }{
                2*Sc^{2/3}
            }
            % = &\\&
            = 
            \frac{
                \, 7.9\E{-2}
                \left(
                    \frac{\rho\,d\,v}{\mu}
                \right)^{-.25}
                \, v
            }{
                2\,Sc^{2/3}
            }
            = &\\&
            = 
            \frac{
                \, 7.9\E{-2}
                \left(
                    \frac{
                        \left(
                            \frac{\mu}{D_A\,Sc}
                        \right)
                        \,d
                        \,v
                    }{
                        \mu
                    }
                \right)^{-.25}
                \, v
            }{
                2\,Sc^{2/3}
            }
            % = &\\&
            = 
            \frac{
                \, 7.9\E{-2}
                \left(
                    \frac{
                        d\,v
                    }{
                        D_A\,Sc
                    }
                \right)^{-.25}
                \, v
            }{
                2\,Sc^{2/3}
            }
            = &\\&
            = 
            \frac{
                \, 7.9\E{-2}
                \left(
                    \frac{
                        5
                        *5\E{2}
                    }{
                        1.0\E{-5}
                        *1\E{3}
                    }
                    % 4.472135954999579e-2
                \right)^{-.25}
                \, 5
            }{
                2\,(1\E{3})^{2/3}
            }
            \cong &\\&
            \cong\qty
            {8.832468511124169e-5}
            {\m/\s}
        &
    \end{flalign*}
\end{questionBox}

\begin{questionBox}2{ % MARK: Q1.2
    Calcule a percentagem de saturação da água à saída do tubo.
} % Q1.2
    \answer{}
    \begin{flalign*}
        &
            \mathemph{\%\sat}:&\\&
            \%\sat
            =\frac{C_{A,L}}{C^*}
            % 
            % 
            % 
            ; &\\[3ex]&
            \ln{\frac
                {C_{A,S}-C_{A,0}}
                {C_{A,S}-C_{A,L}}
            }
            =\ln{\frac
                {C^*-0}
                {C^*-\%\sat*C^*}
            }
            =-\ln{(
                1-\%\sat
            )}
            = &\\&
            = \frac{4\,L}{d}
            \,\frac{k_C}{v}
            % 
            % 
            % 
            \implies &\\[3ex]&
            \implies
            \%\sat
            = 1-\exp{\left(
                -\frac{4\,L}{d}
                \,\frac{k_C}{v}
            \right)}
            \cong &\\&
            \cong 
            1-\exp{\left(
                -\frac{4*120\E{2}}{5}
                \,\frac{
                    \num{8.832468511124169e-5}
                }{5}
                % 0.844016364451515
            \right)}
            \cong &\\&
            \cong\qty
            {15.5983635548485}
            {\percent}
        &
    \end{flalign*}
\end{questionBox}

\begin{questionBox}2{ % MARK: Q1.3
    Calcule a quantidade em \unit{\kg} de ácido benzóico removida durante a primeira hora do processo.
} % Q1.3
    \answer{}
    \begin{flalign*}
        &
            m
            = t\,W
            = t
            \,v\,(\pi\,d^2/4)(C_{A,L}-C_{A,0})
            = t
            \,v\,(\pi\,d^2/4)(\%\sat*C^*-0)
            \cong &\\&
            \cong
             3600
            *5\E{2}\,(\pi\,5^2/4)
            \,(
                \num{15.5983635548485e-2}
                *3\E{-2}
            ) 
            \cong &\\&
            \cong\qty
            {165.387502187780015758}
            {\kg}
        &
    \end{flalign*}
\end{questionBox}

\begin{questionBox}2{ % MARK: Q1.4
    Discuta as vantagens do uso de analogia no cálculo dos coeficientes de transferência de massa.
} % Q1.4
    \answer{}
    Permitem que com dados mais simples de se obter labortorialmente determinar o coeficiente de massa e/ou o calor.
\end{questionBox}

\begin{questionBox}1{ % MARK: Q2
    Num estudo de absorção de um composto A em água, realizado numa coluna de enchimento, obteve-se um coeficiente individual de transferência de massa para a fase líquida, \(k_L=\qty*{2e-5}{\m/\s}\) e verificou-se que, 10\% da resistência global é exercida pela fase líquida. Num determinado ponto da coluna a percentagem molar de A no ar é 15\% e a sua concentração molar no líquido é \qty*{0.01}{\mole/\dm^3}. A pressão total é \qty*{3}{\atm} e a constante de Henry é \qty*{0.5}{\atm} (\(P_A= H \,x_A\)). A concentração molar da água é \(C_L=18^{-1}\E{3}\,\unit{\mole/\dm^3}\).
} % Q2
    
\end{questionBox}

\begin{questionBox}2{ % MARK: Q2.1
    Determine o coeficiente global de transferência de massa baseado na fase líquida, \(K_L\) e o coeficiente individual de transferência de massa para a fase gasosa, \(k_G\).
} % Q2.1
    \answer{}
    \begin{flalign*}
        &
            \text{\emph{Coeff de trasnf de massa da fase líquida}}
            \,\mathemph{K_L}:&\\&
            \frac{K_L}{k_L}=0.1
            \implies
            K_L
            =0.1\,k_L
            =0.1*2\E{-5}
            =\qty*{2e-6}{\m/\s}
            % 
            % 
            % 
            &\\[3ex]&
            \text{\emph{Coeff de trasnf de massa da fase gasosa }}
            \,\mathemph{K_G}:&\\&
            K_G=0.9*k_g
            % 
            % 
            % 
            ; &\\[3ex]&
            K_L^{-1}
            =k_L^{-1}
            +(H'\,k_G)^{-1}
            =k_L^{-1}
            +\left(
                \frac{H}{C_L}
                \,k_G
            \right)^{-1}
            % 
            % 
            % 
            \implies &\\&
            \implies
            k_G
            =\left(
                K_L^{-1}
                -k_L^{-1}
            \right)^{-1}
            \frac{C_L}{H}
            % 
            % 
            % 
            \implies &\\[3ex]&
            \implies
            K_G
            % = &\\&
            = 0.9
            \,\left(
                K_L^{-1}
                -k_L^{-1}
            \right)^{-1}
            \frac{C_L}{H}
            = &\\&
            = 0.9\,\left(
                (2\E{-6})^{-1}
                -(2\E{-5})^{-1}
            \right)^{-1}
            \frac{18^{-1}\E{3}*10^{6}}{0.5}
            % 
            \,\unit{\frac{\mole}{\s\,\m^2\,\atm}}
            \cong&\\&
            \cong\qty
            {222.22222222222224}
            {\frac{\mole}{\s\,\m^2\,\atm}}
        &
    \end{flalign*}
\end{questionBox}

\begin{questionBox}2{ % MARK: Q2.2
    Determine o fluxo molar.
} % Q2.2
    \answer{}
    \begin{flalign*}
        &
            N_A
            =K_L
            \left(
                C_A^*-C_{A,L}
            \right)
            =K_L
            \left(
                \frac{P_A}{H'}
                -C_{A,L}
            \right)
            =K_L
            \left(
                \frac{P\,y_A}{H/C_L}
                -C_{A,L}
            \right)
            = &\\&
            =2\E{-6}
            \left(
                \frac{3*0.15}{0.5/18^{-1}\E{3}}
                -0.01
            \right)
            \,\unit{\frac{\m\,\mole}{\s\,\dm^3}}
            \cong &\\&
            \cong\qty
            {99.98}
            {\frac{\mole}{\s\,\m^2}}
        &
    \end{flalign*}
\end{questionBox}

\begin{questionBox}2{ % MARK: Q2.3
    Calcule as composições interfaciais no referido ponto da coluna.
} % Q2.3
    \answer{}
    \begin{flalign*}
        &
            \mathemph{C_{A,i}}&\\&
            N_A
            = K_L\,(C_{A,i}-C_{A,L})
            \implies &\\&
            \implies
            C_{A,i}
            = C_{A,L}
            + N_A/k_L
            \cong &\\&
            \cong
            \num{0.01}
            + \frac{\num{99.98}}{2\E{-5}}
            \cong &\\&
            \cong\qty
            {4.99900001e6}
            {\mole/\m^3}
            % 
            % 
            % 
            ; &\\[3ex]&
            \mathemph{P_{A,i}}&\\&
            N_A
            =k_G\,(P_A-P_{A,i})
            =k_G\,(P\,y_A-P_{A,i})
            \implies &\\&
            \implies
            P_{A,i}
            = P\,y_A
            - \frac{N_A}{k_G}
            \cong &\\&
            \cong
            3*0.15
            - \frac{\num{99.98}}{\num{222.22222222222224}}
            \cong &\\&
            \cong \qty
            {8.999999999995501e-5}
            {\atm}
        &
    \end{flalign*}
\end{questionBox}

\begin{questionBox}2{ % MARK: Q2.4
    No caso de ocorrer uma reação química irreversível de 2ª ordem explique em que condições o processo de transferência de massa é controlado pelo filme gasoso. Nessas condições qual seria o valor do fluxo nesse mesmo ponto da coluna? Comente.
} % Q2.4
    \answer{}
    \begin{flalign*}
        &
            N_{A}
            =k_G(P_{A,G}-P_{A,i})
            % 
            % 
            % 
            ; &\\[3ex]&
            \text{Reação de 2ª ordem\({}\land{}\)Reação rápida}
            % \implies &\\&
            \therefore
            P_{A,i}=0
            \implies &\\[3ex]&
            \implies
            N_{A}
            =k_G(P_{A,G}-0)
            =k_G(P_{G}\,y_A)
            \cong
            \num{222.22222222222224}
            (3*0.15)
            \,
            \unit{\frac{\mole}{\s\,\m^2}}
            \cong &\\&
            \cong\qty
            {100}
            {\frac{\mole}{\s\,\m^2}}
            % 
            % 
            % 
            ; &\\[3ex]&
            \frac
            {N_{A,\cancel{r}}}
            {N_{A,r}}
            = \frac
            {99.98}
            {100}
            = 99.98\,\%
        &
    \end{flalign*}
\end{questionBox}

\end{document}
% !TEX root = ./FT_II-Tests_Resolutions.2023.2.tex
\providecommand\mainfilename{"./FT_II-Tests_Resolutions.tex"}
\providecommand \subfilename{}
\renewcommand   \subfilename{"./FT_II-Tests_Resolutions.2023.2.tex"}
\documentclass[\mainfilename]{subfiles}

% \tikzset{external/force remake=true} % - remake all

\begin{document}

% \graphicspath{{\subfix{./figures/FT_II-Tests_Resolutions.2023.2}}}
% \tikzsetexternalprefix{./figures/FT_II-Tests_Resolutions.2023.2/graphics/}

\mymakesubfile{2}
[FT II]
{Test 2023.2 Resolution} % Subfile Title
{Test 2023.2 Resolution} % Part Title

\begin{questionBox}1{ % MARK: Q1
    Pretende-se limpar um tubo cilíndrico com \qty*{5}{\cm} de diâmetro e \qty*{120}{\m} de comprimento cuja superfície interior se encontra revestida de ácido benzóico. Para isso faz-se circular água a \qty*{25}{\celsius} no interior do tubo a uma velocidade \qty*{5}{\m/\s}.
} % Q1
    \paragraph*{Dados:}
    \begin{itemize}
        \item \(M(ac. benzoico) = \qty*{122}{\g/\mole}\)
        \item \(\mathscr{D}_{ac.ben,agua}=\qty*{1.0e-5}{\cm^2/\s}\)
        \item Solubilidade Ac Benzoico: \qty*{3e-3}{\g/\cm^3}
        \item \(C_f=7.9\E{-2}\,Re^{-.25}\)
        \item \(Sc=\mu/\rho\,\mathscr{D}_{A,B}=1000\)
        \item \(Re=\rho\,d\,V/\mu\)
        \item Analogia de Reynolds: \(k_C/V=C_f/2\)
        \item Analogia de Chilton-Coulburn: \(Sc^{2/3}\,k_C/V=C_f/2\)
        \item \(
            \ln{\frac
                {C_{A,S}-C_{A,0}}
                {C_{A,S}-C_{A,L}}
            }
            = \frac{4\,L}{d}
            \,\frac{k_C}{V}
        \)
        \item \(W=v\,(\pi\,d^2/4)(C_{A,L}-C_{A,0}); C_{A,S}=C^2\land v:\text{Velocidade}\)
    \end{itemize}
\end{questionBox}

\begin{questionBox}2{ % MARK: Q1.1
    Calcule o coeficiente de transferência de massa, escolhendo a analogia mais adequada. Justifique.
} % Q1.1
\end{questionBox}

\begin{questionBox}2{ % MARK: Q1.2
    Calcule a percentagem de saturação da água à saída do tubo.
} % Q1.2
\end{questionBox}

\begin{questionBox}2{ % MARK: Q1.3
    Calcule a quantidade em kg de ácido benzóico removida durante a primeira hora do processo.
} % Q1.3
\end{questionBox}

\begin{questionBox}2{ % MARK: Q1.4
    Discuta as vantagens do uso de analogia no cálculo dos coeficientes de transferência de massa.
} % Q1.4
\end{questionBox}

\end{document}
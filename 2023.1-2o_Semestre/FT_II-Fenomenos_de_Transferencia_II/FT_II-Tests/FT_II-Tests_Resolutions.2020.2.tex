% !TEX root = ./FT_II-Tests_Resolutions.2020.2.tex
\providecommand\mainfilename{"./FT_II-Tests_Resolutions.tex"}
\providecommand \subfilename{}
\renewcommand   \subfilename{"./FT_II-Tests_Resolutions.2020.2.tex"}
\documentclass[\mainfilename]{subfiles}

% \tikzset{external/force remake=true} % - remake all

\begin{document}

% \graphicspath{{\subfix{./figures/FT_II-Tests_Resolutions.2020.2}}}
% \tikzsetexternalprefix{./figures/FT_II-Tests_Resolutions.2020.2/graphics/}

\mymakesubfile{1}
[FT II]
{Test 2020.2 Resolution} % Subfile Title
{Test 2020.2 Resolution} % Part Title

\begin{questionBox}1{ % MARK: Q1
    Experiências de transferência de calor permitiram obter uma correlação para o coeficiente de transferência de calor, \textit{h}, para um cilindro de um composto A colocado numa corrente de ar:
    \begin{BM}
        Nu=0.43+0.532\,Re^{0.5}\,Pr^{0.33}
    \end{BM}
} % Q1
    \vspace{-6ex}
    \paragraph*{Dados:}
    \begin{align*}
        \qty*{1}{\atm}
        = \qty*{760}{\mmHg}
        = \qty*{1e5}{\Pa}
    \end{align*}\vspace{-5ex}
    \begin{align*}
        P*_A&=\qty*{400}{\mmHg}
        ;& R&=\qty{8.314462618}{\joule/\mole.\kelvin}
        ;& \mathscr{D}_{A-ar}&=\qty*{9e-6}{\m^2/\s}
        \\
        \rho&=\qty*{0.114}{\kg/\m^3}
        ;& \mu&=\qty*{2.1e-5}{\Pa.\s}
        ;& k& =\qty*{2.73e-2}{\W/\m.\K}
        \\
        &&C_p&=\qty*{1002}{\J/\kg.\K}
    \end{align*}\vspace{-3ex}
    \begin{align*}
        Nu&=\frac{h\,d}{k}
        ;&
        Pr&=\frac{\mu\,C_p}{k}
        ;&
        Sc&=\frac{\mu}{\rho\,\mathscr{D}}
        ;&
        Re&=\frac{\rho\,u\,d}{\mu}
    \end{align*}
    Analogia de Cholton--Colburn \(j_H=j_{\mathscr{D}}\)
    \begin{align*}
        \frac{h\,Pr^{2/3}}{\rho\,u\,C_p}
        = \frac{k_c\,Sc^{2/3}}{u}
    \end{align*}
    \paragraph*{Dados da b:}
    \begin{align*}
        u&=\qty*{3}{\m/\s}
        ;& d&=\qty*{1.5}{\cm}
        % \\
        ;& L&=\qty*{10}{\cm}
        ;& T&=\qty*{310}{\K}
    \end{align*}
\end{questionBox}

\begin{questionBox}2{ % MARK: Q1.1
    Utilizando a analogia de Chilton-Colburn calcule o coeficiente de transferência de massa.
} % Q1.1
    \answer{}
    \begin{flalign*}
        &
            \text{Pela analogia:}
            &\\&
            k_c
            = \frac{h}{\rho\,C_p}
            \,\left(
                \frac{Pr}{Sc}
            \right)^{2/3}
            % = &\\&
            = \frac{
                \left(
                    \frac{Nu\,k}{d}
                \right)
            }{
                \rho\,C_p
            }
            \,\left(
                \frac{
                    \left(
                        \frac{\mu\,C_p}{k}
                    \right)
                }{
                    \left(
                        \frac{\mu}{\rho\,\mathscr{D}}
                    \right)
                }
            \right)^{2/3}
            % = &\\&
            = \frac{
                Nu
            }{
                d
            }
            \sqrt[3]{
                \frac
                {k\,\mathscr{D}^2}
                {\rho\,C_p}
            }
            % 
            % 
            % 
            ; &\\[3ex]&
            Nu
            = 0.43+0.532\,Re^{0.5}\,Pr^{0.33}
            = 0.43
            +0.532
            \,\left(
                \frac{\rho\,u\,d}{\mu}
            \right)^{0.5}
            \,\left(
                \frac{\mu\,C_p}{k}
            \right)^{0.33}
            = &\\&
            = 0.43
            +0.532
            \,\mu^{0.33-0.5}
            \,\left(
                \rho\,u\,d
            \right)^{0.5}
            \,\left(
                \frac{C_p}{k}
            \right)^{0.33}
            = &\\&
            = 0.43
            +0.532
            \,(2.3\E{-5})^{-0.17}
            \,\left(
                0.114
                * 3
                * 1.5\E{-2}
            \right)^{0.5}
            \,\left(
                \frac{1002}{2.73\E{-2}}
            \right)^{0.33}
            \cong &\\&
            \cong\num
            {8.058081159215327}
            % 
            % 
            % 
            ; &\\[3ex]&
            \therefore
            k_c
            = \frac{
                Nu
            }{
                d
            }
            \sqrt[3]{
                \frac
                {k\,\mathscr{D}^2}
                {\rho\,C_p}
            }
            \cong &\\&
            \cong \frac{
                \num{8.058081159215327}
            }{
                1.5\E{-2}
            }
            \sqrt[3]{
                \frac
                {2.73\E{-2}*(9\E{-6})^2}
                {0.114*1002}
            }
            \cong\qty
            {1.44244327244535906e-2}
            {\m/s}
        &
    \end{flalign*}
\end{questionBox}

\begin{questionBox}2{ % MARK: Q1.2
    Calcule a velocidade de sublimação de um cilindro de A com \qty*{1.5}{\cm} de diâmetro e \qty*{10}{\cm} de comprimento. O ar a \qty*{310}{\K} tem uma velocidade de \qty*{3}{\m/\s}.
} % Q1.2
    \answer{}
    \begin{flalign*}
        &
            w
            =k_c\,A\,C^*
            =k_c
            \,\left(
                2*\pi\,d^2/4
                +L\,2\,\pi\,d/2
            \right)
            \left(
                \frac{P^*}{R\,T}
            \right)
            \cong &\\&
            \cong
            \num{1.44244327244535906e-2}
            *2\,\pi\,\E{-4}(1.5^2/4+10*1.5/2)
            % 73.071553155441563e-6
            \,\frac{400\E{5}/760}{\num{8.314462618}*310}
            % 0.155190135541862
            \cong &\\&
            \cong\qty
            {1.492103189269384132e-3}
            {\mole/\s}
        &
    \end{flalign*}
\end{questionBox}

\begin{questionBox}2{ % MARK: Q1.3
    Será válido, neste caso usar a analogia de Reynolds? Justifique. Discuta a importância da utilização de analogias no cálculo de coeficientes de transferência de massa.
} % Q1.3
    \answer{}
    \begin{flalign*}
        &
            Sc
            =\frac{u}{\rho\,\mathscr{D}}
            =\frac{3}{0.114*9\E{-6}}
            \cong\num
            {2.923976608187135e6}
            \neq 1
            \implies
        &
    \end{flalign*}
    Condições não conferem para a analogia de Reynolds.\\

    As analogias permitem que a partir de medidas fáceis de adquirir no laboratório, calcular o coef de transferencia de massa e/ou calor
\end{questionBox}

\begin{questionBox}2{ % MARK: Q1.4
    Como poderia aumentar a velocidade de sublimação?
} % Q1.4
    \answer{}
    \begin{flalign*}
        &
            w\propto
            k_c\,A\,C^*
        &
    \end{flalign*}
    Almentando qualquer dessas variáveis temos um almento da velocidade de sublimação
\end{questionBox}

\begin{questionBox}1{ % MARK: Q2
    Pretende-se remover \ch{SO2} de uma mistura gasosa constituída por \ch{SO2} e ar por absorção utilizando água. A coluna usada opera em contracorrente a \qty*{15}{\celsius} e \qty*{1}{\atm}. A linha de equilíbrio é \(y^*=10\,x\). A \% molar de \ch{SO2} no ar à entrada é 10\,\% e à saída é de 1\,\%. Os coeficientes individuais de transferência de massa são:
    \begin{BM}[align*]
        k_y&=\qty*{2}{\mole/\hour.\m^2}
        &
        k_x&=\qty*{20}{\mole/\hour.\m^2}
    \end{BM}
    Determine para o \emph{topo} da coluna
} % Q2
\end{questionBox}

\begin{questionBox}2{ % MARK: Q2.1
    As composições interfaciais
} % Q2.1
    \answer{}
    \begin{flalign*}
        &
            \mathemph{x_{A,i},y_{A,i}:}
            &\\&
            N_A
            =k_x(x_{A,i}-x_{A})
            =k_y(y_{A}-y_{A,i})
            =k_y(y_{A}-10*x_{A,i})
            \implies &\\&
            \implies
            x_{A,i}
            =\frac{
                k_y\,y_{A,2}
                +x_{A,2}\,k_x
            }{
                10\,k_y+k_x
            }
            =\frac{
                2*1\E{-2}
                +(0)\,20
            }{
                10*2+20
            }
            =0.05\,\%
            \implies &\\&
            \implies
            y_{A,i}
            =10*x_{A,i}
            =0.5\,\%
        &
    \end{flalign*}
\end{questionBox}

\begin{questionBox}2{ % MARK: Q2.2
    A \% da resistência total respitante a cada ma das fases.
} % Q2.2
    \answer{}
    \begin{flalign*}
        &
            \text{\% Rest fase gas e liq}
            &\\&
            \frac
            {y_{A}-y_{A,i}}
            {y_{A}-y^*_{A}}
            = \frac
            {0.01-0.5\,\%}
            {0.01-10*0}
            =0.5
        &
    \end{flalign*}
\end{questionBox}

\begin{questionBox}2{ % MARK: Q2.3
    Os coeficientes globais de trasnferência de massa \(K_y\text{ e }K_x\)
} % Q2.3
    \answer{}
    \begin{flalign*}
        &
            K_y
            =\text{(Resist gas)}*k_y
            =0.5*2
            =1
            ; &\\&
            K_x
            =\text{(Resist liq)}*k_x
            =0.5*20
            =10
        &
    \end{flalign*}
\end{questionBox}

\begin{questionBox}2{ % MARK: Q2.4
    O fluxo de \ch{SO2}
} % Q2.4
    \answer{}
    \begin{flalign*}
        &
            N_A
            =k_y(y_{A,2}-y_{A,1})
            =2(1\E{-2}-0.5\E{-2})
            = \qty*{1e-2}{\mole/\hour.\m^2}
        &
    \end{flalign*}
\end{questionBox}

\begin{questionBox}2{ % MARK: Q2.5
    O valor do fluxo quanto usar soluções de \ch{NaOH} com a concentração crítica de \ch{NaOH}. (Comente)
} % Q2.5
    \answer{}
    \begin{flalign*}
        &
            N_{A}
            = k_y(y_{A,2}-y_{A,1})
            = 2(1\E{-2}-0)
            = \qty*{2e-2}{\mole/\hour.\m^2}
            &\\[3ex]&
            y_{A,1}=0
            \quad\text{(Reação imediata na interfáce)}
        &
    \end{flalign*}
\end{questionBox}

\end{document}
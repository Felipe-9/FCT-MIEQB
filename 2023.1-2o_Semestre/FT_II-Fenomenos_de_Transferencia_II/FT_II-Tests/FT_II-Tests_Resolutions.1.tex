% !TEX root = ./FT_II-Tests_Resolutions.1.tex
\providecommand\mainfilename{"./FT_II-Tests_Resolutions.tex"}
\providecommand \subfilename{}
\renewcommand   \subfilename{"./FT_II-Tests_Resolutions.1.tex"}
\documentclass[\mainfilename]{subfiles}

% \tikzset{external/force remake=true} % - remake all

\begin{document}

% \graphicspath{{\subfix{./.build/figures/FT_II-Tests_Resolutions.1}}}

\mymakesubfile{1}
[FT II]
{Teste 1 Resolução} % Subfile Title
{} % Part Title

\begin{questionBox}1{ % Q1
    Carvão, Atm gasosa enriquecida (40\% percent molar de \ch{O2}) a 1400\,\unit{\kelvin}, à P atm (\(1.013*10^5\,\unit{\pascal}\)). Limit pela dif de \ch{O2} sentido oposto a \ch{CO} q se forma instant com carvão. Carvão = Esfera com \(d = 0.6\,\unit{\milli\metre}\) de carbono puro \(\rho=1280\,\unit{\kilo\gram\,\metre^{-3}}\).
} % Q1
    \begin{center}\Large
        \ch{2 C_{(s)} + O2\gas{} -> 2 CO\gas{}}
    \end{center}
    \begin{itemize}
        % \begin{multicols}{2}
            \item \(\chemalpha = \ch{O2}\)
            \item \(\chembeta = \ch{CO}\)
        % \end{multicols}
    \end{itemize}

    \paragraph{considere}
    \begin{itemize}
        \begin{multicols}{2}
            \item \(\mathscr{D}_{\ch{O2}-\text{mist gas}} = 10^{-4}\,\unit{\metre^2/\second}\)
            \item \(R = \qty{8.314462618}{\joule\,\mole^{-1}\,\kelvin^{-1}}\)
            \item para a) e d) estado estacionario
        \end{multicols}
    \end{itemize}

    \begin{questionBox}2{ % Q1.1
        Esquema, eq conserv de massa, condições fronteira
    } % Q1.1
        \begin{center}\large
            \ch{\text{Sup\ da\ Esfera\ de\ Carvão} <>[CO][O2] ATM}
        \end{center}

        \begin{flalign*}
            &
                Q_{\chembeta}
                = N_{\chembeta,r}*S_r
                = N_{\chembeta,r}*4\,\pi\,r^2
                = N_{\chembeta,r_1}*4\,\pi\,r_1^2
                \implies &\\&
                \implies
                N_{\chembeta,r}
                = N_{\chembeta,r_1}\,(r_1/r)^2
            &
        \end{flalign*}

        \begin{BM}[align*]
            \text{C. Fronteira Dif } \ch{CO}
            & \begin{cases}
                r=r_0 &\quad y_{\chembeta} = y_{\chembeta,*}
                \\
                r=\infty &\quad y_{\chembeta,0} = 0
            \end{cases}
            \\
            \text{C. Fronteira Dif } \ch{O2}
            & \begin{cases}
                r=\infty &\quad y_{\chemalpha} = y_{\chemalpha}
                \\
                r=r_0 &\quad y_{\chemalpha,0} = y_{\chemalpha,*}
            \end{cases}
            \\
            \text{C. Fronteira Reação}
            & \begin{cases}
                r=r_0 &\quad t=t_0
                \\
                r=r &\quad t=t
            \end{cases}
        \end{BM}
        Considero não haver CO na atmosfera indo de sua concentrção máxima na superficie para 0 em infinito e \ch{O2} tem sua máxima de 40\% no infinito e alguma mínima na superficie para que a reação ocorra
    \end{questionBox}

    \begin{questionBox}2{ % Q1.2
        Eq da vel de dif do \ch{O2} e valor da vel
    } % Q1.2
        \begin{flalign*}
            &
                N_{\chemalpha,r}
                = y_{\chemalpha}(
                    N_{\chemalpha,r}
                    + N_{\chembeta,r}
                ) - \frac{P\,\mathscr{D}_{\chemalpha,\chembeta}}{R\,T}
                \,\odv{y_a{\chemalpha}}{r}
            &
        \end{flalign*}
    \end{questionBox}

    \begin{questionBox}2{ % Q1.3
        Tempo até arder tudo
    } % Q1.3
        \begin{flalign*}
            &
                Q_{\chembeta}
                = -C_{\chembeta,L}\odv{V}{t}
                = -C_{\chembeta,L}\odv{(\pi\,r^3\,4/3)}{t}
                = -C_{\chembeta,L}\pi\,r^2\,4\,\odv{r}{t}
                = &\\[1.5ex]&
                = N_{\chembeta,r}\,S_r
                = N_{\chembeta,r}\,\pi\,r^2\,4
                \implies &\\&
                \implies
                N_{\chembeta,r} = -C_{\chembeta,L}\,\odv{r}{t}
                ; &\\[3ex]&
                % ------------------------------ N_{\chembeta,r} ----------------------------- %
                N_{\chembeta,r}
                = y_{\chembeta}(N_{\chembeta,r} + N_{\chemalpha,r})
                - \frac{P\,\mathscr{D}_{\chembeta,\chemalpha}}{R\,T}
                \,\odv{y_{\chembeta}}{r}
                \implies &\\&
                \implies
                N_{\chembeta,r}\odif{r}
                = \frac{
                    y_{\chembeta}\,N_{\chemalpha,r}
                    - \frac{P\,\mathscr{D}_{\chembeta,\chemalpha}}{R\,T}
                }{
                    1-y_{\chembeta}
                }\,\odif{y_{\chembeta}}
            &
        \end{flalign*}
    \end{questionBox}

\end{questionBox}


\end{document}
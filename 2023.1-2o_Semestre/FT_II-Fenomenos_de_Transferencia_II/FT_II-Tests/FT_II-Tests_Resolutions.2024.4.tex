% !TEX root = ./FT_II-Tests_Resolutions.2024.4.tex
\providecommand\mainfilename{"./FT_II-Tests_Resolutions.tex"}
\providecommand \subfilename{}
\renewcommand   \subfilename{"./FT_II-Tests_Resolutions.2024.4.tex"}
\documentclass[\mainfilename]{subfiles}

% \tikzset{external/force remake=true} % - remake all

\begin{document}

% \graphicspath{{\subfix{./figures/FT_II-Tests_Resolutions.2024.4}}}
% \tikzsetexternalprefix{./figures/FT_II-Tests_Resolutions.2024.4/graphics/}

\mymakesubfile{4}
[FT II]
{Exame Época Especial 2024.4 Resolução} % Subfile Title
{Exame Época Especial 2024.4 Resolução} % Part Title

\begin{questionBox}1{ % MARK: Q1
    \begin{itemize}
        \begin{multicols}{2}
            \item \qty*{1200}{\kelvin}
            \item \qty*{1}{\atm}
            \item Esta estacionario
            \item Superficie plana
            \item \(S=\qty*{1}{\m^2}\)
            \item Camada de carbono, espessura: \qty*{0.4}{\mm}
            \item Reação instantanea
            \item \ch{2 C\sld{} + O2\gas{} -> 2 CO\gas{}}
            \item Vel de queima lim pela difusão do \ch{O2}
            \item dif atravez de filme gasoso: \qty*{5}{\mm}
            \item \(y_{A,ar}=0.21\)
            \item \(\mathscr{D}_{A,ar}=\qty*{1e-4}{{\m^2/\s}}\)
            \item \(\rho_C=\qty*{1280}{\kilo\gram/\metre^3}\)
            \item \(M_{C}=\qty*{12}{\g/\mole}\)
            \item Assuma estado estacionário
        \end{multicols}
    \end{itemize}
} % Q1
\end{questionBox}

\begin{questionBox}2{ % MARK: Q1.1
    Velo de diff
} % Q1.1
    \answer{}
    \begin{flalign*}
        &
            Q_{O2}
            =S\,N_{O2,z}
            =S\,\left(
                \frac
                {C_{O2}\,\mathscr{D}_{A,B}}
                {\Theta\,\eta_d\,l}
                \ln{\frac
                    {1-\Theta\,y_{A,1}}
                    {1-\Theta\,y_{A,0}}
                }
            \right)
            = &\\&
            =S\,\left(
                \frac
                {\frac{P}{R\,T}\,\mathscr{D}_{A,B}}
                {(-1)\,1\,(L-Z_0)}
                \ln{\frac
                    {1-(-1)*(0)}
                    {1-(-1)\,y_{A,ar}}
                }
            \right)
            = &\\&
            =\mathemph{
                S\,\frac{P}{R\,T}
                \frac
                {\mathscr{D}_{A,B}}
                {L-Z_0}
                \ln{\frac
                    {1+y_{A,ar}}
                    {1}
                }
            }
            % 
            % 
            % 
            ;&\\[3ex]&
            \text{Condições de fronteira Fluxo:}&\\&
            \begin{cases}
                z_0= L ;&\quad y_{0}=y_{O2,ar}=0.21
                \\
                z_1= Z_0 ;&\quad y_{1}=0
                \quad\text{(Reação instantanea)}
            \end{cases}
            % 
            % 
            % 
            ;&\\[3ex]&
            \Theta
            =1+N_{CO}/N_{O2}
            =1+(-2\,N_{O2})/N_{O2}
            =-1
            % 
            % 
            % 
            ; &\\[1ex]&
            \eta_{d,\text{plano}}
            =1
        &
    \end{flalign*}
    \paragraph*{Nota:}
    \begin{itemize}
        \item \(Z_0\) Diametro inicial
        \item \(Z_1\) Diametro após a queima
        \item \(L\) Posição camada de filme gasoso
    \end{itemize}
\end{questionBox}

\begin{questionBox}2{ % MARK: Q1.2
    Calcule a velocidade da difusão
} % Q1.2
    \answer{}
    \begin{flalign*}
        &
            Q_{O2}
            = S
            \,\frac{P}{R\,T}
            \frac
            {\mathscr{D}_{A,B}}
            {L-Z_0}
            \ln{\frac
                {1+y_{A,ar}}
                {1}
            }
            =1
            \,\frac{1}{
                \num{8.20573660809596e-5}
                * 1200
            }
            \frac
            {1\E{-4}}
            {5\E{-3}}
            \ln{\frac
                {1+0.21}
                {1}
            }
            \cong\mathemph{
                \qty
                {3.871688972252645e-2}
                {\mole/\second}
            }
        &
    \end{flalign*}
\end{questionBox}

\begin{questionBox}2{ % MARK: Q1.3
    O que acontece com o dif do O2 se queimar a 1500\,\unit{K}
} % Q1.3
    \answer{}
    \begin{flalign*}
        &
            Q_{O2,z}
            \propto \mathscr{D}_{O2,ar}\,T^{-1}
            \propto T^{3/2-1}
            \implies
            Q_{O2,z,1500K}
            =Q_{O2,z,1200K}
            \,\left(
                \frac{1500}{1200}
            \right)^{1/2}
            \cong &\\&
            \cong
            \num{3.871688972252645e-2}
            \,\left(
                \frac{1500}{1200}
            \right)^{1/2}
            \cong\mathemph{
                \qty
                {4.328679864846606e-2}
                {\mole/\s}
            }
        &
    \end{flalign*}
\end{questionBox}

\begin{questionBox}2{ % MARK: Q1.4
    Considerando dif pseudo estacionário, quanto tempo demora para arder a placa
} % Q1.4
    \answer{}
    \begin{flalign*}
        &
            C_{C}\,\odv{\vol}{t}
            = \frac{\rho_C}{M_C}\,\odv{(S*z)}{t}
            = \frac{\rho_C}{M_C}\,S\,\odv{z}{t}
            = &\\[3ex]&
            = -Q_{O2}
            = -S\,\frac{P}{R\,T}
            \frac
            {\mathscr{D}_{A,B}}
            {L-z}
            \ln{(1+y_{A,ar})}
            % 
            % 
            % 
            \implies &\\[3ex]&
            \implies
            \int_{Z_0}^{Z_1}{
                (L-z)\odif{z}
            }
            = -\int_{Z_0}^{Z_1}{
                (L-z)\odif{(L-z)}
            }
            = -((L-Z_1)^2-(L-Z_0)^2)/2
            % 
            % 
            % 
            = &\\[3ex]&
            = \int_{0}^{t}{
                \left(
                    -
                    \frac{M_C}{\rho_C}
                    \,\frac{P}{R\,T}
                    \,\mathscr{D}_{A,B}
                    \,\ln{(1+y_{A,ar})}
                \right)
                \,\odif{t}
            }
            = 
            \left(
                -
                \frac{M_C}{\rho_C}
                \,\frac{P}{R\,T}
                \,\mathscr{D}_{A,B}
                \,\ln{(1+y_{A,ar})}
            \right)
            \int_{0}^{t}{
                \,\odif{t}
            }
            = &\\&
            = 
            \left(
                -
                \frac{M_C}{\rho_C}
                \,\frac{P}{R\,T}
                \,\mathscr{D}_{A,B}
                \,\ln{(1+y_{A,ar})}
            \right)
            t
            % 
            % 
            % 
            \implies &\\[3ex]&
            \implies
            t
            =\frac{(L-Z_1)^2-(L-Z_0)^2}{
                2
                \,\frac{M_C}{\rho_C}
                \,\frac{P}{R\,T}
                \,\mathscr{D}_{A,B}
                \,\ln{(1+y_{A,ar})}
            }
            % \cong  &\\&
            \cong \frac{(5\E{-3}+0.4\E{-3})^2-(5\E{-3})^2}{
                2
                \,\frac{12}{1280\E3}
                \,\frac{1}{
                    \num{8.20573660809596e-5}
                    * 1200
                }
                * 1\E{-4}
                \,\ln{(1+0.21)}
            }
            \cong &\\&
            \cong\mathemph{
                \qty
                {1146.0975726961823223984}
                {\second}
            }
            \cong\mathemph{
                \qty*{19}{\minute}
                \,\qty
                {6.097572696182322}
                {\second}
            }
            % 
            % 
            % 
            ; &\\[3ex]&
            \text{Condições de fronteira Fluxo:}&\\&
            \begin{cases}
                z_0= L ;&\quad y_{0}=y_{O2,ar}=0.21
                \\
                z_1= z ;&\quad y_{1}=0
                \quad\text{(Reação instantanea)}
            \end{cases}
            % 
            % 
            % 
            ; &\\[1ex]&
            \text{Condições de fronteira Reação:}&\\&
            \begin{cases}
                t_0=0 ;&\quad z_0=Z_0
                \\
                t_1=t ;&\quad z_1=Z_1=Z_0-0.4\E{-3}
                \quad\text{(Camada de carbono)}
            \end{cases}
            % 
            % 
            % 
            ;&\\[3ex]&
            \Theta
            =1+N_{CO}/N_{O2}
            =1+(-2\,N_{O2})/N_{O2}
            =-1
            % 
            % 
            % 
            ; &\\[1ex]&
            \eta_{d,\text{plano}}
            =1
        &
    \end{flalign*}
\end{questionBox}

\begin{questionBox}1{ % MARK: Q2
    \begin{itemize}
        % \begin{multicols}{2}
            \item Reação de eq: \(y_A=0.75\,x_A\)
            \item Em um poonto da coluna
            \begin{itemize}
                \item Fase liq contem 90\% de A (base molar)
                \item Fase gas contem 45\% de A (base molar)
                \item Coeff indiv de transf de massa na fase gasosa \(k_y=\qty*{2}{\mole/\hour\,\m^2}\)
                \item Resistencia de 70\% devido ao filme gasoso
            \end{itemize}
        % \end{multicols}
    \end{itemize}
} % Q2
\end{questionBox}

\begin{questionBox}2{ % MARK: Q2.1
    Coeff glob de trasnferencia de massa \(K_y\)
} % Q2.1
    \answer{}
    \begin{flalign*}
        &
            K_y
            =\text{(Resist gas)}*k_y
            =0.7*2
            =\mathemph{
                \qty*{1.4}{\mole/\hour\,\m^2}
            }
        &
    \end{flalign*}
\end{questionBox}

\begin{questionBox}2{ % MARK: Q2.2
    Fluxo molar de A nesse ponto (comentar)
} % Q2.2
    \answer{}
    \begin{flalign*}
        &
            N_A
            =K_L
            \left(
                C_A^*-C_{A,L}
            \right)
            =K_L
            \left(
                \frac{P_A}{H'}
                -C_{A,L}
            \right)
            =K_L
            \left(
                \frac{P\,y_A}{H/C_L}
                -C_{A,L}
            \right)
        &
    \end{flalign*}
\end{questionBox}

\begin{questionBox}2{ % MARK: Q2.3
    Composição interfacial nas duas fases
} % Q2.3
    \answer{}
    \begin{flalign*}
        &
            \mathemph{C_{A,i}}&\\&
            N_A
            = K_L\,(C_{A,i}-C_{A,L})
            \implies &\\&
            \implies
            C_{A,i}
            = C_{A,L}
            + N_A/k_L
            % 
            % 
            % 
            ; &\\[3ex]&
            \mathemph{P_{A,i}}&\\&
            N_A
            =k_G\,(P_A-P_{A,i})
            =k_G\,(P\,y_A-P_{A,i})
            \implies &\\&
            \implies
            P_{A,i}
            = P\,y_A
            - \frac{N_A}{k_G}
        &
    \end{flalign*}
\end{questionBox}

\begin{questionBox}2{ % MARK: Q2.4
    Será o composto muito solúvel? Justifique
} % Q2.4
    \answer{}
    Um composto é considerado muito solúvel quando tem consideravelmente grande \(k_G\) dessa forma facilitando a trasnferencia
\end{questionBox}

\end{document}
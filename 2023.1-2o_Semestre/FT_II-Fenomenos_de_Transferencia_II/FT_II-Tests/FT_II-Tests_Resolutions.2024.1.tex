% !TEX root = ./FT_II-Tests_Resolutions.2024.1.tex
\providecommand\mainfilename{"./FT_II-Tests_Resolutions.tex"}
\providecommand \subfilename{}
\renewcommand   \subfilename{"./FT_II-Tests_Resolutions.2024.1.tex"}
\documentclass[\mainfilename]{subfiles}

% \tikzset{external/force remake=true} % - remake all

\begin{document}

% \graphicspath{{\subfix{./.build/figures/FT_II-Tests_Resolutions.2024.1}}}
% \tikzsetexternalprefix{./.build/figures/FT_II-Tests_Resolutions.2024.1/graphics/}

\mymakesubfile{1}
[FT II]
{Teste 2024.1 Resolução} % Subfile Title
{Teste 2024.1 Resolução} % Part Title

\begin{questionBox}1{ % MARK: Q1
    \begin{itemize}
        \begin{multicols}{2}
            \item Altura: \(h=\qty*{30}{\centi\metre}\)
            \item Diametro: \(d=\qty*{1}{\centi\metre}\)
            \item Altura liquido A: \(h_0=\qty*{10}{\centi\metre}\)
            \item \(y_{A,h}=0\)
        \end{multicols}
    \end{itemize}
} % Q1
    \paragraph*{Dados a \qty*{25}{\celsius}}
    \begin{itemize}
        \begin{multicols}{2}
            \item \(D_{A,ar}=\qty*{2e-5}{\metre^2/\second}\)
            \item \(R=\qty*{8.314462618}{\joule.\mole^{-1}.\kelvin^{-1}}\)
            \item \(P_A^*=\qty*{80}{\mmHg}\)
            \item \(P=\qty*{1}{\atm}=\qty*{80}{\mmHg}=\qty*{10E5}{\pascal}\)
            \item \(M_A=\qty*{30}{\gram/\mole}\)
            \item \(\rho_A=\qty*{0.9E3}{\kilo\gram/\metre^3}\)
        \end{multicols}
    \end{itemize}

    \begin{questionBox}2{ % MARK: Q1.1
        Expressão para evaporação completa em função do tempo
    } % Q1.1
        \answer{}
        \begin{flalign*}
            &
                \text{Evaporação em geometria linear:}
                &\\&
                N_{A,z}
                = Q_A
                = &\\&
                = -C_{A,ar}\,\odv{V}{t}
                = -C_{A,ar}\,\odv{\pi\,(d/2)^2\,z}{t}
                = -C_{A,ar}\,\pi\,(d/2)^2\,\odv{z}{t}
                % 
                % 
                % 
                \implies &\\[3ex]&
                \implies
                \int_{0}^{t}{N_{A,z}\,\odif{t}}
                = N_{A,z}\,\int_{0}^{t}{\odif{t}}
                = N_{A,z}\,t
                % 
                % 
                % 
                = &\\&
                = \int_{h/3}^{0}{- C_{A,ar}\,\pi\,d^2\,\odif{z}/4}
                = - \frac{C_{A,ar}\,\pi\,d^2}{4}
                \,(-h/3)
                % 
                % 
                % 
                \implies &\\[3ex]&
                \implies
                N_{A,z}
                = \frac{C_{A,ar}\,\pi\,d^2}{12}\,h/t
                % 
                % 
                % 
                &\\[3ex]&
                \text{Condições de fronteira para fluxo:}
                &\\&
                \begin{cases}
                    z=h_0 & y_{A}=y_A^*
                    \\
                    z=h & y_{A}=0
                \end{cases}
                % 
                % 
                % 
                ; &\\[3ex]&
                \text{Fluxo molar:}
                &\\&
                N_A
                = y_{A,z}\,N_{A}
                - C\,\mathscr{D}_{A,ar}
                \,\odv{y_{A,z}}{z}
                \implies &\\[3ex]&
                \implies
                % 
                % 
                % 
                \int{N_A\,\odif{z}}
                = N_A\,(h-h_0)
                % 
                % 
                % 
                = &\\&
                = \int_{y_{A}^*}^{0}{
                    -C\,\mathscr{D}_{A,ar}
                    \,\frac{\odif{y_{A,z}}}{1-y_{A,z}}
                }
                = &\\&
                = C\,\mathscr{D}_{A,ar}
                \,\int_{y_{A}^*}^{0}{
                    \frac{\odif{(1-y_{A,z})}}{1-y_{A,z}}
                }
                = &\\&
                = C\,\mathscr{D}_{A,ar}
                \,\ln(1-y_{A,z}^*)
                \implies &\\&
                \implies
                N_A
                = \frac{C\,\mathscr{D}_{A,ar}}{h-h_0}
                \,\ln(1-y_{A,z}^*)
            &
        \end{flalign*}
    \end{questionBox}
    \begin{questionBox}2{ % MARK: Q1.2
        Calcule esse tempo
    } % Q1.2
    \end{questionBox}
    \begin{questionBox}2{ % MARK: Q1.3
        Novo tempo para metade da altura e dobro do diametro, comente
    } % Q1.3
    \end{questionBox}
\end{questionBox}

\begin{questionBox}1{ % MARK: Q2
    \begin{itemize}
        \begin{multicols}{2}
            \item \(d=\qty*{1}{\centi\metre}\)
            \item \(T=\qty*{1500}{\kelvin}\)
            \item \(P=\qty*{1}{\atm}\)
            \item Velocidade limit pela dif do \ch{O2}
            \item \(M=\qty*{1280}{\kilo\gram/\metre^3}\)
            \item \(\mathscr{D}_{\ch{O2},ar}=\qty*{1e-4}{\metre^2/\second}\)
        \end{multicols}
        \begin{center}\Large
            \ch{3 C + 2 O2 -> 2 CO + CO2}
        \end{center}
    \end{itemize}
} % Q2
    \begin{questionBox}2{ % MARK: Q2.1
        Expr: vel de consumo de \ch{O2} e cond fronteira
    } % Q2.1
        \answer{}
        \begin{flalign*}
            &
                N_{O2,r}
                = &\\&
                = y_{O2,r}(N_C+N_O2 + N_{CO} + N_{CO2})
                - C\,\mathscr{D}_{O2,ar}
                \,\odv{y_{O2,r}}{r}
                = &\\&
                = y_{O2,r}(N_{O2}/3+N_{O2}/2 - N_{O2}/2 - N_{O2})
                - C\,\mathscr{D}_{O2,ar}
                \,\odv{y_{O2,r}}{r}
                = &\\&
                = y_{O2,r}\,N_{O2}
                \,(-2/3)
                - C\,\mathscr{D}_{O2,ar}
                \,\odv{y_{O2,r}}{r}
                % 
                % 
                % 
                \implies &\\[3ex]&
                \text{Condições de fronteira}
                &\\&
                \begin{cases}
                    r=R & y_{O2}=0
                    \\
                    r=\infty & y_{O2}=y_{O2}^*
                \end{cases}
                &\\&
                \implies
                \int_{R}^{\infty}{
                    N_{O2}\,\odif{r}
                }
                = \int_{R}^{\infty}{
                    \frac{Q_{O2}}{4\,\pi\,r^2}
                    \,\odif{r}
                }
                = \frac{Q_{O2}}{4\,\pi}
                \,\int_{R}^{\infty}{
                    \frac{\odif{r}}{r^2}
                }
                = &\\&
                = \frac{Q_{O2}}{4\,\pi}
                \,(R^{-1}-0)
                = \frac{Q_{O2}}{4\,\pi\,R}
                % 
                % 
                % 
                = &\\[3ex]&
                = - C\,\mathscr{D}_{O2,ar}
                \,\frac{\odif{y_{O2,r}}}{1+y_{O2,r}2/3}
                = - \frac{C\,\mathscr{D}_{O2,ar}}{2/3}
                \,\frac{\odif{(1+y_{O2,r}2/3)}}{1+y_{O2,r}2/3}
                = &\\&
                = - \frac{C\,\mathscr{D}_{O2,ar}}{2/3}
                \ln(1+y_{O2,r}^*2/3)
                % 
                % 
                % 
                \implies &\\[3ex]&
                \implies
                Q_{O2}
                = - \frac{4\,\pi\,R\,C\,\mathscr{D}_{O2,ar}}{2/3}
                \ln(1+y_{O2,r}^*2/3)
                = &\\&
                = - \frac{
                    4\,\pi\,R\,P
                    \,\mathscr{D}_{O2,ar}
                }{
                    R\,T\,2/3
                }
                \ln(1+y_{O2,r}^*2/3)
            &
        \end{flalign*}
    \end{questionBox}
    \begin{questionBox}2{ % MARK: Q2.2
        Vel de consumo de C
    } % Q2.2
    \end{questionBox}
\end{questionBox}

\end{document}
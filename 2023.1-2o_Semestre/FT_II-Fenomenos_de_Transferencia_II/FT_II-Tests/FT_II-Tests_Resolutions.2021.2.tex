% !TEX root = ./FT_II-Tests_Resolutions.2021.2.tex
\providecommand\mainfilename{"./FT_II-Tests_Resolutions.tex"}
\providecommand \subfilename{}
\renewcommand   \subfilename{"./FT_II-Tests_Resolutions.2021.2.tex"}
\documentclass[\mainfilename]{subfiles}

% \tikzset{external/force remake=true} % - remake all

\begin{document}

% \graphicspath{{\subfix{./figures/FT_II-Tests_Resolutions.2021.2}}}
% \tikzsetexternalprefix{./figures/FT_II-Tests_Resolutions.2021.2/graphics/}

\mymakesubfile{2}
[FT II]
{Teste 2021.2 Resolução} % Subfile Title
{Teste 2021.2 Resolução} % Part Title

\begin{questionBox}1{ % MARK: Q1
    Ar a \qty*{10}{\celsius} e \qty*{1}{\atm} move-se a uma velocidade de \qty*{30}{\metre/\second} paralelamente a uma placa quadrada de naftaleno com \qty*{3}{\milli\metre} de espessura e \qty*{40}{\centi\metre} de lado.
} % Q1
    \def\naft{\text{naft}}
    \def\ar{\text{ar}}
    \paragraph*{Dados:}
    \begin{itemize}
        \item \(Sh_x=0.3320\,Re_x^{0.5}\,Sc^{0.33}\) Regime \emph{laminar}
        \item \(Sh_x=0.0292\,Re_x^{0.8}\,Sc^{0.33}\) Regime \emph{turbulento}
        \begin{multicols}{2}
            \item \(Re_{x,c}=x_c\,u/v=32\E{5}\)
            \item \(\mathscr{D}_{\naft,\ar}=\qty*{0.051}{\centi\metre^{2}/\second}\)
            \item \(\rho_{\naft}=\qty{1.2}{\gram/\centi\metre^3}\)
            \item \(M_{\naft}=\qty*{128}{\gram/\mole}\)
            \item \(P^*_{\naft}=\qty{1e-3}{\atm}\)
        \end{multicols}
        \begin{multicols}{2}
            \item \(R=\qty{8.314462618}{\joule.\mole^{-1}.\kelvin^{-1}}\)
            \item \(\qty*{1}{\atm}=\qty*{1e5}{\Pa}\)
            \item \(Sh_x=k_C\,x/\mathscr{D}\)
            \item \(Re_x=x\,u/v\)
            \item \(Sc=v/\mathscr{D}=2.57\)
        \end{multicols}
    \end{itemize}
\end{questionBox}

\begin{questionBox}2{ % MARK: Q1.1
    Por quanto tempo deverá ser a placa exposta ao ar de modo a desaparecer completamente? (4v)
} % Q1.1
    \def\naft{\text{naft}}
    \answer{}
    \begin{flalign*}
        &
            t 
            = \frac{n}{W}
            = \frac{
                Vol_{\naft}\,\frac{\rho_{\naft}}{M_{\naft}}
            }{
                \bar{k_C}\,(C_{A,S}-C_{A,\infty})\,A
            }
            = \frac{
                L^2\,D
                \,\frac{\rho_{\naft}}{M_{\naft}}
            }{
                \bar{k_C}
                \,\left(
                    \left(
                        \frac{n}{V}
                    \right)
                    -0
                \right)\,L^2
            }
            = \frac{
                D\,\frac{\rho_{\naft}}{M_{\naft}}
            }{
                \bar{k_C}\,
                \left(\frac{P}{R\,T}\right)
            }
            % 
            % 
            % 
            % ; &\\[3ex]&
            % \text{\emph{Quantidade de Naftaleno}:}
            % n
            % = Vol_{\naft}\,\frac{\rho_{\naft}}{M_{\naft}}
            % = 40^2\,3\E{-1}
            % \,\unit{\centi\metre^3}
            % \,\frac
            % {\qty*{1.200}{\gram/\centi\metre^3}}
            % {\qty*{128}{\gram/\mole}}
            % =
            % \qty*{4.5}{\mole\of{\naft}}
            % 
            % 
            % 
            ; &\\[3ex]&
            \bar{k_C}
            = \frac{
                \int_0^L{k_C\,\odif{x}}
            }{
                \int_0^L{\odif{x}}
            }
            = \frac{
                \int_0^{x_c}{k_{C,L}\,\odif{x}}
                + \int_{x_c}^{L}{k_{C,T}\,\odif{x}}
            }{
                L
            }
            = &\\&
            = \frac{1}{L}
            \left(
                \int_0^{x_c}{
                    \frac{\mathscr{D}}{x}
                    \,Sh_{x,L}
                    \,\odif{x}
                }
                + \int_{x_c}^{L}{
                    \frac{\mathscr{D}}{x}
                    \,Sh_{x,T}
                    \,\odif{x}
                }
            \right)
            = &\\&
            = \frac{1}{L}
            \left(
                \int_0^{x_c}{
                    \mathscr{D}
                    \,0.332
                    \,Re_x^{0.5}
                    \,Sc^{0.33}
                    \,\frac{\odif{x}}{x}
                }
                + \int_{x_c}^{L}{
                    \mathscr{D}
                    \,0.0292
                    \,Re_x^{0.8}
                    \,Sc^{0.33}
                    \,\frac{\odif{x}}{x}
                }
            \right)
            = &\\&
            = \frac{1}{L}
            \left(
                \begin{aligned}
                    &
                    \mathscr{D}
                    \,0.332
                    \,Sc^{0.33}
                    \int_0^{x_c}{
                        \,\left(
                            \frac{x\,u}{v}
                        \right)^{0.5}
                        \,\frac{\odif{x}}{x}
                    }
                    &+\\+&
                    \mathscr{D}
                    \,0.0292
                    \,Sc^{0.33}
                    \int_{x_c}^{L}{
                        \left(
                            \frac{x\,u}{v}
                        \right)^{0.8}
                        \,\frac{\odif{x}}{x}
                    }
                \end{aligned}
            \right)
            = &\\&
            = \frac{\mathscr{D}\,Sc^{0.33}}{L}
            \left(
                \begin{aligned}
                    &
                    0.332
                    \,\frac{u^{0.5}}{v^{0.5}}
                    \int_0^{x_c}{
                        x^{-0.5}\,\odif{x}
                    }
                    &+\\+&
                    0.0292
                    \,\frac{u^{0.8}}{v^{0.8}}
                    \int_{x_c}^{L}{
                        x^{-0.2}\,\odif{x}
                    }
                \end{aligned}
            \right)
            = &\\&
            = \frac{
                0.051\E{-4}
                * 2.57^{0.33}
            }{  0.4 }
            \left(
                \begin{aligned}
                    &
                    0.332
                    \,\frac{(3.2\E{5})^{0.5}}{(0.14)^{0.5}}
                    \,x_c^{0.5}
                    &+\\+&
                    0.0292
                    \,\frac{u^{0.8}}{v^{0.8}}
                    \left(
                        L^{0.8}
                        - x_c^{0.8}
                    \right)
                    &
                \end{aligned}
            \right)
            ; &\\[3ex]&
            \mathemph{x_c}
            &\\&
            \frac{x_c\,u}{v}
            = \frac{x_c\,u}{Sc\,\mathscr{D}}
            = Re_{x_c}
            \implies &\\&
            \implies
            x_c
            = \frac{Re\,Sc\,\mathscr{D}}{u}
        &
    \end{flalign*}
\end{questionBox}

\begin{questionBox}2{ % MARK: Q1.2
    Determine o valor do coeficiente de transferência de massa a uma distância de \qty*{5}{\centi\metre} do início da
    placa. (2v)
} % Q1.2
    \answer{}
    A \qty*{5}{\centi\metre} acida da placa o regime é laminar
    \begin{flalign*}
        &
            \frac{k_C\,x}{\mathscr{D}}
            = Sh_x
            = 0.332
            \,Re_x^{0.5}
            \,Sc^{0.33}
            = 0.332
            \,\left(
                \frac{x\,u}{v}
            \right)^{0.5}
            \,Sc^{0.33}
            \implies &\\[3ex]&
            \implies
            k_C
            = \frac{\mathscr{D}}{x}
            \,0.332
            \,\left(
                \frac{x\,u}{v}
            \right)^{0.5}
            \,Sc^{0.33}
        &
    \end{flalign*}
\end{questionBox}
\begin{questionBox}2{ % MARK: Q1.3
    De acordo com a teoria do filme, qual seria a espessura do filme estagnado junto à superfície, a essa distância? (1v)
} % Q1.3
    \answer{}
    \begin{flalign*}
        &
            k_C
            = \frac{\mathscr{D}_{A,B}}{\delta}
            \implies
            \delta
            \cong
            \frac{\mathscr{D}_{A,B}}{k_C}
        &
    \end{flalign*}
\end{questionBox}

\begin{questionBox}2{ % MARK: Q1.4
    Se pretender comprar ambientadores sólidos perfumados tendo para escolha com a forma de esferas ou de cubos, com as mesmas dimensões e custo, quais escolheria? Justifique a sua resposta. (3v)
} % Q1.4
    \answer{}
    A minha escolha dependeria do meu objetivo as utilizar o am- bientador. Para esferas e cubos com as mesmas dimensões, a áRea exposta de um cubo é maior que a da esfera e o volume do cubo tambem é maior. Isto quere dizeR que, se pretencer um cheiro mais intenso, escolho: o cuba (por ter uma velocida- de de sublimação mais alta) e se pretender que o cheiro dure mais itempo, escolho a esfera.
\end{questionBox}

\begin{questionBox}1{ % MARK: Q2
    Pretende-se remover \ch{SO2} de uma mistura gasosa constituída por \ch{SO2} e ar por absorção em água, usando uma coluna de enchimento. Num dado ponto da coluna a percentagem molar de \ch{SO2} no ar é 22\% e 1\% na água. A coluna opera a uma pressão de \qty*{3.5}{\atm} e à temperatura de \qty*{15}{\celsius}. A linha de equilíbrio é dada por \(y^*= 7.6\,x\). Sabendo que \(k_y = \qty*{2}{\mole/\metre^2.\hour}\) e ambas as fases contribuem com igual resistência calcule:
} % Q2
    \begin{itemize}
        \begin{multicols}{2}
            \item \(y_A=0.22\)
            \item \(x_A=0.01\)
            \item \(P=\qty*{3.5}{\atm}\)
            \item \(T=\qty*{288.15}{\kelvin}\)
            \item \(y^*_A=7.6\,x_A\)
            \item \(k_y=\qty*{2}{\mole/\metre^2.\hour}\)
        \end{multicols}
    \end{itemize}
\end{questionBox}

\begin{questionBox}2{ % MARK: Q2.1
    O coeficiente individual de transferência de massa de massa, \(k_x\).
} % Q2.1
    \answer{}
    \begin{flalign*}
        &
            \frac{k_y^{-1}}{K_y^{-1}}
            = \frac{k_x^{-1}}{K_x^{-1}}
            = 0.5
            % 
            % 
            % 
            ; &\\[3ex]&
            \frac{1}{K_y}
            = \frac{1}{k_y}
            + \frac{m}{k_x}
            % \implies &\\&
            \implies
            k_x
            = m
            \,\left(
                \frac{1}{K_y}
                - \frac{1}{k_y}
            \right)^{-1}
        &
    \end{flalign*}
\end{questionBox}
\begin{questionBox}2{ % MARK: Q2.2
    Os coeficientes globais de transferência de massa, \(K_y\text{ e }K_G\). (2v)
} % Q2.2
    \answer{}
    \begin{flalign*}
        &
            K_G
            \,\left(
                P_{A,G}-P_{A}^*
            \right)
            = N_A
            = K_y
            \,\left(
                y_{A}-y_{A}^*
            \right)
            = K_y
            \,\left(
                y_{A}-7.6\,x_A
            \right)
            % 
            % 
            % 
            ; &\\[3ex]&
            P_{A,G}
            = 0.22*3.5
            = \qty*{0.77}{\atm}
            \quad
            y^*_A=7.6\,x_A=0.076
            \quad
            y_A=0.22
        &
    \end{flalign*}
\end{questionBox}
\begin{questionBox}2{ % MARK: Q2.3
    O fluxo de transferência de massa. (2v)
} % Q2.3
    \answer{}
    \begin{flalign*}
        &
            N_A
            = K_y
            \,\left(
                y_{A}-y_{A}^*
            \right)
        &
    \end{flalign*}
\end{questionBox}

\begin{questionBox}2{ % MARK: Q2.4
    O fluxo de transferência de massa. (2v)
} % Q2.4
    \answer{}
    \begin{flalign*}
        &
            x_{A,i}
            = y_{A,i}/7.6
            ; &\\[3ex]&
            \frac{k_y^{-1}}{K_y^{-1}}
            = \frac
            {y_A-y_{A,i}}
            {y_A-y^*_{A}}
            = 0.5
            \implies &\\&
            \implies 
            = y_{A,i}
            = y_A
            - 0.5
            \,(y_A-y^*_{A})
        &
    \end{flalign*}
\end{questionBox}

\begin{questionBox}2{ % MARK: Q2.5
    Se em vez de usar água como fase líquida, usar uma solução aquosa de \ch{NaOH}, ocorrerá a sequinte reacção química (reacção de segunda ordem irreversível) com uma cinética muito rápida. \ch{SO2 + 2 NaOH -> Na2SO3 + H2O} Explique qual a vantagem de usar uma concentração de \ch{NaOH} superior à crítica e calcule o fluxo de transferência de massa nesta situação. Compare com o valor obtido em 2.b e comente. (2v)
} % Q2.5
    \answer{}
    \begin{flalign*}
        &
            \frac{N_A}{N_A^0}
            ; &\\[3ex]&
            N_A
            = k_y
            \,(y_{A}-y_{A,i})
            = k_y
            \,(y_{A}-0)
            = 2*0.22
            = \qty*{0.44}{\mole/\metre^2.\hour}
            \implies &\\[3ex]&
            \implies
            \frac{N_A}{N_A^0}
            = \frac{0.44}{0.144}
            \cong 4
        &
    \end{flalign*}
\end{questionBox}

\end{document}
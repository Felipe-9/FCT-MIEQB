% !TEX root = ./FT_II-Tests_Resolutions.2022.2.tex
\providecommand\mainfilename{"./FT_II-Tests_Resolutions.tex"}
\providecommand \subfilename{}
\renewcommand   \subfilename{"./FT_II-Tests_Resolutions.2022.2.tex"}
\documentclass[\mainfilename]{subfiles}

% \tikzset{external/force remake=true} % - remake all

\begin{document}

% \graphicspath{{\subfix{./figures/FT_II-Tests_Resolutions.2022.2}}}
% \tikzsetexternalprefix{./figures/FT_II-Tests_Resolutions.2022.2/graphics/}

\mymakesubfile{2}
[FT II]
{Test 2022.2 Resoltion} % Subfile Title
{Test 2022.2 Resoltion} % Part Title

\begin{questionBox}1{ % MARK: Q1
    Obtiveram-se os seguintes dados de coeficiente de atrito (\(C_f\)) para o escoamento de ar ao longo de uma conduta cilíndrica revestida com naftaleno:
} % Q1
    \begin{center}
        \vspace{1ex}
        \setlength\tabcolsep{6mm}        % width
        % \renewcommand\arraystretch{1.25} % height
        \begin{tabular}{C *{5}{C}}
            \toprule
            
                Re
                & 1\E{4}
                & 5\E{4}
                & 1\E{5}
                & 5\E{5}
                & 1\E{6}
            
            \\\midrule
            
                C_f
                & 8.0\E{-3}
                & 6.1\E{-3}
                & 5.0\E{-3}
                & 4.4\E{-3}
                & 4.1\E{-3}
            
            \\\bottomrule
        \end{tabular}
        \vspace{2ex}
    \end{center}
    Faz-se passar ar à temperatura de \qty*{15}{\celsius} e à pressão de \qty*{1}{\atm} através dessa conduta (\qty*{5}{\cm} de diâmetro e \qty*{4.5}{\m} de comprimento) a uma velocidade de \qty*{15}{\m/\s}.

    \paragraph*{Dados:}
    \begin{align*}
        D_{Naf,ar}(15\,\unit{\celsius},1\,\unit{\atm})
        =& \qty*{7.7e-6}{\m/\s}
        ;& \rho_{ar}(15\unit{\celsius}) 
        =& \qty*{1}{\kg/\m^3}
        \\ \mu_{ar}(15\unit{\celsius}) 
        =& \qty*{2.0e-5}{\m^2/\s}
        ;& P^*(15\unit{\celsius})
        =& \qty*{3.5}{\mmHg}
    \end{align*}
    \begin{align*}
        \text{Analogia de Reynolds}
        &\qquad
        \text{Analogia de Chilton-Coulburn}
        \\
        \frac{k_C}{v}=\frac{C_f}{2}
        &\qquad
        \frac{k_C}{v}\,Sc^{2/3}=\frac{C_f}{2}
    \end{align*}
    \begin{align*}
        \ln{\frac
            {C_{A,s}-C_{A,0}}
            {C_{A,s}-C_{A,L}}
        } 
        =& \frac{4\,L}{d}
        \,\frac{k_C}{v}
        ;&
        Sc&=\frac{\mu}{\rho\,\mathscr{D}}
        ;&
        Sh&=\frac{k_c\,d}{\mathscr{D}}
        ;&
        Re&=\frac{\rho\,u\,d}{\mu}
    \end{align*}
\end{questionBox}

\begin{questionBox}2{ % MARK: Q1.1
    O coeficiente de transferência de massa usando a analogia de Chilton-Colburn.
} % Q1.1
    \answer{}
    \begin{flalign*}
        &
            k_C
            = \frac{C_f\,v}{2\,Sc^{2/3}}
            = \frac{C_f\,v}{
                2
                \,\left(
                    \mu
                    /\rho\,\mathscr{D}_{Naf,Ar}
                \right)^{2/3}
            }
            % \cong &\\&
            \cong \frac{
                \num{6.69375e-3}
                *15
            }{
                2
                \,\left(
                    2.0\E{-5}
                    /1*7.7\E{-6}
                    % 2.597402597402597
                \right)^{2/3}
                % 4.186093319612365
            }
            \unit{\m/\s}
            \cong &\\&
            \cong\mathemph{
                \qty
                {2.6568757071760533e-2}
                {\m/\s}
            }
            % 
            % 
            % 
            ; &\\[3ex]&
            \mathemph{C_f}&\\&
            C_f\left(
                Re
            \right)
            = C_f\left(
                \frac{\rho\,d\,v}{\mu}
            \right)
            = C_f\left(
                \frac{1*5\E{-2}*15}{\num*{2.0e-5}}
            \right)
            = C_f\left(
                \num{3.7500e4}
            \right)
            = C_f\left(
                \num{3.7500e4}
            \right)
            \implies &\\&
            \implies
            \frac{
                C_f\left(
                    \num{3.7500e4}
                \right)
                -C_f(1\E{4})
            }{
                \num{3.7500e4}
                -1\E{4}
            }
            = \frac{
                C_f(5\E{4})
                -C_f(1\E{4})
            }{
                5\E{4}
                -1\E{4}
            }
            \implies &\\&
            \implies
            C_f\left(
                \num{3.7500e4}
            \right)
            = \frac{
                C_f(5\E{4})
                -C_f(1\E{4})
            }{
                5\E{4}
                -1\E{4}
            }
            \,\left(
                \num{3.7500e4}
                -1\E{4}
            \right)
            + C_f(1\E{4})
            = &\\&
            = \frac{
                6.1\E{-3}
                -8.0\E{-3}
            }{
                5\E{4}
                -1\E{4}
            }
            \,\left(
                \num{3.7500e4}
                -1\E{4}
            \right)
            + 8\E{-3}
            \cong &\\&
            \cong\num
            {6.69375e-3}
        &
    \end{flalign*}
\end{questionBox}

\begin{questionBox}2{ % MARK: Q1.2
    Pode usar a analogia de Reynolds? Justifique.
} % Q1.2
    \answer{}
    \begin{flalign*}
        &
            Sc
            = \frac{\mu}{\rho\,\mathscr{D}}
            = \frac{2.0\E{-5}}{1*7.7\E{-6}}
            \cong
            \num{2.597402597402597}
            \neq1
        &
    \end{flalign*}
    \(\therefore\) não se pode usar a analogia de Reynalds
\end{questionBox}

\begin{questionBox}2{ % MARK: Q1.3
    A concentração de naftaleno no ar para o comprimento de 1.5 m.
} % Q1.3
    \answer{}
    \begin{flalign*}
        &
            \mathemph{C_{A,15}}:&\\&
            \ln\frac
            {C_{A,s}-C_{A,0}}
            {C_{A,s}-C_{A,L}}
            = \ln\frac
            {C_{A}^*-0}
            {C_{A}^*-C_{A,L}}
            = -\ln(1-C_{A,L}/C_{A}^*)
            = \frac{4\,L}{d}
            \,\frac{K_C}{v}
            \implies &\\&
            \implies
            C_{A,L}
            = C_{A}^*
            \,\left(
                1
                -\exp{\left(
                    -\frac{4\,L}{d}
                    \,\frac{K_C}{v}
                \right)}
            \right)
            % = &\\&
            = \frac{P^*}{R\,T}
            \,\left(
                1
                -\exp{\left(
                    -\frac{4\,L}{d}
                    \,\frac{K_C}{v}
                \right)}
            \right)
            = &\\&
            = \frac
            {3.5/\num{760.00209995982457}}
            {
                \num{8.20573660809596e-5}
                \,(15+273.15)
            }
            % 0.194767753713938
            \,\left(
                1
                -\exp{\left(
                    -\frac{4\,L}{5\E{-2}}
                    \,\frac
                    {\num{2.6568757071760533e-2}}
                    {15}
                    % 0.063961784785245
                \right)}
            \right)
            \cong &\\&
            \cong \num{1.94767753713938e-1}
            \,\left(
                1
                -\exp{\left(
                    -\num{1.41700037716056e-1}
                    \,L
                \right)}
            \right)
            % 0.191480156424951
            \implies &\\&
            \implies
            C_{A,1.5}
            \cong\mathemph{
                \qty
                {3.7294159947681e-2}
                {\mole/\m^3}
            }
        &
    \end{flalign*}
\end{questionBox}

\begin{questionBox}2{ % MARK: Q1.4
    A percentagem de saturação do ar na corrente de saída.
} % Q1.4
    \answer{}
    \begin{flalign*}
        &
            \%\sat
            = \frac{C_{A,L}}{C^*}
            = \frac{C_{A,4.5}}{C^*}
            \cong 
            \frac{
                C^*
                \,\left(
                    1
                    -\exp{\left(
                        -\num{1.41700037716056e-1}
                        *4.5
                    \right)}
                \right)
            }{
                C^*
            }
            % \cong &\\&
            \cong\qty
            {47.146707133685801}
            {\percent}
        &
    \end{flalign*}
\end{questionBox}

\begin{questionBox}2{ % MARK: Q1.5
    Para além de analogias quais os outros métodos que poderá usar para calcular coeficientes de transferência de massa? Quais são os mais utilizados?
} % Q1.5
    \answer{}
    Usamos correlações e medições de queda de pressão, a primeira toma prioridade por temperatura ser mais fáceis de medir no laboratório do que pressões.
\end{questionBox}

\begin{questionBox}1{ % MARK: Q2
    É obtida água com cloro, para utilização no branqueamento de pasta de papel, por absorção de cloro gasoso em água numa coluna de enchimento à temperatura de \qty*{293}{\K} e à pressão de \qty*{1}{\atm}. Num dado ponto da coluna a pressão parcial de cloro no gás é \qty*{125}{\mmHg} e a concentração de cloro no líquido é de \qty*{14}{\milli\M}. Se 80\% da resistência à transferência de massa estiver na fase líquida, calcule:
} % Q2
    \paragraph*{Dados de equilíbrio:}
    \begin{center}
        \vspace{1ex}
        \rowcolors*{2}{background!95!foreground}{}
        \begin{tabular}{L *{6}{C}}
            \toprule
            
                p_{cloro}/\unit{\mmHg}
                & 5 & 10 & 30 & 50 & 100 & 150
                \\
                C_{cloro}/\unit{\milli\M}
                & 6.2 & 8.1 & 13.2 & 17.1 & 25.0 & 32.0

                % 6.2 , 5
                % 8.1 , 10
                % 13.2 , 30
                % 17.1 , 50
                % 25.0 , 100
                % 32.0, 150
            \\\bottomrule
        \end{tabular}
        \vspace{2ex}
    \end{center}
\end{questionBox}

\begin{questionBox}2{ % MARK: Q2.1
    As composições de equilíbrio.
} % Q2.1
    \answer{}
    \begin{flalign*}
        &
            C_{cloro}^*
            =f(p_{cloro}^*)
            =f(125)
            \implies
            \frac
            {f(125)-f(100)}
            {125-100}
            = \frac
            {f(150)-f(100)}
            {150-100}
            \implies &\\&
            \implies
            C_{cloro}^*
            = f(125)
            = \frac
            {f(150)-f(100)}
            {150-100}
            (125-100)
            +f(100)
            = &\\&
            = \frac
            {32.0-25.0}
            {150-100}
            (125-100)
            +25.0
            % = &\\&
            = \mathemph{
                \qty*
                {28.5}
                {\milli\M}
            }
            % 
            % 
            % 
            ; &\\[3ex]&
            P^*_{cloro}
            = f(c_{cloro})
            = f(14)
            \implies &\\&
            \implies
            \frac
            {f(14)-f(13.2)}
            {14-13.2}
            = \frac
            {f(17.1)-f(13.2)}
            {17.1-13.2}
            \implies &\\&
            \implies
            p^*_{cloro}
            = f(14)
            = \frac
            {f(17.1)-f(13.2)}
            {17.1-13.2}
            \,(14-13.2)
            + f(13.2)
            = &\\&
            = \frac
            {50-30}
            {17.1-13.2}
            \,(14-13.2)
            + 30
            \cong\mathemph{
                \qty
                {34.102564102564103}
                {\mmHg}
            }
        &
    \end{flalign*}
\end{questionBox}

\begin{questionBox}2{ % MARK: Q2.2
    As composições interfaciais.
} % Q2.2
    \answer{}
    \begin{flalign*}
        &
            \mathemph{C_{A,i}}&\\&
            \frac
            {C_{A,i}-C_{A,L}}
            {C_{A}^*-C_{A,L}}
            =0.8
            \implies &\\&
            \implies
            C_{A,i}
            = 0.8(C_{A}^*-C_{A,L})+C_{A,L}
            = 0.8(28.5-14)+14
            \cong\mathemph{
                \qty*{25.6}
                {\milli\M}
            }
            % 
            % 
            % 
            ; &\\[3ex]&
            \mathemph{P_{A,i}}&\\&
            \frac
            {P_{A}-P_{A,i}}
            {P_{A}-P_{A}^*}
            =0.2
            \implies &\\&
            \implies
            P_{A,i}
            =P_A-0.2(P_{A}-P_{A}^*)
            =125-0.2(125-\num{34.102564102564103})
            \cong\mathemph{
                \qty
                {106.82051282051282}
                {\mmHg}
            }
        &
    \end{flalign*}
\end{questionBox}

\begin{questionBox}2{ % MARK: Q2.3
    O coeficiente global de transferência de massa \(K_G\) sabendo que o coeficiente individual de transferência de massa \(k_G\) é \qty*{3.3e-4}{\mole/\hour\,\m^2\,\mmHg}.
} % Q2.3
    \answer{}
    \begin{flalign*}
        &
            K_G
            =0.2\,k_G
            =0.2*3.3\E{-4}
            =\mathemph{
                \qty*
                {6.6e-5}
                {\mole/\hour\,\m^2\,\mmHg}
            }
        &
    \end{flalign*}
\end{questionBox}

\begin{questionBox}2{ % MARK: Q2.4
    O fluxo de cloro nesse ponto da coluna.
} % Q2.4
    \answer{}
    \begin{flalign*}
        &
            N_A
            =K_G\,(P_{A,i}-P_A^*)
            \cong
            6.6\E{-5}
            \,(125-\num{34.102564102564103})
            \cong\mathemph{
                \qty
                {5.9992307692307694e-3}
                {\mole/\hour\,\m^2}
            }
        &
    \end{flalign*}
\end{questionBox}

\begin{questionBox}2{ % MARK: Q2.5
    Se o processo se realizar a uma temperatura mais elevada, qual espera ser o efeito no fluxo? Justifique a sua resposta.
} % Q2.5
    \answer{}
    \begin{flalign*}
        &
            N_A\propto-P_A^*
            \land
            P_A^*\propto T
            \therefore
            N_A\propto-T
        &
    \end{flalign*}
\end{questionBox}

\begin{questionBox}2{ % MARK: Q2.6
    Será importante usar uma reação química na fase líquida neste caso? Justifique a sua resposta.
} % Q2.6
    \answer{}
    É importante a reação química porque a resistencia na fase líquida é elevada, usar a reação química contribuia para diminuir essa resistencia
\end{questionBox}

\end{document}
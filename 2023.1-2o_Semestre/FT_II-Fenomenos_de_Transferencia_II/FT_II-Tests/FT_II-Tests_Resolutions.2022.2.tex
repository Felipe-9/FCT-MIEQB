% !TEX root = ./FT_II-Tests_Resolutions.2022.2.tex
\providecommand\mainfilename{"./FT_II-Tests_Resolutions.tex"}
\providecommand \subfilename{}
\renewcommand   \subfilename{"./FT_II-Tests_Resolutions.2022.2.tex"}
\documentclass[\mainfilename]{subfiles}

% \tikzset{external/force remake=true} % - remake all

\begin{document}

% \graphicspath{{\subfix{./figures/FT_II-Tests_Resolutions.2022.2}}}
% \tikzsetexternalprefix{./figures/FT_II-Tests_Resolutions.2022.2/graphics/}

\mymakesubfile{2}
[FT II]
{Test 2022.2 Resoltion} % Subfile Title
{Test 2022.2 Resoltion} % Part Title

\begin{questionBox}1{ % MARK: Q1
    Obtiveram-se os seguintes dados de coeficiente de atrito (Cf) para o escoamento de ar ao longo de uma conduta cilíndrica revestida com naftaleno:
} % Q1
    \begin{center}
        \vspace{1ex}
        \setlength\tabcolsep{3mm}        % width
        % \renewcommand\arraystretch{1.25} % height
        \begin{tabular}{C *{5}{C}}
            \toprule
            
                Re
                & 1\E{-4}
                & 5\E{-4}
                & 1\E{-5}
                & 5\E{-5}
                & 1\E{-6}
            
            \\\midrule
            
                C_f
                & 8.0\E{-3}
                & 6.1\E{-3}
                & 5.0\E{-3}
                & 4.4\E{-3}
                & 4.1\E{-3}
            
            \\\bottomrule
        \end{tabular}
        \vspace{2ex}
    \end{center}
    Faz-se passar ar à temperatura de \qty*{15}{\celsius} e à pressão de \qty*{1}{\atm} através dessa conduta (\qty*{5}{\cm} de diâmetro e \qty*{4.5}{\m} de comprimento) a uma velocidade de \qty*{15}{\m/\s}.


    \paragraph*{Dados:}
    \begin{align*}
        D_{Naf,ar}(15\,\unit{\celsius},1\,\unit{\atm})
        =& \qty*{7.7e-6}{\m/\s}
        ;& \rho_{ar}(15\unit{\celsius}) 
        =& \qty*{1}{\kg/\m^3}
        \\ \mu_{ar}(15\unit{\celsius}) 
        =& \qty*{2.0e-5}{\m^2/\s}
        ;& P^*(15\unit{\celsius})
        =& \qty*{3.5}{\mmHg}
    \end{align*}
    \begin{multicols}{2}
        Analogia de Reynolds
        \begin{align*}
            \frac{k_c}{V}=\frac{C_f}{2}
        \end{align*}

        Analogia de Chilton-Coulburn
        \begin{align*}
            \frac{k_C}{V}\,Sc^{2/3}=\frac{C_f}{2}
        \end{align*}
    \end{multicols}
    \begin{align*}
        \ln{\frac
            {C_{A,s}-C_{A,0}}
            {C_{A,s}-C_{A,L}}
        } 
        = \frac{4\,L}{d}
        \,\frac{k_C}{V}
    \end{align*}
    \begin{align*}
        Sc&=\frac{\mu}{\rho\,\mathscr{D}}
        ;&
        Sh&=\frac{k_c\,d}{\mathscr{D}}
        ;&
        Re&=\frac{\rho\,u\,d}{\mu}
    \end{align*}
\end{questionBox}

\begin{questionBox}2{ % MARK: Q1.1
    O coeficiente de transferência de massa usando a analogia de Chilton-Colburn.
} % Q1.1
    \answer{}
    \begin{flalign*}
        &
            k_C
            =\frac{C_f\,v}{Sc^{3/2}\,2}
            % 
            % 
            % 
            ; &\\[3ex]&
            Re
            = \frac{\rho\,d\,v}{\mu}
            = \frac{1*5\E{-2}*15}{2\E{-5}}
            \implies &\\&
            \implies
            C_f(Re)\cong\dots
            % 
            % 
            % 
            ; &\\[3ex]&
            \therefore
            k_C
            =\frac{C_f\,v}{Sc^{3/2}\,2}
            =\dots
        &
    \end{flalign*}
\end{questionBox}

\begin{questionBox}2{ % MARK: Q1.2
    Pode usar a analogia de Reynolds? Justifique.
} % Q1.2
    \answer{}
    Não
    \begin{flalign*}
        &
            Sc\neq1
        &
    \end{flalign*}
\end{questionBox}

\begin{questionBox}2{ % MARK: Q1.3
    A concentração de naftaleno no ar para o comprimento de 1.5 m.
} % Q1.3
    \answer{}
    \begin{flalign*}
        &
            k_C
            =\ln{(1-C_{A,L}/C^*)^{-1}}
            \,\frac{d\,v}{4\,l}
            \implies &\\&
            \implies
            C_{A,1.5}
            = 
            C^*\left(
                1
                -\exp{\left(
                    -\frac
                    {4\,k_C\,l}{d\,v}
                \right)}
            \right)
        &
    \end{flalign*}
\end{questionBox}

\begin{questionBox}2{ % MARK: Q1.4
    A percentagem de saturação do ar na corrente de saída.
} % Q1.4
    \answer{}
    \begin{flalign*}
        &
            \%_{sat}
            =\frac{C_{A,L}}{C^*}
            \cong\frac{1.92}{0.0917}
            \cong
            47.76\%
        &
    \end{flalign*}
\end{questionBox}

\begin{questionBox}2{ % MARK: Q1.5
    Para além de analogias quais os outros métodos que poderá usar para calcular coeficientes de transferência de massa? Quais são os mais utilizados?
} % Q1.5
    \answer{}
    Correlações experimentaias e queda de pressão, é mais fácil nedir no lab. temp do q pressão.
\end{questionBox}

\end{document}
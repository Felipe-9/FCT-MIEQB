% !TEX root = ./FT_II-Tests_Resolutions.3.1.tex
\providecommand\mainfilename{"./FT_II-Tests_Resolutions.tex"}
\providecommand \subfilename{}
\renewcommand   \subfilename{"./FT_II-Tests_Resolutions.3.1.tex"}
\documentclass[\mainfilename]{subfiles}

% \tikzset{external/force remake=true} % - remake all

\begin{document}

% \graphicspath{{\subfix{./.build/figures/FT_II-Tests_Resolutions.3.1}}}
% \tikzsetexternalprefix{./.build/figures/FT_II-Tests_Resolutions.3.1/}

\mymakesubfile{2}
[FT II]
{Exame de recurso resolução} % Subfile Title
{Exame de recurso resolução} % Part Title

\begin{questionBox}1{ % Q1
    Tubo de ensaio
} % Q1
    \begin{itemize}
        \begin{multicols}{2}
            \item 5\,\unit{\centi\metre} de altura
            \item Contem metanol
            \item \(z_0=2\,\unit{\centi\metre}\) nivel inicial
            \item Corrente de ar
            \item \(25\,\unit{\celsius},1\,\unit{\atm}\)
            \item \(D_{met,ar}=0.196\,\unit{\centi\metre^2.\second^{-1}}\) Coef de dif
            \item \(p=126\,\unit{\mmHg}\) respectivamente
            \item \(M_{met}=32\,\unit{\gram.\mole^{-1}}\)
            \item \(\rho_{met}=0.792\,\unit{\gram.\centi\metre^{-3}}\)
        \end{multicols}
    \end{itemize}

    \begin{questionBox}2{ % Q1.1
        Calc nivel de met no fim de 24h
    } % Q1.1
        
    \end{questionBox}

    \begin{questionBox}2{ % Q1.2
        Explique oq acontece se a temp dobrar
    } % Q1.2
        body
    \end{questionBox}
\end{questionBox}

\begin{questionBox}1{ % Q2
    Gas a
} % Q2
    \begin{itemize}
        \begin{multicols}{2}
            \item \(T_A=600\,\unit{\kelvin}\)
            \item \(p_A=101.3\,\unit{\kilo\pascal}=1\,\unit{\atm}\)
            \item \(z_1=1\,\unit{\milli\metre}\)
            \item \ch{A -> B}
            \item \(N_A\propto -N_B\)
            \item \(D_A=0.15\,\unit{\centi\metre^3.\second^{-1}}\)
        \end{multicols}
    \end{itemize}

    \begin{questionBox}2{ % Q2.1
        Fração de a se a reac for inst, just
    } % Q2.1
        \answer{}
        \begin{flalign*}
            &
                f\begin{cases}
                    r_0=0 & y_{A,z}=y_{A,*}
                    \\
                    r_1=1*10^{-3} & y_{A,z}=y_{A,z_2}
                \end{cases}
                &\\&
                N_{A,z}
                =y_{A,z}(N_{A,z}+N_{B,z})
                - \frac{P\,D_{A,B}}{R\,T}\odv{y_{A,z}}{z}
                \cong
                y_{A,z}\,N_{A,z}
                - \frac{P\,D_{A,B}}{R\,T}\odv{y_{A,z}}{z}
                \implies &\\[5ex]&
                \implies
                \int_{z_0}^{z_1}{N_{A,z}\,\odif{z}}
                = N_{A,z}\int_{z_0}^{z_1}{\odif{z}}
                = N_{A,z}\,z_1
                = &\\[3ex]&
                = \int_{y_{A,z_0}}^{y_{A,z_1}}{
                    -\frac{P\,D_{A,B}}{R\,T}
                    \frac{\odif{y_{A,z}}}{1-y_{A,z}}
                }
                = 
                -\frac{P\,D_{A,B}}{R\,T}
                \int_{y_{A,z_0}}^{y_{A,z_1}}{
                    \frac{\odif{y_{A,z}}}{1-y_{A,z}}
                }
                = &\\& 
                =  
                \frac{P\,D_{A,B}}{R\,T}
                \int_{y_{A,z_0}}^{y_{A,z_1}}{
                    \frac{\odif{(1-y_{A,z})}}{1-y_{A,z}}
                }
                = 
                \frac{P\,D_{A,B}}{R\,T}
                \ln\frac
                {1-y_{A,z_0}}
                {1-y_{A,z_1}}
                = &\\&
                = 
                \frac{P\,D_{A,B}}{R\,T}
                \ln\frac
                {1-p_{A,z_0}/P}
                {1-p_{A,z_1}/P}
                = 
                \frac{P\,D_{A,B}}{R\,T}
                \ln\frac
                {P-p_{A,z_0}}
                {P-p_{A,z_1}}
                \implies &\\[5ex]&
                \implies
                N_{A,z}
                = \frac{z_1\,P\,D_{A,B}}{R\,T}
                \ln\frac
                {P-p_{A,z_0}}
                {P-p_{A,z_1}}
                = &\\&
                = \frac{
                    10^{-3}
                    *1
                    *0.15*10^{-6}
                }{
                    \num{8.314462618}
                    *600
                }
                \ln\frac
                {0.15*10^{-6}-0.15*10^{-6}}
                {0.15*10^{-6}-p_{A,z_1}}
            &
        \end{flalign*}
    \end{questionBox}
\end{questionBox}

% Q3 M = 352 g/mole

\begin{questionBox}1{ % Q3
    Considere
} % Q3
    \begin{itemize}
        \item Solvente não volátil
        \item \(c_{\ch{CO2},t_0}=1.8\,\unit{\mole.\litre^{-1}}\) Concentração de \ch{CO2} inicial no solvente
        \item exposto a atm de \ch{CO2}
        \item \(t_1=40'\)
        \item \(c_{\ch{CO2},4\,\unit{\centi\metre}}=5*10^{-3}\,\unit{\mole.\centi\metre^{-3}}\)
        \item \(D_{\ch{CO2},S}=5*10^{-5}\)
    \end{itemize}

    \begin{BM}
        \frac{C_{A,s}-C_A}{C_{A,s}-C_{A,0}}
        =erf\left(
            z/\sqrt{4\,D\,t}
        \right)
    \end{BM}

    

    Determine a concentração interface
\end{questionBox}

\end{document}
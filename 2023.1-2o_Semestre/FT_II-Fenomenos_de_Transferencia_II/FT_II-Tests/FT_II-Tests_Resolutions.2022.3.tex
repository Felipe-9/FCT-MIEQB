% !TEX root = ./FT_II-Tests_Resolutions.2022.3.tex
\providecommand\mainfilename{"./FT_II-Tests_Resolutions.tex"}
\providecommand \subfilename{}
\renewcommand   \subfilename{"./FT_II-Tests_Resolutions.2022.3.tex"}
\documentclass[\mainfilename]{subfiles}

% \tikzset{external/force remake=true} % - remake all

\begin{document}

% \graphicspath{{\subfix{./figures/FT_II-Tests_Resolutions.2022.3}}}
% \tikzsetexternalprefix{./figures/FT_II-Tests_Resolutions.2022.3/graphics/}

\mymakesubfile{3}
[FT II]
{Exame 2022.3 Resolução} % Subfile Title
{Exame 2022.3 Resolução} % Part Title

\begin{questionBox}1{ % MARK: Q1
    Um tanque com o topo aberto para a atmosfera contém metanol líquido (\ch{CH3OH}, peso molecular \qty*{32}{\g/\mole}) no fundo do tanque. O tanque é mantido a \qty*{30}{\celsius}. O diâmetro do tanque cilíndrico é de \qty*{1.0}{\m}, a altura total do tanque é de \qty*{3.0}{\m} e o nível do líquido no fundo do tanque é mantido em \qty*{0.5}{\m}. O espaço de gás dentro do tanque está estagnado e os vapores de \ch{CH3OH} são imediatamente dispersos assim que saem do tanque.\\[2ex]A \qty*{30}{\celsius}, a pressão de vapor exercida pelo \ch{CH3OH} líquido é de \qty*{163}{\mmHg} e a \qty*{40}{\celsius} a pressão de vapor do \ch{CH3OH} é de \qty*{265}{\mmHg}. O coeficiente de difusão do metanol no ar é \qty*{1.66}{\cm^2/\s} e varia com a temperatura \(T^{3/2}\) .
} % Q1
\end{questionBox}

\begin{questionBox}2{ % MARK: Q1.1
    Qual é a taxa de emissão de vapor de \ch{CH3OH} do tanque em \unit{\kg/\day} quando o tanque está a uma temperatura de \qty*{30}{\celsius}? Deduza a equação necessária e as condições fronteira para este problema.
} % Q1.1
    \answer{}
    \begin{flalign*}
        &
            Taxa
            =M_{A}\,W
            =M_{A}\,S\,N_A
            =M_{A}
            \,\left(
                \pi\,d^2/4
            \right)\,\left(
                \frac
                {c\,\mathscr{D}_{A,B}}
                {\Theta\,\eta_d\,l}
                \ln{\frac
                    {1-\Theta\,y_{A,2}}
                    {1-\Theta\,y_{A,1}}
                }
            \right)
            = &\\&
            =\frac
            {
                M_{A}\,
                \pi\,d^2
                \frac{P}{R\,T}
                \,\mathscr{D}_{A,B}
            }
            {
                4*1*1*(3.0-0.5)
            }
            \ln{\frac
                {1-1*0}
                {1-1\,y_{A,1}}
            }
            \cong  &\\&
            \cong \frac
            {
                32
                \,\pi\,1.0^2
                \frac{1}{
                    \num{8.20573660809596e-5}
                    *(30+273.15)
                }
                *1.66\E{-4}
            }
            {
                4*(3.0-0.5)
            }
            \ln{\frac
                {1}
                {1-\num{0.214473091598848}}
            }
            \cong &\\&
            \cong
            \qty
            {161.946317459157888e-4}
            {\gram/\s}
            \cong\mathemph{
                \qty
                {1.39921618284712415232}
                {\kg/\day}
            }
            % 
            % 
            % 
            ; &\\[3ex]&
            \text{Condições de fronteira fluxo:}&\\&
            \begin{cases}
                z_0=0.5 ;&\quad
                y_{A,0}=P_A^*/P
                \cong\frac{163}{\num{760.00209995982457}}
                \cong\num{0.214473091598848}
                \\
                z_1=3.0 ;&\quad
                y_{A,1}=0
            \end{cases}
            % 
            % 
            % 
            ; &\\[3ex]&
            \eta_{d,\text{plano}}
            =1
            % 
            % 
            % 
            ; &\\[3ex]&
            \Theta
            = 1+N_B/N_A
            = 1+0/N_A
            = 1
        &
    \end{flalign*}
\end{questionBox}

\begin{questionBox}2{ % MARK: Q1.2
    Se a temperatura do tanque for aumentada para \qty*{40}{\celsius}, qual é a \% de aumento na taxa de emissão para um aumento de \qty*{10}{\celsius} na temperatura?
} % Q1.2
    \answer{}
    \begin{flalign*}
        &
            \frac
            {\text{Taxa}_{\qty*{40}{\celsius}}}
            {\text{Taxa}_{\qty*{30}{\celsius}}}
            -1
            =\frac
            {
                \frac
                {
                    M_A
                    \,\pi\,d^2
                    \,\frac{P}{R\,T}
                    \,\mathscr{D}_{A,B,40\unit{\celsius}}
                }
                {4\,(3.0-0.5)}
                \ln{\frac
                    {1}
                    {1-y_{A,0,40\unit{\celsius}}}
                }
            }
            {
                \frac
                {
                    M_A
                    \,\pi\,d^2
                    \,\frac{P}{R\,T}
                    \,\mathscr{D}_{A,B,30\unit{\celsius}}
                }
                {4\,(3.0-0.5)}
                \ln{\frac
                    {1}
                    {1-y_{A,0,30\unit{\celsius}}}
                }
            }
            -1
            = &\\&
            =\frac
            {
                \frac
                {\mathscr{D}_{A,B,40\unit{\celsius}}}
                {40+273.15}
                \ln{\frac
                    {1}
                    {1-y_{A,0,40\unit{\celsius}}}
                }
            }
            {
                \frac
                {\mathscr{D}_{A,B,30\unit{\celsius}}}
                {30+273.15}
                \ln{\frac
                    {1}
                    {1-y_{A,0,30\unit{\celsius}}}
                }
            }
            -1
            \cong &\\&
            \cong\frac
            {
                \frac
                {
                    \mathscr{D}_{A,B,30\unit{\celsius}}
                    \,\left(
                        \frac{40+273.15}{30+273.15}
                    \right)^{3/2}
                }
                {40+273.15}
                \ln{\frac
                    {1}
                    {1-\num{0.348683247077881}}
                }
            }
            {
                \frac
                {\mathscr{D}_{A,B,30\unit{\celsius}}}
                {30+273.15}
                \ln{\frac
                    {1}
                    {1-\num{0.214473091598848}}
                }
            }
            -1
            \cong &\\&
            \cong\frac
            {
                \left(
                    \frac{40+273.15}{30+273.15}
                \right)^{1/2}
                \ln{\frac
                    {1}
                    {1-\num{0.348683247077881}}
                }
            }
            {
                \ln{\frac
                    {1}
                    {1-\num{0.214473091598848}}
                }
            }
            -1
            \cong &\\&
            \cong\mathemph{
                \qty
                {80.5188598786085}
                {\percent}
            }
            % 
            % 
            % 
            ; &\\[3ex]&
            \mathscr{D}_{A,B,40\unit{\celsius}}
            = \mathscr{D}_{A,B,30\unit{\celsius}}
            \,\left(
                \frac{40+273.15}{30+273.15}
            \right)^{3/2}
            % 
            % 
            % 
            ; &\\[3ex]&
            y_{A,0,40\unit{\celsius}}
            =P^*_{A,40\unit{\celsius}}/P
            \cong\frac{265}{\num{760.00209995982457}}
            \cong\num{0.348683247077881}
        &
    \end{flalign*}
\end{questionBox}

\begin{questionBox}1{ % MARK: Q2
    Um reator de leito fluidizado de carvão foi proposto para uma nova fábrica. Se operado a \qty*{1145}{\K}, o processo de combustão em ar (\qty*{21}{\percent\of{\ch{O2}}} e \qty*{79}{\percent\of{\ch{N2}}}) será limitado pela difusão do \ch{O2} em contracorrente ao \ch{CO2}, formado na superfície da partícula. Suponha que o carvão seja carbono sólido puro com densidade de \qty*{1.28e3}{\kg/\m^3} e que a partícula seja esférica com um diâmetro inicial de \qty*{1.5e-4}{\m} (\qty*{150}{\micro\m}). Sob as condições do processo de combustão, a difusividade do \ch{O2} na mistura gasosa a \qty*{1145}{\K} é \qty*{1.3e-4}{\cm^2/\s}. A reação na superfície é: \ch{C\sld{} + O2\gas{} -> CO2\gas} Na superfície da partícula de carvão, a reação é muito rápida.\\[2ex]
    Se for assumido um processo de estado quase estacionário, calcule:
} % Q2
    % \paragraph*{Dados:}
    % \begin{itemize}
    %     \item \(\mathscr{D}_{\ch{CO2},agua}=\qty*{1e-5}{\cm^2/\s}\)
    % \end{itemize}
\end{questionBox}

\begin{questionBox}2{ % MARK: Q2.1
    O tempo necessário para reduzir o diâmetro da partícula de carbono para \qty*{5e-5}{\m} (\qty*{50}{\micro\m}). Deduza as equações necessárias e as condições fronteira para este problema.
} % Q2.1
    \answer{}
    \begin{flalign*}
        &
            C_{A,S}
            \,\odv{\vol}{t}
            = \left(
                \frac{\rho_A}{M_A}
            \right)
            \,\odv{\pi\,r^3\,4/3}{t}
            = \frac{\rho_A}{M_A}
            \,4\,\pi\,r^2
            \,\odv{r}{t}
            = &\\&
            =-Q_{A,r}
            =-\frac{P}{R\,T}
            \,\mathscr{D}_{A,B}
            \,4\,\pi\,r
            \,y_{A,0}
            % 
            % 
            % 
            \implies &\\[3ex]&
            \implies
            \int_{R_0}^{R_1}{
                r\,\odif{r}
            }
            = (R_1^2-R_0^2)/2
            % 
            % 
            % 
            = &\\[3ex]&
            =\int_{0}^{t}{
                \left(
                    -\frac{P}{R\,T}
                    \,\frac{M_A}{\rho_A}
                    \,\mathscr{D}_{A,B}
                    \,y_{A,0}
                \right)
                \,\odif{t}
            }
            =
            \left(
                -\frac{P}{R\,T}
                \,\frac{M_A}{\rho_A}
                \,\mathscr{D}_{A,B}
                \,y_{A,0}
            \right)
            \int_{0}^{t}{
                \,\odif{t}
            }
            = &\\&
            =
            \left(
                -\frac{P}{R\,T}
                \,\frac{M_A}{\rho_A}
                \,\mathscr{D}_{A,B}
                \,y_{A,0}
            \right)
            t
            % 
            % 
            % 
            \implies &\\[3ex]&
            \implies
            t
            =-\frac{R\,T}{P}
            \,\frac{\rho_A}{M_A}
            \frac{
                (R_1^2-R_0^2)/2
            }{
                \mathscr{D}_{A,B}
                \,y_{A,0}
            }
            \cong  &\\&
            \cong -\frac{
                \num{8.20573660809596e-5}
                *1145
            }{
                1
            }
            % 9395.5684162698742e-5
            \,\frac{1.28\E{6}}{12}
            \frac{
                ((0.5\E{-4}/2)^2-(1.5\E{-4}/2)^2)/2
                % 0.25
            }{
                1.3\E{-8}
                *0.21
            }
            \cong &\\&
            \cong\mathemph{
                \qty
                {9177.60040661281973}
                {\second}
            }
            \cong\mathemph{
                \qty*{2}{\hour}
                \,\qty
                {32.960006776880329}
                {\minute}
            }
            % 
            % 
            % 
            ; &\\[3ex]&
            \text{Velocidade do fluxo}&\\&
            Q_{A,r}
            = S\,N_{A,z}
            = -S\,\frac
            {c\,\mathscr{D}_{A,B}}
            {1-\Theta\,y_A}
            \,\odv{y_A}{z}
            = -\frac
            {
                S
                \,\frac{P}{R\,T}
                \,\mathscr{D}_{A,B}
            }
            {1-0*y_A}
            \,\odv{y_A}{z}
            % 
            % 
            % 
            \implies &\\[3ex]&
            \implies
            \int_{\infty}^{r}{
                \frac{Q_{A,r}}{S}\,\odif{r}
            }
            = Q_{A,r}
            \,\int_{\infty}^{r}{
                \frac{\odif{r}}{
                    4\,\pi\,r^2
                }
            }
            = -\frac{Q_{A,r}}{4\,\pi}
            \,(r^{-1}-0)
            % 
            % 
            % 
            = &\\[3ex]&
            = \int_{y_{A,0}}^{0}{
                -\frac{P}{R\,T}
                \,\mathscr{D}_{A,B}
                \,\odif{y_{A}}
            }
            = 
            -\frac{P}{R\,T}
            \,\mathscr{D}_{A,B}
            \int_{y_{A,0}}^{0}{
                \odif{y_{A}}
            }
            = 
            -\frac{P}{R\,T}
            \,\mathscr{D}_{A,B}
            y_{A,0}
            % 
            % 
            % 
            \implies &\\[3ex]&
            \implies
            Q_{A,r}
            = 
            \frac{P}{R\,T}
            \,\mathscr{D}_{A,B}
            \,4\,\pi\,r
            \,y_{A,0}
            % 
            % 
            % 
            ; &\\[3ex]&
            \Theta
            = 1+N_{\ch{CO2}}/N_{\ch{O2}}
            = 1+(-N_{\ch{O2}})/N_{\ch{O2}}
            = 0
            % 
            % 
            % 
            ; &\\[3ex]&
            \text{Condições de fronteira fluxo:}&\\&
            \begin{cases}
                r_0\to\infty ;&\quad 
                y_{A,0}=0.21
                \\
                r_1=r ;&\quad 
                y_{A,1}\cong 0
                \quad\text{(Reação instantanea)}
            \end{cases}
            % 
            % 
            % 
            ; &\\[1ex]&
            \text{Condições de fronteira combustão:}&\\&
            \begin{cases}
                t_0=0 ;&\quad
                r_0=R_0=1.5\E{-4}
                \\
                t_1=t ;&\quad
                r_1=R_1=0.5\E{-4}
            \end{cases}
        &
    \end{flalign*}
\end{questionBox}

\begin{questionBox}2{ % MARK: Q2.2
    Explique por que razão temos neste caso difusão com reacção química heterogénea.
} % Q2.2
    \answer{}
    Temos uma reação heterogenia pois ela se da apenas na interfaçe entre o carvão e o gás ao seu redor, a existencia de multiplas fazes e localização da reação caracteriza como reação heterogenia.
\end{questionBox}

\begin{questionBox}1{ % MARK: Q3
    Um tanque de água profundo tem \ch{O2} dissolvido com uma concentração uniforme \qty*{1}{\g/\L}. Se a concentração de \ch{O2} for subitamente elevada à superfície para \qty*{10}{\g/\L}, calcule:
} % Q3
    \paragraph*{Dados:}
    \begin{align*}
        \frac
        {C_{A,s}-C_A}
        {C_{A,s}-C_{A,0}}
        &=\erf{\xi}
        ;&
        \xi&=\frac{z}{\sqrt{4\,D\,t}}
        ;&
        J_A^*
        =-\mathscr{D}
        \,\pdv{C_A}{z}
        =\sqrt{\mathscr{D}/\pi\,t}
        \,\exp{\left(
            -z^2/4\,\mathscr{D}\,t
        \right)}
        \,(C_{A,s}-C_{A,0})
    \end{align*}
    Em que
    \begin{itemize}
        \begin{multicols}{2}
        \item \(C_A\) é a concentração de \ch{O2} a uma distância (\(z\)) da superfície num determinado instante (\(t\))
        \item \(C_{A,0}\) é a concentração inicial
        \item \(C_{A,s}\) é a concentração na superfície
        \item \(\mathscr{D}\) o coeficiente de difusão.
        \end{multicols}
    \end{itemize}
    \begin{center}
        \vspace{1ex}
        \begin{tabular}{LL *{2}{ | LL}}
            \toprule
            
                \multicolumn{1}{C}{a}
                & \multicolumn{1}{C}{\erf(a)}
                & \multicolumn{1}{C}{a}
                & \multicolumn{1}{C}{\erf(a)}
                & \multicolumn{1}{C}{a}
                & \multicolumn{1}{C}{\erf(a)}
            
            \\\midrule
            
                   0.0  & 0.0     & 0.48 & 0.50275 & 0.96 & 0.82542
                \\ 0.04 & 0.04511 & 0.52 & 0.53790 & 1.00 & 0.84270
                \\ 0.08 & 0.09008 & 0.56 & 0.57162 & 1.10 & 0.88021
                \\ 0.12 & 0.13476 & 0.60 & 0.60386 & 1.20 & 0.91031
                \\ 0.16 & 0.17901 & 0.64 & 0.63459 & 1.30 & 0.93401
                \\ 0.20 & 0.22270 & 0.68 & 0.66378 & 1.40 & 0.95229
                \\ 0.24 & 0.26570 & 0.72 & 0.69143 & 1.50 & 0.96611
                \\ 0.28 & 6.30788 & 0.76 & 0.71754 & 1.60 & 0.97635
                \\ 0.32 & 0.34913 & 0.80 & 0.7421  & 1.70 & 0.98379
                \\ 0.36 & 0.38933 & 0.84 & 0.76514 & 1.80 & 0.98909
                \\ 0.40 & 0.42839 & 0.88 & 0.78669 & 2.00 & 0.99532
                \\ 0.44 & 0.46622 & 0.92 & 0.80677 & 3.24 & 0.99999
            
            \\\bottomrule
        \end{tabular}
        \\[1ex]\tablecaption{Error function values. For negative \textit{a}, erf(\textit{a}) is negative}
        \vspace{2ex}
    \end{center}
    \begin{BM}
        \erf(\myvert{a})
        = 1-\left(
            1
            +0.2784\,\myvert{a}
            +0.2314\,\myvert{a}^2
            +0.0781\,\myvert{a}^4
        \right)^{-4}
    \end{BM}
\end{questionBox}

\begin{questionBox}2{ % MARK: Q3.1
    A concentração de \ch{O2} a \qty*{1}{\mm} de profundidade ao fim de 2 horas?
} % Q3.1
    \answer{}
    \begin{flalign*}
        &
            \frac
            {C_{A,s}-C_A}
            {C_{A,s}-C_{A,0}}
            =\erf\xi
            \implies &\\&
            \implies
            C_{A}
            = C_{A,s}
            -(C_{A,s}-C_{A,0})
            \,\erf{\xi}
            \cong 10
            -(10-1)
            \,\num{0.207923252115868}
            \cong &\\&
            \cong\mathemph{
                \qty
                {8.128690730957188}
                {\mole/\m^3}
            }
            % 
            % 
            % 
            ; &\\[3ex]&
            \erf{\xi}
            = \erf{\left(
                \frac{z}{\sqrt{4\,\mathscr{D}\,t}}
            \right)}
            = \erf{\left(
                \frac{1\E{-3}}{\sqrt{4*1\E{-9}*(2*3600)}}
            \right)}
            \cong &\\&
            \cong \erf{\left(
                \num{0.186338998124982483}
            \right)}
            \cong
            1-\left(
                1
                +0.2784*(\num{0.186338998124982483})
                +0.2314*(\num{0.186338998124982483})^2
                +0.0781*(\num{0.186338998124982483})^4
                % 1.060005659215341
            \right)^{-4}
            \cong\num{
                0.207923252115868
            }
        &
    \end{flalign*}
\end{questionBox}

\begin{questionBox}2{ % MARK: Q3.2
    O fluxo de \ch{O2} na superfície do tanque para esse tempo?
} % Q3.2
    \answer{}
    \begin{flalign*}
        &
            J_A
            =\sqrt{\frac{\mathscr{D}}{\pi\,t}}
            \,(C_{A,s}-C_{A,0})
            =\sqrt{\frac{1\E{-9}}{\pi\,(2*3600)}}
            \,(10-1)
            \cong\mathemph{
                \num{1.89234939151512e-6}
            }
        &
    \end{flalign*}
\end{questionBox}

\begin{questionBox}1{ % MARK: Q4
    Ar seco (\qty*{300}{\K} e \qty*{1.013e5}{\Pa}) circula a uma velocidade de \qty*{1.5}{\m/\s}, num tubo com \qty*{6}{\m} de comprimento e \qty*{0.15}{\m} de diâmetro. A superfície interior do tubo está revestida com um material absorvente (com razão diâmetro/rugosidade, \(d/\varepsilon\), de \num*{10.000}) que está saturado com água.
} % Q4
    \paragraph*{Dados:}
    \begin{itemize}
        \item Difusividade da água em ar (\qty*{300}{\K}): \qty*{2.6e-5}{\m^2/\s}
        \item Viscosidade cinemática do ar (\qty*{300}{\K}): \qty*{1.569e-5}{\m^2\,s}
        \item Pressão de vapor da água a (\qty*{300}{\K}) = \qty*{17.5}{\mmHg}
        \item Constante do gases: \(R=\qty{8.2057366080960e-2}{\litre.\atm/\mole.\kelvin}\)
        \item Factor de atrito \(f = 7.91\E{-3}\,Re^{0.12}\)
    \end{itemize}
    \begin{align*}
        Re&=\frac{\rho\,d\,v}{\mu}
        ;& 
        Sc&=\frac{\mu}{\rho\,\mathscr{D}_{A,B}}
        ;&
        Sh&=\frac{k_c\,d}{\mathscr{D}_{A,B}}
        ;&
        \ln{\frac
            {C_{A,s}-C_{A,0}}
            {C_{A,s}-C_{A,L}}
        } =\frac{4\,L}{d}
        \,\frac{k_C}{v}
    \end{align*}
    Analogia de Chilton-Colburn:
    \begin{align*}
        \frac{k_c}{v}
        \,Sc^{2/3}
        &=\frac{f}{2}
        ;&
        C_{A,s}=C^*
        \land
        v\text{: Velocidade}
    \end{align*}
    Determine:
\end{questionBox}

\begin{questionBox}2{ % MARK: Q4.1
    A concentração de agua à saída do tubo
} % Q4.1
    \answer{}
    \begin{flalign*}
        &
            \ln{\frac
                {C_{A,s}-C_{A,0}}
                {C_{A,s}-C_{A,L}}
            }
            =\frac{4\,k_C\,L}{d\,v}
            \implies &\\&
            \implies
            C_{A,L}
            = C_{A,s}
            \left(
                1
                -\frac
                {
                    1-\frac
                    {C_{A,0}}
                    {C_{A,S}}
                }
                {
                    \exp{\left(
                        \frac{4\,k_C\,L}{d\,v}
                    \right)}
                }
            \right)
            = \frac{P_A^*}{R\,T}
            \left(
                1
                -\frac
                {
                    1-\frac{0}{C_{A,S}}
                }
                {
                    \exp{\left(
                        \frac{4\,k_C\,L}{d\,v}
                    \right)}
                }
            \right)
            = &\\&
            = \frac{P_A^*}{R\,T}
            \left(
                1
                -\exp{\left(
                    \frac{-4\,k_C\,L}{d\,v}
                \right)}
            \right)
            \cong &\\&
            \cong \frac{
                (17.5/\num{760.00209995982457})
            }{
                \num{8.20573660809596e-5}
                * 300
            }
            % 0.93537213721119
            \left(
                1
                -\exp{\left(
                    \frac{
                        -4
                        * \num{26.197418548907646e-3}
                        * 6
                    }{
                        0.15
                        *1.5
                    }
                \right)}
                % 0.938847914445797
            \right)
            \cong &\\&
            \cong\mathemph{
                \qty
                {0.878172180251434}
                {\mole/\m^3}
            }
            % 
            % 
            % 
            ; &\\[3ex]&
            k_C
            = \frac{f\,v}{2\,Sc^{2/3}}
            = \frac{
                (7.91\E{-3}\,Re^{0.12})\,v
            }{
                2
                \,\left(
                    \frac{\mu}{\rho\,\mathscr{D}_{A,B}}
                \right)^{2/3}
            }
            = \frac{
                7.91\E{-3}
                \,\left(
                    \frac{\rho\,d\,v}{\mu}
                \right)^{0.12}
                \,v
            }{
                2
                \,\left(
                    \frac{\mu}{\rho\,\mathscr{D}_{A,B}}
                \right)^{2/3}
            }
            = &\\&
            = \frac{
                7.91\E{-3}
                \,\left(
                    \frac{1*0.15*1.5}{1.569\E{-5}}
                \right)^{0.12}
                % 3.153458710269441
                \,1.5
                % 37.415787597346915
            }{
                2
                \,\left(
                    \frac{
                        1.569\E{-5}
                    }{
                        1
                        *2.6\E{-5}
                    }
                \right)^{2/3}
                % 0.714112108555578
            }
            \cong &\\[3ex]&
            \cong
            \num{26.197418548907646e-3}
        &
    \end{flalign*}
\end{questionBox}

\begin{questionBox}2{ % MARK: Q4.2
    A velocidade de transferencia de agua em \unit{\kg/\hour}
} % Q4.2
    \answer{}
    \begin{flalign*}
        &
            W
            =C_{A,L}\,v\,S
            \cong
            \num{0.878172180251434}
            *1.5
            *\pi\,(0.15)^2/4
            \cong\mathemph{
                \qty{2.327787509117206e-2}
                {\mole/\second}
            }
            \cong\mathemph{
                \qty{1.50840630590794945176}
                {\kg/\hour}
            }
        &
    \end{flalign*}
\end{questionBox}

\begin{questionBox}1{ % MARK: Q5
    Pretende-se remover \ch{SO2} de uma mistura gasosa constituída por \ch{SO2} e ar por absorção utilizando água. A constante de Henry é \qty*{1.5}{\atm}. A coluna usada opera a \qty*{15}{\celsius} e \qty*{3}{\atm}. Num dado ponto da coluna a \% molar de \ch{SO2} na fase gasosa é 20\% e na fase líquida é 1\%. Sabendo que os coeficiente individuais de transferência de massa são \(
        k_y=\qty*{5.6e-4}{\mole/\s\,m^2}
        \text{ e }
        k_x= \qty*{5.6e-3}{\mole/\s\,m^2}
    \).\\
    Determine
} % Q5
\end{questionBox}

\begin{questionBox}2{ % MARK: Q5.1
    As composições interfaciais
} % Q5.1
\end{questionBox}

\begin{questionBox}2{ % MARK: Q5.2
    A \% da resistencia total respeitante a cada uma das fases
} % Q5.2
\end{questionBox}

\begin{questionBox}2{ % MARK: Q5.3
    O coeficiente global de trasnferencia de massa \(K_x\)
} % Q5.3
\end{questionBox}

\begin{questionBox}2{ % MARK: Q5.4
    O fluxo de \ch{SO2}
} % Q5.4
\end{questionBox}

\begin{questionBox}2{ % MARK: Q5.5
    O valor do fluxo quando usar soluções de \ch{NaOH} com a concentração crítica de \ch{NaOH}. Comente
} % Q5.5
\end{questionBox}

\end{document}
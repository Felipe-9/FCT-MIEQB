% !TEX root = ./FT_II-Tests_Resolutions.2024.3.tex
\providecommand\mainfilename{"./FT_II-Tests_Resolutions.tex"}
\providecommand \subfilename{}
\renewcommand   \subfilename{"./FT_II-Tests_Resolutions.2024.3.tex"}
\documentclass[\mainfilename]{subfiles}

% \tikzset{external/force remake=true} % - remake all

\begin{document}

% \graphicspath{{\subfix{./figures/FT_II-Tests_Resolutions.2024.3}}}
% \tikzsetexternalprefix{./figures/FT_II-Tests_Resolutions.2024.3/graphics/}

\mymakesubfile{3}
[FT II]
{Exame 2024.3 Resolução} % Subfile Title
{Exame 2024.3 Resolução} % Part Title

\begin{questionBox}1{ % MARK: Q1
    \begin{itemize}
        \item \ch{{n} A -> A_n}
        \item Espessura difusão: \(\delta\)
        \item \(y_{A,0}\) concentração externa de A
        \item Reação instantanea: \(y_{A,1}=0\)
    \end{itemize}
} % Q1
\end{questionBox}

\begin{questionBox}2{ % MARK: Q1.1
    Geometria plana, comprimento: \textit{L}, largura \textit{W}
} % Q1.1
    \answer{}
    \begin{flalign*}
        &
            \text{Velocidade de isomerização}
            = &\\&
            = Q_A
            = - N_{A,z}\,{S,z}
            = -\frac
            {C\,\mathscr{D}_{A,B}}
            {1-\Theta\,y_A}
            \,\odv{y_A}{z}
            \,S_z
            % 
            % 
            % 
            \implies &\\[3ex]&
            \implies
            \int{Q_A\,\frac{\odif{z}}{S_z}}
            = Q_A
            \,\int_{z_0}^{z_1}{
                \frac{\odif{z}}{S_z}
            }
            =  &\\[3ex]&
            = \int{
                -\frac
                {C\,\mathscr{D}_{A,B}}
                {1-\Theta\,y_A}
                \,\odif{y_A}
            }
            = 
            \frac
            {C\,\mathscr{D}_{A,B}}
            {\Theta}
            \,\int_{y_{A,z_0}}^{y_{A,z_1}}{
                \frac{\odif{(1-\Theta\,y_A)}}{1-\Theta\,y_A}
            }
            = &\\&
            = 
            \frac
            {C\,\mathscr{D}_{A,B}}
            {\Theta}
            \ln{\frac
                {1-\Theta\,y_{A,z_1}}
                {1-\Theta\,y_{A,z_0}}
            }
            % 
            % 
            % 
            ; &\\[3ex]&
            \int_{z_0}^{z_1}{
                \frac{\odif{z}}{S_z}
            }
            = \int_{z_0}^{z_1}{
                \frac{\odif{z}}{L\,W}
            }
            = \frac{\adif{z}}{L\,W}
            = \frac{\delta}{L\,W}
            % 
            % 
            % 
            ; &\\[3ex]&
            \Theta
            = 1+\frac{N_{A_n}}{N_{A}}
            = 1+\frac{-N_{A}/n}{N_{A}}
            = 1-1/n
            % 
            % 
            % 
            \implies &\\[3ex]&
            \implies
            Q_A
            =
            \frac
            {C\,\mathscr{D}_{A,B}\,L\,W}
            {(1-1/n)\,\delta}
            \ln{\frac
                {1-(1-1/n)\,y_{A,z_1}}
                {1-(1-1/n)\,y_{A,z_0}}
            }
            = &\\&
            =
            \frac
            {C\,\mathscr{D}_{A,B}\,L\,W}
            {(1-1/n)\,\delta}
            \ln{(1-(1-1/n)\,y_{A,0})}
            % = &\\&
            % =
            % \frac
            % {C\,\mathscr{D}_{A,B}\,L\,W}
            % {(1-1/n)\,\delta}
            % \ln{(1-(1-1/n)\,y_{A,0})}
        &
    \end{flalign*}
\end{questionBox}

\begin{questionBox}2{ % MARK: Q1.2
    Geometria esférica, raio \textit{R}
} % Q1.2
    \answer{}
    \begin{flalign*}
        &
            Q_A
            = 
            \left(
                \int_{r_0}^{r_1}{
                    \frac{\odif{r}}{S_r}
                }
            \right)^{-1}
            \frac
            {C\,\mathscr{D}_{A,B}}
            {\Theta}
            \ln{\frac
                {1-\Theta\,y_{A,r_1}}
                {1-\Theta\,y_{A,r_0}}
            }
            % 
            % 
            % 
            ; &\\[3ex]&
            \int_{r_0}^{r_1}{
                \frac{\odif{r}}{S_r}
            }
            = \int_{r_0}^{r_1}{
                \frac{\odif{r}}{4\,\pi\,r^2}
            }
            = \frac{r_0^{-1}-r_1^{-1}}{4\,\pi}
            = \frac{(R)^{-1}-(R+\delta)^{-1}}{4\,\pi}
            % 
            % 
            % 
            ; &\\[3ex]&
            \Theta
            = 1+\frac{N_{A_n}}{N_{A}}
            = 1+\frac{-N_{A}/n}{N_{A}}
            = 1-1/n
            % 
            % 
            % 
            \implies &\\[3ex]&
            \implies
            Q_A
            = 
            \frac
            {
                C\,\mathscr{D}_{A,B}
                \,4\,\pi
            }
            {
                (1-1/n)
                \left(
                    R^{-1}-(R+\delta)^{-1}
                \right)
            }
            \ln{(1-(1-1/n)\,y_{A,0})}
        &
    \end{flalign*}
\end{questionBox}

\begin{questionBox}1{ % MARK: Q2
    \begin{itemize}
        \item Espessura liquido: \qty*{1}{\mm}
        \item Espessura ar: \qty*{4}{\mm}
        \item Ar estagnado: \(N_{ar}=0\)
        \item \(P=1\,\unit{\atm}\)
        \item \(T=25\,\unit{\celsius}\)
    \end{itemize}
} % Q2
    \paragraph*{Dados}
    \begin{align*}
        \mathscr{D}_{et}
        =&\qty*{0.132}{\cm^2/\s}
        ;&\rho_{et}
        =&\qty*{0.789}{\g/\cm^3}
        ;&M_{et}
        =&\qty*{46.1}{\g/\mole}
        \\P^*_{et}(25\unit{\celsius})
        =&\qty*{7.87}{\kilo\Pa}
    \end{align*}
\end{questionBox}

\begin{questionBox}2{ % MARK: Q2.1
    Tempo para et evap completamente
} % Q2.1
    \answer{}
    % MARK: Third
    \begin{flalign*}
        &
            -C_{A,L}
            \,\odv{V}{t}
            = -C_{A,L}
            \,\odv{(z\,S)}{t}
            = -C_{A,L}
            \,S
            \,\odv{z}{t}
            % 
            % 
            % 
            = &\\&
            = Q
            = &\\&
            =\left(
                \int_{z}^{5\E{-3}}{
                    \frac{\odif{z}}{S_z}
                }
            \right)^{-1}
            \frac
            {C\,\mathscr{D}_{A,B}}
            {\Theta}
            \ln{\frac
                {1-\Theta\,y_{A,z_1}}
                {1-\Theta\,y_{A,z}}
            }
            = &\\&
            =\left(
                \frac{\adif{z}\big\vert_{z}^{5\E{-3}}}{S}
            \right)^{-1}
            \frac
            {C\,\mathscr{D}_{A,B}}
            {\Theta}
            \ln{\frac
                {1-0}
                {1-\Theta\,y_{A,z_0}}
            }
            = &\\&
            =\frac
            {C\,\mathscr{D}_{A,B}\,S}
            {\Theta\,(5\E{-3}-z)}
            \ln{\frac
                {1}
                {1-\Theta\,y_{A,z_0}}
            }
            % 
            % 
            % 
            ; &\\[3ex]&
            \Theta
            = 1+\frac{N_{B}}{N_{A}}
            = 1+\frac{0}{N_{A}}
            = 1
            % 
            % 
            % 
            \implies &\\[3ex]&
            \implies
            \int_{z}^{0}{
                -C_{A,L}
                (5\E{-3}-z)
                \,\odif{z}
            }
            = C_{A,L}
            \int_{1\E{-3}}^{0}{(5\E{-3}-z)\odif{(5-z)}}
            = C_{A,L}\,((5\E{-3})^2-(5\E{-3}-1\E{-3})^2)/2
            = C_{A,L}\,4.5\E{-6}
            % 
            % 
            % 
            = &\\[3ex]&
            =\int_{0}^{t}{
                C\,\mathscr{D}_{A,B}
                \ln{\frac
                    {1}
                    {1-y_{A,z_0}}
                }
                \odif{t}
            }
            % = &\\&
            =
            C\,\mathscr{D}_{A,B}
            \ln{\frac
                {1}
                {1-y_{A,z}}
            }
            \int_{0}^{t}{
                \odif{t}
            }
            = &\\&
            =C\,\mathscr{D}_{A,B}
            \ln{\frac
                {1}
                {1-y_{A,z}}
            }\,t
            % 
            % 
            % 
            \implies &\\[3ex]&
            \implies
            t
            =
            \frac
            {C_{A,L}\,4.5\E{-6}}
            {C\,\mathscr{D}_{A,B}}
            \left(
                \ln{\frac
                    {1}
                    {1-y_{A,z}}
                }
            \right)^{-1}
            = &\\&
            =
            \frac
            {
                \left(
                    \rho_{et}/M_{et}
                \right)
                \,4.5\E{-6}
            }
            {
                \left(
                    \frac{P}{R\,T}
                \right)
                \,\mathscr{D}_{A,B}
            }
            \left(
                \ln{\frac
                    {1}
                    {1-P^*_{et}/P}
                }
            \right)^{-1}
            = &\\&
            =
            \frac
            {
                \rho_{Et}
                \,4.5\E{-6}
                \,R\,T
            }
            {
                P
                \,\mathscr{D}_{A,B}
                \,M_{Et}
            }
            \left(
                \ln{\frac
                    {1}
                    {1-P^*_{et}/P}
                }
            \right)^{-1}
            \cong &\\&
            \cong
            \frac
            {
                (0.789\E{6})
                * 4.5\E{-6}
                * \num{8.314462618}
                * 298.15
            }
            {
                1\E{5}
                * 0.132\E{-4}
                * 46.1
            }
            % 144.6384167067814262
            \left(
                \ln{\frac
                    {1}
                    {1-7.87\E3/1\E{5}}
                }
            \right)^{-1}
            % 12.199650272864029
            \cong&\\&
            \cong\qty
            {1764.538099843507187}
            {\second}
            % 
            % 
            % 
            ; &\\[3ex]&
            \text{Fronteiras para fluxo}
            &\\&
            \begin{cases}
                z_0=\qty*{1e-3}{\m}
                &\quad y_{A,z_0}\quad\text{(Concentração de et na superfície)}
                \\ z_1=z_0+\qty*{4e-3}{\m}
                &\quad y_{A,z}=0
                \quad\text{(C fora do ar estagnado)}
            \end{cases}
            ; &\\[3ex]&
            \text{Fronteiras para evaporação}
            &\\&
            \begin{cases}
                z_0=\qty*{1e-3}{\m}
                &\quad y_{A,z_0}\quad\text{(Fronteira da evap)}
                \\ z_1=0
                &\quad y_{A,z_1}=0
                \quad\text{(Evaporação completa)}
            \end{cases}
        &
    \end{flalign*}
\end{questionBox}

\begin{questionBox}2{ % MARK: Q2.2
    Efeitos de almentar a temperatura no tempo de evap
} % Q2.2
    \answer{}
    Principal efeito do almento da temperatura se percebe em \(\mathscr{D}_{et}\) onde
    \begin{BM}
        \mathscr{D}\propto T^{3/2}
    \end{BM}
    Tambem será possível ver o efeito na pressão de vapor do liquido contribuindo ainda mais para o fluxo
\end{questionBox}

\begin{questionBox}1{ % MARK: Q3
    \begin{itemize}
        \item Remover composto A
        \item \(T=15\,\unit{\celsius}\)
        \item \(P=2\,\unit{\atm}\)
        \item Contracorrente
        \item Faze gazosa entra na base
        \item \(y_{A,0}=20\%\)
        \item \(y_{A,1}=2\%\)
        \item \(10\)
    \end{itemize}
} % Q3
    body
\end{questionBox}

\end{document}
% !TEX root = ./FT_II-Tests_Resolutions.2020.1.tex
\providecommand\mainfilename{"./FT_II-Tests_Resolutions.tex"}
\providecommand \subfilename{}
\renewcommand   \subfilename{"./FT_II-Tests_Resolutions.2020.1.tex"}
\documentclass[\mainfilename]{subfiles}

% \tikzset{external/force remake=true} % - remake all

\begin{document}

% \graphicspath{{\subfix{./figures/FT_II-Tests_Resolutions.2020.1}}}
% \tikzsetexternalprefix{./figures/FT_II-Tests_Resolutions.2020.1/graphics/}

\mymakesubfile{1}
[FT II]
{Test 2020.1 Resolution} % Subfile Title
{Test 2020.1 Resolution} % Part Title

\begin{questionBox}1{ % MARK: Q1
    Uma partícula de carvão queima numa atmosfera gasosa (21\,\% de percentagem molar em oxigénio) a \qty*{1200}{\K}, à pressão atmosférica (\qty*{1.013e5}{\Pa}). O processo é limitado pela difusão de \ch{O2} em sentido oposto ao do \ch{CO2} que se forma, através de uma reacção instantânea com o carvão à sua superfície. O carvão, constituído por esferas com diâmetro de \qty*{0.2}{\mm}, consiste em carbono puro com uma massa específica de \qty*{1280}{\kg.\m^{-3}}.
    \begin{center}
        \large\bfseries\ch{
            C\sld{} + O2\gas{} -> CO2\gas{}
        }
    \end{center}
} % Q1
    \paragraph*{Considere}
    \begin{align*}
        \mathscr{D}_{\ch{O2},\text{mist gas}}
        &=\qty*{1e-4}{\m^2/\s}
        ;& R&=\qty{8.20573660809596}{\metre^3.\Pa/\mole.\kelvin}
    \end{align*}
    \paragraph*{Nota:} Nas alíneas seguintes, de b) a e) assuma que o processo de difusão ocorre em \emph{estado estacionário}.
\end{questionBox}

\begin{questionBox}2{ % MARK: Q1.1
    Faça uma estimativa do coeficiente de difusão do oxigénio no ar se a pressão for o dobro da presssão atmosférica e a temperatura for \qty*{1500}{\K}.
} % Q1.1
    \answer{}
    \begin{flalign*}
        &
            \begin{cases}
                A:& \ch{O2}
                \\ B:& \text{Mist gasosa}
            \end{cases}
            % 
            % 
            % 
            &\\[2ex]&
            \mathscr{D}_{A,B,1500\,\unit{\kelvin},2\,P_1\,\unit{\atm}}
            = \mathscr{D}_{A,B,1200\,\unit{\kelvin},P_1\,\unit{\atm}}
            \,\frac{P_1}{P_2}
            \,\left(
                \frac{T_2}{T_1}
            \right)^{3/2}
            = &\\&
            = 1\E{-4}
            \,\frac{P_1}{2\,P_1}
            \,\left(
                \frac{1500}{1200}
            \right)^{3/2}
            % \cong &\\&
            \cong\qty
            {6.98771242968684e-5}
            {\m/\s}
        &
    \end{flalign*}
\end{questionBox}

\begin{questionBox}2{ % MARK: Q1.2
    Voltando às condições referidas no enunciado, faça um esquema do processo que está a ocorrer, apresente a respectiva equação de conservação de massa e explicite as condições fronteira que considerou.
} % Q1.2
    \answer{}
    \begin{flalign*}
        &
            \begin{cases}
                    A: \ch{O2}
                \\  B: \ch{CO}
                \\  N_{\ch{CO}}=-N{\ch{O2}}
                \quad\text{Reação Instantanea}
            \end{cases}
            % 
            % 
            % 
            &\\[3ex]&
            N_{A,r}
            = -\frac
            {C\,\mathscr{D}_{A,B}}
            {1-\Theta\,y_A}
            \,\odv{y_A}{r}
            % 
            % 
            % 
            ; &\\[3ex]&
            \Theta
            = 1+N_{B}/N_A
            = 0
            % 
            % 
            % 
            ; &\\[3ex]&
            \therefore
            N_{A,r}
            = -C\,\mathscr{D}_{A,B}
            \,\odv{y_A}{r}
            % &\\&
            \quad
            \begin{cases}
                r_0=R
                \\ y_{A,0}=0
                &\quad\text{(Reação instantanea)}
                \\
                r_1=\infty
                &\quad\text{(Atmosfera)}
                \\
                y_{A,1}=0.21
                &\quad\text{(\unit{\percent\of{\ch{O2}}} na atmosfera)}
            \end{cases}
        &
    \end{flalign*}
\end{questionBox}

\begin{questionBox}2{ % MARK: Q1.3
    Com base na sua resposta à alínea b) deduza uma expressão para a velocidade de difusão do oxigénio.
} % Q1.3
    \answer{}
    \begin{flalign*}
        &
            Q
            = -N_{A,r}\,S_{r}
            = C\,\mathscr{D}_{A,B}
            \,\odv{y_A}{r}
            \,(4\,\pi\,r^2)
            \implies &\\&
            \implies
            \int_{r_0}^{\infty}{
                Q\,\frac{\odif{r}}{r^2}
            }
            = Q\,\int_{r_0}^{\infty}{
                \frac{\odif{r}}{r^2}
            }
            = -Q\,(0-r_0^{-1})
            % 
            % 
            % 
            = &\\[3ex]&
            = \int_{y_{A,r_0}}^{y_{A,\infty}}{
                C\,\mathscr{D}_{A,B}
                \,4\,\pi
                \,\odif{y_{r}}
            }
            = 
            C\,\mathscr{D}_{A,B}
            \,4\,\pi
            \int_{0}^{.21}{
                \odif{y_{A,r}}
            }
            = &\\&
            = 
            \left(
                \frac{P}{R\,T}
            \right)
            \,\mathscr{D}_{A,B}
            \,4\,\pi
            \,0.21
            \implies &\\[3ex]&
            \implies
            Q
            = \frac
            {
                P
                \,\mathscr{D}_{A,B}
                \,4\,\pi
                \,0.21
                \,r_1
            }
            {R\,T}
        &
    \end{flalign*}
\end{questionBox}

\begin{questionBox}2{ % MARK: Q1.4
    Calcule o valor da velocidade de difusão do oxigénio.
} % Q1.4
    \answer{}
    \begin{flalign*}
        &
            Q
            = \frac
            {
                P
                \,\mathscr{D}_{A,B}
                \,4\,\pi
                \,0.21
                \,r_1
            }
            {R\,T}
            = &\\&
            = \frac
            {
                1.013\E5
                * (1\E{-4})
                * 4\,\pi
                * 0.21
                * 0.1\E{-3}
            }
            {
                \num{8.314462618}
                *1200
            }
            \cong &\\&
            \cong\qty
            {2.679311283254509e-7}
            {\mole/\s}
        &
    \end{flalign*}
\end{questionBox}

\begin{questionBox}2{ % MARK: Q1.5
    Quanto tempo demora uma partícula de carvão a arder completamente?
} % Q1.5
    \answer{}
    \begin{flalign*}
        &   
            -C_{A,L}
            \,\odv{V}{t}
            =-C_{A,L}
            \,\odv{(4\,\pi\,r^3/3)}{t}
            =-C_{A,L}
            \,4\,\pi\,r^2
            \,\odv{r}{t}
            = &\\&
            = Q
            = \frac
            {
                P
                \,\mathscr{D}_{A,B}
                \,4\,\pi
                \,0.21
                \,r
            }
            {R\,T}
            % 
            % 
            % 
            \implies &\\[3ex]&
            \implies
            \int_{r_0}^{r_1}{
                -C_{A,L}
                \,4\,\pi\,r
                \,\odif{r}
            }
            = 
            -C_{A,L}
            \,4\,\pi
            \int_{r}^{0}{
                r\,\odif{r}
            }
            = &\\&
            = 
            -C_{A,L}
            \,4\,\pi
            (0^2-r^2)/2
            % 
            % 
            % 
            = &\\[3ex]&
            = \int_0^{t}{
                \frac
                {
                    P
                    \,\mathscr{D}_{A,B}
                    \,4\,\pi
                    \,0.21
                }
                {R\,T}
                \,\odif{t}
            }
            = 
            \frac
            {
                P
                \,\mathscr{D}_{A,B}
                \,4\,\pi
                \,0.21
            }
            {R\,T}
            \int_0^{t}{
                \,\odif{t}
            }
            = &\\&
            = 
            \frac
            {
                P
                \,\mathscr{D}_{A,B}
                \,4\,\pi
                \,0.21
            }
            {R\,T}
            t
            % 
            % 
            % 
            \implies &\\[3ex]&
            \implies
            t
            = \frac{
                C_{A,L}
                \,r^2
                \,R\,T
            }{
                P
                \,\mathscr{D}_{A,B}
                \,2
                *0.21
            }
            = \frac{
                (1280\E{3}/12)
                * (0.1\E{-3})^2
                * \num{8.314462618}
                * 1200
            }{
                (1.013\E{5})
                * 1\E{-4}
                * 2
                * 0.21
            }
        &
    \end{flalign*}
\end{questionBox}

\begin{questionBox}2{ % MARK: Q1.6
    Assuma agora que o processo de difusão ocorre em estado pseudo-estacionário. Nestas circunstâncias, quanto tempo demora uma partícula de carvão a arder completamente. Compare com o resultado obtido na alínea e) e comente.
} % Q1.6
    \answer{}
    \begin{flalign*}
        &
            Q
            =\frac{N}{t}
            \implies &\\&
            \implies
            t
            =\frac{N}{Q}
            =Q^{-1}
            \,\frac{V\,\rho}{M}
            =
            \frac{(4\,\pi\,r^3/3)\,\rho}{M\,Q}
            \cong
            \frac{
                (4*\pi\,(0.1\E{-3})^3/3)
                \,1280
            }{
                12\E{-3}
                \,\num{2.679311283254509e-7}
            }
            \cong &\\&
            \cong
            \qty
            {1.66760872953822525}
            {\s}
        &
    \end{flalign*}
\end{questionBox}

\end{document}
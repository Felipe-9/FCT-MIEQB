% !TEX root = ./FT_II-Slides_Annotations.4.tex
\providecommand\mainfilename{"./FT_II-Slides_Annotations.tex"}
\providecommand \subfilename{}
\renewcommand   \subfilename{"./FT_II-Slides_Annotations.4.tex"}
\documentclass[\mainfilename]{subfiles}

% \tikzset{external/force remake=true} % - remake all

\begin{document}

\graphicspath{{\subfix{./.build/figures/FT_II-Slides_Annotations.4}}}
% \tikzsetexternalprefix{./.build/figures/FT_II-Slides_Annotations.4/}

\mymakesubfile{4}
[FT II]
{Anotações: Difusão em estádo estacionário} % Subfile Title
{Difusão em estádo estacionário} % Part Title

\begin{sectionBox}1{Difusão em estado estacionário para todas as geometrias} % S

    \begin{BM}
        N_{A,z_1}
        = \frac{c\,D_{A,B}}{\Theta\,\eta_d\,z_1}
        \,\ln\frac
        {1-\Theta\,y_{A,1}}
        {1-\Theta\,y_{A,0}}
        % \\[2ex]
        ; \qquad
        \Theta = 1+\frac{N_{B,z}}{N_{A,z}}
    \end{BM}

    \begin{multicols}{2}
        \paragraph*{\(z\):} Dimensão caracteristica
        \begin{itemize}
            \item \textit{z} para película plana
            \item \(r_1\) para cilindros e esferas
        \end{itemize}

        \begin{minipage}{1\textwidth}
            \paragraph*{\(\eta_d\):} Fator adimensional
            \begin{itemize}
                \item 1 para plano
                \item \(\ln(r_2/r_1)\) para cilíndro
                \item \(1-r_2/r_1\) para esféra
            \end{itemize}
        \end{minipage}
    \end{multicols}
    
\end{sectionBox}

\begin{sectionBox}2{Geometria cilíndrica} % S
    
    \section*{Equação de concervação}
    \begin{BM}
        2\,\pi\,r\,L\,N_{A,r}\big\vert_{r}
        = 2\,\pi\,r\,L\,N_{A,r}\big\vert_{r+\adif{r}}
        \implies
        \begin{cases}
            r\,N_{A,r} \text{ é constante}
            \\ 
            r\,N_{B,r} \text{ é constante}
        \end{cases}
    \end{BM}
    
\end{sectionBox}

\begin{sectionBox}2{Casos espeçíficos em dif plana} % S
    
    \subsection*{Difusão através de componente estagnado} % S
    \begin{BM}
        N_{A,z_1}
        = \frac{c\,\mathscr{D}_{A,B}}{z_1}
        \,\ln\frac
        {1-y_{A,1}}
        {1-y_{A,0}}
        \quad
        N_B=0
    \end{BM}

    \subsection*{Contra Difusão Equimolar}
    \begin{BM}
        N_{A,z_1}
        =-\frac
        {c\,\mathscr{D}_{A,B}}
        {z_1}
        (y_{A,1}-y_{A,0})
        ;\quad
        N_B=-N_A
    \end{BM}
    
\end{sectionBox}

\begin{exampleBox}1{ % MARK:E1
    Um componente A difunde-se através de uma camada em repouso de um componente \textit{B} de espessura \textit{Z}. A pressão parcial de \textit{A} num dos lados da camada é \(p_{A,1}\) e no outro lado \(p_{A,2} < p_{A,1}\). Mostre que o fluxo máximo possível de A através dessa camada é dado por:
    \begin{BM}
        N_{A\,\max{}}
        = \frac{\mathscr{D}\,P}{R\,T\,Z}
        \,\ln\frac{P}{P-p_{A,1}}
    \end{BM}
} % E1
    \answer{}
    \begin{flalign*}
        &
            N_{A,\max,z}
            = \frac
            {c\,\mathscr{D}_{A,B}}
            {\Theta\,\eta_d\,z}
            \ln{\frac
                {1-\Theta\,y_{A,2}}
                {1-\Theta\,y_{A,1}}
            }
            = \frac
            {
                \frac{P}{R\,t}
                \,\mathscr{D}_{A,B}
            }
            {\Theta\,\eta_d\,z}
            \ln{\frac
                {1-\Theta\,y_{A,2}}
                {1-\Theta\,y_{A,1}}
            }
            % 
            % 
            % 
            ; &\\[3ex]&
            N_{A,\max,z}\implies y_{A,2}=0
            % 
            % 
            % 
            ; &\\[3ex]&
            \Theta
            = 1+N_{B,z}/N_{A,z}
            = 1
            % 
            % 
            % 
            ; &\\[3ex]&
            \therefore
            N_{A,\max,z}
            = \frac
            {P\,\mathscr{D}_{A,B}}
            {R\,T\,(1)\,z}
            \ln{\frac
                {1}
                {1-y_{A,1}}
            }
            = &\\&
            = \frac
            {P\,\mathscr{D}_{A,B}}
            {R\,T\,z}
            \ln{\frac
                {1}
                {1-P_{A,1}/P}
            }
            = \mathemph{
                \frac
                {P\,\mathscr{D}_{A,B}}
                {z\,R\,T}
                \ln{\frac
                    {P}
                    {P-P_{A,1}}
                }
            }
        &
    \end{flalign*}

    \answer{}
    \begin{center}
        \includegraphics[width=.8\textwidth]{IMG_6431}
    \end{center}
    
\end{exampleBox}

\begin{exampleBox}1{ % MARK: E2
    Moldou-se naftaleno sob a forma de um cilindro de raio \(R_1\) que se deixou sublimar no ar em repouso. Mostre que a velocidade de sublimação é dada por:
    \begin{BM}
        Q
        =\frac
        {2\,\pi\,L\,\mathscr{D}\,P}
        {
            R\,T
            \,\ln(R_2/R_1)
        }
        \,\ln{\frac
            {1-y_{A,2}}
            {1-y^*_{A}}
        }
    \end{BM}
    Sendo a fracção molar correspondente à pressão de vapor do naftaleno e \(y_{A,2}\) a fracção molar correspondente a \(R_2\).\\
    Explique o que sucede à velocidade de sublimação quando \(R_2\) se torna muito grande.
} % E2
    \answer{}
    \begin{flalign*}
        &
            Q
            = N_{A,R_1}\,S_{R_1}
            = \frac
            {c\,\mathscr{D}_{A,B}}
            {\Theta\,R_1\,\ln{(R_2/R_1)}}
            \,\ln{\frac
                {1-\Theta\,y_{A,2}}
                {1-\Theta\,y_{A,1}}
            }
            \,(2\,\pi\,R_1\,L)
            = &\\&
            = \frac{
                \left(\frac{P}{R\,T}\right)
                \,\mathscr{D}_{A,B}
                \,2\,\pi\,L
            }
            {\Theta\,\ln{(R_2/R_1)}}
            \,\ln{\frac
                {1-\Theta\,y_{A,2}}
                {1-\Theta\,y_{A,1}}
            }
            % 
            % 
            % 
            ; &\\[3ex]&
            \Theta
            = 1+N_{B}/N_{A}
            = 1+0/N_{A}
            = 1
            % 
            % 
            % 
            ; &\\[3ex]&
            \therefore
            Q
            % = &\\&
            = \frac
            {
                P
                \,\mathscr{D}_{A,B}
                \,2\,\pi\,L
            }
            {R\,T\,\ln{(R_2/R_1)}}
            \,\ln{\frac
                {1-y_{A,2}}
                {1-y^*_{A}}
            }
        &
    \end{flalign*}

    \answer{}
    \begin{center}
        \includegraphics[width=.8\textwidth]{IMG_6432}
    \end{center}
    \paragraph*{Maximizar Q} removemos \(y_{A,2}\)
    
\end{exampleBox}

\begin{exampleBox}2{ % E2.1
    E se a geometria for esférica
} % E2.1
    \answer{}
    \begin{flalign*}
        &
            Q
            = N_{A,R_1}\,S_{R_1}
            = \frac
            {c\,\mathscr{D}_{A,B}}
            {\Theta\,R_1(1-R_1/R_2)}
            \,\ln{\frac
                {1-\Theta\,y_{A,2}}
                {1-\Theta\,y_{A,1}}
            }
            \,(4\,\pi\,R_1^2)
            = &\\&
            = \frac
            {
                \left(
                    \frac{P}{R\,T}
                \right)
                \,\mathscr{D}_{A,B}
                \,4\,\pi
            }
            {R_1^{-1}-R_2^{-1}}
            \,\ln{\frac
                {1-y_{A,2}}
                {1-y^*_{A}}
            }
            % 
            % 
            % 
            ; &\\[3ex]&
            \lim_{R_2\to\infty}{Q}
            = \lim_{R_2\to\infty}{
                \frac
                {
                    P\,\mathscr{D}_{A,B}
                    \,4\,\pi
                }
                {R\,T\,(R_1^{-1}-R_2^{-1})}
                \,\ln\frac
                {1-y_{A,2}}
                {1-y^*_{A}}
            }
            = &\\&
            = 
            \frac
            {
                P\,\mathscr{D}_{A,B}
                \,4\,\pi
            }
            {
                R\,T
                \,\lim_{R_2\to\infty}{
                    {(R_1^{-1}-R_2^{-1})}
                }
            }
            \,\ln\frac
            {1-y_{A,2}}
            {1-y^*_{A}}
            = &\\&
            = 
            \frac
            {
                P\,\mathscr{D}_{A,B}
                \,4\,\pi
            }
            {
                R\,T
                (R_1^{-1})
            }
            \,\ln\frac
            {1-y_{A,2}}
            {1-y^*_{A}}
        &
    \end{flalign*}

    \answer{}
    \begin{center}
        \includegraphics[width=.8\textwidth]{IMG_6433}
    \end{center}
    
\end{exampleBox}

\end{document}
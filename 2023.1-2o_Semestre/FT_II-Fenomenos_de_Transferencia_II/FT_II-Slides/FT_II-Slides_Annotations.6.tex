% !TEX root = ./FT_II-Slides_Annotations.6.tex
\providecommand\mainfilename{"./FT_II-Slides_Annotations.tex"}
\providecommand \subfilename{}
\renewcommand   \subfilename{"./FT_II-Slides_Annotations.6.tex"}
\documentclass[\mainfilename]{subfiles}

% \tikzset{external/force remake=true} % - remake all

\begin{document}

\graphicspath{{\subfix{./.build/figures/FT_II-Slides_Annotations.6}}}
% \tikzsetexternalprefix{./.build/figures/FT_II-Slides_Annotations.6/graphics/}

\mymakesubfile{6}
[FT II]
{Difusão com reações químicas} % Subfile Title
{Difusão com reações químicas} % Part Title

\begin{sectionBox}1{Sistemas unidimensionais associados com reação química} % MARK: S
    
    \begin{multicols}{2}
        \begin{sectionBox}*2bm{Sistema homogênio} % MARK: S
            Quando a reação ocorre uniformemente em uma unica fase
        \end{sectionBox}
        \begin{sectionBox}*2bm{Sistema Heterogenio} % MARK: S
            Quando a reação ocorre na fronteira entre duas fases
        \end{sectionBox}
    \end{multicols}

    \paragraph*{Reações Homogênias}
    A frequencia de aparecimento da especie A por uma reação homogenia se da pela equação geral diferencial:
    \begin{BM}
        \gdif{N_A}
        + \pdv{C_{A}}{t}
        - R_A
        = 0
    \end{BM}
    \paragraph*{\(R_A\)} é a frequencia em que A varia pela reação química, ex. em uma reação de ordem 1: \(R_A=-k_1\,C_{A}\)

    \paragraph*{Reações Heterogénias}
    Geralmente envolvem uma camada de fluido onde A está disolvido e uma camada sólida onde ocorre a reação, como a frequencia de aparecimento fica dependente da area de contato temos que os fluxos dos componentes são dependentes um dos outros de acordo com a reação
    \paragraph*{Exemplo}
    \begin{center}
        \ch{O2\gas{} + C\sld{} -> CO2\gas{}}
    \end{center}
    De onde tiramos que \(N_{\ch{O2}}=-N_{\ch{CO2}}\)
    
\end{sectionBox}

% W_{O2} = Q_{O2}

\begin{exampleBox}1{ % #MARK: E1
    A seguinte reação queima uma esfera de carbono de raio \textit{r} com \ch{O2\gas{}} puro, encontre o a equação para a velocidade de difusão molar do oxigenio
    \begin{center}\Large
        \ch{2 C\sld{} + O2\gas{} -> 2 CO\gas{}}
    \end{center}
} % E1
    \begin{flalign*}
        &
            Q_{\ch{O2}}
            = W_{\ch{O2}}
            &\\& 
        &
    \end{flalign*}

    \paragraph*{Reações instantâneas:} 
    \(y_{\ch{O2}\vert{}R}=0\) 
    completamente absorvido pela superfície
    \paragraph*{Reações \ch{O2} Puro:} 
    \(y_{\ch{O2}\vert{}\infty}=1\)

    \answer{}
    \begin{flalign*}
        &
            Q_{\ch{O2}}
            = S
            \,N_{\ch{O2}}
            = \left(
                4\,\pi\,r^2
            \right)
            \,\left(
                \frac
                {C_{A,L}\,\mathscr{D}_{A,B}}
                {\Theta\,\eta_d\,r}
                \ln{\frac
                    {1-\Theta\,y_{A,2}}
                    {1-\Theta\,y_{A,1}}
                }
            \right)
            = &\\&
            = \frac
            {
                4\,\pi\,r^2
                \,C_{A,L}\,\mathscr{D}_{A,B}
            }
            {(-1)*1\,r}
            \ln{\frac
                {1-(-1)\,(1)}
                {1-(-1)\,(0)}
            }
            = &\\&
            = \mathemph{
                -4\,\pi\,r
                \,C_{A,L}\,\mathscr{D}_{A,B}
                \,\ln{2}
            }
            % 
            % 
            % 
            ; &\\[3ex]&
            \begin{cases}
                R_1=r;&\quad y_{1}=0
                \\ R_2\to\infty;&\quad y_{2}=1
            \end{cases}
            % 
            % 
            % 
            ; &\\[3ex]&
            \eta_{d,\text{esfera}}
            = 1-R_1/R_2
            \to 1
            % 
            % 
            % 
            ; &\\[3ex]&
            \Theta
            = 1+N_{\ch{CO}}/N_{\ch{O2}}
            = 1+(-2\,N_{\ch{O2}})/N_{\ch{O2}}
            = -1
        &
    \end{flalign*}

    % \begin{center}
    %     \includegraphics[width=.8\textwidth]{Image.jpeg}
    % \end{center}
    
\end{exampleBox}

\begin{exampleBox}1{ % #MARK: E2
    Obtenha uma expressão para o fluxo molar de A quando numa superfície catalítica ocorre a reacção instantânea \ch{{n} A -> A_{n}}. A difusão de A dá-se através de uma camada de espessura \textit{l} e a fracção molar de A no exterior dessa camada é \(y_{A,0}\).
} % E2
    \answer{}
    \begin{flalign*}
        &
            N_{A,z}
            = \frac
            {C_{A}\,\mathscr{D}_{A,B}}
            {\Theta\,\eta_d\,l}
            \ln{\frac
                {1-\Theta\,y_{A,1}}
                {1-\Theta\,y_{A,0}}
            }
            = &\\&
            = \frac
            {\frac{P}{R\,T}\,\mathscr{D}_{A,B}}
            {(1-1/n)\,1\,l}
            \ln{\frac
                {1-(1-1/n)\,(0)}
                {1-(1-1/n)\,y_{A,0}}
            }
            = &\\&
            = \mathemph{
                \frac
                {P\,\mathscr{D}_{A,B}}
                {R\,T\,(1-1/n)\,l}
                \ln{\frac
                    {1}
                    {1-(1-1/n)\,y_{A,0}}
                }
            }
            % 
            % 
            % 
            ; &\\[3ex]&
            \begin{cases}
                    z_0=0;&\quad y_{A,0}
                \\  z_1=l;&\quad y_{A,1}=0
            \end{cases}
            % 
            % 
            % 
            ; &\\[3ex]&
            \eta_{d,\text{plano}}=1
            % 
            % 
            % 
            ; &\\[3ex]&
            \Theta
            = 1+N_{A_n}/N_{A}
            = 1+(-N_{A}/n)/N_{A}
            = 1-1/n
        &
    \end{flalign*}
    
\end{exampleBox}

\begin{exampleBox}1{ % #MARK: E3
    Um cilindro de aço, cuja superfície está revestida por um catalisador, é usado para promover a reacção de dimerização de um composto gasoso A (\ch{2 A -> A2}), à pressão atmosférica e à temperatura de \qty*{50}{\celsius}. Este composto, com uma pressão parcial de \qty*{0.39}{\atm}, difunde-se estacionariamente até à superfície do cilindro, sendo a velocidade de difusão limitada pela difusão de A através de um filme gasoso com \qty*{6}{\milli\metre} de espessura.
} % E3
    
    \begin{itemize}
        \begin{multicols}{3}
            \item \(D_A=\qty*{2.5E-5}{\metre^2/\second}\)
            \item \(d=\qty*{5}{\centi\metre}\)
            \item \(L=\qty*{10}{\centi\metre}\)
        \end{multicols}
    \end{itemize}

    \begin{exampleBox}2{ % #MARK: E3.1
        Determine a velocidade de difusão de A para o caso em que a reacção ocorre somente na superfície lateral exterior do cilindro.
    } % E3.1
        \answer{}
        \begin{flalign*}
            &
                Q_{A,R}
                = S\,N_{A,R}
                = \left(
                    2\,\pi\,r_0\,L
                \right)\,\left(
                    \frac
                    {C_{A}\,\mathscr{D}_{A,B}}
                    {\Theta\,\eta_d\,r_0}
                    \ln{\frac
                        {1-\Theta\,y_{A,1}}
                        {1-\Theta\,y_{A,0}}
                    }
                \right)
                = &\\&
                = \frac
                {
                    2\,\pi\,L
                    \,\frac{P}{R\,T}
                    \,\mathscr{D}_{A,B}
                }
                {(1/2)\,\left(\ln{\frac{r_1}{r_0}}\right)}
                \ln{\frac
                    {1-(1/2)\,y_{A,1}}
                    {1-(1/2)\,(0)}
                }
                = &\\&
                = \frac
                {
                    4\,\pi\,L
                    \,P
                    \,\mathscr{D}_{A,B}
                }
                {R\,T\,\ln{\frac{r_1}{r_0}}}
                \ln{(1-y_{A,1}/2)}
                \cong &\\&
                \cong \frac
                {
                    4\,\pi\,10\E{-2}
                    * 1
                    * 2.5\E{-5}
                }
                {
                    \num{8.20573660809596e-5}
                    * (50+273.15)
                    * \ln{\frac
                        {2.5\E{-2}+6\E{-3}}
                        {2.5\E{-2}}
                    }
                }
                \ln{(1-0.39/2)}
                \cong &\\&
                \cong\mathemph{
                    \qty
                    {-1.19467654735535e-3}
                    {\mole/\m^2\,\s}
                }
                % 
                % 
                % 
                ; &\\[3ex]&
                \text{Condições de fronteira do fluxo}:&\\&
                \begin{cases}
                        r_0=2.5\E{-2};&\quad y_{A,0}=0
                    \\  r_1=2.5\E{-2}+6\E{-3}
                    ;&\quad y_{A,1}=P_{A}/P=0.39/1
                \end{cases}
                % 
                % 
                % 
                ; &\\[3ex]&
                \Theta
                = 1+N_{A_2}/N_{A}
                = 1+(-N_{A}/2)/N_{A}
                = 1/2
                % 
                % 
                % 
                ; &\\[3ex]&
                \eta_{d,\text{Cilindro}}
                = \ln{\frac{r_1}{r_0}}
            &
        \end{flalign*}
        
    \end{exampleBox}
    \begin{exampleBox}2{ % #MARK: E3.2
        Calcule a velocidade de difusão de A para o caso em que a reacção ocorre somente numa das bases do cilindro.
    } % E3.2
        
        \answer{}
        \begin{flalign*}
            &
                Q_{A,z}
                = S\,N_{A,z}
                = \left(
                    \pi\,r_0^2
                \right)\,\left(
                    \frac
                    {C_A\,\mathscr{D}_{A,B}}
                    {\Theta\,\eta_d\,l}
                    \ln{\frac
                        {1-\Theta\,y_{A,1}}
                        {1-\Theta\,y_{A,0}}
                    }
                \right)
                = &\\&
                = \frac
                {
                    \pi\,r_0^2
                    \,\frac{P}{R\,T}\,\mathscr{D}_{A,B}
                }
                {
                    (1/2)\,1\,l
                }
                \ln{\frac
                    {1-(1/2)\,y_{A,1}}
                    {1-(1/2)\,(0)}
                }
                = &\\&
                = \frac
                {
                    2\,\pi\,r_0^2
                    \,P
                    \,\mathscr{D}_{A,B}
                }
                {
                    R\,T\,l
                }
                \ln{(1-y_{A,1}/2)}
                \cong &\\&
                \cong \frac
                {
                    2\,\pi
                    *  (2.5\E{-2})^2
                    *  1
                    *  2.5\E{-5}
                }
                {
                    \num{8.20573660809596e-5}
                    * (50+273.15)
                    * 6\E{-3}
                }
                \ln{(1-0.39/2)}
                \cong &\\&
                \cong\mathemph{
                    \qty
                    {-1.338481876550095e-4}
                    {\mole/\m^2\,\s}
                }
                % 
                % 
                % 
                ; &\\[3ex]&
                \text{Condições de fronteira}&\\&
                \begin{cases}
                        z_0=0;\quad& y_{A,0}=0
                    \\  z_1=6\E{-3};\quad& y_{A,1}=0.39
                \end{cases}
                % 
                % 
                % 
                ; &\\[3ex]&
                \eta_{d,\text{plano}}
                = 1
                % 
                % 
                % 
                ; &\\[3ex]&
                \Theta = 1/2
            &
        \end{flalign*}
        
    \end{exampleBox}
    
\end{exampleBox}

\end{document}
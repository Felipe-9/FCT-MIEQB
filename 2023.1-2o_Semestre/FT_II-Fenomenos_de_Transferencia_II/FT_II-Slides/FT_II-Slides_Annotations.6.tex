% !TEX root = ./FT_II-Slides_Annotations.6.tex
\providecommand\mainfilename{"./FT_II-Slides_Annotations.tex"}
\providecommand \subfilename{}
\renewcommand   \subfilename{"./FT_II-Slides_Annotations.6.tex"}
\documentclass[\mainfilename]{subfiles}

% \tikzset{external/force remake=true} % - remake all

\begin{document}

\graphicspath{{\subfix{./.build/figures/FT_II-Slides_Annotations.6}}}
% \tikzsetexternalprefix{./.build/figures/FT_II-Slides_Annotations.6/graphics/}

\mymakesubfile{6}
[FT II]
{Reação Química Heterogénea} % Subfile Title
{Reação Química Heterogénea} % Part Title

% W_{O2} = Q_{O2}

\begin{exampleBox}1{ % #MARK: E1
    \ch{2 C + O2 -> 2 CO}
} % E1
    \begin{flalign*}
        &
            Q_{\ch{O2}}
            = W_{\ch{O2}}
            &\\& 
        &
    \end{flalign*}

    \paragraph*{Reações instantâneas:} 
    \(y_{\ch{O2}\vert{}R}=0\) 
    completamente absorvido pela superfície
    \paragraph*{Reações \ch{O2} Puro:} 
    \(y_{\ch{O2}\vert{}\infty}=1\)

    \answer{}
    \begin{flalign*}
        &
            Q_{\ch{O2}}
            = 4\,\pi\,r^2
            \,N_{\ch{O2}}
            % 
            % 
            % 
            ; &\\[3ex]&
            \text{\emph{Fluxo Molar \ch{O2}:}}
            &\\&
            N_{\ch{O2}}
            = y_{\ch{O2}}
            \,(N_{\ch{O2}}+N_{\ch{CO}})
            - C\,\mathscr{D}
            \,\odv{y_{\ch{O2}}}{r}
            % 
            % 
            % 
            ; &\\[3ex]&
            \text{\emph{Fluxo Molar \ch{CO}:}}
            &\\&
            \frac{N_{\ch{CO}}}{2}
            = \frac{N_{\ch{O2}}}{-1}
            \implies
            N_{\ch{CO}}
            = -2\,N_{\ch{O2}}
            % 
            % 
            % 
            \implies &\\[3ex]&
            \implies
            \int_{R}^{\infty}{
                N_{\ch{O2}}
                \,\odif{r}
            }
            = \int_{R}^{\infty}{
                \frac{Q_{\ch{O2}}}{4\,\pi\,r^2}
                \,\odif{r}
            }
            = \frac{Q_{\ch{O2}}}{4\,\pi}
            \,\int_{R}^{\infty}{
                \frac{\odif{r}}{r^2}
            }
            = &\\&
            = -\frac{Q_{\ch{O2}}}{4\,\pi}
            \,\adif{(1/r)}
            \,\big\vert_{R}^{\infty}
            = \frac{Q_{\ch{O2}}}{4\,\pi\,R}
            % 
            % 
            % 
            = &\\[3ex]&
            = \int_{y_{\ch{O2}\vert{R}}}^{y_{\ch{O2}\vert{\infty}}}
            - C\,\mathscr{D}
            \,\frac
            {\odif{y_{\ch{O2}}}}
            {y_{\ch{O2}}+1}
            = 
            - C\,\mathscr{D}
            \,\int_{0}^{1}
            \,\frac
            {\odif{y_{\ch{O2}}+1}}
            {y_{\ch{O2}}+1}
            = 
            - C\,\mathscr{D}
            \adif{\left(
                \ln{(y_{\ch{O2}}+1)}
            \right)}
            \Big\vert_{0}^{1}
            = &\\&
            = 
            - C\,\mathscr{D}
            \ln\frac{1+1}{0+1}
            = - C\,\mathscr{D}\ln{2}
            % 
            % 
            % 
            \implies &\\[3ex]&
            \implies
            Q_{\ch{O2}}
            = - 4\,\pi\,R
            \, C\,\mathscr{D}\ln{2}
        &
    \end{flalign*}

    % \begin{center}
    %     \includegraphics[width=.8\textwidth]{Image.jpeg}
    % \end{center}
    
\end{exampleBox}

\begin{exampleBox}1{ % #MARK: E2
    Obtenha uma expressão para o fluxo molar de A quando numa superfície catalítica ocorre a reacção instantânea \ch{{n} A -> A_{n}}. A difusão de A dá-se através de uma camada de espessura \textit{l} e a fracção molar de A no exterior dessa camada é \(y_{A,0}\).
} % E2
    \answer{}
    \begin{flalign*}
        &
            \text{Fluxo em geometria plana:}
            &\\&
            Q_{A,z}=N_{A,z}
            % 
            % 
            % 
            ; &\\[3ex]&
            \text{Fluxo Molar:}
            &\\&
            N_A
            = -C\,\mathscr{D}\,\odv{y_A}{z}
            + y_A\,(N_A+N_{A_n})
            = -C\,\mathscr{D}\,\odv{y_A}{z}
            + y_A\,(N_A+N_A/n)
            = &\\&
            = -C\,\mathscr{D}\,\odv{y_A}{z}
            + y_A\,N_A\,(1-1/n)
            \implies &\\[2ex]&
            \implies
            % 
            % 
            % 
            \int_{0}^{l}{N_A\,\odif{z}}
            = N_A\,\int_{0}^{l}{\odif{z}}
            = N_A\,l
            % 
            % 
            % 
            = &\\[3ex]&
            = \int_{0}^{y_{A,0}}{
                -C\,\mathscr{D}
                \,\frac{\odif{y_A}}
                {1-y_A\,(1-1/n)}
            }
            = -C\,\mathscr{D}
            \,\int_{0}^{y_{A,0}}{
                \frac{\odif{y_A}}
                {1-y_A\,(1-1/n)}
            }
            = &\\&
            = \frac{-C\,\mathscr{D}}{(1-1/n)}
            \,\int_{0}^{y_{A,0}}{
                \frac{\odif{(1-y_A\,(1-1/n))}}
                {1-y_A\,(1-1/n)}
            }
            = \frac{-C\,\mathscr{D}}{(1-1/n)}
            \ln{(1-y_{A,0}(1-1/n))}
            \implies &\\[3ex]&
            \implies
            % 
            % 
            % 
            N_A
            = \frac{-C\,\mathscr{D}}{l\,(1-1/n)}
            \ln{(1-y_{A,0}(1-1/n))}
        &
    \end{flalign*}
    
\end{exampleBox}

\begin{exampleBox}1{ % #MARK: E3
    Um cilindro de aço, cuja superfície está revestida por um catalisador, é usado para promover a reacção de dimerização de um composto gasoso A (\ch{2 A -> A2}), à pressão atmosférica e à temperatura de \qty*{50}{\celsius}. Este composto, com uma pressão parcial de \qty*{0.39}{\atm}, difunde-se estacionariamente até à superfície do cilindro, sendo a velocidade de difusão limitada pela difusão de A através de um filme gasoso com \qty*{6}{\milli\metre} de espessura.
} % E3
    
    \begin{itemize}
        \begin{multicols}{3}
            \item \(D_A=\qty*{2.5E-5}{\metre^2/\second}\)
            \item \(d=\qty*{5}{\centi\metre}\)
            \item \(L=\qty*{10}{\centi\metre}\)
        \end{multicols}
    \end{itemize}

    \begin{exampleBox}2{ % #MARK: E3.1
        Determine a velocidade de difusão de A para o caso em que a reacção ocorre somente na superfície lateral exterior do cilindro.
    } % E3.1
        \answer{}
        \begin{flalign*}
            &
                \text{Fluxo em geometria cilindrica:}
                &\\&
                Q_A = N_{A,R}\,2\,\pi\,r\,L
                % 
                % 
                % 
                ;&\\[3ex]&
                \text{Fluxo molar:}
                &\\&
                N_{A,R}
                = -C\,\mathscr{D}\,\odv{y_A}{r}
                + y_A\,(N_{A,R}+N_{A_2,R})
                = &\\&
                = -C\,\mathscr{D}\,\odv{y_A}{r}
                + y_A\,(N_{A,R}-(1/2)\,N_{A,R})
                = &\\&
                = -C\,\mathscr{D}\,\odv{y_A}{r}
                + y_A\,N_{A,R}/2
                % 
                % 
                % 
                \implies &\\[3ex]&
                \implies
                \int_{R}^{R+\delta}{
                    N_{A,R}\,\odif{r}
                }
                = \int_{R}^{R+\delta}{
                    \frac{Q_{A,R}}{2\,\pi\,r\,L}
                    \,\odif{r}
                }
                = \frac{Q_{A,R}}{2\,\pi\,L}
                \,\int_{R}^{R+\delta}{\frac{\odif{r}}{r}}
                = \frac{Q_{A,R}}{2\,\pi\,L}
                \ln\frac{R+\delta}{R}
                % 
                % 
                % 
                = &\\[3ex]&
                = \int_0^{y_{A,0}}{
                    -C\,\mathscr{D}
                    \,\frac{\odif{y_A}}{1-y_A/2}
                }
                = \frac{C\,\mathscr{D}}{1/2}
                \,\int_0^{y_{A,0}}{
                    \frac{\odif{(1-y_A/2)}}{1-y_A/2}
                }
                = \frac{C\,\mathscr{D}}{1/2}
                \,\ln\frac{1-y_{A,0}/2}{1-0}
                = &\\&
                = \frac{C\,\mathscr{D}}{1/2}
                \,\ln\frac{1-p_{A,0}/2\,P}{1-0}
                % 
                % 
                % 
                \implies &\\[6ex]&
                \implies
                Q_{A,R}
                = \frac{
                    \frac{P}{R\,T}\,\mathscr{D}
                    \,2\,\pi\,L
                }{(1/2)\ln\frac{R+\delta}{R}}
                \,\ln(1-p_{A,0}/2\,P)
                = &\\&
                = \frac{
                    \frac{1}{\num{82.0573660809596}*323.15}
                    * 2.5\E{-1}
                    * 2*\pi*10
                }{(1/2)\ln\frac{2.5+0.6}{2.5}}
                % 0.005507629965672
                \,\ln(1-0.39/2*1)
                % -0.216913001563574
                \cong
                \qty[2]{-1.194676547355e-3}{\mole/\second}
                % -0.001194676547355
            &
        \end{flalign*}
        
    \end{exampleBox}
    \begin{exampleBox}2{ % #MARK: E3.2
        Calcule a velocidade de difusão de A para o caso em que a reacção ocorre somente numa das bases do cilindro.
    } % E3.2
        
        \answer{}
        \begin{flalign*}
            &
                \text{Fluxo em geometria plana:}
                &\\&
                Q_{A,z}=N_{A,z}
                % 
                % 
                % 
                ; &\\[3ex]&
                \text{Fluxo molar:}
                &\\&
                N_{A,z}
                = \frac{C\,\mathscr{D}}{\delta/2}
                \,\ln(1-y_{A,0}/2)
                = \frac{C\,\mathscr{D}}{\delta/2}
                \,\ln(1-p_{A,0}/2*P)
                % 
                % 
                % 
                = &\\&
                = \frac{
                    \frac{1}{
                        \num{82.0573660809596}
                        *323.15
                    }
                    *0.25
                }{0.6/2}
                \,\ln(1-0.39/2*1)
                \cong
                \SI{-6.816832220539635e-6}{\mole/\second}
            &
        \end{flalign*}
        
    \end{exampleBox}
    
\end{exampleBox}

\end{document}
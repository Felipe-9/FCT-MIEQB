% !TEX root = ./FT_II-Slides_Annotations.2024.3.1.tex
\providecommand\mainfilename{"./FT_II-Slides_Annotations.tex"}
\providecommand \subfilename{}
\renewcommand   \subfilename{"./FT_II-Slides_Annotations.2024.3.1.tex"}
\documentclass[\mainfilename]{subfiles}

% \tikzset{external/force remake=true} % - remake all

\begin{document}

% \graphicspath{{\subfix{./figures/FT_II-Slides_Annotations.2024.3.1}}}
% \tikzsetexternalprefix{./figures/FT_II-Slides_Annotations.2024.3.1/graphics/}

\mymakesubfile{1}
[FT II]
{Difusão com reação química Homogénea} % Subfile Title
{Difusão com reação química Homogénea} % Part Title

\begin{exampleBox}1{ % MARK: E1
    Usa-se uma coluna de absorção gasosa para absorver um composto A de uma corrente de ar à pressão atmosférica usando água. Para soluções diluídas os dados de equilíbrio podem ser obtidos por uma recta: \(P_A = 0.5\,C_A\), sendo \(P_A\) em \unit{\atm} e \(CA\) em \unit{\mol/\m^3}. Num determinado ponto da coluna a percentagem molar de A no ar é 7\% e a sua fracção molar no líquido é nula. Os coeficientes individuais de transferência de massa para cada uma das fases são: \(
        k_G = \qty*{0.3}{\mole/\m^2\,\hour\,\atm}
        \text{ e }
        k_L = \qty*{0.25}{\m/\hour}
    \).
} % E1
    
    \begin{exampleBox}2{ % MARK: E1.1
        Determine o coeficiente global de transferência de massa \(K_L\).
    } % E1.1
        \answer{}
        \begin{flalign*}
            &
                K_L
                = \left(
                    \frac
                    {C_{A}^*-C_A}
                    {N_A}
                \right)^{-1}
                = \left(
                    \frac
                    {C_{A}^*-C_{A,i}}
                    {N_A}
                    + \frac
                    {C_{A,i}-C_A}
                    {N_A}
                \right)^{-1}
                = &\\&
                = \left(
                    (k_G\,H)^{-1}
                    + k_L^{-1}
                \right)^{-1}
                = \left(
                    (0.3*0.5)^{-1}
                    + 0.25^{-1}
                \right)^{-1}
                \cong\mathemph{
                    \qty*
                    {9.375e-2}
                    {\m/\hour}
                }
            &
        \end{flalign*}
    \end{exampleBox}
    \begin{exampleBox}2{ % MARK: E1.2
        Determine o fluxo molar e as composições interfaciais no referido ponto da coluna.
    } % E1.2
        \answer{}
        \begin{flalign*}
            &
                \text{Fluxo molar:}&\\&
                N_{A}
                =K_L(C_{A}^*-C_A)
                =K_L((P_{A}/0.5)-C_A)
                =K_L((y_A\,P/0.5)-C_A)
                \cong &\\&
                \cong
                \num{9.375e-2}
                ((0.07*1/0.5)-0)
                \cong\mathemph{
                    \qty
                    {1.3125e-2}
                    {\mole/\m^2\,\hour}
                }
                % 
                % 
                % 
                ; &\\[3ex]&
                \text{Composições interfaciais: }C_{A,i}&\\&
                N_A=0.25(C_{A,i}-0)
                \implies
                C_{A,i}
                \cong
                \frac{\num{1.3125e-2}}{0.25}
                \cong\mathemph{
                    \qty
                    {5.25e-1}
                    {\mole/\metre^3}
                }
                % 
                % 
                % 
                ; &\\[3ex]&
                \text{Composições interfaciais: }P_{A,i}&\\&
                N_A=0.3(0.07-P_{A,i})
                \implies
                P_{A,i}
                \cong
                0.07
                -\frac{\num{1.3125e-2}}{0.3}
                \cong\mathemph{
                    \qty
                    {2.625e-2}
                    {\atm}
                }
            &
        \end{flalign*}
    \end{exampleBox}
    \begin{exampleBox}2{ % MARK: E1.3
        Calcule a percentagem de resistência exercida por cada uma das fases.
    } % E1.3
        \answer{}
        \begin{flalign*}
            &
                \text{\% Fase liquida}
                = \frac{k_L^{-1}}{K_L^{-1}}
                = \frac
                {K_L}
                {k_L}
                \cong \frac
                {\num{9.375e-2}}
                {0.25}
                \cong\mathemph{
                    \qty*
                    {37.5}
                    {\percent}
                }
                % 
                % 
                % 
                ; &\\[3ex]&
                \text{\% Fase gasosa}
                = \frac{(H\,k_g)^{-1}}{K_L^{-1}}
                = \frac
                {K_L}
                {H\,k_g}
                \cong \frac
                {\num{9.375e-2}}
                {0.5*0.3}
                \cong\mathemph{
                    \qty*
                    {62.5}
                    {\percent}
                }
            &
        \end{flalign*}
    \end{exampleBox}
    \begin{exampleBox}2{ % MARK: E1.4
        Determine o coeficiente de transferência de massa na fase líquida no caso de ocorrer uma reacção química irreversível de 1ª ordem com uma constante de velocidade igual a \qty*{30}{\s^{-1}}. Será válido considerar que a reacção é rápida? (\(\mathscr{D}_{A-\ch{H2O}} = \qty*{2.1e-5}{\cm^2/\s}\)).
    } % E1.4
        \answer{}
        \begin{flalign*}
            &
                \frac{K}{K^\circ}
                =Ha
                =\sqrt{\frac{k_L}{\mathscr{D}}}
                \,\delta
                =\sqrt{\frac{k_L}{\mathscr{D}}}
                \,\frac{\mathscr{D}}{k^\circ}
                \implies &\\&
                \implies
                K_L
                =\sqrt{k_L\,\mathscr{D}}
                =\sqrt{30*2.1\E{-9}}
                \cong
                ????
            &
        \end{flalign*}
    \end{exampleBox}
\end{exampleBox}

\end{document}
% !TEX root = ./FT_II-Slides_Annotations.1.tex
\providecommand\mainfilename{"./FT_II-Slides_Annotations.tex"}
\providecommand \subfilename{}
\renewcommand   \subfilename{"./FT_II-Slides_Annotations.1.tex"}
\documentclass[\mainfilename]{subfiles}

% \tikzset{external/force remake=true} % - remake all

\begin{document}

% \graphicspath{{\subfix{./.build/figures/FT_II-Slides_Annotations.1}}}
% \tikzsetexternalprefix{./.build/figures/FT_II-Slides_Annotations.1/}

\mymakesubfile{1}
[FT II]
{Fundamentos: Transferencia de massa} % Subfile Title
{Fundamentos: Transferencia de massa} % Part Title

\begin{sectionBox}1{Operações de transferencia de massa} % S1
    


    % \begin{itemize}
    %     \begin{multicols}{2}
    %         \item Destilação
    %         \item Adsorção Gasosa
    %         \item Sacagem
    %         \item Extração Liquido--Líquido
    %     \end{multicols}
    % \end{itemize}

    \begin{multicols}{2}
            \begin{sectionBox}2{Destilação} % S
                \begin{itemize}
                    \item Líquido--Vapor
                    \item Todos os componentes tem duas fases
                    \item Composição diferente em cada fase
                \end{itemize}
            \end{sectionBox}
        
            \begin{sectionBox}2{Absorção Gasosa} % S
                \begin{itemize}
                    \item Gás--Liquido
                    \item Apenas uma componente se distribui
                \end{itemize}
            \end{sectionBox}
        
            \begin{sectionBox}2{Secagem} % S
                \begin{itemize}
                    \item Gás--Sólido
                    \item Difusão do líquido presente no sólido para o gás
                \end{itemize}
            \end{sectionBox}

            \begin{sectionBox}2{Extração Líquido--Líquido} % S
                \begin{itemize}
                    \item Solução heterogenea
                    \item Solvente + Solução \rightarrow\\Extrato + Resíduo
                \end{itemize}
            \end{sectionBox}
    \end{multicols}
    
\end{sectionBox}

% MARK: Comp
%   ,ad8888ba,
%  d8"'    `"8b
% d8'
% 88              ,adPPYba,   88,dPYba,,adPYba,   8b,dPPYba,    ,adPPYba,   ,adPPYba,
% 88             a8"     "8a  88P'   "88"    "8a  88P'    "8a  a8"     "8a  I8[    ""
% Y8,            8b       d8  88      88      88  88       d8  8b       d8   `"Y8ba,
%  Y8a.    .a8P  "8a,   ,a8"  88      88      88  88b,   ,a8"  "8a,   ,a8"  aa    ]8I
%   `"Y8888Y"'    `"YbbdP"'   88      88      88  88`YbbdP"'    `"YbbdP"'   `"YbbdP"'
%                                                 88
%                                                 88

\begin{sectionBox}1{Composições} % S
    
    \subsection*{Concentração mássica}
    \begin{BM}
        \rho_A=\frac{m_A}{V}
        \qquad
        \rho=\sum_i{\rho_i}
    \end{BM}

    \subsection*{Concentração molar}
    \begin{BM}
        c_A
        =\frac{\rho_A}{M_A}
        =\frac{N_A}{v}
        =\frac{p_A}{R\,T}
    \end{BM}

    \subsection*{Fração molar}
    \begin{BM}
        y_A
        =\frac{c_A}{c}
        =\frac{p_A/(R\,T)}{P/(R\,T)}
        =\frac{p_A}{P}
        % \qquad
        % y_A=\frac{}
    \end{BM}
    
\end{sectionBox}

\begin{exampleBox}1{ % E1
    A composição molar de uma mistura gasosa a \qty*{273}{\kelvin} e \qty*{1.5e5}{\pascal} é
} % E1
    \begin{center}
        \vspace{1ex}
        \begin{tabular}{*{4}{C}}
            \toprule
            
                \ch{O2}
                & \ch{CO}
                & \ch{CO2}
                & \ch{N2}
            
            \\\midrule
            
                7\%
                & 10\%
                & 15\%
                & 68\%
            
            \\\bottomrule
        \end{tabular}
        \vspace{-4ex}
    \end{center}

    \section*{Determine}

    \begin{exampleBox}2{ % E1.1
        A composição em percentagem mássica
    } % E1.1
        \answer{}
        \begin{flalign*}
            &
                % 
                % rho_X
                % 
                \rho_{X}
                = \frac{
                    m_X
                    \,\unit{\frac
                        {\gram\of{X}}
                        {\mole}
                    }
                }{
                    M
                    \,\unit{\frac
                        {\gram}
                        {\mole}
                    }
                }
                = \frac{100\,m_X}{M}\,\unit{\frac{\percent\gram_{X}}{\gram}}
                % 
                % m_X
                % 
                ; &\\[3ex]&
                m_X\,\unit{\frac{\gram_{X}}{\mole}}
                = \frac{M_X\,\unit{\gram\of{X}}}{\unit{\mole\of{X}}}
                \,\frac{c_X\,\unit{\mole\of{X}}}{\unit{\mole}}
                = M_X\,c_X
                \,\unit{\frac{\gram_{X}}{\mole}}
                % 
                % M
                % 
                ; &\\[3ex]&
                M\,\unit{\frac{\gram}{\mole}}
                = \sum{
                    m_X
                    \,\unit{\frac
                        {\gram\of{X}}
                        {\mole}
                    }
                }
                % 
                % rho_X
                % 
                \implies &\\[3ex]&
                \implies
                \rho_X
                = \frac
                {100\,M_X\,c_X}
                {\sum{m_X}}
                \,\unit{\frac
                    {\percent\gram_{X}}
                    {\gram}
                }
            &
        \end{flalign*}
        \begin{center}
            \vspace{1ex}
            \begin{tabular}{L *{4}{C}}
                \toprule
                
                    & \ch{O2}
                    & \ch{CO}
                    & \ch{CO2}
                    & \ch{N2}
                
                \\\midrule
                
                    m_x
                    &  2.24
                    &  2.80
                    &  6.60
                    & 19.04
                    \\
                    \rho_X
                    & \num{7.301173402868318}
                    & \num{09.126466753585398}
                    & \num{21.51238591916558}
                    & \num{62.05997392438071}
                    
                \\\bottomrule
                \multicolumn{5}{R}{M=30.68}
            \end{tabular}
            \vspace{1ex}
        \end{center}
    \end{exampleBox}

    \begin{exampleBox}2{ % E
        A massa específica da mistura gasosa
    } % E
        \answer{}
        Assumindo gás ideal.
        \begin{flalign*}
            &
                \rho
                =\frac{M\,n}{V}
                =\frac{M\,n}{
                    \frac{n\,R\,T}{P}
                }
                =\frac{M\,P}{R\,T}
                \cong\frac{
                    30.68\E-3
                    *1.5\E5
                }{
                    \num{8.314462618}
                    * 273.15
                }
                \cong
                \qty{2.026334899933467}{\kilo\gram.\metre^{-3}}
            &
        \end{flalign*}
    \end{exampleBox}
\end{exampleBox}

% MARK: Velo
% 8b           d8              88
% `8b         d8'              88
%  `8b       d8'               88
%   `8b     d8'     ,adPPYba,  88   ,adPPYba,
%    `8b   d8'     a8P_____88  88  a8"     "8a
%     `8b d8'      8PP"""""""  88  8b       d8
%      `888'       "8b,   ,aa  88  "8a,   ,a8"
%       `8'         `"Ybbd8"'  88   `"YbbdP"'

\begin{sectionBox}1{Velocidades} % S
    
    \subsection*{Velocidade mássica}
    \begin{BM}
        v
        = \frac{
            \sum_{i=1}^{n}{
                \rho_i\,v_i
            }
        }{
            \sum_{i=1}^{n}{\rho_i}
        }
        = \frac{
            \sum_{i=1}^{n}{
                \rho_i\,v_i
            }
        }{\rho}
    \end{BM}

    \subsection*{Velocidade média molar}
    \begin{BM}
        V=\frac{\sum_{i=1}^{n}{c_i\,v_i}}{c}
    \end{BM}

    \subsection*{Velocidade relativa}
    Velocidade do componente \textit{i} relativamente à velocidade média mássica/molar
    \begin{BM}
        \adif{v_i}=v_i-v
        \\
        \adif{V_i}=v_i-V
    \end{BM}
    
\end{sectionBox}

% MARK: Coeff
%   ,ad8888ba,                               ad88     ad88
%  d8"'    `"8b                             d8"      d8"
% d8'                                       88       88
% 88              ,adPPYba,    ,adPPYba,  MM88MMM  MM88MMM
% 88             a8"     "8a  a8P_____88    88       88
% Y8,            8b       d8  8PP"""""""    88       88
%  Y8a.    .a8P  "8a,   ,a8"  "8b,   ,aa    88       88
%   `"Y8888Y"'    `"YbbdP"'    `"Ybbd8"'    88       88
% MARK: Difus
% 88888888ba,    88     ad88
% 88      `"8b   ""    d8"
% 88        `8b        88
% 88         88  88  MM88MMM  88       88  ,adPPYba,
% 88         88  88    88     88       88  I8[    ""
% 88         8P  88    88     88       88   `"Y8ba,
% 88      .a8P   88    88     "8a,   ,a88  aa    ]8I
% 88888888Y"'    88    88      `"YbbdP'Y8  `"YbbdP"'

\begin{sectionBox}1{Coeficiente de Difusão} % S
    
    \begin{BM}
        D=f(P,T,\text{natureza do componente})
        \\
        J_A = - D_{A,B}\,\gdif{c_A}
    \end{BM}

    \subsection*{Valores típicos}
    \begin{description}[
        leftmargin=!,
        labelwidth=\widthof{Líquidos} % Longest item
    ]
        \sisetup{
            % scientific / engineering / input / fixed
            exponent-mode           = input,
            exponent-to-prefix      = false,          % 1000 g -> 1 kg
            % exponent-product        = *,             % x * 10^y
            % fixed-exponent          = 0,
            round-mode              = none,        % figures/places/unsertanty/none
            round-precision         = 1,
            % round-minimum           = 0.01, % <x => 0
            % output-exponent-marker  = {\,\mathrm{E}},
        }
        \item[Gases]    \numrange{1e-5}{1e-4}
        \item[Líquidos] \numrange{.5e-9}{2e-9}
        \item[Sólidos]  \numrange{1e-24}{1e-12}
    \end{description}
    
\end{sectionBox}

% MARK: Lei
% 88                       88
% 88                       ""
% 88
% 88            ,adPPYba,  88
% 88           a8P_____88  88
% 88           8PP"""""""  88
% 88           "8b,   ,aa  88
% 88888888888   `"Ybbd8"'  88
% MARK: Fick
% 88888888888  88              88
% 88           ""              88
% 88                           88
% 88aaaaa      88   ,adPPYba,  88   ,d8
% 88"""""      88  a8"     ""  88 ,a8"
% 88           88  8b          8888[
% 88           88  "8a,   ,aa  88`"Yba,
% 88           88   `"Ybbd8"'  88   `Y8a



\begin{sectionBox}1{Lei da difusão} % S
    
    \subsection*{1ª Lei de Fick}
    \begin{BM}
        J_A=-D_{A,B}\nabla{c_A}
    \end{BM}
    \begin{description}[
        leftmargin=!,
        labelwidth=\widthof{} % Longest item
    ]
       \item[\(D_{A,B}\)] Coeficiente de difusão
    \end{description}

    \subsection*{Sistemas}\vspace{-3ex}
    \begin{multicols}{2}
        \begin{sectionBox}*2{Unidirecional} % S
            \begin{BM}
                J_{A,z}=-D_{A,B}\odv{c_A}{z}
            \end{BM}
        \end{sectionBox}
        \begin{sectionBox}*2{Isobárico e isotérmico} % S
            \begin{BM}
                J_{A,z}=-c\,D_{A,B}\odv{y_A}{z}
            \end{BM}
        \end{sectionBox}
    \end{multicols}
    
\end{sectionBox}

\begin{sectionBox}1{Fluxo máximo (molar) de i} % S
    
    \begin{BM}
        N_{A}
        =c_A\,v_A
        =y_{A}(N_A+N_B)
        -c\,D_{A,B}\,\nabla{y_A}
        \\
        N_{A,z}
        =y_{A}(N_{A,z}+N_{B,z})
        -c\,D_{A,B}\,\odv{y_A}{z}
    \end{BM}
    \begin{flalign*}
        &
            % \frac{N_{A}}{v_A}
            % =c_A
            % =\frac{N_{A,z}}{v_{A,z}}
            % =\frac{N_{A,z}}{v_{A,z}}
            % 
            % J_{A,z}
            % 
            J_{A,z}
            =c_A(v_{A,z}-V_z)
            =-c\,D_{A,B}\,\odv{y_A}{z}
            \implies &\\[3ex]&
            \implies
            c_A\,v_{A,z}
            = &\\&
            =c_A\,v_z-c\,D_{A,B}\,\odv{y_A}{z}
            = &\\&
            =c_A\,\left(
                \frac{c_A\,v_{A,z}+c_B\,v_{B,z}}{c}
            \right)-c\,D_{A,B}\,\odv{y_A}{z}
            = &\\&
            =y_{A}\,\left(
                c_A\,v_{A,z}+c_B\,v_{B,z}
            \right)-c\,D_{A,B}\,\odv{y_A}{z}
            ;&\\&
            N_A=c_A\,v_A
            \implies &\\[3ex]&
            \implies
            c_A\,v_{A,z}
            =N_{A,z}
            =y_{A}\,\left(
                N_{A,z}+N_{B,z}
            \right)-c\,D_{A,B}\,\odv{y_A}{z}
            \implies &\\[3ex]&
            \implies
            N_{A}
            =y_{A}\,\left(
                N_{A}+N_{B}
            \right)-c\,D_{A,B}\,\nabla{y_A}
        &
    \end{flalign*}

    \begin{sectionBox}2{Formas equivalentes para fluxo de massa em sistemas binarios (\textit{A},\textit{B})} % S
        \begin{center}
            \setlength\tabcolsep{3mm}        % width
            % \renewcommand\arraystretch{1.25} % height
            \vspace{1ex}
            \begin{tabular}{*{2}{C} L l}
                \toprule
                
                    \multicolumn{1}{c}{Flux}
                    & \multicolumn{1}{c}{Gradient}
                    & \multicolumn{1}{c}{Fick rate eq}
                    & \multicolumn{1}{c}{Restrictions}
                
                \\\midrule

                    % ================ n_A =============== %
                    \multirow[t]{2}{*}{\(n_A\)} 
                    & \gdif{\omega_A}
                    & n_A=-\rho\,D_{A,B}\,\gdif{\omega_A}+\omega_A(n_A+n_B)
                    & \multirow{4}{*}{Constant \(\rho\)}
                    \\
                    & \gdif{\rho_A}
                    & n_A=-D_{A,B}\gdif{\rho_A}+\omega_A(n_A+n_B)
                    &
                    % ================ j_A =============== %
                    \\[1ex] \multirow[t]{2}{*}{\(j_A\)} 
                    & \gdif{\omega_A}
                    & j_A=-\rho\,D_{A,B}\,\gdif{\omega_A}
                    % & \multirow{2}{*}{Constant \(\rho\)}
                    \\
                    & \gdif{\rho_A}
                    & j_A=-D_{A,B}\gdif{\rho_A}
                    &

                    \\\midrule
                    % ================ N_A =============== %
                    \multirow[t]{2}{*}{\(N_A\)} 
                    & \gdif{y_A}
                    & N_A=-c\,D_{A,B}\,\gdif{y_A}+y_A(N_A+N_B)
                    & \multirow{4}{*}{Constant \textit{c}}
                    \\
                    & \gdif{c_A}
                    & N_A=-D_{A,B}\gdif{c_A}+y_A(N_A+N_B)
                    &
                    % ================ J_A =============== %
                    \\[1ex] \multirow[t]{2}{*}{\(J_A\)} 
                    & \gdif{y_A}
                    & J_A=-c\,D_{A,B}\,\gdif{y_A}
                    % & \multirow{2}{*}{Constant \textit{c}}
                    \\
                    & \gdif{c_A}
                    & J_A=-D_{A,B}\gdif{c_A}
                    &
                
                \\\bottomrule
            \end{tabular}
            \vspace{2ex}
        \end{center}
    \end{sectionBox}
    
\end{sectionBox}

\end{document}
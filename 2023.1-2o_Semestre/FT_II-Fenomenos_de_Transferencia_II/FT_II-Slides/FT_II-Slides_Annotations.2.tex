% !TEX root = ./FT_II-Slides_Annotations.2.tex
\providecommand\mainfilename{"./FT_II-Slides_Annotations.tex"}
\providecommand \subfilename{}
\renewcommand   \subfilename{"./FT_II-Slides_Annotations.2.tex"}
\documentclass[\mainfilename]{subfiles}

% \tikzset{external/force remake=true} % - remake all

\begin{document}

\graphicspath{{\subfix{./.build/figures/FT_II-Slides_Annotations.2/graphics}}}
% \tikzsetexternalprefix{./.build/figures/FT_II-Slides_Annotations.2/}

\mymakesubfile{2}
[FT II]
{Anotações: Coeficientes de difusão} % Subfile Title
{Coeficientes de difusão} % Part Title

% MARK: Coeff
%   ,ad8888ba,                               ad88     ad88
%  d8"'    `"8b                             d8"      d8"
% d8'                                       88       88
% 88              ,adPPYba,    ,adPPYba,  MM88MMM  MM88MMM
% 88             a8"     "8a  a8P_____88    88       88
% Y8,            8b       d8  8PP"""""""    88       88
%  Y8a.    .a8P  "8a,   ,a8"  "8b,   ,aa    88       88
%   `"Y8888Y"'    `"YbbdP"'    `"Ybbd8"'    88       88

% MARK: Difus
% 88888888ba,    88     ad88
% 88      `"8b   ""    d8"
% 88        `8b        88
% 88         88  88  MM88MMM  88       88  ,adPPYba,
% 88         88  88    88     88       88  I8[    ""
% 88         8P  88    88     88       88   `"Y8ba,
% 88      .a8P   88    88     "8a,   ,a88  aa    ]8I
% 88888888Y"'    88    88      `"YbbdP'Y8  `"YbbdP"'

\begin{sectionBox}1{Coeficiente de Difusão} % S
    
    \begin{BM}
        \mathscr{D}=f(P,T,\text{natureza do componente})
        \\
        J_A = - \mathscr{D}_{A,B}\,\gdif{c_A}
    \end{BM}

    \subsection*{Valores típicos}\vspace{1ex}
    \begin{description}[
        leftmargin=!,
        labelwidth=\widthof{Líquidos} % Longest item
    ]
        \sisetup{
            % scientific / engineering / input / fixed
            exponent-mode           = input,
            exponent-to-prefix      = false,          % 1000 g -> 1 kg
            % exponent-product        = *,             % x * 10^y
            % fixed-exponent          = 0,
            round-mode              = none,        % figures/places/unsertanty/none
            round-precision         = 1,
            % round-minimum           = 0.01, % <x => 0
            % output-exponent-marker  = {\,\mathrm{E}},
        }
        \item[Gases]    \numrange*{1e-5}{1e-4}
        \item[Líquidos] \numrange*{.5e-9}{2e-9}
        \item[Sólidos]  \num*{1e-24}\to\num*{1e-12}
    \end{description}

    \subsubsection*{Constante de proporcionalidade entre fluxo e força motriz}
    \begin{BM}
        \dim{\mathscr{D}_{A,B}}
        = \dim\frac{-J_{A,z}}{\odv{c_A}{z}}
        = \frac{M}{L^2\,T}
        \,\frac{1}{(M/L^3)/L}
        = \frac{L^2}{T}
    \end{BM}
    
\end{sectionBox}

\part*{\(\mathscr{D}\) em Gases}

\begin{sectionBox}2{Teoria cinética de Chapman--Enskog} % MARK: S
    
    Misturas gasosas a baixa pressão: coeficientes de transporte são função da energia potencial de interação entre um par de moléculas no gás

    \begin{BM}
        U(r)
        = 4\,\varepsilon
        \left(
            \left(
                \frac{\sigma}{r}
            \right)^{12}
            - \left(
                \frac{\sigma}{r}
            \right)^{6}
        \right)
    \end{BM}

    \begin{itemize}
        \item \(U(r)\) Potencial de energina intramolecular de Lennard--Jones
        \item \(\varepsilon\) Energia de interação, diretamente relacinada com a força de interação entre as particulas ou sua ligação
        \item \(\sigma\) Diâmetro de colisão
    \end{itemize}
    
    \begin{figure}\centering
        % \tikzset{external/remake next=true}
        % {\Large\bfseries{Solução 1}}\par\medskip
        \begin{tikzpicture}
        \begin{axis}
        [
            xmajorgrids=true,
            ymajorgrids=true,
            % xtick=data,
            % xtick distance={1},
            % minor tick num={3},
            % minor x tick num={0},
            xminorgrids=false,
            yminorgrids=true,
            xmin={0},
            % ymin={-1.5},
            % xmax={3},
            % ymax={2},
            % scaled ticks=true,
            % scaled y ticks={base 10:2},
            % legend pos={north west},
            % axis on top,
            ylabel={\(U(r)/\varepsilon\)},
            xlabel={\(r/\sigma\)},
            % legend columns=2,
        ]
            \addplot+[
                no marks,
                draw={Graph},
                mark options={
                    fill={Graph},
                    fill opacity=1,
                    draw={Graph},
                },
                smooth,
                % tension=0.3,
                % curved line,
            ] expression [
                domain=0.97:3,
                restrict y to domain=-10:1,
                samples=50,
            ] {
                4*(x^(-12)-x^(-6))
            };
            % \addlegendentry{8}

            \draw[<->] 
            (0,0) 
            -> node[above,pos={.5}] {\(\sigma\)}
            (1,0);
            \draw[<->] 
            (1.12,0) 
            -> node[right] {\(\varepsilon\)}
            (1.12,-1);
            
        \end{axis}
        
        \end{tikzpicture}
        % \caption{Variação do prato de entrada da coluna B1 para diferentes numeros de pratos}
    \end{figure}

\end{sectionBox}

\begin{sectionBox}2{\(\mathscr{D}\) em gases} % #MARK: Dgas
    
    \begin{BM}
        \mathscr{D}_{A,B}
        = 1.858*10^{-3}
        \,\frac{
            T^{3/2}
            \,\sqrt{
                M_A^{-1}
                +M_{B}^{-1}
            }
        }{
            P
            \,\sigma_{A,B}^2
            \,\Omega_{D}
        }
    \end{BM}
    \begin{description}[
        leftmargin=!,
        labelwidth=\widthof{\(\dim{M_X}=\unit{\gram_{X}.\mole_X^{-1}}\)} % Longest item
    ]
       \item[\(\dim{\mathscr{D}}=\unit{\centi\metre^2.\second^{-1}}\)] Coeficiente de difusão da espécie A na espécie B
       \item[\(\dim{M_X}=\unit{\gram_{X}.\mole_X^{-1}}\)] Massa molecular da substância gasosa \textit{X}
       \item[\(\dim{P}=\unit{\atm}\)] Pressão total
       \item[\(\dim\sigma_X=\unit{\angstrom}\)] Diâmetro de colizão de \textit{X}
       \item[\(\dim\sigma_{A,B}=\unit{\angstrom}\)] Distância limite
       \item[\(\dim{T}=\unit{\kelvin}\)] Temperatura
       \item[\(\dim\Omega=0\)] Integral de colisão
    \end{description}

    \section*{Valores seguintes se encontram tabelados}

    \begin{multicols}{3}
        \begin{sectionBox}*2m{Diametro de colizão} % S
            \begin{BM}
                \sigma_{A,B}
                =\frac{\sigma_A+\sigma_B}{2}
            \end{BM}
        \end{sectionBox}
    
        \begin{sectionBox}*2m{Energia de interação} % S
            \begin{BM}
                \varepsilon_{A,B}
                =\sqrt{\varepsilon_A\,\varepsilon_B}
            \end{BM}
        \end{sectionBox}
    
        \begin{sectionBox}*2m{Integral de colisão} % S
            \begin{BM}
                \Omega_{D}
                = f(k_B\,T\,\varepsilon_{A,B}^{-1})
            \end{BM}
        \end{sectionBox}
    \end{multicols}
    \paragraph*{Nota:} \(k_B=\qty*{1.380649e-23}{\joule.\kelvin^{-1}}\) Constante de boltzman
\end{sectionBox}

\begin{sectionBox}2{Tabela Lennard--Jones Parametros de potencial} % MARK: S
    
    \begin{center}
        \vspace{1ex}
        \begin{tabular}{l l CC}
            \toprule
            
                \multicolumn{2}{l}{Substance}
                & \sigma (\unit{\angstrom})
                & \varepsilon/k (^\circ\unit{\K})
            
            \\\midrule
            
               \ch{Ar} & Argon & 3.542 & 93.3
            \\ \ch{He} & Helium & 2.551 & 10.22
            \\ \ch{Kr} & Krypton & 3.655 & 178.9
            \\ \ch{Ne} & Neon & 2.820 & 32.8
            \\ \ch{Xe} & Xenon & 4.047 & 231.0
            \\ \ch{Air} & Air & 3.711 & 78.6
            \\ \ch{Br2} & Bromine & 4.296 & 507.9
            \\ \ch{CCl4} & Carbon tetrachloride & 5.947 & 322.7
            \\ \ch{CF4} & Carbon tetrafluoride & 4.662 & 134.0
            \\ \ch{CHCl3} & Chloroform & 5.389 & 340.2
            \\ \ch{CH2Cl2} & Methylene chloride & 4.898 & 356.3
            \\ \ch{CH3Br} & Methyl bromide & 4.118 & 449.2
            \\ \ch{CH3Cl} & Methyl chloride & 4.182 & 350.0
            \\ \ch{CH3OH} & Methanol & 3.626 & 481.8
            \\ \ch{CH4} & Methane & 3.758 & 148.6
            \\ \ch{CO} & Carbon monoxide & 3.690 & 91.7
            \\ \ch{CO2} & Carbon dioxide & 3.941 & 195.2
            \\ \ch{CS2} & Carbon disulfide & 4.483 & 467.0
            \\ \ch{C2H2} & Acetylene & 4.033 & 231.8
            \\ \ch{C2H4} & Ethylene & 4.163 & 224.7
            \\ \ch{C2H6} & Ethane & 4.443 & 215.7
            \\ \ch{C2H5Cl} & Ethyl chloride & 4.898 & 300.0
            \\ \ch{C2H5OH} & Ethanol & 4.530 & 362.6
            \\ \ch{CH3OCH3} & Methyl ether & 4.307 & 395.0
            \\ \ch{CH2CHCH3} & Propylene & 4.678 & 298.9
            \\ \ch{CH3CCH} & Methylacetylene & 4.761 & 251.8
            \\ \ch{C3H6} & Cyclopropane & 4.807 & 248.9
            \\ \ch{C3H8} & Propane & 5.118 & 237.1
            \\ \ch{n-C3H7OH} & n-Propyl alcohol & 4.549 & 576.7
            \\ \ch{CH3COCH3} & Acetone & 4.600 & 560.2
            \\ \ch{CH3COOCH3} & Methyl acetate & 4.936 & 469.8
            \\ \ch{n-C4H10} & n-Butane & 4.687 & 531.4
            \\ \ch{iso-C4H10} & Isobutanc & 5.278 & 330.1
            \\ \ch{C2H5OC2H5} & Ethyl ether & 5.678 & 313.8
            \\ \ch{CH3COOC2H5} & Ethyl acetate & 5.205 & 521.3
            
            \\\bottomrule
        \end{tabular}
        \\[1ex]\tablecaption{\textit{Lennard--jones potential parameters found from viscosities}}
        \vspace{2ex}
    \end{center}
    
\end{sectionBox}

\begin{sectionBox}2{The collision integral} % MARK: S
    
    \begin{center}
        \vspace{1ex}
        \begin{tabular}{CC *{2}{ |CC }}
            \toprule
            
                  k_B\,T/\varepsilon & \Omega
                & k_B\,T/\varepsilon & \Omega
                & k_B\,T/\varepsilon & \Omega
            
            \\\midrule
            
               0.30 & 2.6620 & 1.65 & 1.1530 & 4.0 & 0.8836
            \\ 0.35 & 2.4760 & 1.70 & 1.1400 & 4.1 & 0.8788
            \\ 0.40 & 2.3180 & 1.75 & 1.1280 & 4.2 & 0.8740
            \\ 0.45 & 2.1840 & 1.80 & 1.1160 & 4.3 & 0.8694
            \\ 0.50 & 2.0660 & 1.85 & 1.1050 & 4.4 & 0.8652
            \\ 0.55 & 1.9660 & 1.90 & 1.0940 & 4.5 & 0.8610
            \\ 0.60 & 1.8770 & 1.95 & 1.0840 & 4.6 & 0.8568
            \\ 0.65 & 1.7980 & 2.00 & 1.0750 & 4.7 & 0.8530
            \\ 0.70 & 1.7290 & 2.10 & 1.0570 & 4.8 & 0.8492
            \\ 0.75 & 1.6670 & 2.20 & 1.0410 & 4.9 & 0.8456
            \\ 0.80 & 1.6120 & 2.30 & 1.0260 & 5.0 & 0.8422
            \\ 0.85 & 1.5620 & 2.40 & 1.0120 & 6   & 0.8124
            \\ 0.90 & 1.5170 & 2.50 & 0.9996 & 7   & 0.7896
            \\ 0.95 & 1.4760 & 2.60 & 0.9878 & 8   & 0.7712
            \\ 1.00 & 1.4390 & 2.70 & 0.9770 & 9   & 0.7556
            \\ 1.05 & 1.4060 & 2.80 & 0.9672 & 10  & 0.7424
            \\ 1.10 & 1.3750 & 2.90 & 0.9576 & 20  & 0.6640
            \\ 1.15 & 1.3460 & 3.00 & 0.9490 & 30  & 0.6232
            \\ 1.20 & 1.3200 & 3.10 & 0.9406 & 40  & 0.5960
            \\ 1.25 & 1.2960 & 3.20 & 0.9328 & 50  & 0.5756
            \\ 1.30 & 1.2730 & 3.30 & 0.9256 & 60  & 0.5596
            \\ 1.35 & 1.2530 & 3.40 & 0.9186 & 70  & 0.5464
            \\ 1.40 & 1.2330 & 3.50 & 0.9120 & 80  & 0.5352
            \\ 1.45 & 1.2150 & 3.60 & 0.9058 & 90  & 0.5256
            \\ 1.50 & 1.1980 & 3.70 & 0.8998 & 100 & 0.5130
            \\ 1.55 & 1.1820 & 3.80 & 0.8942 & 200 & 0.4644
            \\ 1.60 & 1.1670 & 3.90 & 0.8888 & 300 & 0.4360
            
            \\\bottomrule
        \end{tabular}
        \\[1ex]\tablecaption{The collision integral, Data from Hirschfelder et al. (1954)}
        \vspace{2ex}
    \end{center}
    
\end{sectionBox}

\begin{exampleBox}1{ % MARK: E1
    Faça uma estimativa do coeficiente de difusão do dióxido de carbono em ar a \qty*{20}{\celsius} e à pressão atmosférica, utilizando a equação de Hirschfelder
} % E1
    \paragraph*{Dados}:
    \begin{itemize}
        \item \(M_{ar}=\qty{28.9647}{\gram/\mole}\)
    \end{itemize}
    \answer{}
    \begin{flalign*}
        &
            \mathscr{D}_{\ch{CO2},ar}
            = 1.858\E{-3}
            \,\frac{
                T^{3/2}
                \,\sqrt{
                    M_{\ch{CO2}}^{-1}
                    +M_{ar}^{-1}
                }
            }{
                P
                \,\sigma_{\ch{CO2},ar}^2
                \,\Omega_{D}
            }
            = &\\&
            = 1.858\E{-3}
            \,\frac{
                T^{3/2}
                \,\sqrt{
                    M_{\ch{CO2}}^{-1}
                    +M_{ar}^{-1}
                }
            }{
                P
                \,(0.5*(\sigma_{\ch{CO2}}+\sigma_{ar}))^2
                \,f(k\,T/\varepsilon_{\ch{CO2},ar})
            }
            = &\\&
            = 1.858\E{-3}
            \,\frac{
                (20+273.15)^{3/2}
                \,\sqrt{
                    (44)^{-1}
                    +(\num{28.9647})^{-1}
                }
                % 0.239274019365228
            }{
                1
                *(0.5*(3.941+3.711))^2
                *\num{1.0078677646226}
            }
            % 81.402343185651226
            \cong&\\&
            \cong\qty
            {151.245553638939978e-3}
            {\cm^2/\s}
            % 
            % 
            % 
            ; &\\[3ex]&
            f\left(
                k\,T/\varepsilon_{\ch{CO2},ar}
            \right)
            = f\left(
                k\,T
                \left(
                    \varepsilon_{\ch{CO2}}
                    \,\varepsilon_{ar}
                \right)^{-1/2}
            \right)
            = f\left(
                T
                \,\left(
                    \frac{\varepsilon_{\ch{CO2}}}{k}
                    \,\frac{\varepsilon_{ar}}{k}
                \right)^{-1/2}
            \right)
            \cong&\\&
            \cong f\left(
                (20+273.15)
                \,\left(
                    195.2
                    *78.6
                \right)^{-1/2}
            \right)
            \cong
            f(\num{2.366675521150001})
            \implies &\\&
            \implies
            \frac
            {f(\num{2.366675521150001})-f(2.30)}
            {\num{2.366675521150001}-2.30}
            = \frac
            {f(2.40)-f(2.30)}
            {2.40-2.30}
            \implies &\\&
            \implies
            f(\num{2.366675521150001})
            = \frac
            {f(2.40)-f(2.30)}
            {2.40-2.30}
            \left(
                \num{2.366675521150001}-2.30
            \right)
            +f(2.30)
            \cong&\\&
            \cong \frac
            {1.0120-0.9996}
            {2.40-2.30}
            \left(
                \num{2.366675521150001}-2.30
            \right)
            +0.9996
            \cong&\\&
            \cong\num
            {1.0078677646226}
        &
    \end{flalign*}
    
\end{exampleBox}

\begin{sectionBox}2{Equação de Fuller} % MARK: S
    
    \begin{BM}
        \mathscr{D}_{A,B}
        = \frac{
            1\E{-3}\,T^{1.75}
            \,\sqrt{M_{A}^{-1}+M_{B}^{-1}}
        }{
            P\,\left(
                \left(
                    \sum{v}
                \right)^{1/3}_A
                + \left(
                    \sum{v}
                \right)^{1/3}_B
            \right)^{2}
        }
    \end{BM}

    \begin{itemize}
        \item \(v\) volumes de difusão, tabelados
    \end{itemize}
    
\end{sectionBox}

\begin{sectionBox}1{Variação \(\mathscr{D}\) com a temperatura} % S
    
    \begin{BM}
        \mathscr{D}_{A,B,T_2,P_2}
        = \mathscr{D}_{A,B,T_1,P_1}
        \,\frac{P_1}{P_2}
        \,\left(
            \frac{T_2}{T_1}
        \right)^{3/2}
        \,\frac
        {\Omega_{D,T_1}}
        {\Omega_{D,T_2}}
        \\[2ex]
        D \propto T^{3/2}\,\Omega
        \land
        D \propto P^{-1}
    \end{BM}
    \begin{flalign*}
        &
            \frac
            {\mathscr{D}_{A,B,T_2,P_2}}
            {\mathscr{D}_{A,B,T_1,P_1}}
            = \frac{
                1.858*10^{-3}
                \,\frac{T_2^{3/2}}{
                    P_2
                    \,\sigma_{A,B}^2
                    \,\Omega_{D,T_2}
                }
                \sqrt{
                    M_A^{-1}
                    +M_{B}^{-1}
                }
            }{
                1.858*10^{-3}
                \,\frac{T_1^{3/2}}{
                    P_1
                    \,\sigma_{A,B}^2
                    \,\Omega_{D,T_1}
                }
                \sqrt{
                    M_A^{-1}
                    +M_{B}^{-1}
                }
            }
            = \frac{
                \left(
                    \cfrac{T_2^{3/2}}{
                        P_2
                        \,\Omega_{D,T_2}
                    }
                \right)
            }{
                \left(
                    \cfrac{T_1^{3/2}}{
                        P_1
                        \,\Omega_{D,T_1}
                    }
                \right)
            }
            = &\\&
            = 
            \,\frac{P_1}{P_2}
            \,\left(
                \frac{T_2}{T_1}
            \right)^{3/2}
            \,\frac
            {\Omega_{D,T_1}}
            {\Omega_{D,T_2}}
        &
    \end{flalign*}
    
\end{sectionBox}

\begin{sectionBox}1{Difusão em mistura de gases} % S
    
    \begin{BM}
        \mathscr{D}_{1,m}
        =\left(
            \sum_{i=2}^{n}{
                \frac{y'_{1,i}}{\mathscr{D}_{1,i}}
            }
        \right)^{-1}
        =\left(
            \sum_{i=2}^{n}{
               \frac{
                    \frac{y_i}{\sum_{j=2}^{n}{y_j}}
                }{
                    \mathscr{D}_{1,i}
                }
            }
        \right)^{-1}
        = \frac{\sum_{j=2}^n{y_j}}{\sum_{i=2}^n{y_i/\mathscr{D}_{1,i}}}
        % \\
        % y'_{i}=\frac{y_i}{\sum_{j=2}^{n}{y_j}}
    \end{BM}
    
\end{sectionBox}

\begin{exampleBox}1{ % E
    Determine o coeficiente de difusão do \ch{CO} numa mistura gasosa cuja composição é:
} % E
    
    \begin{center}
        \vspace{1ex}
        \begin{tabular}{*{3}{C}}
            \toprule
            
                y_{\ch{O2}}
                & y_{\ch{N2}}
                & y_{\ch{CO}}
            
            \\\midrule
            
                0.20 & 0.70 & 0.10
            
            \\\bottomrule
        \end{tabular}
        \vspace{2ex}
    \end{center}

    \begin{itemize}
        \item A mistura está à temperatura de \qty*{298}{\kelvin} e à pressão de \qty*{2}{\atm}
        \item Os coeficientes de difusão do \ch{CO} em oxigênio e azoto são:
        \begin{itemize}
            \item \(
                \mathscr{D}_{\ch{CO,O2}} 
                =\qty*{0.185e-4}{\metre^2.\second^{-1}}
                \quad 
                \qty*{273}{\kelvin},
                \qty*{1}{\atm}
            \)
            \item \(
                \mathscr{D}_{\ch{CO,N2}} 
                =\qty*{0.192e-4}{\metre^2.\second^{-1}}
                \quad 
                \qty*{288}{\kelvin},
                \qty*{1}{\atm}
                \)
        \end{itemize}
    \end{itemize}

    \answer{}
    \begin{flalign*}
        &
            % 
            % \mathscr{D}_{CO,M}
            % 
            \text{Coeff de Dif de \ch{CO} na mistura}:
            &\\&
            \mathscr{D}_{\ch{CO},M}
            = \frac{
                \sum_{j=2}^{n}{y_{j}}
            }{
                \sum_{i=2}^{n}{y_i/\mathscr{D}_{\ch{CO},i}}
            }
            % = &\\&
            = \frac{
                0.9
            }{
                \left(
                    \begin{aligned}
                        &
                        0.20/\mathscr{D}_{\ch{CO},\ch{O2},
                            \qty*{298}{\kelvin},
                            \qty*{2}{\atm}
                        }
                        &+\\+&
                        0.70/\mathscr{D}_{\ch{CO},\ch{N2},
                            \qty*{298}{\kelvin},
                            \qty*{2}{\atm}
                        }
                        &
                    \end{aligned}
                \right)
            }
            \cong &\\&
            \cong
            \frac{0.9}{
                \left(
                    \begin{aligned}
                        &
                        0.20/\num{1.05492639372282e-3}
                        &+\\+&
                        0.70/\num{1.01043154819129e-3}
                        &
                    \end{aligned}
                \right)
            }
            \cong
            \qty{1.01999185289502e-3}{\metre^2.\second^{-1}}
            % 
            % D O2
            % 
            ; &\\[3ex]&
            \text{Coeff de Difuão do \ch{CO}:}
            &\\&
            \mathscr{D}_{\ch{CO},\ch{O2},
                \qty*{298}{\kelvin},
                \qty*{2}{\atm}
            }
            % \,\frac
            % {\Omega_{\mathscr{D},T_1}}
            % {\Omega_{\mathscr{D},T_2}}
            = 
            \mathscr{D}_{\ch{CO},\ch{O2},
                \qty*{273}{\kelvin},
                \qty*{1}{\atm}
            }
            \,\frac{1}{2}
            \,\left(
                \frac{298}{273}
            \right)^{3/2}
            = &\\&
            = 
            0.185\,\E{-4}
            \,\frac{1}{2}
            \,\left(
                \frac{298}{273}
            \right)^{3/2}
            \cong\qty
            {1.05492639372282e-3}
            {\m^2/\second}
            % =================== D N2 =================== %
            % 
            % D N2
            % 
            ; &\\[3ex]&
            \text{Coeff de Difs do \ch{NO}}
            &\\&
            \mathscr{D}_{\ch{CO},\ch{N2},
                \qty*{298}{\kelvin},
                \qty*{2}{\atm}
            }
            =
            \mathscr{D}_{\ch{CO},\ch{N2},
                \qty*{288}{\kelvin},
                \qty*{1}{\atm}
            }
            \,\frac{1}{2}
            \,\left(
                \frac{298}{288}
            \right)^{3/2}
            = &\\&
            =
            0.192*10^{-4}
            \,\frac{1}{2}
            \,\left(
                \frac{298}{288}
            \right)^{3/2}
            \cong\qty
            {1.01043154819129e-3}
            {\m^2/\second}
        &
    \end{flalign*}
    
\end{exampleBox}

% MARK: Stokes
%  ad88888ba                         88
% d8"     "8b    ,d                  88
% Y8,            88                  88
% `Y8aaaaa,    MM88MMM   ,adPPYba,   88   ,d8    ,adPPYba,  ,adPPYba,
%   `"""""8b,    88     a8"     "8a  88 ,a8"    a8P_____88  I8[    ""
%         `8b    88     8b       d8  8888[      8PP"""""""   `"Y8ba,
% Y8a     a8P    88,    "8a,   ,a8"  88`"Yba,   "8b,   ,aa  aa    ]8I
%  "Y88888P"     "Y888   `"YbbdP"'   88   `Y8a   `"Ybbd8"'  `"YbbdP"'

% MARK: Einstein
% 88888888888  88                                               88
% 88           ""                            ,d                 ""
% 88                                         88
% 88aaaaa      88  8b,dPPYba,   ,adPPYba,  MM88MMM   ,adPPYba,  88  8b,dPPYba,
% 88"""""      88  88P'   `"8a  I8[    ""    88     a8P_____88  88  88P'   `"8a
% 88           88  88       88   `"Y8ba,     88     8PP"""""""  88  88       88
% 88           88  88       88  aa    ]8I    88,    "8b,   ,aa  88  88       88
% 88888888888  88  88       88  `"YbbdP"'    "Y888   `"Ybbd8"'  88  88       88


\begin{sectionBox}1{\(\mathscr{D}\) em líquidos} % S

    \section*{Stokes--Einstein}
    \begin{BM}
        \mathscr{D}_A
        =\frac{k_B\,T}{6\,\pi\,\mu\,R_A}
    \end{BM}
    \begin{flalign*}
        &
            \left.
                \begin{aligned}
                    \mathscr{D}_A & =u_A\,R\,T
                    \quad& \text{(Nernst--Einstein)}
                    \\
                    u_A & \sim (6\,\pi\,\mu\,R_A)^{-1}
                    \quad& \text{(Stokes)}
                \end{aligned}
            \right\}
            \implies
            \mathscr{D}_A
            =\frac{R\,T}{6\,\pi\,\mu\,R_A}
        &
    \end{flalign*}
    \begin{description}
        \item[\(u_A\)] Mobilidade da partícula
        \item[\(k_B\)] Constante de boltsman: (\qty*{1.380649e-23}{\joule.\kelvin^{-1}})
    \end{description}
    
    \begin{sectionBox}*2{Constante de boltsman} % S
        
        \begin{BM}
            k_B
            = \frac{R}{N_A}
            \cong 
            \frac
            {\qty*{8.314462618}{\joule.\mole^{-1}.\kelvin^{-1}}}
            {\qty*{6.02214076e23}{\mole^{-1}}}
            \cong \\
            \cong
            \qty*{1.380648999974554e-23}{\joule.\kelvin^{-1}}
            \\[2ex]
            k_B
            \underset{exactly}{=}
            \qty*{1.380649e-23}{\joule.\kelvin^{-1}}
        \end{BM}

        Transforma a lei dos gases perfeitos numa verão por molécula
        
    \end{sectionBox}

    \section*{Casos especificos}

    \begin{sectionBox}*2{Prolate ellipsoid} % S
        \begin{BM}
            \mathscr{D}
            = \frac{k_B\,T}{
                6\,\pi\,\mu
                \left(
                    \cfrac{\sqrt{a^2-b^2}}{
                        \ln{\left(
                            \frac{
                                a+\sqrt{a^2-b^2}
                            }{b}
                        \right)}
                    }
                \right)
            }
        \end{BM}
    \end{sectionBox}

    \begin{sectionBox}*2{Oblate ellipsoid} % S
        \begin{BM}
            \mathscr{D}
            = \frac{k_B\,T}{
                6\,\pi\,\mu
                \left(
                    \cfrac{\sqrt{a^2-b^2}}{
                        \tan^{-1}\sqrt{\frac{a^2-b^2}{b^2}}
                    }
                \right)
            }
        \end{BM}
    \end{sectionBox}

\end{sectionBox}

% MARK: Wilke
% I8,        8        ,8I  88  88  88
% `8b       d8b       d8'  ""  88  88
%  "8,     ,8"8,     ,8"       88  88
%   Y8     8P Y8     8P    88  88  88   ,d8    ,adPPYba,
%   `8b   d8' `8b   d8'    88  88  88 ,a8"    a8P_____88
%    `8a a8'   `8a a8'     88  88  8888[      8PP"""""""
%     `8a8'     `8a8'      88  88  88`"Yba,   "8b,   ,aa
%      `8'       `8'       88  88  88   `Y8a   `"Ybbd8"'

% MARK: Chang
%   ,ad8888ba,   88
%  d8"'    `"8b  88
% d8'            88
% 88             88,dPPYba,   ,adPPYYba,  8b,dPPYba,    ,adPPYb,d8
% 88             88P'    "8a  ""     `Y8  88P'   `"8a  a8"    `Y88
% Y8,            88       88  ,adPPPPP88  88       88  8b       88
%  Y8a.    .a8P  88       88  88,    ,88  88       88  "8a,   ,d88
%   `"Y8888Y"'   88       88  `"8bbdP"Y8  88       88   `"YbbdP"Y8
%                                                       aa,    ,88
%                                                        "Y8bbdP"

\begin{sectionBox}2{Correlação de Wilke-Chang} % S
    
    \begin{BM}
        \frac{\mathscr{D}_{A,B}\,\mu_{B}}{T}
        = \frac{
            7.4*10^{-8}
            \,\sqrt{\Phi_B\,M_B}
        }{
            V_A^{0.6}
        }
    \end{BM}

    \begin{itemize}
        \item Soluções diluidas
    \end{itemize}
    \begin{description}[
        leftmargin=!,
        labelwidth=\widthof{\(\dim\mu_B=\unit{\centi\poise}=\qty*{0.1}{\centi\pascal.\second}\)} % Longest item
    ]
       \item[\(\dim{\mathscr{D}}=\unit{\centi\metre^3/\second}\)] Coeficiente de difusão
       \item[\(\dim{M_B}=\unit{\gram\of{B}/\mole}\)] Peso molecular do solvente \textit{B}
       \item[\(\dim\mu_B=\unit{\centi\poise}=\qty*{0.1}{\centi\pascal.\second}\)] Viscosidade do solvente \textit{B}
       \item[\(\dim{V_{A}}=\unit{\centi\metre^3.\mole^{-1}}\)] Volume molar do soluto \textit{A} no seu ponto de vaporização normal
       \item[\(\dim\Phi_B=0\)] parametro de associação
       \vspace{-2ex}
       \begin{itemize}
            \begin{multicols}{2}
                \item 2.26 \to{} Água
                \item 1.90 \to{} Metanol
                \item 1.50 \to{} Etanol
                \item 1.00 \to{} Benzeno, éter\dots\\(não associados / polares)
            \end{multicols}
       \end{itemize}
    \end{description}
    
\end{sectionBox}

\begin{sectionBox}2{Diluição Infinita (Hayduk--Ludie)} % S
    
    \begin{BM}
        \mathscr{D}_{A,B}
        = 13.26*10^{-5}\,\mu_B^{-1.14}\,V_A^{-0.589}
    \end{BM}

    \begin{description}[
        leftmargin=!,
        labelwidth=\widthof{\(\dim\mu_B=\unit{\centi\poise}=\qty*{0.1}{\centi\pascal.\second}\)} % Longest item
    ]
        \item[\(\dim\mu_B=\unit{\centi\poise}=\qty*{0.1}{\centi\pascal.\second}\)] Viscosidade do solvente \textit{B}
        \item[\(\dim{V_{A}}=\unit{\centi\metre^3.\mole^{-1}}\)] Volume molar do soluto \textit{A} no seu ponto de vaporização normal 
    \end{description}
    
\end{sectionBox}

\begin{sectionBox}2{Equação de Sheibel} % S
    
    \begin{BM}
        \frac{\mathscr{D}_{A,B}\,\mu_B}{T}
        =\frac{K}{V_A^{1/3}}
        =\frac{
            8.2*10^{-8}
            \left(
                1 + (3\,V_B/V_A)^{2/3}
            \right)
        }{
            V_A^{1/3}
        }
    \end{BM}
    
    Scheibel elimina o parametro de associação \(\Phi_B\)

\end{sectionBox}

\begin{sectionBox}2{Tabela de volumes moleculares} % S
    
    \begin{center}
        \vspace{1ex}
        \rowcolors*{2}{background!95!foreground}{}
        \begin{tabular}{
            *{2}{l}
            S[table-format=2.1,round-precision=1]
        }
            \toprule
            
                \multicolumn{2}{l}{Compound}
                & \multicolumn{1}{c}{
                    \begin{tabular}{c}
                        Molecular\\volume,
                        \\\hline{}
                        \unit{\cm^3/\g\,\mole}
                    \end{tabular}
                }
            
                \\\midrule
                
                % \rowcolor{Emph}
                Hydrogen,           & \ch{H2}  & 14.3
                \\ Oxygen,          & \ch{O2}  & 25.6
                \\ Nitrogen,        & \ch{N2}  & 31.2
                \\ Air              &          & 29.9
                \\ Carbon monoxide, & \ch{CO}  & 30.7
                \\ Carbon dioxide,  & \ch{CO2} & 34.0
                \\ Carbonyl sulfide,& \ch{COS} & 51.5
                \\ Sulfur dioxide,  & \ch{SO2} & 44.8
                \\ Nitric oxide,    & \ch{NO}  & 23.6
                \\ Nitrous oxide,   & \ch{N2O} & 36.4
                \\ Ammonia,         & \ch{NH3} & 25.8
                \\ Water,           & \ch{H2O} & 18.9
                \\ Hydrogen sulfide,& \ch{H2S} & 32.9
                \\ Bromine,         & \ch{Br2} & 53.2
                \\ Chlorine,        & \ch{Cl2} & 48.4
                \\ Iodine,          & \ch{I2}  & 71.5
            
            \\\bottomrule
        \end{tabular}
        \vspace{2ex}
    \end{center}
    
\end{sectionBox}

\begin{exampleBox}1{ % MARK: E
    Determine o valor do coeficiente de difusão do oxigénio em água à temperatura de \qty*{25}{\celsius} utilizando as correlações de Wilke-Chang e Scheibel e compare com o valor experimental
} % E
    \paragraph*{Dados:} 
    \begin{itemize}
        % \item \(\mathscr{D}_{oxigenio,agua}=\qty*{2.1e-8}{\m^2/\s}\)
        \item \(\mu_{agua}=\qty*{1}{\centi\poise}\)
    \end{itemize}
    \answer{}
    \begin{flalign*}
        &
            \text{Wilke-Chang}&\\&
            \frac{\mathscr{D}_{oxigenio,agua}\,\mu_B}{T}
            = \frac
            {7.4\E{-8}(\Phi_B\,M_B)^{1/2}}
            {V_A^{0.6}}
            \implies &\\&
            \implies
            \mathscr{D}_{oxigenio,agua}
            = \frac{T}{\mu_B}
            \frac
            {7.4\E{-8}(\Phi_B\,M_B)^{1/2}}
            {V_A^{0.6}}
            = \frac{273.15+25}{1}
            \frac
            {7.4\E{-8}(2.26*18)^{1/2}}
            {25.6^{0.6}}
            \cong &\\&
            \cong\qty
            {2.011004425623542674e-5}
            {\cm^2/\s}
            % 
            % 
            % 
            ; &\\[3ex]&
            \text{Scheibel}&\\&
            \frac{\mathscr{D}_{A,B}\,\mu_B}{T}
            = \frac{
                8.2\E{-8}\left(
                    1+(3\,V_B/V_A)^{2/3}
                \right)
            }
            {V_A^{1/3}}
            \implies &\\&
            \implies
            \mathscr{D}_{A,B}
            = \frac{T}{\mu_B}
            \,\frac{
                8.2\E{-8}\left(
                    1+(3\,V_B/V_A)^{2/3}
                \right)
            }
            {V_A^{1/3}}
            = &\\&
            = \frac{273.15+25}{1}
            \,\frac{
                8.2\E{-8}\left(
                    1+(3*18.9/25.6)^{2/3}
                    % 2.699138298806952
                \right)
            }
            {25.6^{1/3}}
            \cong &\\&
            \cong\qty
            {2.239032934947616576e-5}
            {\cm^2/\s}
        &
    \end{flalign*}
    
\end{exampleBox}

% \begin{exampleBox}1{ % E
%     Determine o valor do coeficiente de difusão do oxigénio em água à temperatura de 25\,\unit{\celsius} utilizando as correlações de Wilke--Chang e Scheibel e compare com o valor experimental 
%     \(\mathscr{D}_{\text{oxigénio-água}} = 2.1*10^{-9}\,\unit{\metre^2.\second^{-1}}\).
% } % E
    
%     body
    
% \end{exampleBox}

\part*{\(\mathscr{D}\) em Sólidos}

\begin{sectionBox}1{\(\mathscr{D}\) em sólidos} % S
    
    Difusão através de
    \begin{itemize}
        \begin{multicols}{3}
            \item Meios porosos
            \item não porosos\\(densos)
            \item compósitos
        \end{multicols}
    \end{itemize}

    \begin{sectionBox}*2{Aplicação meios porosos e não porosos} % S
        \begin{itemize}
            \item Processos catalíticos (CatHet)
            \item Processos membranas (permeação de gases e vapores)
            \item Permeação através de embalagens
            \item Liberação controlada de farmacos, agroquímicos,\dots
        \end{itemize}
    \end{sectionBox}
    
\end{sectionBox}

\begin{sectionBox}2{Difusão em meios \emph{porosos}} % S
    
    \subsubsection*{Defin. IUPAC}
    \begin{BM}[align*]
        & d > \qty*{50}{\nano\metre} 
        &\text{Macroporos}
        \\
        2 < & d < \qty*{50}{\nano\metre} 
        &\text{Mesoporos}
        \\
        & d < \qty*{2}{\nano\metre} 
        &\text{Microporos}
    \end{BM}

    \begin{center}
        \includegraphics[width=.7\textwidth]{image-032.png}
    \end{center}

\end{sectionBox}

% MARK: Diff
% 88888888ba,    88     ad88     ad88
% 88      `"8b   ""    d8"      d8"
% 88        `8b        88       88
% 88         88  88  MM88MMM  MM88MMM
% 88         88  88    88       88
% 88         8P  88    88       88
% 88      .a8P   88    88       88
% 88888888Y"'    88    88       88

% MARK: Kn
% 88      a8P                                      88
% 88    ,88'                                       88
% 88  ,88"                                         88
% 88,d88'       8b,dPPYba,   88       88   ,adPPYb,88  ,adPPYba,   ,adPPYba,  8b,dPPYba,
% 8888"88,      88P'   `"8a  88       88  a8"    `Y88  I8[    ""  a8P_____88  88P'   `"8a
% 88P   Y8b     88       88  88       88  8b       88   `"Y8ba,   8PP"""""""  88       88
% 88     "88,   88       88  "8a,   ,a88  "8a,   ,d88  aa    ]8I  "8b,   ,aa  88       88
% 88       Y8b  88       88   `"YbbdP'Y8   `"8bbdP"Y8  `"YbbdP"'   `"Ybbd8"'  88       88

\begin{sectionBox}2{Difusão de Knudsen} % S
    
    Considere uma difusão de baixa densidade por poros capilares bem pequenos onde o diametro dos póros são menores que a distancia média de colisão entre moléculas, estas vao colidir mais com os poros do que consigo próprias.

    \begin{BM}
        K_n=\lambda/d_{poro}
    \end{BM}
    \begin{description}[
        leftmargin=!,
        labelwidth=\widthof{\(d_{\text{poro}}\)} % Longest item
    ]
        \item[\(\lambda\)] Distancia média percorrida livremente pelas partículas (sem colisão)
        \item[\(d_{poro}\)] diametro do poro
    \end{description}

    \begin{sectionBox}*2{Mede a influencia desse tipo de difusão no evento} % S
        \begin{description}[
            leftmargin=!,
            labelwidth=\widthof{\(0.1<K_n<1\)} % Longest item
        ]
            \item[\( 0.1 < K_n < 1\)] A difusão de Knudsen tem parte mesurável porem moderada na difusão geral
            \item[\( 1   < K_n    \)] A dif de K é importante
            \item[\(10   < K_n    \)] A dif de K domina
        \end{description}
    \end{sectionBox}

    \subsubsection*{\(\lambda\) Distancia média percorrida livremente pelas partículas (sem colisão)}
    \begin{BM}
        \lambda
        = \frac{k_B\,T}{
            \sqrt{2}
            \,\pi\,d_{LJ}^2\,P
        }
    \end{BM}
    \begin{description}[
        leftmargin=!,
        labelwidth=\widthof{\(d_{LJ}\)} % Longest item
    ]
        \item[\(\dim P=\unit{\pascal}\)] Pressão do lado da alimentação
        \item[\(\dim{d_{LJ}}=\unit{\metre}\)] Diametro de Lennard--Jones, colisão entre gases que se difundem
        \\(tabelado)
        \item[\(k_B=\qty*{1.380649e-23}{\joule/\kelvin}\)] Constante de Boltzmann
    \end{description}

    \section*{\(\mathscr{D}\) de Knudsen}

    \begin{BM}
        \mathscr{D}_{kn,eff,i}
        = \frac{\varepsilon\,\mathscr{D}_{kn,i}}{\tau}
        = \frac{\varepsilon\,d_{LJ}}{\tau\,3}
        \,\sqrt{\frac{8\,R\,T}{\pi\,M\,W_i}}
        \\[2ex]
        \mathscr{D}_{kn,i}\propto (M\,W_i)^{-1/2}
        ;\quad
        \mathscr{D}_{kn,i}\propto T^{1/2}
        \\
        \mathscr{D}_{kn,i}\cancel{\propto} P
    \end{BM}
    \begin{description}[
        leftmargin=!,
        labelwidth=\widthof{\(\dim{\mathscr{D}}_{kn,eff,i}=\unit{\metre^2.\second^{-1}}\)} % Longest item
    ]
        \item[\(\dim{\mathscr{D}_{kn,i}}=\unit{\metre^2.\second^{-1}}\)] Coeficiente de difusão de Kn do gás \textit{i}
        \item[\(\dim{\mathscr{D}_{kn,eff,i}}=\unit{\metre^2.\second^{-1}}\)] Coeff de diff de Kn efetivo do gás \textit{i}
        \item[\(\dim\varepsilon=0\)] Porosidade do meio poroso
        \item[\(\dim\tau=0\)] tortuosidade do meio poroso
        \item[\(\dim{d_{LJ}}=\unit{\metre}\)] diametro de Lennard--Jones (tabelado)
    \end{description}

    \subsection*{Condições para considerarmos difusão de Knudsen}
    \begin{center}
        \vspace{1ex}
        \begin{tabular}{L *{4}{C}}
            \toprule
            
                d_{poro}/\unit{\nano\metre}
                & <10^3 & <10^2 & <10 & <2
                \\
                p/\unit{\bar}
                & 0.1 & 1 & 10 & 50
            
            \\\bottomrule   
                \multicolumn{5}{R}{
                    K_n>1
                    \land
                    \lambda>d_{poro}
                }
        \end{tabular}
        \vspace{2ex}
    \end{center}

    \subsection*{Seletividade de sep da \(\mathscr{D}_{kn}\)} % S
    
    \begin{BM}
        \alpha
        =\sqrt{\frac
            {M\,W_j}
            {M\,W_i}
        }
    \end{BM}
    
\end{sectionBox}

% #MARK: L-J
\begin{sectionBox}*2{Diametros de Lennard-Jones} % S
    
    \begin{center}
        \vspace{1ex}
        \rowcolors*{2}{background!95!foreground}{}
        \begin{tabular}{
            l
            S[table-format=1.2,round-precision=2]
            S[table-format=1.3,round-precision=3]
        }
            \toprule
            
                \multicolumn{1}{c}{Gás}
                & \multicolumn{1}{c}{
                    \begin{tabular}{c}
                        Diametro\\cinético\\
                        \(d_k\)/\unit{\angstrom}
                    \end{tabular}
                }
                & \multicolumn{1}{c}{
                    \begin{tabular}{c}
                        Diametro de\\
                        Lennard--Jones\\
                        \(d_{LJ}\)/\unit{\angstrom}
                    \end{tabular}
                }
            
            \\\midrule
            
              \ch{He}     & 2.6   & 2.551
            \\\ch{H2}     & 2.89  & 2.827
            \\\ch{O2}     & 3.46  & 3.467
            \\\ch{N2}     & 3.64  & 3.798
            \\\ch{CO}     & 3.76  & 3.69
            \\\ch{CO2}    & 3.3   & 3.941
            \\\ch{CH4}    & 3.8   & 3.758
            \\\ch{C2H6}   & \multicolumn{1}{c}{--} & 4.443
            \\\ch{C2H4}   & 3.9   & 4.163
            \\\ch{C3H8}   & 4.3   & 5.118
            \\\ch{C3H6}   & 4.5   & 4.678
            \\\ch{n-C4H10}& 4.3   & 4.971
            \\\ch{i-C4H10}& 5     & 5.278
            \\\ch{H2O}    & 2.65  & 2.641
            \\\ch{H2S}    & 3.6   & 3.623
            
            \\\bottomrule
        \end{tabular}
        \vspace{2ex}
    \end{center}
    
\end{sectionBox}

\begin{exampleBox}1{ % MARK: E1
    Considerando o transporte de \ch{O2} e de \ch{CO2} através de uma rolha de cortiça natural numa garrafa de vinho a \qty*{23}{\celsius} e a \qty*{1}{\bar}:
} % E1
    \paragraph*{Dados:}
    \begin{itemize}
        \begin{multicols}{3}
            \item \(d_{\ch{O2}}=\qty*{3.467}{\angstrom}\)
            \item \(d_{\ch{CO2}}=\qty*{3.941}{\angstrom}\)
            \item \(k_B=\qty*{1.380649e-23}{\J/\K}\)
            \item \(\qty*{1}{\bar}=\qty*{e5}{\Pa}\)
            \item \(d_{poro}=\qty*{40}{\nm}\)
        \end{multicols}
    \end{itemize}
    \begin{align*}
        \lambda
        =\frac{k_B\,T}{\sqrt{2}\,\pi\,d_{soluto}^2\,P}
    \end{align*}

    \begin{exampleBox}2{ % MARK: E.1
        calcule o livre percurso médio para os gases \ch{O2} e de \ch{CO2}.
    } % E1
        \answer{}
        \begin{flalign*}
            &
                \lambda_{\ch{O2}}
                =\frac{k_B\,T}{\sqrt{2}\,\pi\,\d_{\ch{O2}}^2\,P}
                \cong\frac{
                    \num{1.380649e-23}
                    *(23+273.15)
                }{
                    \sqrt{2}\,\pi
                    *(3.467\E{-10})^2
                    *10^5
                }
                \cong\qty
                {76.56362118298201}
                {\nm}
                % 
                % 
                % 
                ; &\\[3ex]&
                \lambda_{\ch{CO2}}
                =\frac{k_B\,T}{\sqrt{2}\,\pi\,\d_{\ch{O2}}^2\,P}
                \cong\frac{
                    \num{1.380649e-23}
                    *(23+273.15)
                }{
                    \sqrt{2}\,\pi
                    *(3.941\E{-10})^2
                    *10^5
                }
                \cong\qty
                {59.253946277353015}
                {\nm}
            &
        \end{flalign*}
    \end{exampleBox}

    \begin{exampleBox}2{ % MARK: E.2
        Calcule o numero de Knudsen.
    } % E.2
        \answer{}
        \begin{flalign*}
            &
                Kn_{\ch{O2}}
                =\frac{\lambda_{\ch{O2}}}{d_{LJ,\ch{O2}}}
                \cong
                \frac
                {\num{76.56362118298201}}
                {40}
                \cong
                \num{1.91409052957455}
                % 
                % 
                % 
                ; &\\[3ex]&
                Kn_{\ch{CO2}}
                =\frac{\lambda_{\ch{CO2}}}{d_{LJ,\ch{CO2}}}
                \cong
                \frac
                {\num{59.253946277353015}}
                {40}
                \cong
                \num{1.481348656933826}
            &
        \end{flalign*}
    \end{exampleBox}
    \begin{exampleBox}2{ % MARK: E.3
        Será que este transporte segue um comportamento difusivo de Knudsen?
    } % E.3
        \answer{}
        \begin{flalign*}
            &
                \left\{
                    \begin{aligned}
                        Kn_{\ch{O2}}>Kn_{\ch{CO2}}>1
                        \\ \lambda_{\ch{O2}}>\lambda_{\ch{CO2}}>d_{poro}
                        \\ d_{poro}<10^2\land P=\qty*{1}{\bar}
                    \end{aligned}
                \right\}
            &
        \end{flalign*}
        \(\therefore\) Ambos seguem o comportamento difusivo de knudsen
    \end{exampleBox}
    
\end{exampleBox}

%   ,ad8888ba,
%  d8"'    `"8b                  ,d
% d8'        `8b                 88
% 88          88  88       88  MM88MMM  8b,dPPYba,  ,adPPYYba,  ,adPPYba,
% 88          88  88       88    88     88P'   "Y8  ""     `Y8  I8[    ""
% Y8,        ,8P  88       88    88     88          ,adPPPPP88   `"Y8ba,
%  Y8a.    .a8P   "8a,   ,a88    88,    88          88,    ,88  aa    ]8I
%   `"Y8888Y"'     `"YbbdP'Y8    "Y888  88          `"8bbdP"Y8  `"YbbdP"'

% 88888888ba,    88     ad88
% 88      `"8b   ""    d8"
% 88        `8b        88
% 88         88  88  MM88MMM  88       88  ,adPPYba,   ,adPPYba,  ,adPPYba,
% 88         88  88    88     88       88  I8[    ""  a8P_____88  I8[    ""
% 88         8P  88    88     88       88   `"Y8ba,   8PP"""""""   `"Y8ba,
% 88      .a8P   88    88     "8a,   ,a88  aa    ]8I  "8b,   ,aa  aa    ]8I
% 88888888Y"'    88    88      `"YbbdP'Y8  `"YbbdP"'   `"Ybbd8"'  `"YbbdP"'

% #MARK: part*
\part*{Outras difusões}

\begin{sectionBox}1{Difusão superficial} % S
    
    \begin{itemize}
        \item \(
            1\,\unit{\nano\metre}
            < d_{LR}
            < 4\,\unit{\nano\metre}\)
        \item Molec de gas adsv nas paredes do poro
        \item relacionada com a mobilidade das moléculas à superficie
        \item rela. c a natureza química do gás e do mat poroso 
        (\(
            \ch{CO2}
            >\ch{CH4}
            >\ch{N2}
            >\ch{H2}
            >\ch{He}
        \))
        \item Referente a misturas gasosas e vapores
        \item depende fortemente de \textit{T}
    \end{itemize}
\end{sectionBox}

\begin{sectionBox}1{Condensação capilar} % S
    \begin{itemize}
        \item \(
            0.6\,\unit{\nano\metre}
            < d_{LR}
            < 6\,\unit{\nano\metre}
        \)
        \item Moléculas de gás ou vapor condensam denro dos poros e movem-se como líquidos
        \item elevada seletividade para gases ou vapores que condensam
        \item relacionado com a nat quimica do soluto
    \end{itemize}
\end{sectionBox}

\begin{sectionBox}1{Peneiros moleculares} % S
    \begin{itemize}
        \item \(
            0.2\,\unit{\nano\metre}
            < d_{LR}
            < 1\,\unit{\nano\metre}
        \)
        \item Tamanho dos poros comparaveis com o tamanho do gás alvo
        \item elevada seletividade
        \item relaciondado com o tamanho do soluto
        \item referente a mistuas gasosas e vapores
    \end{itemize}
\end{sectionBox}

% #MARK: Diff poros
\begin{sectionBox}1{Dif por meios não poros e sem partição de soluto} % S

    \section*{1ª lei de fick}
    \begin{BM}
        J_i=-\mathscr{D}_i\odv{c_i}{z}
    \end{BM}

    \begin{itemize}
        \item estrutura do meior é considerada homogenea e tratada como ``blackbox''
    \end{itemize}

    \begin{sectionBox}2{Transporte de massa através do filme} % S
        
        \begin{BM}
            J_i=\frac{\mathscr{D}_i}{\delta}\,\adif{c_i}
            ;\qquad
            J_i=\frac{\mathscr{D}_i}{\delta}\,\adif{p_i}
        \end{BM}
        
    \end{sectionBox}
\end{sectionBox}

\begin{sectionBox}1{Solubilização} % #MARK: Solubilização
    
    \subsection*{Difusão depende de}
    \begin{itemize}
        \item Tamanho do soluto que permeia
        \item natureza do material e meio sólido
        \item pode ser necessário considerar resistencias externas ao transporte do soluto (transf de massa externa)
    \end{itemize}
    
    \subsection*{Difusão em meios compósitos para esferas}
    \begin{BM}
        \frac{\mathscr{D}_{eff}-\mathscr{D}}{\mathscr{D}_{eff}+2\,\mathscr{D}}
        = \phi_s
        \,\frac{\mathscr{D}_s-\mathscr{D}}{\mathscr{D}_s+2\,\mathscr{D}}
        \iff
        \frac{\mathscr{D}_{eff}}{\mathscr{D}}
        = \frac{
            \frac{2}{\mathscr{D}_s}
            +\frac{1}{\mathscr{D}}
            -2\,\phi_s\left(
                \frac{1}{\mathscr{D}_s}
                -\frac{1}{\mathscr{D}}
            \right)
        }{
            \frac{2}{\mathscr{D}_s}
            +\frac{1}{\mathscr{D}}
            +\,\phi_s\left(
                \frac{1}{\mathscr{D}_s}
                -\frac{1}{\mathscr{D}}
            \right)
        }
        =\\
        = \frac{
            2\,\mathscr{D}
            +\mathscr{D}_s
            -2\,\phi_s\left(
                \mathscr{D}
                -\mathscr{D}_s
            \right)
        }{
            2\,\mathscr{D}
            +\mathscr{D}_s
            +\,\phi_s\left(
                \mathscr{D}
                -\mathscr{D}_s
            \right)
        }
    \end{BM}
    \begin{description}[
        leftmargin=!,
        labelwidth=\widthof{} % Longest item
    ]
        \item[\(\phi_S\)] fração de volume das esferas do material compósito
        \item[\(\mathscr{D}\)] Coef de dif na fase continua
        \item[\(\mathscr{D}_S\)] Coef de dif da através das esferas (fase dispersa)
    \end{description}

    \begin{itemize}
        \item Depende apenas da fração de volume das esferas (não do tamanho)
        \item A forma da equação depende da geometria
    \end{itemize}

    \subsection*{Casos permeabilidade das esferas}
    \begin{BM}
        \mathscr{D}_S=0
        \quad\text{(Esferas impermeáveis)}
        \implies
        \frac{\mathscr{D}_{eff}}{\mathscr{D}}
        = \frac{2(1-\phi_S)}{2+\phi_S}
        \\
        \mathscr{D}_s\to\infty
        \quad\text{(Esferas muito permeáveis)}
        \implies
        \frac{\mathscr{D}_{eff}}{\mathscr{D}}
        =\frac{1+2\,\phi_s}{1-\phi_s}
    \end{BM}

\end{sectionBox}

\end{document}
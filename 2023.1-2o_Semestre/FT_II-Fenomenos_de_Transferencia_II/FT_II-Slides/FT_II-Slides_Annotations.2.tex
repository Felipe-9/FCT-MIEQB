% !TEX root = ./FT_II-Slides_Annotations.2.tex
\providecommand\mainfilename{"./FT_II-Slides_Annotations.tex"}
\providecommand \subfilename{}
\renewcommand   \subfilename{"./FT_II-Slides_Annotations.2.tex"}
\documentclass[\mainfilename]{subfiles}

% \tikzset{external/force remake=true} % - remake all

\begin{document}

\graphicspath{{\subfix{./.build/figures/FT_II-Slides_Annotations.2/graphics}}}
% \tikzsetexternalprefix{./.build/figures/FT_II-Slides_Annotations.2/}

\mymakesubfile{2}
[FT II]
{Anotações: Coeficientes de difusão} % Subfile Title
{Coeficientes de difusão} % Part Title

% MARK: Coeff
%   ,ad8888ba,                               ad88     ad88
%  d8"'    `"8b                             d8"      d8"
% d8'                                       88       88
% 88              ,adPPYba,    ,adPPYba,  MM88MMM  MM88MMM
% 88             a8"     "8a  a8P_____88    88       88
% Y8,            8b       d8  8PP"""""""    88       88
%  Y8a.    .a8P  "8a,   ,a8"  "8b,   ,aa    88       88
%   `"Y8888Y"'    `"YbbdP"'    `"Ybbd8"'    88       88

% MARK: Difus
% 88888888ba,    88     ad88
% 88      `"8b   ""    d8"
% 88        `8b        88
% 88         88  88  MM88MMM  88       88  ,adPPYba,
% 88         88  88    88     88       88  I8[    ""
% 88         8P  88    88     88       88   `"Y8ba,
% 88      .a8P   88    88     "8a,   ,a88  aa    ]8I
% 88888888Y"'    88    88      `"YbbdP'Y8  `"YbbdP"'

\begin{sectionBox}1{Coeficiente de Difusão} % S
    
    \begin{BM}
        D=f(P,T,\text{natureza do componente})
        \\
        J_A = - D_{A,B}\,\gdif{c_A}
    \end{BM}

    \subsection*{Valores típicos}\vspace{1ex}
    \begin{description}[
        leftmargin=!,
        labelwidth=\widthof{Líquidos} % Longest item
    ]
        \sisetup{
            % scientific / engineering / input / fixed
            exponent-mode           = input,
            exponent-to-prefix      = false,          % 1000 g -> 1 kg
            % exponent-product        = *,             % x * 10^y
            % fixed-exponent          = 0,
            round-mode              = none,        % figures/places/unsertanty/none
            round-precision         = 1,
            % round-minimum           = 0.01, % <x => 0
            % output-exponent-marker  = {\,\mathrm{E}},
        }
        \item[Gases]    \numrange*{1e-5}{1e-4}
        \item[Líquidos] \numrange*{.5e-9}{2e-9}
        \item[Sólidos]  \numrange*{1e-24}{1e-12}
    \end{description}

    \subsubsection*{Constante de proporcionalidade entre fluxo e força motriz}
    \begin{BM}
        \dim{D_{A,B}}
        = \dim\frac{-J_{A,z}}{\odv{c_A}{z}}
        = \frac{M}{L^2\,T}
        \,\frac{1}{(M/L^3)/L}
        = \frac{L^2}{T}
    \end{BM}
    
\end{sectionBox}

\begin{sectionBox}2{\(D\) em gases} % S
    
    \begin{BM}
        D_{A,B}
        = 1.858*10^{-3}
        \,\frac{T^{3/2}}{
            P
            \,\sigma_{A,B}^2
            \,\Omega_{D}
        }
        \sqrt{
            M_A^{-1}
            +M_{B}^{-1}
        }
    \end{BM}
    \begin{description}[
        leftmargin=!,
        labelwidth=\widthof{\(\dim{M_X}=\unit{\gram_{X}.\mole_X^{-1}}\)} % Longest item
    ]
       \item[\(\dim{D}=\unit{\centi\metre^2.\second^{-1}}\)] Coeficiente de difusão da espécie A na espécie B
       \item[\(\dim{M_X}=\unit{\gram_{X}.\mole_X^{-1}}\)] Massa molecular da substância gasosa \textit{X}
       \item[\(\dim{P}=\unit{\atm}\)] Pressão total
       \item[\(\dim\sigma_X=\unit{\angstrom}\)] Diâmetro de colizão de \textit{X}
       \item[\(\dim\sigma_{A,B}=\unit{\angstrom}\)] Distância limite
       \item[\(\dim{T}=\unit{\kelvin}\)] Temperatura
       \item[\(\dim\Omega=0\)] Integral de colisão
    \end{description}

    \section*{Valores seguintes se encontram tabelados}

    \begin{sectionBox}*2{Diametro de colizão} % S
        \begin{BM}
            \sigma_{A,B}
            =\frac{\sigma_A+\sigma_B}{2}
        \end{BM}
    \end{sectionBox}

    \begin{sectionBox}*2{Energia de interação} % S
        \begin{BM}
            \varepsilon_{A,B}
            =\sqrt{\varepsilon_A\,\varepsilon_B}
        \end{BM}
    \end{sectionBox}

    \begin{sectionBox}*2{Integral de colisão} % S
        \begin{BM}
            \Omega_{D}
            = f(k\,T\,\varepsilon_{A,B}^{-1})
        \end{BM}
    \end{sectionBox}
    
\end{sectionBox}

\begin{sectionBox}1{Variação \(D\) com a temperatura} % S
    
    \begin{BM}
        D_{A,B,T_2,P_2}
        = D_{A,B,T_1,P_1}
        \,\frac{P_1}{P_2}
        \,\left(
            \frac{T_2}{T_1}
        \right)^{3/2}
        \,\frac
        {\Omega_{D,T_1}}
        {\Omega_{D,T_2}}
        \\[2ex]
        D \propto T^{3/2}\,\Omega
        \land
        D \propto P^{-1}
    \end{BM}
    \begin{flalign*}
        &
            \frac
            {D_{A,B,T_2,P_2}}
            {D_{A,B,T_1,P_1}}
            = \frac{
                1.858*10^{-3}
                \,\frac{T_2^{3/2}}{
                    P_2
                    \,\sigma_{A,B}^2
                    \,\Omega_{D,T_2}
                }
                \sqrt{
                    M_A^{-1}
                    +M_{B}^{-1}
                }
            }{
                1.858*10^{-3}
                \,\frac{T_1^{3/2}}{
                    P_1
                    \,\sigma_{A,B}^2
                    \,\Omega_{D,T_1}
                }
                \sqrt{
                    M_A^{-1}
                    +M_{B}^{-1}
                }
            }
            = \frac{
                \left(
                    \cfrac{T_2^{3/2}}{
                        P_2
                        \,\Omega_{D,T_2}
                    }
                \right)
            }{
                \left(
                    \cfrac{T_1^{3/2}}{
                        P_1
                        \,\Omega_{D,T_1}
                    }
                \right)
            }
            = &\\&
            = 
            \,\frac{P_1}{P_2}
            \,\left(
                \frac{T_2}{T_1}
            \right)^{3/2}
            \,\frac
            {\Omega_{D,T_1}}
            {\Omega_{D,T_2}}
        &
    \end{flalign*}
    
\end{sectionBox}

\begin{sectionBox}1{Difusão em mistura de gases} % S
    
    \begin{BM}
        D_{1,m}
        =\left(
            \sum_{i=2}^{n}{
                \frac{y'_{1,i}}{D_{1,i}}
            }
        \right)^{-1}
        =\left(
            \sum_{i=2}^{n}{
               \frac{
                    \frac{y_i}{\sum_{j=2}^{n}{y_j}}
                }{
                    D_{1,i}
                }
            }
        \right)^{-1}
        = \frac{\sum_{j=2}^n{y_j}}{\sum_{i=2}^n{y_i/D_{1,i}}}
        % \\
        % y'_{i}=\frac{y_i}{\sum_{j=2}^{n}{y_j}}
    \end{BM}
    
\end{sectionBox}

\begin{exampleBox}1{ % E
    Determine o coeficiente de difusão do \ch{CO} numa mistura gasosa cuja composição é:
} % E
    
    \begin{center}
        \vspace{1ex}
        \begin{tabular}{*{3}{C}}
            \toprule
            
                y_{\ch{O2}}
                & y_{\ch{N2}}
                & y_{\ch{CO}}
            
            \\\midrule
            
                0.20 & 0.70 & 0.10
            
            \\\bottomrule
        \end{tabular}
        \vspace{2ex}
    \end{center}

    \begin{itemize}
        \item A mistura está à temperatura de \qty*{298}{\kelvin} e à pressão de \qty*{2}{\atm}
        \item Os coeficientes de difusão do \ch{CO} em oxigênio e azoto são:
        \begin{itemize}
            \item \(
                D_{\ch{CO,O2}} 
                =\qty*{0.185e-4}{\metre^2.\second^{-1}}
                \quad 
                \qty*{273}{\kelvin},
                \qty*{1}{\atm}
            \)
            \item \(
                D_{\ch{CO,N2}} 
                =\qty*{0.192e-4}{\metre^2.\second^{-1}}
                \quad 
                \qty*{288}{\kelvin},
                \qty*{1}{\atm}
                \)
        \end{itemize}
    \end{itemize}

    \answer{}
    \begin{flalign*}
        &
            % 
            % D_{CO,M}
            % 
            \text{Coeff de Dif de \ch{CO} na mistura}:
            &\\&
            D_{\ch{CO},M}
            = \frac{
                \sum_{j=2}^{n}{y_{j}}
            }{
                \sum_{i=2}^{n}{y_i/D_{\ch{CO},i}}
            }
            % = &\\&
            = \frac{
                0.9
            }{
                \left(
                    \begin{aligned}
                        &
                        0.20/D_{\ch{CO},\ch{O2},
                            \qty*{298}{\kelvin},
                            \qty*{2}{\atm}
                        }
                        &+\\+&
                        0.70/D_{\ch{CO},\ch{N2},
                            \qty*{298}{\kelvin},
                            \qty*{2}{\atm}
                        }
                        &
                    \end{aligned}
                \right)
            }
            % 
            % D O2
            % 
            ; &\\[3ex]&
            \text{Coeff de Difuão do \ch{CO}:}
            &\\&
            D_{\ch{CO},\ch{O2},
                \qty*{298}{\kelvin},
                \qty*{1}{\atm}
            }
            = 
            D_{\ch{CO},\ch{O2},
                \qty*{273}{\kelvin},
                \qty*{2}{\atm}
            }
            \,\frac{1}{2}
            \,\left(
                \frac{298}{273}
            \right)^{3/2}
            = &\\&
            = 
            0.185\,\E{-4}
            \,\frac{1}{2}
            \,\left(
                \frac{298}{273}
            \right)^{3/2}
            \cong
            \qty{1.05492639372282e-3}{\metre^2.\second^{-1}}
            % =================== D N2 =================== %
            % 
            % D N2
            % 
            ; &\\[3ex]&
            \text{Coeff de Difs do \ch{NO}}
            &\\&
            D_{\ch{CO},\ch{N2},
                \qty*{298}{\kelvin},
                \qty*{2}{\atm}
            }
            =
            D_{\ch{CO},\ch{N2},
                \qty*{288}{\kelvin},
                \qty*{1}{\atm}
            }
            \,\frac{1}{2}
            \,\left(
                \frac{298}{288}
            \right)^{3/2}
            = &\\&
            =
            0.192*10^{-4}
            \,\frac{1}{2}
            \,\left(
                \frac{298}{288}
            \right)^{3/2}
            \cong
            \qty{1.01043154819129e-3}{\metre^2.\second^{-1}}
            \implies &\\[3ex]&
            \implies
            D_{\ch{CO},M}
            \cong
            \frac{0.9}{
                \left(
                    \begin{aligned}
                        &
                        0.20/\num{1.05492639372282e-3}
                        &+\\+&
                        0.70/\num{1.01043154819129e-3}
                        &
                    \end{aligned}
                \right)
            }
            \cong
            \qty{1.01999185289502e-3}{\metre^2.\second^{-1}}
        &
    \end{flalign*}
    
\end{exampleBox}

% MARK: Stokes
%  ad88888ba                         88
% d8"     "8b    ,d                  88
% Y8,            88                  88
% `Y8aaaaa,    MM88MMM   ,adPPYba,   88   ,d8    ,adPPYba,  ,adPPYba,
%   `"""""8b,    88     a8"     "8a  88 ,a8"    a8P_____88  I8[    ""
%         `8b    88     8b       d8  8888[      8PP"""""""   `"Y8ba,
% Y8a     a8P    88,    "8a,   ,a8"  88`"Yba,   "8b,   ,aa  aa    ]8I
%  "Y88888P"     "Y888   `"YbbdP"'   88   `Y8a   `"Ybbd8"'  `"YbbdP"'

% MARK: Einstein
% 88888888888  88                                               88
% 88           ""                            ,d                 ""
% 88                                         88
% 88aaaaa      88  8b,dPPYba,   ,adPPYba,  MM88MMM   ,adPPYba,  88  8b,dPPYba,
% 88"""""      88  88P'   `"8a  I8[    ""    88     a8P_____88  88  88P'   `"8a
% 88           88  88       88   `"Y8ba,     88     8PP"""""""  88  88       88
% 88           88  88       88  aa    ]8I    88,    "8b,   ,aa  88  88       88
% 88888888888  88  88       88  `"YbbdP"'    "Y888   `"Ybbd8"'  88  88       88


\begin{sectionBox}1{\(D\) em líquidos} % S

    \section*{Stokes--Einstein}
    \begin{BM}
        D_A
        =\frac{k_B\,T}{6\,\pi\,\mu\,R_A}
    \end{BM}
    \begin{flalign*}
        &
            \left.
                \begin{aligned}
                    D_A & =u_A\,R\,T
                    \quad& \text{(Nernst--Einstein)}
                    \\
                    u_A & \sim (6\,\pi\,\mu\,R_A)^{-1}
                    \quad& \text{(Stokes)}
                \end{aligned}
            \right\}
            \implies
            D_A
            =\frac{R\,T}{6\,\pi\,\mu\,R_A}
        &
    \end{flalign*}
    \begin{description}
        \item[\(u_A\)] Mobilidade da partícula
        \item[\(k_B\)] Constante de boltsman: (\qty*{1.380649e-23}{\joule.\kelvin^{-1}})
    \end{description}
    
    \begin{sectionBox}*2{Constante de boltsman} % S
        
        \begin{BM}
            k_B
            = \frac{R}{N_A}
            \cong 
            \frac
            {\qty*{8.314462618}{\joule.\mole^{-1}.\kelvin^{-1}}}
            {\qty*{6.02214076e23}{\mole^{-1}}}
            \cong \\
            \cong
            \qty*{1.380648999974554e-23}{\joule.\kelvin^{-1}}
            \\[2ex]
            k_B
            \underset{exactly}{=}
            \qty*{1.380649e-23}{\joule.\kelvin^{-1}}
        \end{BM}

        Transforma a lei dos gases perfeitos numa verão por molécula
        
    \end{sectionBox}

    \section*{Casos especificos}

    \begin{sectionBox}*2{Prolate ellipsoid} % S
        \begin{BM}
            D
            = \frac{k_B\,T}{
                6\,\pi\,\mu
                \left(
                    \cfrac{\sqrt{a^2-b^2}}{
                        \ln{\left(
                            \frac{
                                a+\sqrt{a^2-b^2}
                            }{b}
                        \right)}
                    }
                \right)
            }
        \end{BM}
    \end{sectionBox}

    \begin{sectionBox}*2{Oblate ellipsoid} % S
        \begin{BM}
            D
            = \frac{k_B\,T}{
                6\,\pi\,\mu
                \left(
                    \cfrac{\sqrt{a^2-b^2}}{
                        \tan^{-1}\sqrt{\frac{a^2-b^2}{b^2}}
                    }
                \right)
            }
        \end{BM}
    \end{sectionBox}

\end{sectionBox}

% MARK: Wilke
% I8,        8        ,8I  88  88  88
% `8b       d8b       d8'  ""  88  88
%  "8,     ,8"8,     ,8"       88  88
%   Y8     8P Y8     8P    88  88  88   ,d8    ,adPPYba,
%   `8b   d8' `8b   d8'    88  88  88 ,a8"    a8P_____88
%    `8a a8'   `8a a8'     88  88  8888[      8PP"""""""
%     `8a8'     `8a8'      88  88  88`"Yba,   "8b,   ,aa
%      `8'       `8'       88  88  88   `Y8a   `"Ybbd8"'

% MARK: Chang
%   ,ad8888ba,   88
%  d8"'    `"8b  88
% d8'            88
% 88             88,dPPYba,   ,adPPYYba,  8b,dPPYba,    ,adPPYb,d8
% 88             88P'    "8a  ""     `Y8  88P'   `"8a  a8"    `Y88
% Y8,            88       88  ,adPPPPP88  88       88  8b       88
%  Y8a.    .a8P  88       88  88,    ,88  88       88  "8a,   ,d88
%   `"Y8888Y"'   88       88  `"8bbdP"Y8  88       88   `"YbbdP"Y8
%                                                       aa,    ,88
%                                                        "Y8bbdP"

\begin{sectionBox}2{Correlação de Wilke-Chang} % S
    
    \begin{BM}
        \frac{D_{A,B}\,\mu_{B}}{T}
        = \frac{
            7.4*10^{-8}
            \,\sqrt{\Phi_B\,M_B}
        }{
            V_A^{0.6}
        }
    \end{BM}

    \begin{itemize}
        \item Soluções diluidas
    \end{itemize}
    \begin{description}[
        leftmargin=!,
        labelwidth=\widthof{\(\dim\mu_B=\unit{\centi\poise}=\qty*{0.1}{\centi\pascal.\second}\)} % Longest item
    ]
       \item[\(\dim{D}=\unit{\centi\metre^3/\second}\)] Coeficiente de difusão
       \item[\(\dim{M_B}=\unit{\gram\of{B}/\mole}\)] Peso molecular do solvente \textit{B}
       \item[\(\dim\mu_B=\unit{\centi\poise}=\qty*{0.1}{\centi\pascal.\second}\)] Viscosidade do solvente \textit{B}
       \item[\(\dim{V_{A}}=\unit{\centi\metre^3.\mole^{-1}}\)] Volume molar do soluto \textit{A} no seu ponto de vaporização normal
       \item[\(\dim\Phi_B=0\)] parametro de associação
       \vspace{-2ex}
       \begin{itemize}
            \begin{multicols}{2}
                \item 2.26 \to{} Água
                \item 1.90 \to{} Metanol
                \item 1.50 \to{} Etanol
                \item 1.00 \to{} Benzeno, éter\dots\\(não associados / polares)
            \end{multicols}
       \end{itemize}
    \end{description}
    
\end{sectionBox}

\begin{sectionBox}2{Diluição Infinita (Hayduk--Ludie)} % S
    
    \begin{BM}
        D_{A,B}
        = 13.26*10^{-5}\,\mu_B^{-1.14}\,V_A^{-0.589}
    \end{BM}

    \begin{description}[
        leftmargin=!,
        labelwidth=\widthof{\(\dim\mu_B=\unit{\centi\poise}=\qty*{0.1}{\centi\pascal.\second}\)} % Longest item
    ]
        \item[\(\dim\mu_B=\unit{\centi\poise}=\qty*{0.1}{\centi\pascal.\second}\)] Viscosidade do solvente \textit{B}
        \item[\(\dim{V_{A}}=\unit{\centi\metre^3.\mole^{-1}}\)] Volume molar do soluto \textit{A} no seu ponto de vaporização normal 
    \end{description}
    
\end{sectionBox}

\begin{sectionBox}2{Equação de Sheibel} % S
    
    \begin{BM}
        \frac{D_{A,B}\,\mu_B}{T}
        =\frac{K}{V_A^{1/3}}
        =\frac{
            8.2*10^{-8}
            \left(
                1 + (3\,V_B/V_A)^{2/3}
            \right)
        }{
            V_A^{1/3}
        }
    \end{BM}
    
\end{sectionBox}

\begin{sectionBox}2{Tabela de volumes moleculares} % S
    
    \begin{center}
        \vspace{1ex}
        \rowcolors*{2}{background!95!foreground}{}
        \begin{tabular}{l l c}
            \toprule
            
                \multicolumn{2}{c}{Compound}
                & \begin{tabular}{c}
                    Molecular\\volume,\\\unit{\centi\metre^3.\gram^{-1}.\mole^{-1}}
                \end{tabular}
            
                \\\midrule
                
                % \rowcolor{Emph}
                Hydrogen,           & \ch{H2}  & 14.3
                \\ Oxygen,          & \ch{O2}  & 25.6
                \\ Nitrogen,        & \ch{N2}  & 31.2
                \\ Air              &          & 29.9
                \\ Carbon monoxide, & \ch{CO}  & 30.7
                \\ Carbon dioxide,  & \ch{CO2} & 34.0
                \\ Carbonyl sulfide,& \ch{COS} & 51.5
                \\ Sulfur dioxide,  & \ch{SO2} & 44.8
                \\ Nitric oxide,    & \ch{NO}  & 23.6
                \\ Nitrous oxide,   & \ch{N2O} & 36.4
                \\ Ammonia,         & \ch{NH3} & 25.8
                \\ Water,           & \ch{H2O} & 18.9
                \\ Hydrogen sulfide,& \ch{H2S} & 32.9
                \\ Bromine,         & \ch{Br2} & 53.2
                \\ Chlorine,        & \ch{Cl2} & 48.4
                \\ Iodine,          & \ch{I2}  & 71.5
            
            \\\bottomrule
        \end{tabular}
        \vspace{2ex}
    \end{center}
    
\end{sectionBox}

% \begin{exampleBox}1{ % E
%     Determine o valor do coeficiente de difusão do oxigénio em água à temperatura de 25\,\unit{\celsius} utilizando as correlações de Wilke--Chang e Scheibel e compare com o valor experimental 
%     \(D_{\text{oxigénio-água}} = 2.1*10^{-9}\,\unit{\metre^2.\second^{-1}}\).
% } % E
    
%     body
    
% \end{exampleBox}

\begin{sectionBox}1{\(D\) em sólidos} % S
    
    Difusão através de
    \begin{itemize}
        \begin{multicols}{3}
            \item Meios porosos
            \item não porosos\\(densos)
            \item compósitos
        \end{multicols}
    \end{itemize}

    \begin{sectionBox}*2{Aplicação meios porosos e não porosos} % S
        \begin{itemize}
            \item Processos catalíticos (CatHet)
            \item Processos membranas (permeação de gases e vapores)
            \item Permeação através de embalagens
            \item Liberação controlada de farmacos, agroquímicos,\dots
        \end{itemize}
    \end{sectionBox}
    
\end{sectionBox}

\begin{sectionBox}2{Difusão em meios \emph{porosos}} % S
    
    \subsubsection*{Defin. IUPAC}
    \begin{BM}[align*]
        & d > \qty*{50}{\nano\metre} 
        &\text{Macroporos}
        \\
        2 < & d < \qty*{50}{\nano\metre} 
        &\text{Mesoporos}
        \\
        & d < \qty*{2}{\nano\metre} 
        &\text{Microporos}
    \end{BM}

    \begin{center}
        \includegraphics[width=.7\textwidth]{image-032.png}
    \end{center}

\end{sectionBox}

% MARK: Diff
% 88888888ba,    88     ad88     ad88
% 88      `"8b   ""    d8"      d8"
% 88        `8b        88       88
% 88         88  88  MM88MMM  MM88MMM
% 88         88  88    88       88
% 88         8P  88    88       88
% 88      .a8P   88    88       88
% 88888888Y"'    88    88       88

% MARK: Knudse
% 88      a8P                                      88
% 88    ,88'                                       88
% 88  ,88"                                         88
% 88,d88'       8b,dPPYba,   88       88   ,adPPYb,88  ,adPPYba,   ,adPPYba,  8b,dPPYba,
% 8888"88,      88P'   `"8a  88       88  a8"    `Y88  I8[    ""  a8P_____88  88P'   `"8a
% 88P   Y8b     88       88  88       88  8b       88   `"Y8ba,   8PP"""""""  88       88
% 88     "88,   88       88  "8a,   ,a88  "8a,   ,d88  aa    ]8I  "8b,   ,aa  88       88
% 88       Y8b  88       88   `"YbbdP'Y8   `"8bbdP"Y8  `"YbbdP"'   `"Ybbd8"'  88       88

\begin{sectionBox}2{Difusão de Knudsen} % S
    
    Considere uma difusão de baixa densidade por poros capilares bem pequenos onde o diametro dos póros são menores que a distancia média de colisão entre moléculas, estas vao colidir mais com os poros do que consigo próprias.

    \begin{BM}
        K_n=\lambda/d_{LJ}
    \end{BM}
    \begin{description}[
        leftmargin=!,
        labelwidth=\widthof{\(d_{\text{poro}}\)} % Longest item
    ]
        \item[\(\lambda\)] Distancia média percorrida livremente pelas partículas (sem colisão)
        \item[\(d_{LJ}\)] diametro de Lennard--Jones do poro
    \end{description}

    \begin{sectionBox}*2{Mede a influencia desse tipo de difusão no evento} % S
        \begin{description}[
            leftmargin=!,
            labelwidth=\widthof{\(0.1<K_n<1\)} % Longest item
        ]
            \item[\( 0.1 < K_n < 1\)] A difusão de Knudsen tem parte mesurável porem moderada na difusão geral
            \item[\( 1   < K_n    \)] A dif de K é importante
            \item[\(10   < K_n    \)] A dif de K domina
        \end{description}
    \end{sectionBox}

    \subsubsection*{\(\lambda\) Distancia média percorrida livremente pelas partículas (sem colisão)}
    \begin{BM}
        \lambda
        = \frac{k_B\,T}{
            \sqrt{2}
            \,\pi\,d^2\,p
        }
    \end{BM}
    \begin{description}[
        leftmargin=!,
        labelwidth=\widthof{\(d_{LJ}\)} % Longest item
    ]
        \item[\textit{p}] Pressão do lado da alimentação
        \item[\(d_{LJ}\)] Diametro de Lennard--Jones, colisão entre gases que se difundem
        \\(tabelado)
        \item[\(k_B=\qty*{1.380649e-23}{\joule/\kelvin}\)] Constante de Boltzmann
    \end{description}

    \subsection{Condições para considerarmos difusão de Knudsen}
    \begin{center}
        \vspace{1ex}
        \begin{tabular}{L *{4}{C}}
            \toprule
            
                d_{LJ}/\unit{\nano\metre}
                & <10^3 & <10^2 & <10 & <2
                \\
                p/\unit{\bar}
                & 0.1 & 1 & 10 & 50
            
            \\\bottomrule   
                \multicolumn{5}{R}{
                    K_n>1
                    \land
                    \lambda>d_{LD}
                }
        \end{tabular}
        \vspace{2ex}
    \end{center}
    
\end{sectionBox}

\begin{sectionBox}*2{Diametros de Lennard-Jones} % S
    
    \begin{center}
        \vspace{1ex}
        \begin{tabular}{*{3}{C}}
            \toprule
            
                \multicolumn{1}{c}{Gás}
                & \multicolumn{1}{c}{
                    \begin{tabular}{c}
                        Diametro\\cinético\\
                        \(d_k\)/\unit{\angstrom}
                    \end{tabular}
                }
                & \multicolumn{1}{c}{
                    \begin{tabular}{c}
                        Diametro de\\
                        Lennard--Jones\\
                        \(d_{LJ}\)/\unit{\angstrom}
                    \end{tabular}
                }
            
            \\\midrule
            
              \ch{He}     & 2.6   & 2.551
            \\\ch{H2}     & 2.89  & 2.827
            \\\ch{O2}     & 3.46  & 3.467
            \\\ch{N2}     & 3.64  & 3.798
            \\\ch{CO}     & 3.76  & 3.69
            \\\ch{CO2}    & 3.3   & 3.941
            \\\ch{CH4}    & 3.8   & 3.758
            \\\ch{C2H6}   & -     & 4.443
            \\\ch{C2H4}   & 3.9   & 4.163
            \\\ch{C3H8}   & 4.3   & 5.118
            \\\ch{C3H6}   & 4.5   & 4.678
            \\\ch{n-C4H10}& 4.3   & 4.971
            \\\ch{i-C4H10}& 5     & 5.278
            \\\ch{H2O}    & 2.65  & 2.641
            \\\ch{H2S}    & 3.6   & 3.623
            
            \\\bottomrule
        \end{tabular}
        \vspace{2ex}
    \end{center}
    
\end{sectionBox}

\begin{sectionBox}2{\(D\) de Knudsen} % S
    
    \begin{BM}
        D_{Kn,eff,i}
        = \frac{\varepsilon\,D_{Kn,i}}{\tau}
        = \frac{\varepsilon\,d_{LJ}}{\tau\,3}
        \,\sqrt{8\,R\,T}{\pi\,M\,W_i}
        \\[2ex]
        D_{Kn,i}\propto (M\,W_i)^{-1/2}
        \quad
        D_{Kn,i}\propto T^{1/2}
    \end{BM}
    \begin{description}[
        leftmargin=!,
        labelwidth=\widthof{\(\dim{D}_{Kn,eff,i}=\unit{\metre^2.\second^{-1}}\)} % Longest item
    ]
        \item[\(\dim{D_{Kn,i}}=\unit{\metre^2.\second^{-1}}\)] Coeficiente de difusão de Kn do gás \textit{i}
        \item[\(\dim{D}_{Kn,eff,i}=\unit{\metre^2.\second^{-1}}\)] Coeff de diff de Kn efetivo do gás \textit{i}
        \item[\(\dim\varepsilon=0\)] Porosidade do meio poroso
        \item[\(\dim\tau=0\)] tortuosidade do meio poroso
    \end{description}
    
\end{sectionBox}

\begin{sectionBox}2{Seletividade de sep da \textit{D} de \(Kn\)} % S
    
    \begin{BM}
        \alpha
        =\sqrt{\frac
            {M\,W_j}
            {M\,W_i}
        }
    \end{BM}
    
\end{sectionBox}

\part*{Outras difusões}

\begin{sectionBox}1{Difusão superficial} % S
    
    \begin{itemize}
        \item \(
            1\,\unit{\nano\metre}
            < d_{LR}
            < 4\,\unit{\nano\metre}\)
        \item Molec de gas adsv nas paredes do poro
        \item relacionada com a mobilidade das moléculas à superficie
        \item rela. c a natureza química do gás e do mat poroso 
        (\(
            \ch{CO2}
            >\ch{CH4}
            >\ch{N2}
            >\ch{H2}
            >\ch{He}
        \))
        \item Referente a misturas gasosas e vapores
        \item depende fortemente de \textit{T}
    \end{itemize}
\end{sectionBox}

\begin{sectionBox}1{Condensação capilar} % S
    \begin{itemize}
        \item \(
            0.6\,\unit{\nano\metre}
            < d_{LR}
            < 6\,\unit{\nano\metre}
        \)
        \item Moléculas de gás ou vapor condensam denro dos poros e movem-se como líquidos
        \item elevada seletividade para gases ou vapores que condensam
        \item relacionado com a nat quimica do soluto
    \end{itemize}
\end{sectionBox}

\begin{sectionBox}1{Peneiros moleculares} % S
    \begin{itemize}
        \item \(
            0.2\,\unit{\nano\metre}
            < d_{LR}
            < 1\,\unit{\nano\metre}
        \)
        \item Tamanho dos poros comparaveis com o tamanho do gás alvo
        \item elevada seletividade
        \item relaciondado com o tamanho do soluto
        \item referente a mistuas gasosas e vapores
    \end{itemize}
\end{sectionBox}

\begin{sectionBox}1{Dif por meios não poros e sem partição de soluto} % S

    \section*{1ª lei de fick}
    \begin{BM}
        J_i=-D_i\odv{c_i}{z}
    \end{BM}

    \begin{itemize}
        \item estrutura do meior é considerada homogenea e tratada como ``blackbox''
    \end{itemize}

    \begin{sectionBox}2{Transporte de massa através do filme} % S
        
        \begin{BM}
            J_i=\frac{D_i}{\delta}\,\adif{c_i}
            \\
            J_i=\frac{D_i}{\delta}\,\adif{p_i}
        \end{BM}
        
    \end{sectionBox}
\end{sectionBox}

\begin{sectionBox}1{Solubilização} % S
    
    \subsection*{Difusão depende de}
    \begin{itemize}
        \item Tamanho do soluto que permeia
        \item natureza do material e meio sólido
        \item pode ser necessário considerar resistencias externas ao transporte do soluto (transf de massa externa)
    \end{itemize}
    
    \begin{BM}
        \frac{D_{eff}-D}{D_{eff}+2\,D}
        = \phi_s
        \,\frac{D_s-D}{D_s+2\,D}
    \end{BM}
    \begin{description}[
        leftmargin=!,
        labelwidth=\widthof{} % Longest item
    ]
        \item[\(\phi_S\)] fração de volume das esferas do material compósito
        \item[\(D\)] Coef de dif na fase continua
        \item[\(D_S\)] Coef de dif da através das esferas (fase dispersa)
    \end{description}

    \begin{itemize}
        \item Depende apenas da fração de volume das esferas (não do tamanho)
    \end{itemize}

    \subsection*{Se as esferas fossem impermeaveis}
    \begin{BM}
        \frac{D_{eff}}{D}
        =   \frac{2(1-\phi_S)}{2+\phi_S}
        \\
        \lim_{D_S\to\infty}{\frac{D_{eff}}{D}}=1.33
    \end{BM}

\end{sectionBox}

\end{document}
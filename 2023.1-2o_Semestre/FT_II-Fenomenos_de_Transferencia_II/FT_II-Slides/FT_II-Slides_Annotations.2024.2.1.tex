% !TEX root = ./FT_II-Slides_Annotations.2024.2.1.tex
\providecommand\mainfilename{"./FT_II-Slides_Annotations.tex"}
\providecommand \subfilename{}
\renewcommand   \subfilename{"./FT_II-Slides_Annotations.2024.2.1.tex"}
\documentclass[\mainfilename]{subfiles}

% \tikzset{external/force remake=true} % - remake all

\begin{document}

% \graphicspath{{\subfix{./figures/FT_II-Slides_Annotations.2024.2.1}}}
% \tikzsetexternalprefix{./figures/FT_II-Slides_Annotations.2024.2.1/graphics/}

\mymakesubfile{8}
[FT II]
{Convecção -- Análise Dimensional e Correlações} % Subfile Title
{Convecção -- Análise Dimensional e Correlações} % Part Title

\begin{sectionBox}1{Coeficiente de Transferencia de Massa} % MARK: S
    
    \begin{BM}
        N_A = \mathemph{k_C}\,(C_{A,S}-C_{A})
    \end{BM}
    Avaliação de \(k_C\)
    \begin{itemize}
        \item \hyperref{}{2.1}{1.AnaliseDimensional}{Análise Dimensional}
        \item Correlações Experimentais
        \item Analogias entre transferencia de massa, calor e quantidade de movimento
        \item Modelos
        \item Camada Limite
    \end{itemize}
    
\end{sectionBox}

\begin{sectionBox}1{Análise Dimensional} % MARK: S
    \hyperdef{2.1}{1.AnaliseDimensional}{}
    
    \begin{center}
        \vspace{1ex}
        \begin{tabular}{l CC}
            \toprule
            
                \multicolumn{1}{c}{Variável}
                & \multicolumn{1}{c}{Símbolo}
                & \multicolumn{1}{c}{Dimensão}
            
            \\\midrule
            
                Diametro
                & D & \unit{L}
                \\ Massa Esp. Flu.
                & \rho & \unit{M.L^{-3}}
                \\ Viscosidade Flu.
                & \mu & \unit{M.L^{-1}.T^{-1}}
                \\ Velocidade Flu.
                & v & \unit{L.T^{-1}}
                \\ Coef. Difusão
                & \mathscr{D}_{A,B} & \unit{L^2.T^{-1}}
                \\ Coef. Transf. Massa
                & k_C & \unit{L.T^{-1}}
            
            \\\bottomrule
        \end{tabular}
        \vspace{2ex}
    \end{center}

    \paragraph*{Teorema \chempi{} de Bulkiman:}
    \begin{BM}
        i = n-K
    \end{BM}
    \begin{description}[
        leftmargin=!,
        labelwidth=\widthof{\(K\)} % Longest item
    ]
        \item[\(i\)] Nº de Grupos Adimensionais
        \item[\(n\)] Nº de Variáveis
        \item[\(K\)] Nº de Grandezas fundamentais
    \end{description}

\end{sectionBox}


\begin{sectionBox}2{Numero de Sheerwood} % MARK: S
    
    \begin{BM}
        \pi_1
        = \mathscr{D}_{A,B}^{a_1}
        \,\rho^{a_2}
        \,D^{a_3}
        \,k_C
        \implies \\
        \implies
        \pi_1
        = \frac{k_C\,D}{\mathscr{D}_{A,B}}
    \end{BM}

    \begin{flalign*}
        &
            \dim{\pi_1}
            = 1
            = \dim{\left(
                \mathscr{D}_{A,B}^{a_1}
                \,\rho^{a_2}
                \,D^{a_3}
                \,k_C
            \right)}
            = &\\&
            = \left(
                \frac{L^2}{T}
            \right)^{a_1}
            \, \left(
                \frac{M}{L^3}
            \right)^{a_2}
            \, \left(
                L
            \right)^{a_3}
            \,\frac{L}{T}
            = L^{2\,a_1-3\,a_2+a_3+1}
            \,T^{-a_1-1}
            \,M^{a_2}
            \implies &\\&
            \implies
            \begin{cases}
                a_2=0
                \\
                a_1=-1
                \\
                a_3=-1+2=1
            \end{cases}
            % 
            % 
            % 
            \qquad
            \therefore
            \pi_1=\frac{k_C\,D}{\mathscr{D}_{A,B}}
        &
    \end{flalign*}
    
\end{sectionBox}

\begin{sectionBox}2{Numero de Reynalds} % MARK: S
    
    \begin{BM}
        Re
        =\frac{\pi_2}{Sc}
        =\frac{\pi_2}{\pi_3}
        =\frac{D\,v\,\rho}{\mu}
    \end{BM}

    \begin{flalign*}
        &
            \mathemph{\pi_2:}
            \quad
            \pi_2
            = \mathscr{D}_{A,B}^{a_1}
            \,\rho^{a_2}
            \,\D^{a_3}
            \,v
            = \dots
            = \frac{D\,v}{\mathscr{D}_{A,B}}
            % 
            % 
            % 
            ; &\\&
            \mathemph{\pi_3:}
            \quad
            \pi_3
            = \mathscr{D}_{A,B}^{a_1}
            \,\rho^{a_2}
            \,\D^{a_3}
            \,\mu
            = \dots
            = \frac{\mu}{\rho\,\mathscr{D}_{A,B}}
            % 
            % 
            % 
            \implies &\\&
            \implies
            Re
            = \frac{\pi_2}{\pi_3}
            = \frac{
                \frac{D\,v}{\mathscr{D}_{A,B}}
            }{
                \frac{\mu}{\rho\,\mathscr{D}_{A,B}}
            }
            =\frac{D\,v\,\rho}{\mu}
        &
    \end{flalign*}
    
\end{sectionBox}

\begin{sectionBox}2{Numero de Schmidt} % MARK: S
    
    \begin{BM}
        Sc
        = \pi_3
        = \frac{\mu}{\rho\,\mathscr{D}_{A,B}}
    \end{BM}

    Razão entre a difusão molecular de quantidade de movimento e de massa

    \begin{flalign*}
        &
            Sc
            = \pi_3
            = \mathscr{D}_{A,B}^{a_1}
            \,\rho^{a_2}
            \,D^{a_3}
            \,k_C
            = \dots
            = \frac{\mu}{\rho\,\mathscr{D}_{A,B}}
        &
    \end{flalign*}
    
\end{sectionBox}

\part*{Correlações}

\begin{sectionBox}1{Correlações} % MARK: S
    
    \subsection*{Experimentais}
    \begin{multicols}{2}
        Transferencia de Massa
        \begin{BM}
            Sh=\Psi(Re,Sc)
        \end{BM}

        Transferencia de Calor
        \begin{BM}
            Nu=\Psi(Re,Pr)
        \end{BM}
    \end{multicols}
    
\end{sectionBox}

\end{document}
% !TEX root = ./FT_II-Slides_Annotations.3.tex
\providecommand\mainfilename{"./FT_II-Slides_Annotations.tex"}
\providecommand \subfilename{}
\renewcommand   \subfilename{"./FT_II-Slides_Annotations.3.tex"}
\documentclass[\mainfilename]{subfiles}

% \tikzset{external/force remake=true} % - remake all

\begin{document}

% \graphicspath{{\subfix{./figures/FT_II-Slides_Annotations.3}}}
% \tikzsetexternalprefix{./figures/FT_II-Slides_Annotations.3/graphics/}

\mymakesubfile{3}
[FT II]
{Difusão de Eletrolitos} % Subfile Title
{Difusão de Eletrolitos} % Part Title

\begin{sectionBox}1{Velocidade do ion} % MARK: S
    
    \begin{BM}
        v_i
        = u_i\left(
            \gdif{\mu_i}
            + z_i\,F\,\gdif{\Psi}
        \right)
    \end{BM}
    \begin{align*}
        \mu_i&=\mu_i^0+R\,T\,\ln{a_i}
        ;&
        \gdif{\mu_i}&=\frac{R\,T}{c_i}\,\gdif{c_i}
        ;&
        \mathscr{D}_i&=u_i\,R\,T
        \quad(\text{Relação Einstein})
    \end{align*}
    \begin{align*}
        -J_i
        =c_i\,v_i
        =\mathscr{D}
    \end{align*}
    \begin{description}[
        leftmargin=!,
        labelwidth=\widthof{\(\gdif{mu_i}\)} % Longest item
    ]
        \item[\(\mu_i\):] Mobilidade do ion
        \item[\(\gdif{\mu_i}\):] Potencial químico
        \item[\(z_i\):] Carga ionica 
        \item[\(F\):] \qty{9.64853321233100184e4}{\C/\mole} (Constante de Faraday)
        \item[\(\Psi\):] Potencial eletrostático
        \item[\(u_i\):] Propriedade física do ion: \(u_i\sim(6\,\pi\,\eta\,R_0)^{-1}\) (Stokes-Einstein)
        \item[\(R_0\)] Raio efetivo (Efeitos solvatação) 
    \end{description}

\end{sectionBox}

\begin{sectionBox}1{Equação Nerst-Plank} % MARK: S
    
    \begin{BM}
        -J_i
        = \mathscr{D}\left(
            \gdif{c_i}
            +c_i\,z_i
            \,\frac
            {F\,\gdif{\Psi}}
            {R\,T}
        \right)
    \end{BM}
    \begin{itemize}
        \item Soluções Diluidas
    \end{itemize}
    \begin{sectionBox}3{Demonstration} % MARK: S
        \begin{flalign*}
            &
                -J_i
                = -c_i\,v_i
                = -c_i
                \,\left(
                    -u_i\,(
                        \gdif{\mu_i}
                        +z_i\,F\,\gdif{\Psi}
                    )
                \right)
                = c_i
                \,u_i\,\left(
                    \left(
                        \frac{R\,T}{c_i}
                        \,\gdif{c_i}
                    \right)
                    +z_i\,F\,\gdif{\Psi}
                \right)
                = &\\&
                = \mathscr{D}
                \,\left(
                    \gdif{c_i}
                    + z_i\,c_i
                    \,\frac
                    {F\,\gdif{\Psi}}
                    {R\,T}
                \right)
                % 
                % 
                % 
                ; &\\[3ex]&
                \mu_i=\mu_i^0+R\,T\,\ln{a_i}
                \land
                a_i
                \xrightarrow{\text{Sol diluidas}}
                c_i
                % \implies &\\&
                \implies
                \gdif{\mu_i}
                =\frac{R\,T}{c_i}\,\gdif{c_i}
                % Não ha explicação
                % 
                % 
                % 
                ; &\\[3ex]&
                \mathscr{D}_i=u_i\,R\,T
                \quad\text{(Relação Einsntein)}
            &
        \end{flalign*}
    \end{sectionBox}
    
\end{sectionBox}

\begin{sectionBox}1{Tabela dos coeficientes de difusão de ions em agua a 25\unit{\celsius}} % MARK: S
    
    \begin{center}
        \vspace{1ex}
        \setlength\tabcolsep{9mm} % width
        % \renewcommand\arraystretch{1.25} % height
        \rowcolors*{2}{background!95!foreground}{}
        \begin{tabular}{l C | l C}
            \toprule
            
                   Cation
                &  \mathscr{D}_i
                &  Anion
                &  \mathscr{D}_i
            
            \\\midrule
            
                    \ch{H^{+}}        & 9.31 & \ch{OH^-}         & 5.28
                \\  \ch{Li^{+}}       & 1.03 & \ch{F^-}          & 1.47
                \\  \ch{Na^{+}}       & 1.33 & \ch{Cl^-}         & 2.03
                \\  \ch{K^{+}}        & 1.96 & \ch{Br^-}         & 2.08
                \\  \ch{Rb^{+}}       & 2.07 & \ch{I^-}          & 2.05
                \\  \ch{Cs^{+}}       & 2.06 & \ch{NO3^-}        & 1.90
                \\  \ch{Ag^{+}}       & 1.65 & \ch{CH3COO^-}     & 1.09
                \\  \ch{NH4^{+}}      & 1.96 & \ch{CH3CH2COO^-}  & 0.95
                \\  \ch{N(C4H9)4^{+}} & 0.52 & \ch{B(C6H5)4^-}   & 0.53
                \\  \ch{Ca^{2+}}      & 0.79 & \ch{SO4^-}        & 1.06
                \\  \ch{Mg^{2+}}      & 0.71 & \ch{CO3^{2-}}     & 0.92
                \\  \ch{La^{3+}}      & 0.62 & \ch{Fe(CN)6^{3-}} & 0.98

            \\\bottomrule
        \end{tabular}
        \paragraph*{Note:} Values at infinite dilution in \qty*{1e-5}{\cm^2/\second}. Calculated from data of Robinson and Stokes (1960)
        \vspace{2ex}
    \end{center}
    
\end{sectionBox}

\begin{sectionBox}2{Eletrolitos Fortes (1:1)} % MARK: S
    
    \begin{BM}
        J_{+}-J_{-}
        = i/\myvert{z}
    \end{BM}
    \begin{description}[
        leftmargin=!,
        labelwidth=\widthof{z} % Longest item
    ]
        \begin{multicols}{2}
            \item[\(i\)] Densidade de corrente
            \item[\(z\)] Carga ionica
            \item[\(+\)] Cation
            \item[\(-\)] Anion
        \end{multicols}
    \end{description}

    \subsection{Fluxo dos ioes}
    \begin{BM}
        J_1
        = -\frac{
            2
            \,\mathscr{D}_2
            \,\gdif{c_1}
            +i/\myvert{z}
        }{
            1+
            \mathscr{D}_2
            /\mathscr{D}_1
        }
    \end{BM}
    \begin{align*}
        J_+=J_-
        &\impliedby i=0
        \quad\text{(Sem corrente)} 
        \\
        J_1
        = \frac{i/\myvert{z}}{
            1
            +\mathscr{D}_2
            /\mathscr{D}_1
        }
        &\impliedby \gdif{c}=0
        \quad\text{(muito agitado)}
    \end{align*}
    \begin{align*}
        \mathscr{D}
        &=\frac{n}
        {\sum_{i=1}^{n}{\mathscr{D}_i^{-1}}}
        = H(\mathscr{D}_i)
        ;&
        t_i
        =\frac
        {\mathscr{D}_i}
        {\sum{\mathscr{D}_j}}
    \end{align*}
    \begin{description}[
        leftmargin=!,
        labelwidth=\widthof{} % Longest item
    ]
        \item[\(t_i\)] Numero de transferencia (fração da corrente transportada pelo ion \textit{i})
        \item[\(H\)] Média harmonica
    \end{description}
    
\end{sectionBox}

\begin{exampleBox}1{ % MARK: E1
    Difusão Qual o valor do coeficiente de difusão a \qty*{25}{\celsius} de \ch{HCl} em água? Calcule o nº de transferência para o protão nestas condições.
} % E1
    
    \answer{}
    \begin{center}\Large
        \ch{
            HCl + H2O -> H3O^+ + Cl^-
        }
    \end{center}
    \begin{flalign*}
        &
            \text{Coeficiente de difusão}&\\&
            \mathscr{D}
            = \frac{2}{
                \mathscr{D}_{\ch{H3O^+}}^{-1}
                +\mathscr{D}_{\ch{Cl^-}}^{-1}
            }
            \cong \frac{2}{
                9.31^{-1}
                +2.03^{-1}
            }
            \cong\qty
            {3.33320987654321}
            {\cm^2/\s}
            % 
            % 
            % 
            ; &\\[3ex]&
            \text{Numero de transferencia para }\ch{H3O^+}&\\&
            t_{\ch{H3O^+}}
            = \frac
            {\mathscr{D}_{\ch{H3O^+}}}
            {
                \mathscr{D}_{\ch{H3O^+}}
                +\mathscr{D}_{\ch{Cl^-}}
            }
            = \left(
                1
                +\mathscr{D}_{\ch{Cl^-}}
                /\mathscr{D}_{\ch{H3O^+}}
            \right)^{-1}
            = \left(
                1
                +2.03
                /9.31
            \right)^{-1}
            \cong &\\&
            \cong\qty
            {82.0987654320988}
            {\percent}
        &
    \end{flalign*}
    
\end{exampleBox}

\begin{sectionBox}1{Ionic conductivity and diffusion at infinite dilution} % MARK: S
    
    \begin{center}
        \vspace{1ex}
        \setlength\tabcolsep{6mm} % width
        \rowcolors*{2}{background!95!foreground}{}
        \begin{tabular}{
            l
            S[table-format=2.1,round-precision=2]
            S[table-format=1.3,round-precision=3]
        }
            \toprule
            
                Inorganic Cations
                & \multicolumn{1}{C}{
                    \cfrac{
                        \varLambda_{\pm}
                    }{
                        \qty{e-4}{\m^2\,S/\mole}
                    }
                }
                & \multicolumn{1}{C}{
                    \cfrac{
                        \mathscr{D}
                    }{
                        \qty{e-5}{\cm^2/\s}
                    }
                }

            \\\midrule

                \ch{Ag^+}                   & 61.9  & 1.648
            \\  \ch{1/3 Al^{3+}}            & 19    & 0.541
            \\  \ch{1/2 Ba^{2+}}            & 63.6  & 0.847
            \\  \ch{1/2 Be^{2+}}            & 45    & 0.599
            \\  \ch{1/2 Ca^{2+}}            & 59.47 & 0.792
            \\  \ch{1/2 Cd^{2+}}            & 54    & 0.719
            \\  \ch{1/3 Ce^{3+}}            & 69.8  & 0.620
            \\  \ch{1/2 Co^{2+}}            & 55    & 0.732
            \\  \ch{1/3 [Co(NH3)6]^{3+}}    & 101.9 & 0.904
            \\  \ch{1/3 [Co(en)3]^{6+}}     & 74.7  & 0.663
            \\  \ch{1/6 [Co2(trien)3]^{6+}} & 69    & 0.306
            \\  \ch{1/3 Cr^{3+}}            & 67    & 0.595
            \\  \ch{Cs^{+}}                 & 77.2  & 2.056

            \\\toprule

                % \rowcolor{}

                Inorganic Anions
                & \multicolumn{1}{C}{
                    \cfrac{
                        \varLambda_{\pm}
                    }{
                        \qty{e-4}{\m^2\,S/\mole}
                    }
                }
                & \multicolumn{1}{C}{
                    \cfrac{
                        \mathscr{D}
                    }{
                        \qty{e-5}{\cm^2/\s}
                    }
                }

            \\\midrule
            
                \ch{Au(CN)2^-}          & 50    & 1.331
            \\  \ch{Au(CN)4^-}          & 36    & 0.959
            \\  \ch{B(C6H5)4^-}         & 21    & 0.559
            \\  \ch{Br^-}               & 78.1  & 2.080
            \\  \ch{Br3^-}              & 43    & 1.145
            \\  \ch{BrO3^-}             & 55.7  & 1.483
            \\  \ch{CN^-}               & 78    & 2.077
            \\  \ch{CNO^-}              & 64.6  & 1.720
            \\  \ch{1/2 CO3^{2-}}       & 69.3  & 0.923
            \\  \ch{Cl^-}               & 76.31 & 2.032
            \\  \ch{ClO2^-}             & 52    & 1.385
            \\  \ch{ClO3^-}             & 64.6  & 1.720
            \\  \ch{ClO4^-}             & 67.3  & 1.792
            \\  \ch{1/3 [Co(CN)6]^{3-}} & 98.9  & 0.878
            \\  \ch{1/2 CrO4^{2-}}      & 85    & 1.132
            
            \\\bottomrule
        \end{tabular}
        \vspace{2ex}
    \end{center}
    
\end{sectionBox}

\end{document}
% !TEX root = ./FT_II-Slides_Annotations.7.tex
\providecommand\mainfilename{"./FT_II-Slides_Annotations.tex"}
\providecommand \subfilename{}
\renewcommand   \subfilename{"./FT_II-Slides_Annotations.7.tex"}
\documentclass[\mainfilename]{subfiles}

% \tikzset{external/force remake=true} % - remake all

\begin{document}

% \graphicspath{{\subfix{./.build/figures/FT_II-Slides_Annotations.7}}}
% \tikzsetexternalprefix{./.build/figures/FT_II-Slides_Annotations.7/graphics/}

\mymakesubfile{7}
[FT II]
{Difusão em estado Pseudo-Estacionário} % Subfile Title
{Difusão em estado Pseudo-Estacionário} % Part Title


\begin{sectionBox}1{Difusão em estado pseudo estacionário} % S

    \begin{flalign*}
        &
            \text{From an unsteady-state material balance:}&\\&
            \text{In}-\text{Out}
            = 0-Q_{A}
            = -N_{A}\,S
            = -\left(
                \frac
                {C_{A}\,\mathscr{D}_{A,B}}
                {\Theta\,\eta_d\,l}
                \ln{\frac
                    {1-\Theta\,y_{A,2}}
                    {1-\Theta\,y_{A,1}}
                }
            \right)
            \,S
            = &\\&
            =\text{Accumulation}
            = C_{A,L}\odv{\vol}{t}
            %
            %
            %
            ; &\\[3ex]&
            N_{A,z}\,\adif{z}
            = -\frac
            {C_{A}\,\mathscr{D}_{A,B}}
            {1-y_A}
            \,\odv{y_A}{z}
        &
    \end{flalign*}

    \begin{BM}
        z = f(t)
        \begin{cases}
            t_0=0 & z_{0}
            \\
            t & z_{t}
        \end{cases}
        \\
        N_A=f(z)\iff N_A=f(t)
        \\
        Q_A = -C_{A,L}\odv{V}{t}
        \qquad
        N_A = C_{A,L}\odv{z}{t}
    \end{BM}

    \paragraph*{Caracterização:}
    Se a distancia do caminho da difusão variar pequenas quantidade por um longo periodo de tempo, pode-se usar o modelo de difusão em estado pseudo-estacionário.

\end{sectionBox}

\begin{exampleBox}1{ % MARK: E1
    Encontre a equação para encontrar o tempo que um liquido em um tubo evaporando em função da altura do liquido
} % E1
    \paragraph*{Dados:}
    \begin{itemize}
        \item Difusão por um filme estagnado (\(N_{B}=0\))
        % \item Altura do cilindro: L
    \end{itemize}
    \answer{}
    \begin{flalign*}
        &
            C_{A,L}\,\odv{\vol}{t}
            = C_{A,L}\,\odv{(S*z)}{t}
            = C_{A,L}\,S\,\odv{z}{t}
            = -S\,N_{A}
            = &\\&
            = -S\,\left(
                \frac
                {C_{A}\,\mathscr{D}_{A,B}}
                {\Theta\,\eta_d\,(L-z)}
                \ln{\frac
                    {1-\Theta\,y_{A,1}}
                    {1-\Theta\,y_{A,0}}
                }
            \right)
            = &\\&
            = -S\,\frac
            {
                \frac{P}{R\,T}
                \,\mathscr{D}_{A,B}
            }
            {
                1*1\,(L-z)
            }
            \ln{\frac
                {1-1*y_{A,1}}
                {1-1*y_{A,0}}
            }
            = &\\&
            = -S\,\frac
            {
                P
                \,\mathscr{D}_{A,B}
            }
            {
                R\,T\,(L-z)
            }
            \ln{\frac
                {1-y_{A,1}}
                {1-y_{A,0}}
            }
            % 
            % 
            % 
            \implies &\\[3ex]&
            \implies
            \int_{Z}^{z}{
                (L-z)\odif{z}
            }
            = -\int_{Z}^{z}{
                (L-z)\odif{(L-z)}
            }
            = -(
                \adif{(L-z)^2/2}
            )
            = &\\&
            = -\frac{1}{2}(
                (L-z)^2
                -(L-Z)^2
            )
            % = &\\&
            = L\,(z-Z)
            +(Z^2-z^2)/2
            % 
            % 
            % 
            = &\\[3ex]&
            = \int_{0}^{t}{
                \frac
                {-P\,\mathscr{D}_{A,B}}
                {R\,T\,C_{A,L}}
                \ln{\left(
                    \frac
                    {1-y_{A,1}}
                    {1-y_{A,0}}
                \right)
                }
                \,\odif{t}
            }
            = 
            \frac
            {-P\,\mathscr{D}_{A,B}}
            {R\,T\,C_{A,L}}
            \ln{\left(
                \frac
                {1-y_{A,1}}
                {1-y_{A,0}}
            \right)
            }
            \int_{0}^{t}{
                \odif{t}
            }
            = &\\&
            = 
            \frac
            {-P\,\mathscr{D}_{A,B}}
            {R\,T\,C_{A,L}}
            \ln{\left(
                \frac
                {1-y_{A,1}}
                {1-y_{A,0}}
            \right)}
                = \frac{
                    L\,(z-Z)
                    +(Z^2-z^2)/2
                }{
                    \frac
                    {-P\,\mathscr{D}_{A,B}}
                    {R\,T\,C_{A,L}}
                    \ln{\left(
                        \frac
                        {1-y_{A,1}}
                        {1-y_{A,0}}
                    \right)
                    }
                }
            }
            % 
            % 
            % 
            ; &\\[3ex]&
            \text{Condições de fronteira do fluxo}&\\&
            \begin{cases}
                    z_0=z;&\quad y_{A,0}
                \\  z_1=L;&\quad y_{A,1}
            \end{cases}
            % 
            % 
            % 
            ; &\\[3ex]&
            \Theta
            = 1-N_B/N_A
            = 1-0/N_A
            = 1
            % 
            % 
            % 
            ; &\\[3ex]&
            \eta_{d,\text{plano}}
            = 1
            % 
            % 
            % 
            ; &\\[3ex]&
            \text{Condições de fronteira da evaporação}&\\&
            \begin{cases}
                    t_0=0 ;&\quad z_0=Z
                \\  t_1=t ;&\quad z_1=z
            \end{cases}
        &
    \end{flalign*}
\end{exampleBox}

\begin{exampleBox}1{ % MARK: E2
    Encontre a equação para o tempo em que uma esfera arde em função do raio da esfera, considere difusão por filme estagnado
} % E2
    \answer{}
    \begin{flalign*}
        &
            C_{A,L}
            \,\odv{\vol}{t}
            = C_{A,L}
            \,\odv{(\pi\,r^3\,4/3)}{t}
            = C_{A,L}
            \,\pi\,r^2\,4\odv{r}{t}
            % 
            % 
            % 
            = &\\[3ex]&
            = -S\,N_{A}
            = -\left(
                4\,\pi\,r^2
            \right)\,\left(
                \frac
                {C_A\,\mathscr{D}_{A,B}}
                {\Theta\,\eta_d\,r}
                \ln{\frac
                    {1-\Theta\,y_{A,1}}
                    {1-\Theta\,y_{A,0}}
                }
            \right)
            = &\\&
            = -\left(
                4\,\pi\,r^2
            \right)\,\left(
                \frac
                {\frac{P}{R\,T}\,\mathscr{D}_{A,B}}
                {1*1*r}
                \ln{\frac
                    {1-1*y_{A,1}}
                    {1-1*y_{A,0}}
                }
            \right)
            % 
            % 
            % 
            \implies &\\[3ex]&
            \implies
            \int_{R}^{r}{
                r\,\odif{r}
            }
            = (r^2-R^2)/2
            % 
            % 
            % 
            = &\\[3ex]&
            = \int_{0}^{t}{
                -\frac
                {P\,\mathscr{D}_{A,B}}
                {R\,T\,C_{A,L}}
                \ln{\left(\frac
                    {1-y_{A,1}}
                    {1-y_{A,0}}
                \right)
                }
                \,\odif{t}
            }
            % = &\\&
            = 
            -\frac
            {P\,\mathscr{D}_{A,B}}
            {R\,T\,C_{A,L}}
            \ln{\frac
                {1-y_{A,1}}
                {1-y_{A,0}}
            }
            \int_{0}^{t}{
                \odif{t}
            }
            = &\\&
            = 
            -t
            \,\frac
            {P\,\mathscr{D}_{A,B}}
            {R\,T\,C_{A,L}}
            \ln{\frac
                {1-y_{A,1}}
                {1-y_{A,0}}
            }
            % 
            % 
            % 
            \implies &\\[3ex]&
            \implies
            \mathemph{
                t
                = \frac{R^2-r^2}{
                    2\,\frac
                    {P\,\mathscr{D}_{A,B}}
                    {R\,T\,C_{A,L}}
                    \ln{\frac
                        {1-y_{A,1}}
                        {1-y_{A,0}}
                    }
                }
            }
            % 
            % 
            % 
            ; &\\[3ex]&
            \text{Condições de fronteira fluxo:}&\\&
            \begin{cases}
                    r_0=r;&\quad            y_{A,0}
                \\  r_1\to\infty ;&\quad    y_{A,1}
            \end{cases}
            % 
            % 
            % 
            ; &\\[3ex]&
            \eta_{d,\text{Esfera}}
            = 1-r_1/r_0
            % 
            % 
            % 
            ; &\\[3ex]&
            \Theta
            = 1+N_{B}/N_{A}
            = 1+0/N_{A}
            = 1
            % 
            % 
            % 
            ; &\\[3ex]&
            \text{Condições de fronteira arder:}&\\&
            \begin{cases}
                    t_0=0 ;&\quad r_0=R
                \\  t_1=t ;&\quad r_1=r
            \end{cases}
        &
    \end{flalign*}
\end{exampleBox}

\begin{exampleBox}1{ % MARK: E3
    Uma camada de água com \qty*{1}{\mm} de espessura é mantida a \qty*{20}{\celsius} em contacto com o ar seco a \qty*{1}{\atm}. Admitindo que a evaporação se dá por difusão molecular através de uma camada de ar estagnado com \qty*{5}{\mm} de espessura, calcule o tempo necessário para que a água evapore completamente. O coeficiente de difusão de água no ar é \qty*{0.26}{\cm^2/\s} e a pressão de vapor da água a \qty*{20}{\celsius} é \qty*{2.34e-2}{\atm}.
} % E3
    \answer{}
    \begin{flalign*}
        &
            C_{A,L}\,\odv{\vol}{t}
            = \frac{\rho_{A}}{M_A}\,\odv{(S*z)}{t}
            = \frac{\rho_{A}}{M_A}\,S\,\odv{z}{t}
            = &\\&
            = -S\,N_{A}
            = -S\,\left(
                \frac
                {C_A\,\mathscr{D}_{A,B}}
                {\Theta\,\eta_d\,l}
                \ln{\frac
                    {1-\Theta\,y_{A,1}}
                    {1-\Theta\,y_{A,0}}
                }
            \right)
            = &\\&
            = -S\,\left(
                \frac
                {\frac{P}{R\,T}\,\mathscr{D}_{A,B}}
                {1*1*(6\E{-3}-z)}
                \ln{\frac
                    {1-1*y_{A,1}}
                    {1-1*y_{A,0}}
                }
            \right)
            % 
            % 
            % 
            \implies &\\[3ex]&
            \implies
            \int_{1\E{-3}}^{0}{
                (6\E{-3}-z)\,\odif{z}
            }
            = -\int_{1\E{-3}}^{0}{
                (6\E{-3}-z)\,\odif{(6\E{-3}-z)}
            }
            = &\\&
            = -\frac{1}{2}\left(
                (6\E{-3}-0)^2
                -(6\E{-3}-1\E{-3})^2
            \right)
            = -((6\E{-3})^2-(5\E{-3})^2)/2
            % 
            % 
            % 
            = &\\[3ex]&
            = \int_{0}^{t}{
                -\left(
                    \frac
                    {P\,\mathscr{D}_{A,B}}
                    {R\,T}
                    \,\frac{M_A}{\rho_A}
                    \ln{\frac
                        {1-y_{A,1}}
                        {1-y_{A,0}}
                    }
                \right)
                \,\odif{t}
            }
            = 
            -\left(
                \frac
                {P\,\mathscr{D}_{A,B}}
                {R\,T}
                \,\frac{M_A}{\rho_A}
                \ln{\frac
                    {1-y_{A,1}}
                    {1-y_{A,0}}
                }
            \right)
            \int_{0}^{t}{
                \,\odif{t}
            }
            = &\\& 
            =
            -\left(
                \frac
                {P\,\mathscr{D}_{A,B}}
                {R\,T}
                \,\frac{M_A}{\rho_A}
                \ln{\frac
                    {1-y_{A,1}}
                    {1-y_{A,0}}
                }
            \right)
            t
            % 
            % 
            % 
            \implies &\\[3ex]&
            \implies
            t
            =\frac
            {
                (6\E{-3})^2
                -(5\E{-3})^2
            }
            {
                2\,\frac
                {P\,\mathscr{D}_{A,B}}
                {R\,T}
                \,\frac{M_A}{\rho_A}
                \ln{\frac
                    {1-y_{A,1}}
                    {1-y_{A,0}}
                }
            }
            % = &\\&
            =\frac
            {(6\E{-3})^2-(5\E{-3})^2}
            % 11
            {
                2\,\frac
                {1*0.26\E{-4}}
                {
                    \num{8.20573660809596e-5}
                    * (20+273.15)
                }
                \,\frac{18}{1\E{6}}
                \ln{\frac
                    {1-0}
                    {1-2.34\E{-2}}
                }
                % 4.606655483479692e-10
            }
            \cong &\\&
            \cong\mathemph{
                \qty
                {1.193924750770707e4}
                {\second}
            }
            \cong\mathemph{
                \qty*{3}{\hour}
                \qty
                {18.987458461784427}
                {\minute}
            }
            % 
            % 
            % 
            ; &\\[3ex]&
            \text{Condições de fronteira fluxo:}&\\&
            \begin{cases}
                    z_0=z ;&\quad 
                    y_{A,0}=P_{A}/P=2.34\E{-2}/1
                \\  z_1=Z+5\E{-3}=6\E{-3} ;&\quad
                    y_{A,1}=0
            \end{cases}
            % 
            % 
            % 
            ; &\\[3ex]&
            \eta_{d,\text{Plano}}
            = 1
            % 
            % 
            % 
            ; &\\[3ex]&
            \Theta
            = 1+N_B/N_A
            = 1+0/N_A
            = 1
            % 
            % 
            % 
            ; &\\[3ex]&
            \text{Condições de fronteira evaporação:}&\\&
            \begin{cases}
                    t_0=0 ;&\quad
                    z_0=Z=1\E{-3}
                \\  t_1=t ;&\quad
                    z_1=0
            \end{cases}
        &
    \end{flalign*}
\end{exampleBox}

\begin{exampleBox}1{ % MARK: E4
    Calcule o tempo necessário para sublimar completamente uma esfera de naftleno (\ch{C10H8}) cujo diâmetro inicial é \qty*{1}{\cm}. A esfera está colocada numa quantidade ``infinita'' de ar a \qty*{318}{\K}.
} % E4
    \paragraph*{Dados:}
    \begin{itemize}
        \begin{multicols}{3}
            \item \(P^*_{\ch{C10H8}}=\qty*{0.106}{\atm}\)
            \item \(\rho_{\ch{C10H8}}=\qty*{1140}{\kg/\m^3}\)
            \item \(\mathscr{D}_{naft-ar}=\qty*{6.9e-7}{\m^2/\s}\)
        \end{multicols}
    \end{itemize}
    \answer{}
    \begin{flalign*}
        &
            C_{A,L}\,\odv{\vol}{t}
            = \frac{\rho_{A}}{M_A}
            \,\odv{(\pi\,r^3\,4/3)}{t}
            = \frac{\rho_{A}}{M_A}
            \,4\,\pi\,r^2
            \,\odv{r}{t}
            = \frac{\rho_{A}}{M_A}
            \,S
            \,\odv{r}{t}
            % 
            % 
            % 
            = &\\[3ex]&
            = -S\,N_{A,r}
            = -S\,\left(
                \frac
                {C_{A}\,\mathscr{D}_{A,B}}
                {\Theta\,\eta_d\,r}
                \ln{\frac
                    {1-\Theta\,y_{A,1}}
                    {1-\Theta\,y_{A,0}}
                }
            \right)
            = &\\&
            = -S\,\left(
                \frac
                {\frac{P}{R\,T}\,\mathscr{D}_{A,B}}
                {1*1*r}
                \ln{\frac
                    {1-1*(0)}
                    {1-1*y_{A,0}}
                }
            \right)
            % 
            % 
            % 
            \implies &\\[3ex]&
            \implies
            \int_{R_0}^{0}{
                r\,\odif{r}
            }
            = (0^2-R_0^2)/2
            % 
            % 
            % 
            = &\\[3ex]&
            = \int_{0}^{t}{
                -\left(
                    \frac
                    {P\,\mathscr{D}_{A,B}}
                    {R\,T}
                    \,\frac{M_A}{\rho_{A}}
                    \ln{\frac
                        {1}
                        {1-y_{A,0}}
                    }
                \right)
                \,\odif{t}
            }
            = &\\&
            = 
            -\left(
                \frac
                {P\,\mathscr{D}_{A,B}}
                {R\,T}
                \,\frac{M_A}{\rho_{A}}
                \ln{\frac
                    {1}
                    {1-y_{A,0}}
                }
            \right)
            \int_{0}^{t}{
                \,\odif{t}
            }
            = &\\&
            = 
            -\left(
                \frac
                {P\,\mathscr{D}_{A,B}}
                {R\,T}
                \,\frac{M_A}{\rho_{A}}
                \ln{\frac
                    {1}
                    {1-y_{A,0}}
                }
            \right)
            t
            % 
            % 
            % 
            \implies &\\[3ex]&
            \implies
            t
            =
            \frac{
                R\,T
                \,\rho_A
                \,R_0^2/2
            }
            {
                P
                \,\mathscr{D}_{A,B}
                \,M_{A}
                \ln{\frac
                    {1}
                    {1-y_{A,0}}
                }
            }
            \cong &\\&
            \cong
            \frac{
                \num{8.20573660809596e-5}
                * 318
                * 1140\E{3}
                * (0.5\E{-2})^2/2
            }
            {
                1
                * 6.9\E{-7}
                * (10*12+8)
                \ln{\frac
                    {1}
                    {1-0.106}
                }
            }
            \cong &\\&
            \cong\mathemph{
                \qty
                {3.757427061673796775e4}
                {\second}
            }
            \cong\mathemph{
                \qty*
                {10}
                {\hour}
                \,\qty
                {26.237843612299463}
                {\minute}
            }
            % 
            % 
            % 
            ; &\\[3ex]&
            \text{Condições de fronteira fluxo:}&\\&
            \begin{cases}
                    r_0=r ;&\quad
                    y_{A,0} = P^*_{A}/P = 0.106
                \\  r_1\to\infty ;&\quad
                    y_{A,1} = 0
            \end{cases}
            % 
            % 
            % 
            ; &\\[3ex]&
            \eta_{d,\text{Esfera}}
            = 1-r_0/r_1
            \to 1
            % 
            % 
            % 
            ; &\\[3ex]&
            \Theta
            = 1-N_B/N_A
            = 1-0/N_A
            = 1
            % 
            % 
            % 
            ; &\\[3ex]&
            \text{Condições de fronteira sublimação:}&\\&
            \begin{cases}
                    t_0=0 ;&\quad r_0=R_0
                \\  t_1=t ;&\quad r_1=0
            \end{cases}
        &
    \end{flalign*}
\end{exampleBox}

\begin{sectionBox}2{} %

    \begin{BM}
        t = \frac{
            C_{A,l}\,\adif{(z^2)}
        }{
            2\,D_{A,B}\,C
            \,\ln\frac{1-y_{A,1}}{1-y_{A,0}}
        }
    \end{BM}

    \begin{flalign*}
        &
            N_A
            = \frac{D_{A,B}\,C}{z}
            \,\ln\frac{1-y_{A,1}}{1-y_{A,0}}
            % = &\\[3ex]&
            = C_{A,l}\odv{z}{t}
            \implies &\\&
            \implies
            \int\odif{t}
            = t
            = &\\&
            = \int{
                    \frac{
                    C_{A,l}
                }{
                    D_{A,B}\,C
                    \,\ln\frac{1-y_{A,1}}{1-y_{A,0}}
                }
                \,z\,\odif{z}
            }
            =
            \frac{
                C_{A,l}
            }{
                D_{A,B}\,C
                \,\ln\frac{1-y_{A,1}}{1-y_{A,0}}
            }
            \int{
                z\,\odif{z}
            }
            =
            \frac{
                C_{A,l}
            }{
                D_{A,B}\,C
                \,\ln\frac{1-y_{A,1}}{1-y_{A,0}}
            }
            \frac{\adif{(z^2)}}{2}
            = &\\&
            =
            \frac{
                C_{A,l}\,\adif{(z^2)}
            }{
                2\,D_{A,B}\,C
                \,\ln\frac{1-y_{A,1}}{1-y_{A,0}}
            }
        &
    \end{flalign*}

\end{sectionBox}

\begin{sectionBox}2{Geometria esférica}

    \begin{BM}
        t
        = \frac{C_{A,l}}{
            2\,D\,C\,ln{(1-y_{A,0})^{-1}}
        }\adif{(-r^2)}
    \end{BM}

    \begin{flalign*}
        &
            \lim_{
                \substack{
                    r_2 \to\infty
                    \\ y_{A,1} \to0
                }
            }
            -C_{A,l}\,4\,\pi\,r^2\,\odv{r}{t}
            = \frac{4\,\pi\,D\,C}{r_0^{-1}}
            \,\ln{(1-y_{A,0})^{-1}}
        &
    \end{flalign*}

\end{sectionBox}

\begin{exampleBox}1{
    Calcule o tempo necessário para que a água evapore completamente.
} % E

    \begin{itemize}
        \item Uma camada de água com 1\,\unit{\milli\metre} de espessura
        \item É mantida a 20\,\unit{\celsius}
        \item em contato com o ar seco a 1\,\unit{\atm}
        \item Admitindo que a evaporação  se dá por difusão molecular através de uma camada de ar estagnado com 5\,\unit{\milli\metre} de espessura
        \item O coeficiente de difusão de água no ar é 0.26\,\unit{\centi\metre^2/\second}
        \item A pressão de vapor da água a 20\,\unit{\celsius} é 0.0234\,\unit{\atm}
    \end{itemize}

    \answer{}
    \begin{flalign*}
        &
            N_{A}
            = y_A(N_{A}+N_{B})
            - \frac{P\,D_{A,B}}{R\,T}
            \,\odv{y_A}{z}
            = y_A\,N_{A}
            - \frac{P\,D_{A,B}}{R\,T}
            \,\odv{y_A}{z}
            %
            %
            %
            \implies &\\[3ex]&
            \implies
            \int{N_{A}\odif{z}}
            = N_{A}\int{\odif{z}}
            = N_{A}\adif{z}
            %
            %
            %
            = &\\&
            \int{
                -\frac{P\,D_{A,B}}{R\,T}
                \frac{\odif{y_A}}{1-y_A}
            }
            =
            -\frac{P\,D_{A,B}}{R\,T}
            \int{
                \frac{\odif{y_A}}{1-y_A}
            }
            =
            \frac{P\,D_{A,B}}{R\,T}
            \ln\frac{1-y_{A,1}}{1-y_{A,0}}
            %
            %
            %
            \xRightarrow[y_{A,1}=0]{} &\\[3ex]&
            \xRightarrow[y_{A,1}=0]{}
            N_A
            = &\\&
            = \frac{P\,D_{A,B}}{R\,T\,\adif{z}}
            \,\ln\frac{1}{1-y_{A,0}}
            = \frac{P\,D_{A,B}}{R\,T\,\delta}
            \,\ln\frac{1}{1-y_{A,0}}
            = &\\&
            = Q_A/S
            = -C_{A,l}\,\odv{V}{t}
            \,\frac{1}{S}
            = -C_{A,l}\,\left(-S\,\odv{\delta}{t}\right)
            \,\frac{1}{S}
            = C_{A,l}\,\odv{\delta}{t}
            \implies &\\&
            \implies
            \int{C_{A,l}\,\delta\,\odif{\delta}}
            =
            C_{A,l}
            \int{
                \delta\,\odif{\delta}
            }
            = C_{A,l}
            \adif{(\delta^2)}/2
            = &\\&
            = \int{
                \frac{P\,D_{A,B}}{R\,T}
                \ln\frac{1}{1-y_{A,0}}
                \,\odif{t}
            }
            =
            \frac{P\,D_{A,B}}{R\,T}
            \ln\frac{1}{1-y_{A,0}}
            \int{
                \odif{t}
            }
            =
            \frac{P\,D_{A,B}}{R\,T}
            \ln\frac{1}{1-y_{A,0}}
            \adif{t}
            \implies &\\&
            \implies
            \adif{t}
            = \frac{
                C_{A,l}
                \adif{(\delta^2)}/2
            }{
                \frac{P\,D_{A,B}}{R\,T}
                \ln\frac{1}{1-y_{A,0}}
            }
            = \frac{
                C_{A,l}
                \,\adif{(\delta^2)}
                \,R\,T
            }{
                2\,P\,D_{A,B}
                \ln\frac{1}{1-p_{A,0}/P}
            }
            = &\\&
            = \frac{
                \left(
                    \frac{1000\,\unit{\kilo\gram\of{Agua}}}{\unit{\metre^3\of{Agua}}}
                    \frac{\unit{\mole\of{Agua}}}{18\,\unit{\gram\of{Agua}}}
                \right)
                % 55.555555555555556e3
                *((6\E-3)^2-(5\E-3)^2)
                % 0.000011
                *\num{8.20573660809596e-5}
                *(20+273.15)
            % 1470.034919627591061-5
            }{
                2
                * 1
                * (0.26\E-4)
                \ln\frac{1}{1-0.0234}
                % 0.023678127354577
            % 0.1231262622438004e-5
            }
            \cong &\\&
            \cong
            \qty{11939.247507707150908}{\second}
            \frac
            {\unit{\hour}}
            {3600\,\unit{\second}}
            \cong
            \qty{3.316457641029764}{\hour}
        &
    \end{flalign*}

\end{exampleBox}

\end{document}

% !TEX root = ./FT_II-Notes.1.tex
\providecommand\mainfilename{"./FT_II-Notes.tex"}
\providecommand \subfilename{}
\renewcommand   \subfilename{"./FT_II-Notes.1.tex"}
\documentclass[\mainfilename]{subfiles}

% \tikzset{external/force remake=true} % - remake all

\begin{document}

% \graphicspath{{\subfix{./.build/figures/FT_II-Notes.1}}}
% \tikzsetexternalprefix{./.build/figures/FT_II-Notes.1/}

\mymakesubfile{1}
[FT II]
{Anotações: Transferencia de massa entre fases} % Subfile Title
{Transferencia de massa entre fases} % Part Title

\begin{sectionBox}1{Adsorsão gasosa \ch{F\gas{} -> F\lqd{}}} % S
    
    \begin{BM}[align*]
        \omega_A &= N_A\,A_{\text{transf}};
        \\ N_{A,z} &= k_g (\rho_{A,g} - \rho_{A,i});
        \\ N_{A,z} &= k_l (c_{A,i}    - c_{A,l});
        \quad&& c_{l} = \rho_l/M;
        \\ N_{A,z} &= k_x ( x_{A,i}    - x_{A,l});
        \quad&& \begin{cases}
            k_x=k_l\,c_l
            \\
            x_A\cong c_A/c_l \text{ (A muito pequeno no l)}
        \end{cases}
        \\ N_{A,z} &= k_c ( c_{A,g}    - c_{A,i,g});
        \quad&& k_{c}=k_g\,R\,T
        \\ N_{A,z} &= k_y ( y_{A,g}    - y_{A,i});
        \quad&& k_y = k_g\,P
    \end{BM}

    \subsection*{Coeficientes globais}
    \begin{BM}[align*]
        N_{A,z}&=k_g\,(\rho_{A,g}-\rho_{A,*})
        \\
        N_{A,z}&=k_l\,(c_{A,*}-\rho_{A,l})
    \end{BM}
    \begin{description}[
        leftmargin=!,
        labelwidth=\widthof{} % Longest item
    ]
       \item[\(c_{A,*}\)] Concentração em equilíbrio
       \item[\(\rho_{A,*}\)] Pressão equilíbrio
    \end{description}
    
\end{sectionBox}

\begin{sectionBox}1{Linha de equilíbrio} % S
    
    \begin{BM}
        p_A=n\,c_A
        \begin{cases}
            p_{A,i}=n\,c_{A,i}
            \\ p_{A,g}=n\,c_{A,*}
            \\ p_{A,*}=n\,c_{A,l}
        \end{cases}
    \end{BM}
    \begin{BM}
        K_g^{-1}
        =\frac{p_{A,g}-p_{A,i}}{N_{A,z}}
        +\frac{n(c_{A,i}-c_{A,l})}{N_{A,z}}
    \end{BM}

    \paragraph*{Nota:} Um gás mais solúvel tem maior \(k_g\) <=> Transfere-se mais facilmente

    \begin{BM}
        K_g^{-1}=k_g^{-1}+k_l^{-1}\,n
        \\ 
        K_l^{-1}=k_l^{-1}+k_g^{-1}\,n
    \end{BM}

    \subsection*{Metodo gráfico}
    Permite determinar composições interfaciais e de equilíbrio

    \begin{sectionBox}2{Resistencia a fase gasosa} % S
        \begin{BM}
            R
            =\frac{k_g^{-1}}{K_g^{-1}}
            =\frac{p_{A,g}-p_{A,i}}{p_{A,g}-p_{A,*}}
            \qquad
            R\in\myrange*{0,2}
        \end{BM}
    \end{sectionBox}

    \begin{sectionBox}2{Resistencia Fase líquida} % S
        \begin{BM}
            R
            = \frac{k_l^{-1}}{K_l^{-1}}
            = \frac{n\,k_l^{-1}}{K_g^{-1}}
            = \frac{c_{A,i}-c_{A,l}}{c_{A,*}-c_{A,l}}
        \end{BM}
    \end{sectionBox}
    
\end{sectionBox}

\begin{sectionBox}1{Difusão com reação química} % S
    
    Gases pouco solúveis (resistencia fase liquida alta)

    \subsection*{1ª Ordem}
    \begin{BM}
        N_{A,z}=-\mathscr{D}_{A,B}\odv{c_A}{z}
        \qquad
        \text{Cond. Fronteira}
        \begin{cases}
            z=0, & c_A=c_{A,0}
            \\
            z=\delta, & c_A=0
        \end{cases}
        \\[2ex]
        N_A = k_{l,0}\,\frac{H_a}{\tanh{H_a}}\,(c_{A,i}-0)
    \end{BM}
    
\end{sectionBox}

\end{document}
% !TEX root = ./PS-Tests_Resolutions.2021.1.tex
\providecommand\mainfilename{"./PS-Tests_Resolutions.tex"}
\providecommand \subfilename{}
\renewcommand   \subfilename{"./PS-Tests_Resolutions.2021.1.tex"}
\documentclass[\mainfilename]{subfiles}

% \tikzset{external/force remake=true} % - remake all

\begin{document}

\graphicspath{{\subfix{./.build/figures/PS-Tests_Resolutions.2021.1}}}
% \tikzsetexternalprefix{./.build/figures/PS-Tests_Resolutions.2021.1/graphics/}

\mymakesubfile{1}
[PS]
{Test 2021.1 Resolution} % Subfile Title
{Test 2021.1 Resolution} % Part Title

\begin{questionBox}1{ % MARK: Q1
    Pretende-se absorver acetona presente numa mistura gasosa constituída por ar e acetona numa coluna de enchimento com \lemph{área de secção \qty*{0.186}{\metre^2}} e usando água em \lemph{contracorrente}. A composição da acetona na corrente gasosa à entrada é \lemph{\qty*{2.6}{\percent}} e pretende-se obter \lemph{\qty*{0.5}{\percent}} de acetona na corrente de saída. O caudal da mistura gasosa à entrada da coluna é \lemph{\qty*{13.7}{\kilo\mole/\hour}} e o caudal de água é \lemph{\qty*{43.6}{\kilo\mole/\hour}}. A relação de equilíbrio é \lemph{\(y^*= 1.2\,x\)} sendo \textit{y} e \textit{x} as fracções molares de acetona. Determine:
} % Q1
    \begin{BM}
        Z=H_{OG}\,N_{OG}
        ;\quad
        H_{OG} = \frac{G}{A\,K_y\,a}
        ;\quad
        N_{OG} = \int_{y_2}^{y_1}{\frac{\odif{y}}{y-y^*}}
    \end{BM}
    % Absorção
    % Coluna 2 pontos: A e B
    % Anotar dados p cada ponto
    % Entrada do gás é por baixo => Ponto B
    % Caldal da mistura Gasosa => G_b
    % Caldal de Agua => Ponto a quando o Liq entra
    % Fazer equação de eq (lei de Henry)
    % y^*=1.2\,x

    \begin{questionBox}2{ % MARK: Q1.1
        O caudal mínimo de água.
    } % Q1.1
        \answer{}
        \begin{flalign*}
            &
                L_{B,\min}: &\\&
                G_B\,y_B + L_A\,x_a
                = G_B\,y_B
                = &\\[1.5ex]&
                = G_A\,y_A + L_B\,x_B
                = G_A\,y_A + L_{B,\min}\,x_B^*
                = &\\&
                = G_A\,y_A + L_{B,\min}\,\frac{y_B}{1.2}
                ; &\\[3ex]&
                % Verificar se o caldal é constante
                E=\myvert{1-G_s/G_B}:&\\&
                G_B\,y_B
                = G_s\,\frac{y_B}{1-y_B}
                \implies &\\&
                \implies
                E
                = \myvert{1-\frac{G_s}{G_B}}
                = \myvert{1-1+y_B}
                = &\\&
                = 2.6\,\%
                < 10\,\%
                &\\&
                \therefore
                \begin{cases}
                    G_A=G_B=G&=\qty*{13.7}{\kilo\mole/\hour}
                    \\
                    L_A=L_B=L&=\qty*{43.6}{\kilo\mole/\hour}
                \end{cases}
                ; &\\[6ex]&
                \implies
                L_{B\,\min}
                = L_{\min}
                % 
                = (G_B\,y_B-G_A\,y_A)\frac{1.2}{y_B}
                = (G\,y_B-G\,y_A)\frac{1.2}{y_B}
                % 
                = &\\&
                = G\,(1-y_A/y_B)\,1.2
                \cong 13.7\,(1-0.5/2.6)\,1.2
                \cong &\\&
                \cong \qty[1]{13.278461538461538}{\kilo\mole/\hour}
            &
        \end{flalign*}
        % Verificar se o caldal é constante:
        % (Assim podemos usar o G_B dado)
            % G_s * \frac{y_B}{1-y_B} = G_B * y_B
            % G_A * y_A = (1-%) G_B * y_B
                % Quando dizem porcentagem de remoção ao invés da composição de saída
            % G_b Ar mais merdas
            % G_s Ar sem  merdas
            % Encontramos G_s a partir de G_b (\qty*{13.34}{\kilo\mole/\hour})
            % Calculamos o Erro ("Maior menos o menor pelo menor")
                % E = (G_B-G_s)/G_B = 2.6 % < 10% => Constante
                % => G_B = G_A = G \land L_A = L_B = L
                % Por ser constante usamos o G_B a partir de agora
                % L (caldal liquido) tambem é constante
        % Caldal minimo da agua (questão começa aqui)
        % Equilibrio constante ou não a equação é verdade:
        % G_B\,y_B + L_A\,x_a
        % = G_A\,y_A + L_B\,x_B
        % x_a = 0 (inicia limpo)
        % Como pedem L min: L\,x_B => L_{min}\,x_B^*
        % y_B^* = 0.022
        % L_min = \qty*{13.08}{\kilo\mole/\hour}
    \end{questionBox}
    \begin{questionBox}2{ % MARK: Q1.2
        A \% molar de acetona na corrente líquida à saída da coluna.
    } % Q1.2
        \answer{}
        \begin{flalign*}
            &
                x_B: &\\&
                G_B\,y_B 
                = G\,y_B 
                = &\\&
                = G_A\,y_A + L_B\,x_B
                = G\,y_A + L\,x_B
                \implies &\\&
                \implies
                x_B
                = G\,(y_B - y_A)/L
                \cong 13.7\,(2.6 - 0.5)/43.6
                \cong \qty[2]{.659862385321101}{\percent}
            &
        \end{flalign*}
    \end{questionBox}
    \begin{questionBox}2{ % MARK: Q1.3
        O nº de unidades de transferência.
        % Quermos o numero de unidades de transferencia
        % Dois tipos de colunas: Pratos, Enchimentos
            % Pratos:
                % Calculamos log o numero de pratos
            % Enchimentos:
                % Calculamos numero de pratos teoricos q usamos para achar a altura de enchimento
                % N_{OG} = \int_{y_A}^{y_B}{\odif{y}/(y-y^*)}
                % Pode ser aproximado por:
                % N_{OG} = \frac{y_B-Y_A}{\adif{\bar{y_L}}}
                % \adif{\bar{y_L}} 
                % = \frac
                %     {\adif{y_B}-\adif{y_A}}
                %     {\ln{\adif{y_B}/\adif{y_A}}}
                % \adif{y}_i = y_i-y_i^*
                % Da lei de henry pegamos y^* para cada
    } % Q1.3
        \answer{}
        \begin{flalign*}
            &
                N_{OG}: &\\&
                N_{OG}
                = \int_{y_A}^{y_B}{\odif{y}/(y-y^*)}
                \cong 
                \frac{y_B-y_A}{\adif{\bar{y_L}}}
                = &\\&
                = \frac{
                    y_B - y_A
                }{
                    \left(
                        \frac
                        {\adif{y_B}-\adif{y_A}}
                        {\ln{\adif{y_B}/\adif{y_A}}}
                    \right)
                }
                = &\\&
                = \frac{
                    y_B - y_A
                }{
                    \left(
                        \frac
                        {
                            (y_B-y_B^*)
                            -(y_A-y_A^*)
                        }
                        {\ln{
                            \frac
                            {y_B-y_B^*}
                            {y_A-y_A^*}
                        }}
                    \right)
                }
                = &\\&
                = \frac{
                    y_B - y_A
                }{
                    \left(
                        \frac
                        {
                             (y_B-1.2\,x_B)
                            -(y_A-1.2\,x_A)
                        }
                        {\ln{
                            \frac
                            {y_B-1.2\,x_B}
                            {y_A-1.2\,x_A}
                        }}
                    \right)
                }
                = &\\&
                = 
                \left(
                    \frac
                    {
                        1-\frac{1.2\,x_B}{y_B-y_A}
                    }
                    {\ln{
                        \frac
                        {y_B-1.2\,x_B}
                        {y_A}
                    }}
                \right)^{-1}
                \cong
                \left(
                    \frac
                    {
                        1-\frac{
                            1.2
                            *\qty[2]{.659862385321101}{\percent}
                        }{2.6-0.5}
                    }
                    {\ln{
                        \frac
                        {
                            2.6
                            -1.2
                            *\qty[2]{.659862385321101}{\percent}
                        }
                        {0.5}
                    }}
                \right)^{-1}
                \cong
                \num{2.063551038782582}
                \cong 2
            &
        \end{flalign*}
    \end{questionBox}
    \begin{questionBox}2{ % MARK: Q1.4
        A altura de enchimento necessária, sabendo que os coeficientes volumétricos individuais de transferência de massa são:
        \begin{BM}
            k_{y}\,a=\qty*{3.8E-2}{\kilo\mole/\second.\metre^3}
            ; \qquad
            k_{x}\,a=\qty*{6.2E-2}{\kilo\mole/\second.\metre^3}
        \end{BM}
        % Altura do enchimento
        % H_{OG} = \frac{G}{A\,K_{y}\,a}
        % G Segue a mesma unidade (massica/molar) com o coeficiente K_{Y}
        % "a" é uma area
        % para achar o K_y com base nos k_y:
            % \frac{1}{K_y\,a}
            % = \frac{1}{k_y\,a}
            % + \frac{m}{k_x\,a}
            % "m" é o declive da lei re henry
        % Se perguntassem a altura da coluna é o numero de pratos vezes o numero de pratos (unidade de transferencia)
        % Se perguntassem: caso usassemos uma coluna de prátos
            % N 
            % = \frac{\ln{\adif{y_B}/\adif{y_A}}}{\ln{L/m\,g}}
            % = \frac{\ln{\adif{y_B}/\adif{y_A}}}{\ln{A}}
            % "A" fator de absorção
    } % Q1.4
        \answer{}
        \begin{flalign*}
            &
                H_{OG}: &\\&
                H_{OG}
                = \frac{G}{A\,K_y\,a}
                = \frac{G}{A}
                \,\frac{1}{K_y\,a}
                = \frac{G}{L/m\,G}
                \,\left(
                    \frac{1}{k_y\,a}
                    + \frac{m}{k_x\,a}
                \right)
                = &\\&
                = \frac{m\,G^2}{L}
                \,\left(
                    \frac{1}{k_y\,a}
                    + \frac{m}{k_x\,a}
                \right)
                = &\\&
                = \frac{1.2*13.7^2}{43.6}
                \,\left(
                    \frac{1}{3.8\E{-2}*3600}
                    + \frac{1.2}{6.2\E{-2}*3600}
                \right)
                \cong
                \qty{655.345581325316924}{\metre^3}
            &
        \end{flalign*}
    \end{questionBox}
    \begin{questionBox}2{ % MARK: Q1.5
        Discuta como variaria a altura se diminuísse o declive da linha de equilíbrio. Isso implicaria operar a uma temperatura superior ou inferior?
    } % Q1.5
    \end{questionBox}
\end{questionBox}

% Destilação
\begin{questionBox}1{ % MARK: Q2
    Pretende-se dimensionar uma coluna de destilação para fraccionar \lemph{\qty*{100}{\kilo\mole/\hour}} de uma mistura de \lemph{\qty*{55}{\percent.\mole\of{A}} e \qty*{45}{\percent.\mole\of{C}}}. A alimentação encontra-se à temperatura de \lemph{\qty*{110}{\celsius}}. Pretende-se obter um destilado e um resíduo com \lemph{\qty*{95}{\percent.\mole} e \qty*{15}{\percent.\mole}} no composto mais volátil, respectivamente. Os dados de equilíbrio liquido-vapor encontram-se representados na figura junta. Calcule:
} % Q2
    \paragraph*{Dados:}
    \begin{itemize}
        \item Temperaturas de vaporização a \qty*{1}{\bar}
        \begin{itemize}\vspace{-2ex}
            \begin{multicols}{2}
                \item A Puro:      \qty*{80.1}{\celsius}
                \item C Puro:      \qty*{110.6}{\celsius}
                \item Alimentação: \qty*{94}{\celsius}
                \item Destilado:   \qty*{82}{\celsius}
                \item Resíduo:     \qty*{108}{\celsius}
            \end{multicols}
        \end{itemize}
        \item \(C_p\) misturas líquidas \ch{A + C}: \qty*{67}{\joule/\mole.\celsius}
        \item \(\adif{\hat{H}}_{vap}\) Misturas \ch{A + C}: \qty*{40.2}{\kilo\joule/\mole}
        \item Despreze eventuais efeitos da temperatura nos calores sensíveis e latente e eventuais perdas de calor e de pressão na coluna.
    \end{itemize}
    \paragraph*{Curva de equilíbrio:}
    \begin{center}
        \vspace{1ex}
        \begin{tabular}{L *{6}{C}}
            \toprule
            
                x & 0 & 0.20 & 0.40 & 0.60 & 0.80 & 1
                \\
                y & 0 & 0.38 & 0.62 & 0.79 & 0.91 & 1
            
            \\\bottomrule
        \end{tabular}\
        \vspace{2ex}
        \\{\textit{x, y} referem-se às composições do composto mais volátil nas fases líquida e vapor, respectivamente}
    \end{center}
    \begin{BM}[flalign*]
        &
            & y_{n+1}
            & = \frac{L}{V}\,x_n
            + \frac{D\,x_D}{V}
            = \frac{R}{R+1}\,x_n
            + \frac{x_D}{R+1}
        &\\&
            & y_{m+1}
            & = \frac{\bar{L}}{\bar{V}}\,x_m
            - \frac{B\,x_B}{\bar{V}}
        &\\&
            & y_i
            & = \frac{i}{i-1}\,x_i
            - \frac{x_F}{i-1}
        &
    \end{BM}
    % Ateção: Feed (F), Destilado (D) e residuo (B)
        % F = \qty*{100}{\kilo\mole/\hour}
        % Destilação => Dois compostos => O mais volátil é usado no gráfico (volátil = Ponto de eb baixo ou algo do tipo)
        % x_F = 0.55
        % x_D = 0.95
        % x_B = 0.15
        % Rasão de refluxo (caso for pedido)
        %   R=L/D
        % sobre o Feed:
        %   5 diferentes tipos: devo aprender a usar 3
        %   Cada um usamos p achar o "i" que é o declive de uma reta
        % Toda destilação tem 3 retas (diagram de eq {y}\times{x}):
        % Usamos as retas para calcular o numero de pratos p fazer a destilação
            % Reta de Enriching
            % Reta de Stripping
            % Reta de Feed
        % Sabemos q fiz corretamente a reta se todas se encontram no centro
    \begin{questionBox}2{ % MARK: Q2.1
        A razão mínima de refluxo.
        % Refluxo minimo
        % Traçamos x_D até encontrar a reta y=x, 
        % desse ponto traçamos uma reta até encontrar c o ponto q interceta a reta do feed comm a curva de equilíbrio e continuamos até encontrar a ordenada na origem q é:
        %  y_D = x_D/(R_{\min}+1)
        % R_{\min} é o refluxo minimo ou seja a razão entre o L e o D min para q ocorra a destilação
        % para isso precisamos da reta do feed calculada a partir do "i"
        % Para o i precisamos do estado físico da alimentação
        % Sabemos q é vapor super aquecido pela temperatura de feed ser muito maior da temperatura de ebulição (vapor sobreaquecida)
        % i = (\bar{L}-L)/F
        % \bar{L} = L - \miu
        % por ser vapor sobre aquecido sabemos q sai \miu
        % \miu é a quantidade de caldal perdida no feed
        % obtemos \miu a partir de um balanço de energia
        % Balanço energético:
            % F * C_{P,mis,L} * \adif{T} = \miu * \adif{H}_{vap}
            % F costuma cortar por isso pode deixar \miu em função de F
            % \adif{T} = T_F-T_{eb}
        % Aqui temos o i e a reta de Feed de onde usamos p os dois pontos da reta
        % y_i = ...
        % usamos a eq p traçar a reta no gráfico e pegamos dois pontos
        % O primeiro fazemos x_i = x_F que implica q y_i = x_F
        % O segundo ponto experimentamos valores até encontrar
        % Daqui seguimos as direções do começo
    } % Q2.1
        \answer{}
        \begin{itemize}
            \begin{multicols}{2}
                \item A é mais volátil
                \item \(x_F = \qty*{55}{\percent\of{A}}\)
                \item \(x_D = \qty*{95}{\percent\of{A}}\)
                \item \(x_B = \qty*{15}{\percent\of{A}}\)
            \end{multicols}
        \end{itemize}
        \begin{flalign*}
            &
                R_{\min}: &\\&
                y_{n+1}
                = \frac{R_{\min}}{R_{\min}+1}\,x_n
                + \frac{x_{D}}{R_{\min}+1}
                ;&\\[3ex]&
                y_{n+1} \text{ (Reta a partir de dois pontos)}:
                &\\[1.5ex]&
                % 
                % 
                % 
                \text{1º Ponto: Interseção \(y_i\) com a curva de equilibrio} &\\&
                y_i 
                = \frac{i}{i-1}\,x_i
                - \frac{x_F}{i-1}
                % 
                % 
                % 
                ; &\\[3ex]&
                % Tipos de feed:]
                % Liq saturado (no ponto de eb): i=1
                % vapor saturado (no ponto de cond): i=0
                % vapor sobre aquecido (i<0):
                i \text{ (Vapor Sobreaquecido)} &\\&
                i 
                = \frac{\bar{L}-L}{F}
                = \frac{(L-\nu)-L}{F}
                = \frac{-\nu}{F}
                % 
                % 
                % 
                ; &\\[3ex]&
                \nu\text{ (Balanço energético)} &\\&
                \nu\,\adif{\hat{H}}_{vap}
                = F\,C_{p,mist}\,\adif{T}
                \implies &\\&
                \implies
                \nu
                = \frac
                {F\,C_{p,mist}\,\adif{T}}
                {\adif{\hat{H}}_{vap}}
                = \frac
                {F\,67(110-94)}
                {40.2\E{3}}
                \cong
                F\,\num{26.666666666666667e-3}
                \implies &\\&
                \implies
                i
                = \frac{-\nu}{F}
                \cong -\num{26.666666666666667e-3}
                \implies &\\&
                \implies
                y_i
                = \frac{i}{i-1}\,x_i
                - \frac{x_F}{i-1}
                \cong 
                -\num{0.025974025974026}\,x_i
                +\num{0.535714285714286}
                &\\&
                \begin{cases}
                    x_i = x_F 
                    &\implies y_i = 0.55
                    \\
                    x_i = 0   
                    &\implies y_i = \num{0.535714285714286}
                \end{cases}
                &\\[3ex]&
                \text{2º Ponto:} &\\&
                y_D=x_D = 0.95
            &
        \end{flalign*}
        \begin{center}
            \includegraphics[width=.8\textwidth]{Graph.1.pdf}
        \end{center}
        \begin{flalign*}
            &
                \implies
                \frac{x_D}{R_{\min}+1}
                \cong 0.3139
                \implies
                R_{\min}
                \cong \frac{0.95}{0.3139}-1
                \cong \num{2.026441541892322}
            &
        \end{flalign*}
    \end{questionBox}
    \begin{questionBox}2{ % MARK: Q2.2
        Se a coluna operar a uma razão de refluxo \lemph{25\%} superior ao valor mínimo e com uma caldeira total
        e um condensador total, determine o número de andares de equilíbrio necessários. Indique o andar óptimo
        de entrada da alimentação e o número de andares em cada secção da coluna. Quais as aproximações que
        tomou?
    } % Q2.2
        \answer{}
        \begin{flalign*}
            &
                \text{Razão de refluxo:} 
                % &\\&
                R 
                = R_{\min}*1.25
                \cong \num{2.026441541892322}*1.25
                \cong \num{2.533051927365403}
                % 
                % 
                % 
                &\\[3ex]&
                \text{Reta de Enriching:}
                &\\&
                y_{n+1}
                = \frac{R\,x_n+x_D}{R+1}
                \cong
                = \num{0.71695858975226}\,x_n
                + \num{0.268889339735353}
                &\\[3ex]&
                \text{Interseção com o Feed}
                &\\&
                % -\num{0.025974025974026}\,x_i
                % +\num{0.535714285714286}
                - \num{0.025974025974026}\,x_n
                + \num{0.535714285714286}
                = \num{0.71695858975226}\,x_n
                + \num{0.268889339735353}
                \implies &\\&
                \implies
                x_n
                = \frac
                {\num{0.535714285714286}-\num{0.268889339735353}}
                {\num{0.71695858975226}+\num{0.025974025974026}}
                \cong
                \num{0.359150938228882}
                \implies &\\&
                \implies
                y_m 
                \cong 
                -\num{0.025974025974026}
                *\num{0.359150938228882}
                +\num{0.535714285714286}
                \cong \num{0.526385689916133}
                % 
                % 
                % 
                &\\[3ex]&
                \text{Reta de Stripping:}
                &\\&
                y_{m+1}
                = \frac{\bar{L}}{\bar{V}}\,x_m
                - \frac{B\,x_b}{\bar{V}}
                = \frac{\bar{V}+B}{\bar{V}}\,x_m
                - \frac{B\,x_b}{\bar{V}}
                &\\&
                \begin{cases}
                    x_m=x_B
                    &\implies
                    y_{m+1}=x_B\,\frac{\bar{V}+B-B}{\bar{V}}=x_B=0.15
                    \\
                    x_m\cong\num{0.359150938228882}
                    &\implies
                    y_{m+1}\cong \num{0.526385689916133}
                \end{cases}
            &
        \end{flalign*}
        \paragraph*{Grafico:} Traçar os pratos começando de \(x_D\) até o ponto de interseção onde vai ser o numero de pratos ótimo antes da entrada, então continuamos traçando até intersectar \(x=y\)
    \end{questionBox}
    \begin{questionBox}2{ % MARK: Q2.3
        \emph{NÃO} efectuando cálculos diga qual a relação entre o calor fornecido pela caldeira e o calor retirado pelo condensador. Justifique devidamente.
    } % Q2.3
    \end{questionBox}
    \begin{questionBox}2{ % MARK: Q2.4
        Comente a seguinte frase, justificando plenamente a sua resposta:\\\textit{``Quanto mais afastada estiver a linha de equilíbrio da linha diagonal \(x = y\) (gráfico McCabe-Thiele) mais fácil é a separação por destilação e menor o número de andares de equilíbrio necessários''}
    } % Q2.4
    \end{questionBox}
\end{questionBox}

\end{document}
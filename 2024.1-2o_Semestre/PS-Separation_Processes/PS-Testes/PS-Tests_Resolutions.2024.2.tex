% !TEX root = ./PS-Tests_Resolutions.2024.2.tex
\providecommand\mainfilename{"./PS-Tests_Resolutions.tex"}
\providecommand \subfilename{}
\renewcommand   \subfilename{"./PS-Tests_Resolutions.2024.2.tex"}
\documentclass[\mainfilename]{subfiles}

\tikzset{external/force remake=true} % - remake all

\begin{document}

% \graphicspath{{\subfix{./figures/PS-Tests_Resolutions.2024.2}}}
\tikzsetexternalprefix{./figures/PS-Tests_Resolutions.2024.2/graphics/}

\mymakesubfile{2}
[PS]
{Teste 2024.2 Resolução} % Subfile Title
{Teste 2024.2 Resolução} % Part Title

\begin{questionBox}1{ % MARK: Q1
    \begin{itemize}
        \begin{multicols}{2}
            \item B: Solvente
            \item C: Soluto
            \item A: Diluente
            \item Feed: \qty*{40}{\percent.p/p\of{C}}
        \end{multicols}
    \end{itemize}
} % Q1
\end{questionBox}

\begin{questionBox}2{ % MARK: Q1.1
    \begin{itemize}
        \begin{multicols}{2}
            \item \(F=\qty*{2500}{\kilo\gram/\hour}\)
            \item \(S=\qty*{500}{\kilo\gram/\hour}\)
        \end{multicols}
        \item \(x_{R}=?\)
        \item Composições dos caudais de saída e entrada
    \end{itemize}
} % Q1.1
    \answer{}
    \begin{center}
        % \tikzset{external/remake next=true}
        \begin{tikzpicture}[xscale=3.5]
            \PSAndarScheme{1}
        \end{tikzpicture}
    \end{center}
    \begin{flalign*}
        &
            S:
            \begin{cases}
                y_S=0
                \\ \qty{0}{\percent\of{A}}
                \\ \qty{100}{\percent\of{B}}
                \\ \qty{0}{\percent\of{C}}
            \end{cases}
            ; \qquad
            F:
            \begin{cases}
                x_F=0.40
                \\ \qty{40}{\percent\of{A}}
                \\ \qty{0}{\percent\of{B}}
                \\ \qty{60}{\percent\of{C}}
            \end{cases}
            % 
            % 
            % 
            ; &\\[3ex]&
            x_M: &\\&
            M\,x_M
            = (F+S)\,x_M
            = &\\&
            = F\,x_F+S\,y_S
            \implies &\\&
            \implies
            x_M
            = \frac{F\,x_F+S\,y_S}{F+S}
            = \frac{2500*0.40+500*0}{2500+500}
            \,\unit{\percent\of{C}}
            \cong \qty{33.333333333333333}{\percent\of{C}}
            &\\&
            M:\begin{cases}
                x_M=\num{.33333333333333333}
                \\ \qty[2]{49.166666666667}{\percent\of{A}}
                \\ \qty[2]{17.5}{\percent\of{B}}
                \\ \qty[2]{33.333333333333333}{\percent\of{C}}
            \end{cases}
            ;
            E:\begin{cases}
                y_E=\num{.415}
                \\ \qty[1]{2.5}{\percent\of{A}}
                \\ \qty[1]{56}{\percent\of{B}}
                \\ \qty[1]{41.5}{\percent\of{C}}
            \end{cases}
            ;
            R:\begin{cases}
                x_R=\num{.295}
                \\ \qty[1]{69.5}{\percent\of{A}}
                \\ \qty[1]{1}{\percent\of{B}}
                \\ \qty[1]{29.5}{\percent\of{C}}
            \end{cases}
            % 
            % 
            % 
            ; &\\[3ex]&
            E: &\\&
            E\,y_E
            + R\,x_R
            = E\,y_E
            + (M-E)\,x_R
            = M\,x_M
            \implies &\\&
            \implies
            E
            = M\,\frac{x_M-x_R}{y_E-x_R}
            \cong (2500+500)\,\frac{\num{.33333333333333333}-0.295}{0.415-0.295}
            \cong \qty{958.333333333333333}{\kilo\gram/\hour}
            ; &\\[3ex]&
            R 
            = M-E
            \cong 3000-\num{958.333333333333333}
            \cong \qty{2041.666666666666667}{\kilo\gram/\hour}
        &
    \end{flalign*}
\end{questionBox}

\begin{questionBox}2{ % MARK: Q1.2
    \begin{itemize}
        % \begin{multicols}{2}
            \item \(x_{R,2}=x_{R,1}/2\)
            \item \(S_2=?\)
        % \end{multicols}
    \end{itemize}
} % Q1.2
    \answer{}
    \begin{center}
        \tikzset{external/remake next=true}
        \begin{tikzpicture}[xscale=3.5]
            \PSAndarScheme{2}
        \end{tikzpicture}
    \end{center}
    \begin{flalign*}
        &
            x_{R,2}
            = 0.5\,x_{R,1}
            \cong
            \qty{14.75}{\percent\of{C}}
            &\\&
            R_2:\begin{cases}
                x_{R,2}=\num{0.1475}
                \\ \qty{84.35}{\percent\of{A}}
                \\ \qty{.9}{\percent\of{B}}
                \\ \qty{14.75}{\percent\of{C}}
            \end{cases}
            ; E_2:\begin{cases}
                y_{E,2}=\num{.230}
                \\ \qty{1.0}{\percent\of{A}}
                \\ \qty{76.0}{\percent\of{B}}
                \\ \qty{23.0}{\percent\of{C}}
            \end{cases}
            ; M_2:\begin{cases}
                x_{M,2}=\num{.185}
                \\ \qty{45.0}{\percent\of{A}}
                \\ \qty{36.5}{\percent\of{B}}
                \\ \qty{18.5}{\percent\of{C}}
            \end{cases}
            % 
            % 
            % 
            &\\[3ex]&
            S_2: &\\&
            S_2\,y_{S,2}+R_1\,x_{R,1}
            = &\\&
            = M_2\,x_{M,2}
            = (S_2+R_1)\,x_{M,2}
            \implies &\\&
            \implies
            S_2
            = R_1
            \,\frac
            {x_{M,2}-x_{R,1}}
            {y_{S,2}-x_{M,2}}
            \cong \num{2041.666666666666667}
            \,\frac
            {.23-0.295}
            {0-0.23}
            \cong
            \qty{576.9927536231885}{\kilo\gram/\hour}
        &
    \end{flalign*}
\end{questionBox}

\begin{questionBox}2{ % MARK: Q1.3
    comparar diagramas de eq dos solventes D e E para escolher o melhor
} % Q1.3
    \answer{}
    Os solventes D e E possuem pontos fortes diferentes, o solvente D é capaz de fazer extrações de diversas composições de misturas diferenetes entre A e C, enquanto o E so seria capaz de extrair concentrações ja mais concentradas em A, porem faria essa extração com menos andares.\\

    Tendo em consideração a composição 40\%p/p, o solvente ``E'' parece não ser capaz de separar a mistura, dessa forma fica a escolha o solvente D como o mais adequado.
\end{questionBox}

\end{document}
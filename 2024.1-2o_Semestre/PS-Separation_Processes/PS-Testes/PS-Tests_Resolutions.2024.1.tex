% !TEX root = ./PS-Tests_Resolutions.2024.1.tex
\providecommand\mainfilename{"./PS-Tests_Resolutions.tex"}
\providecommand \subfilename{}
\renewcommand   \subfilename{"./PS-Tests_Resolutions.2024.1.tex"}
\documentclass[\mainfilename]{subfiles}

% \tikzset{external/force remake=true} % - remake all

\begin{document}

% \graphicspath{{\subfix{./.build/figures/PS-Tests_Resolutions.2024.1}}}
% \tikzsetexternalprefix{./.build/figures/PS-Tests_Resolutions.2024.1/graphics/}

\mymakesubfile{1}
[PS]
{Teste 2024.1} % Subfile Title
{Teste 2024.1} % Part Title

\begin{questionBox}1{ % MARK: Q1
    \begin{itemize}
        \begin{multicols}{2}
            \item Prentende abs \(\qty*{90}{\percent\of{\ch{CO2}}}\)
            \item Conc inicial \(\qty*{1}{\percent\of{\ch{CO2}}}\)
            \item \(P=\qty*{10}{\atm}\)
            \item \(G=1.5\,G_{\min}\)
            \item \(H=\qty*{900}{\atm}\)
        \end{multicols}
    \end{itemize}
} % Q1
    \begin{questionBox}2{ % MARK: Q1.1
        Caldal mínimo da agua
    } % Q1.1
        \answer{}
        \begin{itemize}
            \item \(y_B = 1\%\)
            \item \(y_A = 1\%*(1-90\%) = 0.1\%\)
        \end{itemize}
        \begin{flalign*}
            &
                L_{\min}: &\\&
                G_B\,y_B + L_A\,x_a
                = G_B\,y_B
                = &\\&
                = G_{A}\,y_{A} + L_{B\,\min}\,x_B^*
                ; &\\[3ex]&
                \text{Verificar se o caldal é constante:}
                &\\&
                E = \myvert{1-G_s/G_B};
                &\\&
                G_B\,y_B
                = G_s\,\frac{y_B}{1-y_B}
                \implies &\\&
                \implies
                E
                = \myvert{1-\frac{G_s}{G_B}}
                = \myvert{1-\frac{y_B(1-y_B)}{y_B}}
                = y_B
                = 1\% < 10\%
                &\\&
                \therefore
                \begin{cases}
                    G_A=G_B=G=&
                    \\
                    L_A=L_B=L=& 1.5\,L_{\min}
                \end{cases}
                &\\[3ex]&
                % y_B^*
                % = \frac{p_{B,\ch{CO2}}}{p_T}
                % = \frac{H}{p_T}\,x_B
                % = \frac{900}{10}\,1\%
                % \cong 
                % 0.9
                &\\[3ex]&
                L_{\min}
                = \frac{G\,(y_B-y_A)}{1.5\,x_B}
                = \frac{G\,(y_B-y_A)}{1.5\,y_B/H}
                = \frac{G\,(1-y_A/y_B)\,H}{1.5}
                % Assumimos
                &\\[3ex]&
                \text{Assumimos caldal mínimo: } \qty*{40}{\kilo\mole/\hour}
            &
        \end{flalign*}
    \end{questionBox}
    \begin{questionBox}2{ % MARK: Q1.2
        A \% molar de \ch{CO2} na corrente líquida à saída da coluna
    } % Q1.2
        \answer{}
        \begin{flalign*}
            &
                x_B: &\\&
                G_B\,y_B 
                = G\,y_B 
                = &\\&
                = G_A\,y_A + L_B\,x_B
                = G\,y_A + L\,x_B
                = G\,y_A + L_{\min}*1.5\,x_B
                \implies &\\&
                \implies
                x_B
                = G\,(y_B - y_A)/L
                \cong G\,(1\% - 0.1\%)/40*1.5
                \cong &\\&
                \cong G\,1.5\E{-4}
            &
        \end{flalign*}
    \end{questionBox}
    \begin{questionBox}2{ % MARK: Q1.3
        A força motriz na base e no topo da coluna. Comente
    } % Q1.3
    \end{questionBox}
    \begin{questionBox}2{ % MARK: Q1.4
        Numero de unidades de transf
    } % Q1.4
        \answer{}
        \begin{flalign*}
            &
                N_{OG}: &\\&
                N_{OG}
                = \int_{y_A}^{y_B}{\odif{y}/(y-y^*)}
                \cong 
                \frac{y_B-y_A}{\adif{\bar{y_L}}}
                = &\\&
                = \frac{
                    y_B - y_A
                }{
                    \left(
                        \frac
                        {\adif{y_B}-\adif{y_A}}
                        {\ln{\adif{y_B}/\adif{y_A}}}
                    \right)
                }
                = &\\&
                = \frac{
                    y_B - y_A
                }{
                    \left(
                        \frac
                        {
                            (y_B-y_B^*)
                            -(y_A-y_A^*)
                        }
                        {\ln{
                            \frac
                            {y_B-y_B^*}
                            {y_A-y_A^*}
                        }}
                    \right)
                }
            &
        \end{flalign*}
    \end{questionBox}
    \begin{questionBox}2{ % MARK: Q1.5
        Discuta o efeito de usar uma pressão inferior, na altura de enchiemnto necessáia para esta separação
    } % Q1.5
    \end{questionBox}
\end{questionBox}

\begin{questionBox}1{ % MARK: Q2
    \begin{itemize}
        \begin{multicols}{2}
            \item \(\qty*{45}{\percent.\mole\of{A}}\)
            \item \(\qty*{93}{\percent.\mole}\)
        \end{multicols}
    \end{itemize}
} % Q2
    \paragraph*{Dados:}
    \begin{itemize}
        \begin{multicols}{2}
            \item Temp de corrente de alimentação: \qty*{110}{\celsius}
            \item \(C_{p,mist}=\qty*{67}{\joule/\mole.\celsius}\)
            \item \(\adif{\hat{H}}_{vap,mist}=\qty*{40.2}{\kilo\joule/\mole}\)
        \end{multicols}
        \item Temp de vap \qty*{1}{\bar}:
        \begin{itemize}
            \begin{multicols}{2}
                \item A puro:      \qty*{ 82}{\celsius}
                \item C puro:      \qty*{115}{\celsius}
                \item Alimentação: \qty*{100}{\celsius}
                \item Destilado:   \qty*{ 84}{\celsius}
                \item Residuo:     \qty*{110}{\celsius}
            \end{multicols}
        \end{itemize}
    \end{itemize}
    \begin{questionBox}2{ % MARK: Q2.1
        Razão mínima de refluxo
    } % Q2.1
        \answer{}
        \begin{itemize}
            \item A é mais volátil
            \begin{multicols}{3}
                \item \(x_F=\qty*{45}{\percent.\mole\of{A}}\)
                \item \(x_D=\qty*{93}{\percent.\mole\of{A}}\)
                \item \(x_B=\qty*{15}{\percent.\mole\of{A}}\)
            \end{multicols}
        \end{itemize}
        \begin{flalign*}
            &
                R_{\min}: &\\&
                y_{n+1}
                = \frac{R_{\min}}{R_{\min}+1}\,x_n
                + \frac{x_{D}}{R_{\min}+1}
                ;&\\[3ex]&
                y_{n+1} \text{ (Reta a partir de dois pontos)}:
                &\\[1.5ex]&
                % 
                % 
                % 
                \text{1º Ponto: Interseção \(y_i\) com a curva de equilibrio} &\\&
                y_i 
                = \frac{i}{i-1}\,x_i
                - \frac{x_F}{i-1}
                % 
                % 
                % 
                ; &\\[3ex]&
                i \text{ (Vapor Sobreaquecido)} &\\&
                i 
                = \frac{\bar{L}-L}{F}
                = \frac{(L-\nu)-L}{F}
                = \frac{-\nu}{F}
                % 
                % 
                % 
                ; &\\[3ex]&
                \text{Balanço Mássico}
                &\\&
                \nu\,\adif{\hat{H}}_{vap}
                = F\,C_{p,mist}\,\adif{T}
                \implies &\\&
                \implies
                \nu
                = \frac
                {F\,C_{p,mist}\,\adif{T}}
                {\adif{\hat{H}}_{vap}}
                = \frac
                {F\,67\,(110-100)}
                {40.2\E{3}}
                \cong
                F\,\num{16.666666666666667e-3}
                \implies &\\&
                \implies
                i
                = \frac{-\nu}{F}
                = -\num{16.666666666666667e-3}
                \implies &\\&
                \implies
                y_i
                = \frac{i}{i-1}\,x_i
                - \frac{x_F}{i-1}
                \cong 
                = \num{-0.016393442622951}\,x_i
                + \num{0.442622950819672}
                &\\&
                \begin{cases}
                    x_i = x_F 
                    &\implies y_i = x_F\frac{i-1}{i-1} = x_F 
                    = 0.45
                    \\
                    x_i = 0   
                    &\implies y_i = \num{0.442622950819672}
                    4.25
                \end{cases}
                &\\[3ex]&
                \text{2º Ponto:} &\\&
                y_D=x_D = 0.93
                &\\[6ex]&
                \implies
                \frac{x_D}{R_{\min}+1}
                \cong 0.280
                \implies
                R_{\min}
                \cong \frac{0.93}{0.280}-1
                \cong \num{2.321428571428571}
            &
        \end{flalign*}
    \end{questionBox}
    \begin{questionBox}2{ % MARK: Q2.2
        \begin{itemize}
            \begin{multicols}{2}
                \item \(R=1.2\,R_{\min}\)
            \end{multicols}
            \item Det numero de andares e o andar otimo de entrada
        \end{itemize}
    } % Q2.2
        \answer{}
        \begin{flalign*}
            &
                \text{Razão de refluxo:} 
                % &\\&
                R 
                = R_{\min}*1.25
                \cong \num{2.321428571428571}*1.2
                \cong \num{2.785714285714285}
                % 
                % 
                % 
                &\\[3ex]&
                \text{Reta de Enriching:}
                &\\&
                y_{n+1}
                = \frac{x_n}{1+1/R}
                + \frac{x_D}{R+1}
                \cong
                \num{0.735849056603774}\,x_n
                + \num{0.245660377358491}
                &\\[3ex]&
                \text{Interseção com o Feed}
                &\\&
                \num{-0.016393442622951}\,x_n
                + \num{0.442622950819672}
                = \num{0.735849056603774}\,x_n
                + \num{0.245660377358491}
                \implies &\\&
                \implies
                x_n
                \cong \frac{
                    \num{0.442622950819672}
                    -\num{0.245660377358491}
                }{
                    \num{0.735849056603774}
                    +\num{0.016393442622951}
                }
                \cong
                \num{0.261833881578946}
                \implies &\\&
                \implies
                y_n
                \cong
                  \num{0.735849056603774}
                * \num{0.261833881578946}
                + \num{0.245660377358491}
                \cong
                \num{0.438330592105263}
                % 
                % 
                % 
                &\\[3ex]&
                \text{Reta de Stripping:}
                &\\&
                y_{m+1}
                = \frac{\bar{L}}{\bar{V}}\,x_m
                - \frac{B\,x_b}{\bar{V}}
                = \frac{\bar{V}+B}{\bar{V}}\,x_m
                - \frac{B\,x_b}{\bar{V}}
                &\\&
                \begin{cases}
                    x_m=x_B
                    &\implies
                    y_{m+1}=x_B\,\frac{\bar{V}+B-B}{\bar{V}}=x_B=0.15
                    \\
                    x_m\cong\num{0.261833881578946}
                    &\implies
                    y_{m+1}\cong \num{0.438330592105263}
                \end{cases}
                &\\[6ex]&
                \begin{cases}
                    \text{Pratos Totais: } 10
                    \\
                    \text{Posição ótima de entrada: } 7
                \end{cases}
            &
        \end{flalign*}
    \end{questionBox}
    \begin{questionBox}2{ % MARK: Q2.3
        Seria possivel cumprir o objetivo apenas com 4 andares?
    } % Q2.3
        \answer{}
        Com um aumento significativo da razão de refluxo, é possivel traçar pratos maiores e consequentemente menos pratos, porem por maior que seja, pela consentração de saída ser tão alta, um numero mínimo de pratos deveria se limitar por baixo a 5 para a curva de equilibrio atual. no caso de mudarmos a curva abre possibilidade de reduzir ainda mais o numero de pratos, inclusive para menos de 4.
    \end{questionBox}
    \begin{questionBox}2{ % MARK: Q2.4
        Comentar a frase
    } % Q2.4
        \answer{}
        Com uma curva de equilíbrio mais larga temos mais espaço por prato, que seria a tal facilidade da separação e consequentemente necessitariamos de menos pratos
    \end{questionBox}
\end{questionBox}

\end{document}  
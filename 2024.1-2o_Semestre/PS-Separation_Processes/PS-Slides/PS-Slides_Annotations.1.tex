% !TEX root = ./PS-Slides_Annotations.1.tex
\providecommand\mainfilename{"./PS-Slides_Annotations.tex"}
\providecommand \subfilename{}
\renewcommand   \subfilename{"./PS-Slides_Annotations.1.tex"}
\documentclass[\mainfilename]{subfiles}

% \tikzset{external/force remake=true} % - remake all

\begin{document}

\graphicspath{{\subfix{./.build/figures/PS-Slides_Annotations.1}}}
% \tikzsetexternalprefix{./.build/figures/PS-Slides_Annotations.1/graphics/}

\mymakesubfile{1}
[PS]
{Gas Absorptium} % Subfile Title
{Gas Absorptium} % Part Title

\begin{sectionBox}1{Absorptium} % S
    
    \begin{center}\Large
        \chemsetup[phases]{pos=side}
        \ch{\gas <>[Absorptium][Desabsorptium] \lqd}
    \end{center}

    \begin{itemize}
        \item Mass transference between the gas phase and the liquid phase
        \item \emph{\(\vv{N_A}\):} Vector that points in the direction of the absorptium
        \item \emph{\textit{Stripping}:} Reverse of absorptium
    \end{itemize}

    \paragraph*{Tipos de absorção:}
    \begin{description}[
        leftmargin=!,
        labelwidth=\widthof{} % Longest item
    ]
        \item[Absorção Física:] 
        Gases muito solúveis;
        \ch{NH3 + Water}
        \item[Absorção Química:] 
        Gases pouco solúveis;
        \ch{SO2 + Water}
    \end{description}
    
\end{sectionBox}

\begin{sectionBox}2{\emph{\textit{r}:} Rate of absorptium} % MARK: S1
    
    \begin{center}
        \begin{multicols}{2}
            \includegraphics[width=.3\textwidth]{Screenshot 2024-04-11 at 16.32.00-cutout.png}
            \\
            \begin{BM}
                \begin{cases}
                       r = k_y&\hspace{-1ex} a\,(y-y_i)
                    \\ r = k_x&\hspace{-1ex} a\,(x_i-x)
                    \\ r = K_y&\hspace{-1ex} a\,(y-y^*)
                    \\ r = K_x&\hspace{-1ex} a\,(x^*-y)
                \end{cases}
            \end{BM}
        \end{multicols}
    \end{center}

    \paragraph*{\(x,y\):} Mole fraction of component being absorbed
    
\end{sectionBox}
\begin{sectionBox}2{\emph{\(N_A\):} Flux of absorptium} % MARK: S
    
    \begin{center}
        \begin{multicols}{2}
            \includegraphics[width=.45\textwidth]{Screenshot 2024-04-11 at 17.12.41-cutout.png}
            \\
            \begin{BM}[align*]
                &
                \begin{cases}
                       N_A = k_g&\hspace{-1ex} a\,(p_{A,g}-p_{A,i})
                    \\ N_A = k_l&\hspace{-1ex} a\,(c_{A,i}-c_{A,l})
                    \\ N_A = K_g&\hspace{-1ex} a\,(p_{A,g}-p_{A}^*)
                    \\ N_A = K_l&\hspace{-1ex} a\,(c_{A}^*-c_{A,l})
                \end{cases}
                \\&
                \begin{aligned}
                    & k: & \text{Solo}
                    \\ 
                    & K: & \text{Global}
                \end{aligned}
            \end{BM}
        \end{multicols}
    \end{center}

    \begin{BM}
        \frac{1}{K_{y}\,a}
        = 
        \frac{1}{k_{y}\,a}
        + \frac{m}{k_{x}\,a}
        ; \qquad
        \frac{1}{K_{x}\,a}
        = 
        \frac{1}{k_{x}\,a}
        + \frac{1}{m\,k_{y}\,a};
        \\
        m: \text{Local equilibrium slope}
    \end{BM}
    \begin{description}[
        leftmargin=!,
        labelwidth=\widthof{\(1/m\,k_x\,a\):} % Longest item
    ]
        \item[\(1/k_y\,a\):] Resistance of mass transfer in the \emph{gas} film 
        \item[\(1/m\,k_x\,a\):] Resistance of mass transfer in the \emph{liquid} film 
    \end{description}
\end{sectionBox}

\begin{sectionBox}2{Henry's Law} % MARK: S
    
    \begin{center}
        \includegraphics[width=.6\textwidth]{image-026.png}
    \end{center}
    \begin{BM}
           y_A=m\,x_{A}
           ; \quad
           m = p_{A}^*/p_t
           \\
           y_a = m\,x_a
           \begin{cases}
                y_A^*=m\,x_{A}
                \\ y_A=m\,x_{A}^*
                \\ p_A = H\,x_A
           \end{cases}
    \end{BM}
    \begin{itemize}
        \item From the given equation \(y_a = m\,x_a\) we derive the equations with \(y_A^*\)
    \end{itemize}
    
\end{sectionBox}

\end{document}
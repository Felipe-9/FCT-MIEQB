% !TEX root = ./PS-Report.3.tex
\providecommand\mainfilename{"./PS-Report.tex"}
\providecommand \subfilename{}
\renewcommand   \subfilename{"./PS-Report.3.tex"}
\documentclass[\mainfilename]{subfiles}

% \tikzset{external/force remake=true} % - remake all

\begin{document}

% \graphicspath{{\subfix{./figures/PS-Report.3}}}
\tikzsetexternalprefix{./figures/PS-Report.3/graphics/}

\mymakesubfile{2}
[PS]
{Simulações} % Subfile Title
{Simulações} % Part Title

\part*{Coluna B1}

\begin{sectionBox}1{NSTAGE} % MARK: S
    
    \emph{Se variou manualmente no Aspen a coluna B1 com diferentes numeros de pratos}, a sequencia de uma forma lógica, começamos com 32, e fomos dividindo por 2 até encontrar valores mnores que o pedido, nesse caso foi em 8 pratos, então subiu de 2 em 2 até 16, isso foi feito para poder construir a curva onde se pode ver o vale com menores erros.

    \begin{center}
        \vspace{1ex}
        % \setlength\tabcolsep{3mm}        % width
        % \renewcommand\arraystretch{1.25} % height
        \begin{tabular}{*2{C} *{3}{R}}
            \toprule
            
                   \multicolumn{1}{c}{\multirow{2}{*}{Simulação}}
                &  \multicolumn{1}{c}{\multirow{2}{*}{NSTAGE}}
                &  \multicolumn{2}{l}{Reculperação}
                &  \multicolumn{1}{c}{\multirow{2}{*}{Error(.95)}}
            \\\cline{3-4} &&
                \multicolumn{1}{c}{Tolueno}
                & \multicolumn{1}{c}{Cumeno}
            
            \\\midrule
            
                    1 & 32 & 100.00\,\% & 100.00\,\% & \num{1.19242808740467E-01}
                \\  2 & 16 &  99.26\,\% &  99.26\,\% & \num{1.55365394982667E-02}
                \\  3 &  8 &  89.39\,\% &  89.29\,\% & \num{1.69839224114106E-02}
                \\  4 & 10 &  94.27\,\% &  94.25\,\% & \num{7.36159871240984E-02}
                \\  5 & 12 &  95.82\,\% &  95.79\,\% & \num{8.97848545538178E-02}
                \\  6 & 14 &  98.50\,\% &  98.50\,\% & \num{1.05224852166814E-01}
            
            \\\bottomrule
        \end{tabular}
        \\[1ex]\tablecaption{Variação do numero de pratos da coluna B1}
        \vspace{2ex}
    \end{center}

    % \tikzset{external/remake next=true}
    % {\Large\bfseries{Solução 1}}\par\medskip
    \begin{figure}\centering
        
        \begin{tikzpicture}
        \begin{axis}
        [
            xmajorgrids=true,
            xtick=data,
            ymajorgrids=true,
            minor tick num=3,
            xminorgrids=true,
            yminorgrids=true,
            % xmin={7.5},xmax={32.5},
            % scaled ticks=true,
            scaled y ticks={base 10:2},
            % legend pos={north west},
            % axis on top,
            ylabel={Error (.95)},
            xlabel={NSTAGE},
        ]
            
            % % Legends
            % \addlegendimage{empty legend}
            % \addlegendentry[Graph]{}
            % ================= Interpolation ================ %
            % \addplot[
            %     no marks,
            %     dashed,
            %     samples=2,
            % ] expression [
            %     domain=0:4000
            % ] {
            %     -3.5953e-4*x+0.0013
            % };
            % ==================== Points ==================== %
            \addplot+[
                mark=*,
                draw={Graph},
                mark options={
                    fill={Graph},
                    fill opacity=1,
                    draw={Graph},
                },
                smooth, 
                tension=0.3,
                % curved line,
            ] coordinates {
                ( 8,1.19242808740467E-01)
                (10,1.55365394982667E-02)
                (12,1.69839224114106E-02)
                (14,7.36159871240984E-02)
                (16,8.97848545538178E-02)
                (32,1.05224852166814E-01)
            };
            
        \end{axis}
        \end{tikzpicture}
        \caption{Variação do numero de pratos da coluna B1}
    \end{figure}
    

    Foram \emph{selecionados de \numrange*{8}{12} (incluindo valores impares)} como o numero de pratos para continuar o estudo, valores menores foram selecionados para considerar quantidades menores de pratos.\\

    Algo a se notar é que os valores escolhidos são altamente dependentes do \emph{numero de refluxo} inicial ultilizado, perchance teriamos usado 1 ou 5 como numero de refluxo, teriamos visto o vale orbitando ao redor de \numrange*{14}{16} ou \numrange*{4}{5} respectivamente, como decidimos \emph{2 como um valor proximo do esperado}, continuamos as simulações de acordo.
    
\end{sectionBox}

\end{document}
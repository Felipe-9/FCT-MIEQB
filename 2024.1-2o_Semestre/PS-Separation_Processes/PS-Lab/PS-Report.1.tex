% !TEX root = ./PS-Report.1.tex
\providecommand\mainfilename{"./PS-Report.tex"}
\providecommand \subfilename{}
\renewcommand   \subfilename{"./PS-Report.1.tex"}
\documentclass[\mainfilename]{subfiles}

% \tikzset{external/force remake=true} % - remake all

\begin{document}

\graphicspath{{\subfix{./figures/PS-Report.1}}}
% \tikzsetexternalprefix{./figures/PS-Report.1/graphics/}

% \mymakesubfile{1}
% [PS]
% {Introdução} % Subfile Title
% {Introdução} % Part Title

\begin{sectionBox}1{Introdução} % MARK: S
    
    Pretende-se planejar um processo de destilação adaptado ao fluxo de uma mistura de Benzeno (\ch{C6H6}), Tolueno (\ch{C7H8}) e Cumeno (\ch{C9H12}) em duas etapas.

    % \paragraph*{Flowsheet}
    \begin{figure}\centering
        \includegraphics[width=.8\textwidth]{flowsheet.png}
        \caption{Flowsheet}
    \end{figure}

    \paragraph*{Composição do Feed (F)}
    \begin{center}
        \vspace{1ex}
        \begin{tabular}{l C}
            \toprule
            
                \multicolumn{1}{c}{Composto}
                & \multicolumn{1}{c}{Fração molar}
            
            \\\midrule
            
                Benzeno & 0.4
                \\ Tolueno & 0.3
                \\ Cumeno & 0.3
            
            \\\bottomrule
        \end{tabular}
        \\[1ex]\tablecaption{Composição do Feed (F) \\em fração molar de cada composto}
        \vspace{2ex}
    \end{center}

    \paragraph*{Objetivos}
    \begin{center}
        \vspace{1ex}
        \begin{tabular}{l c C}
            \toprule
            
                \multicolumn{1}{c}{Composto}
                & \multicolumn{1}{c}{Stream}
                & \multicolumn{1}{c}{Reculperação}
            
            \\\midrule
            
                    Cumeno  & R & 95\%
                \\  Tolueno & D & 95\%
                \\  Tolueno & R2 & 99\%
                \\  Benzeno & D2 & 99\%
            
            \\\bottomrule
        \end{tabular}
        \\[1ex]\tablecaption{Objetivos de reculperação\\ de cada composto nos streams}
        \vspace{2ex}
    \end{center}

    \begin{description}[
        leftmargin=!,
        labelwidth=\widthof{} % Longest item
    ]
        \item[Reculperação] Mede a fração de fluxo molar de saída comparado com entrada da coluna, para D e R é F, para D2 e R2 é D
    \end{description} 

    Os modelos devem \emph{minimizar os gastos de construção e manutenção}, onde o primeiro esta relacionado com o \emph{numero de colunas} e o segundo a energia gasta para manter a coluna funcionando, que por si se relaciona com o \emph{numero de refluxo} da coluna.

    \paragraph*{Modelos encontrados}
    \begin{center}
        \vspace{1ex}
        \setlength\tabcolsep{1.5mm}        % width
        % \renewcommand\arraystretch{1.25} % height
        \begin{tabular}{*{4}{C} | *{4}{C}}
            \toprule
            
                    \multicolumn{4}{l|}{B1}
                &   \multicolumn{4}{l}{B2}
                
                \\  \multicolumn{1}{c}{Modelo}
                &   \multicolumn{1}{c}{NSTAGES}
                &   \multicolumn{1}{c}{FSTAGE}
                &   \multicolumn{1}{c|}{RR}
                &   \multicolumn{1}{c}{Modelo}
                &   \multicolumn{1}{c}{NSTAGES}
                &   \multicolumn{1}{c}{FSTAGE}
                &   \multicolumn{1}{c}{RR}
            
            \\\midrule
            
                    1 &  8 & 6 & 3.299 & 1 & 17 &  8 & 2.623
                \\  2 &  9 & 7 & 2.069 & 2 & 18 &  9 & 2.251
                \\  3 & 10 & 7 & 1.634 & 3 & 19 &  9 & 2.019
                \\  4 & 11 & 8 & 1.432 & 4 & 20 & 10 & 1.833
                \\  5 & 12 & 8 & 1.317 &  
                
            \\\bottomrule
        \end{tabular}
        \\[1ex]\tablecaption{Melhores modelos para resolução do problema}
        \vspace{2ex}
    \end{center}

    \emph{Qualquer combinação dos modelos B1 e B2 é uma resposta ao problema}, mais a frente será estudado as combinações (1,1),(1,4),(5,1) e (5,4) para prever as caracteristicas energéticas de todas as combinações possíveis.

    \begin{description}[
        leftmargin=!,
        labelwidth=\widthof{NSTAGES} % Longest item
    ]
        \item[NSTAGES] Numero de Pratos
        \item[FSTAGE] Prato de entrada
        \item[RR] Numero de Refluxo
    \end{description}
    
\end{sectionBox}

\end{document}
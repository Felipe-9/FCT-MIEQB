% !TEX root = ./PS-Report.7.tex
\providecommand\mainfilename{"./PS-Report.tex"}
\providecommand \subfilename{}
\renewcommand   \subfilename{"./PS-Report.7.tex"}
\documentclass[\mainfilename]{subfiles}

% \tikzset{external/force remake=true} % - remake all

\begin{document}

% \graphicspath{{\subfix{./figures/PS-Report.7}}}
\tikzsetexternalprefix{./figures/PS-Report.7/graphics/}

% \mymakesubfile{7}
% [PS]
% {FSTAGE-2} % Subfile Title
% {FSTAGE-2} % Part Title

\part*{Coluna B2}

\begin{sectionBox}1{NSTAGE-B2} % MARK: S
    
    Da mesma forma que a primeira coluna se variou manualmente o numero de pratos da coluna B2 com o numero de refluxo fixo em 2.

    \begin{center}
        \vspace{1ex}
        % \setlength\tabcolsep{3mm} % width
        \begin{tabular}{*{5}{C}}
            \toprule
            
            \multicolumn{1}{c}{\multirow{2}{*}{Simulação}}
                &  \multicolumn{1}{c}{\multirow{2}{*}{NSTAGE}}
                &  \multicolumn{2}{l}{Reculperação}
                &  \multicolumn{1}{c}{\multirow{2}{*}{Error(.99)}}
            \\\cline{3-4} &&
                \multicolumn{1}{c}{Tolueno}
                & \multicolumn{1}{c}{Cumeno}
            
            \\\midrule
            
                1 & 32 & \qty{99.8121004901713}{\percent} & \qty{99.9931165760466}{\percent} & \num{1.82345158203827E-02}
            \\  2 & 16 & \qty{97.9500619516825}{\percent} & \qty{98.6623209697227}{\percent} & \num{1.40163341272198E-02}
            \\  3 & 20 & \qty{99.2006215148576}{\percent} & \qty{99.5561100541375}{\percent} & \num{7.64375322217292E-03}
            \\  4 & 18 & \qty{98.7194998868427}{\percent} & \qty{99.2122579545395}{\percent} & \num{4.97735421915879E-03}
            
            \\\bottomrule
        \end{tabular}
        \\[1ex]\tablecaption{Variação do numero de pratos da coluna B2}
        \vspace{2ex}
    \end{center}

    % \tikzset{external/remake next=true}
    % {\Large\bfseries{Solução 1}}\par\medskip
    \begin{figure}\centering
        
        \begin{tikzpicture}
        \begin{axis}
        [
            xmajorgrids=true,
            xtick=data,
            ymajorgrids=true,
            minor tick num=3,
            xminorgrids=true,
            yminorgrids=true,
            % xmin={7.5},xmax={32.5},
            % scaled ticks=true,
            scaled y ticks={base 10:3},
            % legend pos={north west},
            % axis on top,
            ylabel={Error (.99)},
            xlabel={NSTAGE},
        ]
            % ==================== Points ==================== %
            \addplot+[
                mark=*,
                draw={Graph},
                mark options={
                    fill={Graph},
                    fill opacity=1,
                    draw={Graph},
                },
                smooth, 
                tension=0.25,
                % curved line,
            ] coordinates {
                (16,1.40163341272198E-02)
                (18,4.97735421915879E-03)
                (20,7.64375322217292E-03)
                (32,1.82345158203827E-02)
            };
            
        \end{axis}
        \end{tikzpicture}
        \caption{Variação do numero de pratos da coluna B2}
    \end{figure}

    Foram \emph{selecionados de \numrange*{17}{20}} como o numero de pratos a serem estudados para a coluna B2
    
\end{sectionBox}


\end{document}
% !TEX root = ./ERQ_II-Tests_Resolutions.2024.1.tex
\providecommand\mainfilename{"./ERQ_II-Tests_Resolutions.tex"}
\providecommand \subfilename{}
\renewcommand   \subfilename{"./ERQ_II-Tests_Resolutions.2024.1.tex"}
\documentclass[\mainfilename]{subfiles}

% \tikzset{external/force remake=true} % - remake all

\begin{document}

% \graphicspath{{\subfix{./.build/figures/ERQ_II-Tests_Resolutions.2024.1}}}
% \tikzsetexternalprefix{./.build/figures/ERQ_II-Tests_Resolutions.2024.1/graphics/}

\mymakesubfile{1}
[ERQ\,II]
{Teste 2024.1 Resolução} % Subfile Title
{Teste 2024.1 Resolução} % Part Title

\begin{questionBox}1{ % MARK: Q1
    \begin{itemize}
        \item Reator CSTR com escoamento modelado em 2 CSTR iguais em serie
        \item Tempo de residencia max: \qty*{10}{\min}
    \end{itemize}
} % Q1
\end{questionBox}

\begin{questionBox}2{ % MARK: Q1.1
    \begin{itemize}
        \item Esquema de associação de reatores
        \item Equações
        \item Caldais, volumes, concentrações
    \end{itemize}
} % Q1.1
    \answer{}
    \begin{center}\Large\bfseries
        \ch{-> R1 -> R2 ->}
    \end{center}
    \begin{center}
        \vspace{1ex}
        \begin{tabular}{*{4}{C}}
            \toprule
            
                & \multicolumn{1}{c}{Volume}
                & \multicolumn{1}{c}{Caldal in}
                & \multicolumn{1}{c}{Caldal out}
                % & \multicolumn{1}{c}{C}
            
            \\\midrule
            
                1
                & V/2
                & \nu
                & \nu
                \\ 
                2
                & V/2
                & \nu
                & \nu
            
            \\\bottomrule
        \end{tabular}
        \vspace{2ex}
    \end{center}
        \begin{flalign*}
            &
                % 
                % R_1
                % 
                \mathemph{R_1:}
                &\\&
                \nu\,C_{in,1}
                = \nu\,C_{in}
                = \nu\,C_{out,1}
                + \frac{V}{2}
                \,\odv{C_{out,1}}{t}
                % 
                % R_2
                % 
                ; &\\[3ex]&
                \mathemph{R_2:}
                &\\&
                \nu\,C_{in,2}
                = \nu\,C_{out,1}
                = \nu\,C_{out,2}
                + \frac{V}{2}
                \,\odv{C_{out,2}}{t}
                = \nu\,C_{out}
                + \frac{V}{2}
                \,\odv{C_{out}}{t}
            &
        \end{flalign*}
\end{questionBox}

\begin{questionBox}2{ % MARK: Q1.2
    Tempo de residencia
} % Q1.2
    \answer{}
    \begin{flalign*}
        &
            E(t) = \LagrangeTransform{g}(s)
            % 
            % g(s)
            % 
            ; &\\[3ex]&
            \mathemph{g(s):}
            g(s)
            =\bar{C}_{out}/\bar{C}_{in}
            % 
            % c out
            % 
            ; &\\[3ex]&
            \mathemph{\bar{C}_{out}/\bar{C}_{in}}
            &\\&
            \nu\,C_{out,1}
            = \nu\,C_{out}
            + \frac{V}{2}
            \,\odv{C_{out}}{t}
            \implies &\\&
            \implies
            C_{out,1}
            = C_{out}
            + \frac{\tau}{2}
            \,\odv{C_{out}}{t}
            \implies &\\&
            \implies
            \LagrangeTransform{C}_{out,1}
            = \bar{C}_{out,1}
            = &\\&
            = \LagrangeTransform{C}_{out}
            + \LagrangeTransform\left(
                \frac{\tau}{2}
                \,\odv{C_{out}}{t}
            \right)
            = \bar{C}_{out}
            + \frac{\tau}{2}\,s\,\bar{C}_{out}
            \implies &\\&
            \implies
            \frac{\bar{C}_{out}}{\bar{C}_{in}}
            = \frac{1}{1+\frac{\tau}{2}\,s}
            \,\frac{\bar{C}_{out,1}}{\bar{C}_{in}}
            % 
            % C out 1
            % 
            ; &\\[3ex]&
            \mathemph{\bar{C}_{out,1}/\bar{C}_{in}}
            &\\&
            \nu\,C_{in}
            = \nu\,C_{out,1}
            + \frac{V}{2}
            \,\odv{C_{out,1}}{t}
            \implies &\\&
            \implies
            C_{in}
            = C_{out,1}
            + \frac{\tau}{2}
            \,\odv{C_{out,1}}{t}
            \implies &\\&
            \implies
            \LagrangeTransform{C}_{in}
            = \bar{C}_{in}
            = \LagrangeTransform{C}_{out,1}
            + \LagrangeTransform{\left(
                \frac{\tau}{2}
                \,\odv{C_{out,1}}{t}
            \right)}
            = &\\&
            = \bar{C}_{out,1}
            + \frac{\tau}{2}\,s\,\bar{C}_{out,1}
            \implies &\\&
            \implies
            \frac{\bar{C}_{out,1}}{\bar{C}_{in}}
            = \frac{1}{1+\frac{\tau}{2}\,s}
            % 
            % 
            % 
            ; &\\[3ex]&
            \implies
            g(s)
            = \frac{1}{1+\frac{\tau}{2}\,s}
            \,\frac{1}{1+\frac{\tau}{2}\,s}
            = \frac{1}{(1+\frac{\tau}{2}\,s)^2}
            = \frac{1}{(2/\tau)^2}
            \,\frac{1}{(
                \frac{2}{\tau}
                + s
            )^2}
            = &\\&
            = \frac{\tau^2}{4}
            \,\frac{1}{(
                \frac{2}{\tau} + s
            )^2}
            \implies &\\[3ex]&
            \implies
            E(t)
            = \LagrangeTransform{g}(s)
            = \frac{\tau^2}{4}
            \,t\,\exp{\left(
                -\frac{2}{\tau}\,t
            \right)}
        &
    \end{flalign*}
\end{questionBox}

\begin{questionBox}2{ % MARK: Q1.3
    Função Culmulativa
} % Q1.3
    \answer{}
    \begin{flalign*}
        &
            F(t)
            = \int_0^t{E(t)\,\odif{t}}
            = \int_0^t{
                \left(
                    \frac{\tau^2}{4}
                    \,t\,\exp{\left(
                        -\frac{2}{\tau}\,t
                    \right)}
                \right)
                \,\odif{t}
            }
            = &\\&
            = 
            \frac{\tau^2}{4}
            \int_0^t{
                \left(
                    t\,\exp{\left(
                        -\frac{2}{\tau}\,t
                    \right)}
                \right)
                \,\odif{t}
            }
            % 
            % Primitiva
            % 
            ; &\\[3ex]&
            \text{\emph{Primitiva:}}
            &\\&
            \odv{}{t}{\left(
                t\,\exp{\left(
                    -\frac{2}{\tau}\,t
                \right)}
            \right)}
            = &\\&
            = \exp{\left(
                -\frac{2}{\tau}\,t
            \right)}
            +
            t
            \,\left(
                -\frac{\tau}{2}
            \right)
            \,\exp{\left(
                -\frac{2}{\tau}\,t
            \right)}
            \implies &\\&
            \implies
            \Primitive\left(
                \odv{}{t}{\left(
                    t\,\exp{\left(
                        -\frac{2}{\tau}\,t
                    \right)}
                \right)}
            \right)
            = t\,\exp{\left(
                -\frac{2}{\tau}\,t
            \right)}
            = &\\&
            = \Primitive{\left(
                \exp{\left(
                    -\frac{2}{\tau}\,t
                \right)}
            \right)}
            + \Primitive{\left(
                t
                \,\left(
                    -\frac{\tau}{2}
                \right)
                \,\exp{\left(
                    -\frac{2}{\tau}\,t
                \right)}
            \right)}
            = &\\&
            = -\frac{\tau}{2}
            \exp{\left(
                -\frac{2}{\tau}\,t
            \right)}
            - \frac{\tau}{2}
            \Primitive{\left(
                t
                \,\exp{\left(
                    -\frac{2}{\tau}\,t
                \right)}
            \right)}
            \implies &\\&
            \implies
            - \left(
                t\,\frac{2}{\tau}+1
            \right)
            \,\exp{\left(
                -\frac{2}{\tau}\,t
            \right)}
            = \Primitive{\left(
                t
                \,\exp{\left(
                    -\frac{2}{\tau}\,t
                \right)}
            \right)}
            \implies &\\[3ex]&
            \implies
            F(t)
            = \frac{\tau^2}{4}
            \adif{\left(
                - \left(
                    t\,\frac{2}{\tau}+1
                \right)
                \,\exp{\left(
                    -\frac{2}{\tau}\,t
                \right)}
            \right)}
            \Big\vert_0^t
            = &\\&
            = \frac{\tau^2}{4}
            \left(
                - \left(
                    t\,\frac{2}{\tau}+1
                \right)
                \,\exp{\left(
                    -\frac{2}{\tau}\,t
                \right)}
                + \left(
                    0*\frac{2}{\tau}+1
                \right)
                \,\exp{\left(
                    -\frac{2}{\tau}*0
                \right)}
            \right)
            = &\\&
            = \frac{\tau^2}{4}
            \left(
                - \left(
                    t\,\frac{2}{\tau}+1
                \right)
                \,\exp{\left(
                    -\frac{2}{\tau}\,t
                \right)}
                + 1
            \right)
        &
    \end{flalign*}
\end{questionBox}

\setcounter{subquestion}{4}

\begin{questionBox}2{ % MARK: Q1.5
    Concentração do tracador de saída
    \begin{itemize}
        \item Traçador Impulso
        \item \(t=\qty*{17}{\min}\)
        \item \(N=\qty*{8}{\min}\)
        \item \(\nu\)
    \end{itemize}
} % Q1.5
    \answer{}
    \begin{flalign*}
        &
            \text{\emph{Traçador Impulso:}}
            &\\&
            C
            =\frac{N}{\nu}\,E(t)
            =\frac{8}{\nu}\,E(17)
            = &\\&
            =\frac{8}{\nu}
            \frac{(1000/\nu)^2}{4}
            \,17*\exp{\left(
                -\frac{2}{1000/\nu}*17
            \right)}
            % 
            % Nu
            % 
            ; &\\[3ex]&
            \mathemph{\nu:}
            &\\&
            \odv[order=2]{E(t)}{t}
            = 0
            = \odv{}{t}{\left(
                \frac{\tau^2}{4}
                \,\exp{\left(
                    -\frac{2}{\tau}\,t
                \right)}
                - \frac{\tau^2}{4}
                \,t
                \,\frac{\tau}{2}
                \,\exp{\left(
                    -\frac{2}{\tau}\,t
                \right)}
            \right)}
            = &\\&
            = -\frac{\tau^2}{4}
            \,\frac{\tau}{2}
            \,\exp{\left(
                -\frac{2}{\tau}\,t
            \right)}
            - \frac{\tau^2}{4}
            \,\frac{\tau}{2}
            \,\exp{\left(
                -\frac{2}{\tau}\,t
            \right)}
            + \frac{\tau^2}{4}
            \,t
            \,\frac{\tau}{2}
            \,\frac{\tau}{2}
            \,\exp{\left(
                -\frac{2}{\tau}\,t
            \right)}
            = &\\&
            = -\frac{\tau^3}{8}
            \,\exp{\left(
                -\frac{2}{\tau}\,t
            \right)}
            - \frac{\tau^3}{8}
            \,\exp{\left(
                -\frac{2}{\tau}\,t
            \right)}
            + \frac{\tau^4}{16}
            \,t
            \,\exp{\left(
                -\frac{2}{\tau}\,t
            \right)}
            \implies &\\&
            \implies
            \odv[order=2]{E}{t}(10)
            = -\frac{(1000/\nu)^3}{8}
            \,\exp{\left(
                -\frac{2}{(1000/\nu)}\,10
            \right)}
            + &\\&
            - \frac{(1000/\nu)^3}{8}
            \,\exp{\left(
                -\frac{2}{(1000/\nu)}\,10
            \right)}
            + &\\&
            + \frac{(1000/\nu)^4}{16}
            \,10
            \,\exp{\left(
                -\frac{2}{(1000/\nu)}\,01
            \right)}
            = &\\&
            = \frac{1\E9}{8\,\nu^3}
            \,\left(
                - 2
                + \frac{1\E{4}}{2\,\nu}
            \right)
            \,\exp{\left(
                -\frac{\nu}{50}
            \right)}
            ; &\\&
            \text{Assumindo: }\nu=\qty*{10}{\deci\metre^3/\min}
            % 
            % 
            % 
            ; &\\[3ex]&
            \implies
            C
            =\frac{8}{10}
            \frac{(1000/10)^2}{4}
            \,17*\exp{\left(
                -\frac{2}{1000/10}*17
            \right)}
            = &\\&
            =\frac{34\E3}{1}
            \,\exp{\left(
                -\frac{1}{85}
            \right)}
            \cong &\\&
            \cong
            \qty{3.36023437410313227463e5}{\M}
        &
    \end{flalign*}
    Concentração absurdamente alta, não faz sentido, um erro a apontar poderia ser ter assumido o caldal porem um caldal de \qty*{10}{\deci\metre^3/\min} é bastante normal.
\end{questionBox}

\begin{questionBox}2{ % MARK: Q1.6
    Traçador por degrau, Concentração de saída
} % Q1.6
    \answer{}
    \begin{flalign*}
        &
            \text{\emph{Traçador degral:}}
            &\\&
            F(t)
            = \frac{C(t)}{C_0}
            \implies &\\&
            \implies
            C(t)=F(t)\,C_0
            % 
            % 
            % 
            ; &\\[3ex]&
            \begin{cases}
                C_0=\qty*{0.2}{\M}
                \\
                t=\qty*{16}{\min}
            \end{cases}
            &\\&
            C(16)
            = &\\&
            = \frac{(1000/10)^2}{4}
            \left(
                - \left(
                    16\,\frac{2}{(1000/10)}+1
                \right)
                \,\exp{\left(
                    -\frac{2}{(1000/10)}\,16
                \right)}
                + 1
            \right)
            \,0.2
            \cong &\\&
            \cong 
            \qty{9.79258364468636e2}{}
        &
    \end{flalign*}
    Pelas contas \(F(t)\) tem dado superior a 1 o que é impossível, falho em perceber a origem do erro
\end{questionBox}

\end{document}
% !TEX root = ./ERQ_II-Exercises_Resolutions.1.tex
\providecommand\mainfilename{"./ERQ_II-Exercises_Resolutions.tex"}
\providecommand \subfilename{}
\renewcommand   \subfilename{"./ERQ_II-Exercises_Resolutions.1.tex"}
\documentclass[\mainfilename]{subfiles}

% \tikzset{external/force remake=true} % - remake all

\begin{document}

\graphicspath{{\subfix{./.build/figures/ERQ_II-Exercises_Resolutions.1}}}
% \tikzsetexternalprefix{./.build/figures/ERQ_II-Exercises_Resolutions.1/graphics/}

\mymakesubfile{1}
[ERQ II]
{Exercises} % Subfile Title
{Exercises} % Part Title

\setcounter{question}{1}

\begin{questionBox}1{ % MARK: Q2
    Considere um reactor CSTR cujo comportamento não ideal pode ser modelado pela associação de reactores ideais em ``by-pass'', esquematizada na figura.
    \begin{center}
        \includegraphics[width=.8\textwidth]{Screenshot 2024-04-05 at 14.52.37-cutout.png}
    \end{center}
} % Q2
    \paragraph*{Transformadas de Laplace:}
    \begin{center}
        \vspace{1ex}
        \begin{tabular}{CC}
            \toprule
            f(s) & F(t)
            \\\midrule
            \frac{1}{s-a}
            & e^{a\,t}
            \\\bottomrule
        \end{tabular}
        \vspace{2ex}
    \end{center}
\end{questionBox}

\begin{questionBox}2{ % MARK: Q2.1
    Escreva as equações do modelo que representa o escoamento no reactor.
} % Q2.1
    \answer{}
    \begin{center}
        \vspace{1ex}
        \begin{tabular}{*{4}{C}}
            \toprule
            
                \multicolumn{1}{c}{Reator}
                & \multicolumn{1}{c}{Volume}
                & \multicolumn{1}{c}{Caldal In}
                & \multicolumn{1}{c}{Caldal Out}
            
            \\\midrule
            
                1
                & (1-\alpha)\,V 
                & (1-\beta)\,\nu
                & (1-\beta)\,\nu
                \\
                2
                & \alpha\,V
                & \beta\,\nu
                & \beta\,\nu
            
            \\\bottomrule
        \end{tabular}
        % \vspace{0ex}
    \end{center}
    \begin{flalign*}
        &
            % 
            % R1
            % 
            \mathemph{R_1:}
            &\\&
            (1-\beta)\,\nu\,C_{in,1}
            = (1-\beta)\,\nu\,C_{in}
            = &\\[1.5ex]&
            = (1-\beta)\,\nu\,C_{out,1}
            + (1-\alpha)\,V\,\odv{C_{out,1}}{t}
            % 
            % R2
            % 
            ; &\\[3ex]&
            \mathemph{R_2:}
                &\\&
            \beta\,\nu\,C_{in,2}
            = \beta\,\nu\,C_{in}
            = \beta\,\nu\,C_{out,2}
            + \alpha\,V\,\odv{C_{out,2}}{t}
            % 
            % Nó
            % 
            ; &\\[3ex]&
            \text{\emph{Nó:}}
            &\\&
            \nu\,C_{out}
            = \beta\,\nu\,C_{out,2}
            + (1-\beta)\,\nu\,C_{out,1}
        &
    \end{flalign*}
\end{questionBox}
\begin{questionBox}2{ % MARK: Q2.2
    Deduza a expressão da distribuição de tempos de residência.
} % Q2.2
    \answer{}
    \begin{flalign*}
        &
            E(t) = \LagrangeTransform{g}(s)
            % 
            % Nó
            % 
            ; &\\[3ex]&
            \mathemph{g(s):}
            &\\&
            g(s)
            =\bar{C}_{out}/\bar{C}_{in}
            =\LagrangeTransform{C}_{out}
            /\LagrangeTransform{C}_{in}
            ; &\\&
            \nu\,C_{out}
            = \beta\,\nu\,C_{out,2}
            + (1-\beta)\,\nu\,C_{out,1}
            \implies &\\&
            \implies
            C_{out}
            = \beta\,C_{out,2}
            + (1-\beta)\,C_{out,1}
            \implies &\\&
            \implies
            \LagrangeTransform{C}_{out}
            = \bar{C}_{out}
            = \beta\,\bar{C}_{out,2}
            + (1-\beta)\,\bar{C}_{out,1}
            \implies &\\&
            \implies
            g(s)
            = \frac{\bar{C}_{out}}{\bar{C}_{in}}
            = \beta
            \,\frac{\bar{C}_{out,2}}{\bar{C}_{in}}
            + (1-\beta)
            \,\frac{\bar{C}_{out,1}}{\bar{C}_{in}}
            % 
            % C_out,1
            % 
            ; &\\[3ex]&
            \mathemph{\bar{C}_{out,1}:}
            &\\&
            (1-\beta)\,\nu\,C_{in}
            = (1-\beta)\,\nu\,C_{out,1}
            + (1-\alpha)\,V\,\odv{C_{out,1}}{t}
            \implies &\\&
            \implies
            (1-\beta)\,C_{in}
            = (1-\beta)\,C_{out,1}
            + (1-\alpha)\tau\,\odv{C_{out,1}}{t}
            \implies &\\&
            \implies
            \LagrangeTransform{((1-\beta)\,C_{in})}
            = (1-\beta)\,\bar{C}_{in}
            = &\\&
            = \LagrangeTransform{
                ((1-\beta)\,C_{out,1})
            }
            + \LagrangeTransform{\left(
                    (1-\alpha)\tau\,\odv{C_{out,1}}{t}
            \right)}
            = &\\&
            = (1-\beta)\,\bar{C}_{out,1}
            + (1-\alpha)\tau\,s\,\bar{C}_{out,1}
            \implies &\\&
            \implies
            \frac{\bar{C}_{out,1}}{\bar{C}_{in}}
            = \frac
            {1-\beta}
            {1-\beta+(1-\alpha)\tau\,s}
            % 
            % C out 2
            % 
            ; &\\[3ex]&
            \mathemph{\bar{C}_{out,2}:}
            &\\&
            \beta\,\nu\,C_{in}
            = \beta\,\nu\,C_{out,2}
            + \alpha\,V\,\odv{C_{out,2}}{t}
            \implies &\\&
            \implies
            \beta\,C_{in}
            = \beta\,C_{out,2}
            + \alpha\,\tau\,\odv{C_{out,2}}{t}
            \implies &\\&
            \implies
            \LagrangeTransform{(\beta\,C_{in})}
            = \beta\,\bar{C}_{in}
            = &\\&
            = \LagrangeTransform{
                (\beta\,C_{out,2})
            }
            + \LagrangeTransform{\left(
                \alpha\,\tau\,\odv{C_{out,2}}{t}
            \right)}
            = &\\&
            = \beta\,\bar{C}_{out,2}
            + \alpha\,\tau\,s\,\bar{C}_{out,2}
            \implies &\\&
            \implies
            \frac{\bar{C}_{out,2}}{\bar{C}_{in}}
            = \frac
            {\beta}
            {\beta + \alpha\,\tau\,s}
            % 
            % g(s)
            % 
            ; &\\[3ex]&
            \implies
            g(s)
            = \beta
            \,\frac{\beta}{\beta + \alpha\,\tau\,s}
            + (1-\beta)
            \,\frac{1-\beta}
            {1-\beta+(1-\alpha)\tau\,s}
            = &\\&
            = \frac{\beta^2}{\alpha\,\tau}
            \,\left(
                \frac{1}{\frac{\beta}{\alpha\,\tau} + s}
            \right)
            + \frac{(1-\beta)^2}
            {(1-\alpha)\tau}
            \left(
                \frac{1}{
                    \frac{1-\beta}{(1-\alpha)\tau}
                    +s
                }
            \right)
            \implies &\\[3ex]&
            \implies
            \LagrangeTransform{g}(s)
            = \frac{\beta^2}{\alpha\,\tau}
            \,\exp{\left(
                -\frac{\beta}{\alpha\,\tau}\,t
            \right)}
            + \frac{(1-\beta)^2}
            {(1-\alpha)\tau}
            \exp{\left(
                -\frac{1-\beta}{(1-\alpha)\tau}\,t
            \right)}
        &
    \end{flalign*}
\end{questionBox}
\begin{questionBox}2{ % MARK: Q2.3
    Deduza a expressão da função cumulativa.
} % Q2.3
    \answer{}
    \begin{flalign*}
        &
            \text{\emph{Função culmulativa:}}
            &\\&
            F(t)
            = \int_0^t{E(t)\,\odif{t}}
            = &\\&
            = \int_0^t{
                \left(
                    \frac{\beta^2}{\alpha\,\tau}
                    \,\exp{\left(
                        -\frac{\beta}{\alpha\,\tau}\,t
                    \right)}
                    + \frac{(1-\beta)^2}
                    {(1-\alpha)\tau}
                    \exp{\left(
                        -\frac{1-\beta}{(1-\alpha)\tau}\,t
                    \right)}
                \right)
                \,\odif{t}
            }
            = &\\&
            = 
            \frac{\beta^2}{\alpha\,\tau}
            \int_0^t{
                \exp{\left(
                    -\frac{\beta}{\alpha\,\tau}\,t
                \right)}
                \,\odif{t}
            }
            + &\\&
            + \frac{(1-\beta)^2}
            {(1-\alpha)\tau}
            \,\int_0^t{
                \exp{\left(
                    -\frac{1-\beta}{(1-\alpha)\tau}\,t
                \right)}
                \odif{t}
            }
            = &\\&
            = 
            \frac{\beta^2}{\alpha\,\tau}
            \,\frac{\alpha\,\tau}{\beta}
            \adif{
                -\exp{\left(
                    -\frac{\beta}{\alpha\,\tau}\,t
                \right)}
            }\,\Big\vert_0^t
            + &\\&
            + \frac{(1-\beta)^2}{(1-\alpha)\tau}
            \,\frac{(1-\alpha)\tau}{1-\beta}
            \adif{
                -\exp{\left(
                    -\frac{1-\beta}{(1-\alpha)\tau}\,t
                \right)}
            }\,\Big\vert_0^t
            = &\\&
            = 
            (\beta)
            \left(
                -\exp{\left(
                    -\frac{\beta}{\alpha\,\tau}\,t
                \right)}
                + \exp{0}
            \right)
            + &\\&
            + (1-\beta)
            \left(
                -\exp{\left(
                    -\frac{1-\beta}{(1-\alpha)\tau}\,t
                \right)}
                +\exp{0}
            \right)
            = &\\&
            = 
            \beta
            \left(
                -\exp{\left(
                    -\frac{\beta}{\alpha\,\tau}\,t
                \right)}
                + 1
            \right)
            + &\\&
            + (1-\beta)
            \left(
                -\exp{\left(
                    -\frac{1-\beta}{(1-\alpha)\tau}\,t
                \right)}
                +1
            \right)
        &
    \end{flalign*}
\end{questionBox}
\begin{questionBox}2{ % MARK: Q2.4
    Sabendo que no reactor é introduzido um traçador, por degrau, com uma concentração à entrada do reactor \(C_0= \qty*{0.1}{\molar}\), a um caudal volumétrico \qty*{10}{\deci\metre^3/\min}, calcule o tempo ao fim do qual a concentração de traçador à saída é 95\% da concentração à entrada. 
    \begin{itemize}
        \item Volume do reactor: \qty*{1}{\metre^3}; 
        \item caudal de by-pass: 10\% do caudal volumétrico à entrada; 
        \item volume do by-pass: 20\% do volume do reactor.
    \end{itemize}
} % Q2.4
\answer{}
\begin{flalign*}
    &
        t:C(t)=0.95\,C_0
        ; &\\[3ex]&
        % 
        % 
        % 
        \text{\emph{Tracador em Degrau:} }
        F(t) = \frac{C(t)}{C_0}
        ; &\\&
        \text{\emph{Normalização da curva \(C\):}}
        &\\&
        \begin{cases}
            \alpha=20\%
            \\ \beta=10\%
        \end{cases}
        &\\&
        F(t)
        = \frac{C(t)}{C_0}
        = \frac{0.95\,C_0}{C_0}
        = 0.95
        = &\\[3ex]&
        = \beta
        \left(
            -\exp{\left(
                -\frac{\beta}{\alpha\,\tau}\,t
            \right)}
            + 1
        \right)
        + &\\&
        + (1-\beta)
        \left(
            -\exp{\left(
                -\frac{1-\beta}{(1-\alpha)\tau}\,t
            \right)}
            +1
        \right)
        &\\&
        = 0.1
        \left(
            -\exp{\left(
                -\frac{0.1}{
                    0.2
                    \,\frac{1000}{10}
                }
                \,t
            \right)}
            + 1
        \right)
        + &\\&
        + (1-0.1)
        \left(
            -\exp{\left(
                -\frac{1-0.1}{
                    (1-0.2)\frac{1000}{10}
                }
                \,t
            \right)}
            +1
        \right)
        \cong &\\&
        \cong
        - \exp{\left(
            -\frac{t}{200}
        \right)}
        - \exp{\left(
            -\frac{9\,t}{800}
        \right)}
        + 1
        \implies &\\[3ex]&
        \implies
        f(t)
        = - \exp{\left(
            -\frac{t}{200}
        \right)}
        - \exp{\left(
            -\frac{9\,t}{800}
        \right)}
        + 0.05
    &
\end{flalign*}
\begin{center}
    \includegraphics[width=.8\textwidth]{Q1d.pdf}
\end{center}
\begin{flalign*}
    &
        t\cong\qty{307.0224}{\min}
    &
\end{flalign*}
% \begin{center}
%     \includegraphics[width=.8\textwidth]{IMG_6605}
%     \includegraphics[width=.8\textwidth]{IMG_6606}
%     \includegraphics[width=.8\textwidth]{IMG_6607}
%     \includegraphics[width=.8\textwidth]{IMG_6608}
% \end{center}
\end{questionBox}

\setcounter{question}{3}

\begin{questionBox}1{ % MARK: Q4
    Considere um reactor contínuo cujo comportamento não ideal pode ser modelado pela associação de reactores ideais esquematizada na figura.
    \begin{center}
        \includegraphics[width=.8\textwidth]{Screenshot 2024-04-05 at 15.35.55-cutout.png}
    \end{center}
} % Q4
    \paragraph*{Transformadas de Laplace:}
    \begin{center}
        \vspace{1ex}
        \begin{tabular}{*{2}{C}}
            \toprule
            
                f(s)
                & F(t)
                \\\midrule
                \frac{1}{s-a}
                & e^{a\,t}
                \\
                \frac{1}{(s-a)^2}
                & t\,e^{a\,t}
            
            \\\bottomrule
        \end{tabular}
        \vspace{2ex}
    \end{center}
\end{questionBox}
\begin{questionBox}2{ % MARK: Q4.1
    Escreva as equações do modelo que representa o escoamento no reactor, tendo em conta que os tanques 1 e 2 são iguais.
} % Q4.1
    \answer{}
    \begin{center}
        \includegraphics[width=1\textwidth]{1712328642588.jpg}
    \end{center}
    \begin{center}
        \vspace{1ex}
        \begin{tabular}{*{4}{C}}
            \toprule
            
                \multicolumn{1}{c}{Reator}
                & \multicolumn{1}{c}{Volume}
                & \multicolumn{1}{c}{Caldal In}
                & \multicolumn{1}{c}{Caldal Out}
            
            \\\midrule
            
                1 
                & \alpha\,V
                & \beta\,\nu
                & \beta\,\nu
                \\ 
                2 
                & (1-\alpha)\,V/2
                & (1-\beta)\,\nu
                & (1-\beta)\,\nu
                \\ 
                3 
                & (1-\alpha)\,V/2
                & (1-\beta)\,\nu
                & (1-\beta)\,\nu
            
            \\\bottomrule
        \end{tabular}
        % \vspace{2ex}
    \end{center}
    \begin{flalign*}
        &
            % 
            % R1
            % 
            \mathemph{R_1:}
            &\\&
            (1-\beta)\,\nu\,C_{in,1}
            = (1-\beta)\,\nu\,C_{in}
            = &\\&
            = (1-\beta)\,\nu\,C_{out,1}
            + (1-\alpha)\,\frac{V}{2}
            \,\odv{C_{out,1}}{t}
            % 
            % R2
            % 
            ; &\\[3ex]&
            \mathemph{R_2:\,}
            &\\&
            (1-\beta)\,\nu\,C_{in,2}
            = (1-\beta)\,\nu\,C_{out,1}
            = &\\&
            = (1-\beta)\,\nu\,C_{out,2}
            + (1-\alpha)\,\frac{V}{2}
            \,\odv{C_{out,2}}{t}
            % 
            % R_3
            % 
            ; &\\[3ex]&
            \mathemph{R_3:}
            &\\&
            \beta\,\nu\,C_{in,3}
            = \beta\,\nu\,C_{in}
            = \beta\,\nu\,C_{out,3}
            + \alpha\,V\,\odv{C_{out,3}}{t}
            % 
            % Addition Knot
            % 
            ; &\\[3ex]&
            \mathemph{\text{Nó}:}
            &\\&
            \nu\,C_{out}
            = (1-\beta)\,\nu\,C_{out,2}
            + \beta\,\nu\,C_{out,3}
        &
    \end{flalign*}
\end{questionBox}
\begin{questionBox}2{ % MARK: Q4.2
    Deduza a expressão da distribuição de tempos de residência.
} % Q4.2
    \answer{}
    \begin{flalign*}
        &
            E(t) = \LagrangeTransform{g}(s)
            ; &\\[3ex]&
            \mathemph{g(s):}
            &\\&
            g(s)
            =\bar{C}_{out}/\bar{C}_{in}
            =\LagrangeTransform{C}_{out}
            /\LagrangeTransform{C}_{in}
            % 
            % 
            % 
            ; &\\&
            \nu\,C_{out}
            = (1-\beta)\,\nu\,C_{out,2}
            + \beta\,\nu\,C_{out,3}
            \implies &\\&
            \implies
            C_{out}
            = (1-\beta)\,C_{out,2}
            + \beta\,C_{out,3}
            \implies &\\&
            \implies
            \LagrangeTransform{C}_{out}
            =\bar{C}_{out}
            = &\\&
            = \LagrangeTransform{((1-\beta)\,C_{out,2})}
            + \LagrangeTransform{(\beta\,C_{out,3})}
            = &\\&
            = (1-\beta)\,\bar{C}_{out,2}
            + \beta\,\bar{C}_{out,3}
            \implies &\\&
            \implies
            g(s)
            = (1-\beta)\,\frac{\bar{C}_{out,2}}{\bar{C}_{in}}
            + \beta\,\frac{\bar{C}_{out,3}}{\bar{C}_{in}}
            % 
            % C out 3
            % 
            ; &\\[3ex]&
            \mathemph{\bar{C}_{out,3}/\bar{C}_{in}:}
            &\\&
            \beta\,\nu\,C_{in}
            = \beta\,\nu\,C_{out,3}
            + \alpha\,V\,\odv{C_{out,3}}{t}
            \implies &\\&
            \implies
            \beta\,C_{in}
            = \beta\,C_{out,3}
            + \alpha\,\tau\,\odv{C_{out,3}}{t}
            \implies &\\&
            \implies
            \LagrangeTransform{(\beta\,C_{in})}
            = \beta\,\bar{C}_{in}
            = &\\&
            = \LagrangeTransform{(\beta\,C_{out,3})}
            + \LagrangeTransform{\left(
                \alpha\,\tau\,\odv{C_{out,3}}{t}
            \right)}
            = \beta\,\bar{C}_{out,3}
            + \alpha\,\tau\,s\,\bar{C}_{out,3}
            \implies &\\&
            \implies
            \frac{\bar{C}_{out,3}}{\bar{C}_{in}}
            = \frac{\beta}{\beta+\alpha\,\tau\,s}
            % 
            % C out 2
            % 
            ; &\\[3ex]&
            \mathemph{\bar{C}_{out,2}/\bar{C}_{in}:}
            &\\&
            (1-\beta)\,\nu\,C_{out,1}
            = (1-\beta)\,\nu\,C_{out,2}
            + (1-\alpha)\,\frac{V}{2}
            \,\odv{C_{out,2}}{t}
            \implies &\\&
            \implies
            (1-\beta)\,C_{out,1}
            = (1-\beta)\,C_{out,2}
            + \frac{1-\alpha}{2}
            \,\tau\,\odv{C_{out,2}}{t}
            \implies &\\&
            \implies
            \LagrangeTransform{((1-\beta)\,C_{out,1})}
            = (1-\beta)\,\bar{C}_{out,1}
            = &\\&
            = \LagrangeTransform{((1-\beta)\,C_{out,2})}
            + \LagrangeTransform{\left(
                \frac{1-\alpha}{2}
                \,\tau\,\odv{C_{out,2}}{t}
            \right)}
            = &\\&
            = (1-\beta)\,\bar{C}_{out,2}
            + \frac{1-\alpha}{2}
            \,\tau\,s\,\bar{C}_{out,2}
            \implies &\\&
            \implies
            \frac{\bar{C}_{out,2}}{\bar{C}_{in}}
            = \frac{(1-\beta)}{
                (1-\beta)
                +\frac{(1-\alpha)\tau}{2}
                \,s
            }
            \,\frac{\bar{C}_{out,1}}{\bar{C}_{in}}
            % 
            % C out 1
            % 
            ; &\\[3ex]&
            \mathemph{\bar{C}_{out,1}/\bar{C}_{in}:}
            &\\&
            (1-\beta)\,\nu\,C_{in}
            = (1-\beta)\,\nu\,C_{out,1}
            + (1-\alpha)\,\frac{V}{2}
            \,\odv{C_{out,1}}{t}
            \implies &\\&
            \implies 
            (1-\beta)\,C_{in}
            = (1-\beta)\,C_{out,1}
            + \frac{1-\alpha}{2}
            \,\tau\,\odv{C_{out,1}}{t}
            \implies &\\&
            \implies 
            \LagrangeTransform{((1-\beta)\,C_{in})}
            = (1-\beta)\,\bar{C}_{in}
            = &\\&
            = \LagrangeTransform{(1-\beta)\,C_{out,1}}
            + \LagrangeTransform{\left(
                \frac{1-\alpha}{2}
                \,\tau\,\odv{C_{out,1}}{t}
            \right)}
            = &\\&
            = (1-\beta)\,\bar{C}_{out,1}
            + \frac{1-\alpha}{2}
            \,\tau\,s\,\bar{C}_{out,1}
            \implies &\\&
            \implies
            \frac{\bar{C}_{out,1}}{\bar{C}_{in}}
            = \frac{(1-\beta)}{
                (1-\beta)
                + \frac{(1-\alpha)\,\tau}{2}\,s
            }
            % 
            % 
            % 
            \implies &\\[3ex]&
            \implies
            \frac{\bar{C}_{out,2}}{\bar{C}_{in}}
            = \frac{(1-\beta)}{
                (1-\beta)
                +\frac{(1-\alpha)\tau}{2}
                \,s
            }
            \,\frac{\bar{C}_{out,1}}{\bar{C}_{in}}
            = &\\&
            =\left(
                \frac{(1-\beta)}{
                    (1-\beta)
                    + \frac{(1-\alpha)\,\tau}{2}\,s
                }
            \right)^2
            % 
            % Preparando g(s)
            % 
            ; &\\[3ex]&
            \implies
            g(s)
            = &\\&
            = (1-\beta)
            \,\left(
                \frac{(1-\beta)}{
                    (1-\beta)
                    + \frac{(1-\alpha)\,\tau}{2}\,s
                }
            \right)^2
            + \beta
            \,\left(
                \frac{\beta}{\beta+\alpha\,\tau\,s}
            \right)
            = &\\&
            = \frac
            {(1-\beta)^3}
            {\left(\frac{(1-\alpha)\,\tau}{2}\right)^2}
            \,\frac{1}{
                \left(
                    \frac{(1-\beta)\,2}{(1-\alpha)\,\tau}
                    + s
                \right)^2
            }
            + \frac{\beta^2}{\alpha\,\tau}
            \,\frac{1}
            {\frac{\beta}{\alpha\,\tau}+s}
            \implies &\\[3ex]&
            \implies
            E(t)
            = \LagrangeTransform{g}(s)
            = &\\&
            = \frac
            {(1-\beta)^3\,4}
            {(1-\alpha)^2\,\tau^2}
            \,t\,\exp{\left(
                -\frac{(1-\beta)\,2}{(1-\alpha)\,\tau}\,t
            \right)}
            + \frac{\beta^2}{\alpha\,\tau}
            \,\exp{\left(
                -\frac{\beta}{\alpha\,\tau}\,t
            \right)}
        &
    \end{flalign*}
\end{questionBox}
\begin{questionBox}2{ % MARK: Q4.3
    Deduza e expressão da função cumulativa.
} % Q4.3
    \answer{}
    \begin{flalign*}
        &
            F(t)
            = \int_0^t{
                E(t)\,\odif{t}
            }
            = &\\&
            = \int_0^t{
                \frac
                {(1-\beta)^3\,4}
                {(1-\alpha)^2\,\tau^2}
                \,t\,\exp{\left(
                    -\frac{(1-\beta)\,2}{(1-\alpha)\,\tau}\,t
                \right)}
                \,\odif{t}
            }
            + &\\&
            + \int_0^t{
                \frac{\beta^2}{\alpha\,\tau}
                \,\exp{\left(
                    -\frac{\beta}{\alpha\,\tau}\,t
                \right)}
                \,\odif{t}
            }
            = &\\&
            = 
            \frac
            {(1-\beta)^3\,4}
            {(1-\alpha)^2\,\tau^2}
            \int_0^t{
                t\,\exp{\left(
                    -\frac{(1-\beta)\,2}{(1-\alpha)\,\tau}\,t
                \right)}
                \,\odif{t}
            }
            + &\\&
            + \frac{\beta^2}{\alpha\,\tau}
            \,\frac{\alpha\,\tau}{\beta}
            \adif{\left(
                -\exp{\left(
                    -\frac{\beta}{\alpha\,\tau}\,t
                \right)}
            \right)}\Big\vert_0^t
            = &\\&
            = 
            \frac
            {(1-\beta)^3\,4}
            {(1-\alpha)^2\,\tau^2}
            \int_0^t{
                t\,\exp{\left(
                    -\frac{(1-\beta)\,2}{(1-\alpha)\,\tau}\,t
                \right)}
                \,\odif{t}
            }
            + &\\&
            + \beta
            \,\left(
                -\exp{\left(
                    -\frac{\beta}{\alpha\,\tau}\,t
                \right)}
                +\exp{0}
            \right)
            % 
            % primitiva
            % 
            ; &\\[3ex]&
            \text{\emph{Primitiva:}}
            &\\&
            \odif{\left(
                t\,\exp{\left(
                    -\frac{(1-\beta)\,2}{(1-\alpha)\,\tau}\,t
                \right)}
            \right)}
            = &\\&
            = \exp{\left(
                -\frac{(1-\beta)\,2}{(1-\alpha)\,\tau}\,t
            \right)}
            + &\\&
            + t
            \,\left(
                -\frac{(1-\beta)\,2}{(1-\alpha)\,\tau}
            \right)
            \,\exp{\left(
                -\frac{(1-\beta)\,2}{(1-\alpha)\,\tau}\,t
            \right)}
            \implies &\\&
            \implies
            \Primitive{\left(
                \odif{\left(
                    t\,\exp{\left(
                        -\frac{(1-\beta)\,2}{(1-\alpha)\,\tau}\,t
                    \right)}
                \right)}
            \right)}
            = t\,\exp{\left(
                -\frac{(1-\beta)\,2}{(1-\alpha)\,\tau}\,t
            \right)}
            = &\\&
            = \Primitive{\left(
                \exp{\left(
                    -\frac{(1-\beta)\,2}{(1-\alpha)\,\tau}\,t
                \right)}
            \right)}
            + &\\&
            + \Primitive\left(
                t
                \,\left(
                    -\frac{(1-\beta)\,2}{(1-\alpha)\,\tau}
                \right)
                \,\exp{\left(
                    -\frac{(1-\beta)\,2}{(1-\alpha)\,\tau}\,t
                \right)}
            \right)
            = &\\&
            = -\frac{(1-\alpha)\,\tau}{(1-\beta)\,2}
            \exp{\left(
                -\frac{(1-\beta)\,2}{(1-\alpha)\,\tau}\,t
            \right)}
            + &\\&
            - \frac{(1-\beta)\,2}{(1-\alpha)\,\tau}
            \,\Primitive\left(
                t
                \,\exp{\left(
                    -\frac{(1-\beta)\,2}{(1-\alpha)\,\tau}\,t
                \right)}
            \right)
            % 
            % 
            \implies &\\&
            \implies
            \Primitive\left(
                t
                \,\exp{\left(
                    -\frac{(1-\beta)\,2}{(1-\alpha)\,\tau}\,t
                \right)}
            \right)
            = &\\&
            = -\frac{(1-\alpha)\,\tau}{(1-\beta)\,2}
            \,\left(
                t+\frac{(1-\alpha)\,\tau}{(1-\beta)\,2}
            \right)
            \exp{\left(
                -\frac{(1-\beta)\,2}{(1-\alpha)\,\tau}\,t
            \right)}
            % 
            % F(t) conc
            % 
            \implies &\\[3ex]&
            \implies
            F(t)
            =
            \frac
            {(1-\beta)^3\,4}
            {(1-\alpha)^2\,\tau^2}
            * &\\&
            * \adif{\left(
                -\frac{(1-\alpha)\,\tau}{(1-\beta)\,2}
                \,\left(
                    t+\frac{(1-\alpha)\,\tau}{(1-\beta)\,2}
                \right)
                \exp{\left(
                    -\frac{(1-\beta)\,2}{(1-\alpha)\,\tau}\,t
                \right)}
            \right)}
            \,\Big\vert_{0}^{t}
            + &\\&
            + \beta
            \,\left(
                -\exp{\left(
                    -\frac{\beta}{\alpha\,\tau}\,t
                \right)}
                + 1
            \right)
            = &\\&
            =
            \frac
            {(1-\beta)^3\,4}
            {(1-\alpha)^2\,\tau^2}
            * &\\&
            *
            \,\left(
                -\frac{(1-\alpha)\,\tau}{(1-\beta)\,2}
                \,\left(
                    t+\frac{(1-\alpha)\,\tau}{(1-\beta)\,2}
                \right)
                \exp{\left(
                    -\frac{(1-\beta)\,2}{(1-\alpha)\,\tau}\,t
                \right)}
                %
            \right.
            + &\\&
            \left.
                + \left(
                    \frac{(1-\alpha)\,\tau}{(1-\beta)\,2}
                \right)^2
                \,\exp{0}
            \right)
            + &\\&
            + \beta
            \,\left(
                -\exp{\left(
                    -\frac{\beta}{\alpha\,\tau}\,t
                \right)}
                + 1
            \right)
            = &\\&
            = (1-\beta)
            % * &\\&
            % *
            \,\Biggr{(}
            % &\\&
                -\left(
                    t
                    \,\frac
                    {(1-\beta)\,2}
                    {(1-\alpha)\,\tau}
                    + 1
                \right)
                \exp{\left(
                    -\frac{(1-\beta)\,2}{(1-\alpha)\,\tau}\,t
                \right)}
                %
                % + &\\&
                + 1
            % &\\&
            \Biggr{)}
            + &\\&
            + \beta
            \,\left(
                -\exp{\left(
                    -\frac{\beta}{\alpha\,\tau}\,t
                \right)}
                + 1
            \right)
        &
    \end{flalign*}
\end{questionBox}
\begin{questionBox}2{ % MARK: Q4.4
    Sabendo que o reactor real tem um volume de \qty*{1}{\metre^3} e que são introduzidos, por impulso, \qty*{6}{\mole} de um tracador, determine o valor da concentração máxima de tracador. à saída do reactor. 
    \begin{itemize}
        \item Caudal volumétrico da alimentação: \qty*{20}{\deci\metre^3/\min}; 
        \item caudal de by-pass: 5\% do caudal volumétrico à entrada; 
        \item volume do reciclo: 8\% do volume activo; 
        \item volumes mortos: 12\% do volume do reactor.
    \end{itemize}
} % Q4.4
    \answer{}
    \begin{flalign*}
        &
            \text{\emph{Traçador por impulso:}}
            &\\&
            \odv{C}{t}
            = \frac{N}{\nu}\,\odv{E(t)}{t}
            % 
            % 
            % 
            ; &\\[3ex]&
            \begin{cases}
                \beta = 0.05 
                ;\quad (1-\beta)=0.95
                \\
                \alpha = 0.08 
                ;\quad (1-\alpha) = 0.92
                \\
                V_m
                = 0.12*1000\,\unit{\deci\metre^3}
                = \qty*{120}{\deci\metre^3}
                \\
                V
                = (1000-120)\,\unit{\deci\metre^3}
                = \qty*{880}{\deci\metre^3}
                \\
                \nu = \qty*{20}{\deci\metre^3/\min}
                \\
                \tau
                =\frac{880}{20}\unit{\min}
                =\qty*{44}{\min}
                \\
                N = \qty*{6}{\mole}
                \\
                t = \qty*{21}{\min}
            \end{cases}
            % 
            % 
            % 
            &\\&
            \implies
            \odv{C}{t}
            = 0
            = &\\&
            = \frac{N}{\nu}
            \left(
                \frac
                {(1-\beta)^3\,4}
                {(1-\alpha)^2\,\tau^2}
                \left(
                    1-\frac{(1-\beta)\,2}{(1-\alpha)\,\tau}\,t
                \right)
                \exp{\left(
                    -\frac{(1-\beta)\,2}{(1-\alpha)\,\tau}\,t
                \right)}
            \right.
            &\\&
            \left.
                - \frac{\beta^3}{\alpha^2\,\tau^2}
                \,\exp{\left(
                    -\frac{\beta}{\alpha\,\tau}\,t
                \right)}
            \right)
            = &\\&
            = \frac{6}{20}
            \left(
                \frac
                {0.95^3*4}
                {0.92^2*44^2}
                \left(
                    1-\frac{0.95*2}{0.92*44}\,t
                \right)
                \exp{\left(
                    -\frac{0.95*2}{0.92*44}\,t
                \right)}
            \right.
            &\\&
            \left.
                - \frac{0.05^3}{0.08^2*44^2}
                \,\exp{\left(
                    -\frac{0.05}{0.08*44}\,t
                \right)}
            \right)
            = &\\&
            = 
            \left(
                \num{6.27871910590698e-4}
                -\num{2.9470272483259045e-5}\,t
            \right)
            \exp{\left(
                -\num{4.693675889328063e-2}\,t
            \right)}
            &\\&
            - \num{3.02653667355372e-6}
            \,\exp{\left(
                -\num{1.420454545454545e-2}\,t
            \right)}
        &
    \end{flalign*}
    \begin{center}
        \includegraphics[width=.8\textwidth]{Q2.d.pdf}
    \end{center}
    \begin{flalign*}
        &
            t=\qty{21.1004}{\min}
            % 
            % 
            % 
            ; &\\[3ex]&
            C
            \cong \frac{N}{\nu}E(\num{21.1004})
            \cong &\\&
            \cong 
            \frac{6}{20}
            \,\Biggr{(}
                \frac
                {0.95^3*4}
                {0.92^2*44^2}
                \,\num{21.1004}
                % 0.013248348462228
                \,\exp{\left(
                    -\frac{0.95*2}{0.92*44}\,\num{21.1004}
                    % -0.470432583992095
                \right)}
                % 0.624731960775984
                % = 0.008276666711851
                + &\\&
                + \frac{0.05^2}{0.08*44}
                % 2.130681818181818e-4
                \,\exp{\left(
                    -\frac{0.05}{0.08*44}\,\num{21.1004}
                    % -0.299721590909091
                \right)}
                % 0.741024499922742
                % = 1.578887428812663e-4
            \Biggr{)}
            \cong &\\&
            \cong
            \qty{8.4345554547322663e-3}{\M}
        &
    \end{flalign*}
\end{questionBox}

\setcounter{question}{12}

\begin{questionBox}1{ % MARK: Q13
    A reacção elementar \ch{A -> B} é conduzida, na fase gasosa, num reactor multitubular de leito fixo, consistindo em 100 tubos de \qty*{2}{\metre} de comprimento e \qty*{2}{\centi\metre} de diâmetro da secção recta, cheios com um catalisador sólido, poroso, na forma de pellets esféricas de \qty*{5}{\milli\metre} de diâmetro. O reagente A é alimentado puro a um caudal de \qty*{100}{\deci\metre/\min}, à temperatura de \qty*{373}{\kelvin} e à pressão de \qty*{6}{\atm}
} % Q13
    \paragraph*{Dados:}
    \begin{itemize}
        \item \(\rho_P=\qty*{1.3}{\gram/\centi\metre^3}\)
        \item Coeficiente de Difusão externo: \(D_A=\qty*{2.7e-7}{\metre^2/\second}\)
        \item Viscosidade Cinemática: \(\nu=\qty*{4e-6}{\metre^2/\second}\)
        \item \(\varepsilon_B=0.45\)
        \item Difusidade efetiva intraparticular: \(D_e=\qty*{1.3e-8}{\metre^2/\second}\)
        \item Constante cinética: \(k'=\qty*{0.023}{\deci\metre^3.\gram^{-1}\of{cat}.\min^{-1}}\)
        \item \(R\cong\qty{8.20573660809596e-2}{\deci\metre^3.\atm.\mole^{-1}.\kelvin^{-1}}\)
    \end{itemize}
    \paragraph*{Formulas:}
    \begin{BM}
        Sh
        = 1.0\,Re^{1/2}\,Sc^{1/3}
        = \frac{k_c\,d_P}{D_A}
        \,\frac{1}{(1/\varepsilon_b)-1}
        % 
        ;\\
        Re = \frac{u\,d_P}{\nu\,(1-\varepsilon_b)}
        % 
        ; 
        Sc = \frac{\nu}{D_A}
        % 
        ;\\
        \phi
        =R\,\sqrt{\frac
            {k'\,\rho_P}{D_e}
        }
        % 
        ;
        \eta
        = \frac{3}{\phi^2}(\phi\,\coth{\phi}-1)
    \end{BM}
    Perfil de concentração de pellet: \(
        \rho
        = \frac{\sinh{(\phi\,\lambda)}}{\lambda\,\cosh{\phi}}
    \)
\end{questionBox}

\begin{questionBox}2{ % MARK: Q13.1
    Calcule o valor da constante cinética aparente, que observaria no caso da ausência de limitações difusionais externas.
} % Q13.1
    \answer{}
    \begin{flalign*}
        &
            k'_{ap}
            = \eta\,k'
            = \left(
                \frac{3}{\phi^2}(\phi\,\coth{\phi}-1)
            \right)
            \,k'
            ; &\\[3ex]&
            % 
            % 
            % 
            \phi
            = R\,\sqrt{\frac
                {k'\,\rho_P}{D_e}
            }
            \cong &\\&
            \cong 
            \qty{8.20573660809596e-2}{\deci\metre^3.\atm.\mole^{-1}.\kelvin^{-1}}\,\sqrt{\frac
                {
                    \qty{0.023e-3}{\frac{\metre^3}{\min}}
                    \,\unit{\frac{\min}{60\,\second}}
                    *\qty{1.3e3}{\metre^2/\second}
                }{
                    \qty{1.3e-8}{\metre^2/\second}
                }
            }
            \cong &\\&
            \cong
            \num{16.065929817854}
            % 
            % 
            % 
            \implies &\\[3ex]&
            \implies
            k'_{ap}
            = 
            \frac{3}{\phi^2}(\phi\,\coth{\phi}-1)
            \,k'
            \cong &\\&
            \cong
            \frac{3}{(\num{16.065929817854})^2}
            \,(\num{16.065929817854}\,\coth{\num{16.065929817854}}-1)
            % 0.175107788740824
            \,\frac{0.023}{60}
            \cong
            \qty{6.712465235064924e-5}{\litre/\sec.\gram}
        &
    \end{flalign*}
\end{questionBox}

\begin{questionBox}2{ % MARK: Q13.2
    Calcule o valor do coeficiente de transferência de massa.
} % Q13.2
    \answer{}
    \begin{flalign*}
        &
            Sh
            = Re^{1/2}\,Sc^{1/3}
            = \left(
                \frac{u\,d_P}{\nu\,(1-\varepsilon_b)}
            \right)^{1/2}
            \,\left(
                \frac{\nu}{D_a}
            \right)^{1/3}
            % 
            % 
            % 
            ; &\\[3ex]&
            u
            = \frac{V_{tubos}}{A_c}
            = \frac{
                \left(
                    V/N_{tubos}
                \right)
            }{
                \left(
                    \varepsilon_b
                    \,\pi\,D^2/4
                \right)
            }
            = \frac{
                \left(
                    V/60*100
                \right)
            }{
                \left(
                    0.45
                    \,\pi\,0.02^2/4
                \right)
            }
            \cong &\\&
            \cong 
            \nu\,\num{1.17892550438441}
            % 
            % 
            % 
            \implies &\\[3ex]&
            \implies
            Sh
            = \left(
                \frac{
                    \nu
                    *\num{1.17892550438441}
                    *0.005
                }{
                    \nu
                    \,(1-0.45)
                }
            \right)^{1/2}
            \,\left(
                \frac{4\E{-6}}{2.7\E{-7}}
            \right)^{1/3}
        &
    \end{flalign*}
\end{questionBox}

\begin{questionBox}2{ % MARK: Q13.3
    Calcule o valor da constante cinética realmente observada.
} % Q13.3
\end{questionBox}

\begin{questionBox}2{ % MARK: Q13.4
    Diga, justificando a sua resposta, se o reactor se encontra em regime cinético, difusional interno, difusional externo ou misto.
} % Q13.4
\end{questionBox}

\begin{questionBox}2{ % MARK: Q13.5
    Calcule a conversão à saída do reactor.
} % Q13.5
\end{questionBox}

\begin{questionBox}2{ % MARK: Q13.6
    Determine o valor da concentração de A no centro das pellets, à saída do reactor.
} % Q13.6
\end{questionBox}

\end{document}
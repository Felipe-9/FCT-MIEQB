% !TEX root = ./ERQ_II-Exercises_Resolutions.2.tex
\providecommand\mainfilename{"./ERQ_II-Exercises_Resolutions.tex"}
\providecommand \subfilename{}
\renewcommand   \subfilename{"./ERQ_II-Exercises_Resolutions.2.tex"}
\documentclass[\mainfilename]{subfiles}

% \tikzset{external/force remake=true} % - remake all

\renewcommand\thequestion{Questão \Alph{question}}
\renewcommand\thesubquestion{Q\Alph{question}.\alph{subquestion}}

\begin{document}

% \graphicspath{{\subfix{./figures/ERQ_II-Exercises_Resolutions.2}}}
% \tikzsetexternalprefix{./figures/ERQ_II-Exercises_Resolutions.2/graphics/}

\mymakesubfile{2}
[ERQ II]
{Exercises 2024.2 Resolutions} % Subfile Title
{Exercises 2024.2 Resolutions} % Part Title

\begin{questionBox}1{ % MARK: Q1
    A reacção de 1ª ordem, na fase líquida, \ch{A -> B}, é conduzida sobre um catalisador na forma de pellets esféricas. Fizeram-se duas experiências no laboratório, em que a reacção foi conduzida em reactor batch (balão de ensaio) carregado com \qty*{150}{\ml} de uma solução \qty*{0.1}{\M\of{A}} e \qty*{0.1}{\g\of{cat}}. Em cada uma das experiências usaram-se ``pellets'' de diferentes diâmetros. Ambas as experiências foram interrompidas ao fim de 1 hora, sendo as misturas analisadas e as respectivas conversões finais as mostradas na tabela. Assuma ausência de limitações difusionais externas.
} % Q1
    \begin{center}
        \vspace{1ex}
        \begin{tabular}{l C C}
            \toprule
            
                Experiencia
                & 1 & 2
                \\ Diâmetro das pellets (\unit{\mm})
                & 0.22 & 1.85
                \\ \(X (\%)\)
                & 93.2 & 29.8
            
            \\\bottomrule
        \end{tabular}
    \end{center}

    \paragraph*{Use:}
    \begin{BM}
        \mathemph{\phi}
        = r_p\sqrt{
            \frac
            {k'\,\rho_c}
            {D\,e}
        }
        ; \qquad
        \mathemph{\eta}
        = \frac{3}{\phi^2}
        \,(\phi\,\coth{\phi}-1)
        ;\qquad
        \mathemph{\rho_c} 
        = \qty*{1.2}{\g.\cm^{-3}}
    \end{BM}
\end{questionBox}

\begin{questionBox}2{ % MARK: Q1.1
    Determine, para cada uma das experiências, o valor da constante cinética aparente
} % Q1.1
    \answer{}
\end{questionBox}

\begin{questionBox}2{ % MARK: Q1.2
    Calcule, para cada uma das experiências, o módulo de Thiele e o factor de efectividade.
} % Q1.2
\end{questionBox}

\begin{questionBox}2{ % MARK: Q1.3
    Usando os dados calculados na alínea anterior, determine os valores da constante cinética intrínseca e da difusividade efectiva.
} % Q1.3
\end{questionBox}

\begin{questionBox}2{ % MARK: Q1.4
    Determine o valor da constante cinética aparente para o caso de pellets de \qty*{2}{\mm} de diâmetro.
} % Q1.4
\end{questionBox}

\begin{questionBox}2{ % MARK: Q1.5
    Sabendo que a um reactor de leito fixo constituído por um único tubo de \qty*{20}{\cm} de diâmetro de secção recta e \qty*{2}{\m} de comprimento, carregado com o mesmo catalisador na forma de pellets esféricas de \qty*{5}{\mm} de diâmetro, é alimentada a mesma solução de A, a um caudal volumétrico de \qty*{30}{\L/\min}, determine a conversão à saída.
} % Q1.5
\end{questionBox}

\begin{questionBox}1{ % MARK: Q2
    A reacção elementar \ch{A -> B} é conduzida, na fase gasosa, num reactor multitubular de leito fixo, consistindo em 50 tubos de \qty*{1}{\m} de comprimento e \qty*{2.5}{\cm} de diâmetro da secção recta, cheios com um catalisador sólido, poroso, na forma de pellets esféricas de \qty*{6}{\mm} de diâmetro. O reagente A é alimentado puro a um caudal de \qty*{100}{\dm^3/\min}. à temperatura de \qty*{673}{\K} e à pressão de \qty*{1}{\atm}, sendo obtida a conversão de \emph{35\% à saída do reator.}
} % Q2
    \paragraph*{Dados:}
    \begin{itemize}
        \item Massa volúmetrica do catalizador: \(\mathemph{\rho_c}=\qty*{1.3}{\g/\cm^3}\)
        \item Viscosidade cinemática: \(\mathemph{v}=\qty*{4e-6}{\m^2/\s}\)
        \item \(\mathemph{\varepsilon_b}=0.46\)
        \item Difusividade efetiva intraparticular: \(\mathemph{\mathscr{D}_e}=\qty*{1.3e-8}{\m^2/\s}\)
        \item Constante cinética intrínseca: \(\mathemph{k'}=\qty*{1.8e-2}{\dm^3/\g\of{cat}.\min}\)
        \item Constante dos gáses perfeitos:
        \\\(\mathemph{R}=\qty{8.2057366080960e-2}{\dm^3.\atm.\mole^{-1}.\kelvin^{-1}}\)
    \end{itemize}
    \begin{BM}
        \mathemph{Sh}
        = 1.0\,Re^{1/2}\,Sc^{1/3}
        = \frac{k_c\,d_p}{\mathscr{D}_A}
        \,\frac{\varepsilon_b}{1-\varepsilon_b}
        ; \quad
        \mathemph{Re}
        = \frac{u\,d_p}{v\,(1-\varepsilon_b)}
        ; \\
        \mathemph{Sc}=v/\mathscr{D}_A
        ; \quad
        \mathemph{\phi}
        = r_p\,\sqrt{
            \frac
            {k'\,\rho_p}
            {\mathscr{D}_e}
        }
        ; \quad
        \mathemph{\eta}
        = (\phi\,\coth{\phi}-1)\,3/\phi^2
    \end{BM}
\end{questionBox}

\begin{questionBox}2{ % MARK: Q2.1
    Determine o valor da massa de catalisador
} % Q2.1
    \answer{}
    \begin{flalign*}
        &
            V_R
            = \frac{W_{cat}}{\rho_{cat}}
            + \varepsilon_b\,V_R
            \implies &\\&
            \implies
            W_{cat}
            = V_R\,\rho_{cat}\,(1-\varepsilon_b)
            = (N_{tubos}\,V_{tubo})
            \,\rho_{cat}\,(1-\varepsilon_b)
            = &\\&
            = 50
            * 1\E{2}
            * \frac{\pi\,2.5^2}{4}
            * 1.3
            * (1-0.46)
            \cong 
            \qty{1.72296722095315223e4}
            {\g\of{cat}}
        &
    \end{flalign*}
\end{questionBox}

\begin{questionBox}2{ % MARK: Q2.2
    Calcule o valor da constante cinética observada
} % Q2.2
    \answer{}
    \begin{flalign*}
        &
            \text{\emph{Lei cinética:}} 
            &\\&
            -r'_{A,obs}
            = k'_{obs}\,C_A
            = k'_{obs}\,C_{A,0}\,(1-X)
            % 
            % 
            % 
            ; &\\[3ex]&
            \text{\emph{Balanço molar ao reator:}}
            &\\&
            \odif{w}
            = F_{A,0}
            \,\frac{\odif{X}}{-r'_{A,obs}}
            \implies &\\&
            \implies
            \int_0^{W}{\odif{w}}
            = W
            = &\\&
            = \int_0^{X}{
                F_{A,0}
                \,\frac{\odif{X}}{-r'_{A,obs}}
            }
            = 
            F_{A,0}
            \,\int_0^{X}{
                \frac{\odif{X}}{k'_{obs}\,C_{A,0}\,(1-X)}
            }
            = &\\&
            = 
            \frac{F_{A,0}}{k'_{obs}\,C_{A,0}}
            \,\int_0^{X}{
                -\frac{\odif{(1-X)}}{1-X}
            }
            = \frac{v_{0}}{k'_{obs}}
            \,\ln\frac{1-0}{1-X}
            % 
            % 
            % 
            \implies &\\[3ex]&
            \implies
            k'_{obs}
            = \frac{v_{0}}{W}
            \,\ln\frac{1}{1-X}
            \cong \frac
            {100}
            {\num{1.72296722095315223e4}}
            \,\ln\frac{1}{1-0.35}
            \,\unit{\dm^3/\min.\g\of{cat}}
            \cong &\\&
            \cong
            \qty{2.50023860497e-3}
            {\dm^3/\min.\g\of{cat}}
        &
    \end{flalign*}
\end{questionBox}

\begin{questionBox}2{ % MARK: Q2.3
    Calcule o valor da constante cinética que se observaria, no caso de ausência de limitações difusionais externas
} % Q2.3
    \answer{}
    \begin{flalign*}
        &
            k'_{ap}
            = \eta\,k'
            = \left(
                (\phi\,\coth{\phi}-1)
                \,3/\phi^2
            \right)
            \,k'
            ; &\\[3ex]&
            \mathemph{\phi:}
            &\\&
            \phi
            = r_p\,\sqrt{
                \frac
                {k'\,\rho_{cat}}
                {\mathscr{D}_e}
            }
            \cong 
            3\E{-2}
            \,\sqrt{
                \frac
                {k'\,1.3\E{3}}
                {(60*1.3\E{-6})}
            }
            \cong 
            \num{122.474487139158905}
            \,\sqrt{k'}
            \cong &\\&
            \cong 
            \num{122.474487139158905}
            \,\sqrt{1.8\E{-2}}
            \cong 
            \num{9.128709291752769}
            \land
            \phi^2\cong
            \num{1.4999999999999998797e4}\,k'
            % 
            % 
            % 
            \implies &\\[3ex]&
            \implies
            k'_{ap}
            = (\phi\,\coth{\phi}-1)
            \,\frac{3\,k'}{\phi^2}
            \cong &\\&
            \cong (
                \num{9.128709291752769}
                \,\coth{\num{9.128709291752769}}
                -1
                % 4.802437875598051
            )
            \,\frac
            {3\,k'}
            {\num{1.4999999999999998797e4}\,k'}
            \cong
            \qty{1.625741901341126e-3}
            {\dm^3/\min.\g\of{cat}}
        &
    \end{flalign*}
\end{questionBox}

\begin{questionBox}2{ % MARK: Q2.4
    Determine o valor do coeficiente de difusão externo
} % Q2.4
    \answer{}
    \begin{flalign*}
        &
            \text{\emph{Difusividade externa}}
            \,\mathemph{\mathscr{D}_A:}
            &\\&
            Sh
            = \frac{k_c\,d_p}{\mathscr{D}_A}
            = 1.0\,Re^{1/2}\,Sc^{1/3}
            = Re^{1/2}
            \,\left(
                \frac{v}{\mathscr{D}_A}
            \right)^{1/3}
            \implies &\\&
            \implies
            \mathscr{D}_A
            = \frac
            {(k_c\,d_p)^{3/2}}
            {Re^{3/4}\,v^{1/2}}
            % 
            % 
            % 
            ; &\\[3ex]&
            \text{\emph{Numero de Reynald}}
            \,\mathemph{Re:}
            &\\&
            Re
            = \frac{u\,d_p}{v\,(1-\varepsilon_b)}
            % 
            % 
            % 
            \cong &\\[3ex]&
            \cong \frac
            {\num{88.573185720706969}\,3\E{-2}}
            {v\,(1-0.46)}
            \cong
            \frac{\num{4.920732540039276}}{v}
            % 
            % 
            % 
            ; &\\[3ex]&
            \text{\emph{Velocidade linear}}
            \,\mathemph{u:}
            &\\&
            u
            = \frac{v_{tubo}}{\varepsilon_b*A_c}
            = \frac
            {v/N_{tubos}}
            {\varepsilon_b\,\pi\,D_{tubo}^2/4}
            = \frac
            {100/50}
            {0.46\,\pi\,(2.5\E{-1})^2/4}
            \cong &\\&
            \cong
            \qty{88.573185720706969}{\dm/\min}
            % 
            % 
            % 
            ; &\\[3ex]&
            \mathemph{k_c:}
            &\\&
            k_c
            = k'_c\,a
            = k'_c
            \,\left(
                \frac
                % {\pi\,(d_p/2)^3\,\rho_c\,4/3}
                {\pi\,d_p^3\,\rho_c/6}
                {\pi\,d_p^2}
            \right)
            = k'_c
            \,\frac
            {d_p\,\rho_c}
            {6}
            % 
            % 
            % 
            = &\\[3ex]&
            = \left(
                \frac
                {k'_{ap}\,k'_{obs}}
                {k'_{ap}- k'_{obs}}
            \right)
            \,\frac
            {d_p\,\rho_c}
            {6}
            % \cong &\\&
            \cong
            \frac
            {
                \num{1.625741901341126e-3}
                *\num{2.50023860497e-3}
            }
            {
                \num{1.625741901341126e-3}
                - \num{2.50023860497e-3}
            }
            \,\frac
            {6\E{-2}\,1.3\E{3}}
            {6}
            \cong &\\&
            \cong
            \qty{7.698258058949917298e3}
            {\dm/\min}
            % 
            % 
            % 
            ; &\\[3ex]&
            \mathemph{k'_c:}
            &\\&
            k'_c
            \,(C_{A,b}-C_{A,out})
            = -r'_A
            = k'_{ap}\,C_{A,out}
            = k'_{obs}\,C_{A,b}
            \implies &\\&
            \implies
            k'_{ap}\,C_{A,out}
            = \frac{k'_{ap}\,k'_c}{k'_{ap}+k'_c}
            \,C_{A,b}
            = k'_{obs}\,C_{A,b}
            \implies &\\&
            \implies
            \frac{k'_{ap}\,k'_c}{k'_{ap}+k'_c}
            = k'_{obs}
            % \implies &\\&
            \implies
            k'_c
            = \frac
            {k'_{ap}\,k'_{obs}}
            {k'_{ap}- k'_{obs}}
            % 
            % 
            % 
            ; &\\[6ex]&
            \therefore
            \mathscr{D}_A
            = \frac
            {(k_c\,d_p)^{3/2}}
            {Re^{3/4}\,v^{1/2}}
            % = &\\&
            \cong \frac
            {
                (
                    \num{7.698258058949917298e3}
                    *6\E{-2}
                )^{3/2}
            }
            {
                (\frac{\num{4.920732540039276}}{v})^{3/4}
                \,v^{1/2}
            }
            \cong \frac
            {
                \num{9.926944036474929667e3}
            }
            {
                \num{4.920732540039276}^{3/4}
                \,v^{-1/4}
            }
            \cong &\\&
            \cong \num{3.004645595705063742e3}
            \,(4\E{-4}*60)^{1/4}
            \cong
            \qty{1.18262229963214645e3}
            {\dm^2/\min}
        &
    \end{flalign*}
\end{questionBox}

\begin{questionBox}2{ % MARK: Q2.5
    Diga, justificando a resposta, se o reactor se encontra a funcionar em regime cinético, difusional interno, difusional externo ou misto.
} % Q2.5
    \answer{}
    Regime de funcionamento corresponde ao paço mais lento.
    \begin{itemize}
        \item Cinético: reação química é mais lenta
        \item Difusional interno: Transf de massa interna é mais lenta
        \item Difusional externo: Transf de massa externa é mais lenta
        \item Difusional misto: Transf de massa interna \(\approx\) externa
    \end{itemize}

    \(\phi\gg{3}\land\eta\ll{1}\implies\) fortes limitações internas, paço mais lento é transferencia de massa interna ou externa.

    \begin{flalign*}
        &
            \frac{k'_c}{k'_{ap}}
            = \frac
            {
                \frac
                {k'_{ap}\,k'_{obs}}
                {k'_{ap}- k'_{obs}}
            }{k'_{ap}}
            = \left(
                \frac{k'_{ap}}{k'_{obs}}
                - 1
            \right)^{-1}
            \cong
            \left(
                \frac
                {\num{7.70283093202181e-3}}
                {\num{2.50023860497e-3}}
                -1
            \right)^{-1}
            \cong
            \num{0.480575537692923}<1
        &
    \end{flalign*}

    \(\therefore\) Paço mais lento é transferencia de massa atravez do filme externo
    \(\implies\) \emph{Regime difusional externo}
\end{questionBox}

\begin{questionBox}1{ % MARK: Q3
    A reacção elementar \ch{A -> B} é conduzida, na fase gasosa, num reactor multitubular de leito fixo, consistindo em 100 tubos de \qty*{2}{\m} de comprimento e \qty*{2}{\cm} de diâmetro da seção recta, cheios com um catalisador sólido, poroso, na forma de pellets esféricas de \qty*{5}{\mm} de diâmetro. O reagente A é alimentado puro a um caudal de \qty*{100}{\dm^3/\min}, à temperatura de \qty*{373}{\K} e à pressão de \qty*{6}{\atm}
} % Q3
    \paragraph*{dados}
    \begin{itemize}
        \item Massa volúmetrica dos pellets: \(\mathemph{\rho_p}=\qty*{1.3}{\g/\cm^3}\)
        \item Coeficiente de difusão externo: \(\mathemph{\mathscr{D}_A}=\qty*{2.7e-7}{\m^2/\s}\)
        \item Viscosidade cinemática: \(\mathemph{v}=\qty*{4e-6}{\m^2/\s}\)
        \item \(\mathemph{\varepsilon_b}=0.45\)
        \item Difusividade efetiva intraparticular: \(\mathemph{\mathscr{D}_e}=\qty*{1.3e-8}{\m^2/\s}\)
        \item Constante cinética intríseca: \(\mathemph{k'}=\qty*{2.3e-2}{\dm^3/\g\of{cat}.\min}\)
        \item Constante dos gáses perfeitos:
        \\\(\mathemph{R}=\qty{8.2057366080960e-2}{\dm^3.\atm.\mole^{-1}.\kelvin^{-1}}\)
    \end{itemize}
    \begin{BM}
        \mathemph{Sh}
        = 1.0\,Re^{1/2}\,Sc^{1/3}
        = \frac{k_c\,d_p}{\mathscr{D}_A}
        \,\frac{\varepsilon_b}{1-\varepsilon_b}
        ; \quad
        \mathemph{Re}
        = \frac{u\,d_p}{v\,(1-\varepsilon_b)}
        ; \\
        \mathemph{Sc}=v/\mathscr{D}_A
        ; \quad
        \mathemph{\phi}
        = r_p\,\sqrt{
            \frac
            {k'\,\rho_p}
            {\mathscr{D}_e}
        }
        ; \quad
        \mathemph{\eta}
        = (\phi\,\coth{\phi}-1)\,3/\phi^2
    \end{BM}
    \paragraph*{Perfil de concentração dos pellets:}
    \begin{BM}
        \mathemph{\varphi} 
        = \frac
        {\sinh{\phi\,\lambda}}
        {\lambda\,\sinh{\phi}}
    \end{BM}
\end{questionBox}

\begin{questionBox}2{ % MARK: Q3.1
    Calcule o valor da constante cinética aparente, que observaria no caso da ausência de limitações difusionais externas.
} % Q3.1
    \answer{}
    \begin{flalign*}
        &
            k'_{ap}
            = \eta\,k'
            = ((\phi\,\coth{\phi}-1)\,3/\phi^2)
            \,k'
            % 
            % 
            % 
            ; &\\[3ex]&
            \mathemph{\phi}
            &\\&
            \phi
            = r_p\,\sqrt{
                \frac
                {k'\,\rho_p}
                {\mathscr{D}_e}
            }
            \cong
            2.5\E{-2}
            \,\sqrt{
                \frac
                {k'\,1.3\E{3}}
                {60*1.3\E{-6}}
            }
            \cong
            \num{1.02062072615965754e2}
            \,\sqrt{k'}
            \cong &\\&
            \cong
            \num{1.02062072615965754e2}
            \,\sqrt{2.3\E{-2}}
            % \cong &\\&
            \cong
            \num{1.5478479684172265e1}
            \land
            \phi^2
            \cong
            \num{1.0416666666666676038e4}\,k'
            % 
            % 
            % 
            ; &\\[6ex]&
            \therefore
            k'_{ap}
            = (\phi\,\coth{\phi}-1)
            \,\frac{3\,k'}{\phi^2}
            \cong &\\&
            \cong 
            (
                \num{1.5478479684172265e1}
                \,\coth{\num{1.5478479684172265e1}}
                -1
            )
            \,\frac
            {3\,k'}
            {\num{1.0416666666666676038e4}\,k'}
            \cong &\\&
            \cong
            \qty{4.169802149041926e-3}{\dm^3/\min.\g\of{cat}}
        &
    \end{flalign*}
\end{questionBox}

\begin{questionBox}2{ % MARK: Q3.2
    Calcule o valor do coeficiente de transferência de massa
} % Q3.2
    \answer{}
    \begin{flalign*}
        &
            % \text{\emph{Coeficiente de transferencia de massa}}
            % \,\mathemph{k'_c:}
            % &\\&
            k'_c
            = k_c\,a^{-1}
            = k_c
            \,\left(
                \frac
                {\pi\,d_c^3\,\rho_c/6}
                {\pi\,d_c^2}
            \right)^{-1}
            = \frac{k_c\,6}{d_c\,\rho_c}
            % 
            % 
            % 
            ; &\\[3ex]&
            \mathemph{k_c:}
            &\\&
            Sh
            = \frac{k_c\,d_p}{\mathscr{D}_A}
            = 1.0\,Re^{1/2}\,Sc^{1/3}
            \implies &\\&
            \implies
            k_c
            = Re^{1/2}\,Sc^{1/3}
            \,\frac{\mathscr{D}_A}{d_p}
            = \left(
                \frac
                {u\,d_p}
                {v\,(1-\varepsilon_b)}
            \right)^{1/2}
            \,\left(
                \frac{v}{\mathscr{D}_a}
            \right)^{1/3}
            \,\frac{\mathscr{D}_A}{d_p}
            = &\\&
            = \frac
            {\mathscr{D}_A^{2/3}}
            {v^{1/6}}
            \sqrt{
                \frac{u}
                {d_p\,(1-\varepsilon_b)}
            }
            \cong &\\[3ex]&
            \cong \frac
            {(2.3\E{-5}*60)^{2/3}}
            % 0.012395176757509
            {(4\E{-4}*60)^{1/6}}
            % 0.537075333693032
            \sqrt{
                \frac{\num{1131.768484209033526}}
                {5\E{-2}\,(1-0.45)}
            }
            % 202.867487803248642
            \cong
            \qty{4.681984466466463}
            {\dm^3/\g\of{cat}.\min}
            % 
            % 
            % 
            ; &\\[3ex]&
            \text{\emph{Velocidade Linear}}
            \,\mathemph{u:}
            &\\&
            u
            = \frac{v_{tubos}}{\varepsilon_b\,A_c}
            = \frac
            {v/N_{tubos}}
            {\varepsilon_b\,\pi\,d_{tubo}^2/4}
            = \frac
            {100/100}
            {0.45*\pi*(5\E{-2})^2/4}
            \cong
            \qty{1131.768484209033526}{\dm/\min}
            % 
            % 
            % 
            ; &\\[6ex]&
            \therefore
            k'_c
            \cong \frac{
                \num{4.681984466466463}*6
            }{
                5\E{-2}
                *1.3\E{3}
            }
            \cong
            \qty{0.432183181519981}
            {\dm^3/\min.\g\of{cat}}
        &
    \end{flalign*}
\end{questionBox}

\begin{questionBox}2{ % MARK: Q3.3
    Calcule o valor da constante cinética realmente observada.
} % Q3.3
    \answer{}
    \begin{flalign*}
        &
            \mathemph{k'_{obs}:}
            &\\&
            k'_c
            \,(C_{A,b}-C_{A,out})
            = -r'_A
            = k'_{ap}\,C_{A,out}
            = k'_{obs}\,C_{A,b}
            \implies &\\&
            \implies
            k'_{ap}\,C_{A,out}
            = \frac{k'_{ap}\,k'_c}{k'_{ap}+k'_c}
            \,C_{A,b}
            = k'_{obs}\,C_{A,b}
            \implies &\\&
            \implies
            k'_{obs}
            = \frac{k'_{ap}\,k'_c}{k'_{ap}+k'_c}
            = \frac{
                \num{4.169802149041926e-3}
                \,\num{0.432183181519981}
            }{
                \num{4.169802149041926e-3}
                +\num{0.432183181519981}
            }
            \cong
            \qty{4.05e-3}
            {\dm^3/\min.\g\of{cat}}
        &
    \end{flalign*}
\end{questionBox}

\begin{questionBox}2{ % MARK: Q3.4
    Diga, justificando a sua resposta, se o reactor se encontra em regime cinético, difusional interno, difusional externo ou misto.
} % Q3.4
    \answer{}
    \begin{BM}
        \eta
        = \frac{k'_{ap}}{k'}
        \cong \frac
        {\num{4.169802149041926e-3}}
        {2.3\E{-2}}
        \cong\num{0.181295745610519}
        \ll1
        \land
        \phi\cong
        \num{15.478479684172265}
        \gg3
    \end{BM}
    \(\therefore\) Fortes limitações difusionais internas
    \begin{BM}
        \frac{k_c}{k'_{ap}}
        \cong \frac
        {\num{4.681984466466463}}
        {\num{4.169802149041926e-3}}
        \cong 
        \num{1122.831323673765674}
        \gg1
    \end{BM}
    \(\therefore\) Efeitos de transferencia de massa externa despresível
    \(\implies\) Regime difusional interno
\end{questionBox}

\begin{questionBox}2{ % MARK: Q3.5
    Calcule a conversão à saída do reator
} % Q3.5
    \answer{}
    \begin{flalign*}
        &
            \int_0^W{\odif{w}}
            = W
            = \int_0^X{
                F_{A,0}
                \,\frac{\odif{X}}{-r'_{A,obs}}
            }
            = \int_0^X{
                F_{A,0}
                \,\frac{\odif{X}}{k'_{obs}\,C_A}
            }
            = &\\&
            = \int_0^X{
                F_{A,0}
                \,\frac{\odif{X}}{k'_{obs}\,C_{A,0}(1-X)}
            }
            = \int_0^X{
                \frac{v_{0}}{k'_{obs}}
                \,\frac{\odif{X}}{(1-X)}
            }
            = &\\&
            = 
            \frac{v_{0}}{k'_{obs}}
            \int_0^X{
                -\frac{\odif{(1-X)}}{(1-X)}
            }
            = \frac{v_{0}}{k'_{obs}}
            \ln\frac{1-0}{1-X}
            \implies &\\&
            \implies
            X
            = 1
            -\exp{\left(
                \frac{W\,k'_{obs}}
                {v_{0}}
            \right)}^{-1}
            = 1
            -\exp{\left(
                \frac{W\,k'_{obs}}
                {v_{0}}
            \right)}^{-1}
            % 
            % 
            % 
            ; &\\[3ex]&
            \mathemph{W:}
            &\\&
            V_R
            =\frac{W}{\rho_p}
            +\varepsilon_b\,V_R
            \implies
            W
            = V_R\,\rho_p\,(1-\varepsilon_b)
            = &\\&
            = N_{tubos}
            \,\frac{\pi\,d_{tubos}^2}{4}
            \,L_{tubos}
            \,\rho_p
            \,(1-\varepsilon_b)
            = &\\&
            = 100
            * \frac{\pi\,(2\E{-1})^2}{4}
            * 2\E{1}
            * 1.3\E{3}
            * (1-0.45)
            \cong
            \qty{44924.77494633404331}{\g}
            % 
            % 
            % 
            ; &\\[6ex]&
            \therefore
            X
            = 1
            -\exp{\left(
                \frac{W\,k'_{obs}}
                {v_{0}}
            \right)}^{-1}
            \cong
            1
            -\exp{\left(
                \frac{
                    \num{44924.77494633404331}
                    *\num{4.05e-3}
                }
                {100}
            \right)}^{-1}
            \cong &\\&
            \cong
            \num{83.7885659203125}\%
        &
    \end{flalign*}
\end{questionBox}

\begin{questionBox}2{ % MARK: Q3.6
    Determine o valor da contreção de A no centro das \textit{pellets}, à saída do reator
} % Q3.6
    \answer{}
    \begin{flalign*}
        &
            C_A
            =\frac
            {C_{A,out}\,\sinh{\phi\,\lambda}}
            {\lambda\,\sinh{\phi}}
            % 
            % 
            % 
            ; &\\[3ex]&
            \mathemph{C_{A,out}:}
            &\\&
            k'_c(C_{A,b}-C_{A,out})
            = k'_{ap}\,C_{A,out}
            \implies
            C_{A,out}
            = \frac{1}{1+k'_{ap}/k'_c}
            \,C_{A,b}
            = &\\&
            = \frac{1}{1+k'_{ap}/k'_c}
            \,C_{A,b,0}(1-X)
            = \frac{1}{1+k'_{ap}/k'_c}
            \,\frac{P_0}{R\,T}
            \,(1-X)
            \cong &\\&
            \cong 
            \frac{1}{
                1
                +\num{4.169802149041926e-3}
                /\num{4.681984466466463}
            }
            \,\frac{6}{
                \num{8.2057366080960e-2}
                *373
            }
            \,(1-\num{0.837885659203125})
            \cong &\\&
            \cong
            \qty{3.175116519862209e-2}{\M}
            % 
            % 
            % 
            ; &\\[3ex]&
            \lambda=1\E{-5}
            % 
            % 
            % 
            ; &\\[6ex]&
            \therefore
            C_A
            = \frac
            {C_{A,out}\,\sinh{\phi\,\lambda}}
            {\lambda\,\sinh{\phi}}
            \cong 
            \frac
            {
                \num{3.1751165198622092e-2}
                \,\sinh{
                    \num{15.478479684172265}
                    \,1\E{-5}
                }
            }
            {
                1\E{-5}
                \,\sinh{\num{15.478479684172265}}
            }
            \cong
            \qty{1.863372369484183e-7}{\M}
        &
    \end{flalign*}
\end{questionBox}

\end{document}